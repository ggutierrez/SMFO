\documentclass[DIV=20]{scrartcl}
\usepackage[utf8]{inputenc}
\usepackage{mathpazo}
\usepackage{subcaption}
\usepackage{pgf}
\usepackage{import}
\usepackage[noabbrev]{cleveref}

\title{Gaussian optimization}
\author{Juan David Gil \and Gustavo Gutiérrez}
\date{}

\begin{document}
\maketitle
\section{Introduction}
This is a report on the course \emph{Statistical Modelling for Optimization}
held at the \emph{Universidad Tecnológica de Pereira} from June 28th to July
1st. The practical part of course consisted of four lab sessions with a common
goal of optimizing the flying time of a paper helicopter model. In this document
we report our findings and caonclusions of the lab sessions.
\section{Problem}

\section{Lab session 1}

\paragraph{Noise between realizations of the experiement.}

\begin{figure}
  \begin{subfigure}[h]{.5\linewidth}
    %% Creator: Matplotlib, PGF backend
%%
%% To include the figure in your LaTeX document, write
%%   \input{<filename>.pgf}
%%
%% Make sure the required packages are loaded in your preamble
%%   \usepackage{pgf}
%%
%% Figures using additional raster images can only be included by \input if
%% they are in the same directory as the main LaTeX file. For loading figures
%% from other directories you can use the `import` package
%%   \usepackage{import}
%% and then include the figures with
%%   \import{<path to file>}{<filename>.pgf}
%%
%% Matplotlib used the following preamble
%%   \usepackage[utf8x]{inputenc}
%%   \usepackage[T1]{fontenc}
%%   \usepackage{cmbright}
%%
\begingroup%
\makeatletter%
\begin{pgfpicture}%
\pgfpathrectangle{\pgfpointorigin}{\pgfqpoint{2.500000in}{2.500000in}}%
\pgfusepath{use as bounding box, clip}%
\begin{pgfscope}%
\pgfsetbuttcap%
\pgfsetmiterjoin%
\definecolor{currentfill}{rgb}{1.000000,1.000000,1.000000}%
\pgfsetfillcolor{currentfill}%
\pgfsetlinewidth{0.000000pt}%
\definecolor{currentstroke}{rgb}{1.000000,1.000000,1.000000}%
\pgfsetstrokecolor{currentstroke}%
\pgfsetdash{}{0pt}%
\pgfpathmoveto{\pgfqpoint{0.000000in}{0.000000in}}%
\pgfpathlineto{\pgfqpoint{2.500000in}{0.000000in}}%
\pgfpathlineto{\pgfqpoint{2.500000in}{2.500000in}}%
\pgfpathlineto{\pgfqpoint{0.000000in}{2.500000in}}%
\pgfpathclose%
\pgfusepath{fill}%
\end{pgfscope}%
\begin{pgfscope}%
\pgfsetbuttcap%
\pgfsetmiterjoin%
\definecolor{currentfill}{rgb}{0.917647,0.917647,0.949020}%
\pgfsetfillcolor{currentfill}%
\pgfsetlinewidth{0.000000pt}%
\definecolor{currentstroke}{rgb}{0.000000,0.000000,0.000000}%
\pgfsetstrokecolor{currentstroke}%
\pgfsetstrokeopacity{0.000000}%
\pgfsetdash{}{0pt}%
\pgfpathmoveto{\pgfqpoint{0.556847in}{0.516222in}}%
\pgfpathlineto{\pgfqpoint{2.279437in}{0.516222in}}%
\pgfpathlineto{\pgfqpoint{2.279437in}{2.299750in}}%
\pgfpathlineto{\pgfqpoint{0.556847in}{2.299750in}}%
\pgfpathclose%
\pgfusepath{fill}%
\end{pgfscope}%
\begin{pgfscope}%
\pgfpathrectangle{\pgfqpoint{0.556847in}{0.516222in}}{\pgfqpoint{1.722590in}{1.783528in}} %
\pgfusepath{clip}%
\pgfsetroundcap%
\pgfsetroundjoin%
\pgfsetlinewidth{0.803000pt}%
\definecolor{currentstroke}{rgb}{1.000000,1.000000,1.000000}%
\pgfsetstrokecolor{currentstroke}%
\pgfsetdash{}{0pt}%
\pgfpathmoveto{\pgfqpoint{0.556847in}{0.516222in}}%
\pgfpathlineto{\pgfqpoint{0.556847in}{2.299750in}}%
\pgfusepath{stroke}%
\end{pgfscope}%
\begin{pgfscope}%
\pgfsetbuttcap%
\pgfsetroundjoin%
\definecolor{currentfill}{rgb}{0.150000,0.150000,0.150000}%
\pgfsetfillcolor{currentfill}%
\pgfsetlinewidth{0.803000pt}%
\definecolor{currentstroke}{rgb}{0.150000,0.150000,0.150000}%
\pgfsetstrokecolor{currentstroke}%
\pgfsetdash{}{0pt}%
\pgfsys@defobject{currentmarker}{\pgfqpoint{0.000000in}{0.000000in}}{\pgfqpoint{0.000000in}{0.000000in}}{%
\pgfpathmoveto{\pgfqpoint{0.000000in}{0.000000in}}%
\pgfpathlineto{\pgfqpoint{0.000000in}{0.000000in}}%
\pgfusepath{stroke,fill}%
}%
\begin{pgfscope}%
\pgfsys@transformshift{0.556847in}{0.516222in}%
\pgfsys@useobject{currentmarker}{}%
\end{pgfscope}%
\end{pgfscope}%
\begin{pgfscope}%
\definecolor{textcolor}{rgb}{0.150000,0.150000,0.150000}%
\pgfsetstrokecolor{textcolor}%
\pgfsetfillcolor{textcolor}%
\pgftext[x=0.556847in,y=0.438444in,,top]{\color{textcolor}\sffamily\fontsize{8.000000}{9.600000}\selectfont 1.5}%
\end{pgfscope}%
\begin{pgfscope}%
\pgfpathrectangle{\pgfqpoint{0.556847in}{0.516222in}}{\pgfqpoint{1.722590in}{1.783528in}} %
\pgfusepath{clip}%
\pgfsetroundcap%
\pgfsetroundjoin%
\pgfsetlinewidth{0.803000pt}%
\definecolor{currentstroke}{rgb}{1.000000,1.000000,1.000000}%
\pgfsetstrokecolor{currentstroke}%
\pgfsetdash{}{0pt}%
\pgfpathmoveto{\pgfqpoint{0.802932in}{0.516222in}}%
\pgfpathlineto{\pgfqpoint{0.802932in}{2.299750in}}%
\pgfusepath{stroke}%
\end{pgfscope}%
\begin{pgfscope}%
\pgfsetbuttcap%
\pgfsetroundjoin%
\definecolor{currentfill}{rgb}{0.150000,0.150000,0.150000}%
\pgfsetfillcolor{currentfill}%
\pgfsetlinewidth{0.803000pt}%
\definecolor{currentstroke}{rgb}{0.150000,0.150000,0.150000}%
\pgfsetstrokecolor{currentstroke}%
\pgfsetdash{}{0pt}%
\pgfsys@defobject{currentmarker}{\pgfqpoint{0.000000in}{0.000000in}}{\pgfqpoint{0.000000in}{0.000000in}}{%
\pgfpathmoveto{\pgfqpoint{0.000000in}{0.000000in}}%
\pgfpathlineto{\pgfqpoint{0.000000in}{0.000000in}}%
\pgfusepath{stroke,fill}%
}%
\begin{pgfscope}%
\pgfsys@transformshift{0.802932in}{0.516222in}%
\pgfsys@useobject{currentmarker}{}%
\end{pgfscope}%
\end{pgfscope}%
\begin{pgfscope}%
\definecolor{textcolor}{rgb}{0.150000,0.150000,0.150000}%
\pgfsetstrokecolor{textcolor}%
\pgfsetfillcolor{textcolor}%
\pgftext[x=0.802932in,y=0.438444in,,top]{\color{textcolor}\sffamily\fontsize{8.000000}{9.600000}\selectfont 2.0}%
\end{pgfscope}%
\begin{pgfscope}%
\pgfpathrectangle{\pgfqpoint{0.556847in}{0.516222in}}{\pgfqpoint{1.722590in}{1.783528in}} %
\pgfusepath{clip}%
\pgfsetroundcap%
\pgfsetroundjoin%
\pgfsetlinewidth{0.803000pt}%
\definecolor{currentstroke}{rgb}{1.000000,1.000000,1.000000}%
\pgfsetstrokecolor{currentstroke}%
\pgfsetdash{}{0pt}%
\pgfpathmoveto{\pgfqpoint{1.049016in}{0.516222in}}%
\pgfpathlineto{\pgfqpoint{1.049016in}{2.299750in}}%
\pgfusepath{stroke}%
\end{pgfscope}%
\begin{pgfscope}%
\pgfsetbuttcap%
\pgfsetroundjoin%
\definecolor{currentfill}{rgb}{0.150000,0.150000,0.150000}%
\pgfsetfillcolor{currentfill}%
\pgfsetlinewidth{0.803000pt}%
\definecolor{currentstroke}{rgb}{0.150000,0.150000,0.150000}%
\pgfsetstrokecolor{currentstroke}%
\pgfsetdash{}{0pt}%
\pgfsys@defobject{currentmarker}{\pgfqpoint{0.000000in}{0.000000in}}{\pgfqpoint{0.000000in}{0.000000in}}{%
\pgfpathmoveto{\pgfqpoint{0.000000in}{0.000000in}}%
\pgfpathlineto{\pgfqpoint{0.000000in}{0.000000in}}%
\pgfusepath{stroke,fill}%
}%
\begin{pgfscope}%
\pgfsys@transformshift{1.049016in}{0.516222in}%
\pgfsys@useobject{currentmarker}{}%
\end{pgfscope}%
\end{pgfscope}%
\begin{pgfscope}%
\definecolor{textcolor}{rgb}{0.150000,0.150000,0.150000}%
\pgfsetstrokecolor{textcolor}%
\pgfsetfillcolor{textcolor}%
\pgftext[x=1.049016in,y=0.438444in,,top]{\color{textcolor}\sffamily\fontsize{8.000000}{9.600000}\selectfont 2.5}%
\end{pgfscope}%
\begin{pgfscope}%
\pgfpathrectangle{\pgfqpoint{0.556847in}{0.516222in}}{\pgfqpoint{1.722590in}{1.783528in}} %
\pgfusepath{clip}%
\pgfsetroundcap%
\pgfsetroundjoin%
\pgfsetlinewidth{0.803000pt}%
\definecolor{currentstroke}{rgb}{1.000000,1.000000,1.000000}%
\pgfsetstrokecolor{currentstroke}%
\pgfsetdash{}{0pt}%
\pgfpathmoveto{\pgfqpoint{1.295100in}{0.516222in}}%
\pgfpathlineto{\pgfqpoint{1.295100in}{2.299750in}}%
\pgfusepath{stroke}%
\end{pgfscope}%
\begin{pgfscope}%
\pgfsetbuttcap%
\pgfsetroundjoin%
\definecolor{currentfill}{rgb}{0.150000,0.150000,0.150000}%
\pgfsetfillcolor{currentfill}%
\pgfsetlinewidth{0.803000pt}%
\definecolor{currentstroke}{rgb}{0.150000,0.150000,0.150000}%
\pgfsetstrokecolor{currentstroke}%
\pgfsetdash{}{0pt}%
\pgfsys@defobject{currentmarker}{\pgfqpoint{0.000000in}{0.000000in}}{\pgfqpoint{0.000000in}{0.000000in}}{%
\pgfpathmoveto{\pgfqpoint{0.000000in}{0.000000in}}%
\pgfpathlineto{\pgfqpoint{0.000000in}{0.000000in}}%
\pgfusepath{stroke,fill}%
}%
\begin{pgfscope}%
\pgfsys@transformshift{1.295100in}{0.516222in}%
\pgfsys@useobject{currentmarker}{}%
\end{pgfscope}%
\end{pgfscope}%
\begin{pgfscope}%
\definecolor{textcolor}{rgb}{0.150000,0.150000,0.150000}%
\pgfsetstrokecolor{textcolor}%
\pgfsetfillcolor{textcolor}%
\pgftext[x=1.295100in,y=0.438444in,,top]{\color{textcolor}\sffamily\fontsize{8.000000}{9.600000}\selectfont 3.0}%
\end{pgfscope}%
\begin{pgfscope}%
\pgfpathrectangle{\pgfqpoint{0.556847in}{0.516222in}}{\pgfqpoint{1.722590in}{1.783528in}} %
\pgfusepath{clip}%
\pgfsetroundcap%
\pgfsetroundjoin%
\pgfsetlinewidth{0.803000pt}%
\definecolor{currentstroke}{rgb}{1.000000,1.000000,1.000000}%
\pgfsetstrokecolor{currentstroke}%
\pgfsetdash{}{0pt}%
\pgfpathmoveto{\pgfqpoint{1.541185in}{0.516222in}}%
\pgfpathlineto{\pgfqpoint{1.541185in}{2.299750in}}%
\pgfusepath{stroke}%
\end{pgfscope}%
\begin{pgfscope}%
\pgfsetbuttcap%
\pgfsetroundjoin%
\definecolor{currentfill}{rgb}{0.150000,0.150000,0.150000}%
\pgfsetfillcolor{currentfill}%
\pgfsetlinewidth{0.803000pt}%
\definecolor{currentstroke}{rgb}{0.150000,0.150000,0.150000}%
\pgfsetstrokecolor{currentstroke}%
\pgfsetdash{}{0pt}%
\pgfsys@defobject{currentmarker}{\pgfqpoint{0.000000in}{0.000000in}}{\pgfqpoint{0.000000in}{0.000000in}}{%
\pgfpathmoveto{\pgfqpoint{0.000000in}{0.000000in}}%
\pgfpathlineto{\pgfqpoint{0.000000in}{0.000000in}}%
\pgfusepath{stroke,fill}%
}%
\begin{pgfscope}%
\pgfsys@transformshift{1.541185in}{0.516222in}%
\pgfsys@useobject{currentmarker}{}%
\end{pgfscope}%
\end{pgfscope}%
\begin{pgfscope}%
\definecolor{textcolor}{rgb}{0.150000,0.150000,0.150000}%
\pgfsetstrokecolor{textcolor}%
\pgfsetfillcolor{textcolor}%
\pgftext[x=1.541185in,y=0.438444in,,top]{\color{textcolor}\sffamily\fontsize{8.000000}{9.600000}\selectfont 3.5}%
\end{pgfscope}%
\begin{pgfscope}%
\pgfpathrectangle{\pgfqpoint{0.556847in}{0.516222in}}{\pgfqpoint{1.722590in}{1.783528in}} %
\pgfusepath{clip}%
\pgfsetroundcap%
\pgfsetroundjoin%
\pgfsetlinewidth{0.803000pt}%
\definecolor{currentstroke}{rgb}{1.000000,1.000000,1.000000}%
\pgfsetstrokecolor{currentstroke}%
\pgfsetdash{}{0pt}%
\pgfpathmoveto{\pgfqpoint{1.787269in}{0.516222in}}%
\pgfpathlineto{\pgfqpoint{1.787269in}{2.299750in}}%
\pgfusepath{stroke}%
\end{pgfscope}%
\begin{pgfscope}%
\pgfsetbuttcap%
\pgfsetroundjoin%
\definecolor{currentfill}{rgb}{0.150000,0.150000,0.150000}%
\pgfsetfillcolor{currentfill}%
\pgfsetlinewidth{0.803000pt}%
\definecolor{currentstroke}{rgb}{0.150000,0.150000,0.150000}%
\pgfsetstrokecolor{currentstroke}%
\pgfsetdash{}{0pt}%
\pgfsys@defobject{currentmarker}{\pgfqpoint{0.000000in}{0.000000in}}{\pgfqpoint{0.000000in}{0.000000in}}{%
\pgfpathmoveto{\pgfqpoint{0.000000in}{0.000000in}}%
\pgfpathlineto{\pgfqpoint{0.000000in}{0.000000in}}%
\pgfusepath{stroke,fill}%
}%
\begin{pgfscope}%
\pgfsys@transformshift{1.787269in}{0.516222in}%
\pgfsys@useobject{currentmarker}{}%
\end{pgfscope}%
\end{pgfscope}%
\begin{pgfscope}%
\definecolor{textcolor}{rgb}{0.150000,0.150000,0.150000}%
\pgfsetstrokecolor{textcolor}%
\pgfsetfillcolor{textcolor}%
\pgftext[x=1.787269in,y=0.438444in,,top]{\color{textcolor}\sffamily\fontsize{8.000000}{9.600000}\selectfont 4.0}%
\end{pgfscope}%
\begin{pgfscope}%
\pgfpathrectangle{\pgfqpoint{0.556847in}{0.516222in}}{\pgfqpoint{1.722590in}{1.783528in}} %
\pgfusepath{clip}%
\pgfsetroundcap%
\pgfsetroundjoin%
\pgfsetlinewidth{0.803000pt}%
\definecolor{currentstroke}{rgb}{1.000000,1.000000,1.000000}%
\pgfsetstrokecolor{currentstroke}%
\pgfsetdash{}{0pt}%
\pgfpathmoveto{\pgfqpoint{2.033353in}{0.516222in}}%
\pgfpathlineto{\pgfqpoint{2.033353in}{2.299750in}}%
\pgfusepath{stroke}%
\end{pgfscope}%
\begin{pgfscope}%
\pgfsetbuttcap%
\pgfsetroundjoin%
\definecolor{currentfill}{rgb}{0.150000,0.150000,0.150000}%
\pgfsetfillcolor{currentfill}%
\pgfsetlinewidth{0.803000pt}%
\definecolor{currentstroke}{rgb}{0.150000,0.150000,0.150000}%
\pgfsetstrokecolor{currentstroke}%
\pgfsetdash{}{0pt}%
\pgfsys@defobject{currentmarker}{\pgfqpoint{0.000000in}{0.000000in}}{\pgfqpoint{0.000000in}{0.000000in}}{%
\pgfpathmoveto{\pgfqpoint{0.000000in}{0.000000in}}%
\pgfpathlineto{\pgfqpoint{0.000000in}{0.000000in}}%
\pgfusepath{stroke,fill}%
}%
\begin{pgfscope}%
\pgfsys@transformshift{2.033353in}{0.516222in}%
\pgfsys@useobject{currentmarker}{}%
\end{pgfscope}%
\end{pgfscope}%
\begin{pgfscope}%
\definecolor{textcolor}{rgb}{0.150000,0.150000,0.150000}%
\pgfsetstrokecolor{textcolor}%
\pgfsetfillcolor{textcolor}%
\pgftext[x=2.033353in,y=0.438444in,,top]{\color{textcolor}\sffamily\fontsize{8.000000}{9.600000}\selectfont 4.5}%
\end{pgfscope}%
\begin{pgfscope}%
\pgfpathrectangle{\pgfqpoint{0.556847in}{0.516222in}}{\pgfqpoint{1.722590in}{1.783528in}} %
\pgfusepath{clip}%
\pgfsetroundcap%
\pgfsetroundjoin%
\pgfsetlinewidth{0.803000pt}%
\definecolor{currentstroke}{rgb}{1.000000,1.000000,1.000000}%
\pgfsetstrokecolor{currentstroke}%
\pgfsetdash{}{0pt}%
\pgfpathmoveto{\pgfqpoint{2.279437in}{0.516222in}}%
\pgfpathlineto{\pgfqpoint{2.279437in}{2.299750in}}%
\pgfusepath{stroke}%
\end{pgfscope}%
\begin{pgfscope}%
\pgfsetbuttcap%
\pgfsetroundjoin%
\definecolor{currentfill}{rgb}{0.150000,0.150000,0.150000}%
\pgfsetfillcolor{currentfill}%
\pgfsetlinewidth{0.803000pt}%
\definecolor{currentstroke}{rgb}{0.150000,0.150000,0.150000}%
\pgfsetstrokecolor{currentstroke}%
\pgfsetdash{}{0pt}%
\pgfsys@defobject{currentmarker}{\pgfqpoint{0.000000in}{0.000000in}}{\pgfqpoint{0.000000in}{0.000000in}}{%
\pgfpathmoveto{\pgfqpoint{0.000000in}{0.000000in}}%
\pgfpathlineto{\pgfqpoint{0.000000in}{0.000000in}}%
\pgfusepath{stroke,fill}%
}%
\begin{pgfscope}%
\pgfsys@transformshift{2.279437in}{0.516222in}%
\pgfsys@useobject{currentmarker}{}%
\end{pgfscope}%
\end{pgfscope}%
\begin{pgfscope}%
\definecolor{textcolor}{rgb}{0.150000,0.150000,0.150000}%
\pgfsetstrokecolor{textcolor}%
\pgfsetfillcolor{textcolor}%
\pgftext[x=2.279437in,y=0.438444in,,top]{\color{textcolor}\sffamily\fontsize{8.000000}{9.600000}\selectfont 5.0}%
\end{pgfscope}%
\begin{pgfscope}%
\definecolor{textcolor}{rgb}{0.150000,0.150000,0.150000}%
\pgfsetstrokecolor{textcolor}%
\pgfsetfillcolor{textcolor}%
\pgftext[x=1.418142in,y=0.273321in,,top]{\color{textcolor}\sffamily\fontsize{8.800000}{10.560000}\selectfont Falling time realization 1}%
\end{pgfscope}%
\begin{pgfscope}%
\pgfpathrectangle{\pgfqpoint{0.556847in}{0.516222in}}{\pgfqpoint{1.722590in}{1.783528in}} %
\pgfusepath{clip}%
\pgfsetroundcap%
\pgfsetroundjoin%
\pgfsetlinewidth{0.803000pt}%
\definecolor{currentstroke}{rgb}{1.000000,1.000000,1.000000}%
\pgfsetstrokecolor{currentstroke}%
\pgfsetdash{}{0pt}%
\pgfpathmoveto{\pgfqpoint{0.556847in}{0.516222in}}%
\pgfpathlineto{\pgfqpoint{2.279437in}{0.516222in}}%
\pgfusepath{stroke}%
\end{pgfscope}%
\begin{pgfscope}%
\pgfsetbuttcap%
\pgfsetroundjoin%
\definecolor{currentfill}{rgb}{0.150000,0.150000,0.150000}%
\pgfsetfillcolor{currentfill}%
\pgfsetlinewidth{0.803000pt}%
\definecolor{currentstroke}{rgb}{0.150000,0.150000,0.150000}%
\pgfsetstrokecolor{currentstroke}%
\pgfsetdash{}{0pt}%
\pgfsys@defobject{currentmarker}{\pgfqpoint{0.000000in}{0.000000in}}{\pgfqpoint{0.000000in}{0.000000in}}{%
\pgfpathmoveto{\pgfqpoint{0.000000in}{0.000000in}}%
\pgfpathlineto{\pgfqpoint{0.000000in}{0.000000in}}%
\pgfusepath{stroke,fill}%
}%
\begin{pgfscope}%
\pgfsys@transformshift{0.556847in}{0.516222in}%
\pgfsys@useobject{currentmarker}{}%
\end{pgfscope}%
\end{pgfscope}%
\begin{pgfscope}%
\definecolor{textcolor}{rgb}{0.150000,0.150000,0.150000}%
\pgfsetstrokecolor{textcolor}%
\pgfsetfillcolor{textcolor}%
\pgftext[x=0.479069in,y=0.516222in,right,]{\color{textcolor}\sffamily\fontsize{8.000000}{9.600000}\selectfont 2.0}%
\end{pgfscope}%
\begin{pgfscope}%
\pgfpathrectangle{\pgfqpoint{0.556847in}{0.516222in}}{\pgfqpoint{1.722590in}{1.783528in}} %
\pgfusepath{clip}%
\pgfsetroundcap%
\pgfsetroundjoin%
\pgfsetlinewidth{0.803000pt}%
\definecolor{currentstroke}{rgb}{1.000000,1.000000,1.000000}%
\pgfsetstrokecolor{currentstroke}%
\pgfsetdash{}{0pt}%
\pgfpathmoveto{\pgfqpoint{0.556847in}{0.771012in}}%
\pgfpathlineto{\pgfqpoint{2.279437in}{0.771012in}}%
\pgfusepath{stroke}%
\end{pgfscope}%
\begin{pgfscope}%
\pgfsetbuttcap%
\pgfsetroundjoin%
\definecolor{currentfill}{rgb}{0.150000,0.150000,0.150000}%
\pgfsetfillcolor{currentfill}%
\pgfsetlinewidth{0.803000pt}%
\definecolor{currentstroke}{rgb}{0.150000,0.150000,0.150000}%
\pgfsetstrokecolor{currentstroke}%
\pgfsetdash{}{0pt}%
\pgfsys@defobject{currentmarker}{\pgfqpoint{0.000000in}{0.000000in}}{\pgfqpoint{0.000000in}{0.000000in}}{%
\pgfpathmoveto{\pgfqpoint{0.000000in}{0.000000in}}%
\pgfpathlineto{\pgfqpoint{0.000000in}{0.000000in}}%
\pgfusepath{stroke,fill}%
}%
\begin{pgfscope}%
\pgfsys@transformshift{0.556847in}{0.771012in}%
\pgfsys@useobject{currentmarker}{}%
\end{pgfscope}%
\end{pgfscope}%
\begin{pgfscope}%
\definecolor{textcolor}{rgb}{0.150000,0.150000,0.150000}%
\pgfsetstrokecolor{textcolor}%
\pgfsetfillcolor{textcolor}%
\pgftext[x=0.479069in,y=0.771012in,right,]{\color{textcolor}\sffamily\fontsize{8.000000}{9.600000}\selectfont 2.5}%
\end{pgfscope}%
\begin{pgfscope}%
\pgfpathrectangle{\pgfqpoint{0.556847in}{0.516222in}}{\pgfqpoint{1.722590in}{1.783528in}} %
\pgfusepath{clip}%
\pgfsetroundcap%
\pgfsetroundjoin%
\pgfsetlinewidth{0.803000pt}%
\definecolor{currentstroke}{rgb}{1.000000,1.000000,1.000000}%
\pgfsetstrokecolor{currentstroke}%
\pgfsetdash{}{0pt}%
\pgfpathmoveto{\pgfqpoint{0.556847in}{1.025802in}}%
\pgfpathlineto{\pgfqpoint{2.279437in}{1.025802in}}%
\pgfusepath{stroke}%
\end{pgfscope}%
\begin{pgfscope}%
\pgfsetbuttcap%
\pgfsetroundjoin%
\definecolor{currentfill}{rgb}{0.150000,0.150000,0.150000}%
\pgfsetfillcolor{currentfill}%
\pgfsetlinewidth{0.803000pt}%
\definecolor{currentstroke}{rgb}{0.150000,0.150000,0.150000}%
\pgfsetstrokecolor{currentstroke}%
\pgfsetdash{}{0pt}%
\pgfsys@defobject{currentmarker}{\pgfqpoint{0.000000in}{0.000000in}}{\pgfqpoint{0.000000in}{0.000000in}}{%
\pgfpathmoveto{\pgfqpoint{0.000000in}{0.000000in}}%
\pgfpathlineto{\pgfqpoint{0.000000in}{0.000000in}}%
\pgfusepath{stroke,fill}%
}%
\begin{pgfscope}%
\pgfsys@transformshift{0.556847in}{1.025802in}%
\pgfsys@useobject{currentmarker}{}%
\end{pgfscope}%
\end{pgfscope}%
\begin{pgfscope}%
\definecolor{textcolor}{rgb}{0.150000,0.150000,0.150000}%
\pgfsetstrokecolor{textcolor}%
\pgfsetfillcolor{textcolor}%
\pgftext[x=0.479069in,y=1.025802in,right,]{\color{textcolor}\sffamily\fontsize{8.000000}{9.600000}\selectfont 3.0}%
\end{pgfscope}%
\begin{pgfscope}%
\pgfpathrectangle{\pgfqpoint{0.556847in}{0.516222in}}{\pgfqpoint{1.722590in}{1.783528in}} %
\pgfusepath{clip}%
\pgfsetroundcap%
\pgfsetroundjoin%
\pgfsetlinewidth{0.803000pt}%
\definecolor{currentstroke}{rgb}{1.000000,1.000000,1.000000}%
\pgfsetstrokecolor{currentstroke}%
\pgfsetdash{}{0pt}%
\pgfpathmoveto{\pgfqpoint{0.556847in}{1.280591in}}%
\pgfpathlineto{\pgfqpoint{2.279437in}{1.280591in}}%
\pgfusepath{stroke}%
\end{pgfscope}%
\begin{pgfscope}%
\pgfsetbuttcap%
\pgfsetroundjoin%
\definecolor{currentfill}{rgb}{0.150000,0.150000,0.150000}%
\pgfsetfillcolor{currentfill}%
\pgfsetlinewidth{0.803000pt}%
\definecolor{currentstroke}{rgb}{0.150000,0.150000,0.150000}%
\pgfsetstrokecolor{currentstroke}%
\pgfsetdash{}{0pt}%
\pgfsys@defobject{currentmarker}{\pgfqpoint{0.000000in}{0.000000in}}{\pgfqpoint{0.000000in}{0.000000in}}{%
\pgfpathmoveto{\pgfqpoint{0.000000in}{0.000000in}}%
\pgfpathlineto{\pgfqpoint{0.000000in}{0.000000in}}%
\pgfusepath{stroke,fill}%
}%
\begin{pgfscope}%
\pgfsys@transformshift{0.556847in}{1.280591in}%
\pgfsys@useobject{currentmarker}{}%
\end{pgfscope}%
\end{pgfscope}%
\begin{pgfscope}%
\definecolor{textcolor}{rgb}{0.150000,0.150000,0.150000}%
\pgfsetstrokecolor{textcolor}%
\pgfsetfillcolor{textcolor}%
\pgftext[x=0.479069in,y=1.280591in,right,]{\color{textcolor}\sffamily\fontsize{8.000000}{9.600000}\selectfont 3.5}%
\end{pgfscope}%
\begin{pgfscope}%
\pgfpathrectangle{\pgfqpoint{0.556847in}{0.516222in}}{\pgfqpoint{1.722590in}{1.783528in}} %
\pgfusepath{clip}%
\pgfsetroundcap%
\pgfsetroundjoin%
\pgfsetlinewidth{0.803000pt}%
\definecolor{currentstroke}{rgb}{1.000000,1.000000,1.000000}%
\pgfsetstrokecolor{currentstroke}%
\pgfsetdash{}{0pt}%
\pgfpathmoveto{\pgfqpoint{0.556847in}{1.535381in}}%
\pgfpathlineto{\pgfqpoint{2.279437in}{1.535381in}}%
\pgfusepath{stroke}%
\end{pgfscope}%
\begin{pgfscope}%
\pgfsetbuttcap%
\pgfsetroundjoin%
\definecolor{currentfill}{rgb}{0.150000,0.150000,0.150000}%
\pgfsetfillcolor{currentfill}%
\pgfsetlinewidth{0.803000pt}%
\definecolor{currentstroke}{rgb}{0.150000,0.150000,0.150000}%
\pgfsetstrokecolor{currentstroke}%
\pgfsetdash{}{0pt}%
\pgfsys@defobject{currentmarker}{\pgfqpoint{0.000000in}{0.000000in}}{\pgfqpoint{0.000000in}{0.000000in}}{%
\pgfpathmoveto{\pgfqpoint{0.000000in}{0.000000in}}%
\pgfpathlineto{\pgfqpoint{0.000000in}{0.000000in}}%
\pgfusepath{stroke,fill}%
}%
\begin{pgfscope}%
\pgfsys@transformshift{0.556847in}{1.535381in}%
\pgfsys@useobject{currentmarker}{}%
\end{pgfscope}%
\end{pgfscope}%
\begin{pgfscope}%
\definecolor{textcolor}{rgb}{0.150000,0.150000,0.150000}%
\pgfsetstrokecolor{textcolor}%
\pgfsetfillcolor{textcolor}%
\pgftext[x=0.479069in,y=1.535381in,right,]{\color{textcolor}\sffamily\fontsize{8.000000}{9.600000}\selectfont 4.0}%
\end{pgfscope}%
\begin{pgfscope}%
\pgfpathrectangle{\pgfqpoint{0.556847in}{0.516222in}}{\pgfqpoint{1.722590in}{1.783528in}} %
\pgfusepath{clip}%
\pgfsetroundcap%
\pgfsetroundjoin%
\pgfsetlinewidth{0.803000pt}%
\definecolor{currentstroke}{rgb}{1.000000,1.000000,1.000000}%
\pgfsetstrokecolor{currentstroke}%
\pgfsetdash{}{0pt}%
\pgfpathmoveto{\pgfqpoint{0.556847in}{1.790171in}}%
\pgfpathlineto{\pgfqpoint{2.279437in}{1.790171in}}%
\pgfusepath{stroke}%
\end{pgfscope}%
\begin{pgfscope}%
\pgfsetbuttcap%
\pgfsetroundjoin%
\definecolor{currentfill}{rgb}{0.150000,0.150000,0.150000}%
\pgfsetfillcolor{currentfill}%
\pgfsetlinewidth{0.803000pt}%
\definecolor{currentstroke}{rgb}{0.150000,0.150000,0.150000}%
\pgfsetstrokecolor{currentstroke}%
\pgfsetdash{}{0pt}%
\pgfsys@defobject{currentmarker}{\pgfqpoint{0.000000in}{0.000000in}}{\pgfqpoint{0.000000in}{0.000000in}}{%
\pgfpathmoveto{\pgfqpoint{0.000000in}{0.000000in}}%
\pgfpathlineto{\pgfqpoint{0.000000in}{0.000000in}}%
\pgfusepath{stroke,fill}%
}%
\begin{pgfscope}%
\pgfsys@transformshift{0.556847in}{1.790171in}%
\pgfsys@useobject{currentmarker}{}%
\end{pgfscope}%
\end{pgfscope}%
\begin{pgfscope}%
\definecolor{textcolor}{rgb}{0.150000,0.150000,0.150000}%
\pgfsetstrokecolor{textcolor}%
\pgfsetfillcolor{textcolor}%
\pgftext[x=0.479069in,y=1.790171in,right,]{\color{textcolor}\sffamily\fontsize{8.000000}{9.600000}\selectfont 4.5}%
\end{pgfscope}%
\begin{pgfscope}%
\pgfpathrectangle{\pgfqpoint{0.556847in}{0.516222in}}{\pgfqpoint{1.722590in}{1.783528in}} %
\pgfusepath{clip}%
\pgfsetroundcap%
\pgfsetroundjoin%
\pgfsetlinewidth{0.803000pt}%
\definecolor{currentstroke}{rgb}{1.000000,1.000000,1.000000}%
\pgfsetstrokecolor{currentstroke}%
\pgfsetdash{}{0pt}%
\pgfpathmoveto{\pgfqpoint{0.556847in}{2.044960in}}%
\pgfpathlineto{\pgfqpoint{2.279437in}{2.044960in}}%
\pgfusepath{stroke}%
\end{pgfscope}%
\begin{pgfscope}%
\pgfsetbuttcap%
\pgfsetroundjoin%
\definecolor{currentfill}{rgb}{0.150000,0.150000,0.150000}%
\pgfsetfillcolor{currentfill}%
\pgfsetlinewidth{0.803000pt}%
\definecolor{currentstroke}{rgb}{0.150000,0.150000,0.150000}%
\pgfsetstrokecolor{currentstroke}%
\pgfsetdash{}{0pt}%
\pgfsys@defobject{currentmarker}{\pgfqpoint{0.000000in}{0.000000in}}{\pgfqpoint{0.000000in}{0.000000in}}{%
\pgfpathmoveto{\pgfqpoint{0.000000in}{0.000000in}}%
\pgfpathlineto{\pgfqpoint{0.000000in}{0.000000in}}%
\pgfusepath{stroke,fill}%
}%
\begin{pgfscope}%
\pgfsys@transformshift{0.556847in}{2.044960in}%
\pgfsys@useobject{currentmarker}{}%
\end{pgfscope}%
\end{pgfscope}%
\begin{pgfscope}%
\definecolor{textcolor}{rgb}{0.150000,0.150000,0.150000}%
\pgfsetstrokecolor{textcolor}%
\pgfsetfillcolor{textcolor}%
\pgftext[x=0.479069in,y=2.044960in,right,]{\color{textcolor}\sffamily\fontsize{8.000000}{9.600000}\selectfont 5.0}%
\end{pgfscope}%
\begin{pgfscope}%
\pgfpathrectangle{\pgfqpoint{0.556847in}{0.516222in}}{\pgfqpoint{1.722590in}{1.783528in}} %
\pgfusepath{clip}%
\pgfsetroundcap%
\pgfsetroundjoin%
\pgfsetlinewidth{0.803000pt}%
\definecolor{currentstroke}{rgb}{1.000000,1.000000,1.000000}%
\pgfsetstrokecolor{currentstroke}%
\pgfsetdash{}{0pt}%
\pgfpathmoveto{\pgfqpoint{0.556847in}{2.299750in}}%
\pgfpathlineto{\pgfqpoint{2.279437in}{2.299750in}}%
\pgfusepath{stroke}%
\end{pgfscope}%
\begin{pgfscope}%
\pgfsetbuttcap%
\pgfsetroundjoin%
\definecolor{currentfill}{rgb}{0.150000,0.150000,0.150000}%
\pgfsetfillcolor{currentfill}%
\pgfsetlinewidth{0.803000pt}%
\definecolor{currentstroke}{rgb}{0.150000,0.150000,0.150000}%
\pgfsetstrokecolor{currentstroke}%
\pgfsetdash{}{0pt}%
\pgfsys@defobject{currentmarker}{\pgfqpoint{0.000000in}{0.000000in}}{\pgfqpoint{0.000000in}{0.000000in}}{%
\pgfpathmoveto{\pgfqpoint{0.000000in}{0.000000in}}%
\pgfpathlineto{\pgfqpoint{0.000000in}{0.000000in}}%
\pgfusepath{stroke,fill}%
}%
\begin{pgfscope}%
\pgfsys@transformshift{0.556847in}{2.299750in}%
\pgfsys@useobject{currentmarker}{}%
\end{pgfscope}%
\end{pgfscope}%
\begin{pgfscope}%
\definecolor{textcolor}{rgb}{0.150000,0.150000,0.150000}%
\pgfsetstrokecolor{textcolor}%
\pgfsetfillcolor{textcolor}%
\pgftext[x=0.479069in,y=2.299750in,right,]{\color{textcolor}\sffamily\fontsize{8.000000}{9.600000}\selectfont 5.5}%
\end{pgfscope}%
\begin{pgfscope}%
\definecolor{textcolor}{rgb}{0.150000,0.150000,0.150000}%
\pgfsetstrokecolor{textcolor}%
\pgfsetfillcolor{textcolor}%
\pgftext[x=0.251677in,y=1.407986in,,bottom,rotate=90.000000]{\color{textcolor}\sffamily\fontsize{8.800000}{10.560000}\selectfont Falling time realization 2}%
\end{pgfscope}%
\begin{pgfscope}%
\pgfpathrectangle{\pgfqpoint{0.556847in}{0.516222in}}{\pgfqpoint{1.722590in}{1.783528in}} %
\pgfusepath{clip}%
\pgfsetbuttcap%
\pgfsetroundjoin%
\definecolor{currentfill}{rgb}{0.298039,0.447059,0.690196}%
\pgfsetfillcolor{currentfill}%
\pgfsetlinewidth{0.240900pt}%
\definecolor{currentstroke}{rgb}{1.000000,1.000000,1.000000}%
\pgfsetstrokecolor{currentstroke}%
\pgfsetdash{}{0pt}%
\pgfpathmoveto{\pgfqpoint{1.738052in}{1.402409in}}%
\pgfpathcurveto{\pgfqpoint{1.746288in}{1.402409in}}{\pgfqpoint{1.754188in}{1.405681in}}{\pgfqpoint{1.760012in}{1.411505in}}%
\pgfpathcurveto{\pgfqpoint{1.765836in}{1.417329in}}{\pgfqpoint{1.769108in}{1.425229in}}{\pgfqpoint{1.769108in}{1.433465in}}%
\pgfpathcurveto{\pgfqpoint{1.769108in}{1.441701in}}{\pgfqpoint{1.765836in}{1.449601in}}{\pgfqpoint{1.760012in}{1.455425in}}%
\pgfpathcurveto{\pgfqpoint{1.754188in}{1.461249in}}{\pgfqpoint{1.746288in}{1.464522in}}{\pgfqpoint{1.738052in}{1.464522in}}%
\pgfpathcurveto{\pgfqpoint{1.729816in}{1.464522in}}{\pgfqpoint{1.721916in}{1.461249in}}{\pgfqpoint{1.716092in}{1.455425in}}%
\pgfpathcurveto{\pgfqpoint{1.710268in}{1.449601in}}{\pgfqpoint{1.706995in}{1.441701in}}{\pgfqpoint{1.706995in}{1.433465in}}%
\pgfpathcurveto{\pgfqpoint{1.706995in}{1.425229in}}{\pgfqpoint{1.710268in}{1.417329in}}{\pgfqpoint{1.716092in}{1.411505in}}%
\pgfpathcurveto{\pgfqpoint{1.721916in}{1.405681in}}{\pgfqpoint{1.729816in}{1.402409in}}{\pgfqpoint{1.738052in}{1.402409in}}%
\pgfpathclose%
\pgfusepath{stroke,fill}%
\end{pgfscope}%
\begin{pgfscope}%
\pgfpathrectangle{\pgfqpoint{0.556847in}{0.516222in}}{\pgfqpoint{1.722590in}{1.783528in}} %
\pgfusepath{clip}%
\pgfsetbuttcap%
\pgfsetroundjoin%
\definecolor{currentfill}{rgb}{0.298039,0.447059,0.690196}%
\pgfsetfillcolor{currentfill}%
\pgfsetlinewidth{0.240900pt}%
\definecolor{currentstroke}{rgb}{1.000000,1.000000,1.000000}%
\pgfsetstrokecolor{currentstroke}%
\pgfsetdash{}{0pt}%
\pgfpathmoveto{\pgfqpoint{1.491968in}{1.249535in}}%
\pgfpathcurveto{\pgfqpoint{1.500204in}{1.249535in}}{\pgfqpoint{1.508104in}{1.252807in}}{\pgfqpoint{1.513928in}{1.258631in}}%
\pgfpathcurveto{\pgfqpoint{1.519752in}{1.264455in}}{\pgfqpoint{1.523024in}{1.272355in}}{\pgfqpoint{1.523024in}{1.280591in}}%
\pgfpathcurveto{\pgfqpoint{1.523024in}{1.288828in}}{\pgfqpoint{1.519752in}{1.296728in}}{\pgfqpoint{1.513928in}{1.302552in}}%
\pgfpathcurveto{\pgfqpoint{1.508104in}{1.308375in}}{\pgfqpoint{1.500204in}{1.311648in}}{\pgfqpoint{1.491968in}{1.311648in}}%
\pgfpathcurveto{\pgfqpoint{1.483731in}{1.311648in}}{\pgfqpoint{1.475831in}{1.308375in}}{\pgfqpoint{1.470007in}{1.302552in}}%
\pgfpathcurveto{\pgfqpoint{1.464183in}{1.296728in}}{\pgfqpoint{1.460911in}{1.288828in}}{\pgfqpoint{1.460911in}{1.280591in}}%
\pgfpathcurveto{\pgfqpoint{1.460911in}{1.272355in}}{\pgfqpoint{1.464183in}{1.264455in}}{\pgfqpoint{1.470007in}{1.258631in}}%
\pgfpathcurveto{\pgfqpoint{1.475831in}{1.252807in}}{\pgfqpoint{1.483731in}{1.249535in}}{\pgfqpoint{1.491968in}{1.249535in}}%
\pgfpathclose%
\pgfusepath{stroke,fill}%
\end{pgfscope}%
\begin{pgfscope}%
\pgfpathrectangle{\pgfqpoint{0.556847in}{0.516222in}}{\pgfqpoint{1.722590in}{1.783528in}} %
\pgfusepath{clip}%
\pgfsetbuttcap%
\pgfsetroundjoin%
\definecolor{currentfill}{rgb}{0.298039,0.447059,0.690196}%
\pgfsetfillcolor{currentfill}%
\pgfsetlinewidth{0.240900pt}%
\definecolor{currentstroke}{rgb}{1.000000,1.000000,1.000000}%
\pgfsetstrokecolor{currentstroke}%
\pgfsetdash{}{0pt}%
\pgfpathmoveto{\pgfqpoint{1.787269in}{1.300493in}}%
\pgfpathcurveto{\pgfqpoint{1.795505in}{1.300493in}}{\pgfqpoint{1.803405in}{1.303765in}}{\pgfqpoint{1.809229in}{1.309589in}}%
\pgfpathcurveto{\pgfqpoint{1.815053in}{1.315413in}}{\pgfqpoint{1.818325in}{1.323313in}}{\pgfqpoint{1.818325in}{1.331549in}}%
\pgfpathcurveto{\pgfqpoint{1.818325in}{1.339785in}}{\pgfqpoint{1.815053in}{1.347686in}}{\pgfqpoint{1.809229in}{1.353509in}}%
\pgfpathcurveto{\pgfqpoint{1.803405in}{1.359333in}}{\pgfqpoint{1.795505in}{1.362606in}}{\pgfqpoint{1.787269in}{1.362606in}}%
\pgfpathcurveto{\pgfqpoint{1.779033in}{1.362606in}}{\pgfqpoint{1.771133in}{1.359333in}}{\pgfqpoint{1.765309in}{1.353509in}}%
\pgfpathcurveto{\pgfqpoint{1.759485in}{1.347686in}}{\pgfqpoint{1.756212in}{1.339785in}}{\pgfqpoint{1.756212in}{1.331549in}}%
\pgfpathcurveto{\pgfqpoint{1.756212in}{1.323313in}}{\pgfqpoint{1.759485in}{1.315413in}}{\pgfqpoint{1.765309in}{1.309589in}}%
\pgfpathcurveto{\pgfqpoint{1.771133in}{1.303765in}}{\pgfqpoint{1.779033in}{1.300493in}}{\pgfqpoint{1.787269in}{1.300493in}}%
\pgfpathclose%
\pgfusepath{stroke,fill}%
\end{pgfscope}%
\begin{pgfscope}%
\pgfpathrectangle{\pgfqpoint{0.556847in}{0.516222in}}{\pgfqpoint{1.722590in}{1.783528in}} %
\pgfusepath{clip}%
\pgfsetbuttcap%
\pgfsetroundjoin%
\definecolor{currentfill}{rgb}{0.298039,0.447059,0.690196}%
\pgfsetfillcolor{currentfill}%
\pgfsetlinewidth{0.240900pt}%
\definecolor{currentstroke}{rgb}{1.000000,1.000000,1.000000}%
\pgfsetstrokecolor{currentstroke}%
\pgfsetdash{}{0pt}%
\pgfpathmoveto{\pgfqpoint{2.033353in}{1.708156in}}%
\pgfpathcurveto{\pgfqpoint{2.041589in}{1.708156in}}{\pgfqpoint{2.049490in}{1.711429in}}{\pgfqpoint{2.055313in}{1.717252in}}%
\pgfpathcurveto{\pgfqpoint{2.061137in}{1.723076in}}{\pgfqpoint{2.064410in}{1.730976in}}{\pgfqpoint{2.064410in}{1.739213in}}%
\pgfpathcurveto{\pgfqpoint{2.064410in}{1.747449in}}{\pgfqpoint{2.061137in}{1.755349in}}{\pgfqpoint{2.055313in}{1.761173in}}%
\pgfpathcurveto{\pgfqpoint{2.049490in}{1.766997in}}{\pgfqpoint{2.041589in}{1.770269in}}{\pgfqpoint{2.033353in}{1.770269in}}%
\pgfpathcurveto{\pgfqpoint{2.025117in}{1.770269in}}{\pgfqpoint{2.017217in}{1.766997in}}{\pgfqpoint{2.011393in}{1.761173in}}%
\pgfpathcurveto{\pgfqpoint{2.005569in}{1.755349in}}{\pgfqpoint{2.002297in}{1.747449in}}{\pgfqpoint{2.002297in}{1.739213in}}%
\pgfpathcurveto{\pgfqpoint{2.002297in}{1.730976in}}{\pgfqpoint{2.005569in}{1.723076in}}{\pgfqpoint{2.011393in}{1.717252in}}%
\pgfpathcurveto{\pgfqpoint{2.017217in}{1.711429in}}{\pgfqpoint{2.025117in}{1.708156in}}{\pgfqpoint{2.033353in}{1.708156in}}%
\pgfpathclose%
\pgfusepath{stroke,fill}%
\end{pgfscope}%
\begin{pgfscope}%
\pgfpathrectangle{\pgfqpoint{0.556847in}{0.516222in}}{\pgfqpoint{1.722590in}{1.783528in}} %
\pgfusepath{clip}%
\pgfsetbuttcap%
\pgfsetroundjoin%
\definecolor{currentfill}{rgb}{0.298039,0.447059,0.690196}%
\pgfsetfillcolor{currentfill}%
\pgfsetlinewidth{0.240900pt}%
\definecolor{currentstroke}{rgb}{1.000000,1.000000,1.000000}%
\pgfsetstrokecolor{currentstroke}%
\pgfsetdash{}{0pt}%
\pgfpathmoveto{\pgfqpoint{1.491968in}{1.198577in}}%
\pgfpathcurveto{\pgfqpoint{1.500204in}{1.198577in}}{\pgfqpoint{1.508104in}{1.201849in}}{\pgfqpoint{1.513928in}{1.207673in}}%
\pgfpathcurveto{\pgfqpoint{1.519752in}{1.213497in}}{\pgfqpoint{1.523024in}{1.221397in}}{\pgfqpoint{1.523024in}{1.229633in}}%
\pgfpathcurveto{\pgfqpoint{1.523024in}{1.237870in}}{\pgfqpoint{1.519752in}{1.245770in}}{\pgfqpoint{1.513928in}{1.251594in}}%
\pgfpathcurveto{\pgfqpoint{1.508104in}{1.257418in}}{\pgfqpoint{1.500204in}{1.260690in}}{\pgfqpoint{1.491968in}{1.260690in}}%
\pgfpathcurveto{\pgfqpoint{1.483731in}{1.260690in}}{\pgfqpoint{1.475831in}{1.257418in}}{\pgfqpoint{1.470007in}{1.251594in}}%
\pgfpathcurveto{\pgfqpoint{1.464183in}{1.245770in}}{\pgfqpoint{1.460911in}{1.237870in}}{\pgfqpoint{1.460911in}{1.229633in}}%
\pgfpathcurveto{\pgfqpoint{1.460911in}{1.221397in}}{\pgfqpoint{1.464183in}{1.213497in}}{\pgfqpoint{1.470007in}{1.207673in}}%
\pgfpathcurveto{\pgfqpoint{1.475831in}{1.201849in}}{\pgfqpoint{1.483731in}{1.198577in}}{\pgfqpoint{1.491968in}{1.198577in}}%
\pgfpathclose%
\pgfusepath{stroke,fill}%
\end{pgfscope}%
\begin{pgfscope}%
\pgfpathrectangle{\pgfqpoint{0.556847in}{0.516222in}}{\pgfqpoint{1.722590in}{1.783528in}} %
\pgfusepath{clip}%
\pgfsetbuttcap%
\pgfsetroundjoin%
\definecolor{currentfill}{rgb}{0.298039,0.447059,0.690196}%
\pgfsetfillcolor{currentfill}%
\pgfsetlinewidth{0.240900pt}%
\definecolor{currentstroke}{rgb}{1.000000,1.000000,1.000000}%
\pgfsetstrokecolor{currentstroke}%
\pgfsetdash{}{0pt}%
\pgfpathmoveto{\pgfqpoint{2.181004in}{1.759114in}}%
\pgfpathcurveto{\pgfqpoint{2.189240in}{1.759114in}}{\pgfqpoint{2.197140in}{1.762386in}}{\pgfqpoint{2.202964in}{1.768210in}}%
\pgfpathcurveto{\pgfqpoint{2.208788in}{1.774034in}}{\pgfqpoint{2.212060in}{1.781934in}}{\pgfqpoint{2.212060in}{1.790171in}}%
\pgfpathcurveto{\pgfqpoint{2.212060in}{1.798407in}}{\pgfqpoint{2.208788in}{1.806307in}}{\pgfqpoint{2.202964in}{1.812131in}}%
\pgfpathcurveto{\pgfqpoint{2.197140in}{1.817955in}}{\pgfqpoint{2.189240in}{1.821227in}}{\pgfqpoint{2.181004in}{1.821227in}}%
\pgfpathcurveto{\pgfqpoint{2.172767in}{1.821227in}}{\pgfqpoint{2.164867in}{1.817955in}}{\pgfqpoint{2.159044in}{1.812131in}}%
\pgfpathcurveto{\pgfqpoint{2.153220in}{1.806307in}}{\pgfqpoint{2.149947in}{1.798407in}}{\pgfqpoint{2.149947in}{1.790171in}}%
\pgfpathcurveto{\pgfqpoint{2.149947in}{1.781934in}}{\pgfqpoint{2.153220in}{1.774034in}}{\pgfqpoint{2.159044in}{1.768210in}}%
\pgfpathcurveto{\pgfqpoint{2.164867in}{1.762386in}}{\pgfqpoint{2.172767in}{1.759114in}}{\pgfqpoint{2.181004in}{1.759114in}}%
\pgfpathclose%
\pgfusepath{stroke,fill}%
\end{pgfscope}%
\begin{pgfscope}%
\pgfpathrectangle{\pgfqpoint{0.556847in}{0.516222in}}{\pgfqpoint{1.722590in}{1.783528in}} %
\pgfusepath{clip}%
\pgfsetbuttcap%
\pgfsetroundjoin%
\definecolor{currentfill}{rgb}{0.298039,0.447059,0.690196}%
\pgfsetfillcolor{currentfill}%
\pgfsetlinewidth{0.240900pt}%
\definecolor{currentstroke}{rgb}{1.000000,1.000000,1.000000}%
\pgfsetstrokecolor{currentstroke}%
\pgfsetdash{}{0pt}%
\pgfpathmoveto{\pgfqpoint{1.491968in}{1.198577in}}%
\pgfpathcurveto{\pgfqpoint{1.500204in}{1.198577in}}{\pgfqpoint{1.508104in}{1.201849in}}{\pgfqpoint{1.513928in}{1.207673in}}%
\pgfpathcurveto{\pgfqpoint{1.519752in}{1.213497in}}{\pgfqpoint{1.523024in}{1.221397in}}{\pgfqpoint{1.523024in}{1.229633in}}%
\pgfpathcurveto{\pgfqpoint{1.523024in}{1.237870in}}{\pgfqpoint{1.519752in}{1.245770in}}{\pgfqpoint{1.513928in}{1.251594in}}%
\pgfpathcurveto{\pgfqpoint{1.508104in}{1.257418in}}{\pgfqpoint{1.500204in}{1.260690in}}{\pgfqpoint{1.491968in}{1.260690in}}%
\pgfpathcurveto{\pgfqpoint{1.483731in}{1.260690in}}{\pgfqpoint{1.475831in}{1.257418in}}{\pgfqpoint{1.470007in}{1.251594in}}%
\pgfpathcurveto{\pgfqpoint{1.464183in}{1.245770in}}{\pgfqpoint{1.460911in}{1.237870in}}{\pgfqpoint{1.460911in}{1.229633in}}%
\pgfpathcurveto{\pgfqpoint{1.460911in}{1.221397in}}{\pgfqpoint{1.464183in}{1.213497in}}{\pgfqpoint{1.470007in}{1.207673in}}%
\pgfpathcurveto{\pgfqpoint{1.475831in}{1.201849in}}{\pgfqpoint{1.483731in}{1.198577in}}{\pgfqpoint{1.491968in}{1.198577in}}%
\pgfpathclose%
\pgfusepath{stroke,fill}%
\end{pgfscope}%
\begin{pgfscope}%
\pgfpathrectangle{\pgfqpoint{0.556847in}{0.516222in}}{\pgfqpoint{1.722590in}{1.783528in}} %
\pgfusepath{clip}%
\pgfsetbuttcap%
\pgfsetroundjoin%
\definecolor{currentfill}{rgb}{0.298039,0.447059,0.690196}%
\pgfsetfillcolor{currentfill}%
\pgfsetlinewidth{0.240900pt}%
\definecolor{currentstroke}{rgb}{1.000000,1.000000,1.000000}%
\pgfsetstrokecolor{currentstroke}%
\pgfsetdash{}{0pt}%
\pgfpathmoveto{\pgfqpoint{2.181004in}{1.962946in}}%
\pgfpathcurveto{\pgfqpoint{2.189240in}{1.962946in}}{\pgfqpoint{2.197140in}{1.966218in}}{\pgfqpoint{2.202964in}{1.972042in}}%
\pgfpathcurveto{\pgfqpoint{2.208788in}{1.977866in}}{\pgfqpoint{2.212060in}{1.985766in}}{\pgfqpoint{2.212060in}{1.994002in}}%
\pgfpathcurveto{\pgfqpoint{2.212060in}{2.002239in}}{\pgfqpoint{2.208788in}{2.010139in}}{\pgfqpoint{2.202964in}{2.015963in}}%
\pgfpathcurveto{\pgfqpoint{2.197140in}{2.021787in}}{\pgfqpoint{2.189240in}{2.025059in}}{\pgfqpoint{2.181004in}{2.025059in}}%
\pgfpathcurveto{\pgfqpoint{2.172767in}{2.025059in}}{\pgfqpoint{2.164867in}{2.021787in}}{\pgfqpoint{2.159044in}{2.015963in}}%
\pgfpathcurveto{\pgfqpoint{2.153220in}{2.010139in}}{\pgfqpoint{2.149947in}{2.002239in}}{\pgfqpoint{2.149947in}{1.994002in}}%
\pgfpathcurveto{\pgfqpoint{2.149947in}{1.985766in}}{\pgfqpoint{2.153220in}{1.977866in}}{\pgfqpoint{2.159044in}{1.972042in}}%
\pgfpathcurveto{\pgfqpoint{2.164867in}{1.966218in}}{\pgfqpoint{2.172767in}{1.962946in}}{\pgfqpoint{2.181004in}{1.962946in}}%
\pgfpathclose%
\pgfusepath{stroke,fill}%
\end{pgfscope}%
\begin{pgfscope}%
\pgfpathrectangle{\pgfqpoint{0.556847in}{0.516222in}}{\pgfqpoint{1.722590in}{1.783528in}} %
\pgfusepath{clip}%
\pgfsetbuttcap%
\pgfsetroundjoin%
\definecolor{currentfill}{rgb}{0.298039,0.447059,0.690196}%
\pgfsetfillcolor{currentfill}%
\pgfsetlinewidth{0.240900pt}%
\definecolor{currentstroke}{rgb}{1.000000,1.000000,1.000000}%
\pgfsetstrokecolor{currentstroke}%
\pgfsetdash{}{0pt}%
\pgfpathmoveto{\pgfqpoint{1.098233in}{0.739955in}}%
\pgfpathcurveto{\pgfqpoint{1.106469in}{0.739955in}}{\pgfqpoint{1.114369in}{0.743228in}}{\pgfqpoint{1.120193in}{0.749052in}}%
\pgfpathcurveto{\pgfqpoint{1.126017in}{0.754876in}}{\pgfqpoint{1.129289in}{0.762776in}}{\pgfqpoint{1.129289in}{0.771012in}}%
\pgfpathcurveto{\pgfqpoint{1.129289in}{0.779248in}}{\pgfqpoint{1.126017in}{0.787148in}}{\pgfqpoint{1.120193in}{0.792972in}}%
\pgfpathcurveto{\pgfqpoint{1.114369in}{0.798796in}}{\pgfqpoint{1.106469in}{0.802068in}}{\pgfqpoint{1.098233in}{0.802068in}}%
\pgfpathcurveto{\pgfqpoint{1.089996in}{0.802068in}}{\pgfqpoint{1.082096in}{0.798796in}}{\pgfqpoint{1.076272in}{0.792972in}}%
\pgfpathcurveto{\pgfqpoint{1.070449in}{0.787148in}}{\pgfqpoint{1.067176in}{0.779248in}}{\pgfqpoint{1.067176in}{0.771012in}}%
\pgfpathcurveto{\pgfqpoint{1.067176in}{0.762776in}}{\pgfqpoint{1.070449in}{0.754876in}}{\pgfqpoint{1.076272in}{0.749052in}}%
\pgfpathcurveto{\pgfqpoint{1.082096in}{0.743228in}}{\pgfqpoint{1.089996in}{0.739955in}}{\pgfqpoint{1.098233in}{0.739955in}}%
\pgfpathclose%
\pgfusepath{stroke,fill}%
\end{pgfscope}%
\begin{pgfscope}%
\pgfpathrectangle{\pgfqpoint{0.556847in}{0.516222in}}{\pgfqpoint{1.722590in}{1.783528in}} %
\pgfusepath{clip}%
\pgfsetbuttcap%
\pgfsetroundjoin%
\definecolor{currentfill}{rgb}{0.298039,0.447059,0.690196}%
\pgfsetfillcolor{currentfill}%
\pgfsetlinewidth{0.240900pt}%
\definecolor{currentstroke}{rgb}{1.000000,1.000000,1.000000}%
\pgfsetstrokecolor{currentstroke}%
\pgfsetdash{}{0pt}%
\pgfpathmoveto{\pgfqpoint{1.541185in}{1.249535in}}%
\pgfpathcurveto{\pgfqpoint{1.549421in}{1.249535in}}{\pgfqpoint{1.557321in}{1.252807in}}{\pgfqpoint{1.563145in}{1.258631in}}%
\pgfpathcurveto{\pgfqpoint{1.568969in}{1.264455in}}{\pgfqpoint{1.572241in}{1.272355in}}{\pgfqpoint{1.572241in}{1.280591in}}%
\pgfpathcurveto{\pgfqpoint{1.572241in}{1.288828in}}{\pgfqpoint{1.568969in}{1.296728in}}{\pgfqpoint{1.563145in}{1.302552in}}%
\pgfpathcurveto{\pgfqpoint{1.557321in}{1.308375in}}{\pgfqpoint{1.549421in}{1.311648in}}{\pgfqpoint{1.541185in}{1.311648in}}%
\pgfpathcurveto{\pgfqpoint{1.532948in}{1.311648in}}{\pgfqpoint{1.525048in}{1.308375in}}{\pgfqpoint{1.519224in}{1.302552in}}%
\pgfpathcurveto{\pgfqpoint{1.513400in}{1.296728in}}{\pgfqpoint{1.510128in}{1.288828in}}{\pgfqpoint{1.510128in}{1.280591in}}%
\pgfpathcurveto{\pgfqpoint{1.510128in}{1.272355in}}{\pgfqpoint{1.513400in}{1.264455in}}{\pgfqpoint{1.519224in}{1.258631in}}%
\pgfpathcurveto{\pgfqpoint{1.525048in}{1.252807in}}{\pgfqpoint{1.532948in}{1.249535in}}{\pgfqpoint{1.541185in}{1.249535in}}%
\pgfpathclose%
\pgfusepath{stroke,fill}%
\end{pgfscope}%
\begin{pgfscope}%
\pgfpathrectangle{\pgfqpoint{0.556847in}{0.516222in}}{\pgfqpoint{1.722590in}{1.783528in}} %
\pgfusepath{clip}%
\pgfsetbuttcap%
\pgfsetroundjoin%
\definecolor{currentfill}{rgb}{0.298039,0.447059,0.690196}%
\pgfsetfillcolor{currentfill}%
\pgfsetlinewidth{0.240900pt}%
\definecolor{currentstroke}{rgb}{1.000000,1.000000,1.000000}%
\pgfsetstrokecolor{currentstroke}%
\pgfsetdash{}{0pt}%
\pgfpathmoveto{\pgfqpoint{1.393534in}{1.249535in}}%
\pgfpathcurveto{\pgfqpoint{1.401770in}{1.249535in}}{\pgfqpoint{1.409670in}{1.252807in}}{\pgfqpoint{1.415494in}{1.258631in}}%
\pgfpathcurveto{\pgfqpoint{1.421318in}{1.264455in}}{\pgfqpoint{1.424590in}{1.272355in}}{\pgfqpoint{1.424590in}{1.280591in}}%
\pgfpathcurveto{\pgfqpoint{1.424590in}{1.288828in}}{\pgfqpoint{1.421318in}{1.296728in}}{\pgfqpoint{1.415494in}{1.302552in}}%
\pgfpathcurveto{\pgfqpoint{1.409670in}{1.308375in}}{\pgfqpoint{1.401770in}{1.311648in}}{\pgfqpoint{1.393534in}{1.311648in}}%
\pgfpathcurveto{\pgfqpoint{1.385298in}{1.311648in}}{\pgfqpoint{1.377398in}{1.308375in}}{\pgfqpoint{1.371574in}{1.302552in}}%
\pgfpathcurveto{\pgfqpoint{1.365750in}{1.296728in}}{\pgfqpoint{1.362477in}{1.288828in}}{\pgfqpoint{1.362477in}{1.280591in}}%
\pgfpathcurveto{\pgfqpoint{1.362477in}{1.272355in}}{\pgfqpoint{1.365750in}{1.264455in}}{\pgfqpoint{1.371574in}{1.258631in}}%
\pgfpathcurveto{\pgfqpoint{1.377398in}{1.252807in}}{\pgfqpoint{1.385298in}{1.249535in}}{\pgfqpoint{1.393534in}{1.249535in}}%
\pgfpathclose%
\pgfusepath{stroke,fill}%
\end{pgfscope}%
\begin{pgfscope}%
\pgfpathrectangle{\pgfqpoint{0.556847in}{0.516222in}}{\pgfqpoint{1.722590in}{1.783528in}} %
\pgfusepath{clip}%
\pgfsetbuttcap%
\pgfsetroundjoin%
\definecolor{currentfill}{rgb}{0.298039,0.447059,0.690196}%
\pgfsetfillcolor{currentfill}%
\pgfsetlinewidth{0.240900pt}%
\definecolor{currentstroke}{rgb}{1.000000,1.000000,1.000000}%
\pgfsetstrokecolor{currentstroke}%
\pgfsetdash{}{0pt}%
\pgfpathmoveto{\pgfqpoint{1.541185in}{1.147619in}}%
\pgfpathcurveto{\pgfqpoint{1.549421in}{1.147619in}}{\pgfqpoint{1.557321in}{1.150891in}}{\pgfqpoint{1.563145in}{1.156715in}}%
\pgfpathcurveto{\pgfqpoint{1.568969in}{1.162539in}}{\pgfqpoint{1.572241in}{1.170439in}}{\pgfqpoint{1.572241in}{1.178675in}}%
\pgfpathcurveto{\pgfqpoint{1.572241in}{1.186912in}}{\pgfqpoint{1.568969in}{1.194812in}}{\pgfqpoint{1.563145in}{1.200636in}}%
\pgfpathcurveto{\pgfqpoint{1.557321in}{1.206460in}}{\pgfqpoint{1.549421in}{1.209732in}}{\pgfqpoint{1.541185in}{1.209732in}}%
\pgfpathcurveto{\pgfqpoint{1.532948in}{1.209732in}}{\pgfqpoint{1.525048in}{1.206460in}}{\pgfqpoint{1.519224in}{1.200636in}}%
\pgfpathcurveto{\pgfqpoint{1.513400in}{1.194812in}}{\pgfqpoint{1.510128in}{1.186912in}}{\pgfqpoint{1.510128in}{1.178675in}}%
\pgfpathcurveto{\pgfqpoint{1.510128in}{1.170439in}}{\pgfqpoint{1.513400in}{1.162539in}}{\pgfqpoint{1.519224in}{1.156715in}}%
\pgfpathcurveto{\pgfqpoint{1.525048in}{1.150891in}}{\pgfqpoint{1.532948in}{1.147619in}}{\pgfqpoint{1.541185in}{1.147619in}}%
\pgfpathclose%
\pgfusepath{stroke,fill}%
\end{pgfscope}%
\begin{pgfscope}%
\pgfpathrectangle{\pgfqpoint{0.556847in}{0.516222in}}{\pgfqpoint{1.722590in}{1.783528in}} %
\pgfusepath{clip}%
\pgfsetbuttcap%
\pgfsetroundjoin%
\definecolor{currentfill}{rgb}{0.298039,0.447059,0.690196}%
\pgfsetfillcolor{currentfill}%
\pgfsetlinewidth{0.240900pt}%
\definecolor{currentstroke}{rgb}{1.000000,1.000000,1.000000}%
\pgfsetstrokecolor{currentstroke}%
\pgfsetdash{}{0pt}%
\pgfpathmoveto{\pgfqpoint{1.639618in}{1.198577in}}%
\pgfpathcurveto{\pgfqpoint{1.647855in}{1.198577in}}{\pgfqpoint{1.655755in}{1.201849in}}{\pgfqpoint{1.661579in}{1.207673in}}%
\pgfpathcurveto{\pgfqpoint{1.667402in}{1.213497in}}{\pgfqpoint{1.670675in}{1.221397in}}{\pgfqpoint{1.670675in}{1.229633in}}%
\pgfpathcurveto{\pgfqpoint{1.670675in}{1.237870in}}{\pgfqpoint{1.667402in}{1.245770in}}{\pgfqpoint{1.661579in}{1.251594in}}%
\pgfpathcurveto{\pgfqpoint{1.655755in}{1.257418in}}{\pgfqpoint{1.647855in}{1.260690in}}{\pgfqpoint{1.639618in}{1.260690in}}%
\pgfpathcurveto{\pgfqpoint{1.631382in}{1.260690in}}{\pgfqpoint{1.623482in}{1.257418in}}{\pgfqpoint{1.617658in}{1.251594in}}%
\pgfpathcurveto{\pgfqpoint{1.611834in}{1.245770in}}{\pgfqpoint{1.608562in}{1.237870in}}{\pgfqpoint{1.608562in}{1.229633in}}%
\pgfpathcurveto{\pgfqpoint{1.608562in}{1.221397in}}{\pgfqpoint{1.611834in}{1.213497in}}{\pgfqpoint{1.617658in}{1.207673in}}%
\pgfpathcurveto{\pgfqpoint{1.623482in}{1.201849in}}{\pgfqpoint{1.631382in}{1.198577in}}{\pgfqpoint{1.639618in}{1.198577in}}%
\pgfpathclose%
\pgfusepath{stroke,fill}%
\end{pgfscope}%
\begin{pgfscope}%
\pgfpathrectangle{\pgfqpoint{0.556847in}{0.516222in}}{\pgfqpoint{1.722590in}{1.783528in}} %
\pgfusepath{clip}%
\pgfsetbuttcap%
\pgfsetroundjoin%
\definecolor{currentfill}{rgb}{0.298039,0.447059,0.690196}%
\pgfsetfillcolor{currentfill}%
\pgfsetlinewidth{0.240900pt}%
\definecolor{currentstroke}{rgb}{1.000000,1.000000,1.000000}%
\pgfsetstrokecolor{currentstroke}%
\pgfsetdash{}{0pt}%
\pgfpathmoveto{\pgfqpoint{0.999799in}{0.638040in}}%
\pgfpathcurveto{\pgfqpoint{1.008035in}{0.638040in}}{\pgfqpoint{1.015935in}{0.641312in}}{\pgfqpoint{1.021759in}{0.647136in}}%
\pgfpathcurveto{\pgfqpoint{1.027583in}{0.652960in}}{\pgfqpoint{1.030856in}{0.660860in}}{\pgfqpoint{1.030856in}{0.669096in}}%
\pgfpathcurveto{\pgfqpoint{1.030856in}{0.677332in}}{\pgfqpoint{1.027583in}{0.685232in}}{\pgfqpoint{1.021759in}{0.691056in}}%
\pgfpathcurveto{\pgfqpoint{1.015935in}{0.696880in}}{\pgfqpoint{1.008035in}{0.700153in}}{\pgfqpoint{0.999799in}{0.700153in}}%
\pgfpathcurveto{\pgfqpoint{0.991563in}{0.700153in}}{\pgfqpoint{0.983663in}{0.696880in}}{\pgfqpoint{0.977839in}{0.691056in}}%
\pgfpathcurveto{\pgfqpoint{0.972015in}{0.685232in}}{\pgfqpoint{0.968743in}{0.677332in}}{\pgfqpoint{0.968743in}{0.669096in}}%
\pgfpathcurveto{\pgfqpoint{0.968743in}{0.660860in}}{\pgfqpoint{0.972015in}{0.652960in}}{\pgfqpoint{0.977839in}{0.647136in}}%
\pgfpathcurveto{\pgfqpoint{0.983663in}{0.641312in}}{\pgfqpoint{0.991563in}{0.638040in}}{\pgfqpoint{0.999799in}{0.638040in}}%
\pgfpathclose%
\pgfusepath{stroke,fill}%
\end{pgfscope}%
\begin{pgfscope}%
\pgfpathrectangle{\pgfqpoint{0.556847in}{0.516222in}}{\pgfqpoint{1.722590in}{1.783528in}} %
\pgfusepath{clip}%
\pgfsetbuttcap%
\pgfsetroundjoin%
\definecolor{currentfill}{rgb}{0.298039,0.447059,0.690196}%
\pgfsetfillcolor{currentfill}%
\pgfsetlinewidth{0.240900pt}%
\definecolor{currentstroke}{rgb}{1.000000,1.000000,1.000000}%
\pgfsetstrokecolor{currentstroke}%
\pgfsetdash{}{0pt}%
\pgfpathmoveto{\pgfqpoint{1.885703in}{1.504324in}}%
\pgfpathcurveto{\pgfqpoint{1.893939in}{1.504324in}}{\pgfqpoint{1.901839in}{1.507597in}}{\pgfqpoint{1.907663in}{1.513421in}}%
\pgfpathcurveto{\pgfqpoint{1.913487in}{1.519245in}}{\pgfqpoint{1.916759in}{1.527145in}}{\pgfqpoint{1.916759in}{1.535381in}}%
\pgfpathcurveto{\pgfqpoint{1.916759in}{1.543617in}}{\pgfqpoint{1.913487in}{1.551517in}}{\pgfqpoint{1.907663in}{1.557341in}}%
\pgfpathcurveto{\pgfqpoint{1.901839in}{1.563165in}}{\pgfqpoint{1.893939in}{1.566437in}}{\pgfqpoint{1.885703in}{1.566437in}}%
\pgfpathcurveto{\pgfqpoint{1.877466in}{1.566437in}}{\pgfqpoint{1.869566in}{1.563165in}}{\pgfqpoint{1.863742in}{1.557341in}}%
\pgfpathcurveto{\pgfqpoint{1.857918in}{1.551517in}}{\pgfqpoint{1.854646in}{1.543617in}}{\pgfqpoint{1.854646in}{1.535381in}}%
\pgfpathcurveto{\pgfqpoint{1.854646in}{1.527145in}}{\pgfqpoint{1.857918in}{1.519245in}}{\pgfqpoint{1.863742in}{1.513421in}}%
\pgfpathcurveto{\pgfqpoint{1.869566in}{1.507597in}}{\pgfqpoint{1.877466in}{1.504324in}}{\pgfqpoint{1.885703in}{1.504324in}}%
\pgfpathclose%
\pgfusepath{stroke,fill}%
\end{pgfscope}%
\begin{pgfscope}%
\pgfpathrectangle{\pgfqpoint{0.556847in}{0.516222in}}{\pgfqpoint{1.722590in}{1.783528in}} %
\pgfusepath{clip}%
\pgfsetbuttcap%
\pgfsetroundjoin%
\definecolor{currentfill}{rgb}{0.298039,0.447059,0.690196}%
\pgfsetfillcolor{currentfill}%
\pgfsetlinewidth{0.240900pt}%
\definecolor{currentstroke}{rgb}{1.000000,1.000000,1.000000}%
\pgfsetstrokecolor{currentstroke}%
\pgfsetdash{}{0pt}%
\pgfpathmoveto{\pgfqpoint{1.787269in}{1.453367in}}%
\pgfpathcurveto{\pgfqpoint{1.795505in}{1.453367in}}{\pgfqpoint{1.803405in}{1.456639in}}{\pgfqpoint{1.809229in}{1.462463in}}%
\pgfpathcurveto{\pgfqpoint{1.815053in}{1.468287in}}{\pgfqpoint{1.818325in}{1.476187in}}{\pgfqpoint{1.818325in}{1.484423in}}%
\pgfpathcurveto{\pgfqpoint{1.818325in}{1.492659in}}{\pgfqpoint{1.815053in}{1.500559in}}{\pgfqpoint{1.809229in}{1.506383in}}%
\pgfpathcurveto{\pgfqpoint{1.803405in}{1.512207in}}{\pgfqpoint{1.795505in}{1.515480in}}{\pgfqpoint{1.787269in}{1.515480in}}%
\pgfpathcurveto{\pgfqpoint{1.779033in}{1.515480in}}{\pgfqpoint{1.771133in}{1.512207in}}{\pgfqpoint{1.765309in}{1.506383in}}%
\pgfpathcurveto{\pgfqpoint{1.759485in}{1.500559in}}{\pgfqpoint{1.756212in}{1.492659in}}{\pgfqpoint{1.756212in}{1.484423in}}%
\pgfpathcurveto{\pgfqpoint{1.756212in}{1.476187in}}{\pgfqpoint{1.759485in}{1.468287in}}{\pgfqpoint{1.765309in}{1.462463in}}%
\pgfpathcurveto{\pgfqpoint{1.771133in}{1.456639in}}{\pgfqpoint{1.779033in}{1.453367in}}{\pgfqpoint{1.787269in}{1.453367in}}%
\pgfpathclose%
\pgfusepath{stroke,fill}%
\end{pgfscope}%
\begin{pgfscope}%
\pgfpathrectangle{\pgfqpoint{0.556847in}{0.516222in}}{\pgfqpoint{1.722590in}{1.783528in}} %
\pgfusepath{clip}%
\pgfsetbuttcap%
\pgfsetroundjoin%
\definecolor{currentfill}{rgb}{0.298039,0.447059,0.690196}%
\pgfsetfillcolor{currentfill}%
\pgfsetlinewidth{0.240900pt}%
\definecolor{currentstroke}{rgb}{1.000000,1.000000,1.000000}%
\pgfsetstrokecolor{currentstroke}%
\pgfsetdash{}{0pt}%
\pgfpathmoveto{\pgfqpoint{1.393534in}{1.045703in}}%
\pgfpathcurveto{\pgfqpoint{1.401770in}{1.045703in}}{\pgfqpoint{1.409670in}{1.048975in}}{\pgfqpoint{1.415494in}{1.054799in}}%
\pgfpathcurveto{\pgfqpoint{1.421318in}{1.060623in}}{\pgfqpoint{1.424590in}{1.068523in}}{\pgfqpoint{1.424590in}{1.076760in}}%
\pgfpathcurveto{\pgfqpoint{1.424590in}{1.084996in}}{\pgfqpoint{1.421318in}{1.092896in}}{\pgfqpoint{1.415494in}{1.098720in}}%
\pgfpathcurveto{\pgfqpoint{1.409670in}{1.104544in}}{\pgfqpoint{1.401770in}{1.107816in}}{\pgfqpoint{1.393534in}{1.107816in}}%
\pgfpathcurveto{\pgfqpoint{1.385298in}{1.107816in}}{\pgfqpoint{1.377398in}{1.104544in}}{\pgfqpoint{1.371574in}{1.098720in}}%
\pgfpathcurveto{\pgfqpoint{1.365750in}{1.092896in}}{\pgfqpoint{1.362477in}{1.084996in}}{\pgfqpoint{1.362477in}{1.076760in}}%
\pgfpathcurveto{\pgfqpoint{1.362477in}{1.068523in}}{\pgfqpoint{1.365750in}{1.060623in}}{\pgfqpoint{1.371574in}{1.054799in}}%
\pgfpathcurveto{\pgfqpoint{1.377398in}{1.048975in}}{\pgfqpoint{1.385298in}{1.045703in}}{\pgfqpoint{1.393534in}{1.045703in}}%
\pgfpathclose%
\pgfusepath{stroke,fill}%
\end{pgfscope}%
\begin{pgfscope}%
\pgfpathrectangle{\pgfqpoint{0.556847in}{0.516222in}}{\pgfqpoint{1.722590in}{1.783528in}} %
\pgfusepath{clip}%
\pgfsetbuttcap%
\pgfsetroundjoin%
\definecolor{currentfill}{rgb}{0.298039,0.447059,0.690196}%
\pgfsetfillcolor{currentfill}%
\pgfsetlinewidth{0.240900pt}%
\definecolor{currentstroke}{rgb}{1.000000,1.000000,1.000000}%
\pgfsetstrokecolor{currentstroke}%
\pgfsetdash{}{0pt}%
\pgfpathmoveto{\pgfqpoint{1.541185in}{1.198577in}}%
\pgfpathcurveto{\pgfqpoint{1.549421in}{1.198577in}}{\pgfqpoint{1.557321in}{1.201849in}}{\pgfqpoint{1.563145in}{1.207673in}}%
\pgfpathcurveto{\pgfqpoint{1.568969in}{1.213497in}}{\pgfqpoint{1.572241in}{1.221397in}}{\pgfqpoint{1.572241in}{1.229633in}}%
\pgfpathcurveto{\pgfqpoint{1.572241in}{1.237870in}}{\pgfqpoint{1.568969in}{1.245770in}}{\pgfqpoint{1.563145in}{1.251594in}}%
\pgfpathcurveto{\pgfqpoint{1.557321in}{1.257418in}}{\pgfqpoint{1.549421in}{1.260690in}}{\pgfqpoint{1.541185in}{1.260690in}}%
\pgfpathcurveto{\pgfqpoint{1.532948in}{1.260690in}}{\pgfqpoint{1.525048in}{1.257418in}}{\pgfqpoint{1.519224in}{1.251594in}}%
\pgfpathcurveto{\pgfqpoint{1.513400in}{1.245770in}}{\pgfqpoint{1.510128in}{1.237870in}}{\pgfqpoint{1.510128in}{1.229633in}}%
\pgfpathcurveto{\pgfqpoint{1.510128in}{1.221397in}}{\pgfqpoint{1.513400in}{1.213497in}}{\pgfqpoint{1.519224in}{1.207673in}}%
\pgfpathcurveto{\pgfqpoint{1.525048in}{1.201849in}}{\pgfqpoint{1.532948in}{1.198577in}}{\pgfqpoint{1.541185in}{1.198577in}}%
\pgfpathclose%
\pgfusepath{stroke,fill}%
\end{pgfscope}%
\begin{pgfscope}%
\pgfpathrectangle{\pgfqpoint{0.556847in}{0.516222in}}{\pgfqpoint{1.722590in}{1.783528in}} %
\pgfusepath{clip}%
\pgfsetbuttcap%
\pgfsetroundjoin%
\definecolor{currentfill}{rgb}{0.298039,0.447059,0.690196}%
\pgfsetfillcolor{currentfill}%
\pgfsetlinewidth{0.240900pt}%
\definecolor{currentstroke}{rgb}{1.000000,1.000000,1.000000}%
\pgfsetstrokecolor{currentstroke}%
\pgfsetdash{}{0pt}%
\pgfpathmoveto{\pgfqpoint{1.196666in}{0.841871in}}%
\pgfpathcurveto{\pgfqpoint{1.204903in}{0.841871in}}{\pgfqpoint{1.212803in}{0.845144in}}{\pgfqpoint{1.218627in}{0.850968in}}%
\pgfpathcurveto{\pgfqpoint{1.224451in}{0.856791in}}{\pgfqpoint{1.227723in}{0.864691in}}{\pgfqpoint{1.227723in}{0.872928in}}%
\pgfpathcurveto{\pgfqpoint{1.227723in}{0.881164in}}{\pgfqpoint{1.224451in}{0.889064in}}{\pgfqpoint{1.218627in}{0.894888in}}%
\pgfpathcurveto{\pgfqpoint{1.212803in}{0.900712in}}{\pgfqpoint{1.204903in}{0.903984in}}{\pgfqpoint{1.196666in}{0.903984in}}%
\pgfpathcurveto{\pgfqpoint{1.188430in}{0.903984in}}{\pgfqpoint{1.180530in}{0.900712in}}{\pgfqpoint{1.174706in}{0.894888in}}%
\pgfpathcurveto{\pgfqpoint{1.168882in}{0.889064in}}{\pgfqpoint{1.165610in}{0.881164in}}{\pgfqpoint{1.165610in}{0.872928in}}%
\pgfpathcurveto{\pgfqpoint{1.165610in}{0.864691in}}{\pgfqpoint{1.168882in}{0.856791in}}{\pgfqpoint{1.174706in}{0.850968in}}%
\pgfpathcurveto{\pgfqpoint{1.180530in}{0.845144in}}{\pgfqpoint{1.188430in}{0.841871in}}{\pgfqpoint{1.196666in}{0.841871in}}%
\pgfpathclose%
\pgfusepath{stroke,fill}%
\end{pgfscope}%
\begin{pgfscope}%
\pgfpathrectangle{\pgfqpoint{0.556847in}{0.516222in}}{\pgfqpoint{1.722590in}{1.783528in}} %
\pgfusepath{clip}%
\pgfsetbuttcap%
\pgfsetroundjoin%
\definecolor{currentfill}{rgb}{0.298039,0.447059,0.690196}%
\pgfsetfillcolor{currentfill}%
\pgfsetlinewidth{0.240900pt}%
\definecolor{currentstroke}{rgb}{1.000000,1.000000,1.000000}%
\pgfsetstrokecolor{currentstroke}%
\pgfsetdash{}{0pt}%
\pgfpathmoveto{\pgfqpoint{1.344317in}{1.147619in}}%
\pgfpathcurveto{\pgfqpoint{1.352553in}{1.147619in}}{\pgfqpoint{1.360453in}{1.150891in}}{\pgfqpoint{1.366277in}{1.156715in}}%
\pgfpathcurveto{\pgfqpoint{1.372101in}{1.162539in}}{\pgfqpoint{1.375374in}{1.170439in}}{\pgfqpoint{1.375374in}{1.178675in}}%
\pgfpathcurveto{\pgfqpoint{1.375374in}{1.186912in}}{\pgfqpoint{1.372101in}{1.194812in}}{\pgfqpoint{1.366277in}{1.200636in}}%
\pgfpathcurveto{\pgfqpoint{1.360453in}{1.206460in}}{\pgfqpoint{1.352553in}{1.209732in}}{\pgfqpoint{1.344317in}{1.209732in}}%
\pgfpathcurveto{\pgfqpoint{1.336081in}{1.209732in}}{\pgfqpoint{1.328181in}{1.206460in}}{\pgfqpoint{1.322357in}{1.200636in}}%
\pgfpathcurveto{\pgfqpoint{1.316533in}{1.194812in}}{\pgfqpoint{1.313261in}{1.186912in}}{\pgfqpoint{1.313261in}{1.178675in}}%
\pgfpathcurveto{\pgfqpoint{1.313261in}{1.170439in}}{\pgfqpoint{1.316533in}{1.162539in}}{\pgfqpoint{1.322357in}{1.156715in}}%
\pgfpathcurveto{\pgfqpoint{1.328181in}{1.150891in}}{\pgfqpoint{1.336081in}{1.147619in}}{\pgfqpoint{1.344317in}{1.147619in}}%
\pgfpathclose%
\pgfusepath{stroke,fill}%
\end{pgfscope}%
\begin{pgfscope}%
\pgfpathrectangle{\pgfqpoint{0.556847in}{0.516222in}}{\pgfqpoint{1.722590in}{1.783528in}} %
\pgfusepath{clip}%
\pgfsetbuttcap%
\pgfsetroundjoin%
\definecolor{currentfill}{rgb}{0.298039,0.447059,0.690196}%
\pgfsetfillcolor{currentfill}%
\pgfsetlinewidth{0.240900pt}%
\definecolor{currentstroke}{rgb}{1.000000,1.000000,1.000000}%
\pgfsetstrokecolor{currentstroke}%
\pgfsetdash{}{0pt}%
\pgfpathmoveto{\pgfqpoint{1.295100in}{1.453367in}}%
\pgfpathcurveto{\pgfqpoint{1.303336in}{1.453367in}}{\pgfqpoint{1.311237in}{1.456639in}}{\pgfqpoint{1.317060in}{1.462463in}}%
\pgfpathcurveto{\pgfqpoint{1.322884in}{1.468287in}}{\pgfqpoint{1.326157in}{1.476187in}}{\pgfqpoint{1.326157in}{1.484423in}}%
\pgfpathcurveto{\pgfqpoint{1.326157in}{1.492659in}}{\pgfqpoint{1.322884in}{1.500559in}}{\pgfqpoint{1.317060in}{1.506383in}}%
\pgfpathcurveto{\pgfqpoint{1.311237in}{1.512207in}}{\pgfqpoint{1.303336in}{1.515480in}}{\pgfqpoint{1.295100in}{1.515480in}}%
\pgfpathcurveto{\pgfqpoint{1.286864in}{1.515480in}}{\pgfqpoint{1.278964in}{1.512207in}}{\pgfqpoint{1.273140in}{1.506383in}}%
\pgfpathcurveto{\pgfqpoint{1.267316in}{1.500559in}}{\pgfqpoint{1.264044in}{1.492659in}}{\pgfqpoint{1.264044in}{1.484423in}}%
\pgfpathcurveto{\pgfqpoint{1.264044in}{1.476187in}}{\pgfqpoint{1.267316in}{1.468287in}}{\pgfqpoint{1.273140in}{1.462463in}}%
\pgfpathcurveto{\pgfqpoint{1.278964in}{1.456639in}}{\pgfqpoint{1.286864in}{1.453367in}}{\pgfqpoint{1.295100in}{1.453367in}}%
\pgfpathclose%
\pgfusepath{stroke,fill}%
\end{pgfscope}%
\begin{pgfscope}%
\pgfpathrectangle{\pgfqpoint{0.556847in}{0.516222in}}{\pgfqpoint{1.722590in}{1.783528in}} %
\pgfusepath{clip}%
\pgfsetbuttcap%
\pgfsetroundjoin%
\definecolor{currentfill}{rgb}{0.298039,0.447059,0.690196}%
\pgfsetfillcolor{currentfill}%
\pgfsetlinewidth{0.240900pt}%
\definecolor{currentstroke}{rgb}{1.000000,1.000000,1.000000}%
\pgfsetstrokecolor{currentstroke}%
\pgfsetdash{}{0pt}%
\pgfpathmoveto{\pgfqpoint{1.836486in}{1.606240in}}%
\pgfpathcurveto{\pgfqpoint{1.844722in}{1.606240in}}{\pgfqpoint{1.852622in}{1.609513in}}{\pgfqpoint{1.858446in}{1.615337in}}%
\pgfpathcurveto{\pgfqpoint{1.864270in}{1.621160in}}{\pgfqpoint{1.867542in}{1.629061in}}{\pgfqpoint{1.867542in}{1.637297in}}%
\pgfpathcurveto{\pgfqpoint{1.867542in}{1.645533in}}{\pgfqpoint{1.864270in}{1.653433in}}{\pgfqpoint{1.858446in}{1.659257in}}%
\pgfpathcurveto{\pgfqpoint{1.852622in}{1.665081in}}{\pgfqpoint{1.844722in}{1.668353in}}{\pgfqpoint{1.836486in}{1.668353in}}%
\pgfpathcurveto{\pgfqpoint{1.828249in}{1.668353in}}{\pgfqpoint{1.820349in}{1.665081in}}{\pgfqpoint{1.814525in}{1.659257in}}%
\pgfpathcurveto{\pgfqpoint{1.808702in}{1.653433in}}{\pgfqpoint{1.805429in}{1.645533in}}{\pgfqpoint{1.805429in}{1.637297in}}%
\pgfpathcurveto{\pgfqpoint{1.805429in}{1.629061in}}{\pgfqpoint{1.808702in}{1.621160in}}{\pgfqpoint{1.814525in}{1.615337in}}%
\pgfpathcurveto{\pgfqpoint{1.820349in}{1.609513in}}{\pgfqpoint{1.828249in}{1.606240in}}{\pgfqpoint{1.836486in}{1.606240in}}%
\pgfpathclose%
\pgfusepath{stroke,fill}%
\end{pgfscope}%
\begin{pgfscope}%
\pgfpathrectangle{\pgfqpoint{0.556847in}{0.516222in}}{\pgfqpoint{1.722590in}{1.783528in}} %
\pgfusepath{clip}%
\pgfsetbuttcap%
\pgfsetroundjoin%
\definecolor{currentfill}{rgb}{0.298039,0.447059,0.690196}%
\pgfsetfillcolor{currentfill}%
\pgfsetlinewidth{0.240900pt}%
\definecolor{currentstroke}{rgb}{1.000000,1.000000,1.000000}%
\pgfsetstrokecolor{currentstroke}%
\pgfsetdash{}{0pt}%
\pgfpathmoveto{\pgfqpoint{1.688835in}{1.198577in}}%
\pgfpathcurveto{\pgfqpoint{1.697071in}{1.198577in}}{\pgfqpoint{1.704971in}{1.201849in}}{\pgfqpoint{1.710795in}{1.207673in}}%
\pgfpathcurveto{\pgfqpoint{1.716619in}{1.213497in}}{\pgfqpoint{1.719892in}{1.221397in}}{\pgfqpoint{1.719892in}{1.229633in}}%
\pgfpathcurveto{\pgfqpoint{1.719892in}{1.237870in}}{\pgfqpoint{1.716619in}{1.245770in}}{\pgfqpoint{1.710795in}{1.251594in}}%
\pgfpathcurveto{\pgfqpoint{1.704971in}{1.257418in}}{\pgfqpoint{1.697071in}{1.260690in}}{\pgfqpoint{1.688835in}{1.260690in}}%
\pgfpathcurveto{\pgfqpoint{1.680599in}{1.260690in}}{\pgfqpoint{1.672699in}{1.257418in}}{\pgfqpoint{1.666875in}{1.251594in}}%
\pgfpathcurveto{\pgfqpoint{1.661051in}{1.245770in}}{\pgfqpoint{1.657779in}{1.237870in}}{\pgfqpoint{1.657779in}{1.229633in}}%
\pgfpathcurveto{\pgfqpoint{1.657779in}{1.221397in}}{\pgfqpoint{1.661051in}{1.213497in}}{\pgfqpoint{1.666875in}{1.207673in}}%
\pgfpathcurveto{\pgfqpoint{1.672699in}{1.201849in}}{\pgfqpoint{1.680599in}{1.198577in}}{\pgfqpoint{1.688835in}{1.198577in}}%
\pgfpathclose%
\pgfusepath{stroke,fill}%
\end{pgfscope}%
\begin{pgfscope}%
\pgfpathrectangle{\pgfqpoint{0.556847in}{0.516222in}}{\pgfqpoint{1.722590in}{1.783528in}} %
\pgfusepath{clip}%
\pgfsetbuttcap%
\pgfsetroundjoin%
\definecolor{currentfill}{rgb}{0.298039,0.447059,0.690196}%
\pgfsetfillcolor{currentfill}%
\pgfsetlinewidth{0.240900pt}%
\definecolor{currentstroke}{rgb}{1.000000,1.000000,1.000000}%
\pgfsetstrokecolor{currentstroke}%
\pgfsetdash{}{0pt}%
\pgfpathmoveto{\pgfqpoint{1.147450in}{1.045703in}}%
\pgfpathcurveto{\pgfqpoint{1.155686in}{1.045703in}}{\pgfqpoint{1.163586in}{1.048975in}}{\pgfqpoint{1.169410in}{1.054799in}}%
\pgfpathcurveto{\pgfqpoint{1.175234in}{1.060623in}}{\pgfqpoint{1.178506in}{1.068523in}}{\pgfqpoint{1.178506in}{1.076760in}}%
\pgfpathcurveto{\pgfqpoint{1.178506in}{1.084996in}}{\pgfqpoint{1.175234in}{1.092896in}}{\pgfqpoint{1.169410in}{1.098720in}}%
\pgfpathcurveto{\pgfqpoint{1.163586in}{1.104544in}}{\pgfqpoint{1.155686in}{1.107816in}}{\pgfqpoint{1.147450in}{1.107816in}}%
\pgfpathcurveto{\pgfqpoint{1.139213in}{1.107816in}}{\pgfqpoint{1.131313in}{1.104544in}}{\pgfqpoint{1.125489in}{1.098720in}}%
\pgfpathcurveto{\pgfqpoint{1.119665in}{1.092896in}}{\pgfqpoint{1.116393in}{1.084996in}}{\pgfqpoint{1.116393in}{1.076760in}}%
\pgfpathcurveto{\pgfqpoint{1.116393in}{1.068523in}}{\pgfqpoint{1.119665in}{1.060623in}}{\pgfqpoint{1.125489in}{1.054799in}}%
\pgfpathcurveto{\pgfqpoint{1.131313in}{1.048975in}}{\pgfqpoint{1.139213in}{1.045703in}}{\pgfqpoint{1.147450in}{1.045703in}}%
\pgfpathclose%
\pgfusepath{stroke,fill}%
\end{pgfscope}%
\begin{pgfscope}%
\pgfpathrectangle{\pgfqpoint{0.556847in}{0.516222in}}{\pgfqpoint{1.722590in}{1.783528in}} %
\pgfusepath{clip}%
\pgfsetbuttcap%
\pgfsetroundjoin%
\definecolor{currentfill}{rgb}{0.298039,0.447059,0.690196}%
\pgfsetfillcolor{currentfill}%
\pgfsetlinewidth{0.240900pt}%
\definecolor{currentstroke}{rgb}{1.000000,1.000000,1.000000}%
\pgfsetstrokecolor{currentstroke}%
\pgfsetdash{}{0pt}%
\pgfpathmoveto{\pgfqpoint{1.984136in}{1.045703in}}%
\pgfpathcurveto{\pgfqpoint{1.992373in}{1.045703in}}{\pgfqpoint{2.000273in}{1.048975in}}{\pgfqpoint{2.006097in}{1.054799in}}%
\pgfpathcurveto{\pgfqpoint{2.011921in}{1.060623in}}{\pgfqpoint{2.015193in}{1.068523in}}{\pgfqpoint{2.015193in}{1.076760in}}%
\pgfpathcurveto{\pgfqpoint{2.015193in}{1.084996in}}{\pgfqpoint{2.011921in}{1.092896in}}{\pgfqpoint{2.006097in}{1.098720in}}%
\pgfpathcurveto{\pgfqpoint{2.000273in}{1.104544in}}{\pgfqpoint{1.992373in}{1.107816in}}{\pgfqpoint{1.984136in}{1.107816in}}%
\pgfpathcurveto{\pgfqpoint{1.975900in}{1.107816in}}{\pgfqpoint{1.968000in}{1.104544in}}{\pgfqpoint{1.962176in}{1.098720in}}%
\pgfpathcurveto{\pgfqpoint{1.956352in}{1.092896in}}{\pgfqpoint{1.953080in}{1.084996in}}{\pgfqpoint{1.953080in}{1.076760in}}%
\pgfpathcurveto{\pgfqpoint{1.953080in}{1.068523in}}{\pgfqpoint{1.956352in}{1.060623in}}{\pgfqpoint{1.962176in}{1.054799in}}%
\pgfpathcurveto{\pgfqpoint{1.968000in}{1.048975in}}{\pgfqpoint{1.975900in}{1.045703in}}{\pgfqpoint{1.984136in}{1.045703in}}%
\pgfpathclose%
\pgfusepath{stroke,fill}%
\end{pgfscope}%
\begin{pgfscope}%
\pgfpathrectangle{\pgfqpoint{0.556847in}{0.516222in}}{\pgfqpoint{1.722590in}{1.783528in}} %
\pgfusepath{clip}%
\pgfsetbuttcap%
\pgfsetroundjoin%
\definecolor{currentfill}{rgb}{0.298039,0.447059,0.690196}%
\pgfsetfillcolor{currentfill}%
\pgfsetlinewidth{0.240900pt}%
\definecolor{currentstroke}{rgb}{1.000000,1.000000,1.000000}%
\pgfsetstrokecolor{currentstroke}%
\pgfsetdash{}{0pt}%
\pgfpathmoveto{\pgfqpoint{2.082570in}{1.606240in}}%
\pgfpathcurveto{\pgfqpoint{2.090806in}{1.606240in}}{\pgfqpoint{2.098706in}{1.609513in}}{\pgfqpoint{2.104530in}{1.615337in}}%
\pgfpathcurveto{\pgfqpoint{2.110354in}{1.621160in}}{\pgfqpoint{2.113627in}{1.629061in}}{\pgfqpoint{2.113627in}{1.637297in}}%
\pgfpathcurveto{\pgfqpoint{2.113627in}{1.645533in}}{\pgfqpoint{2.110354in}{1.653433in}}{\pgfqpoint{2.104530in}{1.659257in}}%
\pgfpathcurveto{\pgfqpoint{2.098706in}{1.665081in}}{\pgfqpoint{2.090806in}{1.668353in}}{\pgfqpoint{2.082570in}{1.668353in}}%
\pgfpathcurveto{\pgfqpoint{2.074334in}{1.668353in}}{\pgfqpoint{2.066434in}{1.665081in}}{\pgfqpoint{2.060610in}{1.659257in}}%
\pgfpathcurveto{\pgfqpoint{2.054786in}{1.653433in}}{\pgfqpoint{2.051514in}{1.645533in}}{\pgfqpoint{2.051514in}{1.637297in}}%
\pgfpathcurveto{\pgfqpoint{2.051514in}{1.629061in}}{\pgfqpoint{2.054786in}{1.621160in}}{\pgfqpoint{2.060610in}{1.615337in}}%
\pgfpathcurveto{\pgfqpoint{2.066434in}{1.609513in}}{\pgfqpoint{2.074334in}{1.606240in}}{\pgfqpoint{2.082570in}{1.606240in}}%
\pgfpathclose%
\pgfusepath{stroke,fill}%
\end{pgfscope}%
\begin{pgfscope}%
\pgfpathrectangle{\pgfqpoint{0.556847in}{0.516222in}}{\pgfqpoint{1.722590in}{1.783528in}} %
\pgfusepath{clip}%
\pgfsetbuttcap%
\pgfsetroundjoin%
\definecolor{currentfill}{rgb}{0.298039,0.447059,0.690196}%
\pgfsetfillcolor{currentfill}%
\pgfsetlinewidth{0.240900pt}%
\definecolor{currentstroke}{rgb}{1.000000,1.000000,1.000000}%
\pgfsetstrokecolor{currentstroke}%
\pgfsetdash{}{0pt}%
\pgfpathmoveto{\pgfqpoint{1.245883in}{0.994745in}}%
\pgfpathcurveto{\pgfqpoint{1.254120in}{0.994745in}}{\pgfqpoint{1.262020in}{0.998017in}}{\pgfqpoint{1.267844in}{1.003841in}}%
\pgfpathcurveto{\pgfqpoint{1.273668in}{1.009665in}}{\pgfqpoint{1.276940in}{1.017565in}}{\pgfqpoint{1.276940in}{1.025802in}}%
\pgfpathcurveto{\pgfqpoint{1.276940in}{1.034038in}}{\pgfqpoint{1.273668in}{1.041938in}}{\pgfqpoint{1.267844in}{1.047762in}}%
\pgfpathcurveto{\pgfqpoint{1.262020in}{1.053586in}}{\pgfqpoint{1.254120in}{1.056858in}}{\pgfqpoint{1.245883in}{1.056858in}}%
\pgfpathcurveto{\pgfqpoint{1.237647in}{1.056858in}}{\pgfqpoint{1.229747in}{1.053586in}}{\pgfqpoint{1.223923in}{1.047762in}}%
\pgfpathcurveto{\pgfqpoint{1.218099in}{1.041938in}}{\pgfqpoint{1.214827in}{1.034038in}}{\pgfqpoint{1.214827in}{1.025802in}}%
\pgfpathcurveto{\pgfqpoint{1.214827in}{1.017565in}}{\pgfqpoint{1.218099in}{1.009665in}}{\pgfqpoint{1.223923in}{1.003841in}}%
\pgfpathcurveto{\pgfqpoint{1.229747in}{0.998017in}}{\pgfqpoint{1.237647in}{0.994745in}}{\pgfqpoint{1.245883in}{0.994745in}}%
\pgfpathclose%
\pgfusepath{stroke,fill}%
\end{pgfscope}%
\begin{pgfscope}%
\pgfpathrectangle{\pgfqpoint{0.556847in}{0.516222in}}{\pgfqpoint{1.722590in}{1.783528in}} %
\pgfusepath{clip}%
\pgfsetbuttcap%
\pgfsetroundjoin%
\definecolor{currentfill}{rgb}{0.298039,0.447059,0.690196}%
\pgfsetfillcolor{currentfill}%
\pgfsetlinewidth{0.240900pt}%
\definecolor{currentstroke}{rgb}{1.000000,1.000000,1.000000}%
\pgfsetstrokecolor{currentstroke}%
\pgfsetdash{}{0pt}%
\pgfpathmoveto{\pgfqpoint{0.802932in}{1.096661in}}%
\pgfpathcurveto{\pgfqpoint{0.811168in}{1.096661in}}{\pgfqpoint{0.819068in}{1.099933in}}{\pgfqpoint{0.824892in}{1.105757in}}%
\pgfpathcurveto{\pgfqpoint{0.830716in}{1.111581in}}{\pgfqpoint{0.833988in}{1.119481in}}{\pgfqpoint{0.833988in}{1.127717in}}%
\pgfpathcurveto{\pgfqpoint{0.833988in}{1.135954in}}{\pgfqpoint{0.830716in}{1.143854in}}{\pgfqpoint{0.824892in}{1.149678in}}%
\pgfpathcurveto{\pgfqpoint{0.819068in}{1.155502in}}{\pgfqpoint{0.811168in}{1.158774in}}{\pgfqpoint{0.802932in}{1.158774in}}%
\pgfpathcurveto{\pgfqpoint{0.794695in}{1.158774in}}{\pgfqpoint{0.786795in}{1.155502in}}{\pgfqpoint{0.780971in}{1.149678in}}%
\pgfpathcurveto{\pgfqpoint{0.775147in}{1.143854in}}{\pgfqpoint{0.771875in}{1.135954in}}{\pgfqpoint{0.771875in}{1.127717in}}%
\pgfpathcurveto{\pgfqpoint{0.771875in}{1.119481in}}{\pgfqpoint{0.775147in}{1.111581in}}{\pgfqpoint{0.780971in}{1.105757in}}%
\pgfpathcurveto{\pgfqpoint{0.786795in}{1.099933in}}{\pgfqpoint{0.794695in}{1.096661in}}{\pgfqpoint{0.802932in}{1.096661in}}%
\pgfpathclose%
\pgfusepath{stroke,fill}%
\end{pgfscope}%
\begin{pgfscope}%
\pgfpathrectangle{\pgfqpoint{0.556847in}{0.516222in}}{\pgfqpoint{1.722590in}{1.783528in}} %
\pgfusepath{clip}%
\pgfsetbuttcap%
\pgfsetroundjoin%
\definecolor{currentfill}{rgb}{0.298039,0.447059,0.690196}%
\pgfsetfillcolor{currentfill}%
\pgfsetlinewidth{0.240900pt}%
\definecolor{currentstroke}{rgb}{1.000000,1.000000,1.000000}%
\pgfsetstrokecolor{currentstroke}%
\pgfsetdash{}{0pt}%
\pgfpathmoveto{\pgfqpoint{1.196666in}{0.892829in}}%
\pgfpathcurveto{\pgfqpoint{1.204903in}{0.892829in}}{\pgfqpoint{1.212803in}{0.896102in}}{\pgfqpoint{1.218627in}{0.901925in}}%
\pgfpathcurveto{\pgfqpoint{1.224451in}{0.907749in}}{\pgfqpoint{1.227723in}{0.915649in}}{\pgfqpoint{1.227723in}{0.923886in}}%
\pgfpathcurveto{\pgfqpoint{1.227723in}{0.932122in}}{\pgfqpoint{1.224451in}{0.940022in}}{\pgfqpoint{1.218627in}{0.945846in}}%
\pgfpathcurveto{\pgfqpoint{1.212803in}{0.951670in}}{\pgfqpoint{1.204903in}{0.954942in}}{\pgfqpoint{1.196666in}{0.954942in}}%
\pgfpathcurveto{\pgfqpoint{1.188430in}{0.954942in}}{\pgfqpoint{1.180530in}{0.951670in}}{\pgfqpoint{1.174706in}{0.945846in}}%
\pgfpathcurveto{\pgfqpoint{1.168882in}{0.940022in}}{\pgfqpoint{1.165610in}{0.932122in}}{\pgfqpoint{1.165610in}{0.923886in}}%
\pgfpathcurveto{\pgfqpoint{1.165610in}{0.915649in}}{\pgfqpoint{1.168882in}{0.907749in}}{\pgfqpoint{1.174706in}{0.901925in}}%
\pgfpathcurveto{\pgfqpoint{1.180530in}{0.896102in}}{\pgfqpoint{1.188430in}{0.892829in}}{\pgfqpoint{1.196666in}{0.892829in}}%
\pgfpathclose%
\pgfusepath{stroke,fill}%
\end{pgfscope}%
\begin{pgfscope}%
\pgfpathrectangle{\pgfqpoint{0.556847in}{0.516222in}}{\pgfqpoint{1.722590in}{1.783528in}} %
\pgfusepath{clip}%
\pgfsetbuttcap%
\pgfsetroundjoin%
\definecolor{currentfill}{rgb}{0.298039,0.447059,0.690196}%
\pgfsetfillcolor{currentfill}%
\pgfsetlinewidth{0.240900pt}%
\definecolor{currentstroke}{rgb}{1.000000,1.000000,1.000000}%
\pgfsetstrokecolor{currentstroke}%
\pgfsetdash{}{0pt}%
\pgfpathmoveto{\pgfqpoint{1.984136in}{1.759114in}}%
\pgfpathcurveto{\pgfqpoint{1.992373in}{1.759114in}}{\pgfqpoint{2.000273in}{1.762386in}}{\pgfqpoint{2.006097in}{1.768210in}}%
\pgfpathcurveto{\pgfqpoint{2.011921in}{1.774034in}}{\pgfqpoint{2.015193in}{1.781934in}}{\pgfqpoint{2.015193in}{1.790171in}}%
\pgfpathcurveto{\pgfqpoint{2.015193in}{1.798407in}}{\pgfqpoint{2.011921in}{1.806307in}}{\pgfqpoint{2.006097in}{1.812131in}}%
\pgfpathcurveto{\pgfqpoint{2.000273in}{1.817955in}}{\pgfqpoint{1.992373in}{1.821227in}}{\pgfqpoint{1.984136in}{1.821227in}}%
\pgfpathcurveto{\pgfqpoint{1.975900in}{1.821227in}}{\pgfqpoint{1.968000in}{1.817955in}}{\pgfqpoint{1.962176in}{1.812131in}}%
\pgfpathcurveto{\pgfqpoint{1.956352in}{1.806307in}}{\pgfqpoint{1.953080in}{1.798407in}}{\pgfqpoint{1.953080in}{1.790171in}}%
\pgfpathcurveto{\pgfqpoint{1.953080in}{1.781934in}}{\pgfqpoint{1.956352in}{1.774034in}}{\pgfqpoint{1.962176in}{1.768210in}}%
\pgfpathcurveto{\pgfqpoint{1.968000in}{1.762386in}}{\pgfqpoint{1.975900in}{1.759114in}}{\pgfqpoint{1.984136in}{1.759114in}}%
\pgfpathclose%
\pgfusepath{stroke,fill}%
\end{pgfscope}%
\begin{pgfscope}%
\pgfsetrectcap%
\pgfsetmiterjoin%
\pgfsetlinewidth{0.000000pt}%
\definecolor{currentstroke}{rgb}{1.000000,1.000000,1.000000}%
\pgfsetstrokecolor{currentstroke}%
\pgfsetdash{}{0pt}%
\pgfpathmoveto{\pgfqpoint{0.556847in}{0.516222in}}%
\pgfpathlineto{\pgfqpoint{0.556847in}{2.299750in}}%
\pgfusepath{}%
\end{pgfscope}%
\begin{pgfscope}%
\pgfsetrectcap%
\pgfsetmiterjoin%
\pgfsetlinewidth{0.000000pt}%
\definecolor{currentstroke}{rgb}{1.000000,1.000000,1.000000}%
\pgfsetstrokecolor{currentstroke}%
\pgfsetdash{}{0pt}%
\pgfpathmoveto{\pgfqpoint{0.556847in}{0.516222in}}%
\pgfpathlineto{\pgfqpoint{2.279437in}{0.516222in}}%
\pgfusepath{}%
\end{pgfscope}%
\end{pgfpicture}%
\makeatother%
\endgroup%

    \caption{Comparison between the times meassured in the two realizations.}
    \label{fig_t1t2}
  \end{subfigure}
  \begin{subfigure}[h]{.5\linewidth}
    %% Creator: Matplotlib, PGF backend
%%
%% To include the figure in your LaTeX document, write
%%   \input{<filename>.pgf}
%%
%% Make sure the required packages are loaded in your preamble
%%   \usepackage{pgf}
%%
%% Figures using additional raster images can only be included by \input if
%% they are in the same directory as the main LaTeX file. For loading figures
%% from other directories you can use the `import` package
%%   \usepackage{import}
%% and then include the figures with
%%   \import{<path to file>}{<filename>.pgf}
%%
%% Matplotlib used the following preamble
%%   \usepackage[utf8x]{inputenc}
%%   \usepackage[T1]{fontenc}
%%   \usepackage{cmbright}
%%
\begingroup%
\makeatletter%
\begin{pgfpicture}%
\pgfpathrectangle{\pgfpointorigin}{\pgfqpoint{2.500000in}{2.500000in}}%
\pgfusepath{use as bounding box, clip}%
\begin{pgfscope}%
\pgfsetbuttcap%
\pgfsetmiterjoin%
\definecolor{currentfill}{rgb}{1.000000,1.000000,1.000000}%
\pgfsetfillcolor{currentfill}%
\pgfsetlinewidth{0.000000pt}%
\definecolor{currentstroke}{rgb}{1.000000,1.000000,1.000000}%
\pgfsetstrokecolor{currentstroke}%
\pgfsetdash{}{0pt}%
\pgfpathmoveto{\pgfqpoint{0.000000in}{0.000000in}}%
\pgfpathlineto{\pgfqpoint{2.500000in}{0.000000in}}%
\pgfpathlineto{\pgfqpoint{2.500000in}{2.500000in}}%
\pgfpathlineto{\pgfqpoint{0.000000in}{2.500000in}}%
\pgfpathclose%
\pgfusepath{fill}%
\end{pgfscope}%
\begin{pgfscope}%
\pgfsetbuttcap%
\pgfsetmiterjoin%
\definecolor{currentfill}{rgb}{0.917647,0.917647,0.949020}%
\pgfsetfillcolor{currentfill}%
\pgfsetlinewidth{0.000000pt}%
\definecolor{currentstroke}{rgb}{0.000000,0.000000,0.000000}%
\pgfsetstrokecolor{currentstroke}%
\pgfsetstrokeopacity{0.000000}%
\pgfsetdash{}{0pt}%
\pgfpathmoveto{\pgfqpoint{0.556847in}{0.516222in}}%
\pgfpathlineto{\pgfqpoint{2.279437in}{0.516222in}}%
\pgfpathlineto{\pgfqpoint{2.279437in}{2.299750in}}%
\pgfpathlineto{\pgfqpoint{0.556847in}{2.299750in}}%
\pgfpathclose%
\pgfusepath{fill}%
\end{pgfscope}%
\begin{pgfscope}%
\pgfpathrectangle{\pgfqpoint{0.556847in}{0.516222in}}{\pgfqpoint{1.722590in}{1.783528in}} %
\pgfusepath{clip}%
\pgfsetroundcap%
\pgfsetroundjoin%
\pgfsetlinewidth{0.803000pt}%
\definecolor{currentstroke}{rgb}{1.000000,1.000000,1.000000}%
\pgfsetstrokecolor{currentstroke}%
\pgfsetdash{}{0pt}%
\pgfpathmoveto{\pgfqpoint{0.556847in}{0.516222in}}%
\pgfpathlineto{\pgfqpoint{0.556847in}{2.299750in}}%
\pgfusepath{stroke}%
\end{pgfscope}%
\begin{pgfscope}%
\pgfsetbuttcap%
\pgfsetroundjoin%
\definecolor{currentfill}{rgb}{0.150000,0.150000,0.150000}%
\pgfsetfillcolor{currentfill}%
\pgfsetlinewidth{0.803000pt}%
\definecolor{currentstroke}{rgb}{0.150000,0.150000,0.150000}%
\pgfsetstrokecolor{currentstroke}%
\pgfsetdash{}{0pt}%
\pgfsys@defobject{currentmarker}{\pgfqpoint{0.000000in}{0.000000in}}{\pgfqpoint{0.000000in}{0.000000in}}{%
\pgfpathmoveto{\pgfqpoint{0.000000in}{0.000000in}}%
\pgfpathlineto{\pgfqpoint{0.000000in}{0.000000in}}%
\pgfusepath{stroke,fill}%
}%
\begin{pgfscope}%
\pgfsys@transformshift{0.556847in}{0.516222in}%
\pgfsys@useobject{currentmarker}{}%
\end{pgfscope}%
\end{pgfscope}%
\begin{pgfscope}%
\definecolor{textcolor}{rgb}{0.150000,0.150000,0.150000}%
\pgfsetstrokecolor{textcolor}%
\pgfsetfillcolor{textcolor}%
\pgftext[x=0.556847in,y=0.438444in,,top]{\color{textcolor}\sffamily\fontsize{8.000000}{9.600000}\selectfont 5}%
\end{pgfscope}%
\begin{pgfscope}%
\pgfpathrectangle{\pgfqpoint{0.556847in}{0.516222in}}{\pgfqpoint{1.722590in}{1.783528in}} %
\pgfusepath{clip}%
\pgfsetroundcap%
\pgfsetroundjoin%
\pgfsetlinewidth{0.803000pt}%
\definecolor{currentstroke}{rgb}{1.000000,1.000000,1.000000}%
\pgfsetstrokecolor{currentstroke}%
\pgfsetdash{}{0pt}%
\pgfpathmoveto{\pgfqpoint{0.843946in}{0.516222in}}%
\pgfpathlineto{\pgfqpoint{0.843946in}{2.299750in}}%
\pgfusepath{stroke}%
\end{pgfscope}%
\begin{pgfscope}%
\pgfsetbuttcap%
\pgfsetroundjoin%
\definecolor{currentfill}{rgb}{0.150000,0.150000,0.150000}%
\pgfsetfillcolor{currentfill}%
\pgfsetlinewidth{0.803000pt}%
\definecolor{currentstroke}{rgb}{0.150000,0.150000,0.150000}%
\pgfsetstrokecolor{currentstroke}%
\pgfsetdash{}{0pt}%
\pgfsys@defobject{currentmarker}{\pgfqpoint{0.000000in}{0.000000in}}{\pgfqpoint{0.000000in}{0.000000in}}{%
\pgfpathmoveto{\pgfqpoint{0.000000in}{0.000000in}}%
\pgfpathlineto{\pgfqpoint{0.000000in}{0.000000in}}%
\pgfusepath{stroke,fill}%
}%
\begin{pgfscope}%
\pgfsys@transformshift{0.843946in}{0.516222in}%
\pgfsys@useobject{currentmarker}{}%
\end{pgfscope}%
\end{pgfscope}%
\begin{pgfscope}%
\definecolor{textcolor}{rgb}{0.150000,0.150000,0.150000}%
\pgfsetstrokecolor{textcolor}%
\pgfsetfillcolor{textcolor}%
\pgftext[x=0.843946in,y=0.438444in,,top]{\color{textcolor}\sffamily\fontsize{8.000000}{9.600000}\selectfont 6}%
\end{pgfscope}%
\begin{pgfscope}%
\pgfpathrectangle{\pgfqpoint{0.556847in}{0.516222in}}{\pgfqpoint{1.722590in}{1.783528in}} %
\pgfusepath{clip}%
\pgfsetroundcap%
\pgfsetroundjoin%
\pgfsetlinewidth{0.803000pt}%
\definecolor{currentstroke}{rgb}{1.000000,1.000000,1.000000}%
\pgfsetstrokecolor{currentstroke}%
\pgfsetdash{}{0pt}%
\pgfpathmoveto{\pgfqpoint{1.131044in}{0.516222in}}%
\pgfpathlineto{\pgfqpoint{1.131044in}{2.299750in}}%
\pgfusepath{stroke}%
\end{pgfscope}%
\begin{pgfscope}%
\pgfsetbuttcap%
\pgfsetroundjoin%
\definecolor{currentfill}{rgb}{0.150000,0.150000,0.150000}%
\pgfsetfillcolor{currentfill}%
\pgfsetlinewidth{0.803000pt}%
\definecolor{currentstroke}{rgb}{0.150000,0.150000,0.150000}%
\pgfsetstrokecolor{currentstroke}%
\pgfsetdash{}{0pt}%
\pgfsys@defobject{currentmarker}{\pgfqpoint{0.000000in}{0.000000in}}{\pgfqpoint{0.000000in}{0.000000in}}{%
\pgfpathmoveto{\pgfqpoint{0.000000in}{0.000000in}}%
\pgfpathlineto{\pgfqpoint{0.000000in}{0.000000in}}%
\pgfusepath{stroke,fill}%
}%
\begin{pgfscope}%
\pgfsys@transformshift{1.131044in}{0.516222in}%
\pgfsys@useobject{currentmarker}{}%
\end{pgfscope}%
\end{pgfscope}%
\begin{pgfscope}%
\definecolor{textcolor}{rgb}{0.150000,0.150000,0.150000}%
\pgfsetstrokecolor{textcolor}%
\pgfsetfillcolor{textcolor}%
\pgftext[x=1.131044in,y=0.438444in,,top]{\color{textcolor}\sffamily\fontsize{8.000000}{9.600000}\selectfont 7}%
\end{pgfscope}%
\begin{pgfscope}%
\pgfpathrectangle{\pgfqpoint{0.556847in}{0.516222in}}{\pgfqpoint{1.722590in}{1.783528in}} %
\pgfusepath{clip}%
\pgfsetroundcap%
\pgfsetroundjoin%
\pgfsetlinewidth{0.803000pt}%
\definecolor{currentstroke}{rgb}{1.000000,1.000000,1.000000}%
\pgfsetstrokecolor{currentstroke}%
\pgfsetdash{}{0pt}%
\pgfpathmoveto{\pgfqpoint{1.418142in}{0.516222in}}%
\pgfpathlineto{\pgfqpoint{1.418142in}{2.299750in}}%
\pgfusepath{stroke}%
\end{pgfscope}%
\begin{pgfscope}%
\pgfsetbuttcap%
\pgfsetroundjoin%
\definecolor{currentfill}{rgb}{0.150000,0.150000,0.150000}%
\pgfsetfillcolor{currentfill}%
\pgfsetlinewidth{0.803000pt}%
\definecolor{currentstroke}{rgb}{0.150000,0.150000,0.150000}%
\pgfsetstrokecolor{currentstroke}%
\pgfsetdash{}{0pt}%
\pgfsys@defobject{currentmarker}{\pgfqpoint{0.000000in}{0.000000in}}{\pgfqpoint{0.000000in}{0.000000in}}{%
\pgfpathmoveto{\pgfqpoint{0.000000in}{0.000000in}}%
\pgfpathlineto{\pgfqpoint{0.000000in}{0.000000in}}%
\pgfusepath{stroke,fill}%
}%
\begin{pgfscope}%
\pgfsys@transformshift{1.418142in}{0.516222in}%
\pgfsys@useobject{currentmarker}{}%
\end{pgfscope}%
\end{pgfscope}%
\begin{pgfscope}%
\definecolor{textcolor}{rgb}{0.150000,0.150000,0.150000}%
\pgfsetstrokecolor{textcolor}%
\pgfsetfillcolor{textcolor}%
\pgftext[x=1.418142in,y=0.438444in,,top]{\color{textcolor}\sffamily\fontsize{8.000000}{9.600000}\selectfont 8}%
\end{pgfscope}%
\begin{pgfscope}%
\pgfpathrectangle{\pgfqpoint{0.556847in}{0.516222in}}{\pgfqpoint{1.722590in}{1.783528in}} %
\pgfusepath{clip}%
\pgfsetroundcap%
\pgfsetroundjoin%
\pgfsetlinewidth{0.803000pt}%
\definecolor{currentstroke}{rgb}{1.000000,1.000000,1.000000}%
\pgfsetstrokecolor{currentstroke}%
\pgfsetdash{}{0pt}%
\pgfpathmoveto{\pgfqpoint{1.705241in}{0.516222in}}%
\pgfpathlineto{\pgfqpoint{1.705241in}{2.299750in}}%
\pgfusepath{stroke}%
\end{pgfscope}%
\begin{pgfscope}%
\pgfsetbuttcap%
\pgfsetroundjoin%
\definecolor{currentfill}{rgb}{0.150000,0.150000,0.150000}%
\pgfsetfillcolor{currentfill}%
\pgfsetlinewidth{0.803000pt}%
\definecolor{currentstroke}{rgb}{0.150000,0.150000,0.150000}%
\pgfsetstrokecolor{currentstroke}%
\pgfsetdash{}{0pt}%
\pgfsys@defobject{currentmarker}{\pgfqpoint{0.000000in}{0.000000in}}{\pgfqpoint{0.000000in}{0.000000in}}{%
\pgfpathmoveto{\pgfqpoint{0.000000in}{0.000000in}}%
\pgfpathlineto{\pgfqpoint{0.000000in}{0.000000in}}%
\pgfusepath{stroke,fill}%
}%
\begin{pgfscope}%
\pgfsys@transformshift{1.705241in}{0.516222in}%
\pgfsys@useobject{currentmarker}{}%
\end{pgfscope}%
\end{pgfscope}%
\begin{pgfscope}%
\definecolor{textcolor}{rgb}{0.150000,0.150000,0.150000}%
\pgfsetstrokecolor{textcolor}%
\pgfsetfillcolor{textcolor}%
\pgftext[x=1.705241in,y=0.438444in,,top]{\color{textcolor}\sffamily\fontsize{8.000000}{9.600000}\selectfont 9}%
\end{pgfscope}%
\begin{pgfscope}%
\pgfpathrectangle{\pgfqpoint{0.556847in}{0.516222in}}{\pgfqpoint{1.722590in}{1.783528in}} %
\pgfusepath{clip}%
\pgfsetroundcap%
\pgfsetroundjoin%
\pgfsetlinewidth{0.803000pt}%
\definecolor{currentstroke}{rgb}{1.000000,1.000000,1.000000}%
\pgfsetstrokecolor{currentstroke}%
\pgfsetdash{}{0pt}%
\pgfpathmoveto{\pgfqpoint{1.992339in}{0.516222in}}%
\pgfpathlineto{\pgfqpoint{1.992339in}{2.299750in}}%
\pgfusepath{stroke}%
\end{pgfscope}%
\begin{pgfscope}%
\pgfsetbuttcap%
\pgfsetroundjoin%
\definecolor{currentfill}{rgb}{0.150000,0.150000,0.150000}%
\pgfsetfillcolor{currentfill}%
\pgfsetlinewidth{0.803000pt}%
\definecolor{currentstroke}{rgb}{0.150000,0.150000,0.150000}%
\pgfsetstrokecolor{currentstroke}%
\pgfsetdash{}{0pt}%
\pgfsys@defobject{currentmarker}{\pgfqpoint{0.000000in}{0.000000in}}{\pgfqpoint{0.000000in}{0.000000in}}{%
\pgfpathmoveto{\pgfqpoint{0.000000in}{0.000000in}}%
\pgfpathlineto{\pgfqpoint{0.000000in}{0.000000in}}%
\pgfusepath{stroke,fill}%
}%
\begin{pgfscope}%
\pgfsys@transformshift{1.992339in}{0.516222in}%
\pgfsys@useobject{currentmarker}{}%
\end{pgfscope}%
\end{pgfscope}%
\begin{pgfscope}%
\definecolor{textcolor}{rgb}{0.150000,0.150000,0.150000}%
\pgfsetstrokecolor{textcolor}%
\pgfsetfillcolor{textcolor}%
\pgftext[x=1.992339in,y=0.438444in,,top]{\color{textcolor}\sffamily\fontsize{8.000000}{9.600000}\selectfont 10}%
\end{pgfscope}%
\begin{pgfscope}%
\pgfpathrectangle{\pgfqpoint{0.556847in}{0.516222in}}{\pgfqpoint{1.722590in}{1.783528in}} %
\pgfusepath{clip}%
\pgfsetroundcap%
\pgfsetroundjoin%
\pgfsetlinewidth{0.803000pt}%
\definecolor{currentstroke}{rgb}{1.000000,1.000000,1.000000}%
\pgfsetstrokecolor{currentstroke}%
\pgfsetdash{}{0pt}%
\pgfpathmoveto{\pgfqpoint{2.279437in}{0.516222in}}%
\pgfpathlineto{\pgfqpoint{2.279437in}{2.299750in}}%
\pgfusepath{stroke}%
\end{pgfscope}%
\begin{pgfscope}%
\pgfsetbuttcap%
\pgfsetroundjoin%
\definecolor{currentfill}{rgb}{0.150000,0.150000,0.150000}%
\pgfsetfillcolor{currentfill}%
\pgfsetlinewidth{0.803000pt}%
\definecolor{currentstroke}{rgb}{0.150000,0.150000,0.150000}%
\pgfsetstrokecolor{currentstroke}%
\pgfsetdash{}{0pt}%
\pgfsys@defobject{currentmarker}{\pgfqpoint{0.000000in}{0.000000in}}{\pgfqpoint{0.000000in}{0.000000in}}{%
\pgfpathmoveto{\pgfqpoint{0.000000in}{0.000000in}}%
\pgfpathlineto{\pgfqpoint{0.000000in}{0.000000in}}%
\pgfusepath{stroke,fill}%
}%
\begin{pgfscope}%
\pgfsys@transformshift{2.279437in}{0.516222in}%
\pgfsys@useobject{currentmarker}{}%
\end{pgfscope}%
\end{pgfscope}%
\begin{pgfscope}%
\definecolor{textcolor}{rgb}{0.150000,0.150000,0.150000}%
\pgfsetstrokecolor{textcolor}%
\pgfsetfillcolor{textcolor}%
\pgftext[x=2.279437in,y=0.438444in,,top]{\color{textcolor}\sffamily\fontsize{8.000000}{9.600000}\selectfont 11}%
\end{pgfscope}%
\begin{pgfscope}%
\definecolor{textcolor}{rgb}{0.150000,0.150000,0.150000}%
\pgfsetstrokecolor{textcolor}%
\pgfsetfillcolor{textcolor}%
\pgftext[x=1.418142in,y=0.273321in,,top]{\color{textcolor}\sffamily\fontsize{8.800000}{10.560000}\selectfont Tail length}%
\end{pgfscope}%
\begin{pgfscope}%
\pgfpathrectangle{\pgfqpoint{0.556847in}{0.516222in}}{\pgfqpoint{1.722590in}{1.783528in}} %
\pgfusepath{clip}%
\pgfsetroundcap%
\pgfsetroundjoin%
\pgfsetlinewidth{0.803000pt}%
\definecolor{currentstroke}{rgb}{1.000000,1.000000,1.000000}%
\pgfsetstrokecolor{currentstroke}%
\pgfsetdash{}{0pt}%
\pgfpathmoveto{\pgfqpoint{0.556847in}{0.516222in}}%
\pgfpathlineto{\pgfqpoint{2.279437in}{0.516222in}}%
\pgfusepath{stroke}%
\end{pgfscope}%
\begin{pgfscope}%
\pgfsetbuttcap%
\pgfsetroundjoin%
\definecolor{currentfill}{rgb}{0.150000,0.150000,0.150000}%
\pgfsetfillcolor{currentfill}%
\pgfsetlinewidth{0.803000pt}%
\definecolor{currentstroke}{rgb}{0.150000,0.150000,0.150000}%
\pgfsetstrokecolor{currentstroke}%
\pgfsetdash{}{0pt}%
\pgfsys@defobject{currentmarker}{\pgfqpoint{0.000000in}{0.000000in}}{\pgfqpoint{0.000000in}{0.000000in}}{%
\pgfpathmoveto{\pgfqpoint{0.000000in}{0.000000in}}%
\pgfpathlineto{\pgfqpoint{0.000000in}{0.000000in}}%
\pgfusepath{stroke,fill}%
}%
\begin{pgfscope}%
\pgfsys@transformshift{0.556847in}{0.516222in}%
\pgfsys@useobject{currentmarker}{}%
\end{pgfscope}%
\end{pgfscope}%
\begin{pgfscope}%
\definecolor{textcolor}{rgb}{0.150000,0.150000,0.150000}%
\pgfsetstrokecolor{textcolor}%
\pgfsetfillcolor{textcolor}%
\pgftext[x=0.479069in,y=0.516222in,right,]{\color{textcolor}\sffamily\fontsize{8.000000}{9.600000}\selectfont 7}%
\end{pgfscope}%
\begin{pgfscope}%
\pgfpathrectangle{\pgfqpoint{0.556847in}{0.516222in}}{\pgfqpoint{1.722590in}{1.783528in}} %
\pgfusepath{clip}%
\pgfsetroundcap%
\pgfsetroundjoin%
\pgfsetlinewidth{0.803000pt}%
\definecolor{currentstroke}{rgb}{1.000000,1.000000,1.000000}%
\pgfsetstrokecolor{currentstroke}%
\pgfsetdash{}{0pt}%
\pgfpathmoveto{\pgfqpoint{0.556847in}{0.813477in}}%
\pgfpathlineto{\pgfqpoint{2.279437in}{0.813477in}}%
\pgfusepath{stroke}%
\end{pgfscope}%
\begin{pgfscope}%
\pgfsetbuttcap%
\pgfsetroundjoin%
\definecolor{currentfill}{rgb}{0.150000,0.150000,0.150000}%
\pgfsetfillcolor{currentfill}%
\pgfsetlinewidth{0.803000pt}%
\definecolor{currentstroke}{rgb}{0.150000,0.150000,0.150000}%
\pgfsetstrokecolor{currentstroke}%
\pgfsetdash{}{0pt}%
\pgfsys@defobject{currentmarker}{\pgfqpoint{0.000000in}{0.000000in}}{\pgfqpoint{0.000000in}{0.000000in}}{%
\pgfpathmoveto{\pgfqpoint{0.000000in}{0.000000in}}%
\pgfpathlineto{\pgfqpoint{0.000000in}{0.000000in}}%
\pgfusepath{stroke,fill}%
}%
\begin{pgfscope}%
\pgfsys@transformshift{0.556847in}{0.813477in}%
\pgfsys@useobject{currentmarker}{}%
\end{pgfscope}%
\end{pgfscope}%
\begin{pgfscope}%
\definecolor{textcolor}{rgb}{0.150000,0.150000,0.150000}%
\pgfsetstrokecolor{textcolor}%
\pgfsetfillcolor{textcolor}%
\pgftext[x=0.479069in,y=0.813477in,right,]{\color{textcolor}\sffamily\fontsize{8.000000}{9.600000}\selectfont 8}%
\end{pgfscope}%
\begin{pgfscope}%
\pgfpathrectangle{\pgfqpoint{0.556847in}{0.516222in}}{\pgfqpoint{1.722590in}{1.783528in}} %
\pgfusepath{clip}%
\pgfsetroundcap%
\pgfsetroundjoin%
\pgfsetlinewidth{0.803000pt}%
\definecolor{currentstroke}{rgb}{1.000000,1.000000,1.000000}%
\pgfsetstrokecolor{currentstroke}%
\pgfsetdash{}{0pt}%
\pgfpathmoveto{\pgfqpoint{0.556847in}{1.110731in}}%
\pgfpathlineto{\pgfqpoint{2.279437in}{1.110731in}}%
\pgfusepath{stroke}%
\end{pgfscope}%
\begin{pgfscope}%
\pgfsetbuttcap%
\pgfsetroundjoin%
\definecolor{currentfill}{rgb}{0.150000,0.150000,0.150000}%
\pgfsetfillcolor{currentfill}%
\pgfsetlinewidth{0.803000pt}%
\definecolor{currentstroke}{rgb}{0.150000,0.150000,0.150000}%
\pgfsetstrokecolor{currentstroke}%
\pgfsetdash{}{0pt}%
\pgfsys@defobject{currentmarker}{\pgfqpoint{0.000000in}{0.000000in}}{\pgfqpoint{0.000000in}{0.000000in}}{%
\pgfpathmoveto{\pgfqpoint{0.000000in}{0.000000in}}%
\pgfpathlineto{\pgfqpoint{0.000000in}{0.000000in}}%
\pgfusepath{stroke,fill}%
}%
\begin{pgfscope}%
\pgfsys@transformshift{0.556847in}{1.110731in}%
\pgfsys@useobject{currentmarker}{}%
\end{pgfscope}%
\end{pgfscope}%
\begin{pgfscope}%
\definecolor{textcolor}{rgb}{0.150000,0.150000,0.150000}%
\pgfsetstrokecolor{textcolor}%
\pgfsetfillcolor{textcolor}%
\pgftext[x=0.479069in,y=1.110731in,right,]{\color{textcolor}\sffamily\fontsize{8.000000}{9.600000}\selectfont 9}%
\end{pgfscope}%
\begin{pgfscope}%
\pgfpathrectangle{\pgfqpoint{0.556847in}{0.516222in}}{\pgfqpoint{1.722590in}{1.783528in}} %
\pgfusepath{clip}%
\pgfsetroundcap%
\pgfsetroundjoin%
\pgfsetlinewidth{0.803000pt}%
\definecolor{currentstroke}{rgb}{1.000000,1.000000,1.000000}%
\pgfsetstrokecolor{currentstroke}%
\pgfsetdash{}{0pt}%
\pgfpathmoveto{\pgfqpoint{0.556847in}{1.407986in}}%
\pgfpathlineto{\pgfqpoint{2.279437in}{1.407986in}}%
\pgfusepath{stroke}%
\end{pgfscope}%
\begin{pgfscope}%
\pgfsetbuttcap%
\pgfsetroundjoin%
\definecolor{currentfill}{rgb}{0.150000,0.150000,0.150000}%
\pgfsetfillcolor{currentfill}%
\pgfsetlinewidth{0.803000pt}%
\definecolor{currentstroke}{rgb}{0.150000,0.150000,0.150000}%
\pgfsetstrokecolor{currentstroke}%
\pgfsetdash{}{0pt}%
\pgfsys@defobject{currentmarker}{\pgfqpoint{0.000000in}{0.000000in}}{\pgfqpoint{0.000000in}{0.000000in}}{%
\pgfpathmoveto{\pgfqpoint{0.000000in}{0.000000in}}%
\pgfpathlineto{\pgfqpoint{0.000000in}{0.000000in}}%
\pgfusepath{stroke,fill}%
}%
\begin{pgfscope}%
\pgfsys@transformshift{0.556847in}{1.407986in}%
\pgfsys@useobject{currentmarker}{}%
\end{pgfscope}%
\end{pgfscope}%
\begin{pgfscope}%
\definecolor{textcolor}{rgb}{0.150000,0.150000,0.150000}%
\pgfsetstrokecolor{textcolor}%
\pgfsetfillcolor{textcolor}%
\pgftext[x=0.479069in,y=1.407986in,right,]{\color{textcolor}\sffamily\fontsize{8.000000}{9.600000}\selectfont 10}%
\end{pgfscope}%
\begin{pgfscope}%
\pgfpathrectangle{\pgfqpoint{0.556847in}{0.516222in}}{\pgfqpoint{1.722590in}{1.783528in}} %
\pgfusepath{clip}%
\pgfsetroundcap%
\pgfsetroundjoin%
\pgfsetlinewidth{0.803000pt}%
\definecolor{currentstroke}{rgb}{1.000000,1.000000,1.000000}%
\pgfsetstrokecolor{currentstroke}%
\pgfsetdash{}{0pt}%
\pgfpathmoveto{\pgfqpoint{0.556847in}{1.705241in}}%
\pgfpathlineto{\pgfqpoint{2.279437in}{1.705241in}}%
\pgfusepath{stroke}%
\end{pgfscope}%
\begin{pgfscope}%
\pgfsetbuttcap%
\pgfsetroundjoin%
\definecolor{currentfill}{rgb}{0.150000,0.150000,0.150000}%
\pgfsetfillcolor{currentfill}%
\pgfsetlinewidth{0.803000pt}%
\definecolor{currentstroke}{rgb}{0.150000,0.150000,0.150000}%
\pgfsetstrokecolor{currentstroke}%
\pgfsetdash{}{0pt}%
\pgfsys@defobject{currentmarker}{\pgfqpoint{0.000000in}{0.000000in}}{\pgfqpoint{0.000000in}{0.000000in}}{%
\pgfpathmoveto{\pgfqpoint{0.000000in}{0.000000in}}%
\pgfpathlineto{\pgfqpoint{0.000000in}{0.000000in}}%
\pgfusepath{stroke,fill}%
}%
\begin{pgfscope}%
\pgfsys@transformshift{0.556847in}{1.705241in}%
\pgfsys@useobject{currentmarker}{}%
\end{pgfscope}%
\end{pgfscope}%
\begin{pgfscope}%
\definecolor{textcolor}{rgb}{0.150000,0.150000,0.150000}%
\pgfsetstrokecolor{textcolor}%
\pgfsetfillcolor{textcolor}%
\pgftext[x=0.479069in,y=1.705241in,right,]{\color{textcolor}\sffamily\fontsize{8.000000}{9.600000}\selectfont 11}%
\end{pgfscope}%
\begin{pgfscope}%
\pgfpathrectangle{\pgfqpoint{0.556847in}{0.516222in}}{\pgfqpoint{1.722590in}{1.783528in}} %
\pgfusepath{clip}%
\pgfsetroundcap%
\pgfsetroundjoin%
\pgfsetlinewidth{0.803000pt}%
\definecolor{currentstroke}{rgb}{1.000000,1.000000,1.000000}%
\pgfsetstrokecolor{currentstroke}%
\pgfsetdash{}{0pt}%
\pgfpathmoveto{\pgfqpoint{0.556847in}{2.002495in}}%
\pgfpathlineto{\pgfqpoint{2.279437in}{2.002495in}}%
\pgfusepath{stroke}%
\end{pgfscope}%
\begin{pgfscope}%
\pgfsetbuttcap%
\pgfsetroundjoin%
\definecolor{currentfill}{rgb}{0.150000,0.150000,0.150000}%
\pgfsetfillcolor{currentfill}%
\pgfsetlinewidth{0.803000pt}%
\definecolor{currentstroke}{rgb}{0.150000,0.150000,0.150000}%
\pgfsetstrokecolor{currentstroke}%
\pgfsetdash{}{0pt}%
\pgfsys@defobject{currentmarker}{\pgfqpoint{0.000000in}{0.000000in}}{\pgfqpoint{0.000000in}{0.000000in}}{%
\pgfpathmoveto{\pgfqpoint{0.000000in}{0.000000in}}%
\pgfpathlineto{\pgfqpoint{0.000000in}{0.000000in}}%
\pgfusepath{stroke,fill}%
}%
\begin{pgfscope}%
\pgfsys@transformshift{0.556847in}{2.002495in}%
\pgfsys@useobject{currentmarker}{}%
\end{pgfscope}%
\end{pgfscope}%
\begin{pgfscope}%
\definecolor{textcolor}{rgb}{0.150000,0.150000,0.150000}%
\pgfsetstrokecolor{textcolor}%
\pgfsetfillcolor{textcolor}%
\pgftext[x=0.479069in,y=2.002495in,right,]{\color{textcolor}\sffamily\fontsize{8.000000}{9.600000}\selectfont 12}%
\end{pgfscope}%
\begin{pgfscope}%
\pgfpathrectangle{\pgfqpoint{0.556847in}{0.516222in}}{\pgfqpoint{1.722590in}{1.783528in}} %
\pgfusepath{clip}%
\pgfsetroundcap%
\pgfsetroundjoin%
\pgfsetlinewidth{0.803000pt}%
\definecolor{currentstroke}{rgb}{1.000000,1.000000,1.000000}%
\pgfsetstrokecolor{currentstroke}%
\pgfsetdash{}{0pt}%
\pgfpathmoveto{\pgfqpoint{0.556847in}{2.299750in}}%
\pgfpathlineto{\pgfqpoint{2.279437in}{2.299750in}}%
\pgfusepath{stroke}%
\end{pgfscope}%
\begin{pgfscope}%
\pgfsetbuttcap%
\pgfsetroundjoin%
\definecolor{currentfill}{rgb}{0.150000,0.150000,0.150000}%
\pgfsetfillcolor{currentfill}%
\pgfsetlinewidth{0.803000pt}%
\definecolor{currentstroke}{rgb}{0.150000,0.150000,0.150000}%
\pgfsetstrokecolor{currentstroke}%
\pgfsetdash{}{0pt}%
\pgfsys@defobject{currentmarker}{\pgfqpoint{0.000000in}{0.000000in}}{\pgfqpoint{0.000000in}{0.000000in}}{%
\pgfpathmoveto{\pgfqpoint{0.000000in}{0.000000in}}%
\pgfpathlineto{\pgfqpoint{0.000000in}{0.000000in}}%
\pgfusepath{stroke,fill}%
}%
\begin{pgfscope}%
\pgfsys@transformshift{0.556847in}{2.299750in}%
\pgfsys@useobject{currentmarker}{}%
\end{pgfscope}%
\end{pgfscope}%
\begin{pgfscope}%
\definecolor{textcolor}{rgb}{0.150000,0.150000,0.150000}%
\pgfsetstrokecolor{textcolor}%
\pgfsetfillcolor{textcolor}%
\pgftext[x=0.479069in,y=2.299750in,right,]{\color{textcolor}\sffamily\fontsize{8.000000}{9.600000}\selectfont 13}%
\end{pgfscope}%
\begin{pgfscope}%
\definecolor{textcolor}{rgb}{0.150000,0.150000,0.150000}%
\pgfsetstrokecolor{textcolor}%
\pgfsetfillcolor{textcolor}%
\pgftext[x=0.286014in,y=1.407986in,,bottom,rotate=90.000000]{\color{textcolor}\sffamily\fontsize{8.800000}{10.560000}\selectfont Arm length}%
\end{pgfscope}%
\begin{pgfscope}%
\pgfpathrectangle{\pgfqpoint{0.556847in}{0.516222in}}{\pgfqpoint{1.722590in}{1.783528in}} %
\pgfusepath{clip}%
\pgfsetbuttcap%
\pgfsetroundjoin%
\definecolor{currentfill}{rgb}{0.298039,0.447059,0.690196}%
\pgfsetfillcolor{currentfill}%
\pgfsetlinewidth{0.240900pt}%
\definecolor{currentstroke}{rgb}{1.000000,1.000000,1.000000}%
\pgfsetstrokecolor{currentstroke}%
\pgfsetdash{}{0pt}%
\pgfpathmoveto{\pgfqpoint{1.021947in}{1.287753in}}%
\pgfpathcurveto{\pgfqpoint{1.030183in}{1.287753in}}{\pgfqpoint{1.038083in}{1.291026in}}{\pgfqpoint{1.043907in}{1.296849in}}%
\pgfpathcurveto{\pgfqpoint{1.049731in}{1.302673in}}{\pgfqpoint{1.053003in}{1.310573in}}{\pgfqpoint{1.053003in}{1.318810in}}%
\pgfpathcurveto{\pgfqpoint{1.053003in}{1.327046in}}{\pgfqpoint{1.049731in}{1.334946in}}{\pgfqpoint{1.043907in}{1.340770in}}%
\pgfpathcurveto{\pgfqpoint{1.038083in}{1.346594in}}{\pgfqpoint{1.030183in}{1.349866in}}{\pgfqpoint{1.021947in}{1.349866in}}%
\pgfpathcurveto{\pgfqpoint{1.013710in}{1.349866in}}{\pgfqpoint{1.005810in}{1.346594in}}{\pgfqpoint{0.999986in}{1.340770in}}%
\pgfpathcurveto{\pgfqpoint{0.994162in}{1.334946in}}{\pgfqpoint{0.990890in}{1.327046in}}{\pgfqpoint{0.990890in}{1.318810in}}%
\pgfpathcurveto{\pgfqpoint{0.990890in}{1.310573in}}{\pgfqpoint{0.994162in}{1.302673in}}{\pgfqpoint{0.999986in}{1.296849in}}%
\pgfpathcurveto{\pgfqpoint{1.005810in}{1.291026in}}{\pgfqpoint{1.013710in}{1.287753in}}{\pgfqpoint{1.021947in}{1.287753in}}%
\pgfpathclose%
\pgfusepath{stroke,fill}%
\end{pgfscope}%
\begin{pgfscope}%
\pgfpathrectangle{\pgfqpoint{0.556847in}{0.516222in}}{\pgfqpoint{1.722590in}{1.783528in}} %
\pgfusepath{clip}%
\pgfsetbuttcap%
\pgfsetroundjoin%
\definecolor{currentfill}{rgb}{0.298039,0.447059,0.690196}%
\pgfsetfillcolor{currentfill}%
\pgfsetlinewidth{0.240900pt}%
\definecolor{currentstroke}{rgb}{1.000000,1.000000,1.000000}%
\pgfsetstrokecolor{currentstroke}%
\pgfsetdash{}{0pt}%
\pgfpathmoveto{\pgfqpoint{2.095695in}{1.763361in}}%
\pgfpathcurveto{\pgfqpoint{2.103931in}{1.763361in}}{\pgfqpoint{2.111831in}{1.766633in}}{\pgfqpoint{2.117655in}{1.772457in}}%
\pgfpathcurveto{\pgfqpoint{2.123479in}{1.778281in}}{\pgfqpoint{2.126751in}{1.786181in}}{\pgfqpoint{2.126751in}{1.794417in}}%
\pgfpathcurveto{\pgfqpoint{2.126751in}{1.802653in}}{\pgfqpoint{2.123479in}{1.810553in}}{\pgfqpoint{2.117655in}{1.816377in}}%
\pgfpathcurveto{\pgfqpoint{2.111831in}{1.822201in}}{\pgfqpoint{2.103931in}{1.825474in}}{\pgfqpoint{2.095695in}{1.825474in}}%
\pgfpathcurveto{\pgfqpoint{2.087458in}{1.825474in}}{\pgfqpoint{2.079558in}{1.822201in}}{\pgfqpoint{2.073734in}{1.816377in}}%
\pgfpathcurveto{\pgfqpoint{2.067910in}{1.810553in}}{\pgfqpoint{2.064638in}{1.802653in}}{\pgfqpoint{2.064638in}{1.794417in}}%
\pgfpathcurveto{\pgfqpoint{2.064638in}{1.786181in}}{\pgfqpoint{2.067910in}{1.778281in}}{\pgfqpoint{2.073734in}{1.772457in}}%
\pgfpathcurveto{\pgfqpoint{2.079558in}{1.766633in}}{\pgfqpoint{2.087458in}{1.763361in}}{\pgfqpoint{2.095695in}{1.763361in}}%
\pgfpathclose%
\pgfusepath{stroke,fill}%
\end{pgfscope}%
\begin{pgfscope}%
\pgfpathrectangle{\pgfqpoint{0.556847in}{0.516222in}}{\pgfqpoint{1.722590in}{1.783528in}} %
\pgfusepath{clip}%
\pgfsetbuttcap%
\pgfsetroundjoin%
\definecolor{currentfill}{rgb}{0.298039,0.447059,0.690196}%
\pgfsetfillcolor{currentfill}%
\pgfsetlinewidth{0.240900pt}%
\definecolor{currentstroke}{rgb}{1.000000,1.000000,1.000000}%
\pgfsetstrokecolor{currentstroke}%
\pgfsetdash{}{0pt}%
\pgfpathmoveto{\pgfqpoint{1.332013in}{1.436381in}}%
\pgfpathcurveto{\pgfqpoint{1.340249in}{1.436381in}}{\pgfqpoint{1.348149in}{1.439653in}}{\pgfqpoint{1.353973in}{1.445477in}}%
\pgfpathcurveto{\pgfqpoint{1.359797in}{1.451301in}}{\pgfqpoint{1.363069in}{1.459201in}}{\pgfqpoint{1.363069in}{1.467437in}}%
\pgfpathcurveto{\pgfqpoint{1.363069in}{1.475673in}}{\pgfqpoint{1.359797in}{1.483573in}}{\pgfqpoint{1.353973in}{1.489397in}}%
\pgfpathcurveto{\pgfqpoint{1.348149in}{1.495221in}}{\pgfqpoint{1.340249in}{1.498494in}}{\pgfqpoint{1.332013in}{1.498494in}}%
\pgfpathcurveto{\pgfqpoint{1.323777in}{1.498494in}}{\pgfqpoint{1.315877in}{1.495221in}}{\pgfqpoint{1.310053in}{1.489397in}}%
\pgfpathcurveto{\pgfqpoint{1.304229in}{1.483573in}}{\pgfqpoint{1.300956in}{1.475673in}}{\pgfqpoint{1.300956in}{1.467437in}}%
\pgfpathcurveto{\pgfqpoint{1.300956in}{1.459201in}}{\pgfqpoint{1.304229in}{1.451301in}}{\pgfqpoint{1.310053in}{1.445477in}}%
\pgfpathcurveto{\pgfqpoint{1.315877in}{1.439653in}}{\pgfqpoint{1.323777in}{1.436381in}}{\pgfqpoint{1.332013in}{1.436381in}}%
\pgfpathclose%
\pgfusepath{stroke,fill}%
\end{pgfscope}%
\begin{pgfscope}%
\pgfpathrectangle{\pgfqpoint{0.556847in}{0.516222in}}{\pgfqpoint{1.722590in}{1.783528in}} %
\pgfusepath{clip}%
\pgfsetbuttcap%
\pgfsetroundjoin%
\definecolor{currentfill}{rgb}{0.298039,0.447059,0.690196}%
\pgfsetfillcolor{currentfill}%
\pgfsetlinewidth{0.240900pt}%
\definecolor{currentstroke}{rgb}{1.000000,1.000000,1.000000}%
\pgfsetstrokecolor{currentstroke}%
\pgfsetdash{}{0pt}%
\pgfpathmoveto{\pgfqpoint{1.507143in}{1.822812in}}%
\pgfpathcurveto{\pgfqpoint{1.515379in}{1.822812in}}{\pgfqpoint{1.523279in}{1.826084in}}{\pgfqpoint{1.529103in}{1.831908in}}%
\pgfpathcurveto{\pgfqpoint{1.534927in}{1.837732in}}{\pgfqpoint{1.538199in}{1.845632in}}{\pgfqpoint{1.538199in}{1.853868in}}%
\pgfpathcurveto{\pgfqpoint{1.538199in}{1.862104in}}{\pgfqpoint{1.534927in}{1.870004in}}{\pgfqpoint{1.529103in}{1.875828in}}%
\pgfpathcurveto{\pgfqpoint{1.523279in}{1.881652in}}{\pgfqpoint{1.515379in}{1.884925in}}{\pgfqpoint{1.507143in}{1.884925in}}%
\pgfpathcurveto{\pgfqpoint{1.498907in}{1.884925in}}{\pgfqpoint{1.491007in}{1.881652in}}{\pgfqpoint{1.485183in}{1.875828in}}%
\pgfpathcurveto{\pgfqpoint{1.479359in}{1.870004in}}{\pgfqpoint{1.476086in}{1.862104in}}{\pgfqpoint{1.476086in}{1.853868in}}%
\pgfpathcurveto{\pgfqpoint{1.476086in}{1.845632in}}{\pgfqpoint{1.479359in}{1.837732in}}{\pgfqpoint{1.485183in}{1.831908in}}%
\pgfpathcurveto{\pgfqpoint{1.491007in}{1.826084in}}{\pgfqpoint{1.498907in}{1.822812in}}{\pgfqpoint{1.507143in}{1.822812in}}%
\pgfpathclose%
\pgfusepath{stroke,fill}%
\end{pgfscope}%
\begin{pgfscope}%
\pgfpathrectangle{\pgfqpoint{0.556847in}{0.516222in}}{\pgfqpoint{1.722590in}{1.783528in}} %
\pgfusepath{clip}%
\pgfsetbuttcap%
\pgfsetroundjoin%
\definecolor{currentfill}{rgb}{0.298039,0.447059,0.690196}%
\pgfsetfillcolor{currentfill}%
\pgfsetlinewidth{0.240900pt}%
\definecolor{currentstroke}{rgb}{1.000000,1.000000,1.000000}%
\pgfsetstrokecolor{currentstroke}%
\pgfsetdash{}{0pt}%
\pgfpathmoveto{\pgfqpoint{1.771273in}{1.733635in}}%
\pgfpathcurveto{\pgfqpoint{1.779510in}{1.733635in}}{\pgfqpoint{1.787410in}{1.736907in}}{\pgfqpoint{1.793234in}{1.742731in}}%
\pgfpathcurveto{\pgfqpoint{1.799058in}{1.748555in}}{\pgfqpoint{1.802330in}{1.756455in}}{\pgfqpoint{1.802330in}{1.764692in}}%
\pgfpathcurveto{\pgfqpoint{1.802330in}{1.772928in}}{\pgfqpoint{1.799058in}{1.780828in}}{\pgfqpoint{1.793234in}{1.786652in}}%
\pgfpathcurveto{\pgfqpoint{1.787410in}{1.792476in}}{\pgfqpoint{1.779510in}{1.795748in}}{\pgfqpoint{1.771273in}{1.795748in}}%
\pgfpathcurveto{\pgfqpoint{1.763037in}{1.795748in}}{\pgfqpoint{1.755137in}{1.792476in}}{\pgfqpoint{1.749313in}{1.786652in}}%
\pgfpathcurveto{\pgfqpoint{1.743489in}{1.780828in}}{\pgfqpoint{1.740217in}{1.772928in}}{\pgfqpoint{1.740217in}{1.764692in}}%
\pgfpathcurveto{\pgfqpoint{1.740217in}{1.756455in}}{\pgfqpoint{1.743489in}{1.748555in}}{\pgfqpoint{1.749313in}{1.742731in}}%
\pgfpathcurveto{\pgfqpoint{1.755137in}{1.736907in}}{\pgfqpoint{1.763037in}{1.733635in}}{\pgfqpoint{1.771273in}{1.733635in}}%
\pgfpathclose%
\pgfusepath{stroke,fill}%
\end{pgfscope}%
\begin{pgfscope}%
\pgfpathrectangle{\pgfqpoint{0.556847in}{0.516222in}}{\pgfqpoint{1.722590in}{1.783528in}} %
\pgfusepath{clip}%
\pgfsetbuttcap%
\pgfsetroundjoin%
\definecolor{currentfill}{rgb}{0.298039,0.447059,0.690196}%
\pgfsetfillcolor{currentfill}%
\pgfsetlinewidth{0.240900pt}%
\definecolor{currentstroke}{rgb}{1.000000,1.000000,1.000000}%
\pgfsetstrokecolor{currentstroke}%
\pgfsetdash{}{0pt}%
\pgfpathmoveto{\pgfqpoint{1.208561in}{1.911988in}}%
\pgfpathcurveto{\pgfqpoint{1.216797in}{1.911988in}}{\pgfqpoint{1.224697in}{1.915260in}}{\pgfqpoint{1.230521in}{1.921084in}}%
\pgfpathcurveto{\pgfqpoint{1.236345in}{1.926908in}}{\pgfqpoint{1.239617in}{1.934808in}}{\pgfqpoint{1.239617in}{1.943044in}}%
\pgfpathcurveto{\pgfqpoint{1.239617in}{1.951281in}}{\pgfqpoint{1.236345in}{1.959181in}}{\pgfqpoint{1.230521in}{1.965005in}}%
\pgfpathcurveto{\pgfqpoint{1.224697in}{1.970829in}}{\pgfqpoint{1.216797in}{1.974101in}}{\pgfqpoint{1.208561in}{1.974101in}}%
\pgfpathcurveto{\pgfqpoint{1.200324in}{1.974101in}}{\pgfqpoint{1.192424in}{1.970829in}}{\pgfqpoint{1.186600in}{1.965005in}}%
\pgfpathcurveto{\pgfqpoint{1.180776in}{1.959181in}}{\pgfqpoint{1.177504in}{1.951281in}}{\pgfqpoint{1.177504in}{1.943044in}}%
\pgfpathcurveto{\pgfqpoint{1.177504in}{1.934808in}}{\pgfqpoint{1.180776in}{1.926908in}}{\pgfqpoint{1.186600in}{1.921084in}}%
\pgfpathcurveto{\pgfqpoint{1.192424in}{1.915260in}}{\pgfqpoint{1.200324in}{1.911988in}}{\pgfqpoint{1.208561in}{1.911988in}}%
\pgfpathclose%
\pgfusepath{stroke,fill}%
\end{pgfscope}%
\begin{pgfscope}%
\pgfpathrectangle{\pgfqpoint{0.556847in}{0.516222in}}{\pgfqpoint{1.722590in}{1.783528in}} %
\pgfusepath{clip}%
\pgfsetbuttcap%
\pgfsetroundjoin%
\definecolor{currentfill}{rgb}{0.298039,0.447059,0.690196}%
\pgfsetfillcolor{currentfill}%
\pgfsetlinewidth{0.240900pt}%
\definecolor{currentstroke}{rgb}{1.000000,1.000000,1.000000}%
\pgfsetstrokecolor{currentstroke}%
\pgfsetdash{}{0pt}%
\pgfpathmoveto{\pgfqpoint{1.329142in}{1.168851in}}%
\pgfpathcurveto{\pgfqpoint{1.337378in}{1.168851in}}{\pgfqpoint{1.345278in}{1.172124in}}{\pgfqpoint{1.351102in}{1.177948in}}%
\pgfpathcurveto{\pgfqpoint{1.356926in}{1.183772in}}{\pgfqpoint{1.360198in}{1.191672in}}{\pgfqpoint{1.360198in}{1.199908in}}%
\pgfpathcurveto{\pgfqpoint{1.360198in}{1.208144in}}{\pgfqpoint{1.356926in}{1.216044in}}{\pgfqpoint{1.351102in}{1.221868in}}%
\pgfpathcurveto{\pgfqpoint{1.345278in}{1.227692in}}{\pgfqpoint{1.337378in}{1.230964in}}{\pgfqpoint{1.329142in}{1.230964in}}%
\pgfpathcurveto{\pgfqpoint{1.320906in}{1.230964in}}{\pgfqpoint{1.313006in}{1.227692in}}{\pgfqpoint{1.307182in}{1.221868in}}%
\pgfpathcurveto{\pgfqpoint{1.301358in}{1.216044in}}{\pgfqpoint{1.298085in}{1.208144in}}{\pgfqpoint{1.298085in}{1.199908in}}%
\pgfpathcurveto{\pgfqpoint{1.298085in}{1.191672in}}{\pgfqpoint{1.301358in}{1.183772in}}{\pgfqpoint{1.307182in}{1.177948in}}%
\pgfpathcurveto{\pgfqpoint{1.313006in}{1.172124in}}{\pgfqpoint{1.320906in}{1.168851in}}{\pgfqpoint{1.329142in}{1.168851in}}%
\pgfpathclose%
\pgfusepath{stroke,fill}%
\end{pgfscope}%
\begin{pgfscope}%
\pgfpathrectangle{\pgfqpoint{0.556847in}{0.516222in}}{\pgfqpoint{1.722590in}{1.783528in}} %
\pgfusepath{clip}%
\pgfsetbuttcap%
\pgfsetroundjoin%
\definecolor{currentfill}{rgb}{0.298039,0.447059,0.690196}%
\pgfsetfillcolor{currentfill}%
\pgfsetlinewidth{0.240900pt}%
\definecolor{currentstroke}{rgb}{1.000000,1.000000,1.000000}%
\pgfsetstrokecolor{currentstroke}%
\pgfsetdash{}{0pt}%
\pgfpathmoveto{\pgfqpoint{1.688015in}{1.674184in}}%
\pgfpathcurveto{\pgfqpoint{1.696251in}{1.674184in}}{\pgfqpoint{1.704151in}{1.677457in}}{\pgfqpoint{1.709975in}{1.683280in}}%
\pgfpathcurveto{\pgfqpoint{1.715799in}{1.689104in}}{\pgfqpoint{1.719071in}{1.697004in}}{\pgfqpoint{1.719071in}{1.705241in}}%
\pgfpathcurveto{\pgfqpoint{1.719071in}{1.713477in}}{\pgfqpoint{1.715799in}{1.721377in}}{\pgfqpoint{1.709975in}{1.727201in}}%
\pgfpathcurveto{\pgfqpoint{1.704151in}{1.733025in}}{\pgfqpoint{1.696251in}{1.736297in}}{\pgfqpoint{1.688015in}{1.736297in}}%
\pgfpathcurveto{\pgfqpoint{1.679779in}{1.736297in}}{\pgfqpoint{1.671879in}{1.733025in}}{\pgfqpoint{1.666055in}{1.727201in}}%
\pgfpathcurveto{\pgfqpoint{1.660231in}{1.721377in}}{\pgfqpoint{1.656958in}{1.713477in}}{\pgfqpoint{1.656958in}{1.705241in}}%
\pgfpathcurveto{\pgfqpoint{1.656958in}{1.697004in}}{\pgfqpoint{1.660231in}{1.689104in}}{\pgfqpoint{1.666055in}{1.683280in}}%
\pgfpathcurveto{\pgfqpoint{1.671879in}{1.677457in}}{\pgfqpoint{1.679779in}{1.674184in}}{\pgfqpoint{1.688015in}{1.674184in}}%
\pgfpathclose%
\pgfusepath{stroke,fill}%
\end{pgfscope}%
\begin{pgfscope}%
\pgfpathrectangle{\pgfqpoint{0.556847in}{0.516222in}}{\pgfqpoint{1.722590in}{1.783528in}} %
\pgfusepath{clip}%
\pgfsetbuttcap%
\pgfsetroundjoin%
\definecolor{currentfill}{rgb}{0.298039,0.447059,0.690196}%
\pgfsetfillcolor{currentfill}%
\pgfsetlinewidth{0.240900pt}%
\definecolor{currentstroke}{rgb}{1.000000,1.000000,1.000000}%
\pgfsetstrokecolor{currentstroke}%
\pgfsetdash{}{0pt}%
\pgfpathmoveto{\pgfqpoint{2.035404in}{1.585008in}}%
\pgfpathcurveto{\pgfqpoint{2.043640in}{1.585008in}}{\pgfqpoint{2.051540in}{1.588280in}}{\pgfqpoint{2.057364in}{1.594104in}}%
\pgfpathcurveto{\pgfqpoint{2.063188in}{1.599928in}}{\pgfqpoint{2.066460in}{1.607828in}}{\pgfqpoint{2.066460in}{1.616064in}}%
\pgfpathcurveto{\pgfqpoint{2.066460in}{1.624301in}}{\pgfqpoint{2.063188in}{1.632201in}}{\pgfqpoint{2.057364in}{1.638025in}}%
\pgfpathcurveto{\pgfqpoint{2.051540in}{1.643849in}}{\pgfqpoint{2.043640in}{1.647121in}}{\pgfqpoint{2.035404in}{1.647121in}}%
\pgfpathcurveto{\pgfqpoint{2.027168in}{1.647121in}}{\pgfqpoint{2.019268in}{1.643849in}}{\pgfqpoint{2.013444in}{1.638025in}}%
\pgfpathcurveto{\pgfqpoint{2.007620in}{1.632201in}}{\pgfqpoint{2.004347in}{1.624301in}}{\pgfqpoint{2.004347in}{1.616064in}}%
\pgfpathcurveto{\pgfqpoint{2.004347in}{1.607828in}}{\pgfqpoint{2.007620in}{1.599928in}}{\pgfqpoint{2.013444in}{1.594104in}}%
\pgfpathcurveto{\pgfqpoint{2.019268in}{1.588280in}}{\pgfqpoint{2.027168in}{1.585008in}}{\pgfqpoint{2.035404in}{1.585008in}}%
\pgfpathclose%
\pgfusepath{stroke,fill}%
\end{pgfscope}%
\begin{pgfscope}%
\pgfpathrectangle{\pgfqpoint{0.556847in}{0.516222in}}{\pgfqpoint{1.722590in}{1.783528in}} %
\pgfusepath{clip}%
\pgfsetbuttcap%
\pgfsetroundjoin%
\definecolor{currentfill}{rgb}{0.298039,0.447059,0.690196}%
\pgfsetfillcolor{currentfill}%
\pgfsetlinewidth{0.240900pt}%
\definecolor{currentstroke}{rgb}{1.000000,1.000000,1.000000}%
\pgfsetstrokecolor{currentstroke}%
\pgfsetdash{}{0pt}%
\pgfpathmoveto{\pgfqpoint{2.087082in}{1.436381in}}%
\pgfpathcurveto{\pgfqpoint{2.095318in}{1.436381in}}{\pgfqpoint{2.103218in}{1.439653in}}{\pgfqpoint{2.109042in}{1.445477in}}%
\pgfpathcurveto{\pgfqpoint{2.114866in}{1.451301in}}{\pgfqpoint{2.118138in}{1.459201in}}{\pgfqpoint{2.118138in}{1.467437in}}%
\pgfpathcurveto{\pgfqpoint{2.118138in}{1.475673in}}{\pgfqpoint{2.114866in}{1.483573in}}{\pgfqpoint{2.109042in}{1.489397in}}%
\pgfpathcurveto{\pgfqpoint{2.103218in}{1.495221in}}{\pgfqpoint{2.095318in}{1.498494in}}{\pgfqpoint{2.087082in}{1.498494in}}%
\pgfpathcurveto{\pgfqpoint{2.078845in}{1.498494in}}{\pgfqpoint{2.070945in}{1.495221in}}{\pgfqpoint{2.065121in}{1.489397in}}%
\pgfpathcurveto{\pgfqpoint{2.059297in}{1.483573in}}{\pgfqpoint{2.056025in}{1.475673in}}{\pgfqpoint{2.056025in}{1.467437in}}%
\pgfpathcurveto{\pgfqpoint{2.056025in}{1.459201in}}{\pgfqpoint{2.059297in}{1.451301in}}{\pgfqpoint{2.065121in}{1.445477in}}%
\pgfpathcurveto{\pgfqpoint{2.070945in}{1.439653in}}{\pgfqpoint{2.078845in}{1.436381in}}{\pgfqpoint{2.087082in}{1.436381in}}%
\pgfpathclose%
\pgfusepath{stroke,fill}%
\end{pgfscope}%
\begin{pgfscope}%
\pgfpathrectangle{\pgfqpoint{0.556847in}{0.516222in}}{\pgfqpoint{1.722590in}{1.783528in}} %
\pgfusepath{clip}%
\pgfsetbuttcap%
\pgfsetroundjoin%
\definecolor{currentfill}{rgb}{0.298039,0.447059,0.690196}%
\pgfsetfillcolor{currentfill}%
\pgfsetlinewidth{0.240900pt}%
\definecolor{currentstroke}{rgb}{1.000000,1.000000,1.000000}%
\pgfsetstrokecolor{currentstroke}%
\pgfsetdash{}{0pt}%
\pgfpathmoveto{\pgfqpoint{0.889881in}{0.633793in}}%
\pgfpathcurveto{\pgfqpoint{0.898118in}{0.633793in}}{\pgfqpoint{0.906018in}{0.637065in}}{\pgfqpoint{0.911842in}{0.642889in}}%
\pgfpathcurveto{\pgfqpoint{0.917666in}{0.648713in}}{\pgfqpoint{0.920938in}{0.656613in}}{\pgfqpoint{0.920938in}{0.664850in}}%
\pgfpathcurveto{\pgfqpoint{0.920938in}{0.673086in}}{\pgfqpoint{0.917666in}{0.680986in}}{\pgfqpoint{0.911842in}{0.686810in}}%
\pgfpathcurveto{\pgfqpoint{0.906018in}{0.692634in}}{\pgfqpoint{0.898118in}{0.695906in}}{\pgfqpoint{0.889881in}{0.695906in}}%
\pgfpathcurveto{\pgfqpoint{0.881645in}{0.695906in}}{\pgfqpoint{0.873745in}{0.692634in}}{\pgfqpoint{0.867921in}{0.686810in}}%
\pgfpathcurveto{\pgfqpoint{0.862097in}{0.680986in}}{\pgfqpoint{0.858825in}{0.673086in}}{\pgfqpoint{0.858825in}{0.664850in}}%
\pgfpathcurveto{\pgfqpoint{0.858825in}{0.656613in}}{\pgfqpoint{0.862097in}{0.648713in}}{\pgfqpoint{0.867921in}{0.642889in}}%
\pgfpathcurveto{\pgfqpoint{0.873745in}{0.637065in}}{\pgfqpoint{0.881645in}{0.633793in}}{\pgfqpoint{0.889881in}{0.633793in}}%
\pgfpathclose%
\pgfusepath{stroke,fill}%
\end{pgfscope}%
\begin{pgfscope}%
\pgfpathrectangle{\pgfqpoint{0.556847in}{0.516222in}}{\pgfqpoint{1.722590in}{1.783528in}} %
\pgfusepath{clip}%
\pgfsetbuttcap%
\pgfsetroundjoin%
\definecolor{currentfill}{rgb}{0.298039,0.447059,0.690196}%
\pgfsetfillcolor{currentfill}%
\pgfsetlinewidth{0.240900pt}%
\definecolor{currentstroke}{rgb}{1.000000,1.000000,1.000000}%
\pgfsetstrokecolor{currentstroke}%
\pgfsetdash{}{0pt}%
\pgfpathmoveto{\pgfqpoint{1.857403in}{1.793086in}}%
\pgfpathcurveto{\pgfqpoint{1.865639in}{1.793086in}}{\pgfqpoint{1.873539in}{1.796358in}}{\pgfqpoint{1.879363in}{1.802182in}}%
\pgfpathcurveto{\pgfqpoint{1.885187in}{1.808006in}}{\pgfqpoint{1.888459in}{1.815906in}}{\pgfqpoint{1.888459in}{1.824143in}}%
\pgfpathcurveto{\pgfqpoint{1.888459in}{1.832379in}}{\pgfqpoint{1.885187in}{1.840279in}}{\pgfqpoint{1.879363in}{1.846103in}}%
\pgfpathcurveto{\pgfqpoint{1.873539in}{1.851927in}}{\pgfqpoint{1.865639in}{1.855199in}}{\pgfqpoint{1.857403in}{1.855199in}}%
\pgfpathcurveto{\pgfqpoint{1.849167in}{1.855199in}}{\pgfqpoint{1.841267in}{1.851927in}}{\pgfqpoint{1.835443in}{1.846103in}}%
\pgfpathcurveto{\pgfqpoint{1.829619in}{1.840279in}}{\pgfqpoint{1.826346in}{1.832379in}}{\pgfqpoint{1.826346in}{1.824143in}}%
\pgfpathcurveto{\pgfqpoint{1.826346in}{1.815906in}}{\pgfqpoint{1.829619in}{1.808006in}}{\pgfqpoint{1.835443in}{1.802182in}}%
\pgfpathcurveto{\pgfqpoint{1.841267in}{1.796358in}}{\pgfqpoint{1.849167in}{1.793086in}}{\pgfqpoint{1.857403in}{1.793086in}}%
\pgfpathclose%
\pgfusepath{stroke,fill}%
\end{pgfscope}%
\begin{pgfscope}%
\pgfpathrectangle{\pgfqpoint{0.556847in}{0.516222in}}{\pgfqpoint{1.722590in}{1.783528in}} %
\pgfusepath{clip}%
\pgfsetbuttcap%
\pgfsetroundjoin%
\definecolor{currentfill}{rgb}{0.298039,0.447059,0.690196}%
\pgfsetfillcolor{currentfill}%
\pgfsetlinewidth{0.240900pt}%
\definecolor{currentstroke}{rgb}{1.000000,1.000000,1.000000}%
\pgfsetstrokecolor{currentstroke}%
\pgfsetdash{}{0pt}%
\pgfpathmoveto{\pgfqpoint{1.142528in}{1.376930in}}%
\pgfpathcurveto{\pgfqpoint{1.150764in}{1.376930in}}{\pgfqpoint{1.158664in}{1.380202in}}{\pgfqpoint{1.164488in}{1.386026in}}%
\pgfpathcurveto{\pgfqpoint{1.170312in}{1.391850in}}{\pgfqpoint{1.173584in}{1.399750in}}{\pgfqpoint{1.173584in}{1.407986in}}%
\pgfpathcurveto{\pgfqpoint{1.173584in}{1.416222in}}{\pgfqpoint{1.170312in}{1.424122in}}{\pgfqpoint{1.164488in}{1.429946in}}%
\pgfpathcurveto{\pgfqpoint{1.158664in}{1.435770in}}{\pgfqpoint{1.150764in}{1.439043in}}{\pgfqpoint{1.142528in}{1.439043in}}%
\pgfpathcurveto{\pgfqpoint{1.134292in}{1.439043in}}{\pgfqpoint{1.126392in}{1.435770in}}{\pgfqpoint{1.120568in}{1.429946in}}%
\pgfpathcurveto{\pgfqpoint{1.114744in}{1.424122in}}{\pgfqpoint{1.111471in}{1.416222in}}{\pgfqpoint{1.111471in}{1.407986in}}%
\pgfpathcurveto{\pgfqpoint{1.111471in}{1.399750in}}{\pgfqpoint{1.114744in}{1.391850in}}{\pgfqpoint{1.120568in}{1.386026in}}%
\pgfpathcurveto{\pgfqpoint{1.126392in}{1.380202in}}{\pgfqpoint{1.134292in}{1.376930in}}{\pgfqpoint{1.142528in}{1.376930in}}%
\pgfpathclose%
\pgfusepath{stroke,fill}%
\end{pgfscope}%
\begin{pgfscope}%
\pgfpathrectangle{\pgfqpoint{0.556847in}{0.516222in}}{\pgfqpoint{1.722590in}{1.783528in}} %
\pgfusepath{clip}%
\pgfsetbuttcap%
\pgfsetroundjoin%
\definecolor{currentfill}{rgb}{0.298039,0.447059,0.690196}%
\pgfsetfillcolor{currentfill}%
\pgfsetlinewidth{0.240900pt}%
\definecolor{currentstroke}{rgb}{1.000000,1.000000,1.000000}%
\pgfsetstrokecolor{currentstroke}%
\pgfsetdash{}{0pt}%
\pgfpathmoveto{\pgfqpoint{1.033431in}{0.693244in}}%
\pgfpathcurveto{\pgfqpoint{1.041667in}{0.693244in}}{\pgfqpoint{1.049567in}{0.696516in}}{\pgfqpoint{1.055391in}{0.702340in}}%
\pgfpathcurveto{\pgfqpoint{1.061215in}{0.708164in}}{\pgfqpoint{1.064487in}{0.716064in}}{\pgfqpoint{1.064487in}{0.724300in}}%
\pgfpathcurveto{\pgfqpoint{1.064487in}{0.732537in}}{\pgfqpoint{1.061215in}{0.740437in}}{\pgfqpoint{1.055391in}{0.746261in}}%
\pgfpathcurveto{\pgfqpoint{1.049567in}{0.752085in}}{\pgfqpoint{1.041667in}{0.755357in}}{\pgfqpoint{1.033431in}{0.755357in}}%
\pgfpathcurveto{\pgfqpoint{1.025194in}{0.755357in}}{\pgfqpoint{1.017294in}{0.752085in}}{\pgfqpoint{1.011470in}{0.746261in}}%
\pgfpathcurveto{\pgfqpoint{1.005646in}{0.740437in}}{\pgfqpoint{1.002374in}{0.732537in}}{\pgfqpoint{1.002374in}{0.724300in}}%
\pgfpathcurveto{\pgfqpoint{1.002374in}{0.716064in}}{\pgfqpoint{1.005646in}{0.708164in}}{\pgfqpoint{1.011470in}{0.702340in}}%
\pgfpathcurveto{\pgfqpoint{1.017294in}{0.696516in}}{\pgfqpoint{1.025194in}{0.693244in}}{\pgfqpoint{1.033431in}{0.693244in}}%
\pgfpathclose%
\pgfusepath{stroke,fill}%
\end{pgfscope}%
\begin{pgfscope}%
\pgfpathrectangle{\pgfqpoint{0.556847in}{0.516222in}}{\pgfqpoint{1.722590in}{1.783528in}} %
\pgfusepath{clip}%
\pgfsetbuttcap%
\pgfsetroundjoin%
\definecolor{currentfill}{rgb}{0.298039,0.447059,0.690196}%
\pgfsetfillcolor{currentfill}%
\pgfsetlinewidth{0.240900pt}%
\definecolor{currentstroke}{rgb}{1.000000,1.000000,1.000000}%
\pgfsetstrokecolor{currentstroke}%
\pgfsetdash{}{0pt}%
\pgfpathmoveto{\pgfqpoint{1.693757in}{2.001164in}}%
\pgfpathcurveto{\pgfqpoint{1.701993in}{2.001164in}}{\pgfqpoint{1.709893in}{2.004437in}}{\pgfqpoint{1.715717in}{2.010261in}}%
\pgfpathcurveto{\pgfqpoint{1.721541in}{2.016085in}}{\pgfqpoint{1.724813in}{2.023985in}}{\pgfqpoint{1.724813in}{2.032221in}}%
\pgfpathcurveto{\pgfqpoint{1.724813in}{2.040457in}}{\pgfqpoint{1.721541in}{2.048357in}}{\pgfqpoint{1.715717in}{2.054181in}}%
\pgfpathcurveto{\pgfqpoint{1.709893in}{2.060005in}}{\pgfqpoint{1.701993in}{2.063277in}}{\pgfqpoint{1.693757in}{2.063277in}}%
\pgfpathcurveto{\pgfqpoint{1.685521in}{2.063277in}}{\pgfqpoint{1.677620in}{2.060005in}}{\pgfqpoint{1.671797in}{2.054181in}}%
\pgfpathcurveto{\pgfqpoint{1.665973in}{2.048357in}}{\pgfqpoint{1.662700in}{2.040457in}}{\pgfqpoint{1.662700in}{2.032221in}}%
\pgfpathcurveto{\pgfqpoint{1.662700in}{2.023985in}}{\pgfqpoint{1.665973in}{2.016085in}}{\pgfqpoint{1.671797in}{2.010261in}}%
\pgfpathcurveto{\pgfqpoint{1.677620in}{2.004437in}}{\pgfqpoint{1.685521in}{2.001164in}}{\pgfqpoint{1.693757in}{2.001164in}}%
\pgfpathclose%
\pgfusepath{stroke,fill}%
\end{pgfscope}%
\begin{pgfscope}%
\pgfpathrectangle{\pgfqpoint{0.556847in}{0.516222in}}{\pgfqpoint{1.722590in}{1.783528in}} %
\pgfusepath{clip}%
\pgfsetbuttcap%
\pgfsetroundjoin%
\definecolor{currentfill}{rgb}{0.298039,0.447059,0.690196}%
\pgfsetfillcolor{currentfill}%
\pgfsetlinewidth{0.240900pt}%
\definecolor{currentstroke}{rgb}{1.000000,1.000000,1.000000}%
\pgfsetstrokecolor{currentstroke}%
\pgfsetdash{}{0pt}%
\pgfpathmoveto{\pgfqpoint{1.774144in}{1.466106in}}%
\pgfpathcurveto{\pgfqpoint{1.782381in}{1.466106in}}{\pgfqpoint{1.790281in}{1.469378in}}{\pgfqpoint{1.796105in}{1.475202in}}%
\pgfpathcurveto{\pgfqpoint{1.801929in}{1.481026in}}{\pgfqpoint{1.805201in}{1.488926in}}{\pgfqpoint{1.805201in}{1.497163in}}%
\pgfpathcurveto{\pgfqpoint{1.805201in}{1.505399in}}{\pgfqpoint{1.801929in}{1.513299in}}{\pgfqpoint{1.796105in}{1.519123in}}%
\pgfpathcurveto{\pgfqpoint{1.790281in}{1.524947in}}{\pgfqpoint{1.782381in}{1.528219in}}{\pgfqpoint{1.774144in}{1.528219in}}%
\pgfpathcurveto{\pgfqpoint{1.765908in}{1.528219in}}{\pgfqpoint{1.758008in}{1.524947in}}{\pgfqpoint{1.752184in}{1.519123in}}%
\pgfpathcurveto{\pgfqpoint{1.746360in}{1.513299in}}{\pgfqpoint{1.743088in}{1.505399in}}{\pgfqpoint{1.743088in}{1.497163in}}%
\pgfpathcurveto{\pgfqpoint{1.743088in}{1.488926in}}{\pgfqpoint{1.746360in}{1.481026in}}{\pgfqpoint{1.752184in}{1.475202in}}%
\pgfpathcurveto{\pgfqpoint{1.758008in}{1.469378in}}{\pgfqpoint{1.765908in}{1.466106in}}{\pgfqpoint{1.774144in}{1.466106in}}%
\pgfpathclose%
\pgfusepath{stroke,fill}%
\end{pgfscope}%
\begin{pgfscope}%
\pgfpathrectangle{\pgfqpoint{0.556847in}{0.516222in}}{\pgfqpoint{1.722590in}{1.783528in}} %
\pgfusepath{clip}%
\pgfsetbuttcap%
\pgfsetroundjoin%
\definecolor{currentfill}{rgb}{0.298039,0.447059,0.690196}%
\pgfsetfillcolor{currentfill}%
\pgfsetlinewidth{0.240900pt}%
\definecolor{currentstroke}{rgb}{1.000000,1.000000,1.000000}%
\pgfsetstrokecolor{currentstroke}%
\pgfsetdash{}{0pt}%
\pgfpathmoveto{\pgfqpoint{1.619111in}{1.258028in}}%
\pgfpathcurveto{\pgfqpoint{1.627348in}{1.258028in}}{\pgfqpoint{1.635248in}{1.261300in}}{\pgfqpoint{1.641071in}{1.267124in}}%
\pgfpathcurveto{\pgfqpoint{1.646895in}{1.272948in}}{\pgfqpoint{1.650168in}{1.280848in}}{\pgfqpoint{1.650168in}{1.289084in}}%
\pgfpathcurveto{\pgfqpoint{1.650168in}{1.297321in}}{\pgfqpoint{1.646895in}{1.305221in}}{\pgfqpoint{1.641071in}{1.311045in}}%
\pgfpathcurveto{\pgfqpoint{1.635248in}{1.316868in}}{\pgfqpoint{1.627348in}{1.320141in}}{\pgfqpoint{1.619111in}{1.320141in}}%
\pgfpathcurveto{\pgfqpoint{1.610875in}{1.320141in}}{\pgfqpoint{1.602975in}{1.316868in}}{\pgfqpoint{1.597151in}{1.311045in}}%
\pgfpathcurveto{\pgfqpoint{1.591327in}{1.305221in}}{\pgfqpoint{1.588055in}{1.297321in}}{\pgfqpoint{1.588055in}{1.289084in}}%
\pgfpathcurveto{\pgfqpoint{1.588055in}{1.280848in}}{\pgfqpoint{1.591327in}{1.272948in}}{\pgfqpoint{1.597151in}{1.267124in}}%
\pgfpathcurveto{\pgfqpoint{1.602975in}{1.261300in}}{\pgfqpoint{1.610875in}{1.258028in}}{\pgfqpoint{1.619111in}{1.258028in}}%
\pgfpathclose%
\pgfusepath{stroke,fill}%
\end{pgfscope}%
\begin{pgfscope}%
\pgfpathrectangle{\pgfqpoint{0.556847in}{0.516222in}}{\pgfqpoint{1.722590in}{1.783528in}} %
\pgfusepath{clip}%
\pgfsetbuttcap%
\pgfsetroundjoin%
\definecolor{currentfill}{rgb}{0.298039,0.447059,0.690196}%
\pgfsetfillcolor{currentfill}%
\pgfsetlinewidth{0.240900pt}%
\definecolor{currentstroke}{rgb}{1.000000,1.000000,1.000000}%
\pgfsetstrokecolor{currentstroke}%
\pgfsetdash{}{0pt}%
\pgfpathmoveto{\pgfqpoint{0.648719in}{0.633793in}}%
\pgfpathcurveto{\pgfqpoint{0.656955in}{0.633793in}}{\pgfqpoint{0.664855in}{0.637065in}}{\pgfqpoint{0.670679in}{0.642889in}}%
\pgfpathcurveto{\pgfqpoint{0.676503in}{0.648713in}}{\pgfqpoint{0.679775in}{0.656613in}}{\pgfqpoint{0.679775in}{0.664850in}}%
\pgfpathcurveto{\pgfqpoint{0.679775in}{0.673086in}}{\pgfqpoint{0.676503in}{0.680986in}}{\pgfqpoint{0.670679in}{0.686810in}}%
\pgfpathcurveto{\pgfqpoint{0.664855in}{0.692634in}}{\pgfqpoint{0.656955in}{0.695906in}}{\pgfqpoint{0.648719in}{0.695906in}}%
\pgfpathcurveto{\pgfqpoint{0.640482in}{0.695906in}}{\pgfqpoint{0.632582in}{0.692634in}}{\pgfqpoint{0.626758in}{0.686810in}}%
\pgfpathcurveto{\pgfqpoint{0.620935in}{0.680986in}}{\pgfqpoint{0.617662in}{0.673086in}}{\pgfqpoint{0.617662in}{0.664850in}}%
\pgfpathcurveto{\pgfqpoint{0.617662in}{0.656613in}}{\pgfqpoint{0.620935in}{0.648713in}}{\pgfqpoint{0.626758in}{0.642889in}}%
\pgfpathcurveto{\pgfqpoint{0.632582in}{0.637065in}}{\pgfqpoint{0.640482in}{0.633793in}}{\pgfqpoint{0.648719in}{0.633793in}}%
\pgfpathclose%
\pgfusepath{stroke,fill}%
\end{pgfscope}%
\begin{pgfscope}%
\pgfpathrectangle{\pgfqpoint{0.556847in}{0.516222in}}{\pgfqpoint{1.722590in}{1.783528in}} %
\pgfusepath{clip}%
\pgfsetbuttcap%
\pgfsetroundjoin%
\definecolor{currentfill}{rgb}{0.298039,0.447059,0.690196}%
\pgfsetfillcolor{currentfill}%
\pgfsetlinewidth{0.240900pt}%
\definecolor{currentstroke}{rgb}{1.000000,1.000000,1.000000}%
\pgfsetstrokecolor{currentstroke}%
\pgfsetdash{}{0pt}%
\pgfpathmoveto{\pgfqpoint{1.297561in}{1.049950in}}%
\pgfpathcurveto{\pgfqpoint{1.305797in}{1.049950in}}{\pgfqpoint{1.313697in}{1.053222in}}{\pgfqpoint{1.319521in}{1.059046in}}%
\pgfpathcurveto{\pgfqpoint{1.325345in}{1.064870in}}{\pgfqpoint{1.328618in}{1.072770in}}{\pgfqpoint{1.328618in}{1.081006in}}%
\pgfpathcurveto{\pgfqpoint{1.328618in}{1.089242in}}{\pgfqpoint{1.325345in}{1.097142in}}{\pgfqpoint{1.319521in}{1.102966in}}%
\pgfpathcurveto{\pgfqpoint{1.313697in}{1.108790in}}{\pgfqpoint{1.305797in}{1.112063in}}{\pgfqpoint{1.297561in}{1.112063in}}%
\pgfpathcurveto{\pgfqpoint{1.289325in}{1.112063in}}{\pgfqpoint{1.281425in}{1.108790in}}{\pgfqpoint{1.275601in}{1.102966in}}%
\pgfpathcurveto{\pgfqpoint{1.269777in}{1.097142in}}{\pgfqpoint{1.266505in}{1.089242in}}{\pgfqpoint{1.266505in}{1.081006in}}%
\pgfpathcurveto{\pgfqpoint{1.266505in}{1.072770in}}{\pgfqpoint{1.269777in}{1.064870in}}{\pgfqpoint{1.275601in}{1.059046in}}%
\pgfpathcurveto{\pgfqpoint{1.281425in}{1.053222in}}{\pgfqpoint{1.289325in}{1.049950in}}{\pgfqpoint{1.297561in}{1.049950in}}%
\pgfpathclose%
\pgfusepath{stroke,fill}%
\end{pgfscope}%
\begin{pgfscope}%
\pgfpathrectangle{\pgfqpoint{0.556847in}{0.516222in}}{\pgfqpoint{1.722590in}{1.783528in}} %
\pgfusepath{clip}%
\pgfsetbuttcap%
\pgfsetroundjoin%
\definecolor{currentfill}{rgb}{0.298039,0.447059,0.690196}%
\pgfsetfillcolor{currentfill}%
\pgfsetlinewidth{0.240900pt}%
\definecolor{currentstroke}{rgb}{1.000000,1.000000,1.000000}%
\pgfsetstrokecolor{currentstroke}%
\pgfsetdash{}{0pt}%
\pgfpathmoveto{\pgfqpoint{1.871758in}{1.139126in}}%
\pgfpathcurveto{\pgfqpoint{1.879994in}{1.139126in}}{\pgfqpoint{1.887894in}{1.142398in}}{\pgfqpoint{1.893718in}{1.148222in}}%
\pgfpathcurveto{\pgfqpoint{1.899542in}{1.154046in}}{\pgfqpoint{1.902814in}{1.161946in}}{\pgfqpoint{1.902814in}{1.170182in}}%
\pgfpathcurveto{\pgfqpoint{1.902814in}{1.178419in}}{\pgfqpoint{1.899542in}{1.186319in}}{\pgfqpoint{1.893718in}{1.192143in}}%
\pgfpathcurveto{\pgfqpoint{1.887894in}{1.197967in}}{\pgfqpoint{1.879994in}{1.201239in}}{\pgfqpoint{1.871758in}{1.201239in}}%
\pgfpathcurveto{\pgfqpoint{1.863522in}{1.201239in}}{\pgfqpoint{1.855621in}{1.197967in}}{\pgfqpoint{1.849798in}{1.192143in}}%
\pgfpathcurveto{\pgfqpoint{1.843974in}{1.186319in}}{\pgfqpoint{1.840701in}{1.178419in}}{\pgfqpoint{1.840701in}{1.170182in}}%
\pgfpathcurveto{\pgfqpoint{1.840701in}{1.161946in}}{\pgfqpoint{1.843974in}{1.154046in}}{\pgfqpoint{1.849798in}{1.148222in}}%
\pgfpathcurveto{\pgfqpoint{1.855621in}{1.142398in}}{\pgfqpoint{1.863522in}{1.139126in}}{\pgfqpoint{1.871758in}{1.139126in}}%
\pgfpathclose%
\pgfusepath{stroke,fill}%
\end{pgfscope}%
\begin{pgfscope}%
\pgfpathrectangle{\pgfqpoint{0.556847in}{0.516222in}}{\pgfqpoint{1.722590in}{1.783528in}} %
\pgfusepath{clip}%
\pgfsetbuttcap%
\pgfsetroundjoin%
\definecolor{currentfill}{rgb}{0.298039,0.447059,0.690196}%
\pgfsetfillcolor{currentfill}%
\pgfsetlinewidth{0.240900pt}%
\definecolor{currentstroke}{rgb}{1.000000,1.000000,1.000000}%
\pgfsetstrokecolor{currentstroke}%
\pgfsetdash{}{0pt}%
\pgfpathmoveto{\pgfqpoint{0.869784in}{1.495831in}}%
\pgfpathcurveto{\pgfqpoint{0.878021in}{1.495831in}}{\pgfqpoint{0.885921in}{1.499104in}}{\pgfqpoint{0.891745in}{1.504928in}}%
\pgfpathcurveto{\pgfqpoint{0.897569in}{1.510752in}}{\pgfqpoint{0.900841in}{1.518652in}}{\pgfqpoint{0.900841in}{1.526888in}}%
\pgfpathcurveto{\pgfqpoint{0.900841in}{1.535124in}}{\pgfqpoint{0.897569in}{1.543024in}}{\pgfqpoint{0.891745in}{1.548848in}}%
\pgfpathcurveto{\pgfqpoint{0.885921in}{1.554672in}}{\pgfqpoint{0.878021in}{1.557944in}}{\pgfqpoint{0.869784in}{1.557944in}}%
\pgfpathcurveto{\pgfqpoint{0.861548in}{1.557944in}}{\pgfqpoint{0.853648in}{1.554672in}}{\pgfqpoint{0.847824in}{1.548848in}}%
\pgfpathcurveto{\pgfqpoint{0.842000in}{1.543024in}}{\pgfqpoint{0.838728in}{1.535124in}}{\pgfqpoint{0.838728in}{1.526888in}}%
\pgfpathcurveto{\pgfqpoint{0.838728in}{1.518652in}}{\pgfqpoint{0.842000in}{1.510752in}}{\pgfqpoint{0.847824in}{1.504928in}}%
\pgfpathcurveto{\pgfqpoint{0.853648in}{1.499104in}}{\pgfqpoint{0.861548in}{1.495831in}}{\pgfqpoint{0.869784in}{1.495831in}}%
\pgfpathclose%
\pgfusepath{stroke,fill}%
\end{pgfscope}%
\begin{pgfscope}%
\pgfpathrectangle{\pgfqpoint{0.556847in}{0.516222in}}{\pgfqpoint{1.722590in}{1.783528in}} %
\pgfusepath{clip}%
\pgfsetbuttcap%
\pgfsetroundjoin%
\definecolor{currentfill}{rgb}{0.298039,0.447059,0.690196}%
\pgfsetfillcolor{currentfill}%
\pgfsetlinewidth{0.240900pt}%
\definecolor{currentstroke}{rgb}{1.000000,1.000000,1.000000}%
\pgfsetstrokecolor{currentstroke}%
\pgfsetdash{}{0pt}%
\pgfpathmoveto{\pgfqpoint{0.731977in}{0.901322in}}%
\pgfpathcurveto{\pgfqpoint{0.740214in}{0.901322in}}{\pgfqpoint{0.748114in}{0.904595in}}{\pgfqpoint{0.753937in}{0.910418in}}%
\pgfpathcurveto{\pgfqpoint{0.759761in}{0.916242in}}{\pgfqpoint{0.763034in}{0.924142in}}{\pgfqpoint{0.763034in}{0.932379in}}%
\pgfpathcurveto{\pgfqpoint{0.763034in}{0.940615in}}{\pgfqpoint{0.759761in}{0.948515in}}{\pgfqpoint{0.753937in}{0.954339in}}%
\pgfpathcurveto{\pgfqpoint{0.748114in}{0.960163in}}{\pgfqpoint{0.740214in}{0.963435in}}{\pgfqpoint{0.731977in}{0.963435in}}%
\pgfpathcurveto{\pgfqpoint{0.723741in}{0.963435in}}{\pgfqpoint{0.715841in}{0.960163in}}{\pgfqpoint{0.710017in}{0.954339in}}%
\pgfpathcurveto{\pgfqpoint{0.704193in}{0.948515in}}{\pgfqpoint{0.700921in}{0.940615in}}{\pgfqpoint{0.700921in}{0.932379in}}%
\pgfpathcurveto{\pgfqpoint{0.700921in}{0.924142in}}{\pgfqpoint{0.704193in}{0.916242in}}{\pgfqpoint{0.710017in}{0.910418in}}%
\pgfpathcurveto{\pgfqpoint{0.715841in}{0.904595in}}{\pgfqpoint{0.723741in}{0.901322in}}{\pgfqpoint{0.731977in}{0.901322in}}%
\pgfpathclose%
\pgfusepath{stroke,fill}%
\end{pgfscope}%
\begin{pgfscope}%
\pgfpathrectangle{\pgfqpoint{0.556847in}{0.516222in}}{\pgfqpoint{1.722590in}{1.783528in}} %
\pgfusepath{clip}%
\pgfsetbuttcap%
\pgfsetroundjoin%
\definecolor{currentfill}{rgb}{0.298039,0.447059,0.690196}%
\pgfsetfillcolor{currentfill}%
\pgfsetlinewidth{0.240900pt}%
\definecolor{currentstroke}{rgb}{1.000000,1.000000,1.000000}%
\pgfsetstrokecolor{currentstroke}%
\pgfsetdash{}{0pt}%
\pgfpathmoveto{\pgfqpoint{1.745435in}{1.644459in}}%
\pgfpathcurveto{\pgfqpoint{1.753671in}{1.644459in}}{\pgfqpoint{1.761571in}{1.647731in}}{\pgfqpoint{1.767395in}{1.653555in}}%
\pgfpathcurveto{\pgfqpoint{1.773219in}{1.659379in}}{\pgfqpoint{1.776491in}{1.667279in}}{\pgfqpoint{1.776491in}{1.675515in}}%
\pgfpathcurveto{\pgfqpoint{1.776491in}{1.683752in}}{\pgfqpoint{1.773219in}{1.691652in}}{\pgfqpoint{1.767395in}{1.697476in}}%
\pgfpathcurveto{\pgfqpoint{1.761571in}{1.703299in}}{\pgfqpoint{1.753671in}{1.706572in}}{\pgfqpoint{1.745435in}{1.706572in}}%
\pgfpathcurveto{\pgfqpoint{1.737198in}{1.706572in}}{\pgfqpoint{1.729298in}{1.703299in}}{\pgfqpoint{1.723474in}{1.697476in}}%
\pgfpathcurveto{\pgfqpoint{1.717650in}{1.691652in}}{\pgfqpoint{1.714378in}{1.683752in}}{\pgfqpoint{1.714378in}{1.675515in}}%
\pgfpathcurveto{\pgfqpoint{1.714378in}{1.667279in}}{\pgfqpoint{1.717650in}{1.659379in}}{\pgfqpoint{1.723474in}{1.653555in}}%
\pgfpathcurveto{\pgfqpoint{1.729298in}{1.647731in}}{\pgfqpoint{1.737198in}{1.644459in}}{\pgfqpoint{1.745435in}{1.644459in}}%
\pgfpathclose%
\pgfusepath{stroke,fill}%
\end{pgfscope}%
\begin{pgfscope}%
\pgfpathrectangle{\pgfqpoint{0.556847in}{0.516222in}}{\pgfqpoint{1.722590in}{1.783528in}} %
\pgfusepath{clip}%
\pgfsetbuttcap%
\pgfsetroundjoin%
\definecolor{currentfill}{rgb}{0.298039,0.447059,0.690196}%
\pgfsetfillcolor{currentfill}%
\pgfsetlinewidth{0.240900pt}%
\definecolor{currentstroke}{rgb}{1.000000,1.000000,1.000000}%
\pgfsetstrokecolor{currentstroke}%
\pgfsetdash{}{0pt}%
\pgfpathmoveto{\pgfqpoint{0.697525in}{0.871597in}}%
\pgfpathcurveto{\pgfqpoint{0.705762in}{0.871597in}}{\pgfqpoint{0.713662in}{0.874869in}}{\pgfqpoint{0.719486in}{0.880693in}}%
\pgfpathcurveto{\pgfqpoint{0.725310in}{0.886517in}}{\pgfqpoint{0.728582in}{0.894417in}}{\pgfqpoint{0.728582in}{0.902653in}}%
\pgfpathcurveto{\pgfqpoint{0.728582in}{0.910890in}}{\pgfqpoint{0.725310in}{0.918790in}}{\pgfqpoint{0.719486in}{0.924614in}}%
\pgfpathcurveto{\pgfqpoint{0.713662in}{0.930437in}}{\pgfqpoint{0.705762in}{0.933710in}}{\pgfqpoint{0.697525in}{0.933710in}}%
\pgfpathcurveto{\pgfqpoint{0.689289in}{0.933710in}}{\pgfqpoint{0.681389in}{0.930437in}}{\pgfqpoint{0.675565in}{0.924614in}}%
\pgfpathcurveto{\pgfqpoint{0.669741in}{0.918790in}}{\pgfqpoint{0.666469in}{0.910890in}}{\pgfqpoint{0.666469in}{0.902653in}}%
\pgfpathcurveto{\pgfqpoint{0.666469in}{0.894417in}}{\pgfqpoint{0.669741in}{0.886517in}}{\pgfqpoint{0.675565in}{0.880693in}}%
\pgfpathcurveto{\pgfqpoint{0.681389in}{0.874869in}}{\pgfqpoint{0.689289in}{0.871597in}}{\pgfqpoint{0.697525in}{0.871597in}}%
\pgfpathclose%
\pgfusepath{stroke,fill}%
\end{pgfscope}%
\begin{pgfscope}%
\pgfpathrectangle{\pgfqpoint{0.556847in}{0.516222in}}{\pgfqpoint{1.722590in}{1.783528in}} %
\pgfusepath{clip}%
\pgfsetbuttcap%
\pgfsetroundjoin%
\definecolor{currentfill}{rgb}{0.298039,0.447059,0.690196}%
\pgfsetfillcolor{currentfill}%
\pgfsetlinewidth{0.240900pt}%
\definecolor{currentstroke}{rgb}{1.000000,1.000000,1.000000}%
\pgfsetstrokecolor{currentstroke}%
\pgfsetdash{}{0pt}%
\pgfpathmoveto{\pgfqpoint{1.458336in}{1.198577in}}%
\pgfpathcurveto{\pgfqpoint{1.466572in}{1.198577in}}{\pgfqpoint{1.474472in}{1.201849in}}{\pgfqpoint{1.480296in}{1.207673in}}%
\pgfpathcurveto{\pgfqpoint{1.486120in}{1.213497in}}{\pgfqpoint{1.489393in}{1.221397in}}{\pgfqpoint{1.489393in}{1.229633in}}%
\pgfpathcurveto{\pgfqpoint{1.489393in}{1.237870in}}{\pgfqpoint{1.486120in}{1.245770in}}{\pgfqpoint{1.480296in}{1.251594in}}%
\pgfpathcurveto{\pgfqpoint{1.474472in}{1.257418in}}{\pgfqpoint{1.466572in}{1.260690in}}{\pgfqpoint{1.458336in}{1.260690in}}%
\pgfpathcurveto{\pgfqpoint{1.450100in}{1.260690in}}{\pgfqpoint{1.442200in}{1.257418in}}{\pgfqpoint{1.436376in}{1.251594in}}%
\pgfpathcurveto{\pgfqpoint{1.430552in}{1.245770in}}{\pgfqpoint{1.427280in}{1.237870in}}{\pgfqpoint{1.427280in}{1.229633in}}%
\pgfpathcurveto{\pgfqpoint{1.427280in}{1.221397in}}{\pgfqpoint{1.430552in}{1.213497in}}{\pgfqpoint{1.436376in}{1.207673in}}%
\pgfpathcurveto{\pgfqpoint{1.442200in}{1.201849in}}{\pgfqpoint{1.450100in}{1.198577in}}{\pgfqpoint{1.458336in}{1.198577in}}%
\pgfpathclose%
\pgfusepath{stroke,fill}%
\end{pgfscope}%
\begin{pgfscope}%
\pgfpathrectangle{\pgfqpoint{0.556847in}{0.516222in}}{\pgfqpoint{1.722590in}{1.783528in}} %
\pgfusepath{clip}%
\pgfsetbuttcap%
\pgfsetroundjoin%
\definecolor{currentfill}{rgb}{0.298039,0.447059,0.690196}%
\pgfsetfillcolor{currentfill}%
\pgfsetlinewidth{0.240900pt}%
\definecolor{currentstroke}{rgb}{1.000000,1.000000,1.000000}%
\pgfsetstrokecolor{currentstroke}%
\pgfsetdash{}{0pt}%
\pgfpathmoveto{\pgfqpoint{0.887010in}{1.406655in}}%
\pgfpathcurveto{\pgfqpoint{0.895247in}{1.406655in}}{\pgfqpoint{0.903147in}{1.409927in}}{\pgfqpoint{0.908971in}{1.415751in}}%
\pgfpathcurveto{\pgfqpoint{0.914795in}{1.421575in}}{\pgfqpoint{0.918067in}{1.429475in}}{\pgfqpoint{0.918067in}{1.437712in}}%
\pgfpathcurveto{\pgfqpoint{0.918067in}{1.445948in}}{\pgfqpoint{0.914795in}{1.453848in}}{\pgfqpoint{0.908971in}{1.459672in}}%
\pgfpathcurveto{\pgfqpoint{0.903147in}{1.465496in}}{\pgfqpoint{0.895247in}{1.468768in}}{\pgfqpoint{0.887010in}{1.468768in}}%
\pgfpathcurveto{\pgfqpoint{0.878774in}{1.468768in}}{\pgfqpoint{0.870874in}{1.465496in}}{\pgfqpoint{0.865050in}{1.459672in}}%
\pgfpathcurveto{\pgfqpoint{0.859226in}{1.453848in}}{\pgfqpoint{0.855954in}{1.445948in}}{\pgfqpoint{0.855954in}{1.437712in}}%
\pgfpathcurveto{\pgfqpoint{0.855954in}{1.429475in}}{\pgfqpoint{0.859226in}{1.421575in}}{\pgfqpoint{0.865050in}{1.415751in}}%
\pgfpathcurveto{\pgfqpoint{0.870874in}{1.409927in}}{\pgfqpoint{0.878774in}{1.406655in}}{\pgfqpoint{0.887010in}{1.406655in}}%
\pgfpathclose%
\pgfusepath{stroke,fill}%
\end{pgfscope}%
\begin{pgfscope}%
\pgfpathrectangle{\pgfqpoint{0.556847in}{0.516222in}}{\pgfqpoint{1.722590in}{1.783528in}} %
\pgfusepath{clip}%
\pgfsetbuttcap%
\pgfsetroundjoin%
\definecolor{currentfill}{rgb}{0.298039,0.447059,0.690196}%
\pgfsetfillcolor{currentfill}%
\pgfsetlinewidth{0.240900pt}%
\definecolor{currentstroke}{rgb}{1.000000,1.000000,1.000000}%
\pgfsetstrokecolor{currentstroke}%
\pgfsetdash{}{0pt}%
\pgfpathmoveto{\pgfqpoint{1.271722in}{1.139126in}}%
\pgfpathcurveto{\pgfqpoint{1.279958in}{1.139126in}}{\pgfqpoint{1.287859in}{1.142398in}}{\pgfqpoint{1.293682in}{1.148222in}}%
\pgfpathcurveto{\pgfqpoint{1.299506in}{1.154046in}}{\pgfqpoint{1.302779in}{1.161946in}}{\pgfqpoint{1.302779in}{1.170182in}}%
\pgfpathcurveto{\pgfqpoint{1.302779in}{1.178419in}}{\pgfqpoint{1.299506in}{1.186319in}}{\pgfqpoint{1.293682in}{1.192143in}}%
\pgfpathcurveto{\pgfqpoint{1.287859in}{1.197967in}}{\pgfqpoint{1.279958in}{1.201239in}}{\pgfqpoint{1.271722in}{1.201239in}}%
\pgfpathcurveto{\pgfqpoint{1.263486in}{1.201239in}}{\pgfqpoint{1.255586in}{1.197967in}}{\pgfqpoint{1.249762in}{1.192143in}}%
\pgfpathcurveto{\pgfqpoint{1.243938in}{1.186319in}}{\pgfqpoint{1.240666in}{1.178419in}}{\pgfqpoint{1.240666in}{1.170182in}}%
\pgfpathcurveto{\pgfqpoint{1.240666in}{1.161946in}}{\pgfqpoint{1.243938in}{1.154046in}}{\pgfqpoint{1.249762in}{1.148222in}}%
\pgfpathcurveto{\pgfqpoint{1.255586in}{1.142398in}}{\pgfqpoint{1.263486in}{1.139126in}}{\pgfqpoint{1.271722in}{1.139126in}}%
\pgfpathclose%
\pgfusepath{stroke,fill}%
\end{pgfscope}%
\begin{pgfscope}%
\pgfpathrectangle{\pgfqpoint{0.556847in}{0.516222in}}{\pgfqpoint{1.722590in}{1.783528in}} %
\pgfusepath{clip}%
\pgfsetbuttcap%
\pgfsetroundjoin%
\definecolor{currentfill}{rgb}{0.298039,0.447059,0.690196}%
\pgfsetfillcolor{currentfill}%
\pgfsetlinewidth{0.240900pt}%
\definecolor{currentstroke}{rgb}{1.000000,1.000000,1.000000}%
\pgfsetstrokecolor{currentstroke}%
\pgfsetdash{}{0pt}%
\pgfpathmoveto{\pgfqpoint{1.007592in}{1.079675in}}%
\pgfpathcurveto{\pgfqpoint{1.015828in}{1.079675in}}{\pgfqpoint{1.023728in}{1.082947in}}{\pgfqpoint{1.029552in}{1.088771in}}%
\pgfpathcurveto{\pgfqpoint{1.035376in}{1.094595in}}{\pgfqpoint{1.038648in}{1.102495in}}{\pgfqpoint{1.038648in}{1.110731in}}%
\pgfpathcurveto{\pgfqpoint{1.038648in}{1.118968in}}{\pgfqpoint{1.035376in}{1.126868in}}{\pgfqpoint{1.029552in}{1.132692in}}%
\pgfpathcurveto{\pgfqpoint{1.023728in}{1.138516in}}{\pgfqpoint{1.015828in}{1.141788in}}{\pgfqpoint{1.007592in}{1.141788in}}%
\pgfpathcurveto{\pgfqpoint{0.999355in}{1.141788in}}{\pgfqpoint{0.991455in}{1.138516in}}{\pgfqpoint{0.985631in}{1.132692in}}%
\pgfpathcurveto{\pgfqpoint{0.979807in}{1.126868in}}{\pgfqpoint{0.976535in}{1.118968in}}{\pgfqpoint{0.976535in}{1.110731in}}%
\pgfpathcurveto{\pgfqpoint{0.976535in}{1.102495in}}{\pgfqpoint{0.979807in}{1.094595in}}{\pgfqpoint{0.985631in}{1.088771in}}%
\pgfpathcurveto{\pgfqpoint{0.991455in}{1.082947in}}{\pgfqpoint{0.999355in}{1.079675in}}{\pgfqpoint{1.007592in}{1.079675in}}%
\pgfpathclose%
\pgfusepath{stroke,fill}%
\end{pgfscope}%
\begin{pgfscope}%
\pgfpathrectangle{\pgfqpoint{0.556847in}{0.516222in}}{\pgfqpoint{1.722590in}{1.783528in}} %
\pgfusepath{clip}%
\pgfsetbuttcap%
\pgfsetroundjoin%
\definecolor{currentfill}{rgb}{0.298039,0.447059,0.690196}%
\pgfsetfillcolor{currentfill}%
\pgfsetlinewidth{0.240900pt}%
\definecolor{currentstroke}{rgb}{1.000000,1.000000,1.000000}%
\pgfsetstrokecolor{currentstroke}%
\pgfsetdash{}{0pt}%
\pgfpathmoveto{\pgfqpoint{1.673660in}{1.168851in}}%
\pgfpathcurveto{\pgfqpoint{1.681896in}{1.168851in}}{\pgfqpoint{1.689796in}{1.172124in}}{\pgfqpoint{1.695620in}{1.177948in}}%
\pgfpathcurveto{\pgfqpoint{1.701444in}{1.183772in}}{\pgfqpoint{1.704716in}{1.191672in}}{\pgfqpoint{1.704716in}{1.199908in}}%
\pgfpathcurveto{\pgfqpoint{1.704716in}{1.208144in}}{\pgfqpoint{1.701444in}{1.216044in}}{\pgfqpoint{1.695620in}{1.221868in}}%
\pgfpathcurveto{\pgfqpoint{1.689796in}{1.227692in}}{\pgfqpoint{1.681896in}{1.230964in}}{\pgfqpoint{1.673660in}{1.230964in}}%
\pgfpathcurveto{\pgfqpoint{1.665424in}{1.230964in}}{\pgfqpoint{1.657524in}{1.227692in}}{\pgfqpoint{1.651700in}{1.221868in}}%
\pgfpathcurveto{\pgfqpoint{1.645876in}{1.216044in}}{\pgfqpoint{1.642603in}{1.208144in}}{\pgfqpoint{1.642603in}{1.199908in}}%
\pgfpathcurveto{\pgfqpoint{1.642603in}{1.191672in}}{\pgfqpoint{1.645876in}{1.183772in}}{\pgfqpoint{1.651700in}{1.177948in}}%
\pgfpathcurveto{\pgfqpoint{1.657524in}{1.172124in}}{\pgfqpoint{1.665424in}{1.168851in}}{\pgfqpoint{1.673660in}{1.168851in}}%
\pgfpathclose%
\pgfusepath{stroke,fill}%
\end{pgfscope}%
\begin{pgfscope}%
\pgfpathrectangle{\pgfqpoint{0.556847in}{0.516222in}}{\pgfqpoint{1.722590in}{1.783528in}} %
\pgfusepath{clip}%
\pgfsetbuttcap%
\pgfsetroundjoin%
\definecolor{currentfill}{rgb}{0.298039,0.447059,0.690196}%
\pgfsetfillcolor{currentfill}%
\pgfsetlinewidth{0.240900pt}%
\definecolor{currentstroke}{rgb}{1.000000,1.000000,1.000000}%
\pgfsetstrokecolor{currentstroke}%
\pgfsetdash{}{0pt}%
\pgfpathmoveto{\pgfqpoint{1.202819in}{1.763361in}}%
\pgfpathcurveto{\pgfqpoint{1.211055in}{1.763361in}}{\pgfqpoint{1.218955in}{1.766633in}}{\pgfqpoint{1.224779in}{1.772457in}}%
\pgfpathcurveto{\pgfqpoint{1.230603in}{1.778281in}}{\pgfqpoint{1.233875in}{1.786181in}}{\pgfqpoint{1.233875in}{1.794417in}}%
\pgfpathcurveto{\pgfqpoint{1.233875in}{1.802653in}}{\pgfqpoint{1.230603in}{1.810553in}}{\pgfqpoint{1.224779in}{1.816377in}}%
\pgfpathcurveto{\pgfqpoint{1.218955in}{1.822201in}}{\pgfqpoint{1.211055in}{1.825474in}}{\pgfqpoint{1.202819in}{1.825474in}}%
\pgfpathcurveto{\pgfqpoint{1.194582in}{1.825474in}}{\pgfqpoint{1.186682in}{1.822201in}}{\pgfqpoint{1.180858in}{1.816377in}}%
\pgfpathcurveto{\pgfqpoint{1.175034in}{1.810553in}}{\pgfqpoint{1.171762in}{1.802653in}}{\pgfqpoint{1.171762in}{1.794417in}}%
\pgfpathcurveto{\pgfqpoint{1.171762in}{1.786181in}}{\pgfqpoint{1.175034in}{1.778281in}}{\pgfqpoint{1.180858in}{1.772457in}}%
\pgfpathcurveto{\pgfqpoint{1.186682in}{1.766633in}}{\pgfqpoint{1.194582in}{1.763361in}}{\pgfqpoint{1.202819in}{1.763361in}}%
\pgfpathclose%
\pgfusepath{stroke,fill}%
\end{pgfscope}%
\begin{pgfscope}%
\pgfsetrectcap%
\pgfsetmiterjoin%
\pgfsetlinewidth{0.000000pt}%
\definecolor{currentstroke}{rgb}{1.000000,1.000000,1.000000}%
\pgfsetstrokecolor{currentstroke}%
\pgfsetdash{}{0pt}%
\pgfpathmoveto{\pgfqpoint{0.556847in}{0.516222in}}%
\pgfpathlineto{\pgfqpoint{0.556847in}{2.299750in}}%
\pgfusepath{}%
\end{pgfscope}%
\begin{pgfscope}%
\pgfsetrectcap%
\pgfsetmiterjoin%
\pgfsetlinewidth{0.000000pt}%
\definecolor{currentstroke}{rgb}{1.000000,1.000000,1.000000}%
\pgfsetstrokecolor{currentstroke}%
\pgfsetdash{}{0pt}%
\pgfpathmoveto{\pgfqpoint{0.556847in}{0.516222in}}%
\pgfpathlineto{\pgfqpoint{2.279437in}{0.516222in}}%
\pgfusepath{}%
\end{pgfscope}%
\end{pgfpicture}%
\makeatother%
\endgroup%

    \caption{Comparison between the tail and the arm lengths.}
    \label{fig_tl_al}
  \end{subfigure}
\end{figure}

\paragraph{Relations.}
According to the provided data, there seems to be a linear correlation between
the wing length and the falling times in the two realizations. This correlation
is plotted in~\cref{fig_wl_times}.
\begin{figure}
  \centering
  %% Creator: Matplotlib, PGF backend
%%
%% To include the figure in your LaTeX document, write
%%   \input{<filename>.pgf}
%%
%% Make sure the required packages are loaded in your preamble
%%   \usepackage{pgf}
%%
%% Figures using additional raster images can only be included by \input if
%% they are in the same directory as the main LaTeX file. For loading figures
%% from other directories you can use the `import` package
%%   \usepackage{import}
%% and then include the figures with
%%   \import{<path to file>}{<filename>.pgf}
%%
%% Matplotlib used the following preamble
%%   \usepackage[utf8x]{inputenc}
%%   \usepackage[T1]{fontenc}
%%   \usepackage{cmbright}
%%
\begingroup%
\makeatletter%
\begin{pgfpicture}%
\pgfpathrectangle{\pgfpointorigin}{\pgfqpoint{5.000000in}{2.500000in}}%
\pgfusepath{use as bounding box, clip}%
\begin{pgfscope}%
\pgfsetbuttcap%
\pgfsetmiterjoin%
\definecolor{currentfill}{rgb}{1.000000,1.000000,1.000000}%
\pgfsetfillcolor{currentfill}%
\pgfsetlinewidth{0.000000pt}%
\definecolor{currentstroke}{rgb}{1.000000,1.000000,1.000000}%
\pgfsetstrokecolor{currentstroke}%
\pgfsetdash{}{0pt}%
\pgfpathmoveto{\pgfqpoint{0.000000in}{0.000000in}}%
\pgfpathlineto{\pgfqpoint{5.000000in}{0.000000in}}%
\pgfpathlineto{\pgfqpoint{5.000000in}{2.500000in}}%
\pgfpathlineto{\pgfqpoint{0.000000in}{2.500000in}}%
\pgfpathclose%
\pgfusepath{fill}%
\end{pgfscope}%
\begin{pgfscope}%
\pgfsetbuttcap%
\pgfsetmiterjoin%
\definecolor{currentfill}{rgb}{0.917647,0.917647,0.949020}%
\pgfsetfillcolor{currentfill}%
\pgfsetlinewidth{0.000000pt}%
\definecolor{currentstroke}{rgb}{0.000000,0.000000,0.000000}%
\pgfsetstrokecolor{currentstroke}%
\pgfsetstrokeopacity{0.000000}%
\pgfsetdash{}{0pt}%
\pgfpathmoveto{\pgfqpoint{0.556847in}{0.516222in}}%
\pgfpathlineto{\pgfqpoint{2.519580in}{0.516222in}}%
\pgfpathlineto{\pgfqpoint{2.519580in}{2.299750in}}%
\pgfpathlineto{\pgfqpoint{0.556847in}{2.299750in}}%
\pgfpathclose%
\pgfusepath{fill}%
\end{pgfscope}%
\begin{pgfscope}%
\pgfpathrectangle{\pgfqpoint{0.556847in}{0.516222in}}{\pgfqpoint{1.962733in}{1.783528in}} %
\pgfusepath{clip}%
\pgfsetroundcap%
\pgfsetroundjoin%
\pgfsetlinewidth{0.803000pt}%
\definecolor{currentstroke}{rgb}{1.000000,1.000000,1.000000}%
\pgfsetstrokecolor{currentstroke}%
\pgfsetdash{}{0pt}%
\pgfpathmoveto{\pgfqpoint{0.556847in}{0.516222in}}%
\pgfpathlineto{\pgfqpoint{0.556847in}{2.299750in}}%
\pgfusepath{stroke}%
\end{pgfscope}%
\begin{pgfscope}%
\pgfsetbuttcap%
\pgfsetroundjoin%
\definecolor{currentfill}{rgb}{0.150000,0.150000,0.150000}%
\pgfsetfillcolor{currentfill}%
\pgfsetlinewidth{0.803000pt}%
\definecolor{currentstroke}{rgb}{0.150000,0.150000,0.150000}%
\pgfsetstrokecolor{currentstroke}%
\pgfsetdash{}{0pt}%
\pgfsys@defobject{currentmarker}{\pgfqpoint{0.000000in}{0.000000in}}{\pgfqpoint{0.000000in}{0.000000in}}{%
\pgfpathmoveto{\pgfqpoint{0.000000in}{0.000000in}}%
\pgfpathlineto{\pgfqpoint{0.000000in}{0.000000in}}%
\pgfusepath{stroke,fill}%
}%
\begin{pgfscope}%
\pgfsys@transformshift{0.556847in}{0.516222in}%
\pgfsys@useobject{currentmarker}{}%
\end{pgfscope}%
\end{pgfscope}%
\begin{pgfscope}%
\definecolor{textcolor}{rgb}{0.150000,0.150000,0.150000}%
\pgfsetstrokecolor{textcolor}%
\pgfsetfillcolor{textcolor}%
\pgftext[x=0.556847in,y=0.438444in,,top]{\color{textcolor}\sffamily\fontsize{8.000000}{9.600000}\selectfont 1.5}%
\end{pgfscope}%
\begin{pgfscope}%
\pgfpathrectangle{\pgfqpoint{0.556847in}{0.516222in}}{\pgfqpoint{1.962733in}{1.783528in}} %
\pgfusepath{clip}%
\pgfsetroundcap%
\pgfsetroundjoin%
\pgfsetlinewidth{0.803000pt}%
\definecolor{currentstroke}{rgb}{1.000000,1.000000,1.000000}%
\pgfsetstrokecolor{currentstroke}%
\pgfsetdash{}{0pt}%
\pgfpathmoveto{\pgfqpoint{0.837238in}{0.516222in}}%
\pgfpathlineto{\pgfqpoint{0.837238in}{2.299750in}}%
\pgfusepath{stroke}%
\end{pgfscope}%
\begin{pgfscope}%
\pgfsetbuttcap%
\pgfsetroundjoin%
\definecolor{currentfill}{rgb}{0.150000,0.150000,0.150000}%
\pgfsetfillcolor{currentfill}%
\pgfsetlinewidth{0.803000pt}%
\definecolor{currentstroke}{rgb}{0.150000,0.150000,0.150000}%
\pgfsetstrokecolor{currentstroke}%
\pgfsetdash{}{0pt}%
\pgfsys@defobject{currentmarker}{\pgfqpoint{0.000000in}{0.000000in}}{\pgfqpoint{0.000000in}{0.000000in}}{%
\pgfpathmoveto{\pgfqpoint{0.000000in}{0.000000in}}%
\pgfpathlineto{\pgfqpoint{0.000000in}{0.000000in}}%
\pgfusepath{stroke,fill}%
}%
\begin{pgfscope}%
\pgfsys@transformshift{0.837238in}{0.516222in}%
\pgfsys@useobject{currentmarker}{}%
\end{pgfscope}%
\end{pgfscope}%
\begin{pgfscope}%
\definecolor{textcolor}{rgb}{0.150000,0.150000,0.150000}%
\pgfsetstrokecolor{textcolor}%
\pgfsetfillcolor{textcolor}%
\pgftext[x=0.837238in,y=0.438444in,,top]{\color{textcolor}\sffamily\fontsize{8.000000}{9.600000}\selectfont 2.0}%
\end{pgfscope}%
\begin{pgfscope}%
\pgfpathrectangle{\pgfqpoint{0.556847in}{0.516222in}}{\pgfqpoint{1.962733in}{1.783528in}} %
\pgfusepath{clip}%
\pgfsetroundcap%
\pgfsetroundjoin%
\pgfsetlinewidth{0.803000pt}%
\definecolor{currentstroke}{rgb}{1.000000,1.000000,1.000000}%
\pgfsetstrokecolor{currentstroke}%
\pgfsetdash{}{0pt}%
\pgfpathmoveto{\pgfqpoint{1.117628in}{0.516222in}}%
\pgfpathlineto{\pgfqpoint{1.117628in}{2.299750in}}%
\pgfusepath{stroke}%
\end{pgfscope}%
\begin{pgfscope}%
\pgfsetbuttcap%
\pgfsetroundjoin%
\definecolor{currentfill}{rgb}{0.150000,0.150000,0.150000}%
\pgfsetfillcolor{currentfill}%
\pgfsetlinewidth{0.803000pt}%
\definecolor{currentstroke}{rgb}{0.150000,0.150000,0.150000}%
\pgfsetstrokecolor{currentstroke}%
\pgfsetdash{}{0pt}%
\pgfsys@defobject{currentmarker}{\pgfqpoint{0.000000in}{0.000000in}}{\pgfqpoint{0.000000in}{0.000000in}}{%
\pgfpathmoveto{\pgfqpoint{0.000000in}{0.000000in}}%
\pgfpathlineto{\pgfqpoint{0.000000in}{0.000000in}}%
\pgfusepath{stroke,fill}%
}%
\begin{pgfscope}%
\pgfsys@transformshift{1.117628in}{0.516222in}%
\pgfsys@useobject{currentmarker}{}%
\end{pgfscope}%
\end{pgfscope}%
\begin{pgfscope}%
\definecolor{textcolor}{rgb}{0.150000,0.150000,0.150000}%
\pgfsetstrokecolor{textcolor}%
\pgfsetfillcolor{textcolor}%
\pgftext[x=1.117628in,y=0.438444in,,top]{\color{textcolor}\sffamily\fontsize{8.000000}{9.600000}\selectfont 2.5}%
\end{pgfscope}%
\begin{pgfscope}%
\pgfpathrectangle{\pgfqpoint{0.556847in}{0.516222in}}{\pgfqpoint{1.962733in}{1.783528in}} %
\pgfusepath{clip}%
\pgfsetroundcap%
\pgfsetroundjoin%
\pgfsetlinewidth{0.803000pt}%
\definecolor{currentstroke}{rgb}{1.000000,1.000000,1.000000}%
\pgfsetstrokecolor{currentstroke}%
\pgfsetdash{}{0pt}%
\pgfpathmoveto{\pgfqpoint{1.398018in}{0.516222in}}%
\pgfpathlineto{\pgfqpoint{1.398018in}{2.299750in}}%
\pgfusepath{stroke}%
\end{pgfscope}%
\begin{pgfscope}%
\pgfsetbuttcap%
\pgfsetroundjoin%
\definecolor{currentfill}{rgb}{0.150000,0.150000,0.150000}%
\pgfsetfillcolor{currentfill}%
\pgfsetlinewidth{0.803000pt}%
\definecolor{currentstroke}{rgb}{0.150000,0.150000,0.150000}%
\pgfsetstrokecolor{currentstroke}%
\pgfsetdash{}{0pt}%
\pgfsys@defobject{currentmarker}{\pgfqpoint{0.000000in}{0.000000in}}{\pgfqpoint{0.000000in}{0.000000in}}{%
\pgfpathmoveto{\pgfqpoint{0.000000in}{0.000000in}}%
\pgfpathlineto{\pgfqpoint{0.000000in}{0.000000in}}%
\pgfusepath{stroke,fill}%
}%
\begin{pgfscope}%
\pgfsys@transformshift{1.398018in}{0.516222in}%
\pgfsys@useobject{currentmarker}{}%
\end{pgfscope}%
\end{pgfscope}%
\begin{pgfscope}%
\definecolor{textcolor}{rgb}{0.150000,0.150000,0.150000}%
\pgfsetstrokecolor{textcolor}%
\pgfsetfillcolor{textcolor}%
\pgftext[x=1.398018in,y=0.438444in,,top]{\color{textcolor}\sffamily\fontsize{8.000000}{9.600000}\selectfont 3.0}%
\end{pgfscope}%
\begin{pgfscope}%
\pgfpathrectangle{\pgfqpoint{0.556847in}{0.516222in}}{\pgfqpoint{1.962733in}{1.783528in}} %
\pgfusepath{clip}%
\pgfsetroundcap%
\pgfsetroundjoin%
\pgfsetlinewidth{0.803000pt}%
\definecolor{currentstroke}{rgb}{1.000000,1.000000,1.000000}%
\pgfsetstrokecolor{currentstroke}%
\pgfsetdash{}{0pt}%
\pgfpathmoveto{\pgfqpoint{1.678409in}{0.516222in}}%
\pgfpathlineto{\pgfqpoint{1.678409in}{2.299750in}}%
\pgfusepath{stroke}%
\end{pgfscope}%
\begin{pgfscope}%
\pgfsetbuttcap%
\pgfsetroundjoin%
\definecolor{currentfill}{rgb}{0.150000,0.150000,0.150000}%
\pgfsetfillcolor{currentfill}%
\pgfsetlinewidth{0.803000pt}%
\definecolor{currentstroke}{rgb}{0.150000,0.150000,0.150000}%
\pgfsetstrokecolor{currentstroke}%
\pgfsetdash{}{0pt}%
\pgfsys@defobject{currentmarker}{\pgfqpoint{0.000000in}{0.000000in}}{\pgfqpoint{0.000000in}{0.000000in}}{%
\pgfpathmoveto{\pgfqpoint{0.000000in}{0.000000in}}%
\pgfpathlineto{\pgfqpoint{0.000000in}{0.000000in}}%
\pgfusepath{stroke,fill}%
}%
\begin{pgfscope}%
\pgfsys@transformshift{1.678409in}{0.516222in}%
\pgfsys@useobject{currentmarker}{}%
\end{pgfscope}%
\end{pgfscope}%
\begin{pgfscope}%
\definecolor{textcolor}{rgb}{0.150000,0.150000,0.150000}%
\pgfsetstrokecolor{textcolor}%
\pgfsetfillcolor{textcolor}%
\pgftext[x=1.678409in,y=0.438444in,,top]{\color{textcolor}\sffamily\fontsize{8.000000}{9.600000}\selectfont 3.5}%
\end{pgfscope}%
\begin{pgfscope}%
\pgfpathrectangle{\pgfqpoint{0.556847in}{0.516222in}}{\pgfqpoint{1.962733in}{1.783528in}} %
\pgfusepath{clip}%
\pgfsetroundcap%
\pgfsetroundjoin%
\pgfsetlinewidth{0.803000pt}%
\definecolor{currentstroke}{rgb}{1.000000,1.000000,1.000000}%
\pgfsetstrokecolor{currentstroke}%
\pgfsetdash{}{0pt}%
\pgfpathmoveto{\pgfqpoint{1.958799in}{0.516222in}}%
\pgfpathlineto{\pgfqpoint{1.958799in}{2.299750in}}%
\pgfusepath{stroke}%
\end{pgfscope}%
\begin{pgfscope}%
\pgfsetbuttcap%
\pgfsetroundjoin%
\definecolor{currentfill}{rgb}{0.150000,0.150000,0.150000}%
\pgfsetfillcolor{currentfill}%
\pgfsetlinewidth{0.803000pt}%
\definecolor{currentstroke}{rgb}{0.150000,0.150000,0.150000}%
\pgfsetstrokecolor{currentstroke}%
\pgfsetdash{}{0pt}%
\pgfsys@defobject{currentmarker}{\pgfqpoint{0.000000in}{0.000000in}}{\pgfqpoint{0.000000in}{0.000000in}}{%
\pgfpathmoveto{\pgfqpoint{0.000000in}{0.000000in}}%
\pgfpathlineto{\pgfqpoint{0.000000in}{0.000000in}}%
\pgfusepath{stroke,fill}%
}%
\begin{pgfscope}%
\pgfsys@transformshift{1.958799in}{0.516222in}%
\pgfsys@useobject{currentmarker}{}%
\end{pgfscope}%
\end{pgfscope}%
\begin{pgfscope}%
\definecolor{textcolor}{rgb}{0.150000,0.150000,0.150000}%
\pgfsetstrokecolor{textcolor}%
\pgfsetfillcolor{textcolor}%
\pgftext[x=1.958799in,y=0.438444in,,top]{\color{textcolor}\sffamily\fontsize{8.000000}{9.600000}\selectfont 4.0}%
\end{pgfscope}%
\begin{pgfscope}%
\pgfpathrectangle{\pgfqpoint{0.556847in}{0.516222in}}{\pgfqpoint{1.962733in}{1.783528in}} %
\pgfusepath{clip}%
\pgfsetroundcap%
\pgfsetroundjoin%
\pgfsetlinewidth{0.803000pt}%
\definecolor{currentstroke}{rgb}{1.000000,1.000000,1.000000}%
\pgfsetstrokecolor{currentstroke}%
\pgfsetdash{}{0pt}%
\pgfpathmoveto{\pgfqpoint{2.239189in}{0.516222in}}%
\pgfpathlineto{\pgfqpoint{2.239189in}{2.299750in}}%
\pgfusepath{stroke}%
\end{pgfscope}%
\begin{pgfscope}%
\pgfsetbuttcap%
\pgfsetroundjoin%
\definecolor{currentfill}{rgb}{0.150000,0.150000,0.150000}%
\pgfsetfillcolor{currentfill}%
\pgfsetlinewidth{0.803000pt}%
\definecolor{currentstroke}{rgb}{0.150000,0.150000,0.150000}%
\pgfsetstrokecolor{currentstroke}%
\pgfsetdash{}{0pt}%
\pgfsys@defobject{currentmarker}{\pgfqpoint{0.000000in}{0.000000in}}{\pgfqpoint{0.000000in}{0.000000in}}{%
\pgfpathmoveto{\pgfqpoint{0.000000in}{0.000000in}}%
\pgfpathlineto{\pgfqpoint{0.000000in}{0.000000in}}%
\pgfusepath{stroke,fill}%
}%
\begin{pgfscope}%
\pgfsys@transformshift{2.239189in}{0.516222in}%
\pgfsys@useobject{currentmarker}{}%
\end{pgfscope}%
\end{pgfscope}%
\begin{pgfscope}%
\definecolor{textcolor}{rgb}{0.150000,0.150000,0.150000}%
\pgfsetstrokecolor{textcolor}%
\pgfsetfillcolor{textcolor}%
\pgftext[x=2.239189in,y=0.438444in,,top]{\color{textcolor}\sffamily\fontsize{8.000000}{9.600000}\selectfont 4.5}%
\end{pgfscope}%
\begin{pgfscope}%
\pgfpathrectangle{\pgfqpoint{0.556847in}{0.516222in}}{\pgfqpoint{1.962733in}{1.783528in}} %
\pgfusepath{clip}%
\pgfsetroundcap%
\pgfsetroundjoin%
\pgfsetlinewidth{0.803000pt}%
\definecolor{currentstroke}{rgb}{1.000000,1.000000,1.000000}%
\pgfsetstrokecolor{currentstroke}%
\pgfsetdash{}{0pt}%
\pgfpathmoveto{\pgfqpoint{2.519580in}{0.516222in}}%
\pgfpathlineto{\pgfqpoint{2.519580in}{2.299750in}}%
\pgfusepath{stroke}%
\end{pgfscope}%
\begin{pgfscope}%
\pgfsetbuttcap%
\pgfsetroundjoin%
\definecolor{currentfill}{rgb}{0.150000,0.150000,0.150000}%
\pgfsetfillcolor{currentfill}%
\pgfsetlinewidth{0.803000pt}%
\definecolor{currentstroke}{rgb}{0.150000,0.150000,0.150000}%
\pgfsetstrokecolor{currentstroke}%
\pgfsetdash{}{0pt}%
\pgfsys@defobject{currentmarker}{\pgfqpoint{0.000000in}{0.000000in}}{\pgfqpoint{0.000000in}{0.000000in}}{%
\pgfpathmoveto{\pgfqpoint{0.000000in}{0.000000in}}%
\pgfpathlineto{\pgfqpoint{0.000000in}{0.000000in}}%
\pgfusepath{stroke,fill}%
}%
\begin{pgfscope}%
\pgfsys@transformshift{2.519580in}{0.516222in}%
\pgfsys@useobject{currentmarker}{}%
\end{pgfscope}%
\end{pgfscope}%
\begin{pgfscope}%
\definecolor{textcolor}{rgb}{0.150000,0.150000,0.150000}%
\pgfsetstrokecolor{textcolor}%
\pgfsetfillcolor{textcolor}%
\pgftext[x=2.519580in,y=0.438444in,,top]{\color{textcolor}\sffamily\fontsize{8.000000}{9.600000}\selectfont 5.0}%
\end{pgfscope}%
\begin{pgfscope}%
\definecolor{textcolor}{rgb}{0.150000,0.150000,0.150000}%
\pgfsetstrokecolor{textcolor}%
\pgfsetfillcolor{textcolor}%
\pgftext[x=1.538214in,y=0.273321in,,top]{\color{textcolor}\sffamily\fontsize{8.800000}{10.560000}\selectfont Falling time realization 1}%
\end{pgfscope}%
\begin{pgfscope}%
\pgfpathrectangle{\pgfqpoint{0.556847in}{0.516222in}}{\pgfqpoint{1.962733in}{1.783528in}} %
\pgfusepath{clip}%
\pgfsetroundcap%
\pgfsetroundjoin%
\pgfsetlinewidth{0.803000pt}%
\definecolor{currentstroke}{rgb}{1.000000,1.000000,1.000000}%
\pgfsetstrokecolor{currentstroke}%
\pgfsetdash{}{0pt}%
\pgfpathmoveto{\pgfqpoint{0.556847in}{0.516222in}}%
\pgfpathlineto{\pgfqpoint{2.519580in}{0.516222in}}%
\pgfusepath{stroke}%
\end{pgfscope}%
\begin{pgfscope}%
\pgfsetbuttcap%
\pgfsetroundjoin%
\definecolor{currentfill}{rgb}{0.150000,0.150000,0.150000}%
\pgfsetfillcolor{currentfill}%
\pgfsetlinewidth{0.803000pt}%
\definecolor{currentstroke}{rgb}{0.150000,0.150000,0.150000}%
\pgfsetstrokecolor{currentstroke}%
\pgfsetdash{}{0pt}%
\pgfsys@defobject{currentmarker}{\pgfqpoint{0.000000in}{0.000000in}}{\pgfqpoint{0.000000in}{0.000000in}}{%
\pgfpathmoveto{\pgfqpoint{0.000000in}{0.000000in}}%
\pgfpathlineto{\pgfqpoint{0.000000in}{0.000000in}}%
\pgfusepath{stroke,fill}%
}%
\begin{pgfscope}%
\pgfsys@transformshift{0.556847in}{0.516222in}%
\pgfsys@useobject{currentmarker}{}%
\end{pgfscope}%
\end{pgfscope}%
\begin{pgfscope}%
\definecolor{textcolor}{rgb}{0.150000,0.150000,0.150000}%
\pgfsetstrokecolor{textcolor}%
\pgfsetfillcolor{textcolor}%
\pgftext[x=0.479069in,y=0.516222in,right,]{\color{textcolor}\sffamily\fontsize{8.000000}{9.600000}\selectfont 3.0}%
\end{pgfscope}%
\begin{pgfscope}%
\pgfpathrectangle{\pgfqpoint{0.556847in}{0.516222in}}{\pgfqpoint{1.962733in}{1.783528in}} %
\pgfusepath{clip}%
\pgfsetroundcap%
\pgfsetroundjoin%
\pgfsetlinewidth{0.803000pt}%
\definecolor{currentstroke}{rgb}{1.000000,1.000000,1.000000}%
\pgfsetstrokecolor{currentstroke}%
\pgfsetdash{}{0pt}%
\pgfpathmoveto{\pgfqpoint{0.556847in}{0.739163in}}%
\pgfpathlineto{\pgfqpoint{2.519580in}{0.739163in}}%
\pgfusepath{stroke}%
\end{pgfscope}%
\begin{pgfscope}%
\pgfsetbuttcap%
\pgfsetroundjoin%
\definecolor{currentfill}{rgb}{0.150000,0.150000,0.150000}%
\pgfsetfillcolor{currentfill}%
\pgfsetlinewidth{0.803000pt}%
\definecolor{currentstroke}{rgb}{0.150000,0.150000,0.150000}%
\pgfsetstrokecolor{currentstroke}%
\pgfsetdash{}{0pt}%
\pgfsys@defobject{currentmarker}{\pgfqpoint{0.000000in}{0.000000in}}{\pgfqpoint{0.000000in}{0.000000in}}{%
\pgfpathmoveto{\pgfqpoint{0.000000in}{0.000000in}}%
\pgfpathlineto{\pgfqpoint{0.000000in}{0.000000in}}%
\pgfusepath{stroke,fill}%
}%
\begin{pgfscope}%
\pgfsys@transformshift{0.556847in}{0.739163in}%
\pgfsys@useobject{currentmarker}{}%
\end{pgfscope}%
\end{pgfscope}%
\begin{pgfscope}%
\definecolor{textcolor}{rgb}{0.150000,0.150000,0.150000}%
\pgfsetstrokecolor{textcolor}%
\pgfsetfillcolor{textcolor}%
\pgftext[x=0.479069in,y=0.739163in,right,]{\color{textcolor}\sffamily\fontsize{8.000000}{9.600000}\selectfont 3.5}%
\end{pgfscope}%
\begin{pgfscope}%
\pgfpathrectangle{\pgfqpoint{0.556847in}{0.516222in}}{\pgfqpoint{1.962733in}{1.783528in}} %
\pgfusepath{clip}%
\pgfsetroundcap%
\pgfsetroundjoin%
\pgfsetlinewidth{0.803000pt}%
\definecolor{currentstroke}{rgb}{1.000000,1.000000,1.000000}%
\pgfsetstrokecolor{currentstroke}%
\pgfsetdash{}{0pt}%
\pgfpathmoveto{\pgfqpoint{0.556847in}{0.962104in}}%
\pgfpathlineto{\pgfqpoint{2.519580in}{0.962104in}}%
\pgfusepath{stroke}%
\end{pgfscope}%
\begin{pgfscope}%
\pgfsetbuttcap%
\pgfsetroundjoin%
\definecolor{currentfill}{rgb}{0.150000,0.150000,0.150000}%
\pgfsetfillcolor{currentfill}%
\pgfsetlinewidth{0.803000pt}%
\definecolor{currentstroke}{rgb}{0.150000,0.150000,0.150000}%
\pgfsetstrokecolor{currentstroke}%
\pgfsetdash{}{0pt}%
\pgfsys@defobject{currentmarker}{\pgfqpoint{0.000000in}{0.000000in}}{\pgfqpoint{0.000000in}{0.000000in}}{%
\pgfpathmoveto{\pgfqpoint{0.000000in}{0.000000in}}%
\pgfpathlineto{\pgfqpoint{0.000000in}{0.000000in}}%
\pgfusepath{stroke,fill}%
}%
\begin{pgfscope}%
\pgfsys@transformshift{0.556847in}{0.962104in}%
\pgfsys@useobject{currentmarker}{}%
\end{pgfscope}%
\end{pgfscope}%
\begin{pgfscope}%
\definecolor{textcolor}{rgb}{0.150000,0.150000,0.150000}%
\pgfsetstrokecolor{textcolor}%
\pgfsetfillcolor{textcolor}%
\pgftext[x=0.479069in,y=0.962104in,right,]{\color{textcolor}\sffamily\fontsize{8.000000}{9.600000}\selectfont 4.0}%
\end{pgfscope}%
\begin{pgfscope}%
\pgfpathrectangle{\pgfqpoint{0.556847in}{0.516222in}}{\pgfqpoint{1.962733in}{1.783528in}} %
\pgfusepath{clip}%
\pgfsetroundcap%
\pgfsetroundjoin%
\pgfsetlinewidth{0.803000pt}%
\definecolor{currentstroke}{rgb}{1.000000,1.000000,1.000000}%
\pgfsetstrokecolor{currentstroke}%
\pgfsetdash{}{0pt}%
\pgfpathmoveto{\pgfqpoint{0.556847in}{1.185045in}}%
\pgfpathlineto{\pgfqpoint{2.519580in}{1.185045in}}%
\pgfusepath{stroke}%
\end{pgfscope}%
\begin{pgfscope}%
\pgfsetbuttcap%
\pgfsetroundjoin%
\definecolor{currentfill}{rgb}{0.150000,0.150000,0.150000}%
\pgfsetfillcolor{currentfill}%
\pgfsetlinewidth{0.803000pt}%
\definecolor{currentstroke}{rgb}{0.150000,0.150000,0.150000}%
\pgfsetstrokecolor{currentstroke}%
\pgfsetdash{}{0pt}%
\pgfsys@defobject{currentmarker}{\pgfqpoint{0.000000in}{0.000000in}}{\pgfqpoint{0.000000in}{0.000000in}}{%
\pgfpathmoveto{\pgfqpoint{0.000000in}{0.000000in}}%
\pgfpathlineto{\pgfqpoint{0.000000in}{0.000000in}}%
\pgfusepath{stroke,fill}%
}%
\begin{pgfscope}%
\pgfsys@transformshift{0.556847in}{1.185045in}%
\pgfsys@useobject{currentmarker}{}%
\end{pgfscope}%
\end{pgfscope}%
\begin{pgfscope}%
\definecolor{textcolor}{rgb}{0.150000,0.150000,0.150000}%
\pgfsetstrokecolor{textcolor}%
\pgfsetfillcolor{textcolor}%
\pgftext[x=0.479069in,y=1.185045in,right,]{\color{textcolor}\sffamily\fontsize{8.000000}{9.600000}\selectfont 4.5}%
\end{pgfscope}%
\begin{pgfscope}%
\pgfpathrectangle{\pgfqpoint{0.556847in}{0.516222in}}{\pgfqpoint{1.962733in}{1.783528in}} %
\pgfusepath{clip}%
\pgfsetroundcap%
\pgfsetroundjoin%
\pgfsetlinewidth{0.803000pt}%
\definecolor{currentstroke}{rgb}{1.000000,1.000000,1.000000}%
\pgfsetstrokecolor{currentstroke}%
\pgfsetdash{}{0pt}%
\pgfpathmoveto{\pgfqpoint{0.556847in}{1.407986in}}%
\pgfpathlineto{\pgfqpoint{2.519580in}{1.407986in}}%
\pgfusepath{stroke}%
\end{pgfscope}%
\begin{pgfscope}%
\pgfsetbuttcap%
\pgfsetroundjoin%
\definecolor{currentfill}{rgb}{0.150000,0.150000,0.150000}%
\pgfsetfillcolor{currentfill}%
\pgfsetlinewidth{0.803000pt}%
\definecolor{currentstroke}{rgb}{0.150000,0.150000,0.150000}%
\pgfsetstrokecolor{currentstroke}%
\pgfsetdash{}{0pt}%
\pgfsys@defobject{currentmarker}{\pgfqpoint{0.000000in}{0.000000in}}{\pgfqpoint{0.000000in}{0.000000in}}{%
\pgfpathmoveto{\pgfqpoint{0.000000in}{0.000000in}}%
\pgfpathlineto{\pgfqpoint{0.000000in}{0.000000in}}%
\pgfusepath{stroke,fill}%
}%
\begin{pgfscope}%
\pgfsys@transformshift{0.556847in}{1.407986in}%
\pgfsys@useobject{currentmarker}{}%
\end{pgfscope}%
\end{pgfscope}%
\begin{pgfscope}%
\definecolor{textcolor}{rgb}{0.150000,0.150000,0.150000}%
\pgfsetstrokecolor{textcolor}%
\pgfsetfillcolor{textcolor}%
\pgftext[x=0.479069in,y=1.407986in,right,]{\color{textcolor}\sffamily\fontsize{8.000000}{9.600000}\selectfont 5.0}%
\end{pgfscope}%
\begin{pgfscope}%
\pgfpathrectangle{\pgfqpoint{0.556847in}{0.516222in}}{\pgfqpoint{1.962733in}{1.783528in}} %
\pgfusepath{clip}%
\pgfsetroundcap%
\pgfsetroundjoin%
\pgfsetlinewidth{0.803000pt}%
\definecolor{currentstroke}{rgb}{1.000000,1.000000,1.000000}%
\pgfsetstrokecolor{currentstroke}%
\pgfsetdash{}{0pt}%
\pgfpathmoveto{\pgfqpoint{0.556847in}{1.630927in}}%
\pgfpathlineto{\pgfqpoint{2.519580in}{1.630927in}}%
\pgfusepath{stroke}%
\end{pgfscope}%
\begin{pgfscope}%
\pgfsetbuttcap%
\pgfsetroundjoin%
\definecolor{currentfill}{rgb}{0.150000,0.150000,0.150000}%
\pgfsetfillcolor{currentfill}%
\pgfsetlinewidth{0.803000pt}%
\definecolor{currentstroke}{rgb}{0.150000,0.150000,0.150000}%
\pgfsetstrokecolor{currentstroke}%
\pgfsetdash{}{0pt}%
\pgfsys@defobject{currentmarker}{\pgfqpoint{0.000000in}{0.000000in}}{\pgfqpoint{0.000000in}{0.000000in}}{%
\pgfpathmoveto{\pgfqpoint{0.000000in}{0.000000in}}%
\pgfpathlineto{\pgfqpoint{0.000000in}{0.000000in}}%
\pgfusepath{stroke,fill}%
}%
\begin{pgfscope}%
\pgfsys@transformshift{0.556847in}{1.630927in}%
\pgfsys@useobject{currentmarker}{}%
\end{pgfscope}%
\end{pgfscope}%
\begin{pgfscope}%
\definecolor{textcolor}{rgb}{0.150000,0.150000,0.150000}%
\pgfsetstrokecolor{textcolor}%
\pgfsetfillcolor{textcolor}%
\pgftext[x=0.479069in,y=1.630927in,right,]{\color{textcolor}\sffamily\fontsize{8.000000}{9.600000}\selectfont 5.5}%
\end{pgfscope}%
\begin{pgfscope}%
\pgfpathrectangle{\pgfqpoint{0.556847in}{0.516222in}}{\pgfqpoint{1.962733in}{1.783528in}} %
\pgfusepath{clip}%
\pgfsetroundcap%
\pgfsetroundjoin%
\pgfsetlinewidth{0.803000pt}%
\definecolor{currentstroke}{rgb}{1.000000,1.000000,1.000000}%
\pgfsetstrokecolor{currentstroke}%
\pgfsetdash{}{0pt}%
\pgfpathmoveto{\pgfqpoint{0.556847in}{1.853868in}}%
\pgfpathlineto{\pgfqpoint{2.519580in}{1.853868in}}%
\pgfusepath{stroke}%
\end{pgfscope}%
\begin{pgfscope}%
\pgfsetbuttcap%
\pgfsetroundjoin%
\definecolor{currentfill}{rgb}{0.150000,0.150000,0.150000}%
\pgfsetfillcolor{currentfill}%
\pgfsetlinewidth{0.803000pt}%
\definecolor{currentstroke}{rgb}{0.150000,0.150000,0.150000}%
\pgfsetstrokecolor{currentstroke}%
\pgfsetdash{}{0pt}%
\pgfsys@defobject{currentmarker}{\pgfqpoint{0.000000in}{0.000000in}}{\pgfqpoint{0.000000in}{0.000000in}}{%
\pgfpathmoveto{\pgfqpoint{0.000000in}{0.000000in}}%
\pgfpathlineto{\pgfqpoint{0.000000in}{0.000000in}}%
\pgfusepath{stroke,fill}%
}%
\begin{pgfscope}%
\pgfsys@transformshift{0.556847in}{1.853868in}%
\pgfsys@useobject{currentmarker}{}%
\end{pgfscope}%
\end{pgfscope}%
\begin{pgfscope}%
\definecolor{textcolor}{rgb}{0.150000,0.150000,0.150000}%
\pgfsetstrokecolor{textcolor}%
\pgfsetfillcolor{textcolor}%
\pgftext[x=0.479069in,y=1.853868in,right,]{\color{textcolor}\sffamily\fontsize{8.000000}{9.600000}\selectfont 6.0}%
\end{pgfscope}%
\begin{pgfscope}%
\pgfpathrectangle{\pgfqpoint{0.556847in}{0.516222in}}{\pgfqpoint{1.962733in}{1.783528in}} %
\pgfusepath{clip}%
\pgfsetroundcap%
\pgfsetroundjoin%
\pgfsetlinewidth{0.803000pt}%
\definecolor{currentstroke}{rgb}{1.000000,1.000000,1.000000}%
\pgfsetstrokecolor{currentstroke}%
\pgfsetdash{}{0pt}%
\pgfpathmoveto{\pgfqpoint{0.556847in}{2.076809in}}%
\pgfpathlineto{\pgfqpoint{2.519580in}{2.076809in}}%
\pgfusepath{stroke}%
\end{pgfscope}%
\begin{pgfscope}%
\pgfsetbuttcap%
\pgfsetroundjoin%
\definecolor{currentfill}{rgb}{0.150000,0.150000,0.150000}%
\pgfsetfillcolor{currentfill}%
\pgfsetlinewidth{0.803000pt}%
\definecolor{currentstroke}{rgb}{0.150000,0.150000,0.150000}%
\pgfsetstrokecolor{currentstroke}%
\pgfsetdash{}{0pt}%
\pgfsys@defobject{currentmarker}{\pgfqpoint{0.000000in}{0.000000in}}{\pgfqpoint{0.000000in}{0.000000in}}{%
\pgfpathmoveto{\pgfqpoint{0.000000in}{0.000000in}}%
\pgfpathlineto{\pgfqpoint{0.000000in}{0.000000in}}%
\pgfusepath{stroke,fill}%
}%
\begin{pgfscope}%
\pgfsys@transformshift{0.556847in}{2.076809in}%
\pgfsys@useobject{currentmarker}{}%
\end{pgfscope}%
\end{pgfscope}%
\begin{pgfscope}%
\definecolor{textcolor}{rgb}{0.150000,0.150000,0.150000}%
\pgfsetstrokecolor{textcolor}%
\pgfsetfillcolor{textcolor}%
\pgftext[x=0.479069in,y=2.076809in,right,]{\color{textcolor}\sffamily\fontsize{8.000000}{9.600000}\selectfont 6.5}%
\end{pgfscope}%
\begin{pgfscope}%
\pgfpathrectangle{\pgfqpoint{0.556847in}{0.516222in}}{\pgfqpoint{1.962733in}{1.783528in}} %
\pgfusepath{clip}%
\pgfsetroundcap%
\pgfsetroundjoin%
\pgfsetlinewidth{0.803000pt}%
\definecolor{currentstroke}{rgb}{1.000000,1.000000,1.000000}%
\pgfsetstrokecolor{currentstroke}%
\pgfsetdash{}{0pt}%
\pgfpathmoveto{\pgfqpoint{0.556847in}{2.299750in}}%
\pgfpathlineto{\pgfqpoint{2.519580in}{2.299750in}}%
\pgfusepath{stroke}%
\end{pgfscope}%
\begin{pgfscope}%
\pgfsetbuttcap%
\pgfsetroundjoin%
\definecolor{currentfill}{rgb}{0.150000,0.150000,0.150000}%
\pgfsetfillcolor{currentfill}%
\pgfsetlinewidth{0.803000pt}%
\definecolor{currentstroke}{rgb}{0.150000,0.150000,0.150000}%
\pgfsetstrokecolor{currentstroke}%
\pgfsetdash{}{0pt}%
\pgfsys@defobject{currentmarker}{\pgfqpoint{0.000000in}{0.000000in}}{\pgfqpoint{0.000000in}{0.000000in}}{%
\pgfpathmoveto{\pgfqpoint{0.000000in}{0.000000in}}%
\pgfpathlineto{\pgfqpoint{0.000000in}{0.000000in}}%
\pgfusepath{stroke,fill}%
}%
\begin{pgfscope}%
\pgfsys@transformshift{0.556847in}{2.299750in}%
\pgfsys@useobject{currentmarker}{}%
\end{pgfscope}%
\end{pgfscope}%
\begin{pgfscope}%
\definecolor{textcolor}{rgb}{0.150000,0.150000,0.150000}%
\pgfsetstrokecolor{textcolor}%
\pgfsetfillcolor{textcolor}%
\pgftext[x=0.479069in,y=2.299750in,right,]{\color{textcolor}\sffamily\fontsize{8.000000}{9.600000}\selectfont 7.0}%
\end{pgfscope}%
\begin{pgfscope}%
\definecolor{textcolor}{rgb}{0.150000,0.150000,0.150000}%
\pgfsetstrokecolor{textcolor}%
\pgfsetfillcolor{textcolor}%
\pgftext[x=0.251677in,y=1.407986in,,bottom,rotate=90.000000]{\color{textcolor}\sffamily\fontsize{8.800000}{10.560000}\selectfont Wing length}%
\end{pgfscope}%
\begin{pgfscope}%
\pgfpathrectangle{\pgfqpoint{0.556847in}{0.516222in}}{\pgfqpoint{1.962733in}{1.783528in}} %
\pgfusepath{clip}%
\pgfsetbuttcap%
\pgfsetroundjoin%
\definecolor{currentfill}{rgb}{0.298039,0.447059,0.690196}%
\pgfsetfillcolor{currentfill}%
\pgfsetlinewidth{0.240900pt}%
\definecolor{currentstroke}{rgb}{1.000000,1.000000,1.000000}%
\pgfsetstrokecolor{currentstroke}%
\pgfsetdash{}{0pt}%
\pgfpathmoveto{\pgfqpoint{1.902721in}{1.153989in}}%
\pgfpathcurveto{\pgfqpoint{1.910957in}{1.153989in}}{\pgfqpoint{1.918857in}{1.157261in}}{\pgfqpoint{1.924681in}{1.163085in}}%
\pgfpathcurveto{\pgfqpoint{1.930505in}{1.168909in}}{\pgfqpoint{1.933778in}{1.176809in}}{\pgfqpoint{1.933778in}{1.185045in}}%
\pgfpathcurveto{\pgfqpoint{1.933778in}{1.193281in}}{\pgfqpoint{1.930505in}{1.201181in}}{\pgfqpoint{1.924681in}{1.207005in}}%
\pgfpathcurveto{\pgfqpoint{1.918857in}{1.212829in}}{\pgfqpoint{1.910957in}{1.216102in}}{\pgfqpoint{1.902721in}{1.216102in}}%
\pgfpathcurveto{\pgfqpoint{1.894485in}{1.216102in}}{\pgfqpoint{1.886585in}{1.212829in}}{\pgfqpoint{1.880761in}{1.207005in}}%
\pgfpathcurveto{\pgfqpoint{1.874937in}{1.201181in}}{\pgfqpoint{1.871665in}{1.193281in}}{\pgfqpoint{1.871665in}{1.185045in}}%
\pgfpathcurveto{\pgfqpoint{1.871665in}{1.176809in}}{\pgfqpoint{1.874937in}{1.168909in}}{\pgfqpoint{1.880761in}{1.163085in}}%
\pgfpathcurveto{\pgfqpoint{1.886585in}{1.157261in}}{\pgfqpoint{1.894485in}{1.153989in}}{\pgfqpoint{1.902721in}{1.153989in}}%
\pgfpathclose%
\pgfusepath{stroke,fill}%
\end{pgfscope}%
\begin{pgfscope}%
\pgfpathrectangle{\pgfqpoint{0.556847in}{0.516222in}}{\pgfqpoint{1.962733in}{1.783528in}} %
\pgfusepath{clip}%
\pgfsetbuttcap%
\pgfsetroundjoin%
\definecolor{currentfill}{rgb}{0.298039,0.447059,0.690196}%
\pgfsetfillcolor{currentfill}%
\pgfsetlinewidth{0.240900pt}%
\definecolor{currentstroke}{rgb}{1.000000,1.000000,1.000000}%
\pgfsetstrokecolor{currentstroke}%
\pgfsetdash{}{0pt}%
\pgfpathmoveto{\pgfqpoint{1.622331in}{0.846330in}}%
\pgfpathcurveto{\pgfqpoint{1.630567in}{0.846330in}}{\pgfqpoint{1.638467in}{0.849602in}}{\pgfqpoint{1.644291in}{0.855426in}}%
\pgfpathcurveto{\pgfqpoint{1.650115in}{0.861250in}}{\pgfqpoint{1.653387in}{0.869150in}}{\pgfqpoint{1.653387in}{0.877387in}}%
\pgfpathcurveto{\pgfqpoint{1.653387in}{0.885623in}}{\pgfqpoint{1.650115in}{0.893523in}}{\pgfqpoint{1.644291in}{0.899347in}}%
\pgfpathcurveto{\pgfqpoint{1.638467in}{0.905171in}}{\pgfqpoint{1.630567in}{0.908443in}}{\pgfqpoint{1.622331in}{0.908443in}}%
\pgfpathcurveto{\pgfqpoint{1.614094in}{0.908443in}}{\pgfqpoint{1.606194in}{0.905171in}}{\pgfqpoint{1.600370in}{0.899347in}}%
\pgfpathcurveto{\pgfqpoint{1.594546in}{0.893523in}}{\pgfqpoint{1.591274in}{0.885623in}}{\pgfqpoint{1.591274in}{0.877387in}}%
\pgfpathcurveto{\pgfqpoint{1.591274in}{0.869150in}}{\pgfqpoint{1.594546in}{0.861250in}}{\pgfqpoint{1.600370in}{0.855426in}}%
\pgfpathcurveto{\pgfqpoint{1.606194in}{0.849602in}}{\pgfqpoint{1.614094in}{0.846330in}}{\pgfqpoint{1.622331in}{0.846330in}}%
\pgfpathclose%
\pgfusepath{stroke,fill}%
\end{pgfscope}%
\begin{pgfscope}%
\pgfpathrectangle{\pgfqpoint{0.556847in}{0.516222in}}{\pgfqpoint{1.962733in}{1.783528in}} %
\pgfusepath{clip}%
\pgfsetbuttcap%
\pgfsetroundjoin%
\definecolor{currentfill}{rgb}{0.298039,0.447059,0.690196}%
\pgfsetfillcolor{currentfill}%
\pgfsetlinewidth{0.240900pt}%
\definecolor{currentstroke}{rgb}{1.000000,1.000000,1.000000}%
\pgfsetstrokecolor{currentstroke}%
\pgfsetdash{}{0pt}%
\pgfpathmoveto{\pgfqpoint{1.958799in}{1.318965in}}%
\pgfpathcurveto{\pgfqpoint{1.967035in}{1.318965in}}{\pgfqpoint{1.974935in}{1.322237in}}{\pgfqpoint{1.980759in}{1.328061in}}%
\pgfpathcurveto{\pgfqpoint{1.986583in}{1.333885in}}{\pgfqpoint{1.989856in}{1.341785in}}{\pgfqpoint{1.989856in}{1.350021in}}%
\pgfpathcurveto{\pgfqpoint{1.989856in}{1.358258in}}{\pgfqpoint{1.986583in}{1.366158in}}{\pgfqpoint{1.980759in}{1.371982in}}%
\pgfpathcurveto{\pgfqpoint{1.974935in}{1.377806in}}{\pgfqpoint{1.967035in}{1.381078in}}{\pgfqpoint{1.958799in}{1.381078in}}%
\pgfpathcurveto{\pgfqpoint{1.950563in}{1.381078in}}{\pgfqpoint{1.942663in}{1.377806in}}{\pgfqpoint{1.936839in}{1.371982in}}%
\pgfpathcurveto{\pgfqpoint{1.931015in}{1.366158in}}{\pgfqpoint{1.927743in}{1.358258in}}{\pgfqpoint{1.927743in}{1.350021in}}%
\pgfpathcurveto{\pgfqpoint{1.927743in}{1.341785in}}{\pgfqpoint{1.931015in}{1.333885in}}{\pgfqpoint{1.936839in}{1.328061in}}%
\pgfpathcurveto{\pgfqpoint{1.942663in}{1.322237in}}{\pgfqpoint{1.950563in}{1.318965in}}{\pgfqpoint{1.958799in}{1.318965in}}%
\pgfpathclose%
\pgfusepath{stroke,fill}%
\end{pgfscope}%
\begin{pgfscope}%
\pgfpathrectangle{\pgfqpoint{0.556847in}{0.516222in}}{\pgfqpoint{1.962733in}{1.783528in}} %
\pgfusepath{clip}%
\pgfsetbuttcap%
\pgfsetroundjoin%
\definecolor{currentfill}{rgb}{0.298039,0.447059,0.690196}%
\pgfsetfillcolor{currentfill}%
\pgfsetlinewidth{0.240900pt}%
\definecolor{currentstroke}{rgb}{1.000000,1.000000,1.000000}%
\pgfsetstrokecolor{currentstroke}%
\pgfsetdash{}{0pt}%
\pgfpathmoveto{\pgfqpoint{2.239189in}{1.666753in}}%
\pgfpathcurveto{\pgfqpoint{2.247426in}{1.666753in}}{\pgfqpoint{2.255326in}{1.670025in}}{\pgfqpoint{2.261150in}{1.675849in}}%
\pgfpathcurveto{\pgfqpoint{2.266974in}{1.681673in}}{\pgfqpoint{2.270246in}{1.689573in}}{\pgfqpoint{2.270246in}{1.697809in}}%
\pgfpathcurveto{\pgfqpoint{2.270246in}{1.706046in}}{\pgfqpoint{2.266974in}{1.713946in}}{\pgfqpoint{2.261150in}{1.719770in}}%
\pgfpathcurveto{\pgfqpoint{2.255326in}{1.725594in}}{\pgfqpoint{2.247426in}{1.728866in}}{\pgfqpoint{2.239189in}{1.728866in}}%
\pgfpathcurveto{\pgfqpoint{2.230953in}{1.728866in}}{\pgfqpoint{2.223053in}{1.725594in}}{\pgfqpoint{2.217229in}{1.719770in}}%
\pgfpathcurveto{\pgfqpoint{2.211405in}{1.713946in}}{\pgfqpoint{2.208133in}{1.706046in}}{\pgfqpoint{2.208133in}{1.697809in}}%
\pgfpathcurveto{\pgfqpoint{2.208133in}{1.689573in}}{\pgfqpoint{2.211405in}{1.681673in}}{\pgfqpoint{2.217229in}{1.675849in}}%
\pgfpathcurveto{\pgfqpoint{2.223053in}{1.670025in}}{\pgfqpoint{2.230953in}{1.666753in}}{\pgfqpoint{2.239189in}{1.666753in}}%
\pgfpathclose%
\pgfusepath{stroke,fill}%
\end{pgfscope}%
\begin{pgfscope}%
\pgfpathrectangle{\pgfqpoint{0.556847in}{0.516222in}}{\pgfqpoint{1.962733in}{1.783528in}} %
\pgfusepath{clip}%
\pgfsetbuttcap%
\pgfsetroundjoin%
\definecolor{currentfill}{rgb}{0.298039,0.447059,0.690196}%
\pgfsetfillcolor{currentfill}%
\pgfsetlinewidth{0.240900pt}%
\definecolor{currentstroke}{rgb}{1.000000,1.000000,1.000000}%
\pgfsetstrokecolor{currentstroke}%
\pgfsetdash{}{0pt}%
\pgfpathmoveto{\pgfqpoint{1.622331in}{0.980095in}}%
\pgfpathcurveto{\pgfqpoint{1.630567in}{0.980095in}}{\pgfqpoint{1.638467in}{0.983367in}}{\pgfqpoint{1.644291in}{0.989191in}}%
\pgfpathcurveto{\pgfqpoint{1.650115in}{0.995015in}}{\pgfqpoint{1.653387in}{1.002915in}}{\pgfqpoint{1.653387in}{1.011151in}}%
\pgfpathcurveto{\pgfqpoint{1.653387in}{1.019387in}}{\pgfqpoint{1.650115in}{1.027288in}}{\pgfqpoint{1.644291in}{1.033111in}}%
\pgfpathcurveto{\pgfqpoint{1.638467in}{1.038935in}}{\pgfqpoint{1.630567in}{1.042208in}}{\pgfqpoint{1.622331in}{1.042208in}}%
\pgfpathcurveto{\pgfqpoint{1.614094in}{1.042208in}}{\pgfqpoint{1.606194in}{1.038935in}}{\pgfqpoint{1.600370in}{1.033111in}}%
\pgfpathcurveto{\pgfqpoint{1.594546in}{1.027288in}}{\pgfqpoint{1.591274in}{1.019387in}}{\pgfqpoint{1.591274in}{1.011151in}}%
\pgfpathcurveto{\pgfqpoint{1.591274in}{1.002915in}}{\pgfqpoint{1.594546in}{0.995015in}}{\pgfqpoint{1.600370in}{0.989191in}}%
\pgfpathcurveto{\pgfqpoint{1.606194in}{0.983367in}}{\pgfqpoint{1.614094in}{0.980095in}}{\pgfqpoint{1.622331in}{0.980095in}}%
\pgfpathclose%
\pgfusepath{stroke,fill}%
\end{pgfscope}%
\begin{pgfscope}%
\pgfpathrectangle{\pgfqpoint{0.556847in}{0.516222in}}{\pgfqpoint{1.962733in}{1.783528in}} %
\pgfusepath{clip}%
\pgfsetbuttcap%
\pgfsetroundjoin%
\definecolor{currentfill}{rgb}{0.298039,0.447059,0.690196}%
\pgfsetfillcolor{currentfill}%
\pgfsetlinewidth{0.240900pt}%
\definecolor{currentstroke}{rgb}{1.000000,1.000000,1.000000}%
\pgfsetstrokecolor{currentstroke}%
\pgfsetdash{}{0pt}%
\pgfpathmoveto{\pgfqpoint{2.407424in}{2.036835in}}%
\pgfpathcurveto{\pgfqpoint{2.415660in}{2.036835in}}{\pgfqpoint{2.423560in}{2.040107in}}{\pgfqpoint{2.429384in}{2.045931in}}%
\pgfpathcurveto{\pgfqpoint{2.435208in}{2.051755in}}{\pgfqpoint{2.438480in}{2.059655in}}{\pgfqpoint{2.438480in}{2.067891in}}%
\pgfpathcurveto{\pgfqpoint{2.438480in}{2.076128in}}{\pgfqpoint{2.435208in}{2.084028in}}{\pgfqpoint{2.429384in}{2.089852in}}%
\pgfpathcurveto{\pgfqpoint{2.423560in}{2.095676in}}{\pgfqpoint{2.415660in}{2.098948in}}{\pgfqpoint{2.407424in}{2.098948in}}%
\pgfpathcurveto{\pgfqpoint{2.399187in}{2.098948in}}{\pgfqpoint{2.391287in}{2.095676in}}{\pgfqpoint{2.385463in}{2.089852in}}%
\pgfpathcurveto{\pgfqpoint{2.379640in}{2.084028in}}{\pgfqpoint{2.376367in}{2.076128in}}{\pgfqpoint{2.376367in}{2.067891in}}%
\pgfpathcurveto{\pgfqpoint{2.376367in}{2.059655in}}{\pgfqpoint{2.379640in}{2.051755in}}{\pgfqpoint{2.385463in}{2.045931in}}%
\pgfpathcurveto{\pgfqpoint{2.391287in}{2.040107in}}{\pgfqpoint{2.399187in}{2.036835in}}{\pgfqpoint{2.407424in}{2.036835in}}%
\pgfpathclose%
\pgfusepath{stroke,fill}%
\end{pgfscope}%
\begin{pgfscope}%
\pgfpathrectangle{\pgfqpoint{0.556847in}{0.516222in}}{\pgfqpoint{1.962733in}{1.783528in}} %
\pgfusepath{clip}%
\pgfsetbuttcap%
\pgfsetroundjoin%
\definecolor{currentfill}{rgb}{0.298039,0.447059,0.690196}%
\pgfsetfillcolor{currentfill}%
\pgfsetlinewidth{0.240900pt}%
\definecolor{currentstroke}{rgb}{1.000000,1.000000,1.000000}%
\pgfsetstrokecolor{currentstroke}%
\pgfsetdash{}{0pt}%
\pgfpathmoveto{\pgfqpoint{1.622331in}{0.739318in}}%
\pgfpathcurveto{\pgfqpoint{1.630567in}{0.739318in}}{\pgfqpoint{1.638467in}{0.742591in}}{\pgfqpoint{1.644291in}{0.748415in}}%
\pgfpathcurveto{\pgfqpoint{1.650115in}{0.754239in}}{\pgfqpoint{1.653387in}{0.762139in}}{\pgfqpoint{1.653387in}{0.770375in}}%
\pgfpathcurveto{\pgfqpoint{1.653387in}{0.778611in}}{\pgfqpoint{1.650115in}{0.786511in}}{\pgfqpoint{1.644291in}{0.792335in}}%
\pgfpathcurveto{\pgfqpoint{1.638467in}{0.798159in}}{\pgfqpoint{1.630567in}{0.801431in}}{\pgfqpoint{1.622331in}{0.801431in}}%
\pgfpathcurveto{\pgfqpoint{1.614094in}{0.801431in}}{\pgfqpoint{1.606194in}{0.798159in}}{\pgfqpoint{1.600370in}{0.792335in}}%
\pgfpathcurveto{\pgfqpoint{1.594546in}{0.786511in}}{\pgfqpoint{1.591274in}{0.778611in}}{\pgfqpoint{1.591274in}{0.770375in}}%
\pgfpathcurveto{\pgfqpoint{1.591274in}{0.762139in}}{\pgfqpoint{1.594546in}{0.754239in}}{\pgfqpoint{1.600370in}{0.748415in}}%
\pgfpathcurveto{\pgfqpoint{1.606194in}{0.742591in}}{\pgfqpoint{1.614094in}{0.739318in}}{\pgfqpoint{1.622331in}{0.739318in}}%
\pgfpathclose%
\pgfusepath{stroke,fill}%
\end{pgfscope}%
\begin{pgfscope}%
\pgfpathrectangle{\pgfqpoint{0.556847in}{0.516222in}}{\pgfqpoint{1.962733in}{1.783528in}} %
\pgfusepath{clip}%
\pgfsetbuttcap%
\pgfsetroundjoin%
\definecolor{currentfill}{rgb}{0.298039,0.447059,0.690196}%
\pgfsetfillcolor{currentfill}%
\pgfsetlinewidth{0.240900pt}%
\definecolor{currentstroke}{rgb}{1.000000,1.000000,1.000000}%
\pgfsetstrokecolor{currentstroke}%
\pgfsetdash{}{0pt}%
\pgfpathmoveto{\pgfqpoint{2.407424in}{1.760388in}}%
\pgfpathcurveto{\pgfqpoint{2.415660in}{1.760388in}}{\pgfqpoint{2.423560in}{1.763660in}}{\pgfqpoint{2.429384in}{1.769484in}}%
\pgfpathcurveto{\pgfqpoint{2.435208in}{1.775308in}}{\pgfqpoint{2.438480in}{1.783208in}}{\pgfqpoint{2.438480in}{1.791445in}}%
\pgfpathcurveto{\pgfqpoint{2.438480in}{1.799681in}}{\pgfqpoint{2.435208in}{1.807581in}}{\pgfqpoint{2.429384in}{1.813405in}}%
\pgfpathcurveto{\pgfqpoint{2.423560in}{1.819229in}}{\pgfqpoint{2.415660in}{1.822501in}}{\pgfqpoint{2.407424in}{1.822501in}}%
\pgfpathcurveto{\pgfqpoint{2.399187in}{1.822501in}}{\pgfqpoint{2.391287in}{1.819229in}}{\pgfqpoint{2.385463in}{1.813405in}}%
\pgfpathcurveto{\pgfqpoint{2.379640in}{1.807581in}}{\pgfqpoint{2.376367in}{1.799681in}}{\pgfqpoint{2.376367in}{1.791445in}}%
\pgfpathcurveto{\pgfqpoint{2.376367in}{1.783208in}}{\pgfqpoint{2.379640in}{1.775308in}}{\pgfqpoint{2.385463in}{1.769484in}}%
\pgfpathcurveto{\pgfqpoint{2.391287in}{1.763660in}}{\pgfqpoint{2.399187in}{1.760388in}}{\pgfqpoint{2.407424in}{1.760388in}}%
\pgfpathclose%
\pgfusepath{stroke,fill}%
\end{pgfscope}%
\begin{pgfscope}%
\pgfpathrectangle{\pgfqpoint{0.556847in}{0.516222in}}{\pgfqpoint{1.962733in}{1.783528in}} %
\pgfusepath{clip}%
\pgfsetbuttcap%
\pgfsetroundjoin%
\definecolor{currentfill}{rgb}{0.298039,0.447059,0.690196}%
\pgfsetfillcolor{currentfill}%
\pgfsetlinewidth{0.240900pt}%
\definecolor{currentstroke}{rgb}{1.000000,1.000000,1.000000}%
\pgfsetstrokecolor{currentstroke}%
\pgfsetdash{}{0pt}%
\pgfpathmoveto{\pgfqpoint{1.173706in}{0.837412in}}%
\pgfpathcurveto{\pgfqpoint{1.181942in}{0.837412in}}{\pgfqpoint{1.189842in}{0.840685in}}{\pgfqpoint{1.195666in}{0.846509in}}%
\pgfpathcurveto{\pgfqpoint{1.201490in}{0.852333in}}{\pgfqpoint{1.204763in}{0.860233in}}{\pgfqpoint{1.204763in}{0.868469in}}%
\pgfpathcurveto{\pgfqpoint{1.204763in}{0.876705in}}{\pgfqpoint{1.201490in}{0.884605in}}{\pgfqpoint{1.195666in}{0.890429in}}%
\pgfpathcurveto{\pgfqpoint{1.189842in}{0.896253in}}{\pgfqpoint{1.181942in}{0.899525in}}{\pgfqpoint{1.173706in}{0.899525in}}%
\pgfpathcurveto{\pgfqpoint{1.165470in}{0.899525in}}{\pgfqpoint{1.157570in}{0.896253in}}{\pgfqpoint{1.151746in}{0.890429in}}%
\pgfpathcurveto{\pgfqpoint{1.145922in}{0.884605in}}{\pgfqpoint{1.142650in}{0.876705in}}{\pgfqpoint{1.142650in}{0.868469in}}%
\pgfpathcurveto{\pgfqpoint{1.142650in}{0.860233in}}{\pgfqpoint{1.145922in}{0.852333in}}{\pgfqpoint{1.151746in}{0.846509in}}%
\pgfpathcurveto{\pgfqpoint{1.157570in}{0.840685in}}{\pgfqpoint{1.165470in}{0.837412in}}{\pgfqpoint{1.173706in}{0.837412in}}%
\pgfpathclose%
\pgfusepath{stroke,fill}%
\end{pgfscope}%
\begin{pgfscope}%
\pgfpathrectangle{\pgfqpoint{0.556847in}{0.516222in}}{\pgfqpoint{1.962733in}{1.783528in}} %
\pgfusepath{clip}%
\pgfsetbuttcap%
\pgfsetroundjoin%
\definecolor{currentfill}{rgb}{0.298039,0.447059,0.690196}%
\pgfsetfillcolor{currentfill}%
\pgfsetlinewidth{0.240900pt}%
\definecolor{currentstroke}{rgb}{1.000000,1.000000,1.000000}%
\pgfsetstrokecolor{currentstroke}%
\pgfsetdash{}{0pt}%
\pgfpathmoveto{\pgfqpoint{1.678409in}{1.394765in}}%
\pgfpathcurveto{\pgfqpoint{1.686645in}{1.394765in}}{\pgfqpoint{1.694545in}{1.398037in}}{\pgfqpoint{1.700369in}{1.403861in}}%
\pgfpathcurveto{\pgfqpoint{1.706193in}{1.409685in}}{\pgfqpoint{1.709465in}{1.417585in}}{\pgfqpoint{1.709465in}{1.425821in}}%
\pgfpathcurveto{\pgfqpoint{1.709465in}{1.434058in}}{\pgfqpoint{1.706193in}{1.441958in}}{\pgfqpoint{1.700369in}{1.447782in}}%
\pgfpathcurveto{\pgfqpoint{1.694545in}{1.453606in}}{\pgfqpoint{1.686645in}{1.456878in}}{\pgfqpoint{1.678409in}{1.456878in}}%
\pgfpathcurveto{\pgfqpoint{1.670172in}{1.456878in}}{\pgfqpoint{1.662272in}{1.453606in}}{\pgfqpoint{1.656448in}{1.447782in}}%
\pgfpathcurveto{\pgfqpoint{1.650625in}{1.441958in}}{\pgfqpoint{1.647352in}{1.434058in}}{\pgfqpoint{1.647352in}{1.425821in}}%
\pgfpathcurveto{\pgfqpoint{1.647352in}{1.417585in}}{\pgfqpoint{1.650625in}{1.409685in}}{\pgfqpoint{1.656448in}{1.403861in}}%
\pgfpathcurveto{\pgfqpoint{1.662272in}{1.398037in}}{\pgfqpoint{1.670172in}{1.394765in}}{\pgfqpoint{1.678409in}{1.394765in}}%
\pgfpathclose%
\pgfusepath{stroke,fill}%
\end{pgfscope}%
\begin{pgfscope}%
\pgfpathrectangle{\pgfqpoint{0.556847in}{0.516222in}}{\pgfqpoint{1.962733in}{1.783528in}} %
\pgfusepath{clip}%
\pgfsetbuttcap%
\pgfsetroundjoin%
\definecolor{currentfill}{rgb}{0.298039,0.447059,0.690196}%
\pgfsetfillcolor{currentfill}%
\pgfsetlinewidth{0.240900pt}%
\definecolor{currentstroke}{rgb}{1.000000,1.000000,1.000000}%
\pgfsetstrokecolor{currentstroke}%
\pgfsetdash{}{0pt}%
\pgfpathmoveto{\pgfqpoint{1.510175in}{0.806201in}}%
\pgfpathcurveto{\pgfqpoint{1.518411in}{0.806201in}}{\pgfqpoint{1.526311in}{0.809473in}}{\pgfqpoint{1.532135in}{0.815297in}}%
\pgfpathcurveto{\pgfqpoint{1.537959in}{0.821121in}}{\pgfqpoint{1.541231in}{0.829021in}}{\pgfqpoint{1.541231in}{0.837257in}}%
\pgfpathcurveto{\pgfqpoint{1.541231in}{0.845494in}}{\pgfqpoint{1.537959in}{0.853394in}}{\pgfqpoint{1.532135in}{0.859217in}}%
\pgfpathcurveto{\pgfqpoint{1.526311in}{0.865041in}}{\pgfqpoint{1.518411in}{0.868314in}}{\pgfqpoint{1.510175in}{0.868314in}}%
\pgfpathcurveto{\pgfqpoint{1.501938in}{0.868314in}}{\pgfqpoint{1.494038in}{0.865041in}}{\pgfqpoint{1.488214in}{0.859217in}}%
\pgfpathcurveto{\pgfqpoint{1.482390in}{0.853394in}}{\pgfqpoint{1.479118in}{0.845494in}}{\pgfqpoint{1.479118in}{0.837257in}}%
\pgfpathcurveto{\pgfqpoint{1.479118in}{0.829021in}}{\pgfqpoint{1.482390in}{0.821121in}}{\pgfqpoint{1.488214in}{0.815297in}}%
\pgfpathcurveto{\pgfqpoint{1.494038in}{0.809473in}}{\pgfqpoint{1.501938in}{0.806201in}}{\pgfqpoint{1.510175in}{0.806201in}}%
\pgfpathclose%
\pgfusepath{stroke,fill}%
\end{pgfscope}%
\begin{pgfscope}%
\pgfpathrectangle{\pgfqpoint{0.556847in}{0.516222in}}{\pgfqpoint{1.962733in}{1.783528in}} %
\pgfusepath{clip}%
\pgfsetbuttcap%
\pgfsetroundjoin%
\definecolor{currentfill}{rgb}{0.298039,0.447059,0.690196}%
\pgfsetfillcolor{currentfill}%
\pgfsetlinewidth{0.240900pt}%
\definecolor{currentstroke}{rgb}{1.000000,1.000000,1.000000}%
\pgfsetstrokecolor{currentstroke}%
\pgfsetdash{}{0pt}%
\pgfpathmoveto{\pgfqpoint{1.678409in}{0.792824in}}%
\pgfpathcurveto{\pgfqpoint{1.686645in}{0.792824in}}{\pgfqpoint{1.694545in}{0.796097in}}{\pgfqpoint{1.700369in}{0.801921in}}%
\pgfpathcurveto{\pgfqpoint{1.706193in}{0.807744in}}{\pgfqpoint{1.709465in}{0.815644in}}{\pgfqpoint{1.709465in}{0.823881in}}%
\pgfpathcurveto{\pgfqpoint{1.709465in}{0.832117in}}{\pgfqpoint{1.706193in}{0.840017in}}{\pgfqpoint{1.700369in}{0.845841in}}%
\pgfpathcurveto{\pgfqpoint{1.694545in}{0.851665in}}{\pgfqpoint{1.686645in}{0.854937in}}{\pgfqpoint{1.678409in}{0.854937in}}%
\pgfpathcurveto{\pgfqpoint{1.670172in}{0.854937in}}{\pgfqpoint{1.662272in}{0.851665in}}{\pgfqpoint{1.656448in}{0.845841in}}%
\pgfpathcurveto{\pgfqpoint{1.650625in}{0.840017in}}{\pgfqpoint{1.647352in}{0.832117in}}{\pgfqpoint{1.647352in}{0.823881in}}%
\pgfpathcurveto{\pgfqpoint{1.647352in}{0.815644in}}{\pgfqpoint{1.650625in}{0.807744in}}{\pgfqpoint{1.656448in}{0.801921in}}%
\pgfpathcurveto{\pgfqpoint{1.662272in}{0.796097in}}{\pgfqpoint{1.670172in}{0.792824in}}{\pgfqpoint{1.678409in}{0.792824in}}%
\pgfpathclose%
\pgfusepath{stroke,fill}%
\end{pgfscope}%
\begin{pgfscope}%
\pgfpathrectangle{\pgfqpoint{0.556847in}{0.516222in}}{\pgfqpoint{1.962733in}{1.783528in}} %
\pgfusepath{clip}%
\pgfsetbuttcap%
\pgfsetroundjoin%
\definecolor{currentfill}{rgb}{0.298039,0.447059,0.690196}%
\pgfsetfillcolor{currentfill}%
\pgfsetlinewidth{0.240900pt}%
\definecolor{currentstroke}{rgb}{1.000000,1.000000,1.000000}%
\pgfsetstrokecolor{currentstroke}%
\pgfsetdash{}{0pt}%
\pgfpathmoveto{\pgfqpoint{1.790565in}{2.094800in}}%
\pgfpathcurveto{\pgfqpoint{1.798801in}{2.094800in}}{\pgfqpoint{1.806701in}{2.098072in}}{\pgfqpoint{1.812525in}{2.103896in}}%
\pgfpathcurveto{\pgfqpoint{1.818349in}{2.109720in}}{\pgfqpoint{1.821621in}{2.117620in}}{\pgfqpoint{1.821621in}{2.125856in}}%
\pgfpathcurveto{\pgfqpoint{1.821621in}{2.134092in}}{\pgfqpoint{1.818349in}{2.141992in}}{\pgfqpoint{1.812525in}{2.147816in}}%
\pgfpathcurveto{\pgfqpoint{1.806701in}{2.153640in}}{\pgfqpoint{1.798801in}{2.156913in}}{\pgfqpoint{1.790565in}{2.156913in}}%
\pgfpathcurveto{\pgfqpoint{1.782329in}{2.156913in}}{\pgfqpoint{1.774429in}{2.153640in}}{\pgfqpoint{1.768605in}{2.147816in}}%
\pgfpathcurveto{\pgfqpoint{1.762781in}{2.141992in}}{\pgfqpoint{1.759508in}{2.134092in}}{\pgfqpoint{1.759508in}{2.125856in}}%
\pgfpathcurveto{\pgfqpoint{1.759508in}{2.117620in}}{\pgfqpoint{1.762781in}{2.109720in}}{\pgfqpoint{1.768605in}{2.103896in}}%
\pgfpathcurveto{\pgfqpoint{1.774429in}{2.098072in}}{\pgfqpoint{1.782329in}{2.094800in}}{\pgfqpoint{1.790565in}{2.094800in}}%
\pgfpathclose%
\pgfusepath{stroke,fill}%
\end{pgfscope}%
\begin{pgfscope}%
\pgfpathrectangle{\pgfqpoint{0.556847in}{0.516222in}}{\pgfqpoint{1.962733in}{1.783528in}} %
\pgfusepath{clip}%
\pgfsetbuttcap%
\pgfsetroundjoin%
\definecolor{currentfill}{rgb}{0.298039,0.447059,0.690196}%
\pgfsetfillcolor{currentfill}%
\pgfsetlinewidth{0.240900pt}%
\definecolor{currentstroke}{rgb}{1.000000,1.000000,1.000000}%
\pgfsetstrokecolor{currentstroke}%
\pgfsetdash{}{0pt}%
\pgfpathmoveto{\pgfqpoint{1.061550in}{1.033601in}}%
\pgfpathcurveto{\pgfqpoint{1.069786in}{1.033601in}}{\pgfqpoint{1.077686in}{1.036873in}}{\pgfqpoint{1.083510in}{1.042697in}}%
\pgfpathcurveto{\pgfqpoint{1.089334in}{1.048521in}}{\pgfqpoint{1.092606in}{1.056421in}}{\pgfqpoint{1.092606in}{1.064657in}}%
\pgfpathcurveto{\pgfqpoint{1.092606in}{1.072893in}}{\pgfqpoint{1.089334in}{1.080793in}}{\pgfqpoint{1.083510in}{1.086617in}}%
\pgfpathcurveto{\pgfqpoint{1.077686in}{1.092441in}}{\pgfqpoint{1.069786in}{1.095714in}}{\pgfqpoint{1.061550in}{1.095714in}}%
\pgfpathcurveto{\pgfqpoint{1.053314in}{1.095714in}}{\pgfqpoint{1.045414in}{1.092441in}}{\pgfqpoint{1.039590in}{1.086617in}}%
\pgfpathcurveto{\pgfqpoint{1.033766in}{1.080793in}}{\pgfqpoint{1.030493in}{1.072893in}}{\pgfqpoint{1.030493in}{1.064657in}}%
\pgfpathcurveto{\pgfqpoint{1.030493in}{1.056421in}}{\pgfqpoint{1.033766in}{1.048521in}}{\pgfqpoint{1.039590in}{1.042697in}}%
\pgfpathcurveto{\pgfqpoint{1.045414in}{1.036873in}}{\pgfqpoint{1.053314in}{1.033601in}}{\pgfqpoint{1.061550in}{1.033601in}}%
\pgfpathclose%
\pgfusepath{stroke,fill}%
\end{pgfscope}%
\begin{pgfscope}%
\pgfpathrectangle{\pgfqpoint{0.556847in}{0.516222in}}{\pgfqpoint{1.962733in}{1.783528in}} %
\pgfusepath{clip}%
\pgfsetbuttcap%
\pgfsetroundjoin%
\definecolor{currentfill}{rgb}{0.298039,0.447059,0.690196}%
\pgfsetfillcolor{currentfill}%
\pgfsetlinewidth{0.240900pt}%
\definecolor{currentstroke}{rgb}{1.000000,1.000000,1.000000}%
\pgfsetstrokecolor{currentstroke}%
\pgfsetdash{}{0pt}%
\pgfpathmoveto{\pgfqpoint{2.070955in}{1.457188in}}%
\pgfpathcurveto{\pgfqpoint{2.079192in}{1.457188in}}{\pgfqpoint{2.087092in}{1.460461in}}{\pgfqpoint{2.092916in}{1.466285in}}%
\pgfpathcurveto{\pgfqpoint{2.098739in}{1.472109in}}{\pgfqpoint{2.102012in}{1.480009in}}{\pgfqpoint{2.102012in}{1.488245in}}%
\pgfpathcurveto{\pgfqpoint{2.102012in}{1.496481in}}{\pgfqpoint{2.098739in}{1.504381in}}{\pgfqpoint{2.092916in}{1.510205in}}%
\pgfpathcurveto{\pgfqpoint{2.087092in}{1.516029in}}{\pgfqpoint{2.079192in}{1.519301in}}{\pgfqpoint{2.070955in}{1.519301in}}%
\pgfpathcurveto{\pgfqpoint{2.062719in}{1.519301in}}{\pgfqpoint{2.054819in}{1.516029in}}{\pgfqpoint{2.048995in}{1.510205in}}%
\pgfpathcurveto{\pgfqpoint{2.043171in}{1.504381in}}{\pgfqpoint{2.039899in}{1.496481in}}{\pgfqpoint{2.039899in}{1.488245in}}%
\pgfpathcurveto{\pgfqpoint{2.039899in}{1.480009in}}{\pgfqpoint{2.043171in}{1.472109in}}{\pgfqpoint{2.048995in}{1.466285in}}%
\pgfpathcurveto{\pgfqpoint{2.054819in}{1.460461in}}{\pgfqpoint{2.062719in}{1.457188in}}{\pgfqpoint{2.070955in}{1.457188in}}%
\pgfpathclose%
\pgfusepath{stroke,fill}%
\end{pgfscope}%
\begin{pgfscope}%
\pgfpathrectangle{\pgfqpoint{0.556847in}{0.516222in}}{\pgfqpoint{1.962733in}{1.783528in}} %
\pgfusepath{clip}%
\pgfsetbuttcap%
\pgfsetroundjoin%
\definecolor{currentfill}{rgb}{0.298039,0.447059,0.690196}%
\pgfsetfillcolor{currentfill}%
\pgfsetlinewidth{0.240900pt}%
\definecolor{currentstroke}{rgb}{1.000000,1.000000,1.000000}%
\pgfsetstrokecolor{currentstroke}%
\pgfsetdash{}{0pt}%
\pgfpathmoveto{\pgfqpoint{1.958799in}{1.372471in}}%
\pgfpathcurveto{\pgfqpoint{1.967035in}{1.372471in}}{\pgfqpoint{1.974935in}{1.375743in}}{\pgfqpoint{1.980759in}{1.381567in}}%
\pgfpathcurveto{\pgfqpoint{1.986583in}{1.387391in}}{\pgfqpoint{1.989856in}{1.395291in}}{\pgfqpoint{1.989856in}{1.403527in}}%
\pgfpathcurveto{\pgfqpoint{1.989856in}{1.411764in}}{\pgfqpoint{1.986583in}{1.419664in}}{\pgfqpoint{1.980759in}{1.425488in}}%
\pgfpathcurveto{\pgfqpoint{1.974935in}{1.431311in}}{\pgfqpoint{1.967035in}{1.434584in}}{\pgfqpoint{1.958799in}{1.434584in}}%
\pgfpathcurveto{\pgfqpoint{1.950563in}{1.434584in}}{\pgfqpoint{1.942663in}{1.431311in}}{\pgfqpoint{1.936839in}{1.425488in}}%
\pgfpathcurveto{\pgfqpoint{1.931015in}{1.419664in}}{\pgfqpoint{1.927743in}{1.411764in}}{\pgfqpoint{1.927743in}{1.403527in}}%
\pgfpathcurveto{\pgfqpoint{1.927743in}{1.395291in}}{\pgfqpoint{1.931015in}{1.387391in}}{\pgfqpoint{1.936839in}{1.381567in}}%
\pgfpathcurveto{\pgfqpoint{1.942663in}{1.375743in}}{\pgfqpoint{1.950563in}{1.372471in}}{\pgfqpoint{1.958799in}{1.372471in}}%
\pgfpathclose%
\pgfusepath{stroke,fill}%
\end{pgfscope}%
\begin{pgfscope}%
\pgfpathrectangle{\pgfqpoint{0.556847in}{0.516222in}}{\pgfqpoint{1.962733in}{1.783528in}} %
\pgfusepath{clip}%
\pgfsetbuttcap%
\pgfsetroundjoin%
\definecolor{currentfill}{rgb}{0.298039,0.447059,0.690196}%
\pgfsetfillcolor{currentfill}%
\pgfsetlinewidth{0.240900pt}%
\definecolor{currentstroke}{rgb}{1.000000,1.000000,1.000000}%
\pgfsetstrokecolor{currentstroke}%
\pgfsetdash{}{0pt}%
\pgfpathmoveto{\pgfqpoint{1.510175in}{0.672436in}}%
\pgfpathcurveto{\pgfqpoint{1.518411in}{0.672436in}}{\pgfqpoint{1.526311in}{0.675708in}}{\pgfqpoint{1.532135in}{0.681532in}}%
\pgfpathcurveto{\pgfqpoint{1.537959in}{0.687356in}}{\pgfqpoint{1.541231in}{0.695256in}}{\pgfqpoint{1.541231in}{0.703493in}}%
\pgfpathcurveto{\pgfqpoint{1.541231in}{0.711729in}}{\pgfqpoint{1.537959in}{0.719629in}}{\pgfqpoint{1.532135in}{0.725453in}}%
\pgfpathcurveto{\pgfqpoint{1.526311in}{0.731277in}}{\pgfqpoint{1.518411in}{0.734549in}}{\pgfqpoint{1.510175in}{0.734549in}}%
\pgfpathcurveto{\pgfqpoint{1.501938in}{0.734549in}}{\pgfqpoint{1.494038in}{0.731277in}}{\pgfqpoint{1.488214in}{0.725453in}}%
\pgfpathcurveto{\pgfqpoint{1.482390in}{0.719629in}}{\pgfqpoint{1.479118in}{0.711729in}}{\pgfqpoint{1.479118in}{0.703493in}}%
\pgfpathcurveto{\pgfqpoint{1.479118in}{0.695256in}}{\pgfqpoint{1.482390in}{0.687356in}}{\pgfqpoint{1.488214in}{0.681532in}}%
\pgfpathcurveto{\pgfqpoint{1.494038in}{0.675708in}}{\pgfqpoint{1.501938in}{0.672436in}}{\pgfqpoint{1.510175in}{0.672436in}}%
\pgfpathclose%
\pgfusepath{stroke,fill}%
\end{pgfscope}%
\begin{pgfscope}%
\pgfpathrectangle{\pgfqpoint{0.556847in}{0.516222in}}{\pgfqpoint{1.962733in}{1.783528in}} %
\pgfusepath{clip}%
\pgfsetbuttcap%
\pgfsetroundjoin%
\definecolor{currentfill}{rgb}{0.298039,0.447059,0.690196}%
\pgfsetfillcolor{currentfill}%
\pgfsetlinewidth{0.240900pt}%
\definecolor{currentstroke}{rgb}{1.000000,1.000000,1.000000}%
\pgfsetstrokecolor{currentstroke}%
\pgfsetdash{}{0pt}%
\pgfpathmoveto{\pgfqpoint{1.678409in}{0.895377in}}%
\pgfpathcurveto{\pgfqpoint{1.686645in}{0.895377in}}{\pgfqpoint{1.694545in}{0.898649in}}{\pgfqpoint{1.700369in}{0.904473in}}%
\pgfpathcurveto{\pgfqpoint{1.706193in}{0.910297in}}{\pgfqpoint{1.709465in}{0.918197in}}{\pgfqpoint{1.709465in}{0.926434in}}%
\pgfpathcurveto{\pgfqpoint{1.709465in}{0.934670in}}{\pgfqpoint{1.706193in}{0.942570in}}{\pgfqpoint{1.700369in}{0.948394in}}%
\pgfpathcurveto{\pgfqpoint{1.694545in}{0.954218in}}{\pgfqpoint{1.686645in}{0.957490in}}{\pgfqpoint{1.678409in}{0.957490in}}%
\pgfpathcurveto{\pgfqpoint{1.670172in}{0.957490in}}{\pgfqpoint{1.662272in}{0.954218in}}{\pgfqpoint{1.656448in}{0.948394in}}%
\pgfpathcurveto{\pgfqpoint{1.650625in}{0.942570in}}{\pgfqpoint{1.647352in}{0.934670in}}{\pgfqpoint{1.647352in}{0.926434in}}%
\pgfpathcurveto{\pgfqpoint{1.647352in}{0.918197in}}{\pgfqpoint{1.650625in}{0.910297in}}{\pgfqpoint{1.656448in}{0.904473in}}%
\pgfpathcurveto{\pgfqpoint{1.662272in}{0.898649in}}{\pgfqpoint{1.670172in}{0.895377in}}{\pgfqpoint{1.678409in}{0.895377in}}%
\pgfpathclose%
\pgfusepath{stroke,fill}%
\end{pgfscope}%
\begin{pgfscope}%
\pgfpathrectangle{\pgfqpoint{0.556847in}{0.516222in}}{\pgfqpoint{1.962733in}{1.783528in}} %
\pgfusepath{clip}%
\pgfsetbuttcap%
\pgfsetroundjoin%
\definecolor{currentfill}{rgb}{0.298039,0.447059,0.690196}%
\pgfsetfillcolor{currentfill}%
\pgfsetlinewidth{0.240900pt}%
\definecolor{currentstroke}{rgb}{1.000000,1.000000,1.000000}%
\pgfsetstrokecolor{currentstroke}%
\pgfsetdash{}{0pt}%
\pgfpathmoveto{\pgfqpoint{1.285862in}{0.975636in}}%
\pgfpathcurveto{\pgfqpoint{1.294098in}{0.975636in}}{\pgfqpoint{1.301999in}{0.978908in}}{\pgfqpoint{1.307822in}{0.984732in}}%
\pgfpathcurveto{\pgfqpoint{1.313646in}{0.990556in}}{\pgfqpoint{1.316919in}{0.998456in}}{\pgfqpoint{1.316919in}{1.006692in}}%
\pgfpathcurveto{\pgfqpoint{1.316919in}{1.014929in}}{\pgfqpoint{1.313646in}{1.022829in}}{\pgfqpoint{1.307822in}{1.028653in}}%
\pgfpathcurveto{\pgfqpoint{1.301999in}{1.034477in}}{\pgfqpoint{1.294098in}{1.037749in}}{\pgfqpoint{1.285862in}{1.037749in}}%
\pgfpathcurveto{\pgfqpoint{1.277626in}{1.037749in}}{\pgfqpoint{1.269726in}{1.034477in}}{\pgfqpoint{1.263902in}{1.028653in}}%
\pgfpathcurveto{\pgfqpoint{1.258078in}{1.022829in}}{\pgfqpoint{1.254806in}{1.014929in}}{\pgfqpoint{1.254806in}{1.006692in}}%
\pgfpathcurveto{\pgfqpoint{1.254806in}{0.998456in}}{\pgfqpoint{1.258078in}{0.990556in}}{\pgfqpoint{1.263902in}{0.984732in}}%
\pgfpathcurveto{\pgfqpoint{1.269726in}{0.978908in}}{\pgfqpoint{1.277626in}{0.975636in}}{\pgfqpoint{1.285862in}{0.975636in}}%
\pgfpathclose%
\pgfusepath{stroke,fill}%
\end{pgfscope}%
\begin{pgfscope}%
\pgfpathrectangle{\pgfqpoint{0.556847in}{0.516222in}}{\pgfqpoint{1.962733in}{1.783528in}} %
\pgfusepath{clip}%
\pgfsetbuttcap%
\pgfsetroundjoin%
\definecolor{currentfill}{rgb}{0.298039,0.447059,0.690196}%
\pgfsetfillcolor{currentfill}%
\pgfsetlinewidth{0.240900pt}%
\definecolor{currentstroke}{rgb}{1.000000,1.000000,1.000000}%
\pgfsetstrokecolor{currentstroke}%
\pgfsetdash{}{0pt}%
\pgfpathmoveto{\pgfqpoint{1.454096in}{0.730401in}}%
\pgfpathcurveto{\pgfqpoint{1.462333in}{0.730401in}}{\pgfqpoint{1.470233in}{0.733673in}}{\pgfqpoint{1.476057in}{0.739497in}}%
\pgfpathcurveto{\pgfqpoint{1.481881in}{0.745321in}}{\pgfqpoint{1.485153in}{0.753221in}}{\pgfqpoint{1.485153in}{0.761457in}}%
\pgfpathcurveto{\pgfqpoint{1.485153in}{0.769694in}}{\pgfqpoint{1.481881in}{0.777594in}}{\pgfqpoint{1.476057in}{0.783418in}}%
\pgfpathcurveto{\pgfqpoint{1.470233in}{0.789241in}}{\pgfqpoint{1.462333in}{0.792514in}}{\pgfqpoint{1.454096in}{0.792514in}}%
\pgfpathcurveto{\pgfqpoint{1.445860in}{0.792514in}}{\pgfqpoint{1.437960in}{0.789241in}}{\pgfqpoint{1.432136in}{0.783418in}}%
\pgfpathcurveto{\pgfqpoint{1.426312in}{0.777594in}}{\pgfqpoint{1.423040in}{0.769694in}}{\pgfqpoint{1.423040in}{0.761457in}}%
\pgfpathcurveto{\pgfqpoint{1.423040in}{0.753221in}}{\pgfqpoint{1.426312in}{0.745321in}}{\pgfqpoint{1.432136in}{0.739497in}}%
\pgfpathcurveto{\pgfqpoint{1.437960in}{0.733673in}}{\pgfqpoint{1.445860in}{0.730401in}}{\pgfqpoint{1.454096in}{0.730401in}}%
\pgfpathclose%
\pgfusepath{stroke,fill}%
\end{pgfscope}%
\begin{pgfscope}%
\pgfpathrectangle{\pgfqpoint{0.556847in}{0.516222in}}{\pgfqpoint{1.962733in}{1.783528in}} %
\pgfusepath{clip}%
\pgfsetbuttcap%
\pgfsetroundjoin%
\definecolor{currentfill}{rgb}{0.298039,0.447059,0.690196}%
\pgfsetfillcolor{currentfill}%
\pgfsetlinewidth{0.240900pt}%
\definecolor{currentstroke}{rgb}{1.000000,1.000000,1.000000}%
\pgfsetstrokecolor{currentstroke}%
\pgfsetdash{}{0pt}%
\pgfpathmoveto{\pgfqpoint{1.398018in}{1.956576in}}%
\pgfpathcurveto{\pgfqpoint{1.406255in}{1.956576in}}{\pgfqpoint{1.414155in}{1.959848in}}{\pgfqpoint{1.419979in}{1.965672in}}%
\pgfpathcurveto{\pgfqpoint{1.425803in}{1.971496in}}{\pgfqpoint{1.429075in}{1.979396in}}{\pgfqpoint{1.429075in}{1.987633in}}%
\pgfpathcurveto{\pgfqpoint{1.429075in}{1.995869in}}{\pgfqpoint{1.425803in}{2.003769in}}{\pgfqpoint{1.419979in}{2.009593in}}%
\pgfpathcurveto{\pgfqpoint{1.414155in}{2.015417in}}{\pgfqpoint{1.406255in}{2.018689in}}{\pgfqpoint{1.398018in}{2.018689in}}%
\pgfpathcurveto{\pgfqpoint{1.389782in}{2.018689in}}{\pgfqpoint{1.381882in}{2.015417in}}{\pgfqpoint{1.376058in}{2.009593in}}%
\pgfpathcurveto{\pgfqpoint{1.370234in}{2.003769in}}{\pgfqpoint{1.366962in}{1.995869in}}{\pgfqpoint{1.366962in}{1.987633in}}%
\pgfpathcurveto{\pgfqpoint{1.366962in}{1.979396in}}{\pgfqpoint{1.370234in}{1.971496in}}{\pgfqpoint{1.376058in}{1.965672in}}%
\pgfpathcurveto{\pgfqpoint{1.381882in}{1.959848in}}{\pgfqpoint{1.389782in}{1.956576in}}{\pgfqpoint{1.398018in}{1.956576in}}%
\pgfpathclose%
\pgfusepath{stroke,fill}%
\end{pgfscope}%
\begin{pgfscope}%
\pgfpathrectangle{\pgfqpoint{0.556847in}{0.516222in}}{\pgfqpoint{1.962733in}{1.783528in}} %
\pgfusepath{clip}%
\pgfsetbuttcap%
\pgfsetroundjoin%
\definecolor{currentfill}{rgb}{0.298039,0.447059,0.690196}%
\pgfsetfillcolor{currentfill}%
\pgfsetlinewidth{0.240900pt}%
\definecolor{currentstroke}{rgb}{1.000000,1.000000,1.000000}%
\pgfsetstrokecolor{currentstroke}%
\pgfsetdash{}{0pt}%
\pgfpathmoveto{\pgfqpoint{2.014877in}{1.390306in}}%
\pgfpathcurveto{\pgfqpoint{2.023113in}{1.390306in}}{\pgfqpoint{2.031014in}{1.393578in}}{\pgfqpoint{2.036837in}{1.399402in}}%
\pgfpathcurveto{\pgfqpoint{2.042661in}{1.405226in}}{\pgfqpoint{2.045934in}{1.413126in}}{\pgfqpoint{2.045934in}{1.421363in}}%
\pgfpathcurveto{\pgfqpoint{2.045934in}{1.429599in}}{\pgfqpoint{2.042661in}{1.437499in}}{\pgfqpoint{2.036837in}{1.443323in}}%
\pgfpathcurveto{\pgfqpoint{2.031014in}{1.449147in}}{\pgfqpoint{2.023113in}{1.452419in}}{\pgfqpoint{2.014877in}{1.452419in}}%
\pgfpathcurveto{\pgfqpoint{2.006641in}{1.452419in}}{\pgfqpoint{1.998741in}{1.449147in}}{\pgfqpoint{1.992917in}{1.443323in}}%
\pgfpathcurveto{\pgfqpoint{1.987093in}{1.437499in}}{\pgfqpoint{1.983821in}{1.429599in}}{\pgfqpoint{1.983821in}{1.421363in}}%
\pgfpathcurveto{\pgfqpoint{1.983821in}{1.413126in}}{\pgfqpoint{1.987093in}{1.405226in}}{\pgfqpoint{1.992917in}{1.399402in}}%
\pgfpathcurveto{\pgfqpoint{1.998741in}{1.393578in}}{\pgfqpoint{2.006641in}{1.390306in}}{\pgfqpoint{2.014877in}{1.390306in}}%
\pgfpathclose%
\pgfusepath{stroke,fill}%
\end{pgfscope}%
\begin{pgfscope}%
\pgfpathrectangle{\pgfqpoint{0.556847in}{0.516222in}}{\pgfqpoint{1.962733in}{1.783528in}} %
\pgfusepath{clip}%
\pgfsetbuttcap%
\pgfsetroundjoin%
\definecolor{currentfill}{rgb}{0.298039,0.447059,0.690196}%
\pgfsetfillcolor{currentfill}%
\pgfsetlinewidth{0.240900pt}%
\definecolor{currentstroke}{rgb}{1.000000,1.000000,1.000000}%
\pgfsetstrokecolor{currentstroke}%
\pgfsetdash{}{0pt}%
\pgfpathmoveto{\pgfqpoint{1.846643in}{1.952117in}}%
\pgfpathcurveto{\pgfqpoint{1.854879in}{1.952117in}}{\pgfqpoint{1.862779in}{1.955390in}}{\pgfqpoint{1.868603in}{1.961214in}}%
\pgfpathcurveto{\pgfqpoint{1.874427in}{1.967037in}}{\pgfqpoint{1.877699in}{1.974938in}}{\pgfqpoint{1.877699in}{1.983174in}}%
\pgfpathcurveto{\pgfqpoint{1.877699in}{1.991410in}}{\pgfqpoint{1.874427in}{1.999310in}}{\pgfqpoint{1.868603in}{2.005134in}}%
\pgfpathcurveto{\pgfqpoint{1.862779in}{2.010958in}}{\pgfqpoint{1.854879in}{2.014230in}}{\pgfqpoint{1.846643in}{2.014230in}}%
\pgfpathcurveto{\pgfqpoint{1.838407in}{2.014230in}}{\pgfqpoint{1.830507in}{2.010958in}}{\pgfqpoint{1.824683in}{2.005134in}}%
\pgfpathcurveto{\pgfqpoint{1.818859in}{1.999310in}}{\pgfqpoint{1.815586in}{1.991410in}}{\pgfqpoint{1.815586in}{1.983174in}}%
\pgfpathcurveto{\pgfqpoint{1.815586in}{1.974938in}}{\pgfqpoint{1.818859in}{1.967037in}}{\pgfqpoint{1.824683in}{1.961214in}}%
\pgfpathcurveto{\pgfqpoint{1.830507in}{1.955390in}}{\pgfqpoint{1.838407in}{1.952117in}}{\pgfqpoint{1.846643in}{1.952117in}}%
\pgfpathclose%
\pgfusepath{stroke,fill}%
\end{pgfscope}%
\begin{pgfscope}%
\pgfpathrectangle{\pgfqpoint{0.556847in}{0.516222in}}{\pgfqpoint{1.962733in}{1.783528in}} %
\pgfusepath{clip}%
\pgfsetbuttcap%
\pgfsetroundjoin%
\definecolor{currentfill}{rgb}{0.298039,0.447059,0.690196}%
\pgfsetfillcolor{currentfill}%
\pgfsetlinewidth{0.240900pt}%
\definecolor{currentstroke}{rgb}{1.000000,1.000000,1.000000}%
\pgfsetstrokecolor{currentstroke}%
\pgfsetdash{}{0pt}%
\pgfpathmoveto{\pgfqpoint{1.229784in}{1.702423in}}%
\pgfpathcurveto{\pgfqpoint{1.238020in}{1.702423in}}{\pgfqpoint{1.245920in}{1.705696in}}{\pgfqpoint{1.251744in}{1.711520in}}%
\pgfpathcurveto{\pgfqpoint{1.257568in}{1.717344in}}{\pgfqpoint{1.260841in}{1.725244in}}{\pgfqpoint{1.260841in}{1.733480in}}%
\pgfpathcurveto{\pgfqpoint{1.260841in}{1.741716in}}{\pgfqpoint{1.257568in}{1.749616in}}{\pgfqpoint{1.251744in}{1.755440in}}%
\pgfpathcurveto{\pgfqpoint{1.245920in}{1.761264in}}{\pgfqpoint{1.238020in}{1.764536in}}{\pgfqpoint{1.229784in}{1.764536in}}%
\pgfpathcurveto{\pgfqpoint{1.221548in}{1.764536in}}{\pgfqpoint{1.213648in}{1.761264in}}{\pgfqpoint{1.207824in}{1.755440in}}%
\pgfpathcurveto{\pgfqpoint{1.202000in}{1.749616in}}{\pgfqpoint{1.198728in}{1.741716in}}{\pgfqpoint{1.198728in}{1.733480in}}%
\pgfpathcurveto{\pgfqpoint{1.198728in}{1.725244in}}{\pgfqpoint{1.202000in}{1.717344in}}{\pgfqpoint{1.207824in}{1.711520in}}%
\pgfpathcurveto{\pgfqpoint{1.213648in}{1.705696in}}{\pgfqpoint{1.221548in}{1.702423in}}{\pgfqpoint{1.229784in}{1.702423in}}%
\pgfpathclose%
\pgfusepath{stroke,fill}%
\end{pgfscope}%
\begin{pgfscope}%
\pgfpathrectangle{\pgfqpoint{0.556847in}{0.516222in}}{\pgfqpoint{1.962733in}{1.783528in}} %
\pgfusepath{clip}%
\pgfsetbuttcap%
\pgfsetroundjoin%
\definecolor{currentfill}{rgb}{0.298039,0.447059,0.690196}%
\pgfsetfillcolor{currentfill}%
\pgfsetlinewidth{0.240900pt}%
\definecolor{currentstroke}{rgb}{1.000000,1.000000,1.000000}%
\pgfsetstrokecolor{currentstroke}%
\pgfsetdash{}{0pt}%
\pgfpathmoveto{\pgfqpoint{2.183111in}{1.715800in}}%
\pgfpathcurveto{\pgfqpoint{2.191348in}{1.715800in}}{\pgfqpoint{2.199248in}{1.719072in}}{\pgfqpoint{2.205072in}{1.724896in}}%
\pgfpathcurveto{\pgfqpoint{2.210896in}{1.730720in}}{\pgfqpoint{2.214168in}{1.738620in}}{\pgfqpoint{2.214168in}{1.746856in}}%
\pgfpathcurveto{\pgfqpoint{2.214168in}{1.755093in}}{\pgfqpoint{2.210896in}{1.762993in}}{\pgfqpoint{2.205072in}{1.768817in}}%
\pgfpathcurveto{\pgfqpoint{2.199248in}{1.774641in}}{\pgfqpoint{2.191348in}{1.777913in}}{\pgfqpoint{2.183111in}{1.777913in}}%
\pgfpathcurveto{\pgfqpoint{2.174875in}{1.777913in}}{\pgfqpoint{2.166975in}{1.774641in}}{\pgfqpoint{2.161151in}{1.768817in}}%
\pgfpathcurveto{\pgfqpoint{2.155327in}{1.762993in}}{\pgfqpoint{2.152055in}{1.755093in}}{\pgfqpoint{2.152055in}{1.746856in}}%
\pgfpathcurveto{\pgfqpoint{2.152055in}{1.738620in}}{\pgfqpoint{2.155327in}{1.730720in}}{\pgfqpoint{2.161151in}{1.724896in}}%
\pgfpathcurveto{\pgfqpoint{2.166975in}{1.719072in}}{\pgfqpoint{2.174875in}{1.715800in}}{\pgfqpoint{2.183111in}{1.715800in}}%
\pgfpathclose%
\pgfusepath{stroke,fill}%
\end{pgfscope}%
\begin{pgfscope}%
\pgfpathrectangle{\pgfqpoint{0.556847in}{0.516222in}}{\pgfqpoint{1.962733in}{1.783528in}} %
\pgfusepath{clip}%
\pgfsetbuttcap%
\pgfsetroundjoin%
\definecolor{currentfill}{rgb}{0.298039,0.447059,0.690196}%
\pgfsetfillcolor{currentfill}%
\pgfsetlinewidth{0.240900pt}%
\definecolor{currentstroke}{rgb}{1.000000,1.000000,1.000000}%
\pgfsetstrokecolor{currentstroke}%
\pgfsetdash{}{0pt}%
\pgfpathmoveto{\pgfqpoint{2.295268in}{1.671212in}}%
\pgfpathcurveto{\pgfqpoint{2.303504in}{1.671212in}}{\pgfqpoint{2.311404in}{1.674484in}}{\pgfqpoint{2.317228in}{1.680308in}}%
\pgfpathcurveto{\pgfqpoint{2.323052in}{1.686132in}}{\pgfqpoint{2.326324in}{1.694032in}}{\pgfqpoint{2.326324in}{1.702268in}}%
\pgfpathcurveto{\pgfqpoint{2.326324in}{1.710504in}}{\pgfqpoint{2.323052in}{1.718405in}}{\pgfqpoint{2.317228in}{1.724228in}}%
\pgfpathcurveto{\pgfqpoint{2.311404in}{1.730052in}}{\pgfqpoint{2.303504in}{1.733325in}}{\pgfqpoint{2.295268in}{1.733325in}}%
\pgfpathcurveto{\pgfqpoint{2.287031in}{1.733325in}}{\pgfqpoint{2.279131in}{1.730052in}}{\pgfqpoint{2.273307in}{1.724228in}}%
\pgfpathcurveto{\pgfqpoint{2.267483in}{1.718405in}}{\pgfqpoint{2.264211in}{1.710504in}}{\pgfqpoint{2.264211in}{1.702268in}}%
\pgfpathcurveto{\pgfqpoint{2.264211in}{1.694032in}}{\pgfqpoint{2.267483in}{1.686132in}}{\pgfqpoint{2.273307in}{1.680308in}}%
\pgfpathcurveto{\pgfqpoint{2.279131in}{1.674484in}}{\pgfqpoint{2.287031in}{1.671212in}}{\pgfqpoint{2.295268in}{1.671212in}}%
\pgfpathclose%
\pgfusepath{stroke,fill}%
\end{pgfscope}%
\begin{pgfscope}%
\pgfpathrectangle{\pgfqpoint{0.556847in}{0.516222in}}{\pgfqpoint{1.962733in}{1.783528in}} %
\pgfusepath{clip}%
\pgfsetbuttcap%
\pgfsetroundjoin%
\definecolor{currentfill}{rgb}{0.298039,0.447059,0.690196}%
\pgfsetfillcolor{currentfill}%
\pgfsetlinewidth{0.240900pt}%
\definecolor{currentstroke}{rgb}{1.000000,1.000000,1.000000}%
\pgfsetstrokecolor{currentstroke}%
\pgfsetdash{}{0pt}%
\pgfpathmoveto{\pgfqpoint{1.341940in}{0.725942in}}%
\pgfpathcurveto{\pgfqpoint{1.350177in}{0.725942in}}{\pgfqpoint{1.358077in}{0.729214in}}{\pgfqpoint{1.363901in}{0.735038in}}%
\pgfpathcurveto{\pgfqpoint{1.369724in}{0.740862in}}{\pgfqpoint{1.372997in}{0.748762in}}{\pgfqpoint{1.372997in}{0.756998in}}%
\pgfpathcurveto{\pgfqpoint{1.372997in}{0.765235in}}{\pgfqpoint{1.369724in}{0.773135in}}{\pgfqpoint{1.363901in}{0.778959in}}%
\pgfpathcurveto{\pgfqpoint{1.358077in}{0.784783in}}{\pgfqpoint{1.350177in}{0.788055in}}{\pgfqpoint{1.341940in}{0.788055in}}%
\pgfpathcurveto{\pgfqpoint{1.333704in}{0.788055in}}{\pgfqpoint{1.325804in}{0.784783in}}{\pgfqpoint{1.319980in}{0.778959in}}%
\pgfpathcurveto{\pgfqpoint{1.314156in}{0.773135in}}{\pgfqpoint{1.310884in}{0.765235in}}{\pgfqpoint{1.310884in}{0.756998in}}%
\pgfpathcurveto{\pgfqpoint{1.310884in}{0.748762in}}{\pgfqpoint{1.314156in}{0.740862in}}{\pgfqpoint{1.319980in}{0.735038in}}%
\pgfpathcurveto{\pgfqpoint{1.325804in}{0.729214in}}{\pgfqpoint{1.333704in}{0.725942in}}{\pgfqpoint{1.341940in}{0.725942in}}%
\pgfpathclose%
\pgfusepath{stroke,fill}%
\end{pgfscope}%
\begin{pgfscope}%
\pgfpathrectangle{\pgfqpoint{0.556847in}{0.516222in}}{\pgfqpoint{1.962733in}{1.783528in}} %
\pgfusepath{clip}%
\pgfsetbuttcap%
\pgfsetroundjoin%
\definecolor{currentfill}{rgb}{0.298039,0.447059,0.690196}%
\pgfsetfillcolor{currentfill}%
\pgfsetlinewidth{0.240900pt}%
\definecolor{currentstroke}{rgb}{1.000000,1.000000,1.000000}%
\pgfsetstrokecolor{currentstroke}%
\pgfsetdash{}{0pt}%
\pgfpathmoveto{\pgfqpoint{0.837238in}{1.943200in}}%
\pgfpathcurveto{\pgfqpoint{0.845474in}{1.943200in}}{\pgfqpoint{0.853374in}{1.946472in}}{\pgfqpoint{0.859198in}{1.952296in}}%
\pgfpathcurveto{\pgfqpoint{0.865022in}{1.958120in}}{\pgfqpoint{0.868294in}{1.966020in}}{\pgfqpoint{0.868294in}{1.974256in}}%
\pgfpathcurveto{\pgfqpoint{0.868294in}{1.982492in}}{\pgfqpoint{0.865022in}{1.990393in}}{\pgfqpoint{0.859198in}{1.996216in}}%
\pgfpathcurveto{\pgfqpoint{0.853374in}{2.002040in}}{\pgfqpoint{0.845474in}{2.005313in}}{\pgfqpoint{0.837238in}{2.005313in}}%
\pgfpathcurveto{\pgfqpoint{0.829001in}{2.005313in}}{\pgfqpoint{0.821101in}{2.002040in}}{\pgfqpoint{0.815277in}{1.996216in}}%
\pgfpathcurveto{\pgfqpoint{0.809453in}{1.990393in}}{\pgfqpoint{0.806181in}{1.982492in}}{\pgfqpoint{0.806181in}{1.974256in}}%
\pgfpathcurveto{\pgfqpoint{0.806181in}{1.966020in}}{\pgfqpoint{0.809453in}{1.958120in}}{\pgfqpoint{0.815277in}{1.952296in}}%
\pgfpathcurveto{\pgfqpoint{0.821101in}{1.946472in}}{\pgfqpoint{0.829001in}{1.943200in}}{\pgfqpoint{0.837238in}{1.943200in}}%
\pgfpathclose%
\pgfusepath{stroke,fill}%
\end{pgfscope}%
\begin{pgfscope}%
\pgfpathrectangle{\pgfqpoint{0.556847in}{0.516222in}}{\pgfqpoint{1.962733in}{1.783528in}} %
\pgfusepath{clip}%
\pgfsetbuttcap%
\pgfsetroundjoin%
\definecolor{currentfill}{rgb}{0.298039,0.447059,0.690196}%
\pgfsetfillcolor{currentfill}%
\pgfsetlinewidth{0.240900pt}%
\definecolor{currentstroke}{rgb}{1.000000,1.000000,1.000000}%
\pgfsetstrokecolor{currentstroke}%
\pgfsetdash{}{0pt}%
\pgfpathmoveto{\pgfqpoint{1.285862in}{1.381388in}}%
\pgfpathcurveto{\pgfqpoint{1.294098in}{1.381388in}}{\pgfqpoint{1.301999in}{1.384661in}}{\pgfqpoint{1.307822in}{1.390485in}}%
\pgfpathcurveto{\pgfqpoint{1.313646in}{1.396309in}}{\pgfqpoint{1.316919in}{1.404209in}}{\pgfqpoint{1.316919in}{1.412445in}}%
\pgfpathcurveto{\pgfqpoint{1.316919in}{1.420681in}}{\pgfqpoint{1.313646in}{1.428581in}}{\pgfqpoint{1.307822in}{1.434405in}}%
\pgfpathcurveto{\pgfqpoint{1.301999in}{1.440229in}}{\pgfqpoint{1.294098in}{1.443501in}}{\pgfqpoint{1.285862in}{1.443501in}}%
\pgfpathcurveto{\pgfqpoint{1.277626in}{1.443501in}}{\pgfqpoint{1.269726in}{1.440229in}}{\pgfqpoint{1.263902in}{1.434405in}}%
\pgfpathcurveto{\pgfqpoint{1.258078in}{1.428581in}}{\pgfqpoint{1.254806in}{1.420681in}}{\pgfqpoint{1.254806in}{1.412445in}}%
\pgfpathcurveto{\pgfqpoint{1.254806in}{1.404209in}}{\pgfqpoint{1.258078in}{1.396309in}}{\pgfqpoint{1.263902in}{1.390485in}}%
\pgfpathcurveto{\pgfqpoint{1.269726in}{1.384661in}}{\pgfqpoint{1.277626in}{1.381388in}}{\pgfqpoint{1.285862in}{1.381388in}}%
\pgfpathclose%
\pgfusepath{stroke,fill}%
\end{pgfscope}%
\begin{pgfscope}%
\pgfpathrectangle{\pgfqpoint{0.556847in}{0.516222in}}{\pgfqpoint{1.962733in}{1.783528in}} %
\pgfusepath{clip}%
\pgfsetbuttcap%
\pgfsetroundjoin%
\definecolor{currentfill}{rgb}{0.298039,0.447059,0.690196}%
\pgfsetfillcolor{currentfill}%
\pgfsetlinewidth{0.240900pt}%
\definecolor{currentstroke}{rgb}{1.000000,1.000000,1.000000}%
\pgfsetstrokecolor{currentstroke}%
\pgfsetdash{}{0pt}%
\pgfpathmoveto{\pgfqpoint{2.183111in}{1.987788in}}%
\pgfpathcurveto{\pgfqpoint{2.191348in}{1.987788in}}{\pgfqpoint{2.199248in}{1.991060in}}{\pgfqpoint{2.205072in}{1.996884in}}%
\pgfpathcurveto{\pgfqpoint{2.210896in}{2.002708in}}{\pgfqpoint{2.214168in}{2.010608in}}{\pgfqpoint{2.214168in}{2.018844in}}%
\pgfpathcurveto{\pgfqpoint{2.214168in}{2.027081in}}{\pgfqpoint{2.210896in}{2.034981in}}{\pgfqpoint{2.205072in}{2.040805in}}%
\pgfpathcurveto{\pgfqpoint{2.199248in}{2.046629in}}{\pgfqpoint{2.191348in}{2.049901in}}{\pgfqpoint{2.183111in}{2.049901in}}%
\pgfpathcurveto{\pgfqpoint{2.174875in}{2.049901in}}{\pgfqpoint{2.166975in}{2.046629in}}{\pgfqpoint{2.161151in}{2.040805in}}%
\pgfpathcurveto{\pgfqpoint{2.155327in}{2.034981in}}{\pgfqpoint{2.152055in}{2.027081in}}{\pgfqpoint{2.152055in}{2.018844in}}%
\pgfpathcurveto{\pgfqpoint{2.152055in}{2.010608in}}{\pgfqpoint{2.155327in}{2.002708in}}{\pgfqpoint{2.161151in}{1.996884in}}%
\pgfpathcurveto{\pgfqpoint{2.166975in}{1.991060in}}{\pgfqpoint{2.174875in}{1.987788in}}{\pgfqpoint{2.183111in}{1.987788in}}%
\pgfpathclose%
\pgfusepath{stroke,fill}%
\end{pgfscope}%
\begin{pgfscope}%
\pgfsetrectcap%
\pgfsetmiterjoin%
\pgfsetlinewidth{0.000000pt}%
\definecolor{currentstroke}{rgb}{1.000000,1.000000,1.000000}%
\pgfsetstrokecolor{currentstroke}%
\pgfsetdash{}{0pt}%
\pgfpathmoveto{\pgfqpoint{0.556847in}{0.516222in}}%
\pgfpathlineto{\pgfqpoint{0.556847in}{2.299750in}}%
\pgfusepath{}%
\end{pgfscope}%
\begin{pgfscope}%
\pgfsetrectcap%
\pgfsetmiterjoin%
\pgfsetlinewidth{0.000000pt}%
\definecolor{currentstroke}{rgb}{1.000000,1.000000,1.000000}%
\pgfsetstrokecolor{currentstroke}%
\pgfsetdash{}{0pt}%
\pgfpathmoveto{\pgfqpoint{0.556847in}{0.516222in}}%
\pgfpathlineto{\pgfqpoint{2.519580in}{0.516222in}}%
\pgfusepath{}%
\end{pgfscope}%
\begin{pgfscope}%
\pgfsetbuttcap%
\pgfsetmiterjoin%
\definecolor{currentfill}{rgb}{0.917647,0.917647,0.949020}%
\pgfsetfillcolor{currentfill}%
\pgfsetlinewidth{0.000000pt}%
\definecolor{currentstroke}{rgb}{0.000000,0.000000,0.000000}%
\pgfsetstrokecolor{currentstroke}%
\pgfsetstrokeopacity{0.000000}%
\pgfsetdash{}{0pt}%
\pgfpathmoveto{\pgfqpoint{2.816705in}{0.516222in}}%
\pgfpathlineto{\pgfqpoint{4.779438in}{0.516222in}}%
\pgfpathlineto{\pgfqpoint{4.779438in}{2.299750in}}%
\pgfpathlineto{\pgfqpoint{2.816705in}{2.299750in}}%
\pgfpathclose%
\pgfusepath{fill}%
\end{pgfscope}%
\begin{pgfscope}%
\pgfpathrectangle{\pgfqpoint{2.816705in}{0.516222in}}{\pgfqpoint{1.962733in}{1.783528in}} %
\pgfusepath{clip}%
\pgfsetroundcap%
\pgfsetroundjoin%
\pgfsetlinewidth{0.803000pt}%
\definecolor{currentstroke}{rgb}{1.000000,1.000000,1.000000}%
\pgfsetstrokecolor{currentstroke}%
\pgfsetdash{}{0pt}%
\pgfpathmoveto{\pgfqpoint{2.816705in}{0.516222in}}%
\pgfpathlineto{\pgfqpoint{2.816705in}{2.299750in}}%
\pgfusepath{stroke}%
\end{pgfscope}%
\begin{pgfscope}%
\pgfsetbuttcap%
\pgfsetroundjoin%
\definecolor{currentfill}{rgb}{0.150000,0.150000,0.150000}%
\pgfsetfillcolor{currentfill}%
\pgfsetlinewidth{0.803000pt}%
\definecolor{currentstroke}{rgb}{0.150000,0.150000,0.150000}%
\pgfsetstrokecolor{currentstroke}%
\pgfsetdash{}{0pt}%
\pgfsys@defobject{currentmarker}{\pgfqpoint{0.000000in}{0.000000in}}{\pgfqpoint{0.000000in}{0.000000in}}{%
\pgfpathmoveto{\pgfqpoint{0.000000in}{0.000000in}}%
\pgfpathlineto{\pgfqpoint{0.000000in}{0.000000in}}%
\pgfusepath{stroke,fill}%
}%
\begin{pgfscope}%
\pgfsys@transformshift{2.816705in}{0.516222in}%
\pgfsys@useobject{currentmarker}{}%
\end{pgfscope}%
\end{pgfscope}%
\begin{pgfscope}%
\definecolor{textcolor}{rgb}{0.150000,0.150000,0.150000}%
\pgfsetstrokecolor{textcolor}%
\pgfsetfillcolor{textcolor}%
\pgftext[x=2.816705in,y=0.438444in,,top]{\color{textcolor}\sffamily\fontsize{8.000000}{9.600000}\selectfont 2.0}%
\end{pgfscope}%
\begin{pgfscope}%
\pgfpathrectangle{\pgfqpoint{2.816705in}{0.516222in}}{\pgfqpoint{1.962733in}{1.783528in}} %
\pgfusepath{clip}%
\pgfsetroundcap%
\pgfsetroundjoin%
\pgfsetlinewidth{0.803000pt}%
\definecolor{currentstroke}{rgb}{1.000000,1.000000,1.000000}%
\pgfsetstrokecolor{currentstroke}%
\pgfsetdash{}{0pt}%
\pgfpathmoveto{\pgfqpoint{3.097095in}{0.516222in}}%
\pgfpathlineto{\pgfqpoint{3.097095in}{2.299750in}}%
\pgfusepath{stroke}%
\end{pgfscope}%
\begin{pgfscope}%
\pgfsetbuttcap%
\pgfsetroundjoin%
\definecolor{currentfill}{rgb}{0.150000,0.150000,0.150000}%
\pgfsetfillcolor{currentfill}%
\pgfsetlinewidth{0.803000pt}%
\definecolor{currentstroke}{rgb}{0.150000,0.150000,0.150000}%
\pgfsetstrokecolor{currentstroke}%
\pgfsetdash{}{0pt}%
\pgfsys@defobject{currentmarker}{\pgfqpoint{0.000000in}{0.000000in}}{\pgfqpoint{0.000000in}{0.000000in}}{%
\pgfpathmoveto{\pgfqpoint{0.000000in}{0.000000in}}%
\pgfpathlineto{\pgfqpoint{0.000000in}{0.000000in}}%
\pgfusepath{stroke,fill}%
}%
\begin{pgfscope}%
\pgfsys@transformshift{3.097095in}{0.516222in}%
\pgfsys@useobject{currentmarker}{}%
\end{pgfscope}%
\end{pgfscope}%
\begin{pgfscope}%
\definecolor{textcolor}{rgb}{0.150000,0.150000,0.150000}%
\pgfsetstrokecolor{textcolor}%
\pgfsetfillcolor{textcolor}%
\pgftext[x=3.097095in,y=0.438444in,,top]{\color{textcolor}\sffamily\fontsize{8.000000}{9.600000}\selectfont 2.5}%
\end{pgfscope}%
\begin{pgfscope}%
\pgfpathrectangle{\pgfqpoint{2.816705in}{0.516222in}}{\pgfqpoint{1.962733in}{1.783528in}} %
\pgfusepath{clip}%
\pgfsetroundcap%
\pgfsetroundjoin%
\pgfsetlinewidth{0.803000pt}%
\definecolor{currentstroke}{rgb}{1.000000,1.000000,1.000000}%
\pgfsetstrokecolor{currentstroke}%
\pgfsetdash{}{0pt}%
\pgfpathmoveto{\pgfqpoint{3.377486in}{0.516222in}}%
\pgfpathlineto{\pgfqpoint{3.377486in}{2.299750in}}%
\pgfusepath{stroke}%
\end{pgfscope}%
\begin{pgfscope}%
\pgfsetbuttcap%
\pgfsetroundjoin%
\definecolor{currentfill}{rgb}{0.150000,0.150000,0.150000}%
\pgfsetfillcolor{currentfill}%
\pgfsetlinewidth{0.803000pt}%
\definecolor{currentstroke}{rgb}{0.150000,0.150000,0.150000}%
\pgfsetstrokecolor{currentstroke}%
\pgfsetdash{}{0pt}%
\pgfsys@defobject{currentmarker}{\pgfqpoint{0.000000in}{0.000000in}}{\pgfqpoint{0.000000in}{0.000000in}}{%
\pgfpathmoveto{\pgfqpoint{0.000000in}{0.000000in}}%
\pgfpathlineto{\pgfqpoint{0.000000in}{0.000000in}}%
\pgfusepath{stroke,fill}%
}%
\begin{pgfscope}%
\pgfsys@transformshift{3.377486in}{0.516222in}%
\pgfsys@useobject{currentmarker}{}%
\end{pgfscope}%
\end{pgfscope}%
\begin{pgfscope}%
\definecolor{textcolor}{rgb}{0.150000,0.150000,0.150000}%
\pgfsetstrokecolor{textcolor}%
\pgfsetfillcolor{textcolor}%
\pgftext[x=3.377486in,y=0.438444in,,top]{\color{textcolor}\sffamily\fontsize{8.000000}{9.600000}\selectfont 3.0}%
\end{pgfscope}%
\begin{pgfscope}%
\pgfpathrectangle{\pgfqpoint{2.816705in}{0.516222in}}{\pgfqpoint{1.962733in}{1.783528in}} %
\pgfusepath{clip}%
\pgfsetroundcap%
\pgfsetroundjoin%
\pgfsetlinewidth{0.803000pt}%
\definecolor{currentstroke}{rgb}{1.000000,1.000000,1.000000}%
\pgfsetstrokecolor{currentstroke}%
\pgfsetdash{}{0pt}%
\pgfpathmoveto{\pgfqpoint{3.657876in}{0.516222in}}%
\pgfpathlineto{\pgfqpoint{3.657876in}{2.299750in}}%
\pgfusepath{stroke}%
\end{pgfscope}%
\begin{pgfscope}%
\pgfsetbuttcap%
\pgfsetroundjoin%
\definecolor{currentfill}{rgb}{0.150000,0.150000,0.150000}%
\pgfsetfillcolor{currentfill}%
\pgfsetlinewidth{0.803000pt}%
\definecolor{currentstroke}{rgb}{0.150000,0.150000,0.150000}%
\pgfsetstrokecolor{currentstroke}%
\pgfsetdash{}{0pt}%
\pgfsys@defobject{currentmarker}{\pgfqpoint{0.000000in}{0.000000in}}{\pgfqpoint{0.000000in}{0.000000in}}{%
\pgfpathmoveto{\pgfqpoint{0.000000in}{0.000000in}}%
\pgfpathlineto{\pgfqpoint{0.000000in}{0.000000in}}%
\pgfusepath{stroke,fill}%
}%
\begin{pgfscope}%
\pgfsys@transformshift{3.657876in}{0.516222in}%
\pgfsys@useobject{currentmarker}{}%
\end{pgfscope}%
\end{pgfscope}%
\begin{pgfscope}%
\definecolor{textcolor}{rgb}{0.150000,0.150000,0.150000}%
\pgfsetstrokecolor{textcolor}%
\pgfsetfillcolor{textcolor}%
\pgftext[x=3.657876in,y=0.438444in,,top]{\color{textcolor}\sffamily\fontsize{8.000000}{9.600000}\selectfont 3.5}%
\end{pgfscope}%
\begin{pgfscope}%
\pgfpathrectangle{\pgfqpoint{2.816705in}{0.516222in}}{\pgfqpoint{1.962733in}{1.783528in}} %
\pgfusepath{clip}%
\pgfsetroundcap%
\pgfsetroundjoin%
\pgfsetlinewidth{0.803000pt}%
\definecolor{currentstroke}{rgb}{1.000000,1.000000,1.000000}%
\pgfsetstrokecolor{currentstroke}%
\pgfsetdash{}{0pt}%
\pgfpathmoveto{\pgfqpoint{3.938266in}{0.516222in}}%
\pgfpathlineto{\pgfqpoint{3.938266in}{2.299750in}}%
\pgfusepath{stroke}%
\end{pgfscope}%
\begin{pgfscope}%
\pgfsetbuttcap%
\pgfsetroundjoin%
\definecolor{currentfill}{rgb}{0.150000,0.150000,0.150000}%
\pgfsetfillcolor{currentfill}%
\pgfsetlinewidth{0.803000pt}%
\definecolor{currentstroke}{rgb}{0.150000,0.150000,0.150000}%
\pgfsetstrokecolor{currentstroke}%
\pgfsetdash{}{0pt}%
\pgfsys@defobject{currentmarker}{\pgfqpoint{0.000000in}{0.000000in}}{\pgfqpoint{0.000000in}{0.000000in}}{%
\pgfpathmoveto{\pgfqpoint{0.000000in}{0.000000in}}%
\pgfpathlineto{\pgfqpoint{0.000000in}{0.000000in}}%
\pgfusepath{stroke,fill}%
}%
\begin{pgfscope}%
\pgfsys@transformshift{3.938266in}{0.516222in}%
\pgfsys@useobject{currentmarker}{}%
\end{pgfscope}%
\end{pgfscope}%
\begin{pgfscope}%
\definecolor{textcolor}{rgb}{0.150000,0.150000,0.150000}%
\pgfsetstrokecolor{textcolor}%
\pgfsetfillcolor{textcolor}%
\pgftext[x=3.938266in,y=0.438444in,,top]{\color{textcolor}\sffamily\fontsize{8.000000}{9.600000}\selectfont 4.0}%
\end{pgfscope}%
\begin{pgfscope}%
\pgfpathrectangle{\pgfqpoint{2.816705in}{0.516222in}}{\pgfqpoint{1.962733in}{1.783528in}} %
\pgfusepath{clip}%
\pgfsetroundcap%
\pgfsetroundjoin%
\pgfsetlinewidth{0.803000pt}%
\definecolor{currentstroke}{rgb}{1.000000,1.000000,1.000000}%
\pgfsetstrokecolor{currentstroke}%
\pgfsetdash{}{0pt}%
\pgfpathmoveto{\pgfqpoint{4.218657in}{0.516222in}}%
\pgfpathlineto{\pgfqpoint{4.218657in}{2.299750in}}%
\pgfusepath{stroke}%
\end{pgfscope}%
\begin{pgfscope}%
\pgfsetbuttcap%
\pgfsetroundjoin%
\definecolor{currentfill}{rgb}{0.150000,0.150000,0.150000}%
\pgfsetfillcolor{currentfill}%
\pgfsetlinewidth{0.803000pt}%
\definecolor{currentstroke}{rgb}{0.150000,0.150000,0.150000}%
\pgfsetstrokecolor{currentstroke}%
\pgfsetdash{}{0pt}%
\pgfsys@defobject{currentmarker}{\pgfqpoint{0.000000in}{0.000000in}}{\pgfqpoint{0.000000in}{0.000000in}}{%
\pgfpathmoveto{\pgfqpoint{0.000000in}{0.000000in}}%
\pgfpathlineto{\pgfqpoint{0.000000in}{0.000000in}}%
\pgfusepath{stroke,fill}%
}%
\begin{pgfscope}%
\pgfsys@transformshift{4.218657in}{0.516222in}%
\pgfsys@useobject{currentmarker}{}%
\end{pgfscope}%
\end{pgfscope}%
\begin{pgfscope}%
\definecolor{textcolor}{rgb}{0.150000,0.150000,0.150000}%
\pgfsetstrokecolor{textcolor}%
\pgfsetfillcolor{textcolor}%
\pgftext[x=4.218657in,y=0.438444in,,top]{\color{textcolor}\sffamily\fontsize{8.000000}{9.600000}\selectfont 4.5}%
\end{pgfscope}%
\begin{pgfscope}%
\pgfpathrectangle{\pgfqpoint{2.816705in}{0.516222in}}{\pgfqpoint{1.962733in}{1.783528in}} %
\pgfusepath{clip}%
\pgfsetroundcap%
\pgfsetroundjoin%
\pgfsetlinewidth{0.803000pt}%
\definecolor{currentstroke}{rgb}{1.000000,1.000000,1.000000}%
\pgfsetstrokecolor{currentstroke}%
\pgfsetdash{}{0pt}%
\pgfpathmoveto{\pgfqpoint{4.499047in}{0.516222in}}%
\pgfpathlineto{\pgfqpoint{4.499047in}{2.299750in}}%
\pgfusepath{stroke}%
\end{pgfscope}%
\begin{pgfscope}%
\pgfsetbuttcap%
\pgfsetroundjoin%
\definecolor{currentfill}{rgb}{0.150000,0.150000,0.150000}%
\pgfsetfillcolor{currentfill}%
\pgfsetlinewidth{0.803000pt}%
\definecolor{currentstroke}{rgb}{0.150000,0.150000,0.150000}%
\pgfsetstrokecolor{currentstroke}%
\pgfsetdash{}{0pt}%
\pgfsys@defobject{currentmarker}{\pgfqpoint{0.000000in}{0.000000in}}{\pgfqpoint{0.000000in}{0.000000in}}{%
\pgfpathmoveto{\pgfqpoint{0.000000in}{0.000000in}}%
\pgfpathlineto{\pgfqpoint{0.000000in}{0.000000in}}%
\pgfusepath{stroke,fill}%
}%
\begin{pgfscope}%
\pgfsys@transformshift{4.499047in}{0.516222in}%
\pgfsys@useobject{currentmarker}{}%
\end{pgfscope}%
\end{pgfscope}%
\begin{pgfscope}%
\definecolor{textcolor}{rgb}{0.150000,0.150000,0.150000}%
\pgfsetstrokecolor{textcolor}%
\pgfsetfillcolor{textcolor}%
\pgftext[x=4.499047in,y=0.438444in,,top]{\color{textcolor}\sffamily\fontsize{8.000000}{9.600000}\selectfont 5.0}%
\end{pgfscope}%
\begin{pgfscope}%
\pgfpathrectangle{\pgfqpoint{2.816705in}{0.516222in}}{\pgfqpoint{1.962733in}{1.783528in}} %
\pgfusepath{clip}%
\pgfsetroundcap%
\pgfsetroundjoin%
\pgfsetlinewidth{0.803000pt}%
\definecolor{currentstroke}{rgb}{1.000000,1.000000,1.000000}%
\pgfsetstrokecolor{currentstroke}%
\pgfsetdash{}{0pt}%
\pgfpathmoveto{\pgfqpoint{4.779438in}{0.516222in}}%
\pgfpathlineto{\pgfqpoint{4.779438in}{2.299750in}}%
\pgfusepath{stroke}%
\end{pgfscope}%
\begin{pgfscope}%
\pgfsetbuttcap%
\pgfsetroundjoin%
\definecolor{currentfill}{rgb}{0.150000,0.150000,0.150000}%
\pgfsetfillcolor{currentfill}%
\pgfsetlinewidth{0.803000pt}%
\definecolor{currentstroke}{rgb}{0.150000,0.150000,0.150000}%
\pgfsetstrokecolor{currentstroke}%
\pgfsetdash{}{0pt}%
\pgfsys@defobject{currentmarker}{\pgfqpoint{0.000000in}{0.000000in}}{\pgfqpoint{0.000000in}{0.000000in}}{%
\pgfpathmoveto{\pgfqpoint{0.000000in}{0.000000in}}%
\pgfpathlineto{\pgfqpoint{0.000000in}{0.000000in}}%
\pgfusepath{stroke,fill}%
}%
\begin{pgfscope}%
\pgfsys@transformshift{4.779438in}{0.516222in}%
\pgfsys@useobject{currentmarker}{}%
\end{pgfscope}%
\end{pgfscope}%
\begin{pgfscope}%
\definecolor{textcolor}{rgb}{0.150000,0.150000,0.150000}%
\pgfsetstrokecolor{textcolor}%
\pgfsetfillcolor{textcolor}%
\pgftext[x=4.779438in,y=0.438444in,,top]{\color{textcolor}\sffamily\fontsize{8.000000}{9.600000}\selectfont 5.5}%
\end{pgfscope}%
\begin{pgfscope}%
\definecolor{textcolor}{rgb}{0.150000,0.150000,0.150000}%
\pgfsetstrokecolor{textcolor}%
\pgfsetfillcolor{textcolor}%
\pgftext[x=3.798071in,y=0.273321in,,top]{\color{textcolor}\sffamily\fontsize{8.800000}{10.560000}\selectfont Falling time realization 2}%
\end{pgfscope}%
\begin{pgfscope}%
\pgfpathrectangle{\pgfqpoint{2.816705in}{0.516222in}}{\pgfqpoint{1.962733in}{1.783528in}} %
\pgfusepath{clip}%
\pgfsetroundcap%
\pgfsetroundjoin%
\pgfsetlinewidth{0.803000pt}%
\definecolor{currentstroke}{rgb}{1.000000,1.000000,1.000000}%
\pgfsetstrokecolor{currentstroke}%
\pgfsetdash{}{0pt}%
\pgfpathmoveto{\pgfqpoint{2.816705in}{0.516222in}}%
\pgfpathlineto{\pgfqpoint{4.779438in}{0.516222in}}%
\pgfusepath{stroke}%
\end{pgfscope}%
\begin{pgfscope}%
\pgfsetbuttcap%
\pgfsetroundjoin%
\definecolor{currentfill}{rgb}{0.150000,0.150000,0.150000}%
\pgfsetfillcolor{currentfill}%
\pgfsetlinewidth{0.803000pt}%
\definecolor{currentstroke}{rgb}{0.150000,0.150000,0.150000}%
\pgfsetstrokecolor{currentstroke}%
\pgfsetdash{}{0pt}%
\pgfsys@defobject{currentmarker}{\pgfqpoint{0.000000in}{0.000000in}}{\pgfqpoint{0.000000in}{0.000000in}}{%
\pgfpathmoveto{\pgfqpoint{0.000000in}{0.000000in}}%
\pgfpathlineto{\pgfqpoint{0.000000in}{0.000000in}}%
\pgfusepath{stroke,fill}%
}%
\begin{pgfscope}%
\pgfsys@transformshift{2.816705in}{0.516222in}%
\pgfsys@useobject{currentmarker}{}%
\end{pgfscope}%
\end{pgfscope}%
\begin{pgfscope}%
\pgfpathrectangle{\pgfqpoint{2.816705in}{0.516222in}}{\pgfqpoint{1.962733in}{1.783528in}} %
\pgfusepath{clip}%
\pgfsetroundcap%
\pgfsetroundjoin%
\pgfsetlinewidth{0.803000pt}%
\definecolor{currentstroke}{rgb}{1.000000,1.000000,1.000000}%
\pgfsetstrokecolor{currentstroke}%
\pgfsetdash{}{0pt}%
\pgfpathmoveto{\pgfqpoint{2.816705in}{0.739163in}}%
\pgfpathlineto{\pgfqpoint{4.779438in}{0.739163in}}%
\pgfusepath{stroke}%
\end{pgfscope}%
\begin{pgfscope}%
\pgfsetbuttcap%
\pgfsetroundjoin%
\definecolor{currentfill}{rgb}{0.150000,0.150000,0.150000}%
\pgfsetfillcolor{currentfill}%
\pgfsetlinewidth{0.803000pt}%
\definecolor{currentstroke}{rgb}{0.150000,0.150000,0.150000}%
\pgfsetstrokecolor{currentstroke}%
\pgfsetdash{}{0pt}%
\pgfsys@defobject{currentmarker}{\pgfqpoint{0.000000in}{0.000000in}}{\pgfqpoint{0.000000in}{0.000000in}}{%
\pgfpathmoveto{\pgfqpoint{0.000000in}{0.000000in}}%
\pgfpathlineto{\pgfqpoint{0.000000in}{0.000000in}}%
\pgfusepath{stroke,fill}%
}%
\begin{pgfscope}%
\pgfsys@transformshift{2.816705in}{0.739163in}%
\pgfsys@useobject{currentmarker}{}%
\end{pgfscope}%
\end{pgfscope}%
\begin{pgfscope}%
\pgfpathrectangle{\pgfqpoint{2.816705in}{0.516222in}}{\pgfqpoint{1.962733in}{1.783528in}} %
\pgfusepath{clip}%
\pgfsetroundcap%
\pgfsetroundjoin%
\pgfsetlinewidth{0.803000pt}%
\definecolor{currentstroke}{rgb}{1.000000,1.000000,1.000000}%
\pgfsetstrokecolor{currentstroke}%
\pgfsetdash{}{0pt}%
\pgfpathmoveto{\pgfqpoint{2.816705in}{0.962104in}}%
\pgfpathlineto{\pgfqpoint{4.779438in}{0.962104in}}%
\pgfusepath{stroke}%
\end{pgfscope}%
\begin{pgfscope}%
\pgfsetbuttcap%
\pgfsetroundjoin%
\definecolor{currentfill}{rgb}{0.150000,0.150000,0.150000}%
\pgfsetfillcolor{currentfill}%
\pgfsetlinewidth{0.803000pt}%
\definecolor{currentstroke}{rgb}{0.150000,0.150000,0.150000}%
\pgfsetstrokecolor{currentstroke}%
\pgfsetdash{}{0pt}%
\pgfsys@defobject{currentmarker}{\pgfqpoint{0.000000in}{0.000000in}}{\pgfqpoint{0.000000in}{0.000000in}}{%
\pgfpathmoveto{\pgfqpoint{0.000000in}{0.000000in}}%
\pgfpathlineto{\pgfqpoint{0.000000in}{0.000000in}}%
\pgfusepath{stroke,fill}%
}%
\begin{pgfscope}%
\pgfsys@transformshift{2.816705in}{0.962104in}%
\pgfsys@useobject{currentmarker}{}%
\end{pgfscope}%
\end{pgfscope}%
\begin{pgfscope}%
\pgfpathrectangle{\pgfqpoint{2.816705in}{0.516222in}}{\pgfqpoint{1.962733in}{1.783528in}} %
\pgfusepath{clip}%
\pgfsetroundcap%
\pgfsetroundjoin%
\pgfsetlinewidth{0.803000pt}%
\definecolor{currentstroke}{rgb}{1.000000,1.000000,1.000000}%
\pgfsetstrokecolor{currentstroke}%
\pgfsetdash{}{0pt}%
\pgfpathmoveto{\pgfqpoint{2.816705in}{1.185045in}}%
\pgfpathlineto{\pgfqpoint{4.779438in}{1.185045in}}%
\pgfusepath{stroke}%
\end{pgfscope}%
\begin{pgfscope}%
\pgfsetbuttcap%
\pgfsetroundjoin%
\definecolor{currentfill}{rgb}{0.150000,0.150000,0.150000}%
\pgfsetfillcolor{currentfill}%
\pgfsetlinewidth{0.803000pt}%
\definecolor{currentstroke}{rgb}{0.150000,0.150000,0.150000}%
\pgfsetstrokecolor{currentstroke}%
\pgfsetdash{}{0pt}%
\pgfsys@defobject{currentmarker}{\pgfqpoint{0.000000in}{0.000000in}}{\pgfqpoint{0.000000in}{0.000000in}}{%
\pgfpathmoveto{\pgfqpoint{0.000000in}{0.000000in}}%
\pgfpathlineto{\pgfqpoint{0.000000in}{0.000000in}}%
\pgfusepath{stroke,fill}%
}%
\begin{pgfscope}%
\pgfsys@transformshift{2.816705in}{1.185045in}%
\pgfsys@useobject{currentmarker}{}%
\end{pgfscope}%
\end{pgfscope}%
\begin{pgfscope}%
\pgfpathrectangle{\pgfqpoint{2.816705in}{0.516222in}}{\pgfqpoint{1.962733in}{1.783528in}} %
\pgfusepath{clip}%
\pgfsetroundcap%
\pgfsetroundjoin%
\pgfsetlinewidth{0.803000pt}%
\definecolor{currentstroke}{rgb}{1.000000,1.000000,1.000000}%
\pgfsetstrokecolor{currentstroke}%
\pgfsetdash{}{0pt}%
\pgfpathmoveto{\pgfqpoint{2.816705in}{1.407986in}}%
\pgfpathlineto{\pgfqpoint{4.779438in}{1.407986in}}%
\pgfusepath{stroke}%
\end{pgfscope}%
\begin{pgfscope}%
\pgfsetbuttcap%
\pgfsetroundjoin%
\definecolor{currentfill}{rgb}{0.150000,0.150000,0.150000}%
\pgfsetfillcolor{currentfill}%
\pgfsetlinewidth{0.803000pt}%
\definecolor{currentstroke}{rgb}{0.150000,0.150000,0.150000}%
\pgfsetstrokecolor{currentstroke}%
\pgfsetdash{}{0pt}%
\pgfsys@defobject{currentmarker}{\pgfqpoint{0.000000in}{0.000000in}}{\pgfqpoint{0.000000in}{0.000000in}}{%
\pgfpathmoveto{\pgfqpoint{0.000000in}{0.000000in}}%
\pgfpathlineto{\pgfqpoint{0.000000in}{0.000000in}}%
\pgfusepath{stroke,fill}%
}%
\begin{pgfscope}%
\pgfsys@transformshift{2.816705in}{1.407986in}%
\pgfsys@useobject{currentmarker}{}%
\end{pgfscope}%
\end{pgfscope}%
\begin{pgfscope}%
\pgfpathrectangle{\pgfqpoint{2.816705in}{0.516222in}}{\pgfqpoint{1.962733in}{1.783528in}} %
\pgfusepath{clip}%
\pgfsetroundcap%
\pgfsetroundjoin%
\pgfsetlinewidth{0.803000pt}%
\definecolor{currentstroke}{rgb}{1.000000,1.000000,1.000000}%
\pgfsetstrokecolor{currentstroke}%
\pgfsetdash{}{0pt}%
\pgfpathmoveto{\pgfqpoint{2.816705in}{1.630927in}}%
\pgfpathlineto{\pgfqpoint{4.779438in}{1.630927in}}%
\pgfusepath{stroke}%
\end{pgfscope}%
\begin{pgfscope}%
\pgfsetbuttcap%
\pgfsetroundjoin%
\definecolor{currentfill}{rgb}{0.150000,0.150000,0.150000}%
\pgfsetfillcolor{currentfill}%
\pgfsetlinewidth{0.803000pt}%
\definecolor{currentstroke}{rgb}{0.150000,0.150000,0.150000}%
\pgfsetstrokecolor{currentstroke}%
\pgfsetdash{}{0pt}%
\pgfsys@defobject{currentmarker}{\pgfqpoint{0.000000in}{0.000000in}}{\pgfqpoint{0.000000in}{0.000000in}}{%
\pgfpathmoveto{\pgfqpoint{0.000000in}{0.000000in}}%
\pgfpathlineto{\pgfqpoint{0.000000in}{0.000000in}}%
\pgfusepath{stroke,fill}%
}%
\begin{pgfscope}%
\pgfsys@transformshift{2.816705in}{1.630927in}%
\pgfsys@useobject{currentmarker}{}%
\end{pgfscope}%
\end{pgfscope}%
\begin{pgfscope}%
\pgfpathrectangle{\pgfqpoint{2.816705in}{0.516222in}}{\pgfqpoint{1.962733in}{1.783528in}} %
\pgfusepath{clip}%
\pgfsetroundcap%
\pgfsetroundjoin%
\pgfsetlinewidth{0.803000pt}%
\definecolor{currentstroke}{rgb}{1.000000,1.000000,1.000000}%
\pgfsetstrokecolor{currentstroke}%
\pgfsetdash{}{0pt}%
\pgfpathmoveto{\pgfqpoint{2.816705in}{1.853868in}}%
\pgfpathlineto{\pgfqpoint{4.779438in}{1.853868in}}%
\pgfusepath{stroke}%
\end{pgfscope}%
\begin{pgfscope}%
\pgfsetbuttcap%
\pgfsetroundjoin%
\definecolor{currentfill}{rgb}{0.150000,0.150000,0.150000}%
\pgfsetfillcolor{currentfill}%
\pgfsetlinewidth{0.803000pt}%
\definecolor{currentstroke}{rgb}{0.150000,0.150000,0.150000}%
\pgfsetstrokecolor{currentstroke}%
\pgfsetdash{}{0pt}%
\pgfsys@defobject{currentmarker}{\pgfqpoint{0.000000in}{0.000000in}}{\pgfqpoint{0.000000in}{0.000000in}}{%
\pgfpathmoveto{\pgfqpoint{0.000000in}{0.000000in}}%
\pgfpathlineto{\pgfqpoint{0.000000in}{0.000000in}}%
\pgfusepath{stroke,fill}%
}%
\begin{pgfscope}%
\pgfsys@transformshift{2.816705in}{1.853868in}%
\pgfsys@useobject{currentmarker}{}%
\end{pgfscope}%
\end{pgfscope}%
\begin{pgfscope}%
\pgfpathrectangle{\pgfqpoint{2.816705in}{0.516222in}}{\pgfqpoint{1.962733in}{1.783528in}} %
\pgfusepath{clip}%
\pgfsetroundcap%
\pgfsetroundjoin%
\pgfsetlinewidth{0.803000pt}%
\definecolor{currentstroke}{rgb}{1.000000,1.000000,1.000000}%
\pgfsetstrokecolor{currentstroke}%
\pgfsetdash{}{0pt}%
\pgfpathmoveto{\pgfqpoint{2.816705in}{2.076809in}}%
\pgfpathlineto{\pgfqpoint{4.779438in}{2.076809in}}%
\pgfusepath{stroke}%
\end{pgfscope}%
\begin{pgfscope}%
\pgfsetbuttcap%
\pgfsetroundjoin%
\definecolor{currentfill}{rgb}{0.150000,0.150000,0.150000}%
\pgfsetfillcolor{currentfill}%
\pgfsetlinewidth{0.803000pt}%
\definecolor{currentstroke}{rgb}{0.150000,0.150000,0.150000}%
\pgfsetstrokecolor{currentstroke}%
\pgfsetdash{}{0pt}%
\pgfsys@defobject{currentmarker}{\pgfqpoint{0.000000in}{0.000000in}}{\pgfqpoint{0.000000in}{0.000000in}}{%
\pgfpathmoveto{\pgfqpoint{0.000000in}{0.000000in}}%
\pgfpathlineto{\pgfqpoint{0.000000in}{0.000000in}}%
\pgfusepath{stroke,fill}%
}%
\begin{pgfscope}%
\pgfsys@transformshift{2.816705in}{2.076809in}%
\pgfsys@useobject{currentmarker}{}%
\end{pgfscope}%
\end{pgfscope}%
\begin{pgfscope}%
\pgfpathrectangle{\pgfqpoint{2.816705in}{0.516222in}}{\pgfqpoint{1.962733in}{1.783528in}} %
\pgfusepath{clip}%
\pgfsetroundcap%
\pgfsetroundjoin%
\pgfsetlinewidth{0.803000pt}%
\definecolor{currentstroke}{rgb}{1.000000,1.000000,1.000000}%
\pgfsetstrokecolor{currentstroke}%
\pgfsetdash{}{0pt}%
\pgfpathmoveto{\pgfqpoint{2.816705in}{2.299750in}}%
\pgfpathlineto{\pgfqpoint{4.779438in}{2.299750in}}%
\pgfusepath{stroke}%
\end{pgfscope}%
\begin{pgfscope}%
\pgfsetbuttcap%
\pgfsetroundjoin%
\definecolor{currentfill}{rgb}{0.150000,0.150000,0.150000}%
\pgfsetfillcolor{currentfill}%
\pgfsetlinewidth{0.803000pt}%
\definecolor{currentstroke}{rgb}{0.150000,0.150000,0.150000}%
\pgfsetstrokecolor{currentstroke}%
\pgfsetdash{}{0pt}%
\pgfsys@defobject{currentmarker}{\pgfqpoint{0.000000in}{0.000000in}}{\pgfqpoint{0.000000in}{0.000000in}}{%
\pgfpathmoveto{\pgfqpoint{0.000000in}{0.000000in}}%
\pgfpathlineto{\pgfqpoint{0.000000in}{0.000000in}}%
\pgfusepath{stroke,fill}%
}%
\begin{pgfscope}%
\pgfsys@transformshift{2.816705in}{2.299750in}%
\pgfsys@useobject{currentmarker}{}%
\end{pgfscope}%
\end{pgfscope}%
\begin{pgfscope}%
\pgfpathrectangle{\pgfqpoint{2.816705in}{0.516222in}}{\pgfqpoint{1.962733in}{1.783528in}} %
\pgfusepath{clip}%
\pgfsetbuttcap%
\pgfsetroundjoin%
\definecolor{currentfill}{rgb}{0.298039,0.447059,0.690196}%
\pgfsetfillcolor{currentfill}%
\pgfsetlinewidth{0.240900pt}%
\definecolor{currentstroke}{rgb}{1.000000,1.000000,1.000000}%
\pgfsetstrokecolor{currentstroke}%
\pgfsetdash{}{0pt}%
\pgfpathmoveto{\pgfqpoint{3.826110in}{1.153989in}}%
\pgfpathcurveto{\pgfqpoint{3.834346in}{1.153989in}}{\pgfqpoint{3.842247in}{1.157261in}}{\pgfqpoint{3.848070in}{1.163085in}}%
\pgfpathcurveto{\pgfqpoint{3.853894in}{1.168909in}}{\pgfqpoint{3.857167in}{1.176809in}}{\pgfqpoint{3.857167in}{1.185045in}}%
\pgfpathcurveto{\pgfqpoint{3.857167in}{1.193281in}}{\pgfqpoint{3.853894in}{1.201181in}}{\pgfqpoint{3.848070in}{1.207005in}}%
\pgfpathcurveto{\pgfqpoint{3.842247in}{1.212829in}}{\pgfqpoint{3.834346in}{1.216102in}}{\pgfqpoint{3.826110in}{1.216102in}}%
\pgfpathcurveto{\pgfqpoint{3.817874in}{1.216102in}}{\pgfqpoint{3.809974in}{1.212829in}}{\pgfqpoint{3.804150in}{1.207005in}}%
\pgfpathcurveto{\pgfqpoint{3.798326in}{1.201181in}}{\pgfqpoint{3.795054in}{1.193281in}}{\pgfqpoint{3.795054in}{1.185045in}}%
\pgfpathcurveto{\pgfqpoint{3.795054in}{1.176809in}}{\pgfqpoint{3.798326in}{1.168909in}}{\pgfqpoint{3.804150in}{1.163085in}}%
\pgfpathcurveto{\pgfqpoint{3.809974in}{1.157261in}}{\pgfqpoint{3.817874in}{1.153989in}}{\pgfqpoint{3.826110in}{1.153989in}}%
\pgfpathclose%
\pgfusepath{stroke,fill}%
\end{pgfscope}%
\begin{pgfscope}%
\pgfpathrectangle{\pgfqpoint{2.816705in}{0.516222in}}{\pgfqpoint{1.962733in}{1.783528in}} %
\pgfusepath{clip}%
\pgfsetbuttcap%
\pgfsetroundjoin%
\definecolor{currentfill}{rgb}{0.298039,0.447059,0.690196}%
\pgfsetfillcolor{currentfill}%
\pgfsetlinewidth{0.240900pt}%
\definecolor{currentstroke}{rgb}{1.000000,1.000000,1.000000}%
\pgfsetstrokecolor{currentstroke}%
\pgfsetdash{}{0pt}%
\pgfpathmoveto{\pgfqpoint{3.657876in}{0.846330in}}%
\pgfpathcurveto{\pgfqpoint{3.666112in}{0.846330in}}{\pgfqpoint{3.674012in}{0.849602in}}{\pgfqpoint{3.679836in}{0.855426in}}%
\pgfpathcurveto{\pgfqpoint{3.685660in}{0.861250in}}{\pgfqpoint{3.688932in}{0.869150in}}{\pgfqpoint{3.688932in}{0.877387in}}%
\pgfpathcurveto{\pgfqpoint{3.688932in}{0.885623in}}{\pgfqpoint{3.685660in}{0.893523in}}{\pgfqpoint{3.679836in}{0.899347in}}%
\pgfpathcurveto{\pgfqpoint{3.674012in}{0.905171in}}{\pgfqpoint{3.666112in}{0.908443in}}{\pgfqpoint{3.657876in}{0.908443in}}%
\pgfpathcurveto{\pgfqpoint{3.649640in}{0.908443in}}{\pgfqpoint{3.641740in}{0.905171in}}{\pgfqpoint{3.635916in}{0.899347in}}%
\pgfpathcurveto{\pgfqpoint{3.630092in}{0.893523in}}{\pgfqpoint{3.626819in}{0.885623in}}{\pgfqpoint{3.626819in}{0.877387in}}%
\pgfpathcurveto{\pgfqpoint{3.626819in}{0.869150in}}{\pgfqpoint{3.630092in}{0.861250in}}{\pgfqpoint{3.635916in}{0.855426in}}%
\pgfpathcurveto{\pgfqpoint{3.641740in}{0.849602in}}{\pgfqpoint{3.649640in}{0.846330in}}{\pgfqpoint{3.657876in}{0.846330in}}%
\pgfpathclose%
\pgfusepath{stroke,fill}%
\end{pgfscope}%
\begin{pgfscope}%
\pgfpathrectangle{\pgfqpoint{2.816705in}{0.516222in}}{\pgfqpoint{1.962733in}{1.783528in}} %
\pgfusepath{clip}%
\pgfsetbuttcap%
\pgfsetroundjoin%
\definecolor{currentfill}{rgb}{0.298039,0.447059,0.690196}%
\pgfsetfillcolor{currentfill}%
\pgfsetlinewidth{0.240900pt}%
\definecolor{currentstroke}{rgb}{1.000000,1.000000,1.000000}%
\pgfsetstrokecolor{currentstroke}%
\pgfsetdash{}{0pt}%
\pgfpathmoveto{\pgfqpoint{3.713954in}{1.318965in}}%
\pgfpathcurveto{\pgfqpoint{3.722190in}{1.318965in}}{\pgfqpoint{3.730090in}{1.322237in}}{\pgfqpoint{3.735914in}{1.328061in}}%
\pgfpathcurveto{\pgfqpoint{3.741738in}{1.333885in}}{\pgfqpoint{3.745011in}{1.341785in}}{\pgfqpoint{3.745011in}{1.350021in}}%
\pgfpathcurveto{\pgfqpoint{3.745011in}{1.358258in}}{\pgfqpoint{3.741738in}{1.366158in}}{\pgfqpoint{3.735914in}{1.371982in}}%
\pgfpathcurveto{\pgfqpoint{3.730090in}{1.377806in}}{\pgfqpoint{3.722190in}{1.381078in}}{\pgfqpoint{3.713954in}{1.381078in}}%
\pgfpathcurveto{\pgfqpoint{3.705718in}{1.381078in}}{\pgfqpoint{3.697818in}{1.377806in}}{\pgfqpoint{3.691994in}{1.371982in}}%
\pgfpathcurveto{\pgfqpoint{3.686170in}{1.366158in}}{\pgfqpoint{3.682898in}{1.358258in}}{\pgfqpoint{3.682898in}{1.350021in}}%
\pgfpathcurveto{\pgfqpoint{3.682898in}{1.341785in}}{\pgfqpoint{3.686170in}{1.333885in}}{\pgfqpoint{3.691994in}{1.328061in}}%
\pgfpathcurveto{\pgfqpoint{3.697818in}{1.322237in}}{\pgfqpoint{3.705718in}{1.318965in}}{\pgfqpoint{3.713954in}{1.318965in}}%
\pgfpathclose%
\pgfusepath{stroke,fill}%
\end{pgfscope}%
\begin{pgfscope}%
\pgfpathrectangle{\pgfqpoint{2.816705in}{0.516222in}}{\pgfqpoint{1.962733in}{1.783528in}} %
\pgfusepath{clip}%
\pgfsetbuttcap%
\pgfsetroundjoin%
\definecolor{currentfill}{rgb}{0.298039,0.447059,0.690196}%
\pgfsetfillcolor{currentfill}%
\pgfsetlinewidth{0.240900pt}%
\definecolor{currentstroke}{rgb}{1.000000,1.000000,1.000000}%
\pgfsetstrokecolor{currentstroke}%
\pgfsetdash{}{0pt}%
\pgfpathmoveto{\pgfqpoint{4.162579in}{1.666753in}}%
\pgfpathcurveto{\pgfqpoint{4.170815in}{1.666753in}}{\pgfqpoint{4.178715in}{1.670025in}}{\pgfqpoint{4.184539in}{1.675849in}}%
\pgfpathcurveto{\pgfqpoint{4.190363in}{1.681673in}}{\pgfqpoint{4.193635in}{1.689573in}}{\pgfqpoint{4.193635in}{1.697809in}}%
\pgfpathcurveto{\pgfqpoint{4.193635in}{1.706046in}}{\pgfqpoint{4.190363in}{1.713946in}}{\pgfqpoint{4.184539in}{1.719770in}}%
\pgfpathcurveto{\pgfqpoint{4.178715in}{1.725594in}}{\pgfqpoint{4.170815in}{1.728866in}}{\pgfqpoint{4.162579in}{1.728866in}}%
\pgfpathcurveto{\pgfqpoint{4.154342in}{1.728866in}}{\pgfqpoint{4.146442in}{1.725594in}}{\pgfqpoint{4.140618in}{1.719770in}}%
\pgfpathcurveto{\pgfqpoint{4.134794in}{1.713946in}}{\pgfqpoint{4.131522in}{1.706046in}}{\pgfqpoint{4.131522in}{1.697809in}}%
\pgfpathcurveto{\pgfqpoint{4.131522in}{1.689573in}}{\pgfqpoint{4.134794in}{1.681673in}}{\pgfqpoint{4.140618in}{1.675849in}}%
\pgfpathcurveto{\pgfqpoint{4.146442in}{1.670025in}}{\pgfqpoint{4.154342in}{1.666753in}}{\pgfqpoint{4.162579in}{1.666753in}}%
\pgfpathclose%
\pgfusepath{stroke,fill}%
\end{pgfscope}%
\begin{pgfscope}%
\pgfpathrectangle{\pgfqpoint{2.816705in}{0.516222in}}{\pgfqpoint{1.962733in}{1.783528in}} %
\pgfusepath{clip}%
\pgfsetbuttcap%
\pgfsetroundjoin%
\definecolor{currentfill}{rgb}{0.298039,0.447059,0.690196}%
\pgfsetfillcolor{currentfill}%
\pgfsetlinewidth{0.240900pt}%
\definecolor{currentstroke}{rgb}{1.000000,1.000000,1.000000}%
\pgfsetstrokecolor{currentstroke}%
\pgfsetdash{}{0pt}%
\pgfpathmoveto{\pgfqpoint{3.601798in}{0.980095in}}%
\pgfpathcurveto{\pgfqpoint{3.610034in}{0.980095in}}{\pgfqpoint{3.617934in}{0.983367in}}{\pgfqpoint{3.623758in}{0.989191in}}%
\pgfpathcurveto{\pgfqpoint{3.629582in}{0.995015in}}{\pgfqpoint{3.632854in}{1.002915in}}{\pgfqpoint{3.632854in}{1.011151in}}%
\pgfpathcurveto{\pgfqpoint{3.632854in}{1.019387in}}{\pgfqpoint{3.629582in}{1.027288in}}{\pgfqpoint{3.623758in}{1.033111in}}%
\pgfpathcurveto{\pgfqpoint{3.617934in}{1.038935in}}{\pgfqpoint{3.610034in}{1.042208in}}{\pgfqpoint{3.601798in}{1.042208in}}%
\pgfpathcurveto{\pgfqpoint{3.593562in}{1.042208in}}{\pgfqpoint{3.585662in}{1.038935in}}{\pgfqpoint{3.579838in}{1.033111in}}%
\pgfpathcurveto{\pgfqpoint{3.574014in}{1.027288in}}{\pgfqpoint{3.570741in}{1.019387in}}{\pgfqpoint{3.570741in}{1.011151in}}%
\pgfpathcurveto{\pgfqpoint{3.570741in}{1.002915in}}{\pgfqpoint{3.574014in}{0.995015in}}{\pgfqpoint{3.579838in}{0.989191in}}%
\pgfpathcurveto{\pgfqpoint{3.585662in}{0.983367in}}{\pgfqpoint{3.593562in}{0.980095in}}{\pgfqpoint{3.601798in}{0.980095in}}%
\pgfpathclose%
\pgfusepath{stroke,fill}%
\end{pgfscope}%
\begin{pgfscope}%
\pgfpathrectangle{\pgfqpoint{2.816705in}{0.516222in}}{\pgfqpoint{1.962733in}{1.783528in}} %
\pgfusepath{clip}%
\pgfsetbuttcap%
\pgfsetroundjoin%
\definecolor{currentfill}{rgb}{0.298039,0.447059,0.690196}%
\pgfsetfillcolor{currentfill}%
\pgfsetlinewidth{0.240900pt}%
\definecolor{currentstroke}{rgb}{1.000000,1.000000,1.000000}%
\pgfsetstrokecolor{currentstroke}%
\pgfsetdash{}{0pt}%
\pgfpathmoveto{\pgfqpoint{4.218657in}{2.036835in}}%
\pgfpathcurveto{\pgfqpoint{4.226893in}{2.036835in}}{\pgfqpoint{4.234793in}{2.040107in}}{\pgfqpoint{4.240617in}{2.045931in}}%
\pgfpathcurveto{\pgfqpoint{4.246441in}{2.051755in}}{\pgfqpoint{4.249713in}{2.059655in}}{\pgfqpoint{4.249713in}{2.067891in}}%
\pgfpathcurveto{\pgfqpoint{4.249713in}{2.076128in}}{\pgfqpoint{4.246441in}{2.084028in}}{\pgfqpoint{4.240617in}{2.089852in}}%
\pgfpathcurveto{\pgfqpoint{4.234793in}{2.095676in}}{\pgfqpoint{4.226893in}{2.098948in}}{\pgfqpoint{4.218657in}{2.098948in}}%
\pgfpathcurveto{\pgfqpoint{4.210420in}{2.098948in}}{\pgfqpoint{4.202520in}{2.095676in}}{\pgfqpoint{4.196696in}{2.089852in}}%
\pgfpathcurveto{\pgfqpoint{4.190873in}{2.084028in}}{\pgfqpoint{4.187600in}{2.076128in}}{\pgfqpoint{4.187600in}{2.067891in}}%
\pgfpathcurveto{\pgfqpoint{4.187600in}{2.059655in}}{\pgfqpoint{4.190873in}{2.051755in}}{\pgfqpoint{4.196696in}{2.045931in}}%
\pgfpathcurveto{\pgfqpoint{4.202520in}{2.040107in}}{\pgfqpoint{4.210420in}{2.036835in}}{\pgfqpoint{4.218657in}{2.036835in}}%
\pgfpathclose%
\pgfusepath{stroke,fill}%
\end{pgfscope}%
\begin{pgfscope}%
\pgfpathrectangle{\pgfqpoint{2.816705in}{0.516222in}}{\pgfqpoint{1.962733in}{1.783528in}} %
\pgfusepath{clip}%
\pgfsetbuttcap%
\pgfsetroundjoin%
\definecolor{currentfill}{rgb}{0.298039,0.447059,0.690196}%
\pgfsetfillcolor{currentfill}%
\pgfsetlinewidth{0.240900pt}%
\definecolor{currentstroke}{rgb}{1.000000,1.000000,1.000000}%
\pgfsetstrokecolor{currentstroke}%
\pgfsetdash{}{0pt}%
\pgfpathmoveto{\pgfqpoint{3.601798in}{0.739318in}}%
\pgfpathcurveto{\pgfqpoint{3.610034in}{0.739318in}}{\pgfqpoint{3.617934in}{0.742591in}}{\pgfqpoint{3.623758in}{0.748415in}}%
\pgfpathcurveto{\pgfqpoint{3.629582in}{0.754239in}}{\pgfqpoint{3.632854in}{0.762139in}}{\pgfqpoint{3.632854in}{0.770375in}}%
\pgfpathcurveto{\pgfqpoint{3.632854in}{0.778611in}}{\pgfqpoint{3.629582in}{0.786511in}}{\pgfqpoint{3.623758in}{0.792335in}}%
\pgfpathcurveto{\pgfqpoint{3.617934in}{0.798159in}}{\pgfqpoint{3.610034in}{0.801431in}}{\pgfqpoint{3.601798in}{0.801431in}}%
\pgfpathcurveto{\pgfqpoint{3.593562in}{0.801431in}}{\pgfqpoint{3.585662in}{0.798159in}}{\pgfqpoint{3.579838in}{0.792335in}}%
\pgfpathcurveto{\pgfqpoint{3.574014in}{0.786511in}}{\pgfqpoint{3.570741in}{0.778611in}}{\pgfqpoint{3.570741in}{0.770375in}}%
\pgfpathcurveto{\pgfqpoint{3.570741in}{0.762139in}}{\pgfqpoint{3.574014in}{0.754239in}}{\pgfqpoint{3.579838in}{0.748415in}}%
\pgfpathcurveto{\pgfqpoint{3.585662in}{0.742591in}}{\pgfqpoint{3.593562in}{0.739318in}}{\pgfqpoint{3.601798in}{0.739318in}}%
\pgfpathclose%
\pgfusepath{stroke,fill}%
\end{pgfscope}%
\begin{pgfscope}%
\pgfpathrectangle{\pgfqpoint{2.816705in}{0.516222in}}{\pgfqpoint{1.962733in}{1.783528in}} %
\pgfusepath{clip}%
\pgfsetbuttcap%
\pgfsetroundjoin%
\definecolor{currentfill}{rgb}{0.298039,0.447059,0.690196}%
\pgfsetfillcolor{currentfill}%
\pgfsetlinewidth{0.240900pt}%
\definecolor{currentstroke}{rgb}{1.000000,1.000000,1.000000}%
\pgfsetstrokecolor{currentstroke}%
\pgfsetdash{}{0pt}%
\pgfpathmoveto{\pgfqpoint{4.442969in}{1.760388in}}%
\pgfpathcurveto{\pgfqpoint{4.451205in}{1.760388in}}{\pgfqpoint{4.459105in}{1.763660in}}{\pgfqpoint{4.464929in}{1.769484in}}%
\pgfpathcurveto{\pgfqpoint{4.470753in}{1.775308in}}{\pgfqpoint{4.474026in}{1.783208in}}{\pgfqpoint{4.474026in}{1.791445in}}%
\pgfpathcurveto{\pgfqpoint{4.474026in}{1.799681in}}{\pgfqpoint{4.470753in}{1.807581in}}{\pgfqpoint{4.464929in}{1.813405in}}%
\pgfpathcurveto{\pgfqpoint{4.459105in}{1.819229in}}{\pgfqpoint{4.451205in}{1.822501in}}{\pgfqpoint{4.442969in}{1.822501in}}%
\pgfpathcurveto{\pgfqpoint{4.434733in}{1.822501in}}{\pgfqpoint{4.426833in}{1.819229in}}{\pgfqpoint{4.421009in}{1.813405in}}%
\pgfpathcurveto{\pgfqpoint{4.415185in}{1.807581in}}{\pgfqpoint{4.411913in}{1.799681in}}{\pgfqpoint{4.411913in}{1.791445in}}%
\pgfpathcurveto{\pgfqpoint{4.411913in}{1.783208in}}{\pgfqpoint{4.415185in}{1.775308in}}{\pgfqpoint{4.421009in}{1.769484in}}%
\pgfpathcurveto{\pgfqpoint{4.426833in}{1.763660in}}{\pgfqpoint{4.434733in}{1.760388in}}{\pgfqpoint{4.442969in}{1.760388in}}%
\pgfpathclose%
\pgfusepath{stroke,fill}%
\end{pgfscope}%
\begin{pgfscope}%
\pgfpathrectangle{\pgfqpoint{2.816705in}{0.516222in}}{\pgfqpoint{1.962733in}{1.783528in}} %
\pgfusepath{clip}%
\pgfsetbuttcap%
\pgfsetroundjoin%
\definecolor{currentfill}{rgb}{0.298039,0.447059,0.690196}%
\pgfsetfillcolor{currentfill}%
\pgfsetlinewidth{0.240900pt}%
\definecolor{currentstroke}{rgb}{1.000000,1.000000,1.000000}%
\pgfsetstrokecolor{currentstroke}%
\pgfsetdash{}{0pt}%
\pgfpathmoveto{\pgfqpoint{3.097095in}{0.837412in}}%
\pgfpathcurveto{\pgfqpoint{3.105332in}{0.837412in}}{\pgfqpoint{3.113232in}{0.840685in}}{\pgfqpoint{3.119055in}{0.846509in}}%
\pgfpathcurveto{\pgfqpoint{3.124879in}{0.852333in}}{\pgfqpoint{3.128152in}{0.860233in}}{\pgfqpoint{3.128152in}{0.868469in}}%
\pgfpathcurveto{\pgfqpoint{3.128152in}{0.876705in}}{\pgfqpoint{3.124879in}{0.884605in}}{\pgfqpoint{3.119055in}{0.890429in}}%
\pgfpathcurveto{\pgfqpoint{3.113232in}{0.896253in}}{\pgfqpoint{3.105332in}{0.899525in}}{\pgfqpoint{3.097095in}{0.899525in}}%
\pgfpathcurveto{\pgfqpoint{3.088859in}{0.899525in}}{\pgfqpoint{3.080959in}{0.896253in}}{\pgfqpoint{3.075135in}{0.890429in}}%
\pgfpathcurveto{\pgfqpoint{3.069311in}{0.884605in}}{\pgfqpoint{3.066039in}{0.876705in}}{\pgfqpoint{3.066039in}{0.868469in}}%
\pgfpathcurveto{\pgfqpoint{3.066039in}{0.860233in}}{\pgfqpoint{3.069311in}{0.852333in}}{\pgfqpoint{3.075135in}{0.846509in}}%
\pgfpathcurveto{\pgfqpoint{3.080959in}{0.840685in}}{\pgfqpoint{3.088859in}{0.837412in}}{\pgfqpoint{3.097095in}{0.837412in}}%
\pgfpathclose%
\pgfusepath{stroke,fill}%
\end{pgfscope}%
\begin{pgfscope}%
\pgfpathrectangle{\pgfqpoint{2.816705in}{0.516222in}}{\pgfqpoint{1.962733in}{1.783528in}} %
\pgfusepath{clip}%
\pgfsetbuttcap%
\pgfsetroundjoin%
\definecolor{currentfill}{rgb}{0.298039,0.447059,0.690196}%
\pgfsetfillcolor{currentfill}%
\pgfsetlinewidth{0.240900pt}%
\definecolor{currentstroke}{rgb}{1.000000,1.000000,1.000000}%
\pgfsetstrokecolor{currentstroke}%
\pgfsetdash{}{0pt}%
\pgfpathmoveto{\pgfqpoint{3.657876in}{1.394765in}}%
\pgfpathcurveto{\pgfqpoint{3.666112in}{1.394765in}}{\pgfqpoint{3.674012in}{1.398037in}}{\pgfqpoint{3.679836in}{1.403861in}}%
\pgfpathcurveto{\pgfqpoint{3.685660in}{1.409685in}}{\pgfqpoint{3.688932in}{1.417585in}}{\pgfqpoint{3.688932in}{1.425821in}}%
\pgfpathcurveto{\pgfqpoint{3.688932in}{1.434058in}}{\pgfqpoint{3.685660in}{1.441958in}}{\pgfqpoint{3.679836in}{1.447782in}}%
\pgfpathcurveto{\pgfqpoint{3.674012in}{1.453606in}}{\pgfqpoint{3.666112in}{1.456878in}}{\pgfqpoint{3.657876in}{1.456878in}}%
\pgfpathcurveto{\pgfqpoint{3.649640in}{1.456878in}}{\pgfqpoint{3.641740in}{1.453606in}}{\pgfqpoint{3.635916in}{1.447782in}}%
\pgfpathcurveto{\pgfqpoint{3.630092in}{1.441958in}}{\pgfqpoint{3.626819in}{1.434058in}}{\pgfqpoint{3.626819in}{1.425821in}}%
\pgfpathcurveto{\pgfqpoint{3.626819in}{1.417585in}}{\pgfqpoint{3.630092in}{1.409685in}}{\pgfqpoint{3.635916in}{1.403861in}}%
\pgfpathcurveto{\pgfqpoint{3.641740in}{1.398037in}}{\pgfqpoint{3.649640in}{1.394765in}}{\pgfqpoint{3.657876in}{1.394765in}}%
\pgfpathclose%
\pgfusepath{stroke,fill}%
\end{pgfscope}%
\begin{pgfscope}%
\pgfpathrectangle{\pgfqpoint{2.816705in}{0.516222in}}{\pgfqpoint{1.962733in}{1.783528in}} %
\pgfusepath{clip}%
\pgfsetbuttcap%
\pgfsetroundjoin%
\definecolor{currentfill}{rgb}{0.298039,0.447059,0.690196}%
\pgfsetfillcolor{currentfill}%
\pgfsetlinewidth{0.240900pt}%
\definecolor{currentstroke}{rgb}{1.000000,1.000000,1.000000}%
\pgfsetstrokecolor{currentstroke}%
\pgfsetdash{}{0pt}%
\pgfpathmoveto{\pgfqpoint{3.657876in}{0.806201in}}%
\pgfpathcurveto{\pgfqpoint{3.666112in}{0.806201in}}{\pgfqpoint{3.674012in}{0.809473in}}{\pgfqpoint{3.679836in}{0.815297in}}%
\pgfpathcurveto{\pgfqpoint{3.685660in}{0.821121in}}{\pgfqpoint{3.688932in}{0.829021in}}{\pgfqpoint{3.688932in}{0.837257in}}%
\pgfpathcurveto{\pgfqpoint{3.688932in}{0.845494in}}{\pgfqpoint{3.685660in}{0.853394in}}{\pgfqpoint{3.679836in}{0.859217in}}%
\pgfpathcurveto{\pgfqpoint{3.674012in}{0.865041in}}{\pgfqpoint{3.666112in}{0.868314in}}{\pgfqpoint{3.657876in}{0.868314in}}%
\pgfpathcurveto{\pgfqpoint{3.649640in}{0.868314in}}{\pgfqpoint{3.641740in}{0.865041in}}{\pgfqpoint{3.635916in}{0.859217in}}%
\pgfpathcurveto{\pgfqpoint{3.630092in}{0.853394in}}{\pgfqpoint{3.626819in}{0.845494in}}{\pgfqpoint{3.626819in}{0.837257in}}%
\pgfpathcurveto{\pgfqpoint{3.626819in}{0.829021in}}{\pgfqpoint{3.630092in}{0.821121in}}{\pgfqpoint{3.635916in}{0.815297in}}%
\pgfpathcurveto{\pgfqpoint{3.641740in}{0.809473in}}{\pgfqpoint{3.649640in}{0.806201in}}{\pgfqpoint{3.657876in}{0.806201in}}%
\pgfpathclose%
\pgfusepath{stroke,fill}%
\end{pgfscope}%
\begin{pgfscope}%
\pgfpathrectangle{\pgfqpoint{2.816705in}{0.516222in}}{\pgfqpoint{1.962733in}{1.783528in}} %
\pgfusepath{clip}%
\pgfsetbuttcap%
\pgfsetroundjoin%
\definecolor{currentfill}{rgb}{0.298039,0.447059,0.690196}%
\pgfsetfillcolor{currentfill}%
\pgfsetlinewidth{0.240900pt}%
\definecolor{currentstroke}{rgb}{1.000000,1.000000,1.000000}%
\pgfsetstrokecolor{currentstroke}%
\pgfsetdash{}{0pt}%
\pgfpathmoveto{\pgfqpoint{3.545720in}{0.792824in}}%
\pgfpathcurveto{\pgfqpoint{3.553956in}{0.792824in}}{\pgfqpoint{3.561856in}{0.796097in}}{\pgfqpoint{3.567680in}{0.801921in}}%
\pgfpathcurveto{\pgfqpoint{3.573504in}{0.807744in}}{\pgfqpoint{3.576776in}{0.815644in}}{\pgfqpoint{3.576776in}{0.823881in}}%
\pgfpathcurveto{\pgfqpoint{3.576776in}{0.832117in}}{\pgfqpoint{3.573504in}{0.840017in}}{\pgfqpoint{3.567680in}{0.845841in}}%
\pgfpathcurveto{\pgfqpoint{3.561856in}{0.851665in}}{\pgfqpoint{3.553956in}{0.854937in}}{\pgfqpoint{3.545720in}{0.854937in}}%
\pgfpathcurveto{\pgfqpoint{3.537484in}{0.854937in}}{\pgfqpoint{3.529584in}{0.851665in}}{\pgfqpoint{3.523760in}{0.845841in}}%
\pgfpathcurveto{\pgfqpoint{3.517936in}{0.840017in}}{\pgfqpoint{3.514663in}{0.832117in}}{\pgfqpoint{3.514663in}{0.823881in}}%
\pgfpathcurveto{\pgfqpoint{3.514663in}{0.815644in}}{\pgfqpoint{3.517936in}{0.807744in}}{\pgfqpoint{3.523760in}{0.801921in}}%
\pgfpathcurveto{\pgfqpoint{3.529584in}{0.796097in}}{\pgfqpoint{3.537484in}{0.792824in}}{\pgfqpoint{3.545720in}{0.792824in}}%
\pgfpathclose%
\pgfusepath{stroke,fill}%
\end{pgfscope}%
\begin{pgfscope}%
\pgfpathrectangle{\pgfqpoint{2.816705in}{0.516222in}}{\pgfqpoint{1.962733in}{1.783528in}} %
\pgfusepath{clip}%
\pgfsetbuttcap%
\pgfsetroundjoin%
\definecolor{currentfill}{rgb}{0.298039,0.447059,0.690196}%
\pgfsetfillcolor{currentfill}%
\pgfsetlinewidth{0.240900pt}%
\definecolor{currentstroke}{rgb}{1.000000,1.000000,1.000000}%
\pgfsetstrokecolor{currentstroke}%
\pgfsetdash{}{0pt}%
\pgfpathmoveto{\pgfqpoint{3.601798in}{2.094800in}}%
\pgfpathcurveto{\pgfqpoint{3.610034in}{2.094800in}}{\pgfqpoint{3.617934in}{2.098072in}}{\pgfqpoint{3.623758in}{2.103896in}}%
\pgfpathcurveto{\pgfqpoint{3.629582in}{2.109720in}}{\pgfqpoint{3.632854in}{2.117620in}}{\pgfqpoint{3.632854in}{2.125856in}}%
\pgfpathcurveto{\pgfqpoint{3.632854in}{2.134092in}}{\pgfqpoint{3.629582in}{2.141992in}}{\pgfqpoint{3.623758in}{2.147816in}}%
\pgfpathcurveto{\pgfqpoint{3.617934in}{2.153640in}}{\pgfqpoint{3.610034in}{2.156913in}}{\pgfqpoint{3.601798in}{2.156913in}}%
\pgfpathcurveto{\pgfqpoint{3.593562in}{2.156913in}}{\pgfqpoint{3.585662in}{2.153640in}}{\pgfqpoint{3.579838in}{2.147816in}}%
\pgfpathcurveto{\pgfqpoint{3.574014in}{2.141992in}}{\pgfqpoint{3.570741in}{2.134092in}}{\pgfqpoint{3.570741in}{2.125856in}}%
\pgfpathcurveto{\pgfqpoint{3.570741in}{2.117620in}}{\pgfqpoint{3.574014in}{2.109720in}}{\pgfqpoint{3.579838in}{2.103896in}}%
\pgfpathcurveto{\pgfqpoint{3.585662in}{2.098072in}}{\pgfqpoint{3.593562in}{2.094800in}}{\pgfqpoint{3.601798in}{2.094800in}}%
\pgfpathclose%
\pgfusepath{stroke,fill}%
\end{pgfscope}%
\begin{pgfscope}%
\pgfpathrectangle{\pgfqpoint{2.816705in}{0.516222in}}{\pgfqpoint{1.962733in}{1.783528in}} %
\pgfusepath{clip}%
\pgfsetbuttcap%
\pgfsetroundjoin%
\definecolor{currentfill}{rgb}{0.298039,0.447059,0.690196}%
\pgfsetfillcolor{currentfill}%
\pgfsetlinewidth{0.240900pt}%
\definecolor{currentstroke}{rgb}{1.000000,1.000000,1.000000}%
\pgfsetstrokecolor{currentstroke}%
\pgfsetdash{}{0pt}%
\pgfpathmoveto{\pgfqpoint{2.984939in}{1.033601in}}%
\pgfpathcurveto{\pgfqpoint{2.993175in}{1.033601in}}{\pgfqpoint{3.001075in}{1.036873in}}{\pgfqpoint{3.006899in}{1.042697in}}%
\pgfpathcurveto{\pgfqpoint{3.012723in}{1.048521in}}{\pgfqpoint{3.015996in}{1.056421in}}{\pgfqpoint{3.015996in}{1.064657in}}%
\pgfpathcurveto{\pgfqpoint{3.015996in}{1.072893in}}{\pgfqpoint{3.012723in}{1.080793in}}{\pgfqpoint{3.006899in}{1.086617in}}%
\pgfpathcurveto{\pgfqpoint{3.001075in}{1.092441in}}{\pgfqpoint{2.993175in}{1.095714in}}{\pgfqpoint{2.984939in}{1.095714in}}%
\pgfpathcurveto{\pgfqpoint{2.976703in}{1.095714in}}{\pgfqpoint{2.968803in}{1.092441in}}{\pgfqpoint{2.962979in}{1.086617in}}%
\pgfpathcurveto{\pgfqpoint{2.957155in}{1.080793in}}{\pgfqpoint{2.953883in}{1.072893in}}{\pgfqpoint{2.953883in}{1.064657in}}%
\pgfpathcurveto{\pgfqpoint{2.953883in}{1.056421in}}{\pgfqpoint{2.957155in}{1.048521in}}{\pgfqpoint{2.962979in}{1.042697in}}%
\pgfpathcurveto{\pgfqpoint{2.968803in}{1.036873in}}{\pgfqpoint{2.976703in}{1.033601in}}{\pgfqpoint{2.984939in}{1.033601in}}%
\pgfpathclose%
\pgfusepath{stroke,fill}%
\end{pgfscope}%
\begin{pgfscope}%
\pgfpathrectangle{\pgfqpoint{2.816705in}{0.516222in}}{\pgfqpoint{1.962733in}{1.783528in}} %
\pgfusepath{clip}%
\pgfsetbuttcap%
\pgfsetroundjoin%
\definecolor{currentfill}{rgb}{0.298039,0.447059,0.690196}%
\pgfsetfillcolor{currentfill}%
\pgfsetlinewidth{0.240900pt}%
\definecolor{currentstroke}{rgb}{1.000000,1.000000,1.000000}%
\pgfsetstrokecolor{currentstroke}%
\pgfsetdash{}{0pt}%
\pgfpathmoveto{\pgfqpoint{3.938266in}{1.457188in}}%
\pgfpathcurveto{\pgfqpoint{3.946503in}{1.457188in}}{\pgfqpoint{3.954403in}{1.460461in}}{\pgfqpoint{3.960227in}{1.466285in}}%
\pgfpathcurveto{\pgfqpoint{3.966051in}{1.472109in}}{\pgfqpoint{3.969323in}{1.480009in}}{\pgfqpoint{3.969323in}{1.488245in}}%
\pgfpathcurveto{\pgfqpoint{3.969323in}{1.496481in}}{\pgfqpoint{3.966051in}{1.504381in}}{\pgfqpoint{3.960227in}{1.510205in}}%
\pgfpathcurveto{\pgfqpoint{3.954403in}{1.516029in}}{\pgfqpoint{3.946503in}{1.519301in}}{\pgfqpoint{3.938266in}{1.519301in}}%
\pgfpathcurveto{\pgfqpoint{3.930030in}{1.519301in}}{\pgfqpoint{3.922130in}{1.516029in}}{\pgfqpoint{3.916306in}{1.510205in}}%
\pgfpathcurveto{\pgfqpoint{3.910482in}{1.504381in}}{\pgfqpoint{3.907210in}{1.496481in}}{\pgfqpoint{3.907210in}{1.488245in}}%
\pgfpathcurveto{\pgfqpoint{3.907210in}{1.480009in}}{\pgfqpoint{3.910482in}{1.472109in}}{\pgfqpoint{3.916306in}{1.466285in}}%
\pgfpathcurveto{\pgfqpoint{3.922130in}{1.460461in}}{\pgfqpoint{3.930030in}{1.457188in}}{\pgfqpoint{3.938266in}{1.457188in}}%
\pgfpathclose%
\pgfusepath{stroke,fill}%
\end{pgfscope}%
\begin{pgfscope}%
\pgfpathrectangle{\pgfqpoint{2.816705in}{0.516222in}}{\pgfqpoint{1.962733in}{1.783528in}} %
\pgfusepath{clip}%
\pgfsetbuttcap%
\pgfsetroundjoin%
\definecolor{currentfill}{rgb}{0.298039,0.447059,0.690196}%
\pgfsetfillcolor{currentfill}%
\pgfsetlinewidth{0.240900pt}%
\definecolor{currentstroke}{rgb}{1.000000,1.000000,1.000000}%
\pgfsetstrokecolor{currentstroke}%
\pgfsetdash{}{0pt}%
\pgfpathmoveto{\pgfqpoint{3.882188in}{1.372471in}}%
\pgfpathcurveto{\pgfqpoint{3.890425in}{1.372471in}}{\pgfqpoint{3.898325in}{1.375743in}}{\pgfqpoint{3.904149in}{1.381567in}}%
\pgfpathcurveto{\pgfqpoint{3.909972in}{1.387391in}}{\pgfqpoint{3.913245in}{1.395291in}}{\pgfqpoint{3.913245in}{1.403527in}}%
\pgfpathcurveto{\pgfqpoint{3.913245in}{1.411764in}}{\pgfqpoint{3.909972in}{1.419664in}}{\pgfqpoint{3.904149in}{1.425488in}}%
\pgfpathcurveto{\pgfqpoint{3.898325in}{1.431311in}}{\pgfqpoint{3.890425in}{1.434584in}}{\pgfqpoint{3.882188in}{1.434584in}}%
\pgfpathcurveto{\pgfqpoint{3.873952in}{1.434584in}}{\pgfqpoint{3.866052in}{1.431311in}}{\pgfqpoint{3.860228in}{1.425488in}}%
\pgfpathcurveto{\pgfqpoint{3.854404in}{1.419664in}}{\pgfqpoint{3.851132in}{1.411764in}}{\pgfqpoint{3.851132in}{1.403527in}}%
\pgfpathcurveto{\pgfqpoint{3.851132in}{1.395291in}}{\pgfqpoint{3.854404in}{1.387391in}}{\pgfqpoint{3.860228in}{1.381567in}}%
\pgfpathcurveto{\pgfqpoint{3.866052in}{1.375743in}}{\pgfqpoint{3.873952in}{1.372471in}}{\pgfqpoint{3.882188in}{1.372471in}}%
\pgfpathclose%
\pgfusepath{stroke,fill}%
\end{pgfscope}%
\begin{pgfscope}%
\pgfpathrectangle{\pgfqpoint{2.816705in}{0.516222in}}{\pgfqpoint{1.962733in}{1.783528in}} %
\pgfusepath{clip}%
\pgfsetbuttcap%
\pgfsetroundjoin%
\definecolor{currentfill}{rgb}{0.298039,0.447059,0.690196}%
\pgfsetfillcolor{currentfill}%
\pgfsetlinewidth{0.240900pt}%
\definecolor{currentstroke}{rgb}{1.000000,1.000000,1.000000}%
\pgfsetstrokecolor{currentstroke}%
\pgfsetdash{}{0pt}%
\pgfpathmoveto{\pgfqpoint{3.433564in}{0.672436in}}%
\pgfpathcurveto{\pgfqpoint{3.441800in}{0.672436in}}{\pgfqpoint{3.449700in}{0.675708in}}{\pgfqpoint{3.455524in}{0.681532in}}%
\pgfpathcurveto{\pgfqpoint{3.461348in}{0.687356in}}{\pgfqpoint{3.464620in}{0.695256in}}{\pgfqpoint{3.464620in}{0.703493in}}%
\pgfpathcurveto{\pgfqpoint{3.464620in}{0.711729in}}{\pgfqpoint{3.461348in}{0.719629in}}{\pgfqpoint{3.455524in}{0.725453in}}%
\pgfpathcurveto{\pgfqpoint{3.449700in}{0.731277in}}{\pgfqpoint{3.441800in}{0.734549in}}{\pgfqpoint{3.433564in}{0.734549in}}%
\pgfpathcurveto{\pgfqpoint{3.425327in}{0.734549in}}{\pgfqpoint{3.417427in}{0.731277in}}{\pgfqpoint{3.411603in}{0.725453in}}%
\pgfpathcurveto{\pgfqpoint{3.405779in}{0.719629in}}{\pgfqpoint{3.402507in}{0.711729in}}{\pgfqpoint{3.402507in}{0.703493in}}%
\pgfpathcurveto{\pgfqpoint{3.402507in}{0.695256in}}{\pgfqpoint{3.405779in}{0.687356in}}{\pgfqpoint{3.411603in}{0.681532in}}%
\pgfpathcurveto{\pgfqpoint{3.417427in}{0.675708in}}{\pgfqpoint{3.425327in}{0.672436in}}{\pgfqpoint{3.433564in}{0.672436in}}%
\pgfpathclose%
\pgfusepath{stroke,fill}%
\end{pgfscope}%
\begin{pgfscope}%
\pgfpathrectangle{\pgfqpoint{2.816705in}{0.516222in}}{\pgfqpoint{1.962733in}{1.783528in}} %
\pgfusepath{clip}%
\pgfsetbuttcap%
\pgfsetroundjoin%
\definecolor{currentfill}{rgb}{0.298039,0.447059,0.690196}%
\pgfsetfillcolor{currentfill}%
\pgfsetlinewidth{0.240900pt}%
\definecolor{currentstroke}{rgb}{1.000000,1.000000,1.000000}%
\pgfsetstrokecolor{currentstroke}%
\pgfsetdash{}{0pt}%
\pgfpathmoveto{\pgfqpoint{3.601798in}{0.895377in}}%
\pgfpathcurveto{\pgfqpoint{3.610034in}{0.895377in}}{\pgfqpoint{3.617934in}{0.898649in}}{\pgfqpoint{3.623758in}{0.904473in}}%
\pgfpathcurveto{\pgfqpoint{3.629582in}{0.910297in}}{\pgfqpoint{3.632854in}{0.918197in}}{\pgfqpoint{3.632854in}{0.926434in}}%
\pgfpathcurveto{\pgfqpoint{3.632854in}{0.934670in}}{\pgfqpoint{3.629582in}{0.942570in}}{\pgfqpoint{3.623758in}{0.948394in}}%
\pgfpathcurveto{\pgfqpoint{3.617934in}{0.954218in}}{\pgfqpoint{3.610034in}{0.957490in}}{\pgfqpoint{3.601798in}{0.957490in}}%
\pgfpathcurveto{\pgfqpoint{3.593562in}{0.957490in}}{\pgfqpoint{3.585662in}{0.954218in}}{\pgfqpoint{3.579838in}{0.948394in}}%
\pgfpathcurveto{\pgfqpoint{3.574014in}{0.942570in}}{\pgfqpoint{3.570741in}{0.934670in}}{\pgfqpoint{3.570741in}{0.926434in}}%
\pgfpathcurveto{\pgfqpoint{3.570741in}{0.918197in}}{\pgfqpoint{3.574014in}{0.910297in}}{\pgfqpoint{3.579838in}{0.904473in}}%
\pgfpathcurveto{\pgfqpoint{3.585662in}{0.898649in}}{\pgfqpoint{3.593562in}{0.895377in}}{\pgfqpoint{3.601798in}{0.895377in}}%
\pgfpathclose%
\pgfusepath{stroke,fill}%
\end{pgfscope}%
\begin{pgfscope}%
\pgfpathrectangle{\pgfqpoint{2.816705in}{0.516222in}}{\pgfqpoint{1.962733in}{1.783528in}} %
\pgfusepath{clip}%
\pgfsetbuttcap%
\pgfsetroundjoin%
\definecolor{currentfill}{rgb}{0.298039,0.447059,0.690196}%
\pgfsetfillcolor{currentfill}%
\pgfsetlinewidth{0.240900pt}%
\definecolor{currentstroke}{rgb}{1.000000,1.000000,1.000000}%
\pgfsetstrokecolor{currentstroke}%
\pgfsetdash{}{0pt}%
\pgfpathmoveto{\pgfqpoint{3.209251in}{0.975636in}}%
\pgfpathcurveto{\pgfqpoint{3.217488in}{0.975636in}}{\pgfqpoint{3.225388in}{0.978908in}}{\pgfqpoint{3.231212in}{0.984732in}}%
\pgfpathcurveto{\pgfqpoint{3.237036in}{0.990556in}}{\pgfqpoint{3.240308in}{0.998456in}}{\pgfqpoint{3.240308in}{1.006692in}}%
\pgfpathcurveto{\pgfqpoint{3.240308in}{1.014929in}}{\pgfqpoint{3.237036in}{1.022829in}}{\pgfqpoint{3.231212in}{1.028653in}}%
\pgfpathcurveto{\pgfqpoint{3.225388in}{1.034477in}}{\pgfqpoint{3.217488in}{1.037749in}}{\pgfqpoint{3.209251in}{1.037749in}}%
\pgfpathcurveto{\pgfqpoint{3.201015in}{1.037749in}}{\pgfqpoint{3.193115in}{1.034477in}}{\pgfqpoint{3.187291in}{1.028653in}}%
\pgfpathcurveto{\pgfqpoint{3.181467in}{1.022829in}}{\pgfqpoint{3.178195in}{1.014929in}}{\pgfqpoint{3.178195in}{1.006692in}}%
\pgfpathcurveto{\pgfqpoint{3.178195in}{0.998456in}}{\pgfqpoint{3.181467in}{0.990556in}}{\pgfqpoint{3.187291in}{0.984732in}}%
\pgfpathcurveto{\pgfqpoint{3.193115in}{0.978908in}}{\pgfqpoint{3.201015in}{0.975636in}}{\pgfqpoint{3.209251in}{0.975636in}}%
\pgfpathclose%
\pgfusepath{stroke,fill}%
\end{pgfscope}%
\begin{pgfscope}%
\pgfpathrectangle{\pgfqpoint{2.816705in}{0.516222in}}{\pgfqpoint{1.962733in}{1.783528in}} %
\pgfusepath{clip}%
\pgfsetbuttcap%
\pgfsetroundjoin%
\definecolor{currentfill}{rgb}{0.298039,0.447059,0.690196}%
\pgfsetfillcolor{currentfill}%
\pgfsetlinewidth{0.240900pt}%
\definecolor{currentstroke}{rgb}{1.000000,1.000000,1.000000}%
\pgfsetstrokecolor{currentstroke}%
\pgfsetdash{}{0pt}%
\pgfpathmoveto{\pgfqpoint{3.545720in}{0.730401in}}%
\pgfpathcurveto{\pgfqpoint{3.553956in}{0.730401in}}{\pgfqpoint{3.561856in}{0.733673in}}{\pgfqpoint{3.567680in}{0.739497in}}%
\pgfpathcurveto{\pgfqpoint{3.573504in}{0.745321in}}{\pgfqpoint{3.576776in}{0.753221in}}{\pgfqpoint{3.576776in}{0.761457in}}%
\pgfpathcurveto{\pgfqpoint{3.576776in}{0.769694in}}{\pgfqpoint{3.573504in}{0.777594in}}{\pgfqpoint{3.567680in}{0.783418in}}%
\pgfpathcurveto{\pgfqpoint{3.561856in}{0.789241in}}{\pgfqpoint{3.553956in}{0.792514in}}{\pgfqpoint{3.545720in}{0.792514in}}%
\pgfpathcurveto{\pgfqpoint{3.537484in}{0.792514in}}{\pgfqpoint{3.529584in}{0.789241in}}{\pgfqpoint{3.523760in}{0.783418in}}%
\pgfpathcurveto{\pgfqpoint{3.517936in}{0.777594in}}{\pgfqpoint{3.514663in}{0.769694in}}{\pgfqpoint{3.514663in}{0.761457in}}%
\pgfpathcurveto{\pgfqpoint{3.514663in}{0.753221in}}{\pgfqpoint{3.517936in}{0.745321in}}{\pgfqpoint{3.523760in}{0.739497in}}%
\pgfpathcurveto{\pgfqpoint{3.529584in}{0.733673in}}{\pgfqpoint{3.537484in}{0.730401in}}{\pgfqpoint{3.545720in}{0.730401in}}%
\pgfpathclose%
\pgfusepath{stroke,fill}%
\end{pgfscope}%
\begin{pgfscope}%
\pgfpathrectangle{\pgfqpoint{2.816705in}{0.516222in}}{\pgfqpoint{1.962733in}{1.783528in}} %
\pgfusepath{clip}%
\pgfsetbuttcap%
\pgfsetroundjoin%
\definecolor{currentfill}{rgb}{0.298039,0.447059,0.690196}%
\pgfsetfillcolor{currentfill}%
\pgfsetlinewidth{0.240900pt}%
\definecolor{currentstroke}{rgb}{1.000000,1.000000,1.000000}%
\pgfsetstrokecolor{currentstroke}%
\pgfsetdash{}{0pt}%
\pgfpathmoveto{\pgfqpoint{3.882188in}{1.956576in}}%
\pgfpathcurveto{\pgfqpoint{3.890425in}{1.956576in}}{\pgfqpoint{3.898325in}{1.959848in}}{\pgfqpoint{3.904149in}{1.965672in}}%
\pgfpathcurveto{\pgfqpoint{3.909972in}{1.971496in}}{\pgfqpoint{3.913245in}{1.979396in}}{\pgfqpoint{3.913245in}{1.987633in}}%
\pgfpathcurveto{\pgfqpoint{3.913245in}{1.995869in}}{\pgfqpoint{3.909972in}{2.003769in}}{\pgfqpoint{3.904149in}{2.009593in}}%
\pgfpathcurveto{\pgfqpoint{3.898325in}{2.015417in}}{\pgfqpoint{3.890425in}{2.018689in}}{\pgfqpoint{3.882188in}{2.018689in}}%
\pgfpathcurveto{\pgfqpoint{3.873952in}{2.018689in}}{\pgfqpoint{3.866052in}{2.015417in}}{\pgfqpoint{3.860228in}{2.009593in}}%
\pgfpathcurveto{\pgfqpoint{3.854404in}{2.003769in}}{\pgfqpoint{3.851132in}{1.995869in}}{\pgfqpoint{3.851132in}{1.987633in}}%
\pgfpathcurveto{\pgfqpoint{3.851132in}{1.979396in}}{\pgfqpoint{3.854404in}{1.971496in}}{\pgfqpoint{3.860228in}{1.965672in}}%
\pgfpathcurveto{\pgfqpoint{3.866052in}{1.959848in}}{\pgfqpoint{3.873952in}{1.956576in}}{\pgfqpoint{3.882188in}{1.956576in}}%
\pgfpathclose%
\pgfusepath{stroke,fill}%
\end{pgfscope}%
\begin{pgfscope}%
\pgfpathrectangle{\pgfqpoint{2.816705in}{0.516222in}}{\pgfqpoint{1.962733in}{1.783528in}} %
\pgfusepath{clip}%
\pgfsetbuttcap%
\pgfsetroundjoin%
\definecolor{currentfill}{rgb}{0.298039,0.447059,0.690196}%
\pgfsetfillcolor{currentfill}%
\pgfsetlinewidth{0.240900pt}%
\definecolor{currentstroke}{rgb}{1.000000,1.000000,1.000000}%
\pgfsetstrokecolor{currentstroke}%
\pgfsetdash{}{0pt}%
\pgfpathmoveto{\pgfqpoint{4.050423in}{1.390306in}}%
\pgfpathcurveto{\pgfqpoint{4.058659in}{1.390306in}}{\pgfqpoint{4.066559in}{1.393578in}}{\pgfqpoint{4.072383in}{1.399402in}}%
\pgfpathcurveto{\pgfqpoint{4.078207in}{1.405226in}}{\pgfqpoint{4.081479in}{1.413126in}}{\pgfqpoint{4.081479in}{1.421363in}}%
\pgfpathcurveto{\pgfqpoint{4.081479in}{1.429599in}}{\pgfqpoint{4.078207in}{1.437499in}}{\pgfqpoint{4.072383in}{1.443323in}}%
\pgfpathcurveto{\pgfqpoint{4.066559in}{1.449147in}}{\pgfqpoint{4.058659in}{1.452419in}}{\pgfqpoint{4.050423in}{1.452419in}}%
\pgfpathcurveto{\pgfqpoint{4.042186in}{1.452419in}}{\pgfqpoint{4.034286in}{1.449147in}}{\pgfqpoint{4.028462in}{1.443323in}}%
\pgfpathcurveto{\pgfqpoint{4.022638in}{1.437499in}}{\pgfqpoint{4.019366in}{1.429599in}}{\pgfqpoint{4.019366in}{1.421363in}}%
\pgfpathcurveto{\pgfqpoint{4.019366in}{1.413126in}}{\pgfqpoint{4.022638in}{1.405226in}}{\pgfqpoint{4.028462in}{1.399402in}}%
\pgfpathcurveto{\pgfqpoint{4.034286in}{1.393578in}}{\pgfqpoint{4.042186in}{1.390306in}}{\pgfqpoint{4.050423in}{1.390306in}}%
\pgfpathclose%
\pgfusepath{stroke,fill}%
\end{pgfscope}%
\begin{pgfscope}%
\pgfpathrectangle{\pgfqpoint{2.816705in}{0.516222in}}{\pgfqpoint{1.962733in}{1.783528in}} %
\pgfusepath{clip}%
\pgfsetbuttcap%
\pgfsetroundjoin%
\definecolor{currentfill}{rgb}{0.298039,0.447059,0.690196}%
\pgfsetfillcolor{currentfill}%
\pgfsetlinewidth{0.240900pt}%
\definecolor{currentstroke}{rgb}{1.000000,1.000000,1.000000}%
\pgfsetstrokecolor{currentstroke}%
\pgfsetdash{}{0pt}%
\pgfpathmoveto{\pgfqpoint{3.601798in}{1.952117in}}%
\pgfpathcurveto{\pgfqpoint{3.610034in}{1.952117in}}{\pgfqpoint{3.617934in}{1.955390in}}{\pgfqpoint{3.623758in}{1.961214in}}%
\pgfpathcurveto{\pgfqpoint{3.629582in}{1.967037in}}{\pgfqpoint{3.632854in}{1.974938in}}{\pgfqpoint{3.632854in}{1.983174in}}%
\pgfpathcurveto{\pgfqpoint{3.632854in}{1.991410in}}{\pgfqpoint{3.629582in}{1.999310in}}{\pgfqpoint{3.623758in}{2.005134in}}%
\pgfpathcurveto{\pgfqpoint{3.617934in}{2.010958in}}{\pgfqpoint{3.610034in}{2.014230in}}{\pgfqpoint{3.601798in}{2.014230in}}%
\pgfpathcurveto{\pgfqpoint{3.593562in}{2.014230in}}{\pgfqpoint{3.585662in}{2.010958in}}{\pgfqpoint{3.579838in}{2.005134in}}%
\pgfpathcurveto{\pgfqpoint{3.574014in}{1.999310in}}{\pgfqpoint{3.570741in}{1.991410in}}{\pgfqpoint{3.570741in}{1.983174in}}%
\pgfpathcurveto{\pgfqpoint{3.570741in}{1.974938in}}{\pgfqpoint{3.574014in}{1.967037in}}{\pgfqpoint{3.579838in}{1.961214in}}%
\pgfpathcurveto{\pgfqpoint{3.585662in}{1.955390in}}{\pgfqpoint{3.593562in}{1.952117in}}{\pgfqpoint{3.601798in}{1.952117in}}%
\pgfpathclose%
\pgfusepath{stroke,fill}%
\end{pgfscope}%
\begin{pgfscope}%
\pgfpathrectangle{\pgfqpoint{2.816705in}{0.516222in}}{\pgfqpoint{1.962733in}{1.783528in}} %
\pgfusepath{clip}%
\pgfsetbuttcap%
\pgfsetroundjoin%
\definecolor{currentfill}{rgb}{0.298039,0.447059,0.690196}%
\pgfsetfillcolor{currentfill}%
\pgfsetlinewidth{0.240900pt}%
\definecolor{currentstroke}{rgb}{1.000000,1.000000,1.000000}%
\pgfsetstrokecolor{currentstroke}%
\pgfsetdash{}{0pt}%
\pgfpathmoveto{\pgfqpoint{3.433564in}{1.702423in}}%
\pgfpathcurveto{\pgfqpoint{3.441800in}{1.702423in}}{\pgfqpoint{3.449700in}{1.705696in}}{\pgfqpoint{3.455524in}{1.711520in}}%
\pgfpathcurveto{\pgfqpoint{3.461348in}{1.717344in}}{\pgfqpoint{3.464620in}{1.725244in}}{\pgfqpoint{3.464620in}{1.733480in}}%
\pgfpathcurveto{\pgfqpoint{3.464620in}{1.741716in}}{\pgfqpoint{3.461348in}{1.749616in}}{\pgfqpoint{3.455524in}{1.755440in}}%
\pgfpathcurveto{\pgfqpoint{3.449700in}{1.761264in}}{\pgfqpoint{3.441800in}{1.764536in}}{\pgfqpoint{3.433564in}{1.764536in}}%
\pgfpathcurveto{\pgfqpoint{3.425327in}{1.764536in}}{\pgfqpoint{3.417427in}{1.761264in}}{\pgfqpoint{3.411603in}{1.755440in}}%
\pgfpathcurveto{\pgfqpoint{3.405779in}{1.749616in}}{\pgfqpoint{3.402507in}{1.741716in}}{\pgfqpoint{3.402507in}{1.733480in}}%
\pgfpathcurveto{\pgfqpoint{3.402507in}{1.725244in}}{\pgfqpoint{3.405779in}{1.717344in}}{\pgfqpoint{3.411603in}{1.711520in}}%
\pgfpathcurveto{\pgfqpoint{3.417427in}{1.705696in}}{\pgfqpoint{3.425327in}{1.702423in}}{\pgfqpoint{3.433564in}{1.702423in}}%
\pgfpathclose%
\pgfusepath{stroke,fill}%
\end{pgfscope}%
\begin{pgfscope}%
\pgfpathrectangle{\pgfqpoint{2.816705in}{0.516222in}}{\pgfqpoint{1.962733in}{1.783528in}} %
\pgfusepath{clip}%
\pgfsetbuttcap%
\pgfsetroundjoin%
\definecolor{currentfill}{rgb}{0.298039,0.447059,0.690196}%
\pgfsetfillcolor{currentfill}%
\pgfsetlinewidth{0.240900pt}%
\definecolor{currentstroke}{rgb}{1.000000,1.000000,1.000000}%
\pgfsetstrokecolor{currentstroke}%
\pgfsetdash{}{0pt}%
\pgfpathmoveto{\pgfqpoint{3.433564in}{1.715800in}}%
\pgfpathcurveto{\pgfqpoint{3.441800in}{1.715800in}}{\pgfqpoint{3.449700in}{1.719072in}}{\pgfqpoint{3.455524in}{1.724896in}}%
\pgfpathcurveto{\pgfqpoint{3.461348in}{1.730720in}}{\pgfqpoint{3.464620in}{1.738620in}}{\pgfqpoint{3.464620in}{1.746856in}}%
\pgfpathcurveto{\pgfqpoint{3.464620in}{1.755093in}}{\pgfqpoint{3.461348in}{1.762993in}}{\pgfqpoint{3.455524in}{1.768817in}}%
\pgfpathcurveto{\pgfqpoint{3.449700in}{1.774641in}}{\pgfqpoint{3.441800in}{1.777913in}}{\pgfqpoint{3.433564in}{1.777913in}}%
\pgfpathcurveto{\pgfqpoint{3.425327in}{1.777913in}}{\pgfqpoint{3.417427in}{1.774641in}}{\pgfqpoint{3.411603in}{1.768817in}}%
\pgfpathcurveto{\pgfqpoint{3.405779in}{1.762993in}}{\pgfqpoint{3.402507in}{1.755093in}}{\pgfqpoint{3.402507in}{1.746856in}}%
\pgfpathcurveto{\pgfqpoint{3.402507in}{1.738620in}}{\pgfqpoint{3.405779in}{1.730720in}}{\pgfqpoint{3.411603in}{1.724896in}}%
\pgfpathcurveto{\pgfqpoint{3.417427in}{1.719072in}}{\pgfqpoint{3.425327in}{1.715800in}}{\pgfqpoint{3.433564in}{1.715800in}}%
\pgfpathclose%
\pgfusepath{stroke,fill}%
\end{pgfscope}%
\begin{pgfscope}%
\pgfpathrectangle{\pgfqpoint{2.816705in}{0.516222in}}{\pgfqpoint{1.962733in}{1.783528in}} %
\pgfusepath{clip}%
\pgfsetbuttcap%
\pgfsetroundjoin%
\definecolor{currentfill}{rgb}{0.298039,0.447059,0.690196}%
\pgfsetfillcolor{currentfill}%
\pgfsetlinewidth{0.240900pt}%
\definecolor{currentstroke}{rgb}{1.000000,1.000000,1.000000}%
\pgfsetstrokecolor{currentstroke}%
\pgfsetdash{}{0pt}%
\pgfpathmoveto{\pgfqpoint{4.050423in}{1.671212in}}%
\pgfpathcurveto{\pgfqpoint{4.058659in}{1.671212in}}{\pgfqpoint{4.066559in}{1.674484in}}{\pgfqpoint{4.072383in}{1.680308in}}%
\pgfpathcurveto{\pgfqpoint{4.078207in}{1.686132in}}{\pgfqpoint{4.081479in}{1.694032in}}{\pgfqpoint{4.081479in}{1.702268in}}%
\pgfpathcurveto{\pgfqpoint{4.081479in}{1.710504in}}{\pgfqpoint{4.078207in}{1.718405in}}{\pgfqpoint{4.072383in}{1.724228in}}%
\pgfpathcurveto{\pgfqpoint{4.066559in}{1.730052in}}{\pgfqpoint{4.058659in}{1.733325in}}{\pgfqpoint{4.050423in}{1.733325in}}%
\pgfpathcurveto{\pgfqpoint{4.042186in}{1.733325in}}{\pgfqpoint{4.034286in}{1.730052in}}{\pgfqpoint{4.028462in}{1.724228in}}%
\pgfpathcurveto{\pgfqpoint{4.022638in}{1.718405in}}{\pgfqpoint{4.019366in}{1.710504in}}{\pgfqpoint{4.019366in}{1.702268in}}%
\pgfpathcurveto{\pgfqpoint{4.019366in}{1.694032in}}{\pgfqpoint{4.022638in}{1.686132in}}{\pgfqpoint{4.028462in}{1.680308in}}%
\pgfpathcurveto{\pgfqpoint{4.034286in}{1.674484in}}{\pgfqpoint{4.042186in}{1.671212in}}{\pgfqpoint{4.050423in}{1.671212in}}%
\pgfpathclose%
\pgfusepath{stroke,fill}%
\end{pgfscope}%
\begin{pgfscope}%
\pgfpathrectangle{\pgfqpoint{2.816705in}{0.516222in}}{\pgfqpoint{1.962733in}{1.783528in}} %
\pgfusepath{clip}%
\pgfsetbuttcap%
\pgfsetroundjoin%
\definecolor{currentfill}{rgb}{0.298039,0.447059,0.690196}%
\pgfsetfillcolor{currentfill}%
\pgfsetlinewidth{0.240900pt}%
\definecolor{currentstroke}{rgb}{1.000000,1.000000,1.000000}%
\pgfsetstrokecolor{currentstroke}%
\pgfsetdash{}{0pt}%
\pgfpathmoveto{\pgfqpoint{3.377486in}{0.725942in}}%
\pgfpathcurveto{\pgfqpoint{3.385722in}{0.725942in}}{\pgfqpoint{3.393622in}{0.729214in}}{\pgfqpoint{3.399446in}{0.735038in}}%
\pgfpathcurveto{\pgfqpoint{3.405270in}{0.740862in}}{\pgfqpoint{3.408542in}{0.748762in}}{\pgfqpoint{3.408542in}{0.756998in}}%
\pgfpathcurveto{\pgfqpoint{3.408542in}{0.765235in}}{\pgfqpoint{3.405270in}{0.773135in}}{\pgfqpoint{3.399446in}{0.778959in}}%
\pgfpathcurveto{\pgfqpoint{3.393622in}{0.784783in}}{\pgfqpoint{3.385722in}{0.788055in}}{\pgfqpoint{3.377486in}{0.788055in}}%
\pgfpathcurveto{\pgfqpoint{3.369249in}{0.788055in}}{\pgfqpoint{3.361349in}{0.784783in}}{\pgfqpoint{3.355525in}{0.778959in}}%
\pgfpathcurveto{\pgfqpoint{3.349701in}{0.773135in}}{\pgfqpoint{3.346429in}{0.765235in}}{\pgfqpoint{3.346429in}{0.756998in}}%
\pgfpathcurveto{\pgfqpoint{3.346429in}{0.748762in}}{\pgfqpoint{3.349701in}{0.740862in}}{\pgfqpoint{3.355525in}{0.735038in}}%
\pgfpathcurveto{\pgfqpoint{3.361349in}{0.729214in}}{\pgfqpoint{3.369249in}{0.725942in}}{\pgfqpoint{3.377486in}{0.725942in}}%
\pgfpathclose%
\pgfusepath{stroke,fill}%
\end{pgfscope}%
\begin{pgfscope}%
\pgfpathrectangle{\pgfqpoint{2.816705in}{0.516222in}}{\pgfqpoint{1.962733in}{1.783528in}} %
\pgfusepath{clip}%
\pgfsetbuttcap%
\pgfsetroundjoin%
\definecolor{currentfill}{rgb}{0.298039,0.447059,0.690196}%
\pgfsetfillcolor{currentfill}%
\pgfsetlinewidth{0.240900pt}%
\definecolor{currentstroke}{rgb}{1.000000,1.000000,1.000000}%
\pgfsetstrokecolor{currentstroke}%
\pgfsetdash{}{0pt}%
\pgfpathmoveto{\pgfqpoint{3.489642in}{1.943200in}}%
\pgfpathcurveto{\pgfqpoint{3.497878in}{1.943200in}}{\pgfqpoint{3.505778in}{1.946472in}}{\pgfqpoint{3.511602in}{1.952296in}}%
\pgfpathcurveto{\pgfqpoint{3.517426in}{1.958120in}}{\pgfqpoint{3.520698in}{1.966020in}}{\pgfqpoint{3.520698in}{1.974256in}}%
\pgfpathcurveto{\pgfqpoint{3.520698in}{1.982492in}}{\pgfqpoint{3.517426in}{1.990393in}}{\pgfqpoint{3.511602in}{1.996216in}}%
\pgfpathcurveto{\pgfqpoint{3.505778in}{2.002040in}}{\pgfqpoint{3.497878in}{2.005313in}}{\pgfqpoint{3.489642in}{2.005313in}}%
\pgfpathcurveto{\pgfqpoint{3.481405in}{2.005313in}}{\pgfqpoint{3.473505in}{2.002040in}}{\pgfqpoint{3.467682in}{1.996216in}}%
\pgfpathcurveto{\pgfqpoint{3.461858in}{1.990393in}}{\pgfqpoint{3.458585in}{1.982492in}}{\pgfqpoint{3.458585in}{1.974256in}}%
\pgfpathcurveto{\pgfqpoint{3.458585in}{1.966020in}}{\pgfqpoint{3.461858in}{1.958120in}}{\pgfqpoint{3.467682in}{1.952296in}}%
\pgfpathcurveto{\pgfqpoint{3.473505in}{1.946472in}}{\pgfqpoint{3.481405in}{1.943200in}}{\pgfqpoint{3.489642in}{1.943200in}}%
\pgfpathclose%
\pgfusepath{stroke,fill}%
\end{pgfscope}%
\begin{pgfscope}%
\pgfpathrectangle{\pgfqpoint{2.816705in}{0.516222in}}{\pgfqpoint{1.962733in}{1.783528in}} %
\pgfusepath{clip}%
\pgfsetbuttcap%
\pgfsetroundjoin%
\definecolor{currentfill}{rgb}{0.298039,0.447059,0.690196}%
\pgfsetfillcolor{currentfill}%
\pgfsetlinewidth{0.240900pt}%
\definecolor{currentstroke}{rgb}{1.000000,1.000000,1.000000}%
\pgfsetstrokecolor{currentstroke}%
\pgfsetdash{}{0pt}%
\pgfpathmoveto{\pgfqpoint{3.265329in}{1.381388in}}%
\pgfpathcurveto{\pgfqpoint{3.273566in}{1.381388in}}{\pgfqpoint{3.281466in}{1.384661in}}{\pgfqpoint{3.287290in}{1.390485in}}%
\pgfpathcurveto{\pgfqpoint{3.293114in}{1.396309in}}{\pgfqpoint{3.296386in}{1.404209in}}{\pgfqpoint{3.296386in}{1.412445in}}%
\pgfpathcurveto{\pgfqpoint{3.296386in}{1.420681in}}{\pgfqpoint{3.293114in}{1.428581in}}{\pgfqpoint{3.287290in}{1.434405in}}%
\pgfpathcurveto{\pgfqpoint{3.281466in}{1.440229in}}{\pgfqpoint{3.273566in}{1.443501in}}{\pgfqpoint{3.265329in}{1.443501in}}%
\pgfpathcurveto{\pgfqpoint{3.257093in}{1.443501in}}{\pgfqpoint{3.249193in}{1.440229in}}{\pgfqpoint{3.243369in}{1.434405in}}%
\pgfpathcurveto{\pgfqpoint{3.237545in}{1.428581in}}{\pgfqpoint{3.234273in}{1.420681in}}{\pgfqpoint{3.234273in}{1.412445in}}%
\pgfpathcurveto{\pgfqpoint{3.234273in}{1.404209in}}{\pgfqpoint{3.237545in}{1.396309in}}{\pgfqpoint{3.243369in}{1.390485in}}%
\pgfpathcurveto{\pgfqpoint{3.249193in}{1.384661in}}{\pgfqpoint{3.257093in}{1.381388in}}{\pgfqpoint{3.265329in}{1.381388in}}%
\pgfpathclose%
\pgfusepath{stroke,fill}%
\end{pgfscope}%
\begin{pgfscope}%
\pgfpathrectangle{\pgfqpoint{2.816705in}{0.516222in}}{\pgfqpoint{1.962733in}{1.783528in}} %
\pgfusepath{clip}%
\pgfsetbuttcap%
\pgfsetroundjoin%
\definecolor{currentfill}{rgb}{0.298039,0.447059,0.690196}%
\pgfsetfillcolor{currentfill}%
\pgfsetlinewidth{0.240900pt}%
\definecolor{currentstroke}{rgb}{1.000000,1.000000,1.000000}%
\pgfsetstrokecolor{currentstroke}%
\pgfsetdash{}{0pt}%
\pgfpathmoveto{\pgfqpoint{4.218657in}{1.987788in}}%
\pgfpathcurveto{\pgfqpoint{4.226893in}{1.987788in}}{\pgfqpoint{4.234793in}{1.991060in}}{\pgfqpoint{4.240617in}{1.996884in}}%
\pgfpathcurveto{\pgfqpoint{4.246441in}{2.002708in}}{\pgfqpoint{4.249713in}{2.010608in}}{\pgfqpoint{4.249713in}{2.018844in}}%
\pgfpathcurveto{\pgfqpoint{4.249713in}{2.027081in}}{\pgfqpoint{4.246441in}{2.034981in}}{\pgfqpoint{4.240617in}{2.040805in}}%
\pgfpathcurveto{\pgfqpoint{4.234793in}{2.046629in}}{\pgfqpoint{4.226893in}{2.049901in}}{\pgfqpoint{4.218657in}{2.049901in}}%
\pgfpathcurveto{\pgfqpoint{4.210420in}{2.049901in}}{\pgfqpoint{4.202520in}{2.046629in}}{\pgfqpoint{4.196696in}{2.040805in}}%
\pgfpathcurveto{\pgfqpoint{4.190873in}{2.034981in}}{\pgfqpoint{4.187600in}{2.027081in}}{\pgfqpoint{4.187600in}{2.018844in}}%
\pgfpathcurveto{\pgfqpoint{4.187600in}{2.010608in}}{\pgfqpoint{4.190873in}{2.002708in}}{\pgfqpoint{4.196696in}{1.996884in}}%
\pgfpathcurveto{\pgfqpoint{4.202520in}{1.991060in}}{\pgfqpoint{4.210420in}{1.987788in}}{\pgfqpoint{4.218657in}{1.987788in}}%
\pgfpathclose%
\pgfusepath{stroke,fill}%
\end{pgfscope}%
\begin{pgfscope}%
\pgfsetrectcap%
\pgfsetmiterjoin%
\pgfsetlinewidth{0.000000pt}%
\definecolor{currentstroke}{rgb}{1.000000,1.000000,1.000000}%
\pgfsetstrokecolor{currentstroke}%
\pgfsetdash{}{0pt}%
\pgfpathmoveto{\pgfqpoint{2.816705in}{0.516222in}}%
\pgfpathlineto{\pgfqpoint{2.816705in}{2.299750in}}%
\pgfusepath{}%
\end{pgfscope}%
\begin{pgfscope}%
\pgfsetrectcap%
\pgfsetmiterjoin%
\pgfsetlinewidth{0.000000pt}%
\definecolor{currentstroke}{rgb}{1.000000,1.000000,1.000000}%
\pgfsetstrokecolor{currentstroke}%
\pgfsetdash{}{0pt}%
\pgfpathmoveto{\pgfqpoint{2.816705in}{0.516222in}}%
\pgfpathlineto{\pgfqpoint{4.779438in}{0.516222in}}%
\pgfusepath{}%
\end{pgfscope}%
\end{pgfpicture}%
\makeatother%
\endgroup%

  \caption{Correlation between the wing length (in centimeters) and the falling
  times (in seconds) of the two realizations.}
  \label{fig_wl_times}
\end{figure}

\begin{figure}
  \centering
  %% Creator: Matplotlib, PGF backend
%%
%% To include the figure in your LaTeX document, write
%%   \input{<filename>.pgf}
%%
%% Make sure the required packages are loaded in your preamble
%%   \usepackage{pgf}
%%
%% Figures using additional raster images can only be included by \input if
%% they are in the same directory as the main LaTeX file. For loading figures
%% from other directories you can use the `import` package
%%   \usepackage{import}
%% and then include the figures with
%%   \import{<path to file>}{<filename>.pgf}
%%
%% Matplotlib used the following preamble
%%   \usepackage[utf8x]{inputenc}
%%   \usepackage[T1]{fontenc}
%%   \usepackage{cmbright}
%%
\begingroup%
\makeatletter%
\begin{pgfpicture}%
\pgfpathrectangle{\pgfpointorigin}{\pgfqpoint{5.000000in}{2.500000in}}%
\pgfusepath{use as bounding box, clip}%
\begin{pgfscope}%
\pgfsetbuttcap%
\pgfsetmiterjoin%
\definecolor{currentfill}{rgb}{1.000000,1.000000,1.000000}%
\pgfsetfillcolor{currentfill}%
\pgfsetlinewidth{0.000000pt}%
\definecolor{currentstroke}{rgb}{1.000000,1.000000,1.000000}%
\pgfsetstrokecolor{currentstroke}%
\pgfsetdash{}{0pt}%
\pgfpathmoveto{\pgfqpoint{0.000000in}{0.000000in}}%
\pgfpathlineto{\pgfqpoint{5.000000in}{0.000000in}}%
\pgfpathlineto{\pgfqpoint{5.000000in}{2.500000in}}%
\pgfpathlineto{\pgfqpoint{0.000000in}{2.500000in}}%
\pgfpathclose%
\pgfusepath{fill}%
\end{pgfscope}%
\begin{pgfscope}%
\pgfsetbuttcap%
\pgfsetmiterjoin%
\definecolor{currentfill}{rgb}{0.917647,0.917647,0.949020}%
\pgfsetfillcolor{currentfill}%
\pgfsetlinewidth{0.000000pt}%
\definecolor{currentstroke}{rgb}{0.000000,0.000000,0.000000}%
\pgfsetstrokecolor{currentstroke}%
\pgfsetstrokeopacity{0.000000}%
\pgfsetdash{}{0pt}%
\pgfpathmoveto{\pgfqpoint{0.556847in}{0.516222in}}%
\pgfpathlineto{\pgfqpoint{2.519580in}{0.516222in}}%
\pgfpathlineto{\pgfqpoint{2.519580in}{2.299750in}}%
\pgfpathlineto{\pgfqpoint{0.556847in}{2.299750in}}%
\pgfpathclose%
\pgfusepath{fill}%
\end{pgfscope}%
\begin{pgfscope}%
\pgfpathrectangle{\pgfqpoint{0.556847in}{0.516222in}}{\pgfqpoint{1.962733in}{1.783528in}} %
\pgfusepath{clip}%
\pgfsetroundcap%
\pgfsetroundjoin%
\pgfsetlinewidth{0.803000pt}%
\definecolor{currentstroke}{rgb}{1.000000,1.000000,1.000000}%
\pgfsetstrokecolor{currentstroke}%
\pgfsetdash{}{0pt}%
\pgfpathmoveto{\pgfqpoint{0.556847in}{0.516222in}}%
\pgfpathlineto{\pgfqpoint{0.556847in}{2.299750in}}%
\pgfusepath{stroke}%
\end{pgfscope}%
\begin{pgfscope}%
\pgfsetbuttcap%
\pgfsetroundjoin%
\definecolor{currentfill}{rgb}{0.150000,0.150000,0.150000}%
\pgfsetfillcolor{currentfill}%
\pgfsetlinewidth{0.803000pt}%
\definecolor{currentstroke}{rgb}{0.150000,0.150000,0.150000}%
\pgfsetstrokecolor{currentstroke}%
\pgfsetdash{}{0pt}%
\pgfsys@defobject{currentmarker}{\pgfqpoint{0.000000in}{0.000000in}}{\pgfqpoint{0.000000in}{0.000000in}}{%
\pgfpathmoveto{\pgfqpoint{0.000000in}{0.000000in}}%
\pgfpathlineto{\pgfqpoint{0.000000in}{0.000000in}}%
\pgfusepath{stroke,fill}%
}%
\begin{pgfscope}%
\pgfsys@transformshift{0.556847in}{0.516222in}%
\pgfsys@useobject{currentmarker}{}%
\end{pgfscope}%
\end{pgfscope}%
\begin{pgfscope}%
\definecolor{textcolor}{rgb}{0.150000,0.150000,0.150000}%
\pgfsetstrokecolor{textcolor}%
\pgfsetfillcolor{textcolor}%
\pgftext[x=0.556847in,y=0.438444in,,top]{\color{textcolor}\sffamily\fontsize{8.000000}{9.600000}\selectfont 1.5}%
\end{pgfscope}%
\begin{pgfscope}%
\pgfpathrectangle{\pgfqpoint{0.556847in}{0.516222in}}{\pgfqpoint{1.962733in}{1.783528in}} %
\pgfusepath{clip}%
\pgfsetroundcap%
\pgfsetroundjoin%
\pgfsetlinewidth{0.803000pt}%
\definecolor{currentstroke}{rgb}{1.000000,1.000000,1.000000}%
\pgfsetstrokecolor{currentstroke}%
\pgfsetdash{}{0pt}%
\pgfpathmoveto{\pgfqpoint{0.837238in}{0.516222in}}%
\pgfpathlineto{\pgfqpoint{0.837238in}{2.299750in}}%
\pgfusepath{stroke}%
\end{pgfscope}%
\begin{pgfscope}%
\pgfsetbuttcap%
\pgfsetroundjoin%
\definecolor{currentfill}{rgb}{0.150000,0.150000,0.150000}%
\pgfsetfillcolor{currentfill}%
\pgfsetlinewidth{0.803000pt}%
\definecolor{currentstroke}{rgb}{0.150000,0.150000,0.150000}%
\pgfsetstrokecolor{currentstroke}%
\pgfsetdash{}{0pt}%
\pgfsys@defobject{currentmarker}{\pgfqpoint{0.000000in}{0.000000in}}{\pgfqpoint{0.000000in}{0.000000in}}{%
\pgfpathmoveto{\pgfqpoint{0.000000in}{0.000000in}}%
\pgfpathlineto{\pgfqpoint{0.000000in}{0.000000in}}%
\pgfusepath{stroke,fill}%
}%
\begin{pgfscope}%
\pgfsys@transformshift{0.837238in}{0.516222in}%
\pgfsys@useobject{currentmarker}{}%
\end{pgfscope}%
\end{pgfscope}%
\begin{pgfscope}%
\definecolor{textcolor}{rgb}{0.150000,0.150000,0.150000}%
\pgfsetstrokecolor{textcolor}%
\pgfsetfillcolor{textcolor}%
\pgftext[x=0.837238in,y=0.438444in,,top]{\color{textcolor}\sffamily\fontsize{8.000000}{9.600000}\selectfont 2.0}%
\end{pgfscope}%
\begin{pgfscope}%
\pgfpathrectangle{\pgfqpoint{0.556847in}{0.516222in}}{\pgfqpoint{1.962733in}{1.783528in}} %
\pgfusepath{clip}%
\pgfsetroundcap%
\pgfsetroundjoin%
\pgfsetlinewidth{0.803000pt}%
\definecolor{currentstroke}{rgb}{1.000000,1.000000,1.000000}%
\pgfsetstrokecolor{currentstroke}%
\pgfsetdash{}{0pt}%
\pgfpathmoveto{\pgfqpoint{1.117628in}{0.516222in}}%
\pgfpathlineto{\pgfqpoint{1.117628in}{2.299750in}}%
\pgfusepath{stroke}%
\end{pgfscope}%
\begin{pgfscope}%
\pgfsetbuttcap%
\pgfsetroundjoin%
\definecolor{currentfill}{rgb}{0.150000,0.150000,0.150000}%
\pgfsetfillcolor{currentfill}%
\pgfsetlinewidth{0.803000pt}%
\definecolor{currentstroke}{rgb}{0.150000,0.150000,0.150000}%
\pgfsetstrokecolor{currentstroke}%
\pgfsetdash{}{0pt}%
\pgfsys@defobject{currentmarker}{\pgfqpoint{0.000000in}{0.000000in}}{\pgfqpoint{0.000000in}{0.000000in}}{%
\pgfpathmoveto{\pgfqpoint{0.000000in}{0.000000in}}%
\pgfpathlineto{\pgfqpoint{0.000000in}{0.000000in}}%
\pgfusepath{stroke,fill}%
}%
\begin{pgfscope}%
\pgfsys@transformshift{1.117628in}{0.516222in}%
\pgfsys@useobject{currentmarker}{}%
\end{pgfscope}%
\end{pgfscope}%
\begin{pgfscope}%
\definecolor{textcolor}{rgb}{0.150000,0.150000,0.150000}%
\pgfsetstrokecolor{textcolor}%
\pgfsetfillcolor{textcolor}%
\pgftext[x=1.117628in,y=0.438444in,,top]{\color{textcolor}\sffamily\fontsize{8.000000}{9.600000}\selectfont 2.5}%
\end{pgfscope}%
\begin{pgfscope}%
\pgfpathrectangle{\pgfqpoint{0.556847in}{0.516222in}}{\pgfqpoint{1.962733in}{1.783528in}} %
\pgfusepath{clip}%
\pgfsetroundcap%
\pgfsetroundjoin%
\pgfsetlinewidth{0.803000pt}%
\definecolor{currentstroke}{rgb}{1.000000,1.000000,1.000000}%
\pgfsetstrokecolor{currentstroke}%
\pgfsetdash{}{0pt}%
\pgfpathmoveto{\pgfqpoint{1.398018in}{0.516222in}}%
\pgfpathlineto{\pgfqpoint{1.398018in}{2.299750in}}%
\pgfusepath{stroke}%
\end{pgfscope}%
\begin{pgfscope}%
\pgfsetbuttcap%
\pgfsetroundjoin%
\definecolor{currentfill}{rgb}{0.150000,0.150000,0.150000}%
\pgfsetfillcolor{currentfill}%
\pgfsetlinewidth{0.803000pt}%
\definecolor{currentstroke}{rgb}{0.150000,0.150000,0.150000}%
\pgfsetstrokecolor{currentstroke}%
\pgfsetdash{}{0pt}%
\pgfsys@defobject{currentmarker}{\pgfqpoint{0.000000in}{0.000000in}}{\pgfqpoint{0.000000in}{0.000000in}}{%
\pgfpathmoveto{\pgfqpoint{0.000000in}{0.000000in}}%
\pgfpathlineto{\pgfqpoint{0.000000in}{0.000000in}}%
\pgfusepath{stroke,fill}%
}%
\begin{pgfscope}%
\pgfsys@transformshift{1.398018in}{0.516222in}%
\pgfsys@useobject{currentmarker}{}%
\end{pgfscope}%
\end{pgfscope}%
\begin{pgfscope}%
\definecolor{textcolor}{rgb}{0.150000,0.150000,0.150000}%
\pgfsetstrokecolor{textcolor}%
\pgfsetfillcolor{textcolor}%
\pgftext[x=1.398018in,y=0.438444in,,top]{\color{textcolor}\sffamily\fontsize{8.000000}{9.600000}\selectfont 3.0}%
\end{pgfscope}%
\begin{pgfscope}%
\pgfpathrectangle{\pgfqpoint{0.556847in}{0.516222in}}{\pgfqpoint{1.962733in}{1.783528in}} %
\pgfusepath{clip}%
\pgfsetroundcap%
\pgfsetroundjoin%
\pgfsetlinewidth{0.803000pt}%
\definecolor{currentstroke}{rgb}{1.000000,1.000000,1.000000}%
\pgfsetstrokecolor{currentstroke}%
\pgfsetdash{}{0pt}%
\pgfpathmoveto{\pgfqpoint{1.678409in}{0.516222in}}%
\pgfpathlineto{\pgfqpoint{1.678409in}{2.299750in}}%
\pgfusepath{stroke}%
\end{pgfscope}%
\begin{pgfscope}%
\pgfsetbuttcap%
\pgfsetroundjoin%
\definecolor{currentfill}{rgb}{0.150000,0.150000,0.150000}%
\pgfsetfillcolor{currentfill}%
\pgfsetlinewidth{0.803000pt}%
\definecolor{currentstroke}{rgb}{0.150000,0.150000,0.150000}%
\pgfsetstrokecolor{currentstroke}%
\pgfsetdash{}{0pt}%
\pgfsys@defobject{currentmarker}{\pgfqpoint{0.000000in}{0.000000in}}{\pgfqpoint{0.000000in}{0.000000in}}{%
\pgfpathmoveto{\pgfqpoint{0.000000in}{0.000000in}}%
\pgfpathlineto{\pgfqpoint{0.000000in}{0.000000in}}%
\pgfusepath{stroke,fill}%
}%
\begin{pgfscope}%
\pgfsys@transformshift{1.678409in}{0.516222in}%
\pgfsys@useobject{currentmarker}{}%
\end{pgfscope}%
\end{pgfscope}%
\begin{pgfscope}%
\definecolor{textcolor}{rgb}{0.150000,0.150000,0.150000}%
\pgfsetstrokecolor{textcolor}%
\pgfsetfillcolor{textcolor}%
\pgftext[x=1.678409in,y=0.438444in,,top]{\color{textcolor}\sffamily\fontsize{8.000000}{9.600000}\selectfont 3.5}%
\end{pgfscope}%
\begin{pgfscope}%
\pgfpathrectangle{\pgfqpoint{0.556847in}{0.516222in}}{\pgfqpoint{1.962733in}{1.783528in}} %
\pgfusepath{clip}%
\pgfsetroundcap%
\pgfsetroundjoin%
\pgfsetlinewidth{0.803000pt}%
\definecolor{currentstroke}{rgb}{1.000000,1.000000,1.000000}%
\pgfsetstrokecolor{currentstroke}%
\pgfsetdash{}{0pt}%
\pgfpathmoveto{\pgfqpoint{1.958799in}{0.516222in}}%
\pgfpathlineto{\pgfqpoint{1.958799in}{2.299750in}}%
\pgfusepath{stroke}%
\end{pgfscope}%
\begin{pgfscope}%
\pgfsetbuttcap%
\pgfsetroundjoin%
\definecolor{currentfill}{rgb}{0.150000,0.150000,0.150000}%
\pgfsetfillcolor{currentfill}%
\pgfsetlinewidth{0.803000pt}%
\definecolor{currentstroke}{rgb}{0.150000,0.150000,0.150000}%
\pgfsetstrokecolor{currentstroke}%
\pgfsetdash{}{0pt}%
\pgfsys@defobject{currentmarker}{\pgfqpoint{0.000000in}{0.000000in}}{\pgfqpoint{0.000000in}{0.000000in}}{%
\pgfpathmoveto{\pgfqpoint{0.000000in}{0.000000in}}%
\pgfpathlineto{\pgfqpoint{0.000000in}{0.000000in}}%
\pgfusepath{stroke,fill}%
}%
\begin{pgfscope}%
\pgfsys@transformshift{1.958799in}{0.516222in}%
\pgfsys@useobject{currentmarker}{}%
\end{pgfscope}%
\end{pgfscope}%
\begin{pgfscope}%
\definecolor{textcolor}{rgb}{0.150000,0.150000,0.150000}%
\pgfsetstrokecolor{textcolor}%
\pgfsetfillcolor{textcolor}%
\pgftext[x=1.958799in,y=0.438444in,,top]{\color{textcolor}\sffamily\fontsize{8.000000}{9.600000}\selectfont 4.0}%
\end{pgfscope}%
\begin{pgfscope}%
\pgfpathrectangle{\pgfqpoint{0.556847in}{0.516222in}}{\pgfqpoint{1.962733in}{1.783528in}} %
\pgfusepath{clip}%
\pgfsetroundcap%
\pgfsetroundjoin%
\pgfsetlinewidth{0.803000pt}%
\definecolor{currentstroke}{rgb}{1.000000,1.000000,1.000000}%
\pgfsetstrokecolor{currentstroke}%
\pgfsetdash{}{0pt}%
\pgfpathmoveto{\pgfqpoint{2.239189in}{0.516222in}}%
\pgfpathlineto{\pgfqpoint{2.239189in}{2.299750in}}%
\pgfusepath{stroke}%
\end{pgfscope}%
\begin{pgfscope}%
\pgfsetbuttcap%
\pgfsetroundjoin%
\definecolor{currentfill}{rgb}{0.150000,0.150000,0.150000}%
\pgfsetfillcolor{currentfill}%
\pgfsetlinewidth{0.803000pt}%
\definecolor{currentstroke}{rgb}{0.150000,0.150000,0.150000}%
\pgfsetstrokecolor{currentstroke}%
\pgfsetdash{}{0pt}%
\pgfsys@defobject{currentmarker}{\pgfqpoint{0.000000in}{0.000000in}}{\pgfqpoint{0.000000in}{0.000000in}}{%
\pgfpathmoveto{\pgfqpoint{0.000000in}{0.000000in}}%
\pgfpathlineto{\pgfqpoint{0.000000in}{0.000000in}}%
\pgfusepath{stroke,fill}%
}%
\begin{pgfscope}%
\pgfsys@transformshift{2.239189in}{0.516222in}%
\pgfsys@useobject{currentmarker}{}%
\end{pgfscope}%
\end{pgfscope}%
\begin{pgfscope}%
\definecolor{textcolor}{rgb}{0.150000,0.150000,0.150000}%
\pgfsetstrokecolor{textcolor}%
\pgfsetfillcolor{textcolor}%
\pgftext[x=2.239189in,y=0.438444in,,top]{\color{textcolor}\sffamily\fontsize{8.000000}{9.600000}\selectfont 4.5}%
\end{pgfscope}%
\begin{pgfscope}%
\pgfpathrectangle{\pgfqpoint{0.556847in}{0.516222in}}{\pgfqpoint{1.962733in}{1.783528in}} %
\pgfusepath{clip}%
\pgfsetroundcap%
\pgfsetroundjoin%
\pgfsetlinewidth{0.803000pt}%
\definecolor{currentstroke}{rgb}{1.000000,1.000000,1.000000}%
\pgfsetstrokecolor{currentstroke}%
\pgfsetdash{}{0pt}%
\pgfpathmoveto{\pgfqpoint{2.519580in}{0.516222in}}%
\pgfpathlineto{\pgfqpoint{2.519580in}{2.299750in}}%
\pgfusepath{stroke}%
\end{pgfscope}%
\begin{pgfscope}%
\pgfsetbuttcap%
\pgfsetroundjoin%
\definecolor{currentfill}{rgb}{0.150000,0.150000,0.150000}%
\pgfsetfillcolor{currentfill}%
\pgfsetlinewidth{0.803000pt}%
\definecolor{currentstroke}{rgb}{0.150000,0.150000,0.150000}%
\pgfsetstrokecolor{currentstroke}%
\pgfsetdash{}{0pt}%
\pgfsys@defobject{currentmarker}{\pgfqpoint{0.000000in}{0.000000in}}{\pgfqpoint{0.000000in}{0.000000in}}{%
\pgfpathmoveto{\pgfqpoint{0.000000in}{0.000000in}}%
\pgfpathlineto{\pgfqpoint{0.000000in}{0.000000in}}%
\pgfusepath{stroke,fill}%
}%
\begin{pgfscope}%
\pgfsys@transformshift{2.519580in}{0.516222in}%
\pgfsys@useobject{currentmarker}{}%
\end{pgfscope}%
\end{pgfscope}%
\begin{pgfscope}%
\definecolor{textcolor}{rgb}{0.150000,0.150000,0.150000}%
\pgfsetstrokecolor{textcolor}%
\pgfsetfillcolor{textcolor}%
\pgftext[x=2.519580in,y=0.438444in,,top]{\color{textcolor}\sffamily\fontsize{8.000000}{9.600000}\selectfont 5.0}%
\end{pgfscope}%
\begin{pgfscope}%
\definecolor{textcolor}{rgb}{0.150000,0.150000,0.150000}%
\pgfsetstrokecolor{textcolor}%
\pgfsetfillcolor{textcolor}%
\pgftext[x=1.538214in,y=0.273321in,,top]{\color{textcolor}\sffamily\fontsize{8.800000}{10.560000}\selectfont Falling time realization 1}%
\end{pgfscope}%
\begin{pgfscope}%
\pgfpathrectangle{\pgfqpoint{0.556847in}{0.516222in}}{\pgfqpoint{1.962733in}{1.783528in}} %
\pgfusepath{clip}%
\pgfsetroundcap%
\pgfsetroundjoin%
\pgfsetlinewidth{0.803000pt}%
\definecolor{currentstroke}{rgb}{1.000000,1.000000,1.000000}%
\pgfsetstrokecolor{currentstroke}%
\pgfsetdash{}{0pt}%
\pgfpathmoveto{\pgfqpoint{0.556847in}{0.516222in}}%
\pgfpathlineto{\pgfqpoint{2.519580in}{0.516222in}}%
\pgfusepath{stroke}%
\end{pgfscope}%
\begin{pgfscope}%
\pgfsetbuttcap%
\pgfsetroundjoin%
\definecolor{currentfill}{rgb}{0.150000,0.150000,0.150000}%
\pgfsetfillcolor{currentfill}%
\pgfsetlinewidth{0.803000pt}%
\definecolor{currentstroke}{rgb}{0.150000,0.150000,0.150000}%
\pgfsetstrokecolor{currentstroke}%
\pgfsetdash{}{0pt}%
\pgfsys@defobject{currentmarker}{\pgfqpoint{0.000000in}{0.000000in}}{\pgfqpoint{0.000000in}{0.000000in}}{%
\pgfpathmoveto{\pgfqpoint{0.000000in}{0.000000in}}%
\pgfpathlineto{\pgfqpoint{0.000000in}{0.000000in}}%
\pgfusepath{stroke,fill}%
}%
\begin{pgfscope}%
\pgfsys@transformshift{0.556847in}{0.516222in}%
\pgfsys@useobject{currentmarker}{}%
\end{pgfscope}%
\end{pgfscope}%
\begin{pgfscope}%
\definecolor{textcolor}{rgb}{0.150000,0.150000,0.150000}%
\pgfsetstrokecolor{textcolor}%
\pgfsetfillcolor{textcolor}%
\pgftext[x=0.479069in,y=0.516222in,right,]{\color{textcolor}\sffamily\fontsize{8.000000}{9.600000}\selectfont 7}%
\end{pgfscope}%
\begin{pgfscope}%
\pgfpathrectangle{\pgfqpoint{0.556847in}{0.516222in}}{\pgfqpoint{1.962733in}{1.783528in}} %
\pgfusepath{clip}%
\pgfsetroundcap%
\pgfsetroundjoin%
\pgfsetlinewidth{0.803000pt}%
\definecolor{currentstroke}{rgb}{1.000000,1.000000,1.000000}%
\pgfsetstrokecolor{currentstroke}%
\pgfsetdash{}{0pt}%
\pgfpathmoveto{\pgfqpoint{0.556847in}{0.813477in}}%
\pgfpathlineto{\pgfqpoint{2.519580in}{0.813477in}}%
\pgfusepath{stroke}%
\end{pgfscope}%
\begin{pgfscope}%
\pgfsetbuttcap%
\pgfsetroundjoin%
\definecolor{currentfill}{rgb}{0.150000,0.150000,0.150000}%
\pgfsetfillcolor{currentfill}%
\pgfsetlinewidth{0.803000pt}%
\definecolor{currentstroke}{rgb}{0.150000,0.150000,0.150000}%
\pgfsetstrokecolor{currentstroke}%
\pgfsetdash{}{0pt}%
\pgfsys@defobject{currentmarker}{\pgfqpoint{0.000000in}{0.000000in}}{\pgfqpoint{0.000000in}{0.000000in}}{%
\pgfpathmoveto{\pgfqpoint{0.000000in}{0.000000in}}%
\pgfpathlineto{\pgfqpoint{0.000000in}{0.000000in}}%
\pgfusepath{stroke,fill}%
}%
\begin{pgfscope}%
\pgfsys@transformshift{0.556847in}{0.813477in}%
\pgfsys@useobject{currentmarker}{}%
\end{pgfscope}%
\end{pgfscope}%
\begin{pgfscope}%
\definecolor{textcolor}{rgb}{0.150000,0.150000,0.150000}%
\pgfsetstrokecolor{textcolor}%
\pgfsetfillcolor{textcolor}%
\pgftext[x=0.479069in,y=0.813477in,right,]{\color{textcolor}\sffamily\fontsize{8.000000}{9.600000}\selectfont 8}%
\end{pgfscope}%
\begin{pgfscope}%
\pgfpathrectangle{\pgfqpoint{0.556847in}{0.516222in}}{\pgfqpoint{1.962733in}{1.783528in}} %
\pgfusepath{clip}%
\pgfsetroundcap%
\pgfsetroundjoin%
\pgfsetlinewidth{0.803000pt}%
\definecolor{currentstroke}{rgb}{1.000000,1.000000,1.000000}%
\pgfsetstrokecolor{currentstroke}%
\pgfsetdash{}{0pt}%
\pgfpathmoveto{\pgfqpoint{0.556847in}{1.110731in}}%
\pgfpathlineto{\pgfqpoint{2.519580in}{1.110731in}}%
\pgfusepath{stroke}%
\end{pgfscope}%
\begin{pgfscope}%
\pgfsetbuttcap%
\pgfsetroundjoin%
\definecolor{currentfill}{rgb}{0.150000,0.150000,0.150000}%
\pgfsetfillcolor{currentfill}%
\pgfsetlinewidth{0.803000pt}%
\definecolor{currentstroke}{rgb}{0.150000,0.150000,0.150000}%
\pgfsetstrokecolor{currentstroke}%
\pgfsetdash{}{0pt}%
\pgfsys@defobject{currentmarker}{\pgfqpoint{0.000000in}{0.000000in}}{\pgfqpoint{0.000000in}{0.000000in}}{%
\pgfpathmoveto{\pgfqpoint{0.000000in}{0.000000in}}%
\pgfpathlineto{\pgfqpoint{0.000000in}{0.000000in}}%
\pgfusepath{stroke,fill}%
}%
\begin{pgfscope}%
\pgfsys@transformshift{0.556847in}{1.110731in}%
\pgfsys@useobject{currentmarker}{}%
\end{pgfscope}%
\end{pgfscope}%
\begin{pgfscope}%
\definecolor{textcolor}{rgb}{0.150000,0.150000,0.150000}%
\pgfsetstrokecolor{textcolor}%
\pgfsetfillcolor{textcolor}%
\pgftext[x=0.479069in,y=1.110731in,right,]{\color{textcolor}\sffamily\fontsize{8.000000}{9.600000}\selectfont 9}%
\end{pgfscope}%
\begin{pgfscope}%
\pgfpathrectangle{\pgfqpoint{0.556847in}{0.516222in}}{\pgfqpoint{1.962733in}{1.783528in}} %
\pgfusepath{clip}%
\pgfsetroundcap%
\pgfsetroundjoin%
\pgfsetlinewidth{0.803000pt}%
\definecolor{currentstroke}{rgb}{1.000000,1.000000,1.000000}%
\pgfsetstrokecolor{currentstroke}%
\pgfsetdash{}{0pt}%
\pgfpathmoveto{\pgfqpoint{0.556847in}{1.407986in}}%
\pgfpathlineto{\pgfqpoint{2.519580in}{1.407986in}}%
\pgfusepath{stroke}%
\end{pgfscope}%
\begin{pgfscope}%
\pgfsetbuttcap%
\pgfsetroundjoin%
\definecolor{currentfill}{rgb}{0.150000,0.150000,0.150000}%
\pgfsetfillcolor{currentfill}%
\pgfsetlinewidth{0.803000pt}%
\definecolor{currentstroke}{rgb}{0.150000,0.150000,0.150000}%
\pgfsetstrokecolor{currentstroke}%
\pgfsetdash{}{0pt}%
\pgfsys@defobject{currentmarker}{\pgfqpoint{0.000000in}{0.000000in}}{\pgfqpoint{0.000000in}{0.000000in}}{%
\pgfpathmoveto{\pgfqpoint{0.000000in}{0.000000in}}%
\pgfpathlineto{\pgfqpoint{0.000000in}{0.000000in}}%
\pgfusepath{stroke,fill}%
}%
\begin{pgfscope}%
\pgfsys@transformshift{0.556847in}{1.407986in}%
\pgfsys@useobject{currentmarker}{}%
\end{pgfscope}%
\end{pgfscope}%
\begin{pgfscope}%
\definecolor{textcolor}{rgb}{0.150000,0.150000,0.150000}%
\pgfsetstrokecolor{textcolor}%
\pgfsetfillcolor{textcolor}%
\pgftext[x=0.479069in,y=1.407986in,right,]{\color{textcolor}\sffamily\fontsize{8.000000}{9.600000}\selectfont 10}%
\end{pgfscope}%
\begin{pgfscope}%
\pgfpathrectangle{\pgfqpoint{0.556847in}{0.516222in}}{\pgfqpoint{1.962733in}{1.783528in}} %
\pgfusepath{clip}%
\pgfsetroundcap%
\pgfsetroundjoin%
\pgfsetlinewidth{0.803000pt}%
\definecolor{currentstroke}{rgb}{1.000000,1.000000,1.000000}%
\pgfsetstrokecolor{currentstroke}%
\pgfsetdash{}{0pt}%
\pgfpathmoveto{\pgfqpoint{0.556847in}{1.705241in}}%
\pgfpathlineto{\pgfqpoint{2.519580in}{1.705241in}}%
\pgfusepath{stroke}%
\end{pgfscope}%
\begin{pgfscope}%
\pgfsetbuttcap%
\pgfsetroundjoin%
\definecolor{currentfill}{rgb}{0.150000,0.150000,0.150000}%
\pgfsetfillcolor{currentfill}%
\pgfsetlinewidth{0.803000pt}%
\definecolor{currentstroke}{rgb}{0.150000,0.150000,0.150000}%
\pgfsetstrokecolor{currentstroke}%
\pgfsetdash{}{0pt}%
\pgfsys@defobject{currentmarker}{\pgfqpoint{0.000000in}{0.000000in}}{\pgfqpoint{0.000000in}{0.000000in}}{%
\pgfpathmoveto{\pgfqpoint{0.000000in}{0.000000in}}%
\pgfpathlineto{\pgfqpoint{0.000000in}{0.000000in}}%
\pgfusepath{stroke,fill}%
}%
\begin{pgfscope}%
\pgfsys@transformshift{0.556847in}{1.705241in}%
\pgfsys@useobject{currentmarker}{}%
\end{pgfscope}%
\end{pgfscope}%
\begin{pgfscope}%
\definecolor{textcolor}{rgb}{0.150000,0.150000,0.150000}%
\pgfsetstrokecolor{textcolor}%
\pgfsetfillcolor{textcolor}%
\pgftext[x=0.479069in,y=1.705241in,right,]{\color{textcolor}\sffamily\fontsize{8.000000}{9.600000}\selectfont 11}%
\end{pgfscope}%
\begin{pgfscope}%
\pgfpathrectangle{\pgfqpoint{0.556847in}{0.516222in}}{\pgfqpoint{1.962733in}{1.783528in}} %
\pgfusepath{clip}%
\pgfsetroundcap%
\pgfsetroundjoin%
\pgfsetlinewidth{0.803000pt}%
\definecolor{currentstroke}{rgb}{1.000000,1.000000,1.000000}%
\pgfsetstrokecolor{currentstroke}%
\pgfsetdash{}{0pt}%
\pgfpathmoveto{\pgfqpoint{0.556847in}{2.002495in}}%
\pgfpathlineto{\pgfqpoint{2.519580in}{2.002495in}}%
\pgfusepath{stroke}%
\end{pgfscope}%
\begin{pgfscope}%
\pgfsetbuttcap%
\pgfsetroundjoin%
\definecolor{currentfill}{rgb}{0.150000,0.150000,0.150000}%
\pgfsetfillcolor{currentfill}%
\pgfsetlinewidth{0.803000pt}%
\definecolor{currentstroke}{rgb}{0.150000,0.150000,0.150000}%
\pgfsetstrokecolor{currentstroke}%
\pgfsetdash{}{0pt}%
\pgfsys@defobject{currentmarker}{\pgfqpoint{0.000000in}{0.000000in}}{\pgfqpoint{0.000000in}{0.000000in}}{%
\pgfpathmoveto{\pgfqpoint{0.000000in}{0.000000in}}%
\pgfpathlineto{\pgfqpoint{0.000000in}{0.000000in}}%
\pgfusepath{stroke,fill}%
}%
\begin{pgfscope}%
\pgfsys@transformshift{0.556847in}{2.002495in}%
\pgfsys@useobject{currentmarker}{}%
\end{pgfscope}%
\end{pgfscope}%
\begin{pgfscope}%
\definecolor{textcolor}{rgb}{0.150000,0.150000,0.150000}%
\pgfsetstrokecolor{textcolor}%
\pgfsetfillcolor{textcolor}%
\pgftext[x=0.479069in,y=2.002495in,right,]{\color{textcolor}\sffamily\fontsize{8.000000}{9.600000}\selectfont 12}%
\end{pgfscope}%
\begin{pgfscope}%
\pgfpathrectangle{\pgfqpoint{0.556847in}{0.516222in}}{\pgfqpoint{1.962733in}{1.783528in}} %
\pgfusepath{clip}%
\pgfsetroundcap%
\pgfsetroundjoin%
\pgfsetlinewidth{0.803000pt}%
\definecolor{currentstroke}{rgb}{1.000000,1.000000,1.000000}%
\pgfsetstrokecolor{currentstroke}%
\pgfsetdash{}{0pt}%
\pgfpathmoveto{\pgfqpoint{0.556847in}{2.299750in}}%
\pgfpathlineto{\pgfqpoint{2.519580in}{2.299750in}}%
\pgfusepath{stroke}%
\end{pgfscope}%
\begin{pgfscope}%
\pgfsetbuttcap%
\pgfsetroundjoin%
\definecolor{currentfill}{rgb}{0.150000,0.150000,0.150000}%
\pgfsetfillcolor{currentfill}%
\pgfsetlinewidth{0.803000pt}%
\definecolor{currentstroke}{rgb}{0.150000,0.150000,0.150000}%
\pgfsetstrokecolor{currentstroke}%
\pgfsetdash{}{0pt}%
\pgfsys@defobject{currentmarker}{\pgfqpoint{0.000000in}{0.000000in}}{\pgfqpoint{0.000000in}{0.000000in}}{%
\pgfpathmoveto{\pgfqpoint{0.000000in}{0.000000in}}%
\pgfpathlineto{\pgfqpoint{0.000000in}{0.000000in}}%
\pgfusepath{stroke,fill}%
}%
\begin{pgfscope}%
\pgfsys@transformshift{0.556847in}{2.299750in}%
\pgfsys@useobject{currentmarker}{}%
\end{pgfscope}%
\end{pgfscope}%
\begin{pgfscope}%
\definecolor{textcolor}{rgb}{0.150000,0.150000,0.150000}%
\pgfsetstrokecolor{textcolor}%
\pgfsetfillcolor{textcolor}%
\pgftext[x=0.479069in,y=2.299750in,right,]{\color{textcolor}\sffamily\fontsize{8.000000}{9.600000}\selectfont 13}%
\end{pgfscope}%
\begin{pgfscope}%
\definecolor{textcolor}{rgb}{0.150000,0.150000,0.150000}%
\pgfsetstrokecolor{textcolor}%
\pgfsetfillcolor{textcolor}%
\pgftext[x=0.286014in,y=1.407986in,,bottom,rotate=90.000000]{\color{textcolor}\sffamily\fontsize{8.800000}{10.560000}\selectfont Arm length}%
\end{pgfscope}%
\begin{pgfscope}%
\pgfpathrectangle{\pgfqpoint{0.556847in}{0.516222in}}{\pgfqpoint{1.962733in}{1.783528in}} %
\pgfusepath{clip}%
\pgfsetbuttcap%
\pgfsetroundjoin%
\definecolor{currentfill}{rgb}{0.298039,0.447059,0.690196}%
\pgfsetfillcolor{currentfill}%
\pgfsetlinewidth{0.240900pt}%
\definecolor{currentstroke}{rgb}{1.000000,1.000000,1.000000}%
\pgfsetstrokecolor{currentstroke}%
\pgfsetdash{}{0pt}%
\pgfpathmoveto{\pgfqpoint{1.902721in}{1.287753in}}%
\pgfpathcurveto{\pgfqpoint{1.910957in}{1.287753in}}{\pgfqpoint{1.918857in}{1.291026in}}{\pgfqpoint{1.924681in}{1.296849in}}%
\pgfpathcurveto{\pgfqpoint{1.930505in}{1.302673in}}{\pgfqpoint{1.933778in}{1.310573in}}{\pgfqpoint{1.933778in}{1.318810in}}%
\pgfpathcurveto{\pgfqpoint{1.933778in}{1.327046in}}{\pgfqpoint{1.930505in}{1.334946in}}{\pgfqpoint{1.924681in}{1.340770in}}%
\pgfpathcurveto{\pgfqpoint{1.918857in}{1.346594in}}{\pgfqpoint{1.910957in}{1.349866in}}{\pgfqpoint{1.902721in}{1.349866in}}%
\pgfpathcurveto{\pgfqpoint{1.894485in}{1.349866in}}{\pgfqpoint{1.886585in}{1.346594in}}{\pgfqpoint{1.880761in}{1.340770in}}%
\pgfpathcurveto{\pgfqpoint{1.874937in}{1.334946in}}{\pgfqpoint{1.871665in}{1.327046in}}{\pgfqpoint{1.871665in}{1.318810in}}%
\pgfpathcurveto{\pgfqpoint{1.871665in}{1.310573in}}{\pgfqpoint{1.874937in}{1.302673in}}{\pgfqpoint{1.880761in}{1.296849in}}%
\pgfpathcurveto{\pgfqpoint{1.886585in}{1.291026in}}{\pgfqpoint{1.894485in}{1.287753in}}{\pgfqpoint{1.902721in}{1.287753in}}%
\pgfpathclose%
\pgfusepath{stroke,fill}%
\end{pgfscope}%
\begin{pgfscope}%
\pgfpathrectangle{\pgfqpoint{0.556847in}{0.516222in}}{\pgfqpoint{1.962733in}{1.783528in}} %
\pgfusepath{clip}%
\pgfsetbuttcap%
\pgfsetroundjoin%
\definecolor{currentfill}{rgb}{0.298039,0.447059,0.690196}%
\pgfsetfillcolor{currentfill}%
\pgfsetlinewidth{0.240900pt}%
\definecolor{currentstroke}{rgb}{1.000000,1.000000,1.000000}%
\pgfsetstrokecolor{currentstroke}%
\pgfsetdash{}{0pt}%
\pgfpathmoveto{\pgfqpoint{1.622331in}{1.763361in}}%
\pgfpathcurveto{\pgfqpoint{1.630567in}{1.763361in}}{\pgfqpoint{1.638467in}{1.766633in}}{\pgfqpoint{1.644291in}{1.772457in}}%
\pgfpathcurveto{\pgfqpoint{1.650115in}{1.778281in}}{\pgfqpoint{1.653387in}{1.786181in}}{\pgfqpoint{1.653387in}{1.794417in}}%
\pgfpathcurveto{\pgfqpoint{1.653387in}{1.802653in}}{\pgfqpoint{1.650115in}{1.810553in}}{\pgfqpoint{1.644291in}{1.816377in}}%
\pgfpathcurveto{\pgfqpoint{1.638467in}{1.822201in}}{\pgfqpoint{1.630567in}{1.825474in}}{\pgfqpoint{1.622331in}{1.825474in}}%
\pgfpathcurveto{\pgfqpoint{1.614094in}{1.825474in}}{\pgfqpoint{1.606194in}{1.822201in}}{\pgfqpoint{1.600370in}{1.816377in}}%
\pgfpathcurveto{\pgfqpoint{1.594546in}{1.810553in}}{\pgfqpoint{1.591274in}{1.802653in}}{\pgfqpoint{1.591274in}{1.794417in}}%
\pgfpathcurveto{\pgfqpoint{1.591274in}{1.786181in}}{\pgfqpoint{1.594546in}{1.778281in}}{\pgfqpoint{1.600370in}{1.772457in}}%
\pgfpathcurveto{\pgfqpoint{1.606194in}{1.766633in}}{\pgfqpoint{1.614094in}{1.763361in}}{\pgfqpoint{1.622331in}{1.763361in}}%
\pgfpathclose%
\pgfusepath{stroke,fill}%
\end{pgfscope}%
\begin{pgfscope}%
\pgfpathrectangle{\pgfqpoint{0.556847in}{0.516222in}}{\pgfqpoint{1.962733in}{1.783528in}} %
\pgfusepath{clip}%
\pgfsetbuttcap%
\pgfsetroundjoin%
\definecolor{currentfill}{rgb}{0.298039,0.447059,0.690196}%
\pgfsetfillcolor{currentfill}%
\pgfsetlinewidth{0.240900pt}%
\definecolor{currentstroke}{rgb}{1.000000,1.000000,1.000000}%
\pgfsetstrokecolor{currentstroke}%
\pgfsetdash{}{0pt}%
\pgfpathmoveto{\pgfqpoint{1.958799in}{1.436381in}}%
\pgfpathcurveto{\pgfqpoint{1.967035in}{1.436381in}}{\pgfqpoint{1.974935in}{1.439653in}}{\pgfqpoint{1.980759in}{1.445477in}}%
\pgfpathcurveto{\pgfqpoint{1.986583in}{1.451301in}}{\pgfqpoint{1.989856in}{1.459201in}}{\pgfqpoint{1.989856in}{1.467437in}}%
\pgfpathcurveto{\pgfqpoint{1.989856in}{1.475673in}}{\pgfqpoint{1.986583in}{1.483573in}}{\pgfqpoint{1.980759in}{1.489397in}}%
\pgfpathcurveto{\pgfqpoint{1.974935in}{1.495221in}}{\pgfqpoint{1.967035in}{1.498494in}}{\pgfqpoint{1.958799in}{1.498494in}}%
\pgfpathcurveto{\pgfqpoint{1.950563in}{1.498494in}}{\pgfqpoint{1.942663in}{1.495221in}}{\pgfqpoint{1.936839in}{1.489397in}}%
\pgfpathcurveto{\pgfqpoint{1.931015in}{1.483573in}}{\pgfqpoint{1.927743in}{1.475673in}}{\pgfqpoint{1.927743in}{1.467437in}}%
\pgfpathcurveto{\pgfqpoint{1.927743in}{1.459201in}}{\pgfqpoint{1.931015in}{1.451301in}}{\pgfqpoint{1.936839in}{1.445477in}}%
\pgfpathcurveto{\pgfqpoint{1.942663in}{1.439653in}}{\pgfqpoint{1.950563in}{1.436381in}}{\pgfqpoint{1.958799in}{1.436381in}}%
\pgfpathclose%
\pgfusepath{stroke,fill}%
\end{pgfscope}%
\begin{pgfscope}%
\pgfpathrectangle{\pgfqpoint{0.556847in}{0.516222in}}{\pgfqpoint{1.962733in}{1.783528in}} %
\pgfusepath{clip}%
\pgfsetbuttcap%
\pgfsetroundjoin%
\definecolor{currentfill}{rgb}{0.298039,0.447059,0.690196}%
\pgfsetfillcolor{currentfill}%
\pgfsetlinewidth{0.240900pt}%
\definecolor{currentstroke}{rgb}{1.000000,1.000000,1.000000}%
\pgfsetstrokecolor{currentstroke}%
\pgfsetdash{}{0pt}%
\pgfpathmoveto{\pgfqpoint{2.239189in}{1.822812in}}%
\pgfpathcurveto{\pgfqpoint{2.247426in}{1.822812in}}{\pgfqpoint{2.255326in}{1.826084in}}{\pgfqpoint{2.261150in}{1.831908in}}%
\pgfpathcurveto{\pgfqpoint{2.266974in}{1.837732in}}{\pgfqpoint{2.270246in}{1.845632in}}{\pgfqpoint{2.270246in}{1.853868in}}%
\pgfpathcurveto{\pgfqpoint{2.270246in}{1.862104in}}{\pgfqpoint{2.266974in}{1.870004in}}{\pgfqpoint{2.261150in}{1.875828in}}%
\pgfpathcurveto{\pgfqpoint{2.255326in}{1.881652in}}{\pgfqpoint{2.247426in}{1.884925in}}{\pgfqpoint{2.239189in}{1.884925in}}%
\pgfpathcurveto{\pgfqpoint{2.230953in}{1.884925in}}{\pgfqpoint{2.223053in}{1.881652in}}{\pgfqpoint{2.217229in}{1.875828in}}%
\pgfpathcurveto{\pgfqpoint{2.211405in}{1.870004in}}{\pgfqpoint{2.208133in}{1.862104in}}{\pgfqpoint{2.208133in}{1.853868in}}%
\pgfpathcurveto{\pgfqpoint{2.208133in}{1.845632in}}{\pgfqpoint{2.211405in}{1.837732in}}{\pgfqpoint{2.217229in}{1.831908in}}%
\pgfpathcurveto{\pgfqpoint{2.223053in}{1.826084in}}{\pgfqpoint{2.230953in}{1.822812in}}{\pgfqpoint{2.239189in}{1.822812in}}%
\pgfpathclose%
\pgfusepath{stroke,fill}%
\end{pgfscope}%
\begin{pgfscope}%
\pgfpathrectangle{\pgfqpoint{0.556847in}{0.516222in}}{\pgfqpoint{1.962733in}{1.783528in}} %
\pgfusepath{clip}%
\pgfsetbuttcap%
\pgfsetroundjoin%
\definecolor{currentfill}{rgb}{0.298039,0.447059,0.690196}%
\pgfsetfillcolor{currentfill}%
\pgfsetlinewidth{0.240900pt}%
\definecolor{currentstroke}{rgb}{1.000000,1.000000,1.000000}%
\pgfsetstrokecolor{currentstroke}%
\pgfsetdash{}{0pt}%
\pgfpathmoveto{\pgfqpoint{1.622331in}{1.733635in}}%
\pgfpathcurveto{\pgfqpoint{1.630567in}{1.733635in}}{\pgfqpoint{1.638467in}{1.736907in}}{\pgfqpoint{1.644291in}{1.742731in}}%
\pgfpathcurveto{\pgfqpoint{1.650115in}{1.748555in}}{\pgfqpoint{1.653387in}{1.756455in}}{\pgfqpoint{1.653387in}{1.764692in}}%
\pgfpathcurveto{\pgfqpoint{1.653387in}{1.772928in}}{\pgfqpoint{1.650115in}{1.780828in}}{\pgfqpoint{1.644291in}{1.786652in}}%
\pgfpathcurveto{\pgfqpoint{1.638467in}{1.792476in}}{\pgfqpoint{1.630567in}{1.795748in}}{\pgfqpoint{1.622331in}{1.795748in}}%
\pgfpathcurveto{\pgfqpoint{1.614094in}{1.795748in}}{\pgfqpoint{1.606194in}{1.792476in}}{\pgfqpoint{1.600370in}{1.786652in}}%
\pgfpathcurveto{\pgfqpoint{1.594546in}{1.780828in}}{\pgfqpoint{1.591274in}{1.772928in}}{\pgfqpoint{1.591274in}{1.764692in}}%
\pgfpathcurveto{\pgfqpoint{1.591274in}{1.756455in}}{\pgfqpoint{1.594546in}{1.748555in}}{\pgfqpoint{1.600370in}{1.742731in}}%
\pgfpathcurveto{\pgfqpoint{1.606194in}{1.736907in}}{\pgfqpoint{1.614094in}{1.733635in}}{\pgfqpoint{1.622331in}{1.733635in}}%
\pgfpathclose%
\pgfusepath{stroke,fill}%
\end{pgfscope}%
\begin{pgfscope}%
\pgfpathrectangle{\pgfqpoint{0.556847in}{0.516222in}}{\pgfqpoint{1.962733in}{1.783528in}} %
\pgfusepath{clip}%
\pgfsetbuttcap%
\pgfsetroundjoin%
\definecolor{currentfill}{rgb}{0.298039,0.447059,0.690196}%
\pgfsetfillcolor{currentfill}%
\pgfsetlinewidth{0.240900pt}%
\definecolor{currentstroke}{rgb}{1.000000,1.000000,1.000000}%
\pgfsetstrokecolor{currentstroke}%
\pgfsetdash{}{0pt}%
\pgfpathmoveto{\pgfqpoint{2.407424in}{1.911988in}}%
\pgfpathcurveto{\pgfqpoint{2.415660in}{1.911988in}}{\pgfqpoint{2.423560in}{1.915260in}}{\pgfqpoint{2.429384in}{1.921084in}}%
\pgfpathcurveto{\pgfqpoint{2.435208in}{1.926908in}}{\pgfqpoint{2.438480in}{1.934808in}}{\pgfqpoint{2.438480in}{1.943044in}}%
\pgfpathcurveto{\pgfqpoint{2.438480in}{1.951281in}}{\pgfqpoint{2.435208in}{1.959181in}}{\pgfqpoint{2.429384in}{1.965005in}}%
\pgfpathcurveto{\pgfqpoint{2.423560in}{1.970829in}}{\pgfqpoint{2.415660in}{1.974101in}}{\pgfqpoint{2.407424in}{1.974101in}}%
\pgfpathcurveto{\pgfqpoint{2.399187in}{1.974101in}}{\pgfqpoint{2.391287in}{1.970829in}}{\pgfqpoint{2.385463in}{1.965005in}}%
\pgfpathcurveto{\pgfqpoint{2.379640in}{1.959181in}}{\pgfqpoint{2.376367in}{1.951281in}}{\pgfqpoint{2.376367in}{1.943044in}}%
\pgfpathcurveto{\pgfqpoint{2.376367in}{1.934808in}}{\pgfqpoint{2.379640in}{1.926908in}}{\pgfqpoint{2.385463in}{1.921084in}}%
\pgfpathcurveto{\pgfqpoint{2.391287in}{1.915260in}}{\pgfqpoint{2.399187in}{1.911988in}}{\pgfqpoint{2.407424in}{1.911988in}}%
\pgfpathclose%
\pgfusepath{stroke,fill}%
\end{pgfscope}%
\begin{pgfscope}%
\pgfpathrectangle{\pgfqpoint{0.556847in}{0.516222in}}{\pgfqpoint{1.962733in}{1.783528in}} %
\pgfusepath{clip}%
\pgfsetbuttcap%
\pgfsetroundjoin%
\definecolor{currentfill}{rgb}{0.298039,0.447059,0.690196}%
\pgfsetfillcolor{currentfill}%
\pgfsetlinewidth{0.240900pt}%
\definecolor{currentstroke}{rgb}{1.000000,1.000000,1.000000}%
\pgfsetstrokecolor{currentstroke}%
\pgfsetdash{}{0pt}%
\pgfpathmoveto{\pgfqpoint{1.622331in}{1.168851in}}%
\pgfpathcurveto{\pgfqpoint{1.630567in}{1.168851in}}{\pgfqpoint{1.638467in}{1.172124in}}{\pgfqpoint{1.644291in}{1.177948in}}%
\pgfpathcurveto{\pgfqpoint{1.650115in}{1.183772in}}{\pgfqpoint{1.653387in}{1.191672in}}{\pgfqpoint{1.653387in}{1.199908in}}%
\pgfpathcurveto{\pgfqpoint{1.653387in}{1.208144in}}{\pgfqpoint{1.650115in}{1.216044in}}{\pgfqpoint{1.644291in}{1.221868in}}%
\pgfpathcurveto{\pgfqpoint{1.638467in}{1.227692in}}{\pgfqpoint{1.630567in}{1.230964in}}{\pgfqpoint{1.622331in}{1.230964in}}%
\pgfpathcurveto{\pgfqpoint{1.614094in}{1.230964in}}{\pgfqpoint{1.606194in}{1.227692in}}{\pgfqpoint{1.600370in}{1.221868in}}%
\pgfpathcurveto{\pgfqpoint{1.594546in}{1.216044in}}{\pgfqpoint{1.591274in}{1.208144in}}{\pgfqpoint{1.591274in}{1.199908in}}%
\pgfpathcurveto{\pgfqpoint{1.591274in}{1.191672in}}{\pgfqpoint{1.594546in}{1.183772in}}{\pgfqpoint{1.600370in}{1.177948in}}%
\pgfpathcurveto{\pgfqpoint{1.606194in}{1.172124in}}{\pgfqpoint{1.614094in}{1.168851in}}{\pgfqpoint{1.622331in}{1.168851in}}%
\pgfpathclose%
\pgfusepath{stroke,fill}%
\end{pgfscope}%
\begin{pgfscope}%
\pgfpathrectangle{\pgfqpoint{0.556847in}{0.516222in}}{\pgfqpoint{1.962733in}{1.783528in}} %
\pgfusepath{clip}%
\pgfsetbuttcap%
\pgfsetroundjoin%
\definecolor{currentfill}{rgb}{0.298039,0.447059,0.690196}%
\pgfsetfillcolor{currentfill}%
\pgfsetlinewidth{0.240900pt}%
\definecolor{currentstroke}{rgb}{1.000000,1.000000,1.000000}%
\pgfsetstrokecolor{currentstroke}%
\pgfsetdash{}{0pt}%
\pgfpathmoveto{\pgfqpoint{2.407424in}{1.674184in}}%
\pgfpathcurveto{\pgfqpoint{2.415660in}{1.674184in}}{\pgfqpoint{2.423560in}{1.677457in}}{\pgfqpoint{2.429384in}{1.683280in}}%
\pgfpathcurveto{\pgfqpoint{2.435208in}{1.689104in}}{\pgfqpoint{2.438480in}{1.697004in}}{\pgfqpoint{2.438480in}{1.705241in}}%
\pgfpathcurveto{\pgfqpoint{2.438480in}{1.713477in}}{\pgfqpoint{2.435208in}{1.721377in}}{\pgfqpoint{2.429384in}{1.727201in}}%
\pgfpathcurveto{\pgfqpoint{2.423560in}{1.733025in}}{\pgfqpoint{2.415660in}{1.736297in}}{\pgfqpoint{2.407424in}{1.736297in}}%
\pgfpathcurveto{\pgfqpoint{2.399187in}{1.736297in}}{\pgfqpoint{2.391287in}{1.733025in}}{\pgfqpoint{2.385463in}{1.727201in}}%
\pgfpathcurveto{\pgfqpoint{2.379640in}{1.721377in}}{\pgfqpoint{2.376367in}{1.713477in}}{\pgfqpoint{2.376367in}{1.705241in}}%
\pgfpathcurveto{\pgfqpoint{2.376367in}{1.697004in}}{\pgfqpoint{2.379640in}{1.689104in}}{\pgfqpoint{2.385463in}{1.683280in}}%
\pgfpathcurveto{\pgfqpoint{2.391287in}{1.677457in}}{\pgfqpoint{2.399187in}{1.674184in}}{\pgfqpoint{2.407424in}{1.674184in}}%
\pgfpathclose%
\pgfusepath{stroke,fill}%
\end{pgfscope}%
\begin{pgfscope}%
\pgfpathrectangle{\pgfqpoint{0.556847in}{0.516222in}}{\pgfqpoint{1.962733in}{1.783528in}} %
\pgfusepath{clip}%
\pgfsetbuttcap%
\pgfsetroundjoin%
\definecolor{currentfill}{rgb}{0.298039,0.447059,0.690196}%
\pgfsetfillcolor{currentfill}%
\pgfsetlinewidth{0.240900pt}%
\definecolor{currentstroke}{rgb}{1.000000,1.000000,1.000000}%
\pgfsetstrokecolor{currentstroke}%
\pgfsetdash{}{0pt}%
\pgfpathmoveto{\pgfqpoint{1.173706in}{1.585008in}}%
\pgfpathcurveto{\pgfqpoint{1.181942in}{1.585008in}}{\pgfqpoint{1.189842in}{1.588280in}}{\pgfqpoint{1.195666in}{1.594104in}}%
\pgfpathcurveto{\pgfqpoint{1.201490in}{1.599928in}}{\pgfqpoint{1.204763in}{1.607828in}}{\pgfqpoint{1.204763in}{1.616064in}}%
\pgfpathcurveto{\pgfqpoint{1.204763in}{1.624301in}}{\pgfqpoint{1.201490in}{1.632201in}}{\pgfqpoint{1.195666in}{1.638025in}}%
\pgfpathcurveto{\pgfqpoint{1.189842in}{1.643849in}}{\pgfqpoint{1.181942in}{1.647121in}}{\pgfqpoint{1.173706in}{1.647121in}}%
\pgfpathcurveto{\pgfqpoint{1.165470in}{1.647121in}}{\pgfqpoint{1.157570in}{1.643849in}}{\pgfqpoint{1.151746in}{1.638025in}}%
\pgfpathcurveto{\pgfqpoint{1.145922in}{1.632201in}}{\pgfqpoint{1.142650in}{1.624301in}}{\pgfqpoint{1.142650in}{1.616064in}}%
\pgfpathcurveto{\pgfqpoint{1.142650in}{1.607828in}}{\pgfqpoint{1.145922in}{1.599928in}}{\pgfqpoint{1.151746in}{1.594104in}}%
\pgfpathcurveto{\pgfqpoint{1.157570in}{1.588280in}}{\pgfqpoint{1.165470in}{1.585008in}}{\pgfqpoint{1.173706in}{1.585008in}}%
\pgfpathclose%
\pgfusepath{stroke,fill}%
\end{pgfscope}%
\begin{pgfscope}%
\pgfpathrectangle{\pgfqpoint{0.556847in}{0.516222in}}{\pgfqpoint{1.962733in}{1.783528in}} %
\pgfusepath{clip}%
\pgfsetbuttcap%
\pgfsetroundjoin%
\definecolor{currentfill}{rgb}{0.298039,0.447059,0.690196}%
\pgfsetfillcolor{currentfill}%
\pgfsetlinewidth{0.240900pt}%
\definecolor{currentstroke}{rgb}{1.000000,1.000000,1.000000}%
\pgfsetstrokecolor{currentstroke}%
\pgfsetdash{}{0pt}%
\pgfpathmoveto{\pgfqpoint{1.678409in}{1.436381in}}%
\pgfpathcurveto{\pgfqpoint{1.686645in}{1.436381in}}{\pgfqpoint{1.694545in}{1.439653in}}{\pgfqpoint{1.700369in}{1.445477in}}%
\pgfpathcurveto{\pgfqpoint{1.706193in}{1.451301in}}{\pgfqpoint{1.709465in}{1.459201in}}{\pgfqpoint{1.709465in}{1.467437in}}%
\pgfpathcurveto{\pgfqpoint{1.709465in}{1.475673in}}{\pgfqpoint{1.706193in}{1.483573in}}{\pgfqpoint{1.700369in}{1.489397in}}%
\pgfpathcurveto{\pgfqpoint{1.694545in}{1.495221in}}{\pgfqpoint{1.686645in}{1.498494in}}{\pgfqpoint{1.678409in}{1.498494in}}%
\pgfpathcurveto{\pgfqpoint{1.670172in}{1.498494in}}{\pgfqpoint{1.662272in}{1.495221in}}{\pgfqpoint{1.656448in}{1.489397in}}%
\pgfpathcurveto{\pgfqpoint{1.650625in}{1.483573in}}{\pgfqpoint{1.647352in}{1.475673in}}{\pgfqpoint{1.647352in}{1.467437in}}%
\pgfpathcurveto{\pgfqpoint{1.647352in}{1.459201in}}{\pgfqpoint{1.650625in}{1.451301in}}{\pgfqpoint{1.656448in}{1.445477in}}%
\pgfpathcurveto{\pgfqpoint{1.662272in}{1.439653in}}{\pgfqpoint{1.670172in}{1.436381in}}{\pgfqpoint{1.678409in}{1.436381in}}%
\pgfpathclose%
\pgfusepath{stroke,fill}%
\end{pgfscope}%
\begin{pgfscope}%
\pgfpathrectangle{\pgfqpoint{0.556847in}{0.516222in}}{\pgfqpoint{1.962733in}{1.783528in}} %
\pgfusepath{clip}%
\pgfsetbuttcap%
\pgfsetroundjoin%
\definecolor{currentfill}{rgb}{0.298039,0.447059,0.690196}%
\pgfsetfillcolor{currentfill}%
\pgfsetlinewidth{0.240900pt}%
\definecolor{currentstroke}{rgb}{1.000000,1.000000,1.000000}%
\pgfsetstrokecolor{currentstroke}%
\pgfsetdash{}{0pt}%
\pgfpathmoveto{\pgfqpoint{1.510175in}{0.633793in}}%
\pgfpathcurveto{\pgfqpoint{1.518411in}{0.633793in}}{\pgfqpoint{1.526311in}{0.637065in}}{\pgfqpoint{1.532135in}{0.642889in}}%
\pgfpathcurveto{\pgfqpoint{1.537959in}{0.648713in}}{\pgfqpoint{1.541231in}{0.656613in}}{\pgfqpoint{1.541231in}{0.664850in}}%
\pgfpathcurveto{\pgfqpoint{1.541231in}{0.673086in}}{\pgfqpoint{1.537959in}{0.680986in}}{\pgfqpoint{1.532135in}{0.686810in}}%
\pgfpathcurveto{\pgfqpoint{1.526311in}{0.692634in}}{\pgfqpoint{1.518411in}{0.695906in}}{\pgfqpoint{1.510175in}{0.695906in}}%
\pgfpathcurveto{\pgfqpoint{1.501938in}{0.695906in}}{\pgfqpoint{1.494038in}{0.692634in}}{\pgfqpoint{1.488214in}{0.686810in}}%
\pgfpathcurveto{\pgfqpoint{1.482390in}{0.680986in}}{\pgfqpoint{1.479118in}{0.673086in}}{\pgfqpoint{1.479118in}{0.664850in}}%
\pgfpathcurveto{\pgfqpoint{1.479118in}{0.656613in}}{\pgfqpoint{1.482390in}{0.648713in}}{\pgfqpoint{1.488214in}{0.642889in}}%
\pgfpathcurveto{\pgfqpoint{1.494038in}{0.637065in}}{\pgfqpoint{1.501938in}{0.633793in}}{\pgfqpoint{1.510175in}{0.633793in}}%
\pgfpathclose%
\pgfusepath{stroke,fill}%
\end{pgfscope}%
\begin{pgfscope}%
\pgfpathrectangle{\pgfqpoint{0.556847in}{0.516222in}}{\pgfqpoint{1.962733in}{1.783528in}} %
\pgfusepath{clip}%
\pgfsetbuttcap%
\pgfsetroundjoin%
\definecolor{currentfill}{rgb}{0.298039,0.447059,0.690196}%
\pgfsetfillcolor{currentfill}%
\pgfsetlinewidth{0.240900pt}%
\definecolor{currentstroke}{rgb}{1.000000,1.000000,1.000000}%
\pgfsetstrokecolor{currentstroke}%
\pgfsetdash{}{0pt}%
\pgfpathmoveto{\pgfqpoint{1.678409in}{1.793086in}}%
\pgfpathcurveto{\pgfqpoint{1.686645in}{1.793086in}}{\pgfqpoint{1.694545in}{1.796358in}}{\pgfqpoint{1.700369in}{1.802182in}}%
\pgfpathcurveto{\pgfqpoint{1.706193in}{1.808006in}}{\pgfqpoint{1.709465in}{1.815906in}}{\pgfqpoint{1.709465in}{1.824143in}}%
\pgfpathcurveto{\pgfqpoint{1.709465in}{1.832379in}}{\pgfqpoint{1.706193in}{1.840279in}}{\pgfqpoint{1.700369in}{1.846103in}}%
\pgfpathcurveto{\pgfqpoint{1.694545in}{1.851927in}}{\pgfqpoint{1.686645in}{1.855199in}}{\pgfqpoint{1.678409in}{1.855199in}}%
\pgfpathcurveto{\pgfqpoint{1.670172in}{1.855199in}}{\pgfqpoint{1.662272in}{1.851927in}}{\pgfqpoint{1.656448in}{1.846103in}}%
\pgfpathcurveto{\pgfqpoint{1.650625in}{1.840279in}}{\pgfqpoint{1.647352in}{1.832379in}}{\pgfqpoint{1.647352in}{1.824143in}}%
\pgfpathcurveto{\pgfqpoint{1.647352in}{1.815906in}}{\pgfqpoint{1.650625in}{1.808006in}}{\pgfqpoint{1.656448in}{1.802182in}}%
\pgfpathcurveto{\pgfqpoint{1.662272in}{1.796358in}}{\pgfqpoint{1.670172in}{1.793086in}}{\pgfqpoint{1.678409in}{1.793086in}}%
\pgfpathclose%
\pgfusepath{stroke,fill}%
\end{pgfscope}%
\begin{pgfscope}%
\pgfpathrectangle{\pgfqpoint{0.556847in}{0.516222in}}{\pgfqpoint{1.962733in}{1.783528in}} %
\pgfusepath{clip}%
\pgfsetbuttcap%
\pgfsetroundjoin%
\definecolor{currentfill}{rgb}{0.298039,0.447059,0.690196}%
\pgfsetfillcolor{currentfill}%
\pgfsetlinewidth{0.240900pt}%
\definecolor{currentstroke}{rgb}{1.000000,1.000000,1.000000}%
\pgfsetstrokecolor{currentstroke}%
\pgfsetdash{}{0pt}%
\pgfpathmoveto{\pgfqpoint{1.790565in}{1.376930in}}%
\pgfpathcurveto{\pgfqpoint{1.798801in}{1.376930in}}{\pgfqpoint{1.806701in}{1.380202in}}{\pgfqpoint{1.812525in}{1.386026in}}%
\pgfpathcurveto{\pgfqpoint{1.818349in}{1.391850in}}{\pgfqpoint{1.821621in}{1.399750in}}{\pgfqpoint{1.821621in}{1.407986in}}%
\pgfpathcurveto{\pgfqpoint{1.821621in}{1.416222in}}{\pgfqpoint{1.818349in}{1.424122in}}{\pgfqpoint{1.812525in}{1.429946in}}%
\pgfpathcurveto{\pgfqpoint{1.806701in}{1.435770in}}{\pgfqpoint{1.798801in}{1.439043in}}{\pgfqpoint{1.790565in}{1.439043in}}%
\pgfpathcurveto{\pgfqpoint{1.782329in}{1.439043in}}{\pgfqpoint{1.774429in}{1.435770in}}{\pgfqpoint{1.768605in}{1.429946in}}%
\pgfpathcurveto{\pgfqpoint{1.762781in}{1.424122in}}{\pgfqpoint{1.759508in}{1.416222in}}{\pgfqpoint{1.759508in}{1.407986in}}%
\pgfpathcurveto{\pgfqpoint{1.759508in}{1.399750in}}{\pgfqpoint{1.762781in}{1.391850in}}{\pgfqpoint{1.768605in}{1.386026in}}%
\pgfpathcurveto{\pgfqpoint{1.774429in}{1.380202in}}{\pgfqpoint{1.782329in}{1.376930in}}{\pgfqpoint{1.790565in}{1.376930in}}%
\pgfpathclose%
\pgfusepath{stroke,fill}%
\end{pgfscope}%
\begin{pgfscope}%
\pgfpathrectangle{\pgfqpoint{0.556847in}{0.516222in}}{\pgfqpoint{1.962733in}{1.783528in}} %
\pgfusepath{clip}%
\pgfsetbuttcap%
\pgfsetroundjoin%
\definecolor{currentfill}{rgb}{0.298039,0.447059,0.690196}%
\pgfsetfillcolor{currentfill}%
\pgfsetlinewidth{0.240900pt}%
\definecolor{currentstroke}{rgb}{1.000000,1.000000,1.000000}%
\pgfsetstrokecolor{currentstroke}%
\pgfsetdash{}{0pt}%
\pgfpathmoveto{\pgfqpoint{1.061550in}{0.693244in}}%
\pgfpathcurveto{\pgfqpoint{1.069786in}{0.693244in}}{\pgfqpoint{1.077686in}{0.696516in}}{\pgfqpoint{1.083510in}{0.702340in}}%
\pgfpathcurveto{\pgfqpoint{1.089334in}{0.708164in}}{\pgfqpoint{1.092606in}{0.716064in}}{\pgfqpoint{1.092606in}{0.724300in}}%
\pgfpathcurveto{\pgfqpoint{1.092606in}{0.732537in}}{\pgfqpoint{1.089334in}{0.740437in}}{\pgfqpoint{1.083510in}{0.746261in}}%
\pgfpathcurveto{\pgfqpoint{1.077686in}{0.752085in}}{\pgfqpoint{1.069786in}{0.755357in}}{\pgfqpoint{1.061550in}{0.755357in}}%
\pgfpathcurveto{\pgfqpoint{1.053314in}{0.755357in}}{\pgfqpoint{1.045414in}{0.752085in}}{\pgfqpoint{1.039590in}{0.746261in}}%
\pgfpathcurveto{\pgfqpoint{1.033766in}{0.740437in}}{\pgfqpoint{1.030493in}{0.732537in}}{\pgfqpoint{1.030493in}{0.724300in}}%
\pgfpathcurveto{\pgfqpoint{1.030493in}{0.716064in}}{\pgfqpoint{1.033766in}{0.708164in}}{\pgfqpoint{1.039590in}{0.702340in}}%
\pgfpathcurveto{\pgfqpoint{1.045414in}{0.696516in}}{\pgfqpoint{1.053314in}{0.693244in}}{\pgfqpoint{1.061550in}{0.693244in}}%
\pgfpathclose%
\pgfusepath{stroke,fill}%
\end{pgfscope}%
\begin{pgfscope}%
\pgfpathrectangle{\pgfqpoint{0.556847in}{0.516222in}}{\pgfqpoint{1.962733in}{1.783528in}} %
\pgfusepath{clip}%
\pgfsetbuttcap%
\pgfsetroundjoin%
\definecolor{currentfill}{rgb}{0.298039,0.447059,0.690196}%
\pgfsetfillcolor{currentfill}%
\pgfsetlinewidth{0.240900pt}%
\definecolor{currentstroke}{rgb}{1.000000,1.000000,1.000000}%
\pgfsetstrokecolor{currentstroke}%
\pgfsetdash{}{0pt}%
\pgfpathmoveto{\pgfqpoint{2.070955in}{2.001164in}}%
\pgfpathcurveto{\pgfqpoint{2.079192in}{2.001164in}}{\pgfqpoint{2.087092in}{2.004437in}}{\pgfqpoint{2.092916in}{2.010261in}}%
\pgfpathcurveto{\pgfqpoint{2.098739in}{2.016085in}}{\pgfqpoint{2.102012in}{2.023985in}}{\pgfqpoint{2.102012in}{2.032221in}}%
\pgfpathcurveto{\pgfqpoint{2.102012in}{2.040457in}}{\pgfqpoint{2.098739in}{2.048357in}}{\pgfqpoint{2.092916in}{2.054181in}}%
\pgfpathcurveto{\pgfqpoint{2.087092in}{2.060005in}}{\pgfqpoint{2.079192in}{2.063277in}}{\pgfqpoint{2.070955in}{2.063277in}}%
\pgfpathcurveto{\pgfqpoint{2.062719in}{2.063277in}}{\pgfqpoint{2.054819in}{2.060005in}}{\pgfqpoint{2.048995in}{2.054181in}}%
\pgfpathcurveto{\pgfqpoint{2.043171in}{2.048357in}}{\pgfqpoint{2.039899in}{2.040457in}}{\pgfqpoint{2.039899in}{2.032221in}}%
\pgfpathcurveto{\pgfqpoint{2.039899in}{2.023985in}}{\pgfqpoint{2.043171in}{2.016085in}}{\pgfqpoint{2.048995in}{2.010261in}}%
\pgfpathcurveto{\pgfqpoint{2.054819in}{2.004437in}}{\pgfqpoint{2.062719in}{2.001164in}}{\pgfqpoint{2.070955in}{2.001164in}}%
\pgfpathclose%
\pgfusepath{stroke,fill}%
\end{pgfscope}%
\begin{pgfscope}%
\pgfpathrectangle{\pgfqpoint{0.556847in}{0.516222in}}{\pgfqpoint{1.962733in}{1.783528in}} %
\pgfusepath{clip}%
\pgfsetbuttcap%
\pgfsetroundjoin%
\definecolor{currentfill}{rgb}{0.298039,0.447059,0.690196}%
\pgfsetfillcolor{currentfill}%
\pgfsetlinewidth{0.240900pt}%
\definecolor{currentstroke}{rgb}{1.000000,1.000000,1.000000}%
\pgfsetstrokecolor{currentstroke}%
\pgfsetdash{}{0pt}%
\pgfpathmoveto{\pgfqpoint{1.958799in}{1.466106in}}%
\pgfpathcurveto{\pgfqpoint{1.967035in}{1.466106in}}{\pgfqpoint{1.974935in}{1.469378in}}{\pgfqpoint{1.980759in}{1.475202in}}%
\pgfpathcurveto{\pgfqpoint{1.986583in}{1.481026in}}{\pgfqpoint{1.989856in}{1.488926in}}{\pgfqpoint{1.989856in}{1.497163in}}%
\pgfpathcurveto{\pgfqpoint{1.989856in}{1.505399in}}{\pgfqpoint{1.986583in}{1.513299in}}{\pgfqpoint{1.980759in}{1.519123in}}%
\pgfpathcurveto{\pgfqpoint{1.974935in}{1.524947in}}{\pgfqpoint{1.967035in}{1.528219in}}{\pgfqpoint{1.958799in}{1.528219in}}%
\pgfpathcurveto{\pgfqpoint{1.950563in}{1.528219in}}{\pgfqpoint{1.942663in}{1.524947in}}{\pgfqpoint{1.936839in}{1.519123in}}%
\pgfpathcurveto{\pgfqpoint{1.931015in}{1.513299in}}{\pgfqpoint{1.927743in}{1.505399in}}{\pgfqpoint{1.927743in}{1.497163in}}%
\pgfpathcurveto{\pgfqpoint{1.927743in}{1.488926in}}{\pgfqpoint{1.931015in}{1.481026in}}{\pgfqpoint{1.936839in}{1.475202in}}%
\pgfpathcurveto{\pgfqpoint{1.942663in}{1.469378in}}{\pgfqpoint{1.950563in}{1.466106in}}{\pgfqpoint{1.958799in}{1.466106in}}%
\pgfpathclose%
\pgfusepath{stroke,fill}%
\end{pgfscope}%
\begin{pgfscope}%
\pgfpathrectangle{\pgfqpoint{0.556847in}{0.516222in}}{\pgfqpoint{1.962733in}{1.783528in}} %
\pgfusepath{clip}%
\pgfsetbuttcap%
\pgfsetroundjoin%
\definecolor{currentfill}{rgb}{0.298039,0.447059,0.690196}%
\pgfsetfillcolor{currentfill}%
\pgfsetlinewidth{0.240900pt}%
\definecolor{currentstroke}{rgb}{1.000000,1.000000,1.000000}%
\pgfsetstrokecolor{currentstroke}%
\pgfsetdash{}{0pt}%
\pgfpathmoveto{\pgfqpoint{1.510175in}{1.258028in}}%
\pgfpathcurveto{\pgfqpoint{1.518411in}{1.258028in}}{\pgfqpoint{1.526311in}{1.261300in}}{\pgfqpoint{1.532135in}{1.267124in}}%
\pgfpathcurveto{\pgfqpoint{1.537959in}{1.272948in}}{\pgfqpoint{1.541231in}{1.280848in}}{\pgfqpoint{1.541231in}{1.289084in}}%
\pgfpathcurveto{\pgfqpoint{1.541231in}{1.297321in}}{\pgfqpoint{1.537959in}{1.305221in}}{\pgfqpoint{1.532135in}{1.311045in}}%
\pgfpathcurveto{\pgfqpoint{1.526311in}{1.316868in}}{\pgfqpoint{1.518411in}{1.320141in}}{\pgfqpoint{1.510175in}{1.320141in}}%
\pgfpathcurveto{\pgfqpoint{1.501938in}{1.320141in}}{\pgfqpoint{1.494038in}{1.316868in}}{\pgfqpoint{1.488214in}{1.311045in}}%
\pgfpathcurveto{\pgfqpoint{1.482390in}{1.305221in}}{\pgfqpoint{1.479118in}{1.297321in}}{\pgfqpoint{1.479118in}{1.289084in}}%
\pgfpathcurveto{\pgfqpoint{1.479118in}{1.280848in}}{\pgfqpoint{1.482390in}{1.272948in}}{\pgfqpoint{1.488214in}{1.267124in}}%
\pgfpathcurveto{\pgfqpoint{1.494038in}{1.261300in}}{\pgfqpoint{1.501938in}{1.258028in}}{\pgfqpoint{1.510175in}{1.258028in}}%
\pgfpathclose%
\pgfusepath{stroke,fill}%
\end{pgfscope}%
\begin{pgfscope}%
\pgfpathrectangle{\pgfqpoint{0.556847in}{0.516222in}}{\pgfqpoint{1.962733in}{1.783528in}} %
\pgfusepath{clip}%
\pgfsetbuttcap%
\pgfsetroundjoin%
\definecolor{currentfill}{rgb}{0.298039,0.447059,0.690196}%
\pgfsetfillcolor{currentfill}%
\pgfsetlinewidth{0.240900pt}%
\definecolor{currentstroke}{rgb}{1.000000,1.000000,1.000000}%
\pgfsetstrokecolor{currentstroke}%
\pgfsetdash{}{0pt}%
\pgfpathmoveto{\pgfqpoint{1.678409in}{0.633793in}}%
\pgfpathcurveto{\pgfqpoint{1.686645in}{0.633793in}}{\pgfqpoint{1.694545in}{0.637065in}}{\pgfqpoint{1.700369in}{0.642889in}}%
\pgfpathcurveto{\pgfqpoint{1.706193in}{0.648713in}}{\pgfqpoint{1.709465in}{0.656613in}}{\pgfqpoint{1.709465in}{0.664850in}}%
\pgfpathcurveto{\pgfqpoint{1.709465in}{0.673086in}}{\pgfqpoint{1.706193in}{0.680986in}}{\pgfqpoint{1.700369in}{0.686810in}}%
\pgfpathcurveto{\pgfqpoint{1.694545in}{0.692634in}}{\pgfqpoint{1.686645in}{0.695906in}}{\pgfqpoint{1.678409in}{0.695906in}}%
\pgfpathcurveto{\pgfqpoint{1.670172in}{0.695906in}}{\pgfqpoint{1.662272in}{0.692634in}}{\pgfqpoint{1.656448in}{0.686810in}}%
\pgfpathcurveto{\pgfqpoint{1.650625in}{0.680986in}}{\pgfqpoint{1.647352in}{0.673086in}}{\pgfqpoint{1.647352in}{0.664850in}}%
\pgfpathcurveto{\pgfqpoint{1.647352in}{0.656613in}}{\pgfqpoint{1.650625in}{0.648713in}}{\pgfqpoint{1.656448in}{0.642889in}}%
\pgfpathcurveto{\pgfqpoint{1.662272in}{0.637065in}}{\pgfqpoint{1.670172in}{0.633793in}}{\pgfqpoint{1.678409in}{0.633793in}}%
\pgfpathclose%
\pgfusepath{stroke,fill}%
\end{pgfscope}%
\begin{pgfscope}%
\pgfpathrectangle{\pgfqpoint{0.556847in}{0.516222in}}{\pgfqpoint{1.962733in}{1.783528in}} %
\pgfusepath{clip}%
\pgfsetbuttcap%
\pgfsetroundjoin%
\definecolor{currentfill}{rgb}{0.298039,0.447059,0.690196}%
\pgfsetfillcolor{currentfill}%
\pgfsetlinewidth{0.240900pt}%
\definecolor{currentstroke}{rgb}{1.000000,1.000000,1.000000}%
\pgfsetstrokecolor{currentstroke}%
\pgfsetdash{}{0pt}%
\pgfpathmoveto{\pgfqpoint{1.285862in}{1.049950in}}%
\pgfpathcurveto{\pgfqpoint{1.294098in}{1.049950in}}{\pgfqpoint{1.301999in}{1.053222in}}{\pgfqpoint{1.307822in}{1.059046in}}%
\pgfpathcurveto{\pgfqpoint{1.313646in}{1.064870in}}{\pgfqpoint{1.316919in}{1.072770in}}{\pgfqpoint{1.316919in}{1.081006in}}%
\pgfpathcurveto{\pgfqpoint{1.316919in}{1.089242in}}{\pgfqpoint{1.313646in}{1.097142in}}{\pgfqpoint{1.307822in}{1.102966in}}%
\pgfpathcurveto{\pgfqpoint{1.301999in}{1.108790in}}{\pgfqpoint{1.294098in}{1.112063in}}{\pgfqpoint{1.285862in}{1.112063in}}%
\pgfpathcurveto{\pgfqpoint{1.277626in}{1.112063in}}{\pgfqpoint{1.269726in}{1.108790in}}{\pgfqpoint{1.263902in}{1.102966in}}%
\pgfpathcurveto{\pgfqpoint{1.258078in}{1.097142in}}{\pgfqpoint{1.254806in}{1.089242in}}{\pgfqpoint{1.254806in}{1.081006in}}%
\pgfpathcurveto{\pgfqpoint{1.254806in}{1.072770in}}{\pgfqpoint{1.258078in}{1.064870in}}{\pgfqpoint{1.263902in}{1.059046in}}%
\pgfpathcurveto{\pgfqpoint{1.269726in}{1.053222in}}{\pgfqpoint{1.277626in}{1.049950in}}{\pgfqpoint{1.285862in}{1.049950in}}%
\pgfpathclose%
\pgfusepath{stroke,fill}%
\end{pgfscope}%
\begin{pgfscope}%
\pgfpathrectangle{\pgfqpoint{0.556847in}{0.516222in}}{\pgfqpoint{1.962733in}{1.783528in}} %
\pgfusepath{clip}%
\pgfsetbuttcap%
\pgfsetroundjoin%
\definecolor{currentfill}{rgb}{0.298039,0.447059,0.690196}%
\pgfsetfillcolor{currentfill}%
\pgfsetlinewidth{0.240900pt}%
\definecolor{currentstroke}{rgb}{1.000000,1.000000,1.000000}%
\pgfsetstrokecolor{currentstroke}%
\pgfsetdash{}{0pt}%
\pgfpathmoveto{\pgfqpoint{1.454096in}{1.139126in}}%
\pgfpathcurveto{\pgfqpoint{1.462333in}{1.139126in}}{\pgfqpoint{1.470233in}{1.142398in}}{\pgfqpoint{1.476057in}{1.148222in}}%
\pgfpathcurveto{\pgfqpoint{1.481881in}{1.154046in}}{\pgfqpoint{1.485153in}{1.161946in}}{\pgfqpoint{1.485153in}{1.170182in}}%
\pgfpathcurveto{\pgfqpoint{1.485153in}{1.178419in}}{\pgfqpoint{1.481881in}{1.186319in}}{\pgfqpoint{1.476057in}{1.192143in}}%
\pgfpathcurveto{\pgfqpoint{1.470233in}{1.197967in}}{\pgfqpoint{1.462333in}{1.201239in}}{\pgfqpoint{1.454096in}{1.201239in}}%
\pgfpathcurveto{\pgfqpoint{1.445860in}{1.201239in}}{\pgfqpoint{1.437960in}{1.197967in}}{\pgfqpoint{1.432136in}{1.192143in}}%
\pgfpathcurveto{\pgfqpoint{1.426312in}{1.186319in}}{\pgfqpoint{1.423040in}{1.178419in}}{\pgfqpoint{1.423040in}{1.170182in}}%
\pgfpathcurveto{\pgfqpoint{1.423040in}{1.161946in}}{\pgfqpoint{1.426312in}{1.154046in}}{\pgfqpoint{1.432136in}{1.148222in}}%
\pgfpathcurveto{\pgfqpoint{1.437960in}{1.142398in}}{\pgfqpoint{1.445860in}{1.139126in}}{\pgfqpoint{1.454096in}{1.139126in}}%
\pgfpathclose%
\pgfusepath{stroke,fill}%
\end{pgfscope}%
\begin{pgfscope}%
\pgfpathrectangle{\pgfqpoint{0.556847in}{0.516222in}}{\pgfqpoint{1.962733in}{1.783528in}} %
\pgfusepath{clip}%
\pgfsetbuttcap%
\pgfsetroundjoin%
\definecolor{currentfill}{rgb}{0.298039,0.447059,0.690196}%
\pgfsetfillcolor{currentfill}%
\pgfsetlinewidth{0.240900pt}%
\definecolor{currentstroke}{rgb}{1.000000,1.000000,1.000000}%
\pgfsetstrokecolor{currentstroke}%
\pgfsetdash{}{0pt}%
\pgfpathmoveto{\pgfqpoint{1.398018in}{1.495831in}}%
\pgfpathcurveto{\pgfqpoint{1.406255in}{1.495831in}}{\pgfqpoint{1.414155in}{1.499104in}}{\pgfqpoint{1.419979in}{1.504928in}}%
\pgfpathcurveto{\pgfqpoint{1.425803in}{1.510752in}}{\pgfqpoint{1.429075in}{1.518652in}}{\pgfqpoint{1.429075in}{1.526888in}}%
\pgfpathcurveto{\pgfqpoint{1.429075in}{1.535124in}}{\pgfqpoint{1.425803in}{1.543024in}}{\pgfqpoint{1.419979in}{1.548848in}}%
\pgfpathcurveto{\pgfqpoint{1.414155in}{1.554672in}}{\pgfqpoint{1.406255in}{1.557944in}}{\pgfqpoint{1.398018in}{1.557944in}}%
\pgfpathcurveto{\pgfqpoint{1.389782in}{1.557944in}}{\pgfqpoint{1.381882in}{1.554672in}}{\pgfqpoint{1.376058in}{1.548848in}}%
\pgfpathcurveto{\pgfqpoint{1.370234in}{1.543024in}}{\pgfqpoint{1.366962in}{1.535124in}}{\pgfqpoint{1.366962in}{1.526888in}}%
\pgfpathcurveto{\pgfqpoint{1.366962in}{1.518652in}}{\pgfqpoint{1.370234in}{1.510752in}}{\pgfqpoint{1.376058in}{1.504928in}}%
\pgfpathcurveto{\pgfqpoint{1.381882in}{1.499104in}}{\pgfqpoint{1.389782in}{1.495831in}}{\pgfqpoint{1.398018in}{1.495831in}}%
\pgfpathclose%
\pgfusepath{stroke,fill}%
\end{pgfscope}%
\begin{pgfscope}%
\pgfpathrectangle{\pgfqpoint{0.556847in}{0.516222in}}{\pgfqpoint{1.962733in}{1.783528in}} %
\pgfusepath{clip}%
\pgfsetbuttcap%
\pgfsetroundjoin%
\definecolor{currentfill}{rgb}{0.298039,0.447059,0.690196}%
\pgfsetfillcolor{currentfill}%
\pgfsetlinewidth{0.240900pt}%
\definecolor{currentstroke}{rgb}{1.000000,1.000000,1.000000}%
\pgfsetstrokecolor{currentstroke}%
\pgfsetdash{}{0pt}%
\pgfpathmoveto{\pgfqpoint{2.014877in}{0.901322in}}%
\pgfpathcurveto{\pgfqpoint{2.023113in}{0.901322in}}{\pgfqpoint{2.031014in}{0.904595in}}{\pgfqpoint{2.036837in}{0.910418in}}%
\pgfpathcurveto{\pgfqpoint{2.042661in}{0.916242in}}{\pgfqpoint{2.045934in}{0.924142in}}{\pgfqpoint{2.045934in}{0.932379in}}%
\pgfpathcurveto{\pgfqpoint{2.045934in}{0.940615in}}{\pgfqpoint{2.042661in}{0.948515in}}{\pgfqpoint{2.036837in}{0.954339in}}%
\pgfpathcurveto{\pgfqpoint{2.031014in}{0.960163in}}{\pgfqpoint{2.023113in}{0.963435in}}{\pgfqpoint{2.014877in}{0.963435in}}%
\pgfpathcurveto{\pgfqpoint{2.006641in}{0.963435in}}{\pgfqpoint{1.998741in}{0.960163in}}{\pgfqpoint{1.992917in}{0.954339in}}%
\pgfpathcurveto{\pgfqpoint{1.987093in}{0.948515in}}{\pgfqpoint{1.983821in}{0.940615in}}{\pgfqpoint{1.983821in}{0.932379in}}%
\pgfpathcurveto{\pgfqpoint{1.983821in}{0.924142in}}{\pgfqpoint{1.987093in}{0.916242in}}{\pgfqpoint{1.992917in}{0.910418in}}%
\pgfpathcurveto{\pgfqpoint{1.998741in}{0.904595in}}{\pgfqpoint{2.006641in}{0.901322in}}{\pgfqpoint{2.014877in}{0.901322in}}%
\pgfpathclose%
\pgfusepath{stroke,fill}%
\end{pgfscope}%
\begin{pgfscope}%
\pgfpathrectangle{\pgfqpoint{0.556847in}{0.516222in}}{\pgfqpoint{1.962733in}{1.783528in}} %
\pgfusepath{clip}%
\pgfsetbuttcap%
\pgfsetroundjoin%
\definecolor{currentfill}{rgb}{0.298039,0.447059,0.690196}%
\pgfsetfillcolor{currentfill}%
\pgfsetlinewidth{0.240900pt}%
\definecolor{currentstroke}{rgb}{1.000000,1.000000,1.000000}%
\pgfsetstrokecolor{currentstroke}%
\pgfsetdash{}{0pt}%
\pgfpathmoveto{\pgfqpoint{1.846643in}{1.644459in}}%
\pgfpathcurveto{\pgfqpoint{1.854879in}{1.644459in}}{\pgfqpoint{1.862779in}{1.647731in}}{\pgfqpoint{1.868603in}{1.653555in}}%
\pgfpathcurveto{\pgfqpoint{1.874427in}{1.659379in}}{\pgfqpoint{1.877699in}{1.667279in}}{\pgfqpoint{1.877699in}{1.675515in}}%
\pgfpathcurveto{\pgfqpoint{1.877699in}{1.683752in}}{\pgfqpoint{1.874427in}{1.691652in}}{\pgfqpoint{1.868603in}{1.697476in}}%
\pgfpathcurveto{\pgfqpoint{1.862779in}{1.703299in}}{\pgfqpoint{1.854879in}{1.706572in}}{\pgfqpoint{1.846643in}{1.706572in}}%
\pgfpathcurveto{\pgfqpoint{1.838407in}{1.706572in}}{\pgfqpoint{1.830507in}{1.703299in}}{\pgfqpoint{1.824683in}{1.697476in}}%
\pgfpathcurveto{\pgfqpoint{1.818859in}{1.691652in}}{\pgfqpoint{1.815586in}{1.683752in}}{\pgfqpoint{1.815586in}{1.675515in}}%
\pgfpathcurveto{\pgfqpoint{1.815586in}{1.667279in}}{\pgfqpoint{1.818859in}{1.659379in}}{\pgfqpoint{1.824683in}{1.653555in}}%
\pgfpathcurveto{\pgfqpoint{1.830507in}{1.647731in}}{\pgfqpoint{1.838407in}{1.644459in}}{\pgfqpoint{1.846643in}{1.644459in}}%
\pgfpathclose%
\pgfusepath{stroke,fill}%
\end{pgfscope}%
\begin{pgfscope}%
\pgfpathrectangle{\pgfqpoint{0.556847in}{0.516222in}}{\pgfqpoint{1.962733in}{1.783528in}} %
\pgfusepath{clip}%
\pgfsetbuttcap%
\pgfsetroundjoin%
\definecolor{currentfill}{rgb}{0.298039,0.447059,0.690196}%
\pgfsetfillcolor{currentfill}%
\pgfsetlinewidth{0.240900pt}%
\definecolor{currentstroke}{rgb}{1.000000,1.000000,1.000000}%
\pgfsetstrokecolor{currentstroke}%
\pgfsetdash{}{0pt}%
\pgfpathmoveto{\pgfqpoint{1.229784in}{0.871597in}}%
\pgfpathcurveto{\pgfqpoint{1.238020in}{0.871597in}}{\pgfqpoint{1.245920in}{0.874869in}}{\pgfqpoint{1.251744in}{0.880693in}}%
\pgfpathcurveto{\pgfqpoint{1.257568in}{0.886517in}}{\pgfqpoint{1.260841in}{0.894417in}}{\pgfqpoint{1.260841in}{0.902653in}}%
\pgfpathcurveto{\pgfqpoint{1.260841in}{0.910890in}}{\pgfqpoint{1.257568in}{0.918790in}}{\pgfqpoint{1.251744in}{0.924614in}}%
\pgfpathcurveto{\pgfqpoint{1.245920in}{0.930437in}}{\pgfqpoint{1.238020in}{0.933710in}}{\pgfqpoint{1.229784in}{0.933710in}}%
\pgfpathcurveto{\pgfqpoint{1.221548in}{0.933710in}}{\pgfqpoint{1.213648in}{0.930437in}}{\pgfqpoint{1.207824in}{0.924614in}}%
\pgfpathcurveto{\pgfqpoint{1.202000in}{0.918790in}}{\pgfqpoint{1.198728in}{0.910890in}}{\pgfqpoint{1.198728in}{0.902653in}}%
\pgfpathcurveto{\pgfqpoint{1.198728in}{0.894417in}}{\pgfqpoint{1.202000in}{0.886517in}}{\pgfqpoint{1.207824in}{0.880693in}}%
\pgfpathcurveto{\pgfqpoint{1.213648in}{0.874869in}}{\pgfqpoint{1.221548in}{0.871597in}}{\pgfqpoint{1.229784in}{0.871597in}}%
\pgfpathclose%
\pgfusepath{stroke,fill}%
\end{pgfscope}%
\begin{pgfscope}%
\pgfpathrectangle{\pgfqpoint{0.556847in}{0.516222in}}{\pgfqpoint{1.962733in}{1.783528in}} %
\pgfusepath{clip}%
\pgfsetbuttcap%
\pgfsetroundjoin%
\definecolor{currentfill}{rgb}{0.298039,0.447059,0.690196}%
\pgfsetfillcolor{currentfill}%
\pgfsetlinewidth{0.240900pt}%
\definecolor{currentstroke}{rgb}{1.000000,1.000000,1.000000}%
\pgfsetstrokecolor{currentstroke}%
\pgfsetdash{}{0pt}%
\pgfpathmoveto{\pgfqpoint{2.183111in}{1.198577in}}%
\pgfpathcurveto{\pgfqpoint{2.191348in}{1.198577in}}{\pgfqpoint{2.199248in}{1.201849in}}{\pgfqpoint{2.205072in}{1.207673in}}%
\pgfpathcurveto{\pgfqpoint{2.210896in}{1.213497in}}{\pgfqpoint{2.214168in}{1.221397in}}{\pgfqpoint{2.214168in}{1.229633in}}%
\pgfpathcurveto{\pgfqpoint{2.214168in}{1.237870in}}{\pgfqpoint{2.210896in}{1.245770in}}{\pgfqpoint{2.205072in}{1.251594in}}%
\pgfpathcurveto{\pgfqpoint{2.199248in}{1.257418in}}{\pgfqpoint{2.191348in}{1.260690in}}{\pgfqpoint{2.183111in}{1.260690in}}%
\pgfpathcurveto{\pgfqpoint{2.174875in}{1.260690in}}{\pgfqpoint{2.166975in}{1.257418in}}{\pgfqpoint{2.161151in}{1.251594in}}%
\pgfpathcurveto{\pgfqpoint{2.155327in}{1.245770in}}{\pgfqpoint{2.152055in}{1.237870in}}{\pgfqpoint{2.152055in}{1.229633in}}%
\pgfpathcurveto{\pgfqpoint{2.152055in}{1.221397in}}{\pgfqpoint{2.155327in}{1.213497in}}{\pgfqpoint{2.161151in}{1.207673in}}%
\pgfpathcurveto{\pgfqpoint{2.166975in}{1.201849in}}{\pgfqpoint{2.174875in}{1.198577in}}{\pgfqpoint{2.183111in}{1.198577in}}%
\pgfpathclose%
\pgfusepath{stroke,fill}%
\end{pgfscope}%
\begin{pgfscope}%
\pgfpathrectangle{\pgfqpoint{0.556847in}{0.516222in}}{\pgfqpoint{1.962733in}{1.783528in}} %
\pgfusepath{clip}%
\pgfsetbuttcap%
\pgfsetroundjoin%
\definecolor{currentfill}{rgb}{0.298039,0.447059,0.690196}%
\pgfsetfillcolor{currentfill}%
\pgfsetlinewidth{0.240900pt}%
\definecolor{currentstroke}{rgb}{1.000000,1.000000,1.000000}%
\pgfsetstrokecolor{currentstroke}%
\pgfsetdash{}{0pt}%
\pgfpathmoveto{\pgfqpoint{2.295268in}{1.406655in}}%
\pgfpathcurveto{\pgfqpoint{2.303504in}{1.406655in}}{\pgfqpoint{2.311404in}{1.409927in}}{\pgfqpoint{2.317228in}{1.415751in}}%
\pgfpathcurveto{\pgfqpoint{2.323052in}{1.421575in}}{\pgfqpoint{2.326324in}{1.429475in}}{\pgfqpoint{2.326324in}{1.437712in}}%
\pgfpathcurveto{\pgfqpoint{2.326324in}{1.445948in}}{\pgfqpoint{2.323052in}{1.453848in}}{\pgfqpoint{2.317228in}{1.459672in}}%
\pgfpathcurveto{\pgfqpoint{2.311404in}{1.465496in}}{\pgfqpoint{2.303504in}{1.468768in}}{\pgfqpoint{2.295268in}{1.468768in}}%
\pgfpathcurveto{\pgfqpoint{2.287031in}{1.468768in}}{\pgfqpoint{2.279131in}{1.465496in}}{\pgfqpoint{2.273307in}{1.459672in}}%
\pgfpathcurveto{\pgfqpoint{2.267483in}{1.453848in}}{\pgfqpoint{2.264211in}{1.445948in}}{\pgfqpoint{2.264211in}{1.437712in}}%
\pgfpathcurveto{\pgfqpoint{2.264211in}{1.429475in}}{\pgfqpoint{2.267483in}{1.421575in}}{\pgfqpoint{2.273307in}{1.415751in}}%
\pgfpathcurveto{\pgfqpoint{2.279131in}{1.409927in}}{\pgfqpoint{2.287031in}{1.406655in}}{\pgfqpoint{2.295268in}{1.406655in}}%
\pgfpathclose%
\pgfusepath{stroke,fill}%
\end{pgfscope}%
\begin{pgfscope}%
\pgfpathrectangle{\pgfqpoint{0.556847in}{0.516222in}}{\pgfqpoint{1.962733in}{1.783528in}} %
\pgfusepath{clip}%
\pgfsetbuttcap%
\pgfsetroundjoin%
\definecolor{currentfill}{rgb}{0.298039,0.447059,0.690196}%
\pgfsetfillcolor{currentfill}%
\pgfsetlinewidth{0.240900pt}%
\definecolor{currentstroke}{rgb}{1.000000,1.000000,1.000000}%
\pgfsetstrokecolor{currentstroke}%
\pgfsetdash{}{0pt}%
\pgfpathmoveto{\pgfqpoint{1.341940in}{1.139126in}}%
\pgfpathcurveto{\pgfqpoint{1.350177in}{1.139126in}}{\pgfqpoint{1.358077in}{1.142398in}}{\pgfqpoint{1.363901in}{1.148222in}}%
\pgfpathcurveto{\pgfqpoint{1.369724in}{1.154046in}}{\pgfqpoint{1.372997in}{1.161946in}}{\pgfqpoint{1.372997in}{1.170182in}}%
\pgfpathcurveto{\pgfqpoint{1.372997in}{1.178419in}}{\pgfqpoint{1.369724in}{1.186319in}}{\pgfqpoint{1.363901in}{1.192143in}}%
\pgfpathcurveto{\pgfqpoint{1.358077in}{1.197967in}}{\pgfqpoint{1.350177in}{1.201239in}}{\pgfqpoint{1.341940in}{1.201239in}}%
\pgfpathcurveto{\pgfqpoint{1.333704in}{1.201239in}}{\pgfqpoint{1.325804in}{1.197967in}}{\pgfqpoint{1.319980in}{1.192143in}}%
\pgfpathcurveto{\pgfqpoint{1.314156in}{1.186319in}}{\pgfqpoint{1.310884in}{1.178419in}}{\pgfqpoint{1.310884in}{1.170182in}}%
\pgfpathcurveto{\pgfqpoint{1.310884in}{1.161946in}}{\pgfqpoint{1.314156in}{1.154046in}}{\pgfqpoint{1.319980in}{1.148222in}}%
\pgfpathcurveto{\pgfqpoint{1.325804in}{1.142398in}}{\pgfqpoint{1.333704in}{1.139126in}}{\pgfqpoint{1.341940in}{1.139126in}}%
\pgfpathclose%
\pgfusepath{stroke,fill}%
\end{pgfscope}%
\begin{pgfscope}%
\pgfpathrectangle{\pgfqpoint{0.556847in}{0.516222in}}{\pgfqpoint{1.962733in}{1.783528in}} %
\pgfusepath{clip}%
\pgfsetbuttcap%
\pgfsetroundjoin%
\definecolor{currentfill}{rgb}{0.298039,0.447059,0.690196}%
\pgfsetfillcolor{currentfill}%
\pgfsetlinewidth{0.240900pt}%
\definecolor{currentstroke}{rgb}{1.000000,1.000000,1.000000}%
\pgfsetstrokecolor{currentstroke}%
\pgfsetdash{}{0pt}%
\pgfpathmoveto{\pgfqpoint{0.837238in}{1.079675in}}%
\pgfpathcurveto{\pgfqpoint{0.845474in}{1.079675in}}{\pgfqpoint{0.853374in}{1.082947in}}{\pgfqpoint{0.859198in}{1.088771in}}%
\pgfpathcurveto{\pgfqpoint{0.865022in}{1.094595in}}{\pgfqpoint{0.868294in}{1.102495in}}{\pgfqpoint{0.868294in}{1.110731in}}%
\pgfpathcurveto{\pgfqpoint{0.868294in}{1.118968in}}{\pgfqpoint{0.865022in}{1.126868in}}{\pgfqpoint{0.859198in}{1.132692in}}%
\pgfpathcurveto{\pgfqpoint{0.853374in}{1.138516in}}{\pgfqpoint{0.845474in}{1.141788in}}{\pgfqpoint{0.837238in}{1.141788in}}%
\pgfpathcurveto{\pgfqpoint{0.829001in}{1.141788in}}{\pgfqpoint{0.821101in}{1.138516in}}{\pgfqpoint{0.815277in}{1.132692in}}%
\pgfpathcurveto{\pgfqpoint{0.809453in}{1.126868in}}{\pgfqpoint{0.806181in}{1.118968in}}{\pgfqpoint{0.806181in}{1.110731in}}%
\pgfpathcurveto{\pgfqpoint{0.806181in}{1.102495in}}{\pgfqpoint{0.809453in}{1.094595in}}{\pgfqpoint{0.815277in}{1.088771in}}%
\pgfpathcurveto{\pgfqpoint{0.821101in}{1.082947in}}{\pgfqpoint{0.829001in}{1.079675in}}{\pgfqpoint{0.837238in}{1.079675in}}%
\pgfpathclose%
\pgfusepath{stroke,fill}%
\end{pgfscope}%
\begin{pgfscope}%
\pgfpathrectangle{\pgfqpoint{0.556847in}{0.516222in}}{\pgfqpoint{1.962733in}{1.783528in}} %
\pgfusepath{clip}%
\pgfsetbuttcap%
\pgfsetroundjoin%
\definecolor{currentfill}{rgb}{0.298039,0.447059,0.690196}%
\pgfsetfillcolor{currentfill}%
\pgfsetlinewidth{0.240900pt}%
\definecolor{currentstroke}{rgb}{1.000000,1.000000,1.000000}%
\pgfsetstrokecolor{currentstroke}%
\pgfsetdash{}{0pt}%
\pgfpathmoveto{\pgfqpoint{1.285862in}{1.168851in}}%
\pgfpathcurveto{\pgfqpoint{1.294098in}{1.168851in}}{\pgfqpoint{1.301999in}{1.172124in}}{\pgfqpoint{1.307822in}{1.177948in}}%
\pgfpathcurveto{\pgfqpoint{1.313646in}{1.183772in}}{\pgfqpoint{1.316919in}{1.191672in}}{\pgfqpoint{1.316919in}{1.199908in}}%
\pgfpathcurveto{\pgfqpoint{1.316919in}{1.208144in}}{\pgfqpoint{1.313646in}{1.216044in}}{\pgfqpoint{1.307822in}{1.221868in}}%
\pgfpathcurveto{\pgfqpoint{1.301999in}{1.227692in}}{\pgfqpoint{1.294098in}{1.230964in}}{\pgfqpoint{1.285862in}{1.230964in}}%
\pgfpathcurveto{\pgfqpoint{1.277626in}{1.230964in}}{\pgfqpoint{1.269726in}{1.227692in}}{\pgfqpoint{1.263902in}{1.221868in}}%
\pgfpathcurveto{\pgfqpoint{1.258078in}{1.216044in}}{\pgfqpoint{1.254806in}{1.208144in}}{\pgfqpoint{1.254806in}{1.199908in}}%
\pgfpathcurveto{\pgfqpoint{1.254806in}{1.191672in}}{\pgfqpoint{1.258078in}{1.183772in}}{\pgfqpoint{1.263902in}{1.177948in}}%
\pgfpathcurveto{\pgfqpoint{1.269726in}{1.172124in}}{\pgfqpoint{1.277626in}{1.168851in}}{\pgfqpoint{1.285862in}{1.168851in}}%
\pgfpathclose%
\pgfusepath{stroke,fill}%
\end{pgfscope}%
\begin{pgfscope}%
\pgfpathrectangle{\pgfqpoint{0.556847in}{0.516222in}}{\pgfqpoint{1.962733in}{1.783528in}} %
\pgfusepath{clip}%
\pgfsetbuttcap%
\pgfsetroundjoin%
\definecolor{currentfill}{rgb}{0.298039,0.447059,0.690196}%
\pgfsetfillcolor{currentfill}%
\pgfsetlinewidth{0.240900pt}%
\definecolor{currentstroke}{rgb}{1.000000,1.000000,1.000000}%
\pgfsetstrokecolor{currentstroke}%
\pgfsetdash{}{0pt}%
\pgfpathmoveto{\pgfqpoint{2.183111in}{1.763361in}}%
\pgfpathcurveto{\pgfqpoint{2.191348in}{1.763361in}}{\pgfqpoint{2.199248in}{1.766633in}}{\pgfqpoint{2.205072in}{1.772457in}}%
\pgfpathcurveto{\pgfqpoint{2.210896in}{1.778281in}}{\pgfqpoint{2.214168in}{1.786181in}}{\pgfqpoint{2.214168in}{1.794417in}}%
\pgfpathcurveto{\pgfqpoint{2.214168in}{1.802653in}}{\pgfqpoint{2.210896in}{1.810553in}}{\pgfqpoint{2.205072in}{1.816377in}}%
\pgfpathcurveto{\pgfqpoint{2.199248in}{1.822201in}}{\pgfqpoint{2.191348in}{1.825474in}}{\pgfqpoint{2.183111in}{1.825474in}}%
\pgfpathcurveto{\pgfqpoint{2.174875in}{1.825474in}}{\pgfqpoint{2.166975in}{1.822201in}}{\pgfqpoint{2.161151in}{1.816377in}}%
\pgfpathcurveto{\pgfqpoint{2.155327in}{1.810553in}}{\pgfqpoint{2.152055in}{1.802653in}}{\pgfqpoint{2.152055in}{1.794417in}}%
\pgfpathcurveto{\pgfqpoint{2.152055in}{1.786181in}}{\pgfqpoint{2.155327in}{1.778281in}}{\pgfqpoint{2.161151in}{1.772457in}}%
\pgfpathcurveto{\pgfqpoint{2.166975in}{1.766633in}}{\pgfqpoint{2.174875in}{1.763361in}}{\pgfqpoint{2.183111in}{1.763361in}}%
\pgfpathclose%
\pgfusepath{stroke,fill}%
\end{pgfscope}%
\begin{pgfscope}%
\pgfsetrectcap%
\pgfsetmiterjoin%
\pgfsetlinewidth{0.000000pt}%
\definecolor{currentstroke}{rgb}{1.000000,1.000000,1.000000}%
\pgfsetstrokecolor{currentstroke}%
\pgfsetdash{}{0pt}%
\pgfpathmoveto{\pgfqpoint{0.556847in}{0.516222in}}%
\pgfpathlineto{\pgfqpoint{0.556847in}{2.299750in}}%
\pgfusepath{}%
\end{pgfscope}%
\begin{pgfscope}%
\pgfsetrectcap%
\pgfsetmiterjoin%
\pgfsetlinewidth{0.000000pt}%
\definecolor{currentstroke}{rgb}{1.000000,1.000000,1.000000}%
\pgfsetstrokecolor{currentstroke}%
\pgfsetdash{}{0pt}%
\pgfpathmoveto{\pgfqpoint{0.556847in}{0.516222in}}%
\pgfpathlineto{\pgfqpoint{2.519580in}{0.516222in}}%
\pgfusepath{}%
\end{pgfscope}%
\begin{pgfscope}%
\pgfsetbuttcap%
\pgfsetmiterjoin%
\definecolor{currentfill}{rgb}{0.917647,0.917647,0.949020}%
\pgfsetfillcolor{currentfill}%
\pgfsetlinewidth{0.000000pt}%
\definecolor{currentstroke}{rgb}{0.000000,0.000000,0.000000}%
\pgfsetstrokecolor{currentstroke}%
\pgfsetstrokeopacity{0.000000}%
\pgfsetdash{}{0pt}%
\pgfpathmoveto{\pgfqpoint{2.816705in}{0.516222in}}%
\pgfpathlineto{\pgfqpoint{4.779438in}{0.516222in}}%
\pgfpathlineto{\pgfqpoint{4.779438in}{2.299750in}}%
\pgfpathlineto{\pgfqpoint{2.816705in}{2.299750in}}%
\pgfpathclose%
\pgfusepath{fill}%
\end{pgfscope}%
\begin{pgfscope}%
\pgfpathrectangle{\pgfqpoint{2.816705in}{0.516222in}}{\pgfqpoint{1.962733in}{1.783528in}} %
\pgfusepath{clip}%
\pgfsetroundcap%
\pgfsetroundjoin%
\pgfsetlinewidth{0.803000pt}%
\definecolor{currentstroke}{rgb}{1.000000,1.000000,1.000000}%
\pgfsetstrokecolor{currentstroke}%
\pgfsetdash{}{0pt}%
\pgfpathmoveto{\pgfqpoint{2.816705in}{0.516222in}}%
\pgfpathlineto{\pgfqpoint{2.816705in}{2.299750in}}%
\pgfusepath{stroke}%
\end{pgfscope}%
\begin{pgfscope}%
\pgfsetbuttcap%
\pgfsetroundjoin%
\definecolor{currentfill}{rgb}{0.150000,0.150000,0.150000}%
\pgfsetfillcolor{currentfill}%
\pgfsetlinewidth{0.803000pt}%
\definecolor{currentstroke}{rgb}{0.150000,0.150000,0.150000}%
\pgfsetstrokecolor{currentstroke}%
\pgfsetdash{}{0pt}%
\pgfsys@defobject{currentmarker}{\pgfqpoint{0.000000in}{0.000000in}}{\pgfqpoint{0.000000in}{0.000000in}}{%
\pgfpathmoveto{\pgfqpoint{0.000000in}{0.000000in}}%
\pgfpathlineto{\pgfqpoint{0.000000in}{0.000000in}}%
\pgfusepath{stroke,fill}%
}%
\begin{pgfscope}%
\pgfsys@transformshift{2.816705in}{0.516222in}%
\pgfsys@useobject{currentmarker}{}%
\end{pgfscope}%
\end{pgfscope}%
\begin{pgfscope}%
\definecolor{textcolor}{rgb}{0.150000,0.150000,0.150000}%
\pgfsetstrokecolor{textcolor}%
\pgfsetfillcolor{textcolor}%
\pgftext[x=2.816705in,y=0.438444in,,top]{\color{textcolor}\sffamily\fontsize{8.000000}{9.600000}\selectfont 2.0}%
\end{pgfscope}%
\begin{pgfscope}%
\pgfpathrectangle{\pgfqpoint{2.816705in}{0.516222in}}{\pgfqpoint{1.962733in}{1.783528in}} %
\pgfusepath{clip}%
\pgfsetroundcap%
\pgfsetroundjoin%
\pgfsetlinewidth{0.803000pt}%
\definecolor{currentstroke}{rgb}{1.000000,1.000000,1.000000}%
\pgfsetstrokecolor{currentstroke}%
\pgfsetdash{}{0pt}%
\pgfpathmoveto{\pgfqpoint{3.097095in}{0.516222in}}%
\pgfpathlineto{\pgfqpoint{3.097095in}{2.299750in}}%
\pgfusepath{stroke}%
\end{pgfscope}%
\begin{pgfscope}%
\pgfsetbuttcap%
\pgfsetroundjoin%
\definecolor{currentfill}{rgb}{0.150000,0.150000,0.150000}%
\pgfsetfillcolor{currentfill}%
\pgfsetlinewidth{0.803000pt}%
\definecolor{currentstroke}{rgb}{0.150000,0.150000,0.150000}%
\pgfsetstrokecolor{currentstroke}%
\pgfsetdash{}{0pt}%
\pgfsys@defobject{currentmarker}{\pgfqpoint{0.000000in}{0.000000in}}{\pgfqpoint{0.000000in}{0.000000in}}{%
\pgfpathmoveto{\pgfqpoint{0.000000in}{0.000000in}}%
\pgfpathlineto{\pgfqpoint{0.000000in}{0.000000in}}%
\pgfusepath{stroke,fill}%
}%
\begin{pgfscope}%
\pgfsys@transformshift{3.097095in}{0.516222in}%
\pgfsys@useobject{currentmarker}{}%
\end{pgfscope}%
\end{pgfscope}%
\begin{pgfscope}%
\definecolor{textcolor}{rgb}{0.150000,0.150000,0.150000}%
\pgfsetstrokecolor{textcolor}%
\pgfsetfillcolor{textcolor}%
\pgftext[x=3.097095in,y=0.438444in,,top]{\color{textcolor}\sffamily\fontsize{8.000000}{9.600000}\selectfont 2.5}%
\end{pgfscope}%
\begin{pgfscope}%
\pgfpathrectangle{\pgfqpoint{2.816705in}{0.516222in}}{\pgfqpoint{1.962733in}{1.783528in}} %
\pgfusepath{clip}%
\pgfsetroundcap%
\pgfsetroundjoin%
\pgfsetlinewidth{0.803000pt}%
\definecolor{currentstroke}{rgb}{1.000000,1.000000,1.000000}%
\pgfsetstrokecolor{currentstroke}%
\pgfsetdash{}{0pt}%
\pgfpathmoveto{\pgfqpoint{3.377486in}{0.516222in}}%
\pgfpathlineto{\pgfqpoint{3.377486in}{2.299750in}}%
\pgfusepath{stroke}%
\end{pgfscope}%
\begin{pgfscope}%
\pgfsetbuttcap%
\pgfsetroundjoin%
\definecolor{currentfill}{rgb}{0.150000,0.150000,0.150000}%
\pgfsetfillcolor{currentfill}%
\pgfsetlinewidth{0.803000pt}%
\definecolor{currentstroke}{rgb}{0.150000,0.150000,0.150000}%
\pgfsetstrokecolor{currentstroke}%
\pgfsetdash{}{0pt}%
\pgfsys@defobject{currentmarker}{\pgfqpoint{0.000000in}{0.000000in}}{\pgfqpoint{0.000000in}{0.000000in}}{%
\pgfpathmoveto{\pgfqpoint{0.000000in}{0.000000in}}%
\pgfpathlineto{\pgfqpoint{0.000000in}{0.000000in}}%
\pgfusepath{stroke,fill}%
}%
\begin{pgfscope}%
\pgfsys@transformshift{3.377486in}{0.516222in}%
\pgfsys@useobject{currentmarker}{}%
\end{pgfscope}%
\end{pgfscope}%
\begin{pgfscope}%
\definecolor{textcolor}{rgb}{0.150000,0.150000,0.150000}%
\pgfsetstrokecolor{textcolor}%
\pgfsetfillcolor{textcolor}%
\pgftext[x=3.377486in,y=0.438444in,,top]{\color{textcolor}\sffamily\fontsize{8.000000}{9.600000}\selectfont 3.0}%
\end{pgfscope}%
\begin{pgfscope}%
\pgfpathrectangle{\pgfqpoint{2.816705in}{0.516222in}}{\pgfqpoint{1.962733in}{1.783528in}} %
\pgfusepath{clip}%
\pgfsetroundcap%
\pgfsetroundjoin%
\pgfsetlinewidth{0.803000pt}%
\definecolor{currentstroke}{rgb}{1.000000,1.000000,1.000000}%
\pgfsetstrokecolor{currentstroke}%
\pgfsetdash{}{0pt}%
\pgfpathmoveto{\pgfqpoint{3.657876in}{0.516222in}}%
\pgfpathlineto{\pgfqpoint{3.657876in}{2.299750in}}%
\pgfusepath{stroke}%
\end{pgfscope}%
\begin{pgfscope}%
\pgfsetbuttcap%
\pgfsetroundjoin%
\definecolor{currentfill}{rgb}{0.150000,0.150000,0.150000}%
\pgfsetfillcolor{currentfill}%
\pgfsetlinewidth{0.803000pt}%
\definecolor{currentstroke}{rgb}{0.150000,0.150000,0.150000}%
\pgfsetstrokecolor{currentstroke}%
\pgfsetdash{}{0pt}%
\pgfsys@defobject{currentmarker}{\pgfqpoint{0.000000in}{0.000000in}}{\pgfqpoint{0.000000in}{0.000000in}}{%
\pgfpathmoveto{\pgfqpoint{0.000000in}{0.000000in}}%
\pgfpathlineto{\pgfqpoint{0.000000in}{0.000000in}}%
\pgfusepath{stroke,fill}%
}%
\begin{pgfscope}%
\pgfsys@transformshift{3.657876in}{0.516222in}%
\pgfsys@useobject{currentmarker}{}%
\end{pgfscope}%
\end{pgfscope}%
\begin{pgfscope}%
\definecolor{textcolor}{rgb}{0.150000,0.150000,0.150000}%
\pgfsetstrokecolor{textcolor}%
\pgfsetfillcolor{textcolor}%
\pgftext[x=3.657876in,y=0.438444in,,top]{\color{textcolor}\sffamily\fontsize{8.000000}{9.600000}\selectfont 3.5}%
\end{pgfscope}%
\begin{pgfscope}%
\pgfpathrectangle{\pgfqpoint{2.816705in}{0.516222in}}{\pgfqpoint{1.962733in}{1.783528in}} %
\pgfusepath{clip}%
\pgfsetroundcap%
\pgfsetroundjoin%
\pgfsetlinewidth{0.803000pt}%
\definecolor{currentstroke}{rgb}{1.000000,1.000000,1.000000}%
\pgfsetstrokecolor{currentstroke}%
\pgfsetdash{}{0pt}%
\pgfpathmoveto{\pgfqpoint{3.938266in}{0.516222in}}%
\pgfpathlineto{\pgfqpoint{3.938266in}{2.299750in}}%
\pgfusepath{stroke}%
\end{pgfscope}%
\begin{pgfscope}%
\pgfsetbuttcap%
\pgfsetroundjoin%
\definecolor{currentfill}{rgb}{0.150000,0.150000,0.150000}%
\pgfsetfillcolor{currentfill}%
\pgfsetlinewidth{0.803000pt}%
\definecolor{currentstroke}{rgb}{0.150000,0.150000,0.150000}%
\pgfsetstrokecolor{currentstroke}%
\pgfsetdash{}{0pt}%
\pgfsys@defobject{currentmarker}{\pgfqpoint{0.000000in}{0.000000in}}{\pgfqpoint{0.000000in}{0.000000in}}{%
\pgfpathmoveto{\pgfqpoint{0.000000in}{0.000000in}}%
\pgfpathlineto{\pgfqpoint{0.000000in}{0.000000in}}%
\pgfusepath{stroke,fill}%
}%
\begin{pgfscope}%
\pgfsys@transformshift{3.938266in}{0.516222in}%
\pgfsys@useobject{currentmarker}{}%
\end{pgfscope}%
\end{pgfscope}%
\begin{pgfscope}%
\definecolor{textcolor}{rgb}{0.150000,0.150000,0.150000}%
\pgfsetstrokecolor{textcolor}%
\pgfsetfillcolor{textcolor}%
\pgftext[x=3.938266in,y=0.438444in,,top]{\color{textcolor}\sffamily\fontsize{8.000000}{9.600000}\selectfont 4.0}%
\end{pgfscope}%
\begin{pgfscope}%
\pgfpathrectangle{\pgfqpoint{2.816705in}{0.516222in}}{\pgfqpoint{1.962733in}{1.783528in}} %
\pgfusepath{clip}%
\pgfsetroundcap%
\pgfsetroundjoin%
\pgfsetlinewidth{0.803000pt}%
\definecolor{currentstroke}{rgb}{1.000000,1.000000,1.000000}%
\pgfsetstrokecolor{currentstroke}%
\pgfsetdash{}{0pt}%
\pgfpathmoveto{\pgfqpoint{4.218657in}{0.516222in}}%
\pgfpathlineto{\pgfqpoint{4.218657in}{2.299750in}}%
\pgfusepath{stroke}%
\end{pgfscope}%
\begin{pgfscope}%
\pgfsetbuttcap%
\pgfsetroundjoin%
\definecolor{currentfill}{rgb}{0.150000,0.150000,0.150000}%
\pgfsetfillcolor{currentfill}%
\pgfsetlinewidth{0.803000pt}%
\definecolor{currentstroke}{rgb}{0.150000,0.150000,0.150000}%
\pgfsetstrokecolor{currentstroke}%
\pgfsetdash{}{0pt}%
\pgfsys@defobject{currentmarker}{\pgfqpoint{0.000000in}{0.000000in}}{\pgfqpoint{0.000000in}{0.000000in}}{%
\pgfpathmoveto{\pgfqpoint{0.000000in}{0.000000in}}%
\pgfpathlineto{\pgfqpoint{0.000000in}{0.000000in}}%
\pgfusepath{stroke,fill}%
}%
\begin{pgfscope}%
\pgfsys@transformshift{4.218657in}{0.516222in}%
\pgfsys@useobject{currentmarker}{}%
\end{pgfscope}%
\end{pgfscope}%
\begin{pgfscope}%
\definecolor{textcolor}{rgb}{0.150000,0.150000,0.150000}%
\pgfsetstrokecolor{textcolor}%
\pgfsetfillcolor{textcolor}%
\pgftext[x=4.218657in,y=0.438444in,,top]{\color{textcolor}\sffamily\fontsize{8.000000}{9.600000}\selectfont 4.5}%
\end{pgfscope}%
\begin{pgfscope}%
\pgfpathrectangle{\pgfqpoint{2.816705in}{0.516222in}}{\pgfqpoint{1.962733in}{1.783528in}} %
\pgfusepath{clip}%
\pgfsetroundcap%
\pgfsetroundjoin%
\pgfsetlinewidth{0.803000pt}%
\definecolor{currentstroke}{rgb}{1.000000,1.000000,1.000000}%
\pgfsetstrokecolor{currentstroke}%
\pgfsetdash{}{0pt}%
\pgfpathmoveto{\pgfqpoint{4.499047in}{0.516222in}}%
\pgfpathlineto{\pgfqpoint{4.499047in}{2.299750in}}%
\pgfusepath{stroke}%
\end{pgfscope}%
\begin{pgfscope}%
\pgfsetbuttcap%
\pgfsetroundjoin%
\definecolor{currentfill}{rgb}{0.150000,0.150000,0.150000}%
\pgfsetfillcolor{currentfill}%
\pgfsetlinewidth{0.803000pt}%
\definecolor{currentstroke}{rgb}{0.150000,0.150000,0.150000}%
\pgfsetstrokecolor{currentstroke}%
\pgfsetdash{}{0pt}%
\pgfsys@defobject{currentmarker}{\pgfqpoint{0.000000in}{0.000000in}}{\pgfqpoint{0.000000in}{0.000000in}}{%
\pgfpathmoveto{\pgfqpoint{0.000000in}{0.000000in}}%
\pgfpathlineto{\pgfqpoint{0.000000in}{0.000000in}}%
\pgfusepath{stroke,fill}%
}%
\begin{pgfscope}%
\pgfsys@transformshift{4.499047in}{0.516222in}%
\pgfsys@useobject{currentmarker}{}%
\end{pgfscope}%
\end{pgfscope}%
\begin{pgfscope}%
\definecolor{textcolor}{rgb}{0.150000,0.150000,0.150000}%
\pgfsetstrokecolor{textcolor}%
\pgfsetfillcolor{textcolor}%
\pgftext[x=4.499047in,y=0.438444in,,top]{\color{textcolor}\sffamily\fontsize{8.000000}{9.600000}\selectfont 5.0}%
\end{pgfscope}%
\begin{pgfscope}%
\pgfpathrectangle{\pgfqpoint{2.816705in}{0.516222in}}{\pgfqpoint{1.962733in}{1.783528in}} %
\pgfusepath{clip}%
\pgfsetroundcap%
\pgfsetroundjoin%
\pgfsetlinewidth{0.803000pt}%
\definecolor{currentstroke}{rgb}{1.000000,1.000000,1.000000}%
\pgfsetstrokecolor{currentstroke}%
\pgfsetdash{}{0pt}%
\pgfpathmoveto{\pgfqpoint{4.779438in}{0.516222in}}%
\pgfpathlineto{\pgfqpoint{4.779438in}{2.299750in}}%
\pgfusepath{stroke}%
\end{pgfscope}%
\begin{pgfscope}%
\pgfsetbuttcap%
\pgfsetroundjoin%
\definecolor{currentfill}{rgb}{0.150000,0.150000,0.150000}%
\pgfsetfillcolor{currentfill}%
\pgfsetlinewidth{0.803000pt}%
\definecolor{currentstroke}{rgb}{0.150000,0.150000,0.150000}%
\pgfsetstrokecolor{currentstroke}%
\pgfsetdash{}{0pt}%
\pgfsys@defobject{currentmarker}{\pgfqpoint{0.000000in}{0.000000in}}{\pgfqpoint{0.000000in}{0.000000in}}{%
\pgfpathmoveto{\pgfqpoint{0.000000in}{0.000000in}}%
\pgfpathlineto{\pgfqpoint{0.000000in}{0.000000in}}%
\pgfusepath{stroke,fill}%
}%
\begin{pgfscope}%
\pgfsys@transformshift{4.779438in}{0.516222in}%
\pgfsys@useobject{currentmarker}{}%
\end{pgfscope}%
\end{pgfscope}%
\begin{pgfscope}%
\definecolor{textcolor}{rgb}{0.150000,0.150000,0.150000}%
\pgfsetstrokecolor{textcolor}%
\pgfsetfillcolor{textcolor}%
\pgftext[x=4.779438in,y=0.438444in,,top]{\color{textcolor}\sffamily\fontsize{8.000000}{9.600000}\selectfont 5.5}%
\end{pgfscope}%
\begin{pgfscope}%
\definecolor{textcolor}{rgb}{0.150000,0.150000,0.150000}%
\pgfsetstrokecolor{textcolor}%
\pgfsetfillcolor{textcolor}%
\pgftext[x=3.798071in,y=0.273321in,,top]{\color{textcolor}\sffamily\fontsize{8.800000}{10.560000}\selectfont Falling time realization 2}%
\end{pgfscope}%
\begin{pgfscope}%
\pgfpathrectangle{\pgfqpoint{2.816705in}{0.516222in}}{\pgfqpoint{1.962733in}{1.783528in}} %
\pgfusepath{clip}%
\pgfsetroundcap%
\pgfsetroundjoin%
\pgfsetlinewidth{0.803000pt}%
\definecolor{currentstroke}{rgb}{1.000000,1.000000,1.000000}%
\pgfsetstrokecolor{currentstroke}%
\pgfsetdash{}{0pt}%
\pgfpathmoveto{\pgfqpoint{2.816705in}{0.516222in}}%
\pgfpathlineto{\pgfqpoint{4.779438in}{0.516222in}}%
\pgfusepath{stroke}%
\end{pgfscope}%
\begin{pgfscope}%
\pgfsetbuttcap%
\pgfsetroundjoin%
\definecolor{currentfill}{rgb}{0.150000,0.150000,0.150000}%
\pgfsetfillcolor{currentfill}%
\pgfsetlinewidth{0.803000pt}%
\definecolor{currentstroke}{rgb}{0.150000,0.150000,0.150000}%
\pgfsetstrokecolor{currentstroke}%
\pgfsetdash{}{0pt}%
\pgfsys@defobject{currentmarker}{\pgfqpoint{0.000000in}{0.000000in}}{\pgfqpoint{0.000000in}{0.000000in}}{%
\pgfpathmoveto{\pgfqpoint{0.000000in}{0.000000in}}%
\pgfpathlineto{\pgfqpoint{0.000000in}{0.000000in}}%
\pgfusepath{stroke,fill}%
}%
\begin{pgfscope}%
\pgfsys@transformshift{2.816705in}{0.516222in}%
\pgfsys@useobject{currentmarker}{}%
\end{pgfscope}%
\end{pgfscope}%
\begin{pgfscope}%
\pgfpathrectangle{\pgfqpoint{2.816705in}{0.516222in}}{\pgfqpoint{1.962733in}{1.783528in}} %
\pgfusepath{clip}%
\pgfsetroundcap%
\pgfsetroundjoin%
\pgfsetlinewidth{0.803000pt}%
\definecolor{currentstroke}{rgb}{1.000000,1.000000,1.000000}%
\pgfsetstrokecolor{currentstroke}%
\pgfsetdash{}{0pt}%
\pgfpathmoveto{\pgfqpoint{2.816705in}{0.813477in}}%
\pgfpathlineto{\pgfqpoint{4.779438in}{0.813477in}}%
\pgfusepath{stroke}%
\end{pgfscope}%
\begin{pgfscope}%
\pgfsetbuttcap%
\pgfsetroundjoin%
\definecolor{currentfill}{rgb}{0.150000,0.150000,0.150000}%
\pgfsetfillcolor{currentfill}%
\pgfsetlinewidth{0.803000pt}%
\definecolor{currentstroke}{rgb}{0.150000,0.150000,0.150000}%
\pgfsetstrokecolor{currentstroke}%
\pgfsetdash{}{0pt}%
\pgfsys@defobject{currentmarker}{\pgfqpoint{0.000000in}{0.000000in}}{\pgfqpoint{0.000000in}{0.000000in}}{%
\pgfpathmoveto{\pgfqpoint{0.000000in}{0.000000in}}%
\pgfpathlineto{\pgfqpoint{0.000000in}{0.000000in}}%
\pgfusepath{stroke,fill}%
}%
\begin{pgfscope}%
\pgfsys@transformshift{2.816705in}{0.813477in}%
\pgfsys@useobject{currentmarker}{}%
\end{pgfscope}%
\end{pgfscope}%
\begin{pgfscope}%
\pgfpathrectangle{\pgfqpoint{2.816705in}{0.516222in}}{\pgfqpoint{1.962733in}{1.783528in}} %
\pgfusepath{clip}%
\pgfsetroundcap%
\pgfsetroundjoin%
\pgfsetlinewidth{0.803000pt}%
\definecolor{currentstroke}{rgb}{1.000000,1.000000,1.000000}%
\pgfsetstrokecolor{currentstroke}%
\pgfsetdash{}{0pt}%
\pgfpathmoveto{\pgfqpoint{2.816705in}{1.110731in}}%
\pgfpathlineto{\pgfqpoint{4.779438in}{1.110731in}}%
\pgfusepath{stroke}%
\end{pgfscope}%
\begin{pgfscope}%
\pgfsetbuttcap%
\pgfsetroundjoin%
\definecolor{currentfill}{rgb}{0.150000,0.150000,0.150000}%
\pgfsetfillcolor{currentfill}%
\pgfsetlinewidth{0.803000pt}%
\definecolor{currentstroke}{rgb}{0.150000,0.150000,0.150000}%
\pgfsetstrokecolor{currentstroke}%
\pgfsetdash{}{0pt}%
\pgfsys@defobject{currentmarker}{\pgfqpoint{0.000000in}{0.000000in}}{\pgfqpoint{0.000000in}{0.000000in}}{%
\pgfpathmoveto{\pgfqpoint{0.000000in}{0.000000in}}%
\pgfpathlineto{\pgfqpoint{0.000000in}{0.000000in}}%
\pgfusepath{stroke,fill}%
}%
\begin{pgfscope}%
\pgfsys@transformshift{2.816705in}{1.110731in}%
\pgfsys@useobject{currentmarker}{}%
\end{pgfscope}%
\end{pgfscope}%
\begin{pgfscope}%
\pgfpathrectangle{\pgfqpoint{2.816705in}{0.516222in}}{\pgfqpoint{1.962733in}{1.783528in}} %
\pgfusepath{clip}%
\pgfsetroundcap%
\pgfsetroundjoin%
\pgfsetlinewidth{0.803000pt}%
\definecolor{currentstroke}{rgb}{1.000000,1.000000,1.000000}%
\pgfsetstrokecolor{currentstroke}%
\pgfsetdash{}{0pt}%
\pgfpathmoveto{\pgfqpoint{2.816705in}{1.407986in}}%
\pgfpathlineto{\pgfqpoint{4.779438in}{1.407986in}}%
\pgfusepath{stroke}%
\end{pgfscope}%
\begin{pgfscope}%
\pgfsetbuttcap%
\pgfsetroundjoin%
\definecolor{currentfill}{rgb}{0.150000,0.150000,0.150000}%
\pgfsetfillcolor{currentfill}%
\pgfsetlinewidth{0.803000pt}%
\definecolor{currentstroke}{rgb}{0.150000,0.150000,0.150000}%
\pgfsetstrokecolor{currentstroke}%
\pgfsetdash{}{0pt}%
\pgfsys@defobject{currentmarker}{\pgfqpoint{0.000000in}{0.000000in}}{\pgfqpoint{0.000000in}{0.000000in}}{%
\pgfpathmoveto{\pgfqpoint{0.000000in}{0.000000in}}%
\pgfpathlineto{\pgfqpoint{0.000000in}{0.000000in}}%
\pgfusepath{stroke,fill}%
}%
\begin{pgfscope}%
\pgfsys@transformshift{2.816705in}{1.407986in}%
\pgfsys@useobject{currentmarker}{}%
\end{pgfscope}%
\end{pgfscope}%
\begin{pgfscope}%
\pgfpathrectangle{\pgfqpoint{2.816705in}{0.516222in}}{\pgfqpoint{1.962733in}{1.783528in}} %
\pgfusepath{clip}%
\pgfsetroundcap%
\pgfsetroundjoin%
\pgfsetlinewidth{0.803000pt}%
\definecolor{currentstroke}{rgb}{1.000000,1.000000,1.000000}%
\pgfsetstrokecolor{currentstroke}%
\pgfsetdash{}{0pt}%
\pgfpathmoveto{\pgfqpoint{2.816705in}{1.705241in}}%
\pgfpathlineto{\pgfqpoint{4.779438in}{1.705241in}}%
\pgfusepath{stroke}%
\end{pgfscope}%
\begin{pgfscope}%
\pgfsetbuttcap%
\pgfsetroundjoin%
\definecolor{currentfill}{rgb}{0.150000,0.150000,0.150000}%
\pgfsetfillcolor{currentfill}%
\pgfsetlinewidth{0.803000pt}%
\definecolor{currentstroke}{rgb}{0.150000,0.150000,0.150000}%
\pgfsetstrokecolor{currentstroke}%
\pgfsetdash{}{0pt}%
\pgfsys@defobject{currentmarker}{\pgfqpoint{0.000000in}{0.000000in}}{\pgfqpoint{0.000000in}{0.000000in}}{%
\pgfpathmoveto{\pgfqpoint{0.000000in}{0.000000in}}%
\pgfpathlineto{\pgfqpoint{0.000000in}{0.000000in}}%
\pgfusepath{stroke,fill}%
}%
\begin{pgfscope}%
\pgfsys@transformshift{2.816705in}{1.705241in}%
\pgfsys@useobject{currentmarker}{}%
\end{pgfscope}%
\end{pgfscope}%
\begin{pgfscope}%
\pgfpathrectangle{\pgfqpoint{2.816705in}{0.516222in}}{\pgfqpoint{1.962733in}{1.783528in}} %
\pgfusepath{clip}%
\pgfsetroundcap%
\pgfsetroundjoin%
\pgfsetlinewidth{0.803000pt}%
\definecolor{currentstroke}{rgb}{1.000000,1.000000,1.000000}%
\pgfsetstrokecolor{currentstroke}%
\pgfsetdash{}{0pt}%
\pgfpathmoveto{\pgfqpoint{2.816705in}{2.002495in}}%
\pgfpathlineto{\pgfqpoint{4.779438in}{2.002495in}}%
\pgfusepath{stroke}%
\end{pgfscope}%
\begin{pgfscope}%
\pgfsetbuttcap%
\pgfsetroundjoin%
\definecolor{currentfill}{rgb}{0.150000,0.150000,0.150000}%
\pgfsetfillcolor{currentfill}%
\pgfsetlinewidth{0.803000pt}%
\definecolor{currentstroke}{rgb}{0.150000,0.150000,0.150000}%
\pgfsetstrokecolor{currentstroke}%
\pgfsetdash{}{0pt}%
\pgfsys@defobject{currentmarker}{\pgfqpoint{0.000000in}{0.000000in}}{\pgfqpoint{0.000000in}{0.000000in}}{%
\pgfpathmoveto{\pgfqpoint{0.000000in}{0.000000in}}%
\pgfpathlineto{\pgfqpoint{0.000000in}{0.000000in}}%
\pgfusepath{stroke,fill}%
}%
\begin{pgfscope}%
\pgfsys@transformshift{2.816705in}{2.002495in}%
\pgfsys@useobject{currentmarker}{}%
\end{pgfscope}%
\end{pgfscope}%
\begin{pgfscope}%
\pgfpathrectangle{\pgfqpoint{2.816705in}{0.516222in}}{\pgfqpoint{1.962733in}{1.783528in}} %
\pgfusepath{clip}%
\pgfsetroundcap%
\pgfsetroundjoin%
\pgfsetlinewidth{0.803000pt}%
\definecolor{currentstroke}{rgb}{1.000000,1.000000,1.000000}%
\pgfsetstrokecolor{currentstroke}%
\pgfsetdash{}{0pt}%
\pgfpathmoveto{\pgfqpoint{2.816705in}{2.299750in}}%
\pgfpathlineto{\pgfqpoint{4.779438in}{2.299750in}}%
\pgfusepath{stroke}%
\end{pgfscope}%
\begin{pgfscope}%
\pgfsetbuttcap%
\pgfsetroundjoin%
\definecolor{currentfill}{rgb}{0.150000,0.150000,0.150000}%
\pgfsetfillcolor{currentfill}%
\pgfsetlinewidth{0.803000pt}%
\definecolor{currentstroke}{rgb}{0.150000,0.150000,0.150000}%
\pgfsetstrokecolor{currentstroke}%
\pgfsetdash{}{0pt}%
\pgfsys@defobject{currentmarker}{\pgfqpoint{0.000000in}{0.000000in}}{\pgfqpoint{0.000000in}{0.000000in}}{%
\pgfpathmoveto{\pgfqpoint{0.000000in}{0.000000in}}%
\pgfpathlineto{\pgfqpoint{0.000000in}{0.000000in}}%
\pgfusepath{stroke,fill}%
}%
\begin{pgfscope}%
\pgfsys@transformshift{2.816705in}{2.299750in}%
\pgfsys@useobject{currentmarker}{}%
\end{pgfscope}%
\end{pgfscope}%
\begin{pgfscope}%
\pgfpathrectangle{\pgfqpoint{2.816705in}{0.516222in}}{\pgfqpoint{1.962733in}{1.783528in}} %
\pgfusepath{clip}%
\pgfsetbuttcap%
\pgfsetroundjoin%
\definecolor{currentfill}{rgb}{0.298039,0.447059,0.690196}%
\pgfsetfillcolor{currentfill}%
\pgfsetlinewidth{0.240900pt}%
\definecolor{currentstroke}{rgb}{1.000000,1.000000,1.000000}%
\pgfsetstrokecolor{currentstroke}%
\pgfsetdash{}{0pt}%
\pgfpathmoveto{\pgfqpoint{3.826110in}{1.287753in}}%
\pgfpathcurveto{\pgfqpoint{3.834346in}{1.287753in}}{\pgfqpoint{3.842247in}{1.291026in}}{\pgfqpoint{3.848070in}{1.296849in}}%
\pgfpathcurveto{\pgfqpoint{3.853894in}{1.302673in}}{\pgfqpoint{3.857167in}{1.310573in}}{\pgfqpoint{3.857167in}{1.318810in}}%
\pgfpathcurveto{\pgfqpoint{3.857167in}{1.327046in}}{\pgfqpoint{3.853894in}{1.334946in}}{\pgfqpoint{3.848070in}{1.340770in}}%
\pgfpathcurveto{\pgfqpoint{3.842247in}{1.346594in}}{\pgfqpoint{3.834346in}{1.349866in}}{\pgfqpoint{3.826110in}{1.349866in}}%
\pgfpathcurveto{\pgfqpoint{3.817874in}{1.349866in}}{\pgfqpoint{3.809974in}{1.346594in}}{\pgfqpoint{3.804150in}{1.340770in}}%
\pgfpathcurveto{\pgfqpoint{3.798326in}{1.334946in}}{\pgfqpoint{3.795054in}{1.327046in}}{\pgfqpoint{3.795054in}{1.318810in}}%
\pgfpathcurveto{\pgfqpoint{3.795054in}{1.310573in}}{\pgfqpoint{3.798326in}{1.302673in}}{\pgfqpoint{3.804150in}{1.296849in}}%
\pgfpathcurveto{\pgfqpoint{3.809974in}{1.291026in}}{\pgfqpoint{3.817874in}{1.287753in}}{\pgfqpoint{3.826110in}{1.287753in}}%
\pgfpathclose%
\pgfusepath{stroke,fill}%
\end{pgfscope}%
\begin{pgfscope}%
\pgfpathrectangle{\pgfqpoint{2.816705in}{0.516222in}}{\pgfqpoint{1.962733in}{1.783528in}} %
\pgfusepath{clip}%
\pgfsetbuttcap%
\pgfsetroundjoin%
\definecolor{currentfill}{rgb}{0.298039,0.447059,0.690196}%
\pgfsetfillcolor{currentfill}%
\pgfsetlinewidth{0.240900pt}%
\definecolor{currentstroke}{rgb}{1.000000,1.000000,1.000000}%
\pgfsetstrokecolor{currentstroke}%
\pgfsetdash{}{0pt}%
\pgfpathmoveto{\pgfqpoint{3.657876in}{1.763361in}}%
\pgfpathcurveto{\pgfqpoint{3.666112in}{1.763361in}}{\pgfqpoint{3.674012in}{1.766633in}}{\pgfqpoint{3.679836in}{1.772457in}}%
\pgfpathcurveto{\pgfqpoint{3.685660in}{1.778281in}}{\pgfqpoint{3.688932in}{1.786181in}}{\pgfqpoint{3.688932in}{1.794417in}}%
\pgfpathcurveto{\pgfqpoint{3.688932in}{1.802653in}}{\pgfqpoint{3.685660in}{1.810553in}}{\pgfqpoint{3.679836in}{1.816377in}}%
\pgfpathcurveto{\pgfqpoint{3.674012in}{1.822201in}}{\pgfqpoint{3.666112in}{1.825474in}}{\pgfqpoint{3.657876in}{1.825474in}}%
\pgfpathcurveto{\pgfqpoint{3.649640in}{1.825474in}}{\pgfqpoint{3.641740in}{1.822201in}}{\pgfqpoint{3.635916in}{1.816377in}}%
\pgfpathcurveto{\pgfqpoint{3.630092in}{1.810553in}}{\pgfqpoint{3.626819in}{1.802653in}}{\pgfqpoint{3.626819in}{1.794417in}}%
\pgfpathcurveto{\pgfqpoint{3.626819in}{1.786181in}}{\pgfqpoint{3.630092in}{1.778281in}}{\pgfqpoint{3.635916in}{1.772457in}}%
\pgfpathcurveto{\pgfqpoint{3.641740in}{1.766633in}}{\pgfqpoint{3.649640in}{1.763361in}}{\pgfqpoint{3.657876in}{1.763361in}}%
\pgfpathclose%
\pgfusepath{stroke,fill}%
\end{pgfscope}%
\begin{pgfscope}%
\pgfpathrectangle{\pgfqpoint{2.816705in}{0.516222in}}{\pgfqpoint{1.962733in}{1.783528in}} %
\pgfusepath{clip}%
\pgfsetbuttcap%
\pgfsetroundjoin%
\definecolor{currentfill}{rgb}{0.298039,0.447059,0.690196}%
\pgfsetfillcolor{currentfill}%
\pgfsetlinewidth{0.240900pt}%
\definecolor{currentstroke}{rgb}{1.000000,1.000000,1.000000}%
\pgfsetstrokecolor{currentstroke}%
\pgfsetdash{}{0pt}%
\pgfpathmoveto{\pgfqpoint{3.713954in}{1.436381in}}%
\pgfpathcurveto{\pgfqpoint{3.722190in}{1.436381in}}{\pgfqpoint{3.730090in}{1.439653in}}{\pgfqpoint{3.735914in}{1.445477in}}%
\pgfpathcurveto{\pgfqpoint{3.741738in}{1.451301in}}{\pgfqpoint{3.745011in}{1.459201in}}{\pgfqpoint{3.745011in}{1.467437in}}%
\pgfpathcurveto{\pgfqpoint{3.745011in}{1.475673in}}{\pgfqpoint{3.741738in}{1.483573in}}{\pgfqpoint{3.735914in}{1.489397in}}%
\pgfpathcurveto{\pgfqpoint{3.730090in}{1.495221in}}{\pgfqpoint{3.722190in}{1.498494in}}{\pgfqpoint{3.713954in}{1.498494in}}%
\pgfpathcurveto{\pgfqpoint{3.705718in}{1.498494in}}{\pgfqpoint{3.697818in}{1.495221in}}{\pgfqpoint{3.691994in}{1.489397in}}%
\pgfpathcurveto{\pgfqpoint{3.686170in}{1.483573in}}{\pgfqpoint{3.682898in}{1.475673in}}{\pgfqpoint{3.682898in}{1.467437in}}%
\pgfpathcurveto{\pgfqpoint{3.682898in}{1.459201in}}{\pgfqpoint{3.686170in}{1.451301in}}{\pgfqpoint{3.691994in}{1.445477in}}%
\pgfpathcurveto{\pgfqpoint{3.697818in}{1.439653in}}{\pgfqpoint{3.705718in}{1.436381in}}{\pgfqpoint{3.713954in}{1.436381in}}%
\pgfpathclose%
\pgfusepath{stroke,fill}%
\end{pgfscope}%
\begin{pgfscope}%
\pgfpathrectangle{\pgfqpoint{2.816705in}{0.516222in}}{\pgfqpoint{1.962733in}{1.783528in}} %
\pgfusepath{clip}%
\pgfsetbuttcap%
\pgfsetroundjoin%
\definecolor{currentfill}{rgb}{0.298039,0.447059,0.690196}%
\pgfsetfillcolor{currentfill}%
\pgfsetlinewidth{0.240900pt}%
\definecolor{currentstroke}{rgb}{1.000000,1.000000,1.000000}%
\pgfsetstrokecolor{currentstroke}%
\pgfsetdash{}{0pt}%
\pgfpathmoveto{\pgfqpoint{4.162579in}{1.822812in}}%
\pgfpathcurveto{\pgfqpoint{4.170815in}{1.822812in}}{\pgfqpoint{4.178715in}{1.826084in}}{\pgfqpoint{4.184539in}{1.831908in}}%
\pgfpathcurveto{\pgfqpoint{4.190363in}{1.837732in}}{\pgfqpoint{4.193635in}{1.845632in}}{\pgfqpoint{4.193635in}{1.853868in}}%
\pgfpathcurveto{\pgfqpoint{4.193635in}{1.862104in}}{\pgfqpoint{4.190363in}{1.870004in}}{\pgfqpoint{4.184539in}{1.875828in}}%
\pgfpathcurveto{\pgfqpoint{4.178715in}{1.881652in}}{\pgfqpoint{4.170815in}{1.884925in}}{\pgfqpoint{4.162579in}{1.884925in}}%
\pgfpathcurveto{\pgfqpoint{4.154342in}{1.884925in}}{\pgfqpoint{4.146442in}{1.881652in}}{\pgfqpoint{4.140618in}{1.875828in}}%
\pgfpathcurveto{\pgfqpoint{4.134794in}{1.870004in}}{\pgfqpoint{4.131522in}{1.862104in}}{\pgfqpoint{4.131522in}{1.853868in}}%
\pgfpathcurveto{\pgfqpoint{4.131522in}{1.845632in}}{\pgfqpoint{4.134794in}{1.837732in}}{\pgfqpoint{4.140618in}{1.831908in}}%
\pgfpathcurveto{\pgfqpoint{4.146442in}{1.826084in}}{\pgfqpoint{4.154342in}{1.822812in}}{\pgfqpoint{4.162579in}{1.822812in}}%
\pgfpathclose%
\pgfusepath{stroke,fill}%
\end{pgfscope}%
\begin{pgfscope}%
\pgfpathrectangle{\pgfqpoint{2.816705in}{0.516222in}}{\pgfqpoint{1.962733in}{1.783528in}} %
\pgfusepath{clip}%
\pgfsetbuttcap%
\pgfsetroundjoin%
\definecolor{currentfill}{rgb}{0.298039,0.447059,0.690196}%
\pgfsetfillcolor{currentfill}%
\pgfsetlinewidth{0.240900pt}%
\definecolor{currentstroke}{rgb}{1.000000,1.000000,1.000000}%
\pgfsetstrokecolor{currentstroke}%
\pgfsetdash{}{0pt}%
\pgfpathmoveto{\pgfqpoint{3.601798in}{1.733635in}}%
\pgfpathcurveto{\pgfqpoint{3.610034in}{1.733635in}}{\pgfqpoint{3.617934in}{1.736907in}}{\pgfqpoint{3.623758in}{1.742731in}}%
\pgfpathcurveto{\pgfqpoint{3.629582in}{1.748555in}}{\pgfqpoint{3.632854in}{1.756455in}}{\pgfqpoint{3.632854in}{1.764692in}}%
\pgfpathcurveto{\pgfqpoint{3.632854in}{1.772928in}}{\pgfqpoint{3.629582in}{1.780828in}}{\pgfqpoint{3.623758in}{1.786652in}}%
\pgfpathcurveto{\pgfqpoint{3.617934in}{1.792476in}}{\pgfqpoint{3.610034in}{1.795748in}}{\pgfqpoint{3.601798in}{1.795748in}}%
\pgfpathcurveto{\pgfqpoint{3.593562in}{1.795748in}}{\pgfqpoint{3.585662in}{1.792476in}}{\pgfqpoint{3.579838in}{1.786652in}}%
\pgfpathcurveto{\pgfqpoint{3.574014in}{1.780828in}}{\pgfqpoint{3.570741in}{1.772928in}}{\pgfqpoint{3.570741in}{1.764692in}}%
\pgfpathcurveto{\pgfqpoint{3.570741in}{1.756455in}}{\pgfqpoint{3.574014in}{1.748555in}}{\pgfqpoint{3.579838in}{1.742731in}}%
\pgfpathcurveto{\pgfqpoint{3.585662in}{1.736907in}}{\pgfqpoint{3.593562in}{1.733635in}}{\pgfqpoint{3.601798in}{1.733635in}}%
\pgfpathclose%
\pgfusepath{stroke,fill}%
\end{pgfscope}%
\begin{pgfscope}%
\pgfpathrectangle{\pgfqpoint{2.816705in}{0.516222in}}{\pgfqpoint{1.962733in}{1.783528in}} %
\pgfusepath{clip}%
\pgfsetbuttcap%
\pgfsetroundjoin%
\definecolor{currentfill}{rgb}{0.298039,0.447059,0.690196}%
\pgfsetfillcolor{currentfill}%
\pgfsetlinewidth{0.240900pt}%
\definecolor{currentstroke}{rgb}{1.000000,1.000000,1.000000}%
\pgfsetstrokecolor{currentstroke}%
\pgfsetdash{}{0pt}%
\pgfpathmoveto{\pgfqpoint{4.218657in}{1.911988in}}%
\pgfpathcurveto{\pgfqpoint{4.226893in}{1.911988in}}{\pgfqpoint{4.234793in}{1.915260in}}{\pgfqpoint{4.240617in}{1.921084in}}%
\pgfpathcurveto{\pgfqpoint{4.246441in}{1.926908in}}{\pgfqpoint{4.249713in}{1.934808in}}{\pgfqpoint{4.249713in}{1.943044in}}%
\pgfpathcurveto{\pgfqpoint{4.249713in}{1.951281in}}{\pgfqpoint{4.246441in}{1.959181in}}{\pgfqpoint{4.240617in}{1.965005in}}%
\pgfpathcurveto{\pgfqpoint{4.234793in}{1.970829in}}{\pgfqpoint{4.226893in}{1.974101in}}{\pgfqpoint{4.218657in}{1.974101in}}%
\pgfpathcurveto{\pgfqpoint{4.210420in}{1.974101in}}{\pgfqpoint{4.202520in}{1.970829in}}{\pgfqpoint{4.196696in}{1.965005in}}%
\pgfpathcurveto{\pgfqpoint{4.190873in}{1.959181in}}{\pgfqpoint{4.187600in}{1.951281in}}{\pgfqpoint{4.187600in}{1.943044in}}%
\pgfpathcurveto{\pgfqpoint{4.187600in}{1.934808in}}{\pgfqpoint{4.190873in}{1.926908in}}{\pgfqpoint{4.196696in}{1.921084in}}%
\pgfpathcurveto{\pgfqpoint{4.202520in}{1.915260in}}{\pgfqpoint{4.210420in}{1.911988in}}{\pgfqpoint{4.218657in}{1.911988in}}%
\pgfpathclose%
\pgfusepath{stroke,fill}%
\end{pgfscope}%
\begin{pgfscope}%
\pgfpathrectangle{\pgfqpoint{2.816705in}{0.516222in}}{\pgfqpoint{1.962733in}{1.783528in}} %
\pgfusepath{clip}%
\pgfsetbuttcap%
\pgfsetroundjoin%
\definecolor{currentfill}{rgb}{0.298039,0.447059,0.690196}%
\pgfsetfillcolor{currentfill}%
\pgfsetlinewidth{0.240900pt}%
\definecolor{currentstroke}{rgb}{1.000000,1.000000,1.000000}%
\pgfsetstrokecolor{currentstroke}%
\pgfsetdash{}{0pt}%
\pgfpathmoveto{\pgfqpoint{3.601798in}{1.168851in}}%
\pgfpathcurveto{\pgfqpoint{3.610034in}{1.168851in}}{\pgfqpoint{3.617934in}{1.172124in}}{\pgfqpoint{3.623758in}{1.177948in}}%
\pgfpathcurveto{\pgfqpoint{3.629582in}{1.183772in}}{\pgfqpoint{3.632854in}{1.191672in}}{\pgfqpoint{3.632854in}{1.199908in}}%
\pgfpathcurveto{\pgfqpoint{3.632854in}{1.208144in}}{\pgfqpoint{3.629582in}{1.216044in}}{\pgfqpoint{3.623758in}{1.221868in}}%
\pgfpathcurveto{\pgfqpoint{3.617934in}{1.227692in}}{\pgfqpoint{3.610034in}{1.230964in}}{\pgfqpoint{3.601798in}{1.230964in}}%
\pgfpathcurveto{\pgfqpoint{3.593562in}{1.230964in}}{\pgfqpoint{3.585662in}{1.227692in}}{\pgfqpoint{3.579838in}{1.221868in}}%
\pgfpathcurveto{\pgfqpoint{3.574014in}{1.216044in}}{\pgfqpoint{3.570741in}{1.208144in}}{\pgfqpoint{3.570741in}{1.199908in}}%
\pgfpathcurveto{\pgfqpoint{3.570741in}{1.191672in}}{\pgfqpoint{3.574014in}{1.183772in}}{\pgfqpoint{3.579838in}{1.177948in}}%
\pgfpathcurveto{\pgfqpoint{3.585662in}{1.172124in}}{\pgfqpoint{3.593562in}{1.168851in}}{\pgfqpoint{3.601798in}{1.168851in}}%
\pgfpathclose%
\pgfusepath{stroke,fill}%
\end{pgfscope}%
\begin{pgfscope}%
\pgfpathrectangle{\pgfqpoint{2.816705in}{0.516222in}}{\pgfqpoint{1.962733in}{1.783528in}} %
\pgfusepath{clip}%
\pgfsetbuttcap%
\pgfsetroundjoin%
\definecolor{currentfill}{rgb}{0.298039,0.447059,0.690196}%
\pgfsetfillcolor{currentfill}%
\pgfsetlinewidth{0.240900pt}%
\definecolor{currentstroke}{rgb}{1.000000,1.000000,1.000000}%
\pgfsetstrokecolor{currentstroke}%
\pgfsetdash{}{0pt}%
\pgfpathmoveto{\pgfqpoint{4.442969in}{1.674184in}}%
\pgfpathcurveto{\pgfqpoint{4.451205in}{1.674184in}}{\pgfqpoint{4.459105in}{1.677457in}}{\pgfqpoint{4.464929in}{1.683280in}}%
\pgfpathcurveto{\pgfqpoint{4.470753in}{1.689104in}}{\pgfqpoint{4.474026in}{1.697004in}}{\pgfqpoint{4.474026in}{1.705241in}}%
\pgfpathcurveto{\pgfqpoint{4.474026in}{1.713477in}}{\pgfqpoint{4.470753in}{1.721377in}}{\pgfqpoint{4.464929in}{1.727201in}}%
\pgfpathcurveto{\pgfqpoint{4.459105in}{1.733025in}}{\pgfqpoint{4.451205in}{1.736297in}}{\pgfqpoint{4.442969in}{1.736297in}}%
\pgfpathcurveto{\pgfqpoint{4.434733in}{1.736297in}}{\pgfqpoint{4.426833in}{1.733025in}}{\pgfqpoint{4.421009in}{1.727201in}}%
\pgfpathcurveto{\pgfqpoint{4.415185in}{1.721377in}}{\pgfqpoint{4.411913in}{1.713477in}}{\pgfqpoint{4.411913in}{1.705241in}}%
\pgfpathcurveto{\pgfqpoint{4.411913in}{1.697004in}}{\pgfqpoint{4.415185in}{1.689104in}}{\pgfqpoint{4.421009in}{1.683280in}}%
\pgfpathcurveto{\pgfqpoint{4.426833in}{1.677457in}}{\pgfqpoint{4.434733in}{1.674184in}}{\pgfqpoint{4.442969in}{1.674184in}}%
\pgfpathclose%
\pgfusepath{stroke,fill}%
\end{pgfscope}%
\begin{pgfscope}%
\pgfpathrectangle{\pgfqpoint{2.816705in}{0.516222in}}{\pgfqpoint{1.962733in}{1.783528in}} %
\pgfusepath{clip}%
\pgfsetbuttcap%
\pgfsetroundjoin%
\definecolor{currentfill}{rgb}{0.298039,0.447059,0.690196}%
\pgfsetfillcolor{currentfill}%
\pgfsetlinewidth{0.240900pt}%
\definecolor{currentstroke}{rgb}{1.000000,1.000000,1.000000}%
\pgfsetstrokecolor{currentstroke}%
\pgfsetdash{}{0pt}%
\pgfpathmoveto{\pgfqpoint{3.097095in}{1.585008in}}%
\pgfpathcurveto{\pgfqpoint{3.105332in}{1.585008in}}{\pgfqpoint{3.113232in}{1.588280in}}{\pgfqpoint{3.119055in}{1.594104in}}%
\pgfpathcurveto{\pgfqpoint{3.124879in}{1.599928in}}{\pgfqpoint{3.128152in}{1.607828in}}{\pgfqpoint{3.128152in}{1.616064in}}%
\pgfpathcurveto{\pgfqpoint{3.128152in}{1.624301in}}{\pgfqpoint{3.124879in}{1.632201in}}{\pgfqpoint{3.119055in}{1.638025in}}%
\pgfpathcurveto{\pgfqpoint{3.113232in}{1.643849in}}{\pgfqpoint{3.105332in}{1.647121in}}{\pgfqpoint{3.097095in}{1.647121in}}%
\pgfpathcurveto{\pgfqpoint{3.088859in}{1.647121in}}{\pgfqpoint{3.080959in}{1.643849in}}{\pgfqpoint{3.075135in}{1.638025in}}%
\pgfpathcurveto{\pgfqpoint{3.069311in}{1.632201in}}{\pgfqpoint{3.066039in}{1.624301in}}{\pgfqpoint{3.066039in}{1.616064in}}%
\pgfpathcurveto{\pgfqpoint{3.066039in}{1.607828in}}{\pgfqpoint{3.069311in}{1.599928in}}{\pgfqpoint{3.075135in}{1.594104in}}%
\pgfpathcurveto{\pgfqpoint{3.080959in}{1.588280in}}{\pgfqpoint{3.088859in}{1.585008in}}{\pgfqpoint{3.097095in}{1.585008in}}%
\pgfpathclose%
\pgfusepath{stroke,fill}%
\end{pgfscope}%
\begin{pgfscope}%
\pgfpathrectangle{\pgfqpoint{2.816705in}{0.516222in}}{\pgfqpoint{1.962733in}{1.783528in}} %
\pgfusepath{clip}%
\pgfsetbuttcap%
\pgfsetroundjoin%
\definecolor{currentfill}{rgb}{0.298039,0.447059,0.690196}%
\pgfsetfillcolor{currentfill}%
\pgfsetlinewidth{0.240900pt}%
\definecolor{currentstroke}{rgb}{1.000000,1.000000,1.000000}%
\pgfsetstrokecolor{currentstroke}%
\pgfsetdash{}{0pt}%
\pgfpathmoveto{\pgfqpoint{3.657876in}{1.436381in}}%
\pgfpathcurveto{\pgfqpoint{3.666112in}{1.436381in}}{\pgfqpoint{3.674012in}{1.439653in}}{\pgfqpoint{3.679836in}{1.445477in}}%
\pgfpathcurveto{\pgfqpoint{3.685660in}{1.451301in}}{\pgfqpoint{3.688932in}{1.459201in}}{\pgfqpoint{3.688932in}{1.467437in}}%
\pgfpathcurveto{\pgfqpoint{3.688932in}{1.475673in}}{\pgfqpoint{3.685660in}{1.483573in}}{\pgfqpoint{3.679836in}{1.489397in}}%
\pgfpathcurveto{\pgfqpoint{3.674012in}{1.495221in}}{\pgfqpoint{3.666112in}{1.498494in}}{\pgfqpoint{3.657876in}{1.498494in}}%
\pgfpathcurveto{\pgfqpoint{3.649640in}{1.498494in}}{\pgfqpoint{3.641740in}{1.495221in}}{\pgfqpoint{3.635916in}{1.489397in}}%
\pgfpathcurveto{\pgfqpoint{3.630092in}{1.483573in}}{\pgfqpoint{3.626819in}{1.475673in}}{\pgfqpoint{3.626819in}{1.467437in}}%
\pgfpathcurveto{\pgfqpoint{3.626819in}{1.459201in}}{\pgfqpoint{3.630092in}{1.451301in}}{\pgfqpoint{3.635916in}{1.445477in}}%
\pgfpathcurveto{\pgfqpoint{3.641740in}{1.439653in}}{\pgfqpoint{3.649640in}{1.436381in}}{\pgfqpoint{3.657876in}{1.436381in}}%
\pgfpathclose%
\pgfusepath{stroke,fill}%
\end{pgfscope}%
\begin{pgfscope}%
\pgfpathrectangle{\pgfqpoint{2.816705in}{0.516222in}}{\pgfqpoint{1.962733in}{1.783528in}} %
\pgfusepath{clip}%
\pgfsetbuttcap%
\pgfsetroundjoin%
\definecolor{currentfill}{rgb}{0.298039,0.447059,0.690196}%
\pgfsetfillcolor{currentfill}%
\pgfsetlinewidth{0.240900pt}%
\definecolor{currentstroke}{rgb}{1.000000,1.000000,1.000000}%
\pgfsetstrokecolor{currentstroke}%
\pgfsetdash{}{0pt}%
\pgfpathmoveto{\pgfqpoint{3.657876in}{0.633793in}}%
\pgfpathcurveto{\pgfqpoint{3.666112in}{0.633793in}}{\pgfqpoint{3.674012in}{0.637065in}}{\pgfqpoint{3.679836in}{0.642889in}}%
\pgfpathcurveto{\pgfqpoint{3.685660in}{0.648713in}}{\pgfqpoint{3.688932in}{0.656613in}}{\pgfqpoint{3.688932in}{0.664850in}}%
\pgfpathcurveto{\pgfqpoint{3.688932in}{0.673086in}}{\pgfqpoint{3.685660in}{0.680986in}}{\pgfqpoint{3.679836in}{0.686810in}}%
\pgfpathcurveto{\pgfqpoint{3.674012in}{0.692634in}}{\pgfqpoint{3.666112in}{0.695906in}}{\pgfqpoint{3.657876in}{0.695906in}}%
\pgfpathcurveto{\pgfqpoint{3.649640in}{0.695906in}}{\pgfqpoint{3.641740in}{0.692634in}}{\pgfqpoint{3.635916in}{0.686810in}}%
\pgfpathcurveto{\pgfqpoint{3.630092in}{0.680986in}}{\pgfqpoint{3.626819in}{0.673086in}}{\pgfqpoint{3.626819in}{0.664850in}}%
\pgfpathcurveto{\pgfqpoint{3.626819in}{0.656613in}}{\pgfqpoint{3.630092in}{0.648713in}}{\pgfqpoint{3.635916in}{0.642889in}}%
\pgfpathcurveto{\pgfqpoint{3.641740in}{0.637065in}}{\pgfqpoint{3.649640in}{0.633793in}}{\pgfqpoint{3.657876in}{0.633793in}}%
\pgfpathclose%
\pgfusepath{stroke,fill}%
\end{pgfscope}%
\begin{pgfscope}%
\pgfpathrectangle{\pgfqpoint{2.816705in}{0.516222in}}{\pgfqpoint{1.962733in}{1.783528in}} %
\pgfusepath{clip}%
\pgfsetbuttcap%
\pgfsetroundjoin%
\definecolor{currentfill}{rgb}{0.298039,0.447059,0.690196}%
\pgfsetfillcolor{currentfill}%
\pgfsetlinewidth{0.240900pt}%
\definecolor{currentstroke}{rgb}{1.000000,1.000000,1.000000}%
\pgfsetstrokecolor{currentstroke}%
\pgfsetdash{}{0pt}%
\pgfpathmoveto{\pgfqpoint{3.545720in}{1.793086in}}%
\pgfpathcurveto{\pgfqpoint{3.553956in}{1.793086in}}{\pgfqpoint{3.561856in}{1.796358in}}{\pgfqpoint{3.567680in}{1.802182in}}%
\pgfpathcurveto{\pgfqpoint{3.573504in}{1.808006in}}{\pgfqpoint{3.576776in}{1.815906in}}{\pgfqpoint{3.576776in}{1.824143in}}%
\pgfpathcurveto{\pgfqpoint{3.576776in}{1.832379in}}{\pgfqpoint{3.573504in}{1.840279in}}{\pgfqpoint{3.567680in}{1.846103in}}%
\pgfpathcurveto{\pgfqpoint{3.561856in}{1.851927in}}{\pgfqpoint{3.553956in}{1.855199in}}{\pgfqpoint{3.545720in}{1.855199in}}%
\pgfpathcurveto{\pgfqpoint{3.537484in}{1.855199in}}{\pgfqpoint{3.529584in}{1.851927in}}{\pgfqpoint{3.523760in}{1.846103in}}%
\pgfpathcurveto{\pgfqpoint{3.517936in}{1.840279in}}{\pgfqpoint{3.514663in}{1.832379in}}{\pgfqpoint{3.514663in}{1.824143in}}%
\pgfpathcurveto{\pgfqpoint{3.514663in}{1.815906in}}{\pgfqpoint{3.517936in}{1.808006in}}{\pgfqpoint{3.523760in}{1.802182in}}%
\pgfpathcurveto{\pgfqpoint{3.529584in}{1.796358in}}{\pgfqpoint{3.537484in}{1.793086in}}{\pgfqpoint{3.545720in}{1.793086in}}%
\pgfpathclose%
\pgfusepath{stroke,fill}%
\end{pgfscope}%
\begin{pgfscope}%
\pgfpathrectangle{\pgfqpoint{2.816705in}{0.516222in}}{\pgfqpoint{1.962733in}{1.783528in}} %
\pgfusepath{clip}%
\pgfsetbuttcap%
\pgfsetroundjoin%
\definecolor{currentfill}{rgb}{0.298039,0.447059,0.690196}%
\pgfsetfillcolor{currentfill}%
\pgfsetlinewidth{0.240900pt}%
\definecolor{currentstroke}{rgb}{1.000000,1.000000,1.000000}%
\pgfsetstrokecolor{currentstroke}%
\pgfsetdash{}{0pt}%
\pgfpathmoveto{\pgfqpoint{3.601798in}{1.376930in}}%
\pgfpathcurveto{\pgfqpoint{3.610034in}{1.376930in}}{\pgfqpoint{3.617934in}{1.380202in}}{\pgfqpoint{3.623758in}{1.386026in}}%
\pgfpathcurveto{\pgfqpoint{3.629582in}{1.391850in}}{\pgfqpoint{3.632854in}{1.399750in}}{\pgfqpoint{3.632854in}{1.407986in}}%
\pgfpathcurveto{\pgfqpoint{3.632854in}{1.416222in}}{\pgfqpoint{3.629582in}{1.424122in}}{\pgfqpoint{3.623758in}{1.429946in}}%
\pgfpathcurveto{\pgfqpoint{3.617934in}{1.435770in}}{\pgfqpoint{3.610034in}{1.439043in}}{\pgfqpoint{3.601798in}{1.439043in}}%
\pgfpathcurveto{\pgfqpoint{3.593562in}{1.439043in}}{\pgfqpoint{3.585662in}{1.435770in}}{\pgfqpoint{3.579838in}{1.429946in}}%
\pgfpathcurveto{\pgfqpoint{3.574014in}{1.424122in}}{\pgfqpoint{3.570741in}{1.416222in}}{\pgfqpoint{3.570741in}{1.407986in}}%
\pgfpathcurveto{\pgfqpoint{3.570741in}{1.399750in}}{\pgfqpoint{3.574014in}{1.391850in}}{\pgfqpoint{3.579838in}{1.386026in}}%
\pgfpathcurveto{\pgfqpoint{3.585662in}{1.380202in}}{\pgfqpoint{3.593562in}{1.376930in}}{\pgfqpoint{3.601798in}{1.376930in}}%
\pgfpathclose%
\pgfusepath{stroke,fill}%
\end{pgfscope}%
\begin{pgfscope}%
\pgfpathrectangle{\pgfqpoint{2.816705in}{0.516222in}}{\pgfqpoint{1.962733in}{1.783528in}} %
\pgfusepath{clip}%
\pgfsetbuttcap%
\pgfsetroundjoin%
\definecolor{currentfill}{rgb}{0.298039,0.447059,0.690196}%
\pgfsetfillcolor{currentfill}%
\pgfsetlinewidth{0.240900pt}%
\definecolor{currentstroke}{rgb}{1.000000,1.000000,1.000000}%
\pgfsetstrokecolor{currentstroke}%
\pgfsetdash{}{0pt}%
\pgfpathmoveto{\pgfqpoint{2.984939in}{0.693244in}}%
\pgfpathcurveto{\pgfqpoint{2.993175in}{0.693244in}}{\pgfqpoint{3.001075in}{0.696516in}}{\pgfqpoint{3.006899in}{0.702340in}}%
\pgfpathcurveto{\pgfqpoint{3.012723in}{0.708164in}}{\pgfqpoint{3.015996in}{0.716064in}}{\pgfqpoint{3.015996in}{0.724300in}}%
\pgfpathcurveto{\pgfqpoint{3.015996in}{0.732537in}}{\pgfqpoint{3.012723in}{0.740437in}}{\pgfqpoint{3.006899in}{0.746261in}}%
\pgfpathcurveto{\pgfqpoint{3.001075in}{0.752085in}}{\pgfqpoint{2.993175in}{0.755357in}}{\pgfqpoint{2.984939in}{0.755357in}}%
\pgfpathcurveto{\pgfqpoint{2.976703in}{0.755357in}}{\pgfqpoint{2.968803in}{0.752085in}}{\pgfqpoint{2.962979in}{0.746261in}}%
\pgfpathcurveto{\pgfqpoint{2.957155in}{0.740437in}}{\pgfqpoint{2.953883in}{0.732537in}}{\pgfqpoint{2.953883in}{0.724300in}}%
\pgfpathcurveto{\pgfqpoint{2.953883in}{0.716064in}}{\pgfqpoint{2.957155in}{0.708164in}}{\pgfqpoint{2.962979in}{0.702340in}}%
\pgfpathcurveto{\pgfqpoint{2.968803in}{0.696516in}}{\pgfqpoint{2.976703in}{0.693244in}}{\pgfqpoint{2.984939in}{0.693244in}}%
\pgfpathclose%
\pgfusepath{stroke,fill}%
\end{pgfscope}%
\begin{pgfscope}%
\pgfpathrectangle{\pgfqpoint{2.816705in}{0.516222in}}{\pgfqpoint{1.962733in}{1.783528in}} %
\pgfusepath{clip}%
\pgfsetbuttcap%
\pgfsetroundjoin%
\definecolor{currentfill}{rgb}{0.298039,0.447059,0.690196}%
\pgfsetfillcolor{currentfill}%
\pgfsetlinewidth{0.240900pt}%
\definecolor{currentstroke}{rgb}{1.000000,1.000000,1.000000}%
\pgfsetstrokecolor{currentstroke}%
\pgfsetdash{}{0pt}%
\pgfpathmoveto{\pgfqpoint{3.938266in}{2.001164in}}%
\pgfpathcurveto{\pgfqpoint{3.946503in}{2.001164in}}{\pgfqpoint{3.954403in}{2.004437in}}{\pgfqpoint{3.960227in}{2.010261in}}%
\pgfpathcurveto{\pgfqpoint{3.966051in}{2.016085in}}{\pgfqpoint{3.969323in}{2.023985in}}{\pgfqpoint{3.969323in}{2.032221in}}%
\pgfpathcurveto{\pgfqpoint{3.969323in}{2.040457in}}{\pgfqpoint{3.966051in}{2.048357in}}{\pgfqpoint{3.960227in}{2.054181in}}%
\pgfpathcurveto{\pgfqpoint{3.954403in}{2.060005in}}{\pgfqpoint{3.946503in}{2.063277in}}{\pgfqpoint{3.938266in}{2.063277in}}%
\pgfpathcurveto{\pgfqpoint{3.930030in}{2.063277in}}{\pgfqpoint{3.922130in}{2.060005in}}{\pgfqpoint{3.916306in}{2.054181in}}%
\pgfpathcurveto{\pgfqpoint{3.910482in}{2.048357in}}{\pgfqpoint{3.907210in}{2.040457in}}{\pgfqpoint{3.907210in}{2.032221in}}%
\pgfpathcurveto{\pgfqpoint{3.907210in}{2.023985in}}{\pgfqpoint{3.910482in}{2.016085in}}{\pgfqpoint{3.916306in}{2.010261in}}%
\pgfpathcurveto{\pgfqpoint{3.922130in}{2.004437in}}{\pgfqpoint{3.930030in}{2.001164in}}{\pgfqpoint{3.938266in}{2.001164in}}%
\pgfpathclose%
\pgfusepath{stroke,fill}%
\end{pgfscope}%
\begin{pgfscope}%
\pgfpathrectangle{\pgfqpoint{2.816705in}{0.516222in}}{\pgfqpoint{1.962733in}{1.783528in}} %
\pgfusepath{clip}%
\pgfsetbuttcap%
\pgfsetroundjoin%
\definecolor{currentfill}{rgb}{0.298039,0.447059,0.690196}%
\pgfsetfillcolor{currentfill}%
\pgfsetlinewidth{0.240900pt}%
\definecolor{currentstroke}{rgb}{1.000000,1.000000,1.000000}%
\pgfsetstrokecolor{currentstroke}%
\pgfsetdash{}{0pt}%
\pgfpathmoveto{\pgfqpoint{3.882188in}{1.466106in}}%
\pgfpathcurveto{\pgfqpoint{3.890425in}{1.466106in}}{\pgfqpoint{3.898325in}{1.469378in}}{\pgfqpoint{3.904149in}{1.475202in}}%
\pgfpathcurveto{\pgfqpoint{3.909972in}{1.481026in}}{\pgfqpoint{3.913245in}{1.488926in}}{\pgfqpoint{3.913245in}{1.497163in}}%
\pgfpathcurveto{\pgfqpoint{3.913245in}{1.505399in}}{\pgfqpoint{3.909972in}{1.513299in}}{\pgfqpoint{3.904149in}{1.519123in}}%
\pgfpathcurveto{\pgfqpoint{3.898325in}{1.524947in}}{\pgfqpoint{3.890425in}{1.528219in}}{\pgfqpoint{3.882188in}{1.528219in}}%
\pgfpathcurveto{\pgfqpoint{3.873952in}{1.528219in}}{\pgfqpoint{3.866052in}{1.524947in}}{\pgfqpoint{3.860228in}{1.519123in}}%
\pgfpathcurveto{\pgfqpoint{3.854404in}{1.513299in}}{\pgfqpoint{3.851132in}{1.505399in}}{\pgfqpoint{3.851132in}{1.497163in}}%
\pgfpathcurveto{\pgfqpoint{3.851132in}{1.488926in}}{\pgfqpoint{3.854404in}{1.481026in}}{\pgfqpoint{3.860228in}{1.475202in}}%
\pgfpathcurveto{\pgfqpoint{3.866052in}{1.469378in}}{\pgfqpoint{3.873952in}{1.466106in}}{\pgfqpoint{3.882188in}{1.466106in}}%
\pgfpathclose%
\pgfusepath{stroke,fill}%
\end{pgfscope}%
\begin{pgfscope}%
\pgfpathrectangle{\pgfqpoint{2.816705in}{0.516222in}}{\pgfqpoint{1.962733in}{1.783528in}} %
\pgfusepath{clip}%
\pgfsetbuttcap%
\pgfsetroundjoin%
\definecolor{currentfill}{rgb}{0.298039,0.447059,0.690196}%
\pgfsetfillcolor{currentfill}%
\pgfsetlinewidth{0.240900pt}%
\definecolor{currentstroke}{rgb}{1.000000,1.000000,1.000000}%
\pgfsetstrokecolor{currentstroke}%
\pgfsetdash{}{0pt}%
\pgfpathmoveto{\pgfqpoint{3.433564in}{1.258028in}}%
\pgfpathcurveto{\pgfqpoint{3.441800in}{1.258028in}}{\pgfqpoint{3.449700in}{1.261300in}}{\pgfqpoint{3.455524in}{1.267124in}}%
\pgfpathcurveto{\pgfqpoint{3.461348in}{1.272948in}}{\pgfqpoint{3.464620in}{1.280848in}}{\pgfqpoint{3.464620in}{1.289084in}}%
\pgfpathcurveto{\pgfqpoint{3.464620in}{1.297321in}}{\pgfqpoint{3.461348in}{1.305221in}}{\pgfqpoint{3.455524in}{1.311045in}}%
\pgfpathcurveto{\pgfqpoint{3.449700in}{1.316868in}}{\pgfqpoint{3.441800in}{1.320141in}}{\pgfqpoint{3.433564in}{1.320141in}}%
\pgfpathcurveto{\pgfqpoint{3.425327in}{1.320141in}}{\pgfqpoint{3.417427in}{1.316868in}}{\pgfqpoint{3.411603in}{1.311045in}}%
\pgfpathcurveto{\pgfqpoint{3.405779in}{1.305221in}}{\pgfqpoint{3.402507in}{1.297321in}}{\pgfqpoint{3.402507in}{1.289084in}}%
\pgfpathcurveto{\pgfqpoint{3.402507in}{1.280848in}}{\pgfqpoint{3.405779in}{1.272948in}}{\pgfqpoint{3.411603in}{1.267124in}}%
\pgfpathcurveto{\pgfqpoint{3.417427in}{1.261300in}}{\pgfqpoint{3.425327in}{1.258028in}}{\pgfqpoint{3.433564in}{1.258028in}}%
\pgfpathclose%
\pgfusepath{stroke,fill}%
\end{pgfscope}%
\begin{pgfscope}%
\pgfpathrectangle{\pgfqpoint{2.816705in}{0.516222in}}{\pgfqpoint{1.962733in}{1.783528in}} %
\pgfusepath{clip}%
\pgfsetbuttcap%
\pgfsetroundjoin%
\definecolor{currentfill}{rgb}{0.298039,0.447059,0.690196}%
\pgfsetfillcolor{currentfill}%
\pgfsetlinewidth{0.240900pt}%
\definecolor{currentstroke}{rgb}{1.000000,1.000000,1.000000}%
\pgfsetstrokecolor{currentstroke}%
\pgfsetdash{}{0pt}%
\pgfpathmoveto{\pgfqpoint{3.601798in}{0.633793in}}%
\pgfpathcurveto{\pgfqpoint{3.610034in}{0.633793in}}{\pgfqpoint{3.617934in}{0.637065in}}{\pgfqpoint{3.623758in}{0.642889in}}%
\pgfpathcurveto{\pgfqpoint{3.629582in}{0.648713in}}{\pgfqpoint{3.632854in}{0.656613in}}{\pgfqpoint{3.632854in}{0.664850in}}%
\pgfpathcurveto{\pgfqpoint{3.632854in}{0.673086in}}{\pgfqpoint{3.629582in}{0.680986in}}{\pgfqpoint{3.623758in}{0.686810in}}%
\pgfpathcurveto{\pgfqpoint{3.617934in}{0.692634in}}{\pgfqpoint{3.610034in}{0.695906in}}{\pgfqpoint{3.601798in}{0.695906in}}%
\pgfpathcurveto{\pgfqpoint{3.593562in}{0.695906in}}{\pgfqpoint{3.585662in}{0.692634in}}{\pgfqpoint{3.579838in}{0.686810in}}%
\pgfpathcurveto{\pgfqpoint{3.574014in}{0.680986in}}{\pgfqpoint{3.570741in}{0.673086in}}{\pgfqpoint{3.570741in}{0.664850in}}%
\pgfpathcurveto{\pgfqpoint{3.570741in}{0.656613in}}{\pgfqpoint{3.574014in}{0.648713in}}{\pgfqpoint{3.579838in}{0.642889in}}%
\pgfpathcurveto{\pgfqpoint{3.585662in}{0.637065in}}{\pgfqpoint{3.593562in}{0.633793in}}{\pgfqpoint{3.601798in}{0.633793in}}%
\pgfpathclose%
\pgfusepath{stroke,fill}%
\end{pgfscope}%
\begin{pgfscope}%
\pgfpathrectangle{\pgfqpoint{2.816705in}{0.516222in}}{\pgfqpoint{1.962733in}{1.783528in}} %
\pgfusepath{clip}%
\pgfsetbuttcap%
\pgfsetroundjoin%
\definecolor{currentfill}{rgb}{0.298039,0.447059,0.690196}%
\pgfsetfillcolor{currentfill}%
\pgfsetlinewidth{0.240900pt}%
\definecolor{currentstroke}{rgb}{1.000000,1.000000,1.000000}%
\pgfsetstrokecolor{currentstroke}%
\pgfsetdash{}{0pt}%
\pgfpathmoveto{\pgfqpoint{3.209251in}{1.049950in}}%
\pgfpathcurveto{\pgfqpoint{3.217488in}{1.049950in}}{\pgfqpoint{3.225388in}{1.053222in}}{\pgfqpoint{3.231212in}{1.059046in}}%
\pgfpathcurveto{\pgfqpoint{3.237036in}{1.064870in}}{\pgfqpoint{3.240308in}{1.072770in}}{\pgfqpoint{3.240308in}{1.081006in}}%
\pgfpathcurveto{\pgfqpoint{3.240308in}{1.089242in}}{\pgfqpoint{3.237036in}{1.097142in}}{\pgfqpoint{3.231212in}{1.102966in}}%
\pgfpathcurveto{\pgfqpoint{3.225388in}{1.108790in}}{\pgfqpoint{3.217488in}{1.112063in}}{\pgfqpoint{3.209251in}{1.112063in}}%
\pgfpathcurveto{\pgfqpoint{3.201015in}{1.112063in}}{\pgfqpoint{3.193115in}{1.108790in}}{\pgfqpoint{3.187291in}{1.102966in}}%
\pgfpathcurveto{\pgfqpoint{3.181467in}{1.097142in}}{\pgfqpoint{3.178195in}{1.089242in}}{\pgfqpoint{3.178195in}{1.081006in}}%
\pgfpathcurveto{\pgfqpoint{3.178195in}{1.072770in}}{\pgfqpoint{3.181467in}{1.064870in}}{\pgfqpoint{3.187291in}{1.059046in}}%
\pgfpathcurveto{\pgfqpoint{3.193115in}{1.053222in}}{\pgfqpoint{3.201015in}{1.049950in}}{\pgfqpoint{3.209251in}{1.049950in}}%
\pgfpathclose%
\pgfusepath{stroke,fill}%
\end{pgfscope}%
\begin{pgfscope}%
\pgfpathrectangle{\pgfqpoint{2.816705in}{0.516222in}}{\pgfqpoint{1.962733in}{1.783528in}} %
\pgfusepath{clip}%
\pgfsetbuttcap%
\pgfsetroundjoin%
\definecolor{currentfill}{rgb}{0.298039,0.447059,0.690196}%
\pgfsetfillcolor{currentfill}%
\pgfsetlinewidth{0.240900pt}%
\definecolor{currentstroke}{rgb}{1.000000,1.000000,1.000000}%
\pgfsetstrokecolor{currentstroke}%
\pgfsetdash{}{0pt}%
\pgfpathmoveto{\pgfqpoint{3.545720in}{1.139126in}}%
\pgfpathcurveto{\pgfqpoint{3.553956in}{1.139126in}}{\pgfqpoint{3.561856in}{1.142398in}}{\pgfqpoint{3.567680in}{1.148222in}}%
\pgfpathcurveto{\pgfqpoint{3.573504in}{1.154046in}}{\pgfqpoint{3.576776in}{1.161946in}}{\pgfqpoint{3.576776in}{1.170182in}}%
\pgfpathcurveto{\pgfqpoint{3.576776in}{1.178419in}}{\pgfqpoint{3.573504in}{1.186319in}}{\pgfqpoint{3.567680in}{1.192143in}}%
\pgfpathcurveto{\pgfqpoint{3.561856in}{1.197967in}}{\pgfqpoint{3.553956in}{1.201239in}}{\pgfqpoint{3.545720in}{1.201239in}}%
\pgfpathcurveto{\pgfqpoint{3.537484in}{1.201239in}}{\pgfqpoint{3.529584in}{1.197967in}}{\pgfqpoint{3.523760in}{1.192143in}}%
\pgfpathcurveto{\pgfqpoint{3.517936in}{1.186319in}}{\pgfqpoint{3.514663in}{1.178419in}}{\pgfqpoint{3.514663in}{1.170182in}}%
\pgfpathcurveto{\pgfqpoint{3.514663in}{1.161946in}}{\pgfqpoint{3.517936in}{1.154046in}}{\pgfqpoint{3.523760in}{1.148222in}}%
\pgfpathcurveto{\pgfqpoint{3.529584in}{1.142398in}}{\pgfqpoint{3.537484in}{1.139126in}}{\pgfqpoint{3.545720in}{1.139126in}}%
\pgfpathclose%
\pgfusepath{stroke,fill}%
\end{pgfscope}%
\begin{pgfscope}%
\pgfpathrectangle{\pgfqpoint{2.816705in}{0.516222in}}{\pgfqpoint{1.962733in}{1.783528in}} %
\pgfusepath{clip}%
\pgfsetbuttcap%
\pgfsetroundjoin%
\definecolor{currentfill}{rgb}{0.298039,0.447059,0.690196}%
\pgfsetfillcolor{currentfill}%
\pgfsetlinewidth{0.240900pt}%
\definecolor{currentstroke}{rgb}{1.000000,1.000000,1.000000}%
\pgfsetstrokecolor{currentstroke}%
\pgfsetdash{}{0pt}%
\pgfpathmoveto{\pgfqpoint{3.882188in}{1.495831in}}%
\pgfpathcurveto{\pgfqpoint{3.890425in}{1.495831in}}{\pgfqpoint{3.898325in}{1.499104in}}{\pgfqpoint{3.904149in}{1.504928in}}%
\pgfpathcurveto{\pgfqpoint{3.909972in}{1.510752in}}{\pgfqpoint{3.913245in}{1.518652in}}{\pgfqpoint{3.913245in}{1.526888in}}%
\pgfpathcurveto{\pgfqpoint{3.913245in}{1.535124in}}{\pgfqpoint{3.909972in}{1.543024in}}{\pgfqpoint{3.904149in}{1.548848in}}%
\pgfpathcurveto{\pgfqpoint{3.898325in}{1.554672in}}{\pgfqpoint{3.890425in}{1.557944in}}{\pgfqpoint{3.882188in}{1.557944in}}%
\pgfpathcurveto{\pgfqpoint{3.873952in}{1.557944in}}{\pgfqpoint{3.866052in}{1.554672in}}{\pgfqpoint{3.860228in}{1.548848in}}%
\pgfpathcurveto{\pgfqpoint{3.854404in}{1.543024in}}{\pgfqpoint{3.851132in}{1.535124in}}{\pgfqpoint{3.851132in}{1.526888in}}%
\pgfpathcurveto{\pgfqpoint{3.851132in}{1.518652in}}{\pgfqpoint{3.854404in}{1.510752in}}{\pgfqpoint{3.860228in}{1.504928in}}%
\pgfpathcurveto{\pgfqpoint{3.866052in}{1.499104in}}{\pgfqpoint{3.873952in}{1.495831in}}{\pgfqpoint{3.882188in}{1.495831in}}%
\pgfpathclose%
\pgfusepath{stroke,fill}%
\end{pgfscope}%
\begin{pgfscope}%
\pgfpathrectangle{\pgfqpoint{2.816705in}{0.516222in}}{\pgfqpoint{1.962733in}{1.783528in}} %
\pgfusepath{clip}%
\pgfsetbuttcap%
\pgfsetroundjoin%
\definecolor{currentfill}{rgb}{0.298039,0.447059,0.690196}%
\pgfsetfillcolor{currentfill}%
\pgfsetlinewidth{0.240900pt}%
\definecolor{currentstroke}{rgb}{1.000000,1.000000,1.000000}%
\pgfsetstrokecolor{currentstroke}%
\pgfsetdash{}{0pt}%
\pgfpathmoveto{\pgfqpoint{4.050423in}{0.901322in}}%
\pgfpathcurveto{\pgfqpoint{4.058659in}{0.901322in}}{\pgfqpoint{4.066559in}{0.904595in}}{\pgfqpoint{4.072383in}{0.910418in}}%
\pgfpathcurveto{\pgfqpoint{4.078207in}{0.916242in}}{\pgfqpoint{4.081479in}{0.924142in}}{\pgfqpoint{4.081479in}{0.932379in}}%
\pgfpathcurveto{\pgfqpoint{4.081479in}{0.940615in}}{\pgfqpoint{4.078207in}{0.948515in}}{\pgfqpoint{4.072383in}{0.954339in}}%
\pgfpathcurveto{\pgfqpoint{4.066559in}{0.960163in}}{\pgfqpoint{4.058659in}{0.963435in}}{\pgfqpoint{4.050423in}{0.963435in}}%
\pgfpathcurveto{\pgfqpoint{4.042186in}{0.963435in}}{\pgfqpoint{4.034286in}{0.960163in}}{\pgfqpoint{4.028462in}{0.954339in}}%
\pgfpathcurveto{\pgfqpoint{4.022638in}{0.948515in}}{\pgfqpoint{4.019366in}{0.940615in}}{\pgfqpoint{4.019366in}{0.932379in}}%
\pgfpathcurveto{\pgfqpoint{4.019366in}{0.924142in}}{\pgfqpoint{4.022638in}{0.916242in}}{\pgfqpoint{4.028462in}{0.910418in}}%
\pgfpathcurveto{\pgfqpoint{4.034286in}{0.904595in}}{\pgfqpoint{4.042186in}{0.901322in}}{\pgfqpoint{4.050423in}{0.901322in}}%
\pgfpathclose%
\pgfusepath{stroke,fill}%
\end{pgfscope}%
\begin{pgfscope}%
\pgfpathrectangle{\pgfqpoint{2.816705in}{0.516222in}}{\pgfqpoint{1.962733in}{1.783528in}} %
\pgfusepath{clip}%
\pgfsetbuttcap%
\pgfsetroundjoin%
\definecolor{currentfill}{rgb}{0.298039,0.447059,0.690196}%
\pgfsetfillcolor{currentfill}%
\pgfsetlinewidth{0.240900pt}%
\definecolor{currentstroke}{rgb}{1.000000,1.000000,1.000000}%
\pgfsetstrokecolor{currentstroke}%
\pgfsetdash{}{0pt}%
\pgfpathmoveto{\pgfqpoint{3.601798in}{1.644459in}}%
\pgfpathcurveto{\pgfqpoint{3.610034in}{1.644459in}}{\pgfqpoint{3.617934in}{1.647731in}}{\pgfqpoint{3.623758in}{1.653555in}}%
\pgfpathcurveto{\pgfqpoint{3.629582in}{1.659379in}}{\pgfqpoint{3.632854in}{1.667279in}}{\pgfqpoint{3.632854in}{1.675515in}}%
\pgfpathcurveto{\pgfqpoint{3.632854in}{1.683752in}}{\pgfqpoint{3.629582in}{1.691652in}}{\pgfqpoint{3.623758in}{1.697476in}}%
\pgfpathcurveto{\pgfqpoint{3.617934in}{1.703299in}}{\pgfqpoint{3.610034in}{1.706572in}}{\pgfqpoint{3.601798in}{1.706572in}}%
\pgfpathcurveto{\pgfqpoint{3.593562in}{1.706572in}}{\pgfqpoint{3.585662in}{1.703299in}}{\pgfqpoint{3.579838in}{1.697476in}}%
\pgfpathcurveto{\pgfqpoint{3.574014in}{1.691652in}}{\pgfqpoint{3.570741in}{1.683752in}}{\pgfqpoint{3.570741in}{1.675515in}}%
\pgfpathcurveto{\pgfqpoint{3.570741in}{1.667279in}}{\pgfqpoint{3.574014in}{1.659379in}}{\pgfqpoint{3.579838in}{1.653555in}}%
\pgfpathcurveto{\pgfqpoint{3.585662in}{1.647731in}}{\pgfqpoint{3.593562in}{1.644459in}}{\pgfqpoint{3.601798in}{1.644459in}}%
\pgfpathclose%
\pgfusepath{stroke,fill}%
\end{pgfscope}%
\begin{pgfscope}%
\pgfpathrectangle{\pgfqpoint{2.816705in}{0.516222in}}{\pgfqpoint{1.962733in}{1.783528in}} %
\pgfusepath{clip}%
\pgfsetbuttcap%
\pgfsetroundjoin%
\definecolor{currentfill}{rgb}{0.298039,0.447059,0.690196}%
\pgfsetfillcolor{currentfill}%
\pgfsetlinewidth{0.240900pt}%
\definecolor{currentstroke}{rgb}{1.000000,1.000000,1.000000}%
\pgfsetstrokecolor{currentstroke}%
\pgfsetdash{}{0pt}%
\pgfpathmoveto{\pgfqpoint{3.433564in}{0.871597in}}%
\pgfpathcurveto{\pgfqpoint{3.441800in}{0.871597in}}{\pgfqpoint{3.449700in}{0.874869in}}{\pgfqpoint{3.455524in}{0.880693in}}%
\pgfpathcurveto{\pgfqpoint{3.461348in}{0.886517in}}{\pgfqpoint{3.464620in}{0.894417in}}{\pgfqpoint{3.464620in}{0.902653in}}%
\pgfpathcurveto{\pgfqpoint{3.464620in}{0.910890in}}{\pgfqpoint{3.461348in}{0.918790in}}{\pgfqpoint{3.455524in}{0.924614in}}%
\pgfpathcurveto{\pgfqpoint{3.449700in}{0.930437in}}{\pgfqpoint{3.441800in}{0.933710in}}{\pgfqpoint{3.433564in}{0.933710in}}%
\pgfpathcurveto{\pgfqpoint{3.425327in}{0.933710in}}{\pgfqpoint{3.417427in}{0.930437in}}{\pgfqpoint{3.411603in}{0.924614in}}%
\pgfpathcurveto{\pgfqpoint{3.405779in}{0.918790in}}{\pgfqpoint{3.402507in}{0.910890in}}{\pgfqpoint{3.402507in}{0.902653in}}%
\pgfpathcurveto{\pgfqpoint{3.402507in}{0.894417in}}{\pgfqpoint{3.405779in}{0.886517in}}{\pgfqpoint{3.411603in}{0.880693in}}%
\pgfpathcurveto{\pgfqpoint{3.417427in}{0.874869in}}{\pgfqpoint{3.425327in}{0.871597in}}{\pgfqpoint{3.433564in}{0.871597in}}%
\pgfpathclose%
\pgfusepath{stroke,fill}%
\end{pgfscope}%
\begin{pgfscope}%
\pgfpathrectangle{\pgfqpoint{2.816705in}{0.516222in}}{\pgfqpoint{1.962733in}{1.783528in}} %
\pgfusepath{clip}%
\pgfsetbuttcap%
\pgfsetroundjoin%
\definecolor{currentfill}{rgb}{0.298039,0.447059,0.690196}%
\pgfsetfillcolor{currentfill}%
\pgfsetlinewidth{0.240900pt}%
\definecolor{currentstroke}{rgb}{1.000000,1.000000,1.000000}%
\pgfsetstrokecolor{currentstroke}%
\pgfsetdash{}{0pt}%
\pgfpathmoveto{\pgfqpoint{3.433564in}{1.198577in}}%
\pgfpathcurveto{\pgfqpoint{3.441800in}{1.198577in}}{\pgfqpoint{3.449700in}{1.201849in}}{\pgfqpoint{3.455524in}{1.207673in}}%
\pgfpathcurveto{\pgfqpoint{3.461348in}{1.213497in}}{\pgfqpoint{3.464620in}{1.221397in}}{\pgfqpoint{3.464620in}{1.229633in}}%
\pgfpathcurveto{\pgfqpoint{3.464620in}{1.237870in}}{\pgfqpoint{3.461348in}{1.245770in}}{\pgfqpoint{3.455524in}{1.251594in}}%
\pgfpathcurveto{\pgfqpoint{3.449700in}{1.257418in}}{\pgfqpoint{3.441800in}{1.260690in}}{\pgfqpoint{3.433564in}{1.260690in}}%
\pgfpathcurveto{\pgfqpoint{3.425327in}{1.260690in}}{\pgfqpoint{3.417427in}{1.257418in}}{\pgfqpoint{3.411603in}{1.251594in}}%
\pgfpathcurveto{\pgfqpoint{3.405779in}{1.245770in}}{\pgfqpoint{3.402507in}{1.237870in}}{\pgfqpoint{3.402507in}{1.229633in}}%
\pgfpathcurveto{\pgfqpoint{3.402507in}{1.221397in}}{\pgfqpoint{3.405779in}{1.213497in}}{\pgfqpoint{3.411603in}{1.207673in}}%
\pgfpathcurveto{\pgfqpoint{3.417427in}{1.201849in}}{\pgfqpoint{3.425327in}{1.198577in}}{\pgfqpoint{3.433564in}{1.198577in}}%
\pgfpathclose%
\pgfusepath{stroke,fill}%
\end{pgfscope}%
\begin{pgfscope}%
\pgfpathrectangle{\pgfqpoint{2.816705in}{0.516222in}}{\pgfqpoint{1.962733in}{1.783528in}} %
\pgfusepath{clip}%
\pgfsetbuttcap%
\pgfsetroundjoin%
\definecolor{currentfill}{rgb}{0.298039,0.447059,0.690196}%
\pgfsetfillcolor{currentfill}%
\pgfsetlinewidth{0.240900pt}%
\definecolor{currentstroke}{rgb}{1.000000,1.000000,1.000000}%
\pgfsetstrokecolor{currentstroke}%
\pgfsetdash{}{0pt}%
\pgfpathmoveto{\pgfqpoint{4.050423in}{1.406655in}}%
\pgfpathcurveto{\pgfqpoint{4.058659in}{1.406655in}}{\pgfqpoint{4.066559in}{1.409927in}}{\pgfqpoint{4.072383in}{1.415751in}}%
\pgfpathcurveto{\pgfqpoint{4.078207in}{1.421575in}}{\pgfqpoint{4.081479in}{1.429475in}}{\pgfqpoint{4.081479in}{1.437712in}}%
\pgfpathcurveto{\pgfqpoint{4.081479in}{1.445948in}}{\pgfqpoint{4.078207in}{1.453848in}}{\pgfqpoint{4.072383in}{1.459672in}}%
\pgfpathcurveto{\pgfqpoint{4.066559in}{1.465496in}}{\pgfqpoint{4.058659in}{1.468768in}}{\pgfqpoint{4.050423in}{1.468768in}}%
\pgfpathcurveto{\pgfqpoint{4.042186in}{1.468768in}}{\pgfqpoint{4.034286in}{1.465496in}}{\pgfqpoint{4.028462in}{1.459672in}}%
\pgfpathcurveto{\pgfqpoint{4.022638in}{1.453848in}}{\pgfqpoint{4.019366in}{1.445948in}}{\pgfqpoint{4.019366in}{1.437712in}}%
\pgfpathcurveto{\pgfqpoint{4.019366in}{1.429475in}}{\pgfqpoint{4.022638in}{1.421575in}}{\pgfqpoint{4.028462in}{1.415751in}}%
\pgfpathcurveto{\pgfqpoint{4.034286in}{1.409927in}}{\pgfqpoint{4.042186in}{1.406655in}}{\pgfqpoint{4.050423in}{1.406655in}}%
\pgfpathclose%
\pgfusepath{stroke,fill}%
\end{pgfscope}%
\begin{pgfscope}%
\pgfpathrectangle{\pgfqpoint{2.816705in}{0.516222in}}{\pgfqpoint{1.962733in}{1.783528in}} %
\pgfusepath{clip}%
\pgfsetbuttcap%
\pgfsetroundjoin%
\definecolor{currentfill}{rgb}{0.298039,0.447059,0.690196}%
\pgfsetfillcolor{currentfill}%
\pgfsetlinewidth{0.240900pt}%
\definecolor{currentstroke}{rgb}{1.000000,1.000000,1.000000}%
\pgfsetstrokecolor{currentstroke}%
\pgfsetdash{}{0pt}%
\pgfpathmoveto{\pgfqpoint{3.377486in}{1.139126in}}%
\pgfpathcurveto{\pgfqpoint{3.385722in}{1.139126in}}{\pgfqpoint{3.393622in}{1.142398in}}{\pgfqpoint{3.399446in}{1.148222in}}%
\pgfpathcurveto{\pgfqpoint{3.405270in}{1.154046in}}{\pgfqpoint{3.408542in}{1.161946in}}{\pgfqpoint{3.408542in}{1.170182in}}%
\pgfpathcurveto{\pgfqpoint{3.408542in}{1.178419in}}{\pgfqpoint{3.405270in}{1.186319in}}{\pgfqpoint{3.399446in}{1.192143in}}%
\pgfpathcurveto{\pgfqpoint{3.393622in}{1.197967in}}{\pgfqpoint{3.385722in}{1.201239in}}{\pgfqpoint{3.377486in}{1.201239in}}%
\pgfpathcurveto{\pgfqpoint{3.369249in}{1.201239in}}{\pgfqpoint{3.361349in}{1.197967in}}{\pgfqpoint{3.355525in}{1.192143in}}%
\pgfpathcurveto{\pgfqpoint{3.349701in}{1.186319in}}{\pgfqpoint{3.346429in}{1.178419in}}{\pgfqpoint{3.346429in}{1.170182in}}%
\pgfpathcurveto{\pgfqpoint{3.346429in}{1.161946in}}{\pgfqpoint{3.349701in}{1.154046in}}{\pgfqpoint{3.355525in}{1.148222in}}%
\pgfpathcurveto{\pgfqpoint{3.361349in}{1.142398in}}{\pgfqpoint{3.369249in}{1.139126in}}{\pgfqpoint{3.377486in}{1.139126in}}%
\pgfpathclose%
\pgfusepath{stroke,fill}%
\end{pgfscope}%
\begin{pgfscope}%
\pgfpathrectangle{\pgfqpoint{2.816705in}{0.516222in}}{\pgfqpoint{1.962733in}{1.783528in}} %
\pgfusepath{clip}%
\pgfsetbuttcap%
\pgfsetroundjoin%
\definecolor{currentfill}{rgb}{0.298039,0.447059,0.690196}%
\pgfsetfillcolor{currentfill}%
\pgfsetlinewidth{0.240900pt}%
\definecolor{currentstroke}{rgb}{1.000000,1.000000,1.000000}%
\pgfsetstrokecolor{currentstroke}%
\pgfsetdash{}{0pt}%
\pgfpathmoveto{\pgfqpoint{3.489642in}{1.079675in}}%
\pgfpathcurveto{\pgfqpoint{3.497878in}{1.079675in}}{\pgfqpoint{3.505778in}{1.082947in}}{\pgfqpoint{3.511602in}{1.088771in}}%
\pgfpathcurveto{\pgfqpoint{3.517426in}{1.094595in}}{\pgfqpoint{3.520698in}{1.102495in}}{\pgfqpoint{3.520698in}{1.110731in}}%
\pgfpathcurveto{\pgfqpoint{3.520698in}{1.118968in}}{\pgfqpoint{3.517426in}{1.126868in}}{\pgfqpoint{3.511602in}{1.132692in}}%
\pgfpathcurveto{\pgfqpoint{3.505778in}{1.138516in}}{\pgfqpoint{3.497878in}{1.141788in}}{\pgfqpoint{3.489642in}{1.141788in}}%
\pgfpathcurveto{\pgfqpoint{3.481405in}{1.141788in}}{\pgfqpoint{3.473505in}{1.138516in}}{\pgfqpoint{3.467682in}{1.132692in}}%
\pgfpathcurveto{\pgfqpoint{3.461858in}{1.126868in}}{\pgfqpoint{3.458585in}{1.118968in}}{\pgfqpoint{3.458585in}{1.110731in}}%
\pgfpathcurveto{\pgfqpoint{3.458585in}{1.102495in}}{\pgfqpoint{3.461858in}{1.094595in}}{\pgfqpoint{3.467682in}{1.088771in}}%
\pgfpathcurveto{\pgfqpoint{3.473505in}{1.082947in}}{\pgfqpoint{3.481405in}{1.079675in}}{\pgfqpoint{3.489642in}{1.079675in}}%
\pgfpathclose%
\pgfusepath{stroke,fill}%
\end{pgfscope}%
\begin{pgfscope}%
\pgfpathrectangle{\pgfqpoint{2.816705in}{0.516222in}}{\pgfqpoint{1.962733in}{1.783528in}} %
\pgfusepath{clip}%
\pgfsetbuttcap%
\pgfsetroundjoin%
\definecolor{currentfill}{rgb}{0.298039,0.447059,0.690196}%
\pgfsetfillcolor{currentfill}%
\pgfsetlinewidth{0.240900pt}%
\definecolor{currentstroke}{rgb}{1.000000,1.000000,1.000000}%
\pgfsetstrokecolor{currentstroke}%
\pgfsetdash{}{0pt}%
\pgfpathmoveto{\pgfqpoint{3.265329in}{1.168851in}}%
\pgfpathcurveto{\pgfqpoint{3.273566in}{1.168851in}}{\pgfqpoint{3.281466in}{1.172124in}}{\pgfqpoint{3.287290in}{1.177948in}}%
\pgfpathcurveto{\pgfqpoint{3.293114in}{1.183772in}}{\pgfqpoint{3.296386in}{1.191672in}}{\pgfqpoint{3.296386in}{1.199908in}}%
\pgfpathcurveto{\pgfqpoint{3.296386in}{1.208144in}}{\pgfqpoint{3.293114in}{1.216044in}}{\pgfqpoint{3.287290in}{1.221868in}}%
\pgfpathcurveto{\pgfqpoint{3.281466in}{1.227692in}}{\pgfqpoint{3.273566in}{1.230964in}}{\pgfqpoint{3.265329in}{1.230964in}}%
\pgfpathcurveto{\pgfqpoint{3.257093in}{1.230964in}}{\pgfqpoint{3.249193in}{1.227692in}}{\pgfqpoint{3.243369in}{1.221868in}}%
\pgfpathcurveto{\pgfqpoint{3.237545in}{1.216044in}}{\pgfqpoint{3.234273in}{1.208144in}}{\pgfqpoint{3.234273in}{1.199908in}}%
\pgfpathcurveto{\pgfqpoint{3.234273in}{1.191672in}}{\pgfqpoint{3.237545in}{1.183772in}}{\pgfqpoint{3.243369in}{1.177948in}}%
\pgfpathcurveto{\pgfqpoint{3.249193in}{1.172124in}}{\pgfqpoint{3.257093in}{1.168851in}}{\pgfqpoint{3.265329in}{1.168851in}}%
\pgfpathclose%
\pgfusepath{stroke,fill}%
\end{pgfscope}%
\begin{pgfscope}%
\pgfpathrectangle{\pgfqpoint{2.816705in}{0.516222in}}{\pgfqpoint{1.962733in}{1.783528in}} %
\pgfusepath{clip}%
\pgfsetbuttcap%
\pgfsetroundjoin%
\definecolor{currentfill}{rgb}{0.298039,0.447059,0.690196}%
\pgfsetfillcolor{currentfill}%
\pgfsetlinewidth{0.240900pt}%
\definecolor{currentstroke}{rgb}{1.000000,1.000000,1.000000}%
\pgfsetstrokecolor{currentstroke}%
\pgfsetdash{}{0pt}%
\pgfpathmoveto{\pgfqpoint{4.218657in}{1.763361in}}%
\pgfpathcurveto{\pgfqpoint{4.226893in}{1.763361in}}{\pgfqpoint{4.234793in}{1.766633in}}{\pgfqpoint{4.240617in}{1.772457in}}%
\pgfpathcurveto{\pgfqpoint{4.246441in}{1.778281in}}{\pgfqpoint{4.249713in}{1.786181in}}{\pgfqpoint{4.249713in}{1.794417in}}%
\pgfpathcurveto{\pgfqpoint{4.249713in}{1.802653in}}{\pgfqpoint{4.246441in}{1.810553in}}{\pgfqpoint{4.240617in}{1.816377in}}%
\pgfpathcurveto{\pgfqpoint{4.234793in}{1.822201in}}{\pgfqpoint{4.226893in}{1.825474in}}{\pgfqpoint{4.218657in}{1.825474in}}%
\pgfpathcurveto{\pgfqpoint{4.210420in}{1.825474in}}{\pgfqpoint{4.202520in}{1.822201in}}{\pgfqpoint{4.196696in}{1.816377in}}%
\pgfpathcurveto{\pgfqpoint{4.190873in}{1.810553in}}{\pgfqpoint{4.187600in}{1.802653in}}{\pgfqpoint{4.187600in}{1.794417in}}%
\pgfpathcurveto{\pgfqpoint{4.187600in}{1.786181in}}{\pgfqpoint{4.190873in}{1.778281in}}{\pgfqpoint{4.196696in}{1.772457in}}%
\pgfpathcurveto{\pgfqpoint{4.202520in}{1.766633in}}{\pgfqpoint{4.210420in}{1.763361in}}{\pgfqpoint{4.218657in}{1.763361in}}%
\pgfpathclose%
\pgfusepath{stroke,fill}%
\end{pgfscope}%
\begin{pgfscope}%
\pgfsetrectcap%
\pgfsetmiterjoin%
\pgfsetlinewidth{0.000000pt}%
\definecolor{currentstroke}{rgb}{1.000000,1.000000,1.000000}%
\pgfsetstrokecolor{currentstroke}%
\pgfsetdash{}{0pt}%
\pgfpathmoveto{\pgfqpoint{2.816705in}{0.516222in}}%
\pgfpathlineto{\pgfqpoint{2.816705in}{2.299750in}}%
\pgfusepath{}%
\end{pgfscope}%
\begin{pgfscope}%
\pgfsetrectcap%
\pgfsetmiterjoin%
\pgfsetlinewidth{0.000000pt}%
\definecolor{currentstroke}{rgb}{1.000000,1.000000,1.000000}%
\pgfsetstrokecolor{currentstroke}%
\pgfsetdash{}{0pt}%
\pgfpathmoveto{\pgfqpoint{2.816705in}{0.516222in}}%
\pgfpathlineto{\pgfqpoint{4.779438in}{0.516222in}}%
\pgfusepath{}%
\end{pgfscope}%
\end{pgfpicture}%
\makeatother%
\endgroup%

  \caption{Correlation between the arm length (in centimeters) and the falling
  times (in seconds) of the two realizations.}
  \label{fig_al_times}
\end{figure}

\end{document}