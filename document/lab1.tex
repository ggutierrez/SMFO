% -*- root: main.tex -*-
\section{Lab session 1}

\subsection{Linear regression}
For this part we were asked to analyze the data corresponding to two
realizations of the helicopter experiment.

\paragraph{Noise between realizations of the experiment.}

\Cref{fig_t1t2} shows the plot between the times of the experiments for every
realization. The average variance between the time of the experiments is
\SI{0.0425}{\second}. This indicates that there is not much noise present in the
data.

\begin{figure}
  \begin{subfigure}[h]{.5\linewidth}
    %% Creator: Matplotlib, PGF backend
%%
%% To include the figure in your LaTeX document, write
%%   \input{<filename>.pgf}
%%
%% Make sure the required packages are loaded in your preamble
%%   \usepackage{pgf}
%%
%% Figures using additional raster images can only be included by \input if
%% they are in the same directory as the main LaTeX file. For loading figures
%% from other directories you can use the `import` package
%%   \usepackage{import}
%% and then include the figures with
%%   \import{<path to file>}{<filename>.pgf}
%%
%% Matplotlib used the following preamble
%%   \usepackage[utf8x]{inputenc}
%%   \usepackage[T1]{fontenc}
%%   \usepackage{cmbright}
%%
\begingroup%
\makeatletter%
\begin{pgfpicture}%
\pgfpathrectangle{\pgfpointorigin}{\pgfqpoint{2.500000in}{2.500000in}}%
\pgfusepath{use as bounding box, clip}%
\begin{pgfscope}%
\pgfsetbuttcap%
\pgfsetmiterjoin%
\definecolor{currentfill}{rgb}{1.000000,1.000000,1.000000}%
\pgfsetfillcolor{currentfill}%
\pgfsetlinewidth{0.000000pt}%
\definecolor{currentstroke}{rgb}{1.000000,1.000000,1.000000}%
\pgfsetstrokecolor{currentstroke}%
\pgfsetdash{}{0pt}%
\pgfpathmoveto{\pgfqpoint{0.000000in}{0.000000in}}%
\pgfpathlineto{\pgfqpoint{2.500000in}{0.000000in}}%
\pgfpathlineto{\pgfqpoint{2.500000in}{2.500000in}}%
\pgfpathlineto{\pgfqpoint{0.000000in}{2.500000in}}%
\pgfpathclose%
\pgfusepath{fill}%
\end{pgfscope}%
\begin{pgfscope}%
\pgfsetbuttcap%
\pgfsetmiterjoin%
\definecolor{currentfill}{rgb}{0.917647,0.917647,0.949020}%
\pgfsetfillcolor{currentfill}%
\pgfsetlinewidth{0.000000pt}%
\definecolor{currentstroke}{rgb}{0.000000,0.000000,0.000000}%
\pgfsetstrokecolor{currentstroke}%
\pgfsetstrokeopacity{0.000000}%
\pgfsetdash{}{0pt}%
\pgfpathmoveto{\pgfqpoint{0.556847in}{0.516222in}}%
\pgfpathlineto{\pgfqpoint{2.279437in}{0.516222in}}%
\pgfpathlineto{\pgfqpoint{2.279437in}{2.299750in}}%
\pgfpathlineto{\pgfqpoint{0.556847in}{2.299750in}}%
\pgfpathclose%
\pgfusepath{fill}%
\end{pgfscope}%
\begin{pgfscope}%
\pgfpathrectangle{\pgfqpoint{0.556847in}{0.516222in}}{\pgfqpoint{1.722590in}{1.783528in}} %
\pgfusepath{clip}%
\pgfsetroundcap%
\pgfsetroundjoin%
\pgfsetlinewidth{0.803000pt}%
\definecolor{currentstroke}{rgb}{1.000000,1.000000,1.000000}%
\pgfsetstrokecolor{currentstroke}%
\pgfsetdash{}{0pt}%
\pgfpathmoveto{\pgfqpoint{0.556847in}{0.516222in}}%
\pgfpathlineto{\pgfqpoint{0.556847in}{2.299750in}}%
\pgfusepath{stroke}%
\end{pgfscope}%
\begin{pgfscope}%
\pgfsetbuttcap%
\pgfsetroundjoin%
\definecolor{currentfill}{rgb}{0.150000,0.150000,0.150000}%
\pgfsetfillcolor{currentfill}%
\pgfsetlinewidth{0.803000pt}%
\definecolor{currentstroke}{rgb}{0.150000,0.150000,0.150000}%
\pgfsetstrokecolor{currentstroke}%
\pgfsetdash{}{0pt}%
\pgfsys@defobject{currentmarker}{\pgfqpoint{0.000000in}{0.000000in}}{\pgfqpoint{0.000000in}{0.000000in}}{%
\pgfpathmoveto{\pgfqpoint{0.000000in}{0.000000in}}%
\pgfpathlineto{\pgfqpoint{0.000000in}{0.000000in}}%
\pgfusepath{stroke,fill}%
}%
\begin{pgfscope}%
\pgfsys@transformshift{0.556847in}{0.516222in}%
\pgfsys@useobject{currentmarker}{}%
\end{pgfscope}%
\end{pgfscope}%
\begin{pgfscope}%
\definecolor{textcolor}{rgb}{0.150000,0.150000,0.150000}%
\pgfsetstrokecolor{textcolor}%
\pgfsetfillcolor{textcolor}%
\pgftext[x=0.556847in,y=0.438444in,,top]{\color{textcolor}\sffamily\fontsize{8.000000}{9.600000}\selectfont 1.5}%
\end{pgfscope}%
\begin{pgfscope}%
\pgfpathrectangle{\pgfqpoint{0.556847in}{0.516222in}}{\pgfqpoint{1.722590in}{1.783528in}} %
\pgfusepath{clip}%
\pgfsetroundcap%
\pgfsetroundjoin%
\pgfsetlinewidth{0.803000pt}%
\definecolor{currentstroke}{rgb}{1.000000,1.000000,1.000000}%
\pgfsetstrokecolor{currentstroke}%
\pgfsetdash{}{0pt}%
\pgfpathmoveto{\pgfqpoint{0.802932in}{0.516222in}}%
\pgfpathlineto{\pgfqpoint{0.802932in}{2.299750in}}%
\pgfusepath{stroke}%
\end{pgfscope}%
\begin{pgfscope}%
\pgfsetbuttcap%
\pgfsetroundjoin%
\definecolor{currentfill}{rgb}{0.150000,0.150000,0.150000}%
\pgfsetfillcolor{currentfill}%
\pgfsetlinewidth{0.803000pt}%
\definecolor{currentstroke}{rgb}{0.150000,0.150000,0.150000}%
\pgfsetstrokecolor{currentstroke}%
\pgfsetdash{}{0pt}%
\pgfsys@defobject{currentmarker}{\pgfqpoint{0.000000in}{0.000000in}}{\pgfqpoint{0.000000in}{0.000000in}}{%
\pgfpathmoveto{\pgfqpoint{0.000000in}{0.000000in}}%
\pgfpathlineto{\pgfqpoint{0.000000in}{0.000000in}}%
\pgfusepath{stroke,fill}%
}%
\begin{pgfscope}%
\pgfsys@transformshift{0.802932in}{0.516222in}%
\pgfsys@useobject{currentmarker}{}%
\end{pgfscope}%
\end{pgfscope}%
\begin{pgfscope}%
\definecolor{textcolor}{rgb}{0.150000,0.150000,0.150000}%
\pgfsetstrokecolor{textcolor}%
\pgfsetfillcolor{textcolor}%
\pgftext[x=0.802932in,y=0.438444in,,top]{\color{textcolor}\sffamily\fontsize{8.000000}{9.600000}\selectfont 2.0}%
\end{pgfscope}%
\begin{pgfscope}%
\pgfpathrectangle{\pgfqpoint{0.556847in}{0.516222in}}{\pgfqpoint{1.722590in}{1.783528in}} %
\pgfusepath{clip}%
\pgfsetroundcap%
\pgfsetroundjoin%
\pgfsetlinewidth{0.803000pt}%
\definecolor{currentstroke}{rgb}{1.000000,1.000000,1.000000}%
\pgfsetstrokecolor{currentstroke}%
\pgfsetdash{}{0pt}%
\pgfpathmoveto{\pgfqpoint{1.049016in}{0.516222in}}%
\pgfpathlineto{\pgfqpoint{1.049016in}{2.299750in}}%
\pgfusepath{stroke}%
\end{pgfscope}%
\begin{pgfscope}%
\pgfsetbuttcap%
\pgfsetroundjoin%
\definecolor{currentfill}{rgb}{0.150000,0.150000,0.150000}%
\pgfsetfillcolor{currentfill}%
\pgfsetlinewidth{0.803000pt}%
\definecolor{currentstroke}{rgb}{0.150000,0.150000,0.150000}%
\pgfsetstrokecolor{currentstroke}%
\pgfsetdash{}{0pt}%
\pgfsys@defobject{currentmarker}{\pgfqpoint{0.000000in}{0.000000in}}{\pgfqpoint{0.000000in}{0.000000in}}{%
\pgfpathmoveto{\pgfqpoint{0.000000in}{0.000000in}}%
\pgfpathlineto{\pgfqpoint{0.000000in}{0.000000in}}%
\pgfusepath{stroke,fill}%
}%
\begin{pgfscope}%
\pgfsys@transformshift{1.049016in}{0.516222in}%
\pgfsys@useobject{currentmarker}{}%
\end{pgfscope}%
\end{pgfscope}%
\begin{pgfscope}%
\definecolor{textcolor}{rgb}{0.150000,0.150000,0.150000}%
\pgfsetstrokecolor{textcolor}%
\pgfsetfillcolor{textcolor}%
\pgftext[x=1.049016in,y=0.438444in,,top]{\color{textcolor}\sffamily\fontsize{8.000000}{9.600000}\selectfont 2.5}%
\end{pgfscope}%
\begin{pgfscope}%
\pgfpathrectangle{\pgfqpoint{0.556847in}{0.516222in}}{\pgfqpoint{1.722590in}{1.783528in}} %
\pgfusepath{clip}%
\pgfsetroundcap%
\pgfsetroundjoin%
\pgfsetlinewidth{0.803000pt}%
\definecolor{currentstroke}{rgb}{1.000000,1.000000,1.000000}%
\pgfsetstrokecolor{currentstroke}%
\pgfsetdash{}{0pt}%
\pgfpathmoveto{\pgfqpoint{1.295100in}{0.516222in}}%
\pgfpathlineto{\pgfqpoint{1.295100in}{2.299750in}}%
\pgfusepath{stroke}%
\end{pgfscope}%
\begin{pgfscope}%
\pgfsetbuttcap%
\pgfsetroundjoin%
\definecolor{currentfill}{rgb}{0.150000,0.150000,0.150000}%
\pgfsetfillcolor{currentfill}%
\pgfsetlinewidth{0.803000pt}%
\definecolor{currentstroke}{rgb}{0.150000,0.150000,0.150000}%
\pgfsetstrokecolor{currentstroke}%
\pgfsetdash{}{0pt}%
\pgfsys@defobject{currentmarker}{\pgfqpoint{0.000000in}{0.000000in}}{\pgfqpoint{0.000000in}{0.000000in}}{%
\pgfpathmoveto{\pgfqpoint{0.000000in}{0.000000in}}%
\pgfpathlineto{\pgfqpoint{0.000000in}{0.000000in}}%
\pgfusepath{stroke,fill}%
}%
\begin{pgfscope}%
\pgfsys@transformshift{1.295100in}{0.516222in}%
\pgfsys@useobject{currentmarker}{}%
\end{pgfscope}%
\end{pgfscope}%
\begin{pgfscope}%
\definecolor{textcolor}{rgb}{0.150000,0.150000,0.150000}%
\pgfsetstrokecolor{textcolor}%
\pgfsetfillcolor{textcolor}%
\pgftext[x=1.295100in,y=0.438444in,,top]{\color{textcolor}\sffamily\fontsize{8.000000}{9.600000}\selectfont 3.0}%
\end{pgfscope}%
\begin{pgfscope}%
\pgfpathrectangle{\pgfqpoint{0.556847in}{0.516222in}}{\pgfqpoint{1.722590in}{1.783528in}} %
\pgfusepath{clip}%
\pgfsetroundcap%
\pgfsetroundjoin%
\pgfsetlinewidth{0.803000pt}%
\definecolor{currentstroke}{rgb}{1.000000,1.000000,1.000000}%
\pgfsetstrokecolor{currentstroke}%
\pgfsetdash{}{0pt}%
\pgfpathmoveto{\pgfqpoint{1.541185in}{0.516222in}}%
\pgfpathlineto{\pgfqpoint{1.541185in}{2.299750in}}%
\pgfusepath{stroke}%
\end{pgfscope}%
\begin{pgfscope}%
\pgfsetbuttcap%
\pgfsetroundjoin%
\definecolor{currentfill}{rgb}{0.150000,0.150000,0.150000}%
\pgfsetfillcolor{currentfill}%
\pgfsetlinewidth{0.803000pt}%
\definecolor{currentstroke}{rgb}{0.150000,0.150000,0.150000}%
\pgfsetstrokecolor{currentstroke}%
\pgfsetdash{}{0pt}%
\pgfsys@defobject{currentmarker}{\pgfqpoint{0.000000in}{0.000000in}}{\pgfqpoint{0.000000in}{0.000000in}}{%
\pgfpathmoveto{\pgfqpoint{0.000000in}{0.000000in}}%
\pgfpathlineto{\pgfqpoint{0.000000in}{0.000000in}}%
\pgfusepath{stroke,fill}%
}%
\begin{pgfscope}%
\pgfsys@transformshift{1.541185in}{0.516222in}%
\pgfsys@useobject{currentmarker}{}%
\end{pgfscope}%
\end{pgfscope}%
\begin{pgfscope}%
\definecolor{textcolor}{rgb}{0.150000,0.150000,0.150000}%
\pgfsetstrokecolor{textcolor}%
\pgfsetfillcolor{textcolor}%
\pgftext[x=1.541185in,y=0.438444in,,top]{\color{textcolor}\sffamily\fontsize{8.000000}{9.600000}\selectfont 3.5}%
\end{pgfscope}%
\begin{pgfscope}%
\pgfpathrectangle{\pgfqpoint{0.556847in}{0.516222in}}{\pgfqpoint{1.722590in}{1.783528in}} %
\pgfusepath{clip}%
\pgfsetroundcap%
\pgfsetroundjoin%
\pgfsetlinewidth{0.803000pt}%
\definecolor{currentstroke}{rgb}{1.000000,1.000000,1.000000}%
\pgfsetstrokecolor{currentstroke}%
\pgfsetdash{}{0pt}%
\pgfpathmoveto{\pgfqpoint{1.787269in}{0.516222in}}%
\pgfpathlineto{\pgfqpoint{1.787269in}{2.299750in}}%
\pgfusepath{stroke}%
\end{pgfscope}%
\begin{pgfscope}%
\pgfsetbuttcap%
\pgfsetroundjoin%
\definecolor{currentfill}{rgb}{0.150000,0.150000,0.150000}%
\pgfsetfillcolor{currentfill}%
\pgfsetlinewidth{0.803000pt}%
\definecolor{currentstroke}{rgb}{0.150000,0.150000,0.150000}%
\pgfsetstrokecolor{currentstroke}%
\pgfsetdash{}{0pt}%
\pgfsys@defobject{currentmarker}{\pgfqpoint{0.000000in}{0.000000in}}{\pgfqpoint{0.000000in}{0.000000in}}{%
\pgfpathmoveto{\pgfqpoint{0.000000in}{0.000000in}}%
\pgfpathlineto{\pgfqpoint{0.000000in}{0.000000in}}%
\pgfusepath{stroke,fill}%
}%
\begin{pgfscope}%
\pgfsys@transformshift{1.787269in}{0.516222in}%
\pgfsys@useobject{currentmarker}{}%
\end{pgfscope}%
\end{pgfscope}%
\begin{pgfscope}%
\definecolor{textcolor}{rgb}{0.150000,0.150000,0.150000}%
\pgfsetstrokecolor{textcolor}%
\pgfsetfillcolor{textcolor}%
\pgftext[x=1.787269in,y=0.438444in,,top]{\color{textcolor}\sffamily\fontsize{8.000000}{9.600000}\selectfont 4.0}%
\end{pgfscope}%
\begin{pgfscope}%
\pgfpathrectangle{\pgfqpoint{0.556847in}{0.516222in}}{\pgfqpoint{1.722590in}{1.783528in}} %
\pgfusepath{clip}%
\pgfsetroundcap%
\pgfsetroundjoin%
\pgfsetlinewidth{0.803000pt}%
\definecolor{currentstroke}{rgb}{1.000000,1.000000,1.000000}%
\pgfsetstrokecolor{currentstroke}%
\pgfsetdash{}{0pt}%
\pgfpathmoveto{\pgfqpoint{2.033353in}{0.516222in}}%
\pgfpathlineto{\pgfqpoint{2.033353in}{2.299750in}}%
\pgfusepath{stroke}%
\end{pgfscope}%
\begin{pgfscope}%
\pgfsetbuttcap%
\pgfsetroundjoin%
\definecolor{currentfill}{rgb}{0.150000,0.150000,0.150000}%
\pgfsetfillcolor{currentfill}%
\pgfsetlinewidth{0.803000pt}%
\definecolor{currentstroke}{rgb}{0.150000,0.150000,0.150000}%
\pgfsetstrokecolor{currentstroke}%
\pgfsetdash{}{0pt}%
\pgfsys@defobject{currentmarker}{\pgfqpoint{0.000000in}{0.000000in}}{\pgfqpoint{0.000000in}{0.000000in}}{%
\pgfpathmoveto{\pgfqpoint{0.000000in}{0.000000in}}%
\pgfpathlineto{\pgfqpoint{0.000000in}{0.000000in}}%
\pgfusepath{stroke,fill}%
}%
\begin{pgfscope}%
\pgfsys@transformshift{2.033353in}{0.516222in}%
\pgfsys@useobject{currentmarker}{}%
\end{pgfscope}%
\end{pgfscope}%
\begin{pgfscope}%
\definecolor{textcolor}{rgb}{0.150000,0.150000,0.150000}%
\pgfsetstrokecolor{textcolor}%
\pgfsetfillcolor{textcolor}%
\pgftext[x=2.033353in,y=0.438444in,,top]{\color{textcolor}\sffamily\fontsize{8.000000}{9.600000}\selectfont 4.5}%
\end{pgfscope}%
\begin{pgfscope}%
\pgfpathrectangle{\pgfqpoint{0.556847in}{0.516222in}}{\pgfqpoint{1.722590in}{1.783528in}} %
\pgfusepath{clip}%
\pgfsetroundcap%
\pgfsetroundjoin%
\pgfsetlinewidth{0.803000pt}%
\definecolor{currentstroke}{rgb}{1.000000,1.000000,1.000000}%
\pgfsetstrokecolor{currentstroke}%
\pgfsetdash{}{0pt}%
\pgfpathmoveto{\pgfqpoint{2.279437in}{0.516222in}}%
\pgfpathlineto{\pgfqpoint{2.279437in}{2.299750in}}%
\pgfusepath{stroke}%
\end{pgfscope}%
\begin{pgfscope}%
\pgfsetbuttcap%
\pgfsetroundjoin%
\definecolor{currentfill}{rgb}{0.150000,0.150000,0.150000}%
\pgfsetfillcolor{currentfill}%
\pgfsetlinewidth{0.803000pt}%
\definecolor{currentstroke}{rgb}{0.150000,0.150000,0.150000}%
\pgfsetstrokecolor{currentstroke}%
\pgfsetdash{}{0pt}%
\pgfsys@defobject{currentmarker}{\pgfqpoint{0.000000in}{0.000000in}}{\pgfqpoint{0.000000in}{0.000000in}}{%
\pgfpathmoveto{\pgfqpoint{0.000000in}{0.000000in}}%
\pgfpathlineto{\pgfqpoint{0.000000in}{0.000000in}}%
\pgfusepath{stroke,fill}%
}%
\begin{pgfscope}%
\pgfsys@transformshift{2.279437in}{0.516222in}%
\pgfsys@useobject{currentmarker}{}%
\end{pgfscope}%
\end{pgfscope}%
\begin{pgfscope}%
\definecolor{textcolor}{rgb}{0.150000,0.150000,0.150000}%
\pgfsetstrokecolor{textcolor}%
\pgfsetfillcolor{textcolor}%
\pgftext[x=2.279437in,y=0.438444in,,top]{\color{textcolor}\sffamily\fontsize{8.000000}{9.600000}\selectfont 5.0}%
\end{pgfscope}%
\begin{pgfscope}%
\definecolor{textcolor}{rgb}{0.150000,0.150000,0.150000}%
\pgfsetstrokecolor{textcolor}%
\pgfsetfillcolor{textcolor}%
\pgftext[x=1.418142in,y=0.273321in,,top]{\color{textcolor}\sffamily\fontsize{8.800000}{10.560000}\selectfont Falling time realization 1}%
\end{pgfscope}%
\begin{pgfscope}%
\pgfpathrectangle{\pgfqpoint{0.556847in}{0.516222in}}{\pgfqpoint{1.722590in}{1.783528in}} %
\pgfusepath{clip}%
\pgfsetroundcap%
\pgfsetroundjoin%
\pgfsetlinewidth{0.803000pt}%
\definecolor{currentstroke}{rgb}{1.000000,1.000000,1.000000}%
\pgfsetstrokecolor{currentstroke}%
\pgfsetdash{}{0pt}%
\pgfpathmoveto{\pgfqpoint{0.556847in}{0.516222in}}%
\pgfpathlineto{\pgfqpoint{2.279437in}{0.516222in}}%
\pgfusepath{stroke}%
\end{pgfscope}%
\begin{pgfscope}%
\pgfsetbuttcap%
\pgfsetroundjoin%
\definecolor{currentfill}{rgb}{0.150000,0.150000,0.150000}%
\pgfsetfillcolor{currentfill}%
\pgfsetlinewidth{0.803000pt}%
\definecolor{currentstroke}{rgb}{0.150000,0.150000,0.150000}%
\pgfsetstrokecolor{currentstroke}%
\pgfsetdash{}{0pt}%
\pgfsys@defobject{currentmarker}{\pgfqpoint{0.000000in}{0.000000in}}{\pgfqpoint{0.000000in}{0.000000in}}{%
\pgfpathmoveto{\pgfqpoint{0.000000in}{0.000000in}}%
\pgfpathlineto{\pgfqpoint{0.000000in}{0.000000in}}%
\pgfusepath{stroke,fill}%
}%
\begin{pgfscope}%
\pgfsys@transformshift{0.556847in}{0.516222in}%
\pgfsys@useobject{currentmarker}{}%
\end{pgfscope}%
\end{pgfscope}%
\begin{pgfscope}%
\definecolor{textcolor}{rgb}{0.150000,0.150000,0.150000}%
\pgfsetstrokecolor{textcolor}%
\pgfsetfillcolor{textcolor}%
\pgftext[x=0.479069in,y=0.516222in,right,]{\color{textcolor}\sffamily\fontsize{8.000000}{9.600000}\selectfont 2.0}%
\end{pgfscope}%
\begin{pgfscope}%
\pgfpathrectangle{\pgfqpoint{0.556847in}{0.516222in}}{\pgfqpoint{1.722590in}{1.783528in}} %
\pgfusepath{clip}%
\pgfsetroundcap%
\pgfsetroundjoin%
\pgfsetlinewidth{0.803000pt}%
\definecolor{currentstroke}{rgb}{1.000000,1.000000,1.000000}%
\pgfsetstrokecolor{currentstroke}%
\pgfsetdash{}{0pt}%
\pgfpathmoveto{\pgfqpoint{0.556847in}{0.771012in}}%
\pgfpathlineto{\pgfqpoint{2.279437in}{0.771012in}}%
\pgfusepath{stroke}%
\end{pgfscope}%
\begin{pgfscope}%
\pgfsetbuttcap%
\pgfsetroundjoin%
\definecolor{currentfill}{rgb}{0.150000,0.150000,0.150000}%
\pgfsetfillcolor{currentfill}%
\pgfsetlinewidth{0.803000pt}%
\definecolor{currentstroke}{rgb}{0.150000,0.150000,0.150000}%
\pgfsetstrokecolor{currentstroke}%
\pgfsetdash{}{0pt}%
\pgfsys@defobject{currentmarker}{\pgfqpoint{0.000000in}{0.000000in}}{\pgfqpoint{0.000000in}{0.000000in}}{%
\pgfpathmoveto{\pgfqpoint{0.000000in}{0.000000in}}%
\pgfpathlineto{\pgfqpoint{0.000000in}{0.000000in}}%
\pgfusepath{stroke,fill}%
}%
\begin{pgfscope}%
\pgfsys@transformshift{0.556847in}{0.771012in}%
\pgfsys@useobject{currentmarker}{}%
\end{pgfscope}%
\end{pgfscope}%
\begin{pgfscope}%
\definecolor{textcolor}{rgb}{0.150000,0.150000,0.150000}%
\pgfsetstrokecolor{textcolor}%
\pgfsetfillcolor{textcolor}%
\pgftext[x=0.479069in,y=0.771012in,right,]{\color{textcolor}\sffamily\fontsize{8.000000}{9.600000}\selectfont 2.5}%
\end{pgfscope}%
\begin{pgfscope}%
\pgfpathrectangle{\pgfqpoint{0.556847in}{0.516222in}}{\pgfqpoint{1.722590in}{1.783528in}} %
\pgfusepath{clip}%
\pgfsetroundcap%
\pgfsetroundjoin%
\pgfsetlinewidth{0.803000pt}%
\definecolor{currentstroke}{rgb}{1.000000,1.000000,1.000000}%
\pgfsetstrokecolor{currentstroke}%
\pgfsetdash{}{0pt}%
\pgfpathmoveto{\pgfqpoint{0.556847in}{1.025802in}}%
\pgfpathlineto{\pgfqpoint{2.279437in}{1.025802in}}%
\pgfusepath{stroke}%
\end{pgfscope}%
\begin{pgfscope}%
\pgfsetbuttcap%
\pgfsetroundjoin%
\definecolor{currentfill}{rgb}{0.150000,0.150000,0.150000}%
\pgfsetfillcolor{currentfill}%
\pgfsetlinewidth{0.803000pt}%
\definecolor{currentstroke}{rgb}{0.150000,0.150000,0.150000}%
\pgfsetstrokecolor{currentstroke}%
\pgfsetdash{}{0pt}%
\pgfsys@defobject{currentmarker}{\pgfqpoint{0.000000in}{0.000000in}}{\pgfqpoint{0.000000in}{0.000000in}}{%
\pgfpathmoveto{\pgfqpoint{0.000000in}{0.000000in}}%
\pgfpathlineto{\pgfqpoint{0.000000in}{0.000000in}}%
\pgfusepath{stroke,fill}%
}%
\begin{pgfscope}%
\pgfsys@transformshift{0.556847in}{1.025802in}%
\pgfsys@useobject{currentmarker}{}%
\end{pgfscope}%
\end{pgfscope}%
\begin{pgfscope}%
\definecolor{textcolor}{rgb}{0.150000,0.150000,0.150000}%
\pgfsetstrokecolor{textcolor}%
\pgfsetfillcolor{textcolor}%
\pgftext[x=0.479069in,y=1.025802in,right,]{\color{textcolor}\sffamily\fontsize{8.000000}{9.600000}\selectfont 3.0}%
\end{pgfscope}%
\begin{pgfscope}%
\pgfpathrectangle{\pgfqpoint{0.556847in}{0.516222in}}{\pgfqpoint{1.722590in}{1.783528in}} %
\pgfusepath{clip}%
\pgfsetroundcap%
\pgfsetroundjoin%
\pgfsetlinewidth{0.803000pt}%
\definecolor{currentstroke}{rgb}{1.000000,1.000000,1.000000}%
\pgfsetstrokecolor{currentstroke}%
\pgfsetdash{}{0pt}%
\pgfpathmoveto{\pgfqpoint{0.556847in}{1.280591in}}%
\pgfpathlineto{\pgfqpoint{2.279437in}{1.280591in}}%
\pgfusepath{stroke}%
\end{pgfscope}%
\begin{pgfscope}%
\pgfsetbuttcap%
\pgfsetroundjoin%
\definecolor{currentfill}{rgb}{0.150000,0.150000,0.150000}%
\pgfsetfillcolor{currentfill}%
\pgfsetlinewidth{0.803000pt}%
\definecolor{currentstroke}{rgb}{0.150000,0.150000,0.150000}%
\pgfsetstrokecolor{currentstroke}%
\pgfsetdash{}{0pt}%
\pgfsys@defobject{currentmarker}{\pgfqpoint{0.000000in}{0.000000in}}{\pgfqpoint{0.000000in}{0.000000in}}{%
\pgfpathmoveto{\pgfqpoint{0.000000in}{0.000000in}}%
\pgfpathlineto{\pgfqpoint{0.000000in}{0.000000in}}%
\pgfusepath{stroke,fill}%
}%
\begin{pgfscope}%
\pgfsys@transformshift{0.556847in}{1.280591in}%
\pgfsys@useobject{currentmarker}{}%
\end{pgfscope}%
\end{pgfscope}%
\begin{pgfscope}%
\definecolor{textcolor}{rgb}{0.150000,0.150000,0.150000}%
\pgfsetstrokecolor{textcolor}%
\pgfsetfillcolor{textcolor}%
\pgftext[x=0.479069in,y=1.280591in,right,]{\color{textcolor}\sffamily\fontsize{8.000000}{9.600000}\selectfont 3.5}%
\end{pgfscope}%
\begin{pgfscope}%
\pgfpathrectangle{\pgfqpoint{0.556847in}{0.516222in}}{\pgfqpoint{1.722590in}{1.783528in}} %
\pgfusepath{clip}%
\pgfsetroundcap%
\pgfsetroundjoin%
\pgfsetlinewidth{0.803000pt}%
\definecolor{currentstroke}{rgb}{1.000000,1.000000,1.000000}%
\pgfsetstrokecolor{currentstroke}%
\pgfsetdash{}{0pt}%
\pgfpathmoveto{\pgfqpoint{0.556847in}{1.535381in}}%
\pgfpathlineto{\pgfqpoint{2.279437in}{1.535381in}}%
\pgfusepath{stroke}%
\end{pgfscope}%
\begin{pgfscope}%
\pgfsetbuttcap%
\pgfsetroundjoin%
\definecolor{currentfill}{rgb}{0.150000,0.150000,0.150000}%
\pgfsetfillcolor{currentfill}%
\pgfsetlinewidth{0.803000pt}%
\definecolor{currentstroke}{rgb}{0.150000,0.150000,0.150000}%
\pgfsetstrokecolor{currentstroke}%
\pgfsetdash{}{0pt}%
\pgfsys@defobject{currentmarker}{\pgfqpoint{0.000000in}{0.000000in}}{\pgfqpoint{0.000000in}{0.000000in}}{%
\pgfpathmoveto{\pgfqpoint{0.000000in}{0.000000in}}%
\pgfpathlineto{\pgfqpoint{0.000000in}{0.000000in}}%
\pgfusepath{stroke,fill}%
}%
\begin{pgfscope}%
\pgfsys@transformshift{0.556847in}{1.535381in}%
\pgfsys@useobject{currentmarker}{}%
\end{pgfscope}%
\end{pgfscope}%
\begin{pgfscope}%
\definecolor{textcolor}{rgb}{0.150000,0.150000,0.150000}%
\pgfsetstrokecolor{textcolor}%
\pgfsetfillcolor{textcolor}%
\pgftext[x=0.479069in,y=1.535381in,right,]{\color{textcolor}\sffamily\fontsize{8.000000}{9.600000}\selectfont 4.0}%
\end{pgfscope}%
\begin{pgfscope}%
\pgfpathrectangle{\pgfqpoint{0.556847in}{0.516222in}}{\pgfqpoint{1.722590in}{1.783528in}} %
\pgfusepath{clip}%
\pgfsetroundcap%
\pgfsetroundjoin%
\pgfsetlinewidth{0.803000pt}%
\definecolor{currentstroke}{rgb}{1.000000,1.000000,1.000000}%
\pgfsetstrokecolor{currentstroke}%
\pgfsetdash{}{0pt}%
\pgfpathmoveto{\pgfqpoint{0.556847in}{1.790171in}}%
\pgfpathlineto{\pgfqpoint{2.279437in}{1.790171in}}%
\pgfusepath{stroke}%
\end{pgfscope}%
\begin{pgfscope}%
\pgfsetbuttcap%
\pgfsetroundjoin%
\definecolor{currentfill}{rgb}{0.150000,0.150000,0.150000}%
\pgfsetfillcolor{currentfill}%
\pgfsetlinewidth{0.803000pt}%
\definecolor{currentstroke}{rgb}{0.150000,0.150000,0.150000}%
\pgfsetstrokecolor{currentstroke}%
\pgfsetdash{}{0pt}%
\pgfsys@defobject{currentmarker}{\pgfqpoint{0.000000in}{0.000000in}}{\pgfqpoint{0.000000in}{0.000000in}}{%
\pgfpathmoveto{\pgfqpoint{0.000000in}{0.000000in}}%
\pgfpathlineto{\pgfqpoint{0.000000in}{0.000000in}}%
\pgfusepath{stroke,fill}%
}%
\begin{pgfscope}%
\pgfsys@transformshift{0.556847in}{1.790171in}%
\pgfsys@useobject{currentmarker}{}%
\end{pgfscope}%
\end{pgfscope}%
\begin{pgfscope}%
\definecolor{textcolor}{rgb}{0.150000,0.150000,0.150000}%
\pgfsetstrokecolor{textcolor}%
\pgfsetfillcolor{textcolor}%
\pgftext[x=0.479069in,y=1.790171in,right,]{\color{textcolor}\sffamily\fontsize{8.000000}{9.600000}\selectfont 4.5}%
\end{pgfscope}%
\begin{pgfscope}%
\pgfpathrectangle{\pgfqpoint{0.556847in}{0.516222in}}{\pgfqpoint{1.722590in}{1.783528in}} %
\pgfusepath{clip}%
\pgfsetroundcap%
\pgfsetroundjoin%
\pgfsetlinewidth{0.803000pt}%
\definecolor{currentstroke}{rgb}{1.000000,1.000000,1.000000}%
\pgfsetstrokecolor{currentstroke}%
\pgfsetdash{}{0pt}%
\pgfpathmoveto{\pgfqpoint{0.556847in}{2.044960in}}%
\pgfpathlineto{\pgfqpoint{2.279437in}{2.044960in}}%
\pgfusepath{stroke}%
\end{pgfscope}%
\begin{pgfscope}%
\pgfsetbuttcap%
\pgfsetroundjoin%
\definecolor{currentfill}{rgb}{0.150000,0.150000,0.150000}%
\pgfsetfillcolor{currentfill}%
\pgfsetlinewidth{0.803000pt}%
\definecolor{currentstroke}{rgb}{0.150000,0.150000,0.150000}%
\pgfsetstrokecolor{currentstroke}%
\pgfsetdash{}{0pt}%
\pgfsys@defobject{currentmarker}{\pgfqpoint{0.000000in}{0.000000in}}{\pgfqpoint{0.000000in}{0.000000in}}{%
\pgfpathmoveto{\pgfqpoint{0.000000in}{0.000000in}}%
\pgfpathlineto{\pgfqpoint{0.000000in}{0.000000in}}%
\pgfusepath{stroke,fill}%
}%
\begin{pgfscope}%
\pgfsys@transformshift{0.556847in}{2.044960in}%
\pgfsys@useobject{currentmarker}{}%
\end{pgfscope}%
\end{pgfscope}%
\begin{pgfscope}%
\definecolor{textcolor}{rgb}{0.150000,0.150000,0.150000}%
\pgfsetstrokecolor{textcolor}%
\pgfsetfillcolor{textcolor}%
\pgftext[x=0.479069in,y=2.044960in,right,]{\color{textcolor}\sffamily\fontsize{8.000000}{9.600000}\selectfont 5.0}%
\end{pgfscope}%
\begin{pgfscope}%
\pgfpathrectangle{\pgfqpoint{0.556847in}{0.516222in}}{\pgfqpoint{1.722590in}{1.783528in}} %
\pgfusepath{clip}%
\pgfsetroundcap%
\pgfsetroundjoin%
\pgfsetlinewidth{0.803000pt}%
\definecolor{currentstroke}{rgb}{1.000000,1.000000,1.000000}%
\pgfsetstrokecolor{currentstroke}%
\pgfsetdash{}{0pt}%
\pgfpathmoveto{\pgfqpoint{0.556847in}{2.299750in}}%
\pgfpathlineto{\pgfqpoint{2.279437in}{2.299750in}}%
\pgfusepath{stroke}%
\end{pgfscope}%
\begin{pgfscope}%
\pgfsetbuttcap%
\pgfsetroundjoin%
\definecolor{currentfill}{rgb}{0.150000,0.150000,0.150000}%
\pgfsetfillcolor{currentfill}%
\pgfsetlinewidth{0.803000pt}%
\definecolor{currentstroke}{rgb}{0.150000,0.150000,0.150000}%
\pgfsetstrokecolor{currentstroke}%
\pgfsetdash{}{0pt}%
\pgfsys@defobject{currentmarker}{\pgfqpoint{0.000000in}{0.000000in}}{\pgfqpoint{0.000000in}{0.000000in}}{%
\pgfpathmoveto{\pgfqpoint{0.000000in}{0.000000in}}%
\pgfpathlineto{\pgfqpoint{0.000000in}{0.000000in}}%
\pgfusepath{stroke,fill}%
}%
\begin{pgfscope}%
\pgfsys@transformshift{0.556847in}{2.299750in}%
\pgfsys@useobject{currentmarker}{}%
\end{pgfscope}%
\end{pgfscope}%
\begin{pgfscope}%
\definecolor{textcolor}{rgb}{0.150000,0.150000,0.150000}%
\pgfsetstrokecolor{textcolor}%
\pgfsetfillcolor{textcolor}%
\pgftext[x=0.479069in,y=2.299750in,right,]{\color{textcolor}\sffamily\fontsize{8.000000}{9.600000}\selectfont 5.5}%
\end{pgfscope}%
\begin{pgfscope}%
\definecolor{textcolor}{rgb}{0.150000,0.150000,0.150000}%
\pgfsetstrokecolor{textcolor}%
\pgfsetfillcolor{textcolor}%
\pgftext[x=0.251677in,y=1.407986in,,bottom,rotate=90.000000]{\color{textcolor}\sffamily\fontsize{8.800000}{10.560000}\selectfont Falling time realization 2}%
\end{pgfscope}%
\begin{pgfscope}%
\pgfpathrectangle{\pgfqpoint{0.556847in}{0.516222in}}{\pgfqpoint{1.722590in}{1.783528in}} %
\pgfusepath{clip}%
\pgfsetbuttcap%
\pgfsetroundjoin%
\definecolor{currentfill}{rgb}{0.298039,0.447059,0.690196}%
\pgfsetfillcolor{currentfill}%
\pgfsetlinewidth{0.240900pt}%
\definecolor{currentstroke}{rgb}{1.000000,1.000000,1.000000}%
\pgfsetstrokecolor{currentstroke}%
\pgfsetdash{}{0pt}%
\pgfpathmoveto{\pgfqpoint{1.738052in}{1.402409in}}%
\pgfpathcurveto{\pgfqpoint{1.746288in}{1.402409in}}{\pgfqpoint{1.754188in}{1.405681in}}{\pgfqpoint{1.760012in}{1.411505in}}%
\pgfpathcurveto{\pgfqpoint{1.765836in}{1.417329in}}{\pgfqpoint{1.769108in}{1.425229in}}{\pgfqpoint{1.769108in}{1.433465in}}%
\pgfpathcurveto{\pgfqpoint{1.769108in}{1.441701in}}{\pgfqpoint{1.765836in}{1.449601in}}{\pgfqpoint{1.760012in}{1.455425in}}%
\pgfpathcurveto{\pgfqpoint{1.754188in}{1.461249in}}{\pgfqpoint{1.746288in}{1.464522in}}{\pgfqpoint{1.738052in}{1.464522in}}%
\pgfpathcurveto{\pgfqpoint{1.729816in}{1.464522in}}{\pgfqpoint{1.721916in}{1.461249in}}{\pgfqpoint{1.716092in}{1.455425in}}%
\pgfpathcurveto{\pgfqpoint{1.710268in}{1.449601in}}{\pgfqpoint{1.706995in}{1.441701in}}{\pgfqpoint{1.706995in}{1.433465in}}%
\pgfpathcurveto{\pgfqpoint{1.706995in}{1.425229in}}{\pgfqpoint{1.710268in}{1.417329in}}{\pgfqpoint{1.716092in}{1.411505in}}%
\pgfpathcurveto{\pgfqpoint{1.721916in}{1.405681in}}{\pgfqpoint{1.729816in}{1.402409in}}{\pgfqpoint{1.738052in}{1.402409in}}%
\pgfpathclose%
\pgfusepath{stroke,fill}%
\end{pgfscope}%
\begin{pgfscope}%
\pgfpathrectangle{\pgfqpoint{0.556847in}{0.516222in}}{\pgfqpoint{1.722590in}{1.783528in}} %
\pgfusepath{clip}%
\pgfsetbuttcap%
\pgfsetroundjoin%
\definecolor{currentfill}{rgb}{0.298039,0.447059,0.690196}%
\pgfsetfillcolor{currentfill}%
\pgfsetlinewidth{0.240900pt}%
\definecolor{currentstroke}{rgb}{1.000000,1.000000,1.000000}%
\pgfsetstrokecolor{currentstroke}%
\pgfsetdash{}{0pt}%
\pgfpathmoveto{\pgfqpoint{1.491968in}{1.249535in}}%
\pgfpathcurveto{\pgfqpoint{1.500204in}{1.249535in}}{\pgfqpoint{1.508104in}{1.252807in}}{\pgfqpoint{1.513928in}{1.258631in}}%
\pgfpathcurveto{\pgfqpoint{1.519752in}{1.264455in}}{\pgfqpoint{1.523024in}{1.272355in}}{\pgfqpoint{1.523024in}{1.280591in}}%
\pgfpathcurveto{\pgfqpoint{1.523024in}{1.288828in}}{\pgfqpoint{1.519752in}{1.296728in}}{\pgfqpoint{1.513928in}{1.302552in}}%
\pgfpathcurveto{\pgfqpoint{1.508104in}{1.308375in}}{\pgfqpoint{1.500204in}{1.311648in}}{\pgfqpoint{1.491968in}{1.311648in}}%
\pgfpathcurveto{\pgfqpoint{1.483731in}{1.311648in}}{\pgfqpoint{1.475831in}{1.308375in}}{\pgfqpoint{1.470007in}{1.302552in}}%
\pgfpathcurveto{\pgfqpoint{1.464183in}{1.296728in}}{\pgfqpoint{1.460911in}{1.288828in}}{\pgfqpoint{1.460911in}{1.280591in}}%
\pgfpathcurveto{\pgfqpoint{1.460911in}{1.272355in}}{\pgfqpoint{1.464183in}{1.264455in}}{\pgfqpoint{1.470007in}{1.258631in}}%
\pgfpathcurveto{\pgfqpoint{1.475831in}{1.252807in}}{\pgfqpoint{1.483731in}{1.249535in}}{\pgfqpoint{1.491968in}{1.249535in}}%
\pgfpathclose%
\pgfusepath{stroke,fill}%
\end{pgfscope}%
\begin{pgfscope}%
\pgfpathrectangle{\pgfqpoint{0.556847in}{0.516222in}}{\pgfqpoint{1.722590in}{1.783528in}} %
\pgfusepath{clip}%
\pgfsetbuttcap%
\pgfsetroundjoin%
\definecolor{currentfill}{rgb}{0.298039,0.447059,0.690196}%
\pgfsetfillcolor{currentfill}%
\pgfsetlinewidth{0.240900pt}%
\definecolor{currentstroke}{rgb}{1.000000,1.000000,1.000000}%
\pgfsetstrokecolor{currentstroke}%
\pgfsetdash{}{0pt}%
\pgfpathmoveto{\pgfqpoint{1.787269in}{1.300493in}}%
\pgfpathcurveto{\pgfqpoint{1.795505in}{1.300493in}}{\pgfqpoint{1.803405in}{1.303765in}}{\pgfqpoint{1.809229in}{1.309589in}}%
\pgfpathcurveto{\pgfqpoint{1.815053in}{1.315413in}}{\pgfqpoint{1.818325in}{1.323313in}}{\pgfqpoint{1.818325in}{1.331549in}}%
\pgfpathcurveto{\pgfqpoint{1.818325in}{1.339785in}}{\pgfqpoint{1.815053in}{1.347686in}}{\pgfqpoint{1.809229in}{1.353509in}}%
\pgfpathcurveto{\pgfqpoint{1.803405in}{1.359333in}}{\pgfqpoint{1.795505in}{1.362606in}}{\pgfqpoint{1.787269in}{1.362606in}}%
\pgfpathcurveto{\pgfqpoint{1.779033in}{1.362606in}}{\pgfqpoint{1.771133in}{1.359333in}}{\pgfqpoint{1.765309in}{1.353509in}}%
\pgfpathcurveto{\pgfqpoint{1.759485in}{1.347686in}}{\pgfqpoint{1.756212in}{1.339785in}}{\pgfqpoint{1.756212in}{1.331549in}}%
\pgfpathcurveto{\pgfqpoint{1.756212in}{1.323313in}}{\pgfqpoint{1.759485in}{1.315413in}}{\pgfqpoint{1.765309in}{1.309589in}}%
\pgfpathcurveto{\pgfqpoint{1.771133in}{1.303765in}}{\pgfqpoint{1.779033in}{1.300493in}}{\pgfqpoint{1.787269in}{1.300493in}}%
\pgfpathclose%
\pgfusepath{stroke,fill}%
\end{pgfscope}%
\begin{pgfscope}%
\pgfpathrectangle{\pgfqpoint{0.556847in}{0.516222in}}{\pgfqpoint{1.722590in}{1.783528in}} %
\pgfusepath{clip}%
\pgfsetbuttcap%
\pgfsetroundjoin%
\definecolor{currentfill}{rgb}{0.298039,0.447059,0.690196}%
\pgfsetfillcolor{currentfill}%
\pgfsetlinewidth{0.240900pt}%
\definecolor{currentstroke}{rgb}{1.000000,1.000000,1.000000}%
\pgfsetstrokecolor{currentstroke}%
\pgfsetdash{}{0pt}%
\pgfpathmoveto{\pgfqpoint{2.033353in}{1.708156in}}%
\pgfpathcurveto{\pgfqpoint{2.041589in}{1.708156in}}{\pgfqpoint{2.049490in}{1.711429in}}{\pgfqpoint{2.055313in}{1.717252in}}%
\pgfpathcurveto{\pgfqpoint{2.061137in}{1.723076in}}{\pgfqpoint{2.064410in}{1.730976in}}{\pgfqpoint{2.064410in}{1.739213in}}%
\pgfpathcurveto{\pgfqpoint{2.064410in}{1.747449in}}{\pgfqpoint{2.061137in}{1.755349in}}{\pgfqpoint{2.055313in}{1.761173in}}%
\pgfpathcurveto{\pgfqpoint{2.049490in}{1.766997in}}{\pgfqpoint{2.041589in}{1.770269in}}{\pgfqpoint{2.033353in}{1.770269in}}%
\pgfpathcurveto{\pgfqpoint{2.025117in}{1.770269in}}{\pgfqpoint{2.017217in}{1.766997in}}{\pgfqpoint{2.011393in}{1.761173in}}%
\pgfpathcurveto{\pgfqpoint{2.005569in}{1.755349in}}{\pgfqpoint{2.002297in}{1.747449in}}{\pgfqpoint{2.002297in}{1.739213in}}%
\pgfpathcurveto{\pgfqpoint{2.002297in}{1.730976in}}{\pgfqpoint{2.005569in}{1.723076in}}{\pgfqpoint{2.011393in}{1.717252in}}%
\pgfpathcurveto{\pgfqpoint{2.017217in}{1.711429in}}{\pgfqpoint{2.025117in}{1.708156in}}{\pgfqpoint{2.033353in}{1.708156in}}%
\pgfpathclose%
\pgfusepath{stroke,fill}%
\end{pgfscope}%
\begin{pgfscope}%
\pgfpathrectangle{\pgfqpoint{0.556847in}{0.516222in}}{\pgfqpoint{1.722590in}{1.783528in}} %
\pgfusepath{clip}%
\pgfsetbuttcap%
\pgfsetroundjoin%
\definecolor{currentfill}{rgb}{0.298039,0.447059,0.690196}%
\pgfsetfillcolor{currentfill}%
\pgfsetlinewidth{0.240900pt}%
\definecolor{currentstroke}{rgb}{1.000000,1.000000,1.000000}%
\pgfsetstrokecolor{currentstroke}%
\pgfsetdash{}{0pt}%
\pgfpathmoveto{\pgfqpoint{1.491968in}{1.198577in}}%
\pgfpathcurveto{\pgfqpoint{1.500204in}{1.198577in}}{\pgfqpoint{1.508104in}{1.201849in}}{\pgfqpoint{1.513928in}{1.207673in}}%
\pgfpathcurveto{\pgfqpoint{1.519752in}{1.213497in}}{\pgfqpoint{1.523024in}{1.221397in}}{\pgfqpoint{1.523024in}{1.229633in}}%
\pgfpathcurveto{\pgfqpoint{1.523024in}{1.237870in}}{\pgfqpoint{1.519752in}{1.245770in}}{\pgfqpoint{1.513928in}{1.251594in}}%
\pgfpathcurveto{\pgfqpoint{1.508104in}{1.257418in}}{\pgfqpoint{1.500204in}{1.260690in}}{\pgfqpoint{1.491968in}{1.260690in}}%
\pgfpathcurveto{\pgfqpoint{1.483731in}{1.260690in}}{\pgfqpoint{1.475831in}{1.257418in}}{\pgfqpoint{1.470007in}{1.251594in}}%
\pgfpathcurveto{\pgfqpoint{1.464183in}{1.245770in}}{\pgfqpoint{1.460911in}{1.237870in}}{\pgfqpoint{1.460911in}{1.229633in}}%
\pgfpathcurveto{\pgfqpoint{1.460911in}{1.221397in}}{\pgfqpoint{1.464183in}{1.213497in}}{\pgfqpoint{1.470007in}{1.207673in}}%
\pgfpathcurveto{\pgfqpoint{1.475831in}{1.201849in}}{\pgfqpoint{1.483731in}{1.198577in}}{\pgfqpoint{1.491968in}{1.198577in}}%
\pgfpathclose%
\pgfusepath{stroke,fill}%
\end{pgfscope}%
\begin{pgfscope}%
\pgfpathrectangle{\pgfqpoint{0.556847in}{0.516222in}}{\pgfqpoint{1.722590in}{1.783528in}} %
\pgfusepath{clip}%
\pgfsetbuttcap%
\pgfsetroundjoin%
\definecolor{currentfill}{rgb}{0.298039,0.447059,0.690196}%
\pgfsetfillcolor{currentfill}%
\pgfsetlinewidth{0.240900pt}%
\definecolor{currentstroke}{rgb}{1.000000,1.000000,1.000000}%
\pgfsetstrokecolor{currentstroke}%
\pgfsetdash{}{0pt}%
\pgfpathmoveto{\pgfqpoint{2.181004in}{1.759114in}}%
\pgfpathcurveto{\pgfqpoint{2.189240in}{1.759114in}}{\pgfqpoint{2.197140in}{1.762386in}}{\pgfqpoint{2.202964in}{1.768210in}}%
\pgfpathcurveto{\pgfqpoint{2.208788in}{1.774034in}}{\pgfqpoint{2.212060in}{1.781934in}}{\pgfqpoint{2.212060in}{1.790171in}}%
\pgfpathcurveto{\pgfqpoint{2.212060in}{1.798407in}}{\pgfqpoint{2.208788in}{1.806307in}}{\pgfqpoint{2.202964in}{1.812131in}}%
\pgfpathcurveto{\pgfqpoint{2.197140in}{1.817955in}}{\pgfqpoint{2.189240in}{1.821227in}}{\pgfqpoint{2.181004in}{1.821227in}}%
\pgfpathcurveto{\pgfqpoint{2.172767in}{1.821227in}}{\pgfqpoint{2.164867in}{1.817955in}}{\pgfqpoint{2.159044in}{1.812131in}}%
\pgfpathcurveto{\pgfqpoint{2.153220in}{1.806307in}}{\pgfqpoint{2.149947in}{1.798407in}}{\pgfqpoint{2.149947in}{1.790171in}}%
\pgfpathcurveto{\pgfqpoint{2.149947in}{1.781934in}}{\pgfqpoint{2.153220in}{1.774034in}}{\pgfqpoint{2.159044in}{1.768210in}}%
\pgfpathcurveto{\pgfqpoint{2.164867in}{1.762386in}}{\pgfqpoint{2.172767in}{1.759114in}}{\pgfqpoint{2.181004in}{1.759114in}}%
\pgfpathclose%
\pgfusepath{stroke,fill}%
\end{pgfscope}%
\begin{pgfscope}%
\pgfpathrectangle{\pgfqpoint{0.556847in}{0.516222in}}{\pgfqpoint{1.722590in}{1.783528in}} %
\pgfusepath{clip}%
\pgfsetbuttcap%
\pgfsetroundjoin%
\definecolor{currentfill}{rgb}{0.298039,0.447059,0.690196}%
\pgfsetfillcolor{currentfill}%
\pgfsetlinewidth{0.240900pt}%
\definecolor{currentstroke}{rgb}{1.000000,1.000000,1.000000}%
\pgfsetstrokecolor{currentstroke}%
\pgfsetdash{}{0pt}%
\pgfpathmoveto{\pgfqpoint{1.491968in}{1.198577in}}%
\pgfpathcurveto{\pgfqpoint{1.500204in}{1.198577in}}{\pgfqpoint{1.508104in}{1.201849in}}{\pgfqpoint{1.513928in}{1.207673in}}%
\pgfpathcurveto{\pgfqpoint{1.519752in}{1.213497in}}{\pgfqpoint{1.523024in}{1.221397in}}{\pgfqpoint{1.523024in}{1.229633in}}%
\pgfpathcurveto{\pgfqpoint{1.523024in}{1.237870in}}{\pgfqpoint{1.519752in}{1.245770in}}{\pgfqpoint{1.513928in}{1.251594in}}%
\pgfpathcurveto{\pgfqpoint{1.508104in}{1.257418in}}{\pgfqpoint{1.500204in}{1.260690in}}{\pgfqpoint{1.491968in}{1.260690in}}%
\pgfpathcurveto{\pgfqpoint{1.483731in}{1.260690in}}{\pgfqpoint{1.475831in}{1.257418in}}{\pgfqpoint{1.470007in}{1.251594in}}%
\pgfpathcurveto{\pgfqpoint{1.464183in}{1.245770in}}{\pgfqpoint{1.460911in}{1.237870in}}{\pgfqpoint{1.460911in}{1.229633in}}%
\pgfpathcurveto{\pgfqpoint{1.460911in}{1.221397in}}{\pgfqpoint{1.464183in}{1.213497in}}{\pgfqpoint{1.470007in}{1.207673in}}%
\pgfpathcurveto{\pgfqpoint{1.475831in}{1.201849in}}{\pgfqpoint{1.483731in}{1.198577in}}{\pgfqpoint{1.491968in}{1.198577in}}%
\pgfpathclose%
\pgfusepath{stroke,fill}%
\end{pgfscope}%
\begin{pgfscope}%
\pgfpathrectangle{\pgfqpoint{0.556847in}{0.516222in}}{\pgfqpoint{1.722590in}{1.783528in}} %
\pgfusepath{clip}%
\pgfsetbuttcap%
\pgfsetroundjoin%
\definecolor{currentfill}{rgb}{0.298039,0.447059,0.690196}%
\pgfsetfillcolor{currentfill}%
\pgfsetlinewidth{0.240900pt}%
\definecolor{currentstroke}{rgb}{1.000000,1.000000,1.000000}%
\pgfsetstrokecolor{currentstroke}%
\pgfsetdash{}{0pt}%
\pgfpathmoveto{\pgfqpoint{2.181004in}{1.962946in}}%
\pgfpathcurveto{\pgfqpoint{2.189240in}{1.962946in}}{\pgfqpoint{2.197140in}{1.966218in}}{\pgfqpoint{2.202964in}{1.972042in}}%
\pgfpathcurveto{\pgfqpoint{2.208788in}{1.977866in}}{\pgfqpoint{2.212060in}{1.985766in}}{\pgfqpoint{2.212060in}{1.994002in}}%
\pgfpathcurveto{\pgfqpoint{2.212060in}{2.002239in}}{\pgfqpoint{2.208788in}{2.010139in}}{\pgfqpoint{2.202964in}{2.015963in}}%
\pgfpathcurveto{\pgfqpoint{2.197140in}{2.021787in}}{\pgfqpoint{2.189240in}{2.025059in}}{\pgfqpoint{2.181004in}{2.025059in}}%
\pgfpathcurveto{\pgfqpoint{2.172767in}{2.025059in}}{\pgfqpoint{2.164867in}{2.021787in}}{\pgfqpoint{2.159044in}{2.015963in}}%
\pgfpathcurveto{\pgfqpoint{2.153220in}{2.010139in}}{\pgfqpoint{2.149947in}{2.002239in}}{\pgfqpoint{2.149947in}{1.994002in}}%
\pgfpathcurveto{\pgfqpoint{2.149947in}{1.985766in}}{\pgfqpoint{2.153220in}{1.977866in}}{\pgfqpoint{2.159044in}{1.972042in}}%
\pgfpathcurveto{\pgfqpoint{2.164867in}{1.966218in}}{\pgfqpoint{2.172767in}{1.962946in}}{\pgfqpoint{2.181004in}{1.962946in}}%
\pgfpathclose%
\pgfusepath{stroke,fill}%
\end{pgfscope}%
\begin{pgfscope}%
\pgfpathrectangle{\pgfqpoint{0.556847in}{0.516222in}}{\pgfqpoint{1.722590in}{1.783528in}} %
\pgfusepath{clip}%
\pgfsetbuttcap%
\pgfsetroundjoin%
\definecolor{currentfill}{rgb}{0.298039,0.447059,0.690196}%
\pgfsetfillcolor{currentfill}%
\pgfsetlinewidth{0.240900pt}%
\definecolor{currentstroke}{rgb}{1.000000,1.000000,1.000000}%
\pgfsetstrokecolor{currentstroke}%
\pgfsetdash{}{0pt}%
\pgfpathmoveto{\pgfqpoint{1.098233in}{0.739955in}}%
\pgfpathcurveto{\pgfqpoint{1.106469in}{0.739955in}}{\pgfqpoint{1.114369in}{0.743228in}}{\pgfqpoint{1.120193in}{0.749052in}}%
\pgfpathcurveto{\pgfqpoint{1.126017in}{0.754876in}}{\pgfqpoint{1.129289in}{0.762776in}}{\pgfqpoint{1.129289in}{0.771012in}}%
\pgfpathcurveto{\pgfqpoint{1.129289in}{0.779248in}}{\pgfqpoint{1.126017in}{0.787148in}}{\pgfqpoint{1.120193in}{0.792972in}}%
\pgfpathcurveto{\pgfqpoint{1.114369in}{0.798796in}}{\pgfqpoint{1.106469in}{0.802068in}}{\pgfqpoint{1.098233in}{0.802068in}}%
\pgfpathcurveto{\pgfqpoint{1.089996in}{0.802068in}}{\pgfqpoint{1.082096in}{0.798796in}}{\pgfqpoint{1.076272in}{0.792972in}}%
\pgfpathcurveto{\pgfqpoint{1.070449in}{0.787148in}}{\pgfqpoint{1.067176in}{0.779248in}}{\pgfqpoint{1.067176in}{0.771012in}}%
\pgfpathcurveto{\pgfqpoint{1.067176in}{0.762776in}}{\pgfqpoint{1.070449in}{0.754876in}}{\pgfqpoint{1.076272in}{0.749052in}}%
\pgfpathcurveto{\pgfqpoint{1.082096in}{0.743228in}}{\pgfqpoint{1.089996in}{0.739955in}}{\pgfqpoint{1.098233in}{0.739955in}}%
\pgfpathclose%
\pgfusepath{stroke,fill}%
\end{pgfscope}%
\begin{pgfscope}%
\pgfpathrectangle{\pgfqpoint{0.556847in}{0.516222in}}{\pgfqpoint{1.722590in}{1.783528in}} %
\pgfusepath{clip}%
\pgfsetbuttcap%
\pgfsetroundjoin%
\definecolor{currentfill}{rgb}{0.298039,0.447059,0.690196}%
\pgfsetfillcolor{currentfill}%
\pgfsetlinewidth{0.240900pt}%
\definecolor{currentstroke}{rgb}{1.000000,1.000000,1.000000}%
\pgfsetstrokecolor{currentstroke}%
\pgfsetdash{}{0pt}%
\pgfpathmoveto{\pgfqpoint{1.541185in}{1.249535in}}%
\pgfpathcurveto{\pgfqpoint{1.549421in}{1.249535in}}{\pgfqpoint{1.557321in}{1.252807in}}{\pgfqpoint{1.563145in}{1.258631in}}%
\pgfpathcurveto{\pgfqpoint{1.568969in}{1.264455in}}{\pgfqpoint{1.572241in}{1.272355in}}{\pgfqpoint{1.572241in}{1.280591in}}%
\pgfpathcurveto{\pgfqpoint{1.572241in}{1.288828in}}{\pgfqpoint{1.568969in}{1.296728in}}{\pgfqpoint{1.563145in}{1.302552in}}%
\pgfpathcurveto{\pgfqpoint{1.557321in}{1.308375in}}{\pgfqpoint{1.549421in}{1.311648in}}{\pgfqpoint{1.541185in}{1.311648in}}%
\pgfpathcurveto{\pgfqpoint{1.532948in}{1.311648in}}{\pgfqpoint{1.525048in}{1.308375in}}{\pgfqpoint{1.519224in}{1.302552in}}%
\pgfpathcurveto{\pgfqpoint{1.513400in}{1.296728in}}{\pgfqpoint{1.510128in}{1.288828in}}{\pgfqpoint{1.510128in}{1.280591in}}%
\pgfpathcurveto{\pgfqpoint{1.510128in}{1.272355in}}{\pgfqpoint{1.513400in}{1.264455in}}{\pgfqpoint{1.519224in}{1.258631in}}%
\pgfpathcurveto{\pgfqpoint{1.525048in}{1.252807in}}{\pgfqpoint{1.532948in}{1.249535in}}{\pgfqpoint{1.541185in}{1.249535in}}%
\pgfpathclose%
\pgfusepath{stroke,fill}%
\end{pgfscope}%
\begin{pgfscope}%
\pgfpathrectangle{\pgfqpoint{0.556847in}{0.516222in}}{\pgfqpoint{1.722590in}{1.783528in}} %
\pgfusepath{clip}%
\pgfsetbuttcap%
\pgfsetroundjoin%
\definecolor{currentfill}{rgb}{0.298039,0.447059,0.690196}%
\pgfsetfillcolor{currentfill}%
\pgfsetlinewidth{0.240900pt}%
\definecolor{currentstroke}{rgb}{1.000000,1.000000,1.000000}%
\pgfsetstrokecolor{currentstroke}%
\pgfsetdash{}{0pt}%
\pgfpathmoveto{\pgfqpoint{1.393534in}{1.249535in}}%
\pgfpathcurveto{\pgfqpoint{1.401770in}{1.249535in}}{\pgfqpoint{1.409670in}{1.252807in}}{\pgfqpoint{1.415494in}{1.258631in}}%
\pgfpathcurveto{\pgfqpoint{1.421318in}{1.264455in}}{\pgfqpoint{1.424590in}{1.272355in}}{\pgfqpoint{1.424590in}{1.280591in}}%
\pgfpathcurveto{\pgfqpoint{1.424590in}{1.288828in}}{\pgfqpoint{1.421318in}{1.296728in}}{\pgfqpoint{1.415494in}{1.302552in}}%
\pgfpathcurveto{\pgfqpoint{1.409670in}{1.308375in}}{\pgfqpoint{1.401770in}{1.311648in}}{\pgfqpoint{1.393534in}{1.311648in}}%
\pgfpathcurveto{\pgfqpoint{1.385298in}{1.311648in}}{\pgfqpoint{1.377398in}{1.308375in}}{\pgfqpoint{1.371574in}{1.302552in}}%
\pgfpathcurveto{\pgfqpoint{1.365750in}{1.296728in}}{\pgfqpoint{1.362477in}{1.288828in}}{\pgfqpoint{1.362477in}{1.280591in}}%
\pgfpathcurveto{\pgfqpoint{1.362477in}{1.272355in}}{\pgfqpoint{1.365750in}{1.264455in}}{\pgfqpoint{1.371574in}{1.258631in}}%
\pgfpathcurveto{\pgfqpoint{1.377398in}{1.252807in}}{\pgfqpoint{1.385298in}{1.249535in}}{\pgfqpoint{1.393534in}{1.249535in}}%
\pgfpathclose%
\pgfusepath{stroke,fill}%
\end{pgfscope}%
\begin{pgfscope}%
\pgfpathrectangle{\pgfqpoint{0.556847in}{0.516222in}}{\pgfqpoint{1.722590in}{1.783528in}} %
\pgfusepath{clip}%
\pgfsetbuttcap%
\pgfsetroundjoin%
\definecolor{currentfill}{rgb}{0.298039,0.447059,0.690196}%
\pgfsetfillcolor{currentfill}%
\pgfsetlinewidth{0.240900pt}%
\definecolor{currentstroke}{rgb}{1.000000,1.000000,1.000000}%
\pgfsetstrokecolor{currentstroke}%
\pgfsetdash{}{0pt}%
\pgfpathmoveto{\pgfqpoint{1.541185in}{1.147619in}}%
\pgfpathcurveto{\pgfqpoint{1.549421in}{1.147619in}}{\pgfqpoint{1.557321in}{1.150891in}}{\pgfqpoint{1.563145in}{1.156715in}}%
\pgfpathcurveto{\pgfqpoint{1.568969in}{1.162539in}}{\pgfqpoint{1.572241in}{1.170439in}}{\pgfqpoint{1.572241in}{1.178675in}}%
\pgfpathcurveto{\pgfqpoint{1.572241in}{1.186912in}}{\pgfqpoint{1.568969in}{1.194812in}}{\pgfqpoint{1.563145in}{1.200636in}}%
\pgfpathcurveto{\pgfqpoint{1.557321in}{1.206460in}}{\pgfqpoint{1.549421in}{1.209732in}}{\pgfqpoint{1.541185in}{1.209732in}}%
\pgfpathcurveto{\pgfqpoint{1.532948in}{1.209732in}}{\pgfqpoint{1.525048in}{1.206460in}}{\pgfqpoint{1.519224in}{1.200636in}}%
\pgfpathcurveto{\pgfqpoint{1.513400in}{1.194812in}}{\pgfqpoint{1.510128in}{1.186912in}}{\pgfqpoint{1.510128in}{1.178675in}}%
\pgfpathcurveto{\pgfqpoint{1.510128in}{1.170439in}}{\pgfqpoint{1.513400in}{1.162539in}}{\pgfqpoint{1.519224in}{1.156715in}}%
\pgfpathcurveto{\pgfqpoint{1.525048in}{1.150891in}}{\pgfqpoint{1.532948in}{1.147619in}}{\pgfqpoint{1.541185in}{1.147619in}}%
\pgfpathclose%
\pgfusepath{stroke,fill}%
\end{pgfscope}%
\begin{pgfscope}%
\pgfpathrectangle{\pgfqpoint{0.556847in}{0.516222in}}{\pgfqpoint{1.722590in}{1.783528in}} %
\pgfusepath{clip}%
\pgfsetbuttcap%
\pgfsetroundjoin%
\definecolor{currentfill}{rgb}{0.298039,0.447059,0.690196}%
\pgfsetfillcolor{currentfill}%
\pgfsetlinewidth{0.240900pt}%
\definecolor{currentstroke}{rgb}{1.000000,1.000000,1.000000}%
\pgfsetstrokecolor{currentstroke}%
\pgfsetdash{}{0pt}%
\pgfpathmoveto{\pgfqpoint{1.639618in}{1.198577in}}%
\pgfpathcurveto{\pgfqpoint{1.647855in}{1.198577in}}{\pgfqpoint{1.655755in}{1.201849in}}{\pgfqpoint{1.661579in}{1.207673in}}%
\pgfpathcurveto{\pgfqpoint{1.667402in}{1.213497in}}{\pgfqpoint{1.670675in}{1.221397in}}{\pgfqpoint{1.670675in}{1.229633in}}%
\pgfpathcurveto{\pgfqpoint{1.670675in}{1.237870in}}{\pgfqpoint{1.667402in}{1.245770in}}{\pgfqpoint{1.661579in}{1.251594in}}%
\pgfpathcurveto{\pgfqpoint{1.655755in}{1.257418in}}{\pgfqpoint{1.647855in}{1.260690in}}{\pgfqpoint{1.639618in}{1.260690in}}%
\pgfpathcurveto{\pgfqpoint{1.631382in}{1.260690in}}{\pgfqpoint{1.623482in}{1.257418in}}{\pgfqpoint{1.617658in}{1.251594in}}%
\pgfpathcurveto{\pgfqpoint{1.611834in}{1.245770in}}{\pgfqpoint{1.608562in}{1.237870in}}{\pgfqpoint{1.608562in}{1.229633in}}%
\pgfpathcurveto{\pgfqpoint{1.608562in}{1.221397in}}{\pgfqpoint{1.611834in}{1.213497in}}{\pgfqpoint{1.617658in}{1.207673in}}%
\pgfpathcurveto{\pgfqpoint{1.623482in}{1.201849in}}{\pgfqpoint{1.631382in}{1.198577in}}{\pgfqpoint{1.639618in}{1.198577in}}%
\pgfpathclose%
\pgfusepath{stroke,fill}%
\end{pgfscope}%
\begin{pgfscope}%
\pgfpathrectangle{\pgfqpoint{0.556847in}{0.516222in}}{\pgfqpoint{1.722590in}{1.783528in}} %
\pgfusepath{clip}%
\pgfsetbuttcap%
\pgfsetroundjoin%
\definecolor{currentfill}{rgb}{0.298039,0.447059,0.690196}%
\pgfsetfillcolor{currentfill}%
\pgfsetlinewidth{0.240900pt}%
\definecolor{currentstroke}{rgb}{1.000000,1.000000,1.000000}%
\pgfsetstrokecolor{currentstroke}%
\pgfsetdash{}{0pt}%
\pgfpathmoveto{\pgfqpoint{0.999799in}{0.638040in}}%
\pgfpathcurveto{\pgfqpoint{1.008035in}{0.638040in}}{\pgfqpoint{1.015935in}{0.641312in}}{\pgfqpoint{1.021759in}{0.647136in}}%
\pgfpathcurveto{\pgfqpoint{1.027583in}{0.652960in}}{\pgfqpoint{1.030856in}{0.660860in}}{\pgfqpoint{1.030856in}{0.669096in}}%
\pgfpathcurveto{\pgfqpoint{1.030856in}{0.677332in}}{\pgfqpoint{1.027583in}{0.685232in}}{\pgfqpoint{1.021759in}{0.691056in}}%
\pgfpathcurveto{\pgfqpoint{1.015935in}{0.696880in}}{\pgfqpoint{1.008035in}{0.700153in}}{\pgfqpoint{0.999799in}{0.700153in}}%
\pgfpathcurveto{\pgfqpoint{0.991563in}{0.700153in}}{\pgfqpoint{0.983663in}{0.696880in}}{\pgfqpoint{0.977839in}{0.691056in}}%
\pgfpathcurveto{\pgfqpoint{0.972015in}{0.685232in}}{\pgfqpoint{0.968743in}{0.677332in}}{\pgfqpoint{0.968743in}{0.669096in}}%
\pgfpathcurveto{\pgfqpoint{0.968743in}{0.660860in}}{\pgfqpoint{0.972015in}{0.652960in}}{\pgfqpoint{0.977839in}{0.647136in}}%
\pgfpathcurveto{\pgfqpoint{0.983663in}{0.641312in}}{\pgfqpoint{0.991563in}{0.638040in}}{\pgfqpoint{0.999799in}{0.638040in}}%
\pgfpathclose%
\pgfusepath{stroke,fill}%
\end{pgfscope}%
\begin{pgfscope}%
\pgfpathrectangle{\pgfqpoint{0.556847in}{0.516222in}}{\pgfqpoint{1.722590in}{1.783528in}} %
\pgfusepath{clip}%
\pgfsetbuttcap%
\pgfsetroundjoin%
\definecolor{currentfill}{rgb}{0.298039,0.447059,0.690196}%
\pgfsetfillcolor{currentfill}%
\pgfsetlinewidth{0.240900pt}%
\definecolor{currentstroke}{rgb}{1.000000,1.000000,1.000000}%
\pgfsetstrokecolor{currentstroke}%
\pgfsetdash{}{0pt}%
\pgfpathmoveto{\pgfqpoint{1.885703in}{1.504324in}}%
\pgfpathcurveto{\pgfqpoint{1.893939in}{1.504324in}}{\pgfqpoint{1.901839in}{1.507597in}}{\pgfqpoint{1.907663in}{1.513421in}}%
\pgfpathcurveto{\pgfqpoint{1.913487in}{1.519245in}}{\pgfqpoint{1.916759in}{1.527145in}}{\pgfqpoint{1.916759in}{1.535381in}}%
\pgfpathcurveto{\pgfqpoint{1.916759in}{1.543617in}}{\pgfqpoint{1.913487in}{1.551517in}}{\pgfqpoint{1.907663in}{1.557341in}}%
\pgfpathcurveto{\pgfqpoint{1.901839in}{1.563165in}}{\pgfqpoint{1.893939in}{1.566437in}}{\pgfqpoint{1.885703in}{1.566437in}}%
\pgfpathcurveto{\pgfqpoint{1.877466in}{1.566437in}}{\pgfqpoint{1.869566in}{1.563165in}}{\pgfqpoint{1.863742in}{1.557341in}}%
\pgfpathcurveto{\pgfqpoint{1.857918in}{1.551517in}}{\pgfqpoint{1.854646in}{1.543617in}}{\pgfqpoint{1.854646in}{1.535381in}}%
\pgfpathcurveto{\pgfqpoint{1.854646in}{1.527145in}}{\pgfqpoint{1.857918in}{1.519245in}}{\pgfqpoint{1.863742in}{1.513421in}}%
\pgfpathcurveto{\pgfqpoint{1.869566in}{1.507597in}}{\pgfqpoint{1.877466in}{1.504324in}}{\pgfqpoint{1.885703in}{1.504324in}}%
\pgfpathclose%
\pgfusepath{stroke,fill}%
\end{pgfscope}%
\begin{pgfscope}%
\pgfpathrectangle{\pgfqpoint{0.556847in}{0.516222in}}{\pgfqpoint{1.722590in}{1.783528in}} %
\pgfusepath{clip}%
\pgfsetbuttcap%
\pgfsetroundjoin%
\definecolor{currentfill}{rgb}{0.298039,0.447059,0.690196}%
\pgfsetfillcolor{currentfill}%
\pgfsetlinewidth{0.240900pt}%
\definecolor{currentstroke}{rgb}{1.000000,1.000000,1.000000}%
\pgfsetstrokecolor{currentstroke}%
\pgfsetdash{}{0pt}%
\pgfpathmoveto{\pgfqpoint{1.787269in}{1.453367in}}%
\pgfpathcurveto{\pgfqpoint{1.795505in}{1.453367in}}{\pgfqpoint{1.803405in}{1.456639in}}{\pgfqpoint{1.809229in}{1.462463in}}%
\pgfpathcurveto{\pgfqpoint{1.815053in}{1.468287in}}{\pgfqpoint{1.818325in}{1.476187in}}{\pgfqpoint{1.818325in}{1.484423in}}%
\pgfpathcurveto{\pgfqpoint{1.818325in}{1.492659in}}{\pgfqpoint{1.815053in}{1.500559in}}{\pgfqpoint{1.809229in}{1.506383in}}%
\pgfpathcurveto{\pgfqpoint{1.803405in}{1.512207in}}{\pgfqpoint{1.795505in}{1.515480in}}{\pgfqpoint{1.787269in}{1.515480in}}%
\pgfpathcurveto{\pgfqpoint{1.779033in}{1.515480in}}{\pgfqpoint{1.771133in}{1.512207in}}{\pgfqpoint{1.765309in}{1.506383in}}%
\pgfpathcurveto{\pgfqpoint{1.759485in}{1.500559in}}{\pgfqpoint{1.756212in}{1.492659in}}{\pgfqpoint{1.756212in}{1.484423in}}%
\pgfpathcurveto{\pgfqpoint{1.756212in}{1.476187in}}{\pgfqpoint{1.759485in}{1.468287in}}{\pgfqpoint{1.765309in}{1.462463in}}%
\pgfpathcurveto{\pgfqpoint{1.771133in}{1.456639in}}{\pgfqpoint{1.779033in}{1.453367in}}{\pgfqpoint{1.787269in}{1.453367in}}%
\pgfpathclose%
\pgfusepath{stroke,fill}%
\end{pgfscope}%
\begin{pgfscope}%
\pgfpathrectangle{\pgfqpoint{0.556847in}{0.516222in}}{\pgfqpoint{1.722590in}{1.783528in}} %
\pgfusepath{clip}%
\pgfsetbuttcap%
\pgfsetroundjoin%
\definecolor{currentfill}{rgb}{0.298039,0.447059,0.690196}%
\pgfsetfillcolor{currentfill}%
\pgfsetlinewidth{0.240900pt}%
\definecolor{currentstroke}{rgb}{1.000000,1.000000,1.000000}%
\pgfsetstrokecolor{currentstroke}%
\pgfsetdash{}{0pt}%
\pgfpathmoveto{\pgfqpoint{1.393534in}{1.045703in}}%
\pgfpathcurveto{\pgfqpoint{1.401770in}{1.045703in}}{\pgfqpoint{1.409670in}{1.048975in}}{\pgfqpoint{1.415494in}{1.054799in}}%
\pgfpathcurveto{\pgfqpoint{1.421318in}{1.060623in}}{\pgfqpoint{1.424590in}{1.068523in}}{\pgfqpoint{1.424590in}{1.076760in}}%
\pgfpathcurveto{\pgfqpoint{1.424590in}{1.084996in}}{\pgfqpoint{1.421318in}{1.092896in}}{\pgfqpoint{1.415494in}{1.098720in}}%
\pgfpathcurveto{\pgfqpoint{1.409670in}{1.104544in}}{\pgfqpoint{1.401770in}{1.107816in}}{\pgfqpoint{1.393534in}{1.107816in}}%
\pgfpathcurveto{\pgfqpoint{1.385298in}{1.107816in}}{\pgfqpoint{1.377398in}{1.104544in}}{\pgfqpoint{1.371574in}{1.098720in}}%
\pgfpathcurveto{\pgfqpoint{1.365750in}{1.092896in}}{\pgfqpoint{1.362477in}{1.084996in}}{\pgfqpoint{1.362477in}{1.076760in}}%
\pgfpathcurveto{\pgfqpoint{1.362477in}{1.068523in}}{\pgfqpoint{1.365750in}{1.060623in}}{\pgfqpoint{1.371574in}{1.054799in}}%
\pgfpathcurveto{\pgfqpoint{1.377398in}{1.048975in}}{\pgfqpoint{1.385298in}{1.045703in}}{\pgfqpoint{1.393534in}{1.045703in}}%
\pgfpathclose%
\pgfusepath{stroke,fill}%
\end{pgfscope}%
\begin{pgfscope}%
\pgfpathrectangle{\pgfqpoint{0.556847in}{0.516222in}}{\pgfqpoint{1.722590in}{1.783528in}} %
\pgfusepath{clip}%
\pgfsetbuttcap%
\pgfsetroundjoin%
\definecolor{currentfill}{rgb}{0.298039,0.447059,0.690196}%
\pgfsetfillcolor{currentfill}%
\pgfsetlinewidth{0.240900pt}%
\definecolor{currentstroke}{rgb}{1.000000,1.000000,1.000000}%
\pgfsetstrokecolor{currentstroke}%
\pgfsetdash{}{0pt}%
\pgfpathmoveto{\pgfqpoint{1.541185in}{1.198577in}}%
\pgfpathcurveto{\pgfqpoint{1.549421in}{1.198577in}}{\pgfqpoint{1.557321in}{1.201849in}}{\pgfqpoint{1.563145in}{1.207673in}}%
\pgfpathcurveto{\pgfqpoint{1.568969in}{1.213497in}}{\pgfqpoint{1.572241in}{1.221397in}}{\pgfqpoint{1.572241in}{1.229633in}}%
\pgfpathcurveto{\pgfqpoint{1.572241in}{1.237870in}}{\pgfqpoint{1.568969in}{1.245770in}}{\pgfqpoint{1.563145in}{1.251594in}}%
\pgfpathcurveto{\pgfqpoint{1.557321in}{1.257418in}}{\pgfqpoint{1.549421in}{1.260690in}}{\pgfqpoint{1.541185in}{1.260690in}}%
\pgfpathcurveto{\pgfqpoint{1.532948in}{1.260690in}}{\pgfqpoint{1.525048in}{1.257418in}}{\pgfqpoint{1.519224in}{1.251594in}}%
\pgfpathcurveto{\pgfqpoint{1.513400in}{1.245770in}}{\pgfqpoint{1.510128in}{1.237870in}}{\pgfqpoint{1.510128in}{1.229633in}}%
\pgfpathcurveto{\pgfqpoint{1.510128in}{1.221397in}}{\pgfqpoint{1.513400in}{1.213497in}}{\pgfqpoint{1.519224in}{1.207673in}}%
\pgfpathcurveto{\pgfqpoint{1.525048in}{1.201849in}}{\pgfqpoint{1.532948in}{1.198577in}}{\pgfqpoint{1.541185in}{1.198577in}}%
\pgfpathclose%
\pgfusepath{stroke,fill}%
\end{pgfscope}%
\begin{pgfscope}%
\pgfpathrectangle{\pgfqpoint{0.556847in}{0.516222in}}{\pgfqpoint{1.722590in}{1.783528in}} %
\pgfusepath{clip}%
\pgfsetbuttcap%
\pgfsetroundjoin%
\definecolor{currentfill}{rgb}{0.298039,0.447059,0.690196}%
\pgfsetfillcolor{currentfill}%
\pgfsetlinewidth{0.240900pt}%
\definecolor{currentstroke}{rgb}{1.000000,1.000000,1.000000}%
\pgfsetstrokecolor{currentstroke}%
\pgfsetdash{}{0pt}%
\pgfpathmoveto{\pgfqpoint{1.196666in}{0.841871in}}%
\pgfpathcurveto{\pgfqpoint{1.204903in}{0.841871in}}{\pgfqpoint{1.212803in}{0.845144in}}{\pgfqpoint{1.218627in}{0.850968in}}%
\pgfpathcurveto{\pgfqpoint{1.224451in}{0.856791in}}{\pgfqpoint{1.227723in}{0.864691in}}{\pgfqpoint{1.227723in}{0.872928in}}%
\pgfpathcurveto{\pgfqpoint{1.227723in}{0.881164in}}{\pgfqpoint{1.224451in}{0.889064in}}{\pgfqpoint{1.218627in}{0.894888in}}%
\pgfpathcurveto{\pgfqpoint{1.212803in}{0.900712in}}{\pgfqpoint{1.204903in}{0.903984in}}{\pgfqpoint{1.196666in}{0.903984in}}%
\pgfpathcurveto{\pgfqpoint{1.188430in}{0.903984in}}{\pgfqpoint{1.180530in}{0.900712in}}{\pgfqpoint{1.174706in}{0.894888in}}%
\pgfpathcurveto{\pgfqpoint{1.168882in}{0.889064in}}{\pgfqpoint{1.165610in}{0.881164in}}{\pgfqpoint{1.165610in}{0.872928in}}%
\pgfpathcurveto{\pgfqpoint{1.165610in}{0.864691in}}{\pgfqpoint{1.168882in}{0.856791in}}{\pgfqpoint{1.174706in}{0.850968in}}%
\pgfpathcurveto{\pgfqpoint{1.180530in}{0.845144in}}{\pgfqpoint{1.188430in}{0.841871in}}{\pgfqpoint{1.196666in}{0.841871in}}%
\pgfpathclose%
\pgfusepath{stroke,fill}%
\end{pgfscope}%
\begin{pgfscope}%
\pgfpathrectangle{\pgfqpoint{0.556847in}{0.516222in}}{\pgfqpoint{1.722590in}{1.783528in}} %
\pgfusepath{clip}%
\pgfsetbuttcap%
\pgfsetroundjoin%
\definecolor{currentfill}{rgb}{0.298039,0.447059,0.690196}%
\pgfsetfillcolor{currentfill}%
\pgfsetlinewidth{0.240900pt}%
\definecolor{currentstroke}{rgb}{1.000000,1.000000,1.000000}%
\pgfsetstrokecolor{currentstroke}%
\pgfsetdash{}{0pt}%
\pgfpathmoveto{\pgfqpoint{1.344317in}{1.147619in}}%
\pgfpathcurveto{\pgfqpoint{1.352553in}{1.147619in}}{\pgfqpoint{1.360453in}{1.150891in}}{\pgfqpoint{1.366277in}{1.156715in}}%
\pgfpathcurveto{\pgfqpoint{1.372101in}{1.162539in}}{\pgfqpoint{1.375374in}{1.170439in}}{\pgfqpoint{1.375374in}{1.178675in}}%
\pgfpathcurveto{\pgfqpoint{1.375374in}{1.186912in}}{\pgfqpoint{1.372101in}{1.194812in}}{\pgfqpoint{1.366277in}{1.200636in}}%
\pgfpathcurveto{\pgfqpoint{1.360453in}{1.206460in}}{\pgfqpoint{1.352553in}{1.209732in}}{\pgfqpoint{1.344317in}{1.209732in}}%
\pgfpathcurveto{\pgfqpoint{1.336081in}{1.209732in}}{\pgfqpoint{1.328181in}{1.206460in}}{\pgfqpoint{1.322357in}{1.200636in}}%
\pgfpathcurveto{\pgfqpoint{1.316533in}{1.194812in}}{\pgfqpoint{1.313261in}{1.186912in}}{\pgfqpoint{1.313261in}{1.178675in}}%
\pgfpathcurveto{\pgfqpoint{1.313261in}{1.170439in}}{\pgfqpoint{1.316533in}{1.162539in}}{\pgfqpoint{1.322357in}{1.156715in}}%
\pgfpathcurveto{\pgfqpoint{1.328181in}{1.150891in}}{\pgfqpoint{1.336081in}{1.147619in}}{\pgfqpoint{1.344317in}{1.147619in}}%
\pgfpathclose%
\pgfusepath{stroke,fill}%
\end{pgfscope}%
\begin{pgfscope}%
\pgfpathrectangle{\pgfqpoint{0.556847in}{0.516222in}}{\pgfqpoint{1.722590in}{1.783528in}} %
\pgfusepath{clip}%
\pgfsetbuttcap%
\pgfsetroundjoin%
\definecolor{currentfill}{rgb}{0.298039,0.447059,0.690196}%
\pgfsetfillcolor{currentfill}%
\pgfsetlinewidth{0.240900pt}%
\definecolor{currentstroke}{rgb}{1.000000,1.000000,1.000000}%
\pgfsetstrokecolor{currentstroke}%
\pgfsetdash{}{0pt}%
\pgfpathmoveto{\pgfqpoint{1.295100in}{1.453367in}}%
\pgfpathcurveto{\pgfqpoint{1.303336in}{1.453367in}}{\pgfqpoint{1.311237in}{1.456639in}}{\pgfqpoint{1.317060in}{1.462463in}}%
\pgfpathcurveto{\pgfqpoint{1.322884in}{1.468287in}}{\pgfqpoint{1.326157in}{1.476187in}}{\pgfqpoint{1.326157in}{1.484423in}}%
\pgfpathcurveto{\pgfqpoint{1.326157in}{1.492659in}}{\pgfqpoint{1.322884in}{1.500559in}}{\pgfqpoint{1.317060in}{1.506383in}}%
\pgfpathcurveto{\pgfqpoint{1.311237in}{1.512207in}}{\pgfqpoint{1.303336in}{1.515480in}}{\pgfqpoint{1.295100in}{1.515480in}}%
\pgfpathcurveto{\pgfqpoint{1.286864in}{1.515480in}}{\pgfqpoint{1.278964in}{1.512207in}}{\pgfqpoint{1.273140in}{1.506383in}}%
\pgfpathcurveto{\pgfqpoint{1.267316in}{1.500559in}}{\pgfqpoint{1.264044in}{1.492659in}}{\pgfqpoint{1.264044in}{1.484423in}}%
\pgfpathcurveto{\pgfqpoint{1.264044in}{1.476187in}}{\pgfqpoint{1.267316in}{1.468287in}}{\pgfqpoint{1.273140in}{1.462463in}}%
\pgfpathcurveto{\pgfqpoint{1.278964in}{1.456639in}}{\pgfqpoint{1.286864in}{1.453367in}}{\pgfqpoint{1.295100in}{1.453367in}}%
\pgfpathclose%
\pgfusepath{stroke,fill}%
\end{pgfscope}%
\begin{pgfscope}%
\pgfpathrectangle{\pgfqpoint{0.556847in}{0.516222in}}{\pgfqpoint{1.722590in}{1.783528in}} %
\pgfusepath{clip}%
\pgfsetbuttcap%
\pgfsetroundjoin%
\definecolor{currentfill}{rgb}{0.298039,0.447059,0.690196}%
\pgfsetfillcolor{currentfill}%
\pgfsetlinewidth{0.240900pt}%
\definecolor{currentstroke}{rgb}{1.000000,1.000000,1.000000}%
\pgfsetstrokecolor{currentstroke}%
\pgfsetdash{}{0pt}%
\pgfpathmoveto{\pgfqpoint{1.836486in}{1.606240in}}%
\pgfpathcurveto{\pgfqpoint{1.844722in}{1.606240in}}{\pgfqpoint{1.852622in}{1.609513in}}{\pgfqpoint{1.858446in}{1.615337in}}%
\pgfpathcurveto{\pgfqpoint{1.864270in}{1.621160in}}{\pgfqpoint{1.867542in}{1.629061in}}{\pgfqpoint{1.867542in}{1.637297in}}%
\pgfpathcurveto{\pgfqpoint{1.867542in}{1.645533in}}{\pgfqpoint{1.864270in}{1.653433in}}{\pgfqpoint{1.858446in}{1.659257in}}%
\pgfpathcurveto{\pgfqpoint{1.852622in}{1.665081in}}{\pgfqpoint{1.844722in}{1.668353in}}{\pgfqpoint{1.836486in}{1.668353in}}%
\pgfpathcurveto{\pgfqpoint{1.828249in}{1.668353in}}{\pgfqpoint{1.820349in}{1.665081in}}{\pgfqpoint{1.814525in}{1.659257in}}%
\pgfpathcurveto{\pgfqpoint{1.808702in}{1.653433in}}{\pgfqpoint{1.805429in}{1.645533in}}{\pgfqpoint{1.805429in}{1.637297in}}%
\pgfpathcurveto{\pgfqpoint{1.805429in}{1.629061in}}{\pgfqpoint{1.808702in}{1.621160in}}{\pgfqpoint{1.814525in}{1.615337in}}%
\pgfpathcurveto{\pgfqpoint{1.820349in}{1.609513in}}{\pgfqpoint{1.828249in}{1.606240in}}{\pgfqpoint{1.836486in}{1.606240in}}%
\pgfpathclose%
\pgfusepath{stroke,fill}%
\end{pgfscope}%
\begin{pgfscope}%
\pgfpathrectangle{\pgfqpoint{0.556847in}{0.516222in}}{\pgfqpoint{1.722590in}{1.783528in}} %
\pgfusepath{clip}%
\pgfsetbuttcap%
\pgfsetroundjoin%
\definecolor{currentfill}{rgb}{0.298039,0.447059,0.690196}%
\pgfsetfillcolor{currentfill}%
\pgfsetlinewidth{0.240900pt}%
\definecolor{currentstroke}{rgb}{1.000000,1.000000,1.000000}%
\pgfsetstrokecolor{currentstroke}%
\pgfsetdash{}{0pt}%
\pgfpathmoveto{\pgfqpoint{1.688835in}{1.198577in}}%
\pgfpathcurveto{\pgfqpoint{1.697071in}{1.198577in}}{\pgfqpoint{1.704971in}{1.201849in}}{\pgfqpoint{1.710795in}{1.207673in}}%
\pgfpathcurveto{\pgfqpoint{1.716619in}{1.213497in}}{\pgfqpoint{1.719892in}{1.221397in}}{\pgfqpoint{1.719892in}{1.229633in}}%
\pgfpathcurveto{\pgfqpoint{1.719892in}{1.237870in}}{\pgfqpoint{1.716619in}{1.245770in}}{\pgfqpoint{1.710795in}{1.251594in}}%
\pgfpathcurveto{\pgfqpoint{1.704971in}{1.257418in}}{\pgfqpoint{1.697071in}{1.260690in}}{\pgfqpoint{1.688835in}{1.260690in}}%
\pgfpathcurveto{\pgfqpoint{1.680599in}{1.260690in}}{\pgfqpoint{1.672699in}{1.257418in}}{\pgfqpoint{1.666875in}{1.251594in}}%
\pgfpathcurveto{\pgfqpoint{1.661051in}{1.245770in}}{\pgfqpoint{1.657779in}{1.237870in}}{\pgfqpoint{1.657779in}{1.229633in}}%
\pgfpathcurveto{\pgfqpoint{1.657779in}{1.221397in}}{\pgfqpoint{1.661051in}{1.213497in}}{\pgfqpoint{1.666875in}{1.207673in}}%
\pgfpathcurveto{\pgfqpoint{1.672699in}{1.201849in}}{\pgfqpoint{1.680599in}{1.198577in}}{\pgfqpoint{1.688835in}{1.198577in}}%
\pgfpathclose%
\pgfusepath{stroke,fill}%
\end{pgfscope}%
\begin{pgfscope}%
\pgfpathrectangle{\pgfqpoint{0.556847in}{0.516222in}}{\pgfqpoint{1.722590in}{1.783528in}} %
\pgfusepath{clip}%
\pgfsetbuttcap%
\pgfsetroundjoin%
\definecolor{currentfill}{rgb}{0.298039,0.447059,0.690196}%
\pgfsetfillcolor{currentfill}%
\pgfsetlinewidth{0.240900pt}%
\definecolor{currentstroke}{rgb}{1.000000,1.000000,1.000000}%
\pgfsetstrokecolor{currentstroke}%
\pgfsetdash{}{0pt}%
\pgfpathmoveto{\pgfqpoint{1.147450in}{1.045703in}}%
\pgfpathcurveto{\pgfqpoint{1.155686in}{1.045703in}}{\pgfqpoint{1.163586in}{1.048975in}}{\pgfqpoint{1.169410in}{1.054799in}}%
\pgfpathcurveto{\pgfqpoint{1.175234in}{1.060623in}}{\pgfqpoint{1.178506in}{1.068523in}}{\pgfqpoint{1.178506in}{1.076760in}}%
\pgfpathcurveto{\pgfqpoint{1.178506in}{1.084996in}}{\pgfqpoint{1.175234in}{1.092896in}}{\pgfqpoint{1.169410in}{1.098720in}}%
\pgfpathcurveto{\pgfqpoint{1.163586in}{1.104544in}}{\pgfqpoint{1.155686in}{1.107816in}}{\pgfqpoint{1.147450in}{1.107816in}}%
\pgfpathcurveto{\pgfqpoint{1.139213in}{1.107816in}}{\pgfqpoint{1.131313in}{1.104544in}}{\pgfqpoint{1.125489in}{1.098720in}}%
\pgfpathcurveto{\pgfqpoint{1.119665in}{1.092896in}}{\pgfqpoint{1.116393in}{1.084996in}}{\pgfqpoint{1.116393in}{1.076760in}}%
\pgfpathcurveto{\pgfqpoint{1.116393in}{1.068523in}}{\pgfqpoint{1.119665in}{1.060623in}}{\pgfqpoint{1.125489in}{1.054799in}}%
\pgfpathcurveto{\pgfqpoint{1.131313in}{1.048975in}}{\pgfqpoint{1.139213in}{1.045703in}}{\pgfqpoint{1.147450in}{1.045703in}}%
\pgfpathclose%
\pgfusepath{stroke,fill}%
\end{pgfscope}%
\begin{pgfscope}%
\pgfpathrectangle{\pgfqpoint{0.556847in}{0.516222in}}{\pgfqpoint{1.722590in}{1.783528in}} %
\pgfusepath{clip}%
\pgfsetbuttcap%
\pgfsetroundjoin%
\definecolor{currentfill}{rgb}{0.298039,0.447059,0.690196}%
\pgfsetfillcolor{currentfill}%
\pgfsetlinewidth{0.240900pt}%
\definecolor{currentstroke}{rgb}{1.000000,1.000000,1.000000}%
\pgfsetstrokecolor{currentstroke}%
\pgfsetdash{}{0pt}%
\pgfpathmoveto{\pgfqpoint{1.984136in}{1.045703in}}%
\pgfpathcurveto{\pgfqpoint{1.992373in}{1.045703in}}{\pgfqpoint{2.000273in}{1.048975in}}{\pgfqpoint{2.006097in}{1.054799in}}%
\pgfpathcurveto{\pgfqpoint{2.011921in}{1.060623in}}{\pgfqpoint{2.015193in}{1.068523in}}{\pgfqpoint{2.015193in}{1.076760in}}%
\pgfpathcurveto{\pgfqpoint{2.015193in}{1.084996in}}{\pgfqpoint{2.011921in}{1.092896in}}{\pgfqpoint{2.006097in}{1.098720in}}%
\pgfpathcurveto{\pgfqpoint{2.000273in}{1.104544in}}{\pgfqpoint{1.992373in}{1.107816in}}{\pgfqpoint{1.984136in}{1.107816in}}%
\pgfpathcurveto{\pgfqpoint{1.975900in}{1.107816in}}{\pgfqpoint{1.968000in}{1.104544in}}{\pgfqpoint{1.962176in}{1.098720in}}%
\pgfpathcurveto{\pgfqpoint{1.956352in}{1.092896in}}{\pgfqpoint{1.953080in}{1.084996in}}{\pgfqpoint{1.953080in}{1.076760in}}%
\pgfpathcurveto{\pgfqpoint{1.953080in}{1.068523in}}{\pgfqpoint{1.956352in}{1.060623in}}{\pgfqpoint{1.962176in}{1.054799in}}%
\pgfpathcurveto{\pgfqpoint{1.968000in}{1.048975in}}{\pgfqpoint{1.975900in}{1.045703in}}{\pgfqpoint{1.984136in}{1.045703in}}%
\pgfpathclose%
\pgfusepath{stroke,fill}%
\end{pgfscope}%
\begin{pgfscope}%
\pgfpathrectangle{\pgfqpoint{0.556847in}{0.516222in}}{\pgfqpoint{1.722590in}{1.783528in}} %
\pgfusepath{clip}%
\pgfsetbuttcap%
\pgfsetroundjoin%
\definecolor{currentfill}{rgb}{0.298039,0.447059,0.690196}%
\pgfsetfillcolor{currentfill}%
\pgfsetlinewidth{0.240900pt}%
\definecolor{currentstroke}{rgb}{1.000000,1.000000,1.000000}%
\pgfsetstrokecolor{currentstroke}%
\pgfsetdash{}{0pt}%
\pgfpathmoveto{\pgfqpoint{2.082570in}{1.606240in}}%
\pgfpathcurveto{\pgfqpoint{2.090806in}{1.606240in}}{\pgfqpoint{2.098706in}{1.609513in}}{\pgfqpoint{2.104530in}{1.615337in}}%
\pgfpathcurveto{\pgfqpoint{2.110354in}{1.621160in}}{\pgfqpoint{2.113627in}{1.629061in}}{\pgfqpoint{2.113627in}{1.637297in}}%
\pgfpathcurveto{\pgfqpoint{2.113627in}{1.645533in}}{\pgfqpoint{2.110354in}{1.653433in}}{\pgfqpoint{2.104530in}{1.659257in}}%
\pgfpathcurveto{\pgfqpoint{2.098706in}{1.665081in}}{\pgfqpoint{2.090806in}{1.668353in}}{\pgfqpoint{2.082570in}{1.668353in}}%
\pgfpathcurveto{\pgfqpoint{2.074334in}{1.668353in}}{\pgfqpoint{2.066434in}{1.665081in}}{\pgfqpoint{2.060610in}{1.659257in}}%
\pgfpathcurveto{\pgfqpoint{2.054786in}{1.653433in}}{\pgfqpoint{2.051514in}{1.645533in}}{\pgfqpoint{2.051514in}{1.637297in}}%
\pgfpathcurveto{\pgfqpoint{2.051514in}{1.629061in}}{\pgfqpoint{2.054786in}{1.621160in}}{\pgfqpoint{2.060610in}{1.615337in}}%
\pgfpathcurveto{\pgfqpoint{2.066434in}{1.609513in}}{\pgfqpoint{2.074334in}{1.606240in}}{\pgfqpoint{2.082570in}{1.606240in}}%
\pgfpathclose%
\pgfusepath{stroke,fill}%
\end{pgfscope}%
\begin{pgfscope}%
\pgfpathrectangle{\pgfqpoint{0.556847in}{0.516222in}}{\pgfqpoint{1.722590in}{1.783528in}} %
\pgfusepath{clip}%
\pgfsetbuttcap%
\pgfsetroundjoin%
\definecolor{currentfill}{rgb}{0.298039,0.447059,0.690196}%
\pgfsetfillcolor{currentfill}%
\pgfsetlinewidth{0.240900pt}%
\definecolor{currentstroke}{rgb}{1.000000,1.000000,1.000000}%
\pgfsetstrokecolor{currentstroke}%
\pgfsetdash{}{0pt}%
\pgfpathmoveto{\pgfqpoint{1.245883in}{0.994745in}}%
\pgfpathcurveto{\pgfqpoint{1.254120in}{0.994745in}}{\pgfqpoint{1.262020in}{0.998017in}}{\pgfqpoint{1.267844in}{1.003841in}}%
\pgfpathcurveto{\pgfqpoint{1.273668in}{1.009665in}}{\pgfqpoint{1.276940in}{1.017565in}}{\pgfqpoint{1.276940in}{1.025802in}}%
\pgfpathcurveto{\pgfqpoint{1.276940in}{1.034038in}}{\pgfqpoint{1.273668in}{1.041938in}}{\pgfqpoint{1.267844in}{1.047762in}}%
\pgfpathcurveto{\pgfqpoint{1.262020in}{1.053586in}}{\pgfqpoint{1.254120in}{1.056858in}}{\pgfqpoint{1.245883in}{1.056858in}}%
\pgfpathcurveto{\pgfqpoint{1.237647in}{1.056858in}}{\pgfqpoint{1.229747in}{1.053586in}}{\pgfqpoint{1.223923in}{1.047762in}}%
\pgfpathcurveto{\pgfqpoint{1.218099in}{1.041938in}}{\pgfqpoint{1.214827in}{1.034038in}}{\pgfqpoint{1.214827in}{1.025802in}}%
\pgfpathcurveto{\pgfqpoint{1.214827in}{1.017565in}}{\pgfqpoint{1.218099in}{1.009665in}}{\pgfqpoint{1.223923in}{1.003841in}}%
\pgfpathcurveto{\pgfqpoint{1.229747in}{0.998017in}}{\pgfqpoint{1.237647in}{0.994745in}}{\pgfqpoint{1.245883in}{0.994745in}}%
\pgfpathclose%
\pgfusepath{stroke,fill}%
\end{pgfscope}%
\begin{pgfscope}%
\pgfpathrectangle{\pgfqpoint{0.556847in}{0.516222in}}{\pgfqpoint{1.722590in}{1.783528in}} %
\pgfusepath{clip}%
\pgfsetbuttcap%
\pgfsetroundjoin%
\definecolor{currentfill}{rgb}{0.298039,0.447059,0.690196}%
\pgfsetfillcolor{currentfill}%
\pgfsetlinewidth{0.240900pt}%
\definecolor{currentstroke}{rgb}{1.000000,1.000000,1.000000}%
\pgfsetstrokecolor{currentstroke}%
\pgfsetdash{}{0pt}%
\pgfpathmoveto{\pgfqpoint{0.802932in}{1.096661in}}%
\pgfpathcurveto{\pgfqpoint{0.811168in}{1.096661in}}{\pgfqpoint{0.819068in}{1.099933in}}{\pgfqpoint{0.824892in}{1.105757in}}%
\pgfpathcurveto{\pgfqpoint{0.830716in}{1.111581in}}{\pgfqpoint{0.833988in}{1.119481in}}{\pgfqpoint{0.833988in}{1.127717in}}%
\pgfpathcurveto{\pgfqpoint{0.833988in}{1.135954in}}{\pgfqpoint{0.830716in}{1.143854in}}{\pgfqpoint{0.824892in}{1.149678in}}%
\pgfpathcurveto{\pgfqpoint{0.819068in}{1.155502in}}{\pgfqpoint{0.811168in}{1.158774in}}{\pgfqpoint{0.802932in}{1.158774in}}%
\pgfpathcurveto{\pgfqpoint{0.794695in}{1.158774in}}{\pgfqpoint{0.786795in}{1.155502in}}{\pgfqpoint{0.780971in}{1.149678in}}%
\pgfpathcurveto{\pgfqpoint{0.775147in}{1.143854in}}{\pgfqpoint{0.771875in}{1.135954in}}{\pgfqpoint{0.771875in}{1.127717in}}%
\pgfpathcurveto{\pgfqpoint{0.771875in}{1.119481in}}{\pgfqpoint{0.775147in}{1.111581in}}{\pgfqpoint{0.780971in}{1.105757in}}%
\pgfpathcurveto{\pgfqpoint{0.786795in}{1.099933in}}{\pgfqpoint{0.794695in}{1.096661in}}{\pgfqpoint{0.802932in}{1.096661in}}%
\pgfpathclose%
\pgfusepath{stroke,fill}%
\end{pgfscope}%
\begin{pgfscope}%
\pgfpathrectangle{\pgfqpoint{0.556847in}{0.516222in}}{\pgfqpoint{1.722590in}{1.783528in}} %
\pgfusepath{clip}%
\pgfsetbuttcap%
\pgfsetroundjoin%
\definecolor{currentfill}{rgb}{0.298039,0.447059,0.690196}%
\pgfsetfillcolor{currentfill}%
\pgfsetlinewidth{0.240900pt}%
\definecolor{currentstroke}{rgb}{1.000000,1.000000,1.000000}%
\pgfsetstrokecolor{currentstroke}%
\pgfsetdash{}{0pt}%
\pgfpathmoveto{\pgfqpoint{1.196666in}{0.892829in}}%
\pgfpathcurveto{\pgfqpoint{1.204903in}{0.892829in}}{\pgfqpoint{1.212803in}{0.896102in}}{\pgfqpoint{1.218627in}{0.901925in}}%
\pgfpathcurveto{\pgfqpoint{1.224451in}{0.907749in}}{\pgfqpoint{1.227723in}{0.915649in}}{\pgfqpoint{1.227723in}{0.923886in}}%
\pgfpathcurveto{\pgfqpoint{1.227723in}{0.932122in}}{\pgfqpoint{1.224451in}{0.940022in}}{\pgfqpoint{1.218627in}{0.945846in}}%
\pgfpathcurveto{\pgfqpoint{1.212803in}{0.951670in}}{\pgfqpoint{1.204903in}{0.954942in}}{\pgfqpoint{1.196666in}{0.954942in}}%
\pgfpathcurveto{\pgfqpoint{1.188430in}{0.954942in}}{\pgfqpoint{1.180530in}{0.951670in}}{\pgfqpoint{1.174706in}{0.945846in}}%
\pgfpathcurveto{\pgfqpoint{1.168882in}{0.940022in}}{\pgfqpoint{1.165610in}{0.932122in}}{\pgfqpoint{1.165610in}{0.923886in}}%
\pgfpathcurveto{\pgfqpoint{1.165610in}{0.915649in}}{\pgfqpoint{1.168882in}{0.907749in}}{\pgfqpoint{1.174706in}{0.901925in}}%
\pgfpathcurveto{\pgfqpoint{1.180530in}{0.896102in}}{\pgfqpoint{1.188430in}{0.892829in}}{\pgfqpoint{1.196666in}{0.892829in}}%
\pgfpathclose%
\pgfusepath{stroke,fill}%
\end{pgfscope}%
\begin{pgfscope}%
\pgfpathrectangle{\pgfqpoint{0.556847in}{0.516222in}}{\pgfqpoint{1.722590in}{1.783528in}} %
\pgfusepath{clip}%
\pgfsetbuttcap%
\pgfsetroundjoin%
\definecolor{currentfill}{rgb}{0.298039,0.447059,0.690196}%
\pgfsetfillcolor{currentfill}%
\pgfsetlinewidth{0.240900pt}%
\definecolor{currentstroke}{rgb}{1.000000,1.000000,1.000000}%
\pgfsetstrokecolor{currentstroke}%
\pgfsetdash{}{0pt}%
\pgfpathmoveto{\pgfqpoint{1.984136in}{1.759114in}}%
\pgfpathcurveto{\pgfqpoint{1.992373in}{1.759114in}}{\pgfqpoint{2.000273in}{1.762386in}}{\pgfqpoint{2.006097in}{1.768210in}}%
\pgfpathcurveto{\pgfqpoint{2.011921in}{1.774034in}}{\pgfqpoint{2.015193in}{1.781934in}}{\pgfqpoint{2.015193in}{1.790171in}}%
\pgfpathcurveto{\pgfqpoint{2.015193in}{1.798407in}}{\pgfqpoint{2.011921in}{1.806307in}}{\pgfqpoint{2.006097in}{1.812131in}}%
\pgfpathcurveto{\pgfqpoint{2.000273in}{1.817955in}}{\pgfqpoint{1.992373in}{1.821227in}}{\pgfqpoint{1.984136in}{1.821227in}}%
\pgfpathcurveto{\pgfqpoint{1.975900in}{1.821227in}}{\pgfqpoint{1.968000in}{1.817955in}}{\pgfqpoint{1.962176in}{1.812131in}}%
\pgfpathcurveto{\pgfqpoint{1.956352in}{1.806307in}}{\pgfqpoint{1.953080in}{1.798407in}}{\pgfqpoint{1.953080in}{1.790171in}}%
\pgfpathcurveto{\pgfqpoint{1.953080in}{1.781934in}}{\pgfqpoint{1.956352in}{1.774034in}}{\pgfqpoint{1.962176in}{1.768210in}}%
\pgfpathcurveto{\pgfqpoint{1.968000in}{1.762386in}}{\pgfqpoint{1.975900in}{1.759114in}}{\pgfqpoint{1.984136in}{1.759114in}}%
\pgfpathclose%
\pgfusepath{stroke,fill}%
\end{pgfscope}%
\begin{pgfscope}%
\pgfsetrectcap%
\pgfsetmiterjoin%
\pgfsetlinewidth{0.000000pt}%
\definecolor{currentstroke}{rgb}{1.000000,1.000000,1.000000}%
\pgfsetstrokecolor{currentstroke}%
\pgfsetdash{}{0pt}%
\pgfpathmoveto{\pgfqpoint{0.556847in}{0.516222in}}%
\pgfpathlineto{\pgfqpoint{0.556847in}{2.299750in}}%
\pgfusepath{}%
\end{pgfscope}%
\begin{pgfscope}%
\pgfsetrectcap%
\pgfsetmiterjoin%
\pgfsetlinewidth{0.000000pt}%
\definecolor{currentstroke}{rgb}{1.000000,1.000000,1.000000}%
\pgfsetstrokecolor{currentstroke}%
\pgfsetdash{}{0pt}%
\pgfpathmoveto{\pgfqpoint{0.556847in}{0.516222in}}%
\pgfpathlineto{\pgfqpoint{2.279437in}{0.516222in}}%
\pgfusepath{}%
\end{pgfscope}%
\end{pgfpicture}%
\makeatother%
\endgroup%

    \caption{Comparison between the times measured in the two realizations.}
    \label{fig_t1t2}
  \end{subfigure}
  \begin{subfigure}[h]{.5\linewidth}
    %% Creator: Matplotlib, PGF backend
%%
%% To include the figure in your LaTeX document, write
%%   \input{<filename>.pgf}
%%
%% Make sure the required packages are loaded in your preamble
%%   \usepackage{pgf}
%%
%% Figures using additional raster images can only be included by \input if
%% they are in the same directory as the main LaTeX file. For loading figures
%% from other directories you can use the `import` package
%%   \usepackage{import}
%% and then include the figures with
%%   \import{<path to file>}{<filename>.pgf}
%%
%% Matplotlib used the following preamble
%%   \usepackage[utf8x]{inputenc}
%%   \usepackage[T1]{fontenc}
%%   \usepackage{cmbright}
%%
\begingroup%
\makeatletter%
\begin{pgfpicture}%
\pgfpathrectangle{\pgfpointorigin}{\pgfqpoint{2.500000in}{2.500000in}}%
\pgfusepath{use as bounding box, clip}%
\begin{pgfscope}%
\pgfsetbuttcap%
\pgfsetmiterjoin%
\definecolor{currentfill}{rgb}{1.000000,1.000000,1.000000}%
\pgfsetfillcolor{currentfill}%
\pgfsetlinewidth{0.000000pt}%
\definecolor{currentstroke}{rgb}{1.000000,1.000000,1.000000}%
\pgfsetstrokecolor{currentstroke}%
\pgfsetdash{}{0pt}%
\pgfpathmoveto{\pgfqpoint{0.000000in}{0.000000in}}%
\pgfpathlineto{\pgfqpoint{2.500000in}{0.000000in}}%
\pgfpathlineto{\pgfqpoint{2.500000in}{2.500000in}}%
\pgfpathlineto{\pgfqpoint{0.000000in}{2.500000in}}%
\pgfpathclose%
\pgfusepath{fill}%
\end{pgfscope}%
\begin{pgfscope}%
\pgfsetbuttcap%
\pgfsetmiterjoin%
\definecolor{currentfill}{rgb}{0.917647,0.917647,0.949020}%
\pgfsetfillcolor{currentfill}%
\pgfsetlinewidth{0.000000pt}%
\definecolor{currentstroke}{rgb}{0.000000,0.000000,0.000000}%
\pgfsetstrokecolor{currentstroke}%
\pgfsetstrokeopacity{0.000000}%
\pgfsetdash{}{0pt}%
\pgfpathmoveto{\pgfqpoint{0.556847in}{0.516222in}}%
\pgfpathlineto{\pgfqpoint{2.279437in}{0.516222in}}%
\pgfpathlineto{\pgfqpoint{2.279437in}{2.299750in}}%
\pgfpathlineto{\pgfqpoint{0.556847in}{2.299750in}}%
\pgfpathclose%
\pgfusepath{fill}%
\end{pgfscope}%
\begin{pgfscope}%
\pgfpathrectangle{\pgfqpoint{0.556847in}{0.516222in}}{\pgfqpoint{1.722590in}{1.783528in}} %
\pgfusepath{clip}%
\pgfsetroundcap%
\pgfsetroundjoin%
\pgfsetlinewidth{0.803000pt}%
\definecolor{currentstroke}{rgb}{1.000000,1.000000,1.000000}%
\pgfsetstrokecolor{currentstroke}%
\pgfsetdash{}{0pt}%
\pgfpathmoveto{\pgfqpoint{0.556847in}{0.516222in}}%
\pgfpathlineto{\pgfqpoint{0.556847in}{2.299750in}}%
\pgfusepath{stroke}%
\end{pgfscope}%
\begin{pgfscope}%
\pgfsetbuttcap%
\pgfsetroundjoin%
\definecolor{currentfill}{rgb}{0.150000,0.150000,0.150000}%
\pgfsetfillcolor{currentfill}%
\pgfsetlinewidth{0.803000pt}%
\definecolor{currentstroke}{rgb}{0.150000,0.150000,0.150000}%
\pgfsetstrokecolor{currentstroke}%
\pgfsetdash{}{0pt}%
\pgfsys@defobject{currentmarker}{\pgfqpoint{0.000000in}{0.000000in}}{\pgfqpoint{0.000000in}{0.000000in}}{%
\pgfpathmoveto{\pgfqpoint{0.000000in}{0.000000in}}%
\pgfpathlineto{\pgfqpoint{0.000000in}{0.000000in}}%
\pgfusepath{stroke,fill}%
}%
\begin{pgfscope}%
\pgfsys@transformshift{0.556847in}{0.516222in}%
\pgfsys@useobject{currentmarker}{}%
\end{pgfscope}%
\end{pgfscope}%
\begin{pgfscope}%
\definecolor{textcolor}{rgb}{0.150000,0.150000,0.150000}%
\pgfsetstrokecolor{textcolor}%
\pgfsetfillcolor{textcolor}%
\pgftext[x=0.556847in,y=0.438444in,,top]{\color{textcolor}\sffamily\fontsize{8.000000}{9.600000}\selectfont 5}%
\end{pgfscope}%
\begin{pgfscope}%
\pgfpathrectangle{\pgfqpoint{0.556847in}{0.516222in}}{\pgfqpoint{1.722590in}{1.783528in}} %
\pgfusepath{clip}%
\pgfsetroundcap%
\pgfsetroundjoin%
\pgfsetlinewidth{0.803000pt}%
\definecolor{currentstroke}{rgb}{1.000000,1.000000,1.000000}%
\pgfsetstrokecolor{currentstroke}%
\pgfsetdash{}{0pt}%
\pgfpathmoveto{\pgfqpoint{0.843946in}{0.516222in}}%
\pgfpathlineto{\pgfqpoint{0.843946in}{2.299750in}}%
\pgfusepath{stroke}%
\end{pgfscope}%
\begin{pgfscope}%
\pgfsetbuttcap%
\pgfsetroundjoin%
\definecolor{currentfill}{rgb}{0.150000,0.150000,0.150000}%
\pgfsetfillcolor{currentfill}%
\pgfsetlinewidth{0.803000pt}%
\definecolor{currentstroke}{rgb}{0.150000,0.150000,0.150000}%
\pgfsetstrokecolor{currentstroke}%
\pgfsetdash{}{0pt}%
\pgfsys@defobject{currentmarker}{\pgfqpoint{0.000000in}{0.000000in}}{\pgfqpoint{0.000000in}{0.000000in}}{%
\pgfpathmoveto{\pgfqpoint{0.000000in}{0.000000in}}%
\pgfpathlineto{\pgfqpoint{0.000000in}{0.000000in}}%
\pgfusepath{stroke,fill}%
}%
\begin{pgfscope}%
\pgfsys@transformshift{0.843946in}{0.516222in}%
\pgfsys@useobject{currentmarker}{}%
\end{pgfscope}%
\end{pgfscope}%
\begin{pgfscope}%
\definecolor{textcolor}{rgb}{0.150000,0.150000,0.150000}%
\pgfsetstrokecolor{textcolor}%
\pgfsetfillcolor{textcolor}%
\pgftext[x=0.843946in,y=0.438444in,,top]{\color{textcolor}\sffamily\fontsize{8.000000}{9.600000}\selectfont 6}%
\end{pgfscope}%
\begin{pgfscope}%
\pgfpathrectangle{\pgfqpoint{0.556847in}{0.516222in}}{\pgfqpoint{1.722590in}{1.783528in}} %
\pgfusepath{clip}%
\pgfsetroundcap%
\pgfsetroundjoin%
\pgfsetlinewidth{0.803000pt}%
\definecolor{currentstroke}{rgb}{1.000000,1.000000,1.000000}%
\pgfsetstrokecolor{currentstroke}%
\pgfsetdash{}{0pt}%
\pgfpathmoveto{\pgfqpoint{1.131044in}{0.516222in}}%
\pgfpathlineto{\pgfqpoint{1.131044in}{2.299750in}}%
\pgfusepath{stroke}%
\end{pgfscope}%
\begin{pgfscope}%
\pgfsetbuttcap%
\pgfsetroundjoin%
\definecolor{currentfill}{rgb}{0.150000,0.150000,0.150000}%
\pgfsetfillcolor{currentfill}%
\pgfsetlinewidth{0.803000pt}%
\definecolor{currentstroke}{rgb}{0.150000,0.150000,0.150000}%
\pgfsetstrokecolor{currentstroke}%
\pgfsetdash{}{0pt}%
\pgfsys@defobject{currentmarker}{\pgfqpoint{0.000000in}{0.000000in}}{\pgfqpoint{0.000000in}{0.000000in}}{%
\pgfpathmoveto{\pgfqpoint{0.000000in}{0.000000in}}%
\pgfpathlineto{\pgfqpoint{0.000000in}{0.000000in}}%
\pgfusepath{stroke,fill}%
}%
\begin{pgfscope}%
\pgfsys@transformshift{1.131044in}{0.516222in}%
\pgfsys@useobject{currentmarker}{}%
\end{pgfscope}%
\end{pgfscope}%
\begin{pgfscope}%
\definecolor{textcolor}{rgb}{0.150000,0.150000,0.150000}%
\pgfsetstrokecolor{textcolor}%
\pgfsetfillcolor{textcolor}%
\pgftext[x=1.131044in,y=0.438444in,,top]{\color{textcolor}\sffamily\fontsize{8.000000}{9.600000}\selectfont 7}%
\end{pgfscope}%
\begin{pgfscope}%
\pgfpathrectangle{\pgfqpoint{0.556847in}{0.516222in}}{\pgfqpoint{1.722590in}{1.783528in}} %
\pgfusepath{clip}%
\pgfsetroundcap%
\pgfsetroundjoin%
\pgfsetlinewidth{0.803000pt}%
\definecolor{currentstroke}{rgb}{1.000000,1.000000,1.000000}%
\pgfsetstrokecolor{currentstroke}%
\pgfsetdash{}{0pt}%
\pgfpathmoveto{\pgfqpoint{1.418142in}{0.516222in}}%
\pgfpathlineto{\pgfqpoint{1.418142in}{2.299750in}}%
\pgfusepath{stroke}%
\end{pgfscope}%
\begin{pgfscope}%
\pgfsetbuttcap%
\pgfsetroundjoin%
\definecolor{currentfill}{rgb}{0.150000,0.150000,0.150000}%
\pgfsetfillcolor{currentfill}%
\pgfsetlinewidth{0.803000pt}%
\definecolor{currentstroke}{rgb}{0.150000,0.150000,0.150000}%
\pgfsetstrokecolor{currentstroke}%
\pgfsetdash{}{0pt}%
\pgfsys@defobject{currentmarker}{\pgfqpoint{0.000000in}{0.000000in}}{\pgfqpoint{0.000000in}{0.000000in}}{%
\pgfpathmoveto{\pgfqpoint{0.000000in}{0.000000in}}%
\pgfpathlineto{\pgfqpoint{0.000000in}{0.000000in}}%
\pgfusepath{stroke,fill}%
}%
\begin{pgfscope}%
\pgfsys@transformshift{1.418142in}{0.516222in}%
\pgfsys@useobject{currentmarker}{}%
\end{pgfscope}%
\end{pgfscope}%
\begin{pgfscope}%
\definecolor{textcolor}{rgb}{0.150000,0.150000,0.150000}%
\pgfsetstrokecolor{textcolor}%
\pgfsetfillcolor{textcolor}%
\pgftext[x=1.418142in,y=0.438444in,,top]{\color{textcolor}\sffamily\fontsize{8.000000}{9.600000}\selectfont 8}%
\end{pgfscope}%
\begin{pgfscope}%
\pgfpathrectangle{\pgfqpoint{0.556847in}{0.516222in}}{\pgfqpoint{1.722590in}{1.783528in}} %
\pgfusepath{clip}%
\pgfsetroundcap%
\pgfsetroundjoin%
\pgfsetlinewidth{0.803000pt}%
\definecolor{currentstroke}{rgb}{1.000000,1.000000,1.000000}%
\pgfsetstrokecolor{currentstroke}%
\pgfsetdash{}{0pt}%
\pgfpathmoveto{\pgfqpoint{1.705241in}{0.516222in}}%
\pgfpathlineto{\pgfqpoint{1.705241in}{2.299750in}}%
\pgfusepath{stroke}%
\end{pgfscope}%
\begin{pgfscope}%
\pgfsetbuttcap%
\pgfsetroundjoin%
\definecolor{currentfill}{rgb}{0.150000,0.150000,0.150000}%
\pgfsetfillcolor{currentfill}%
\pgfsetlinewidth{0.803000pt}%
\definecolor{currentstroke}{rgb}{0.150000,0.150000,0.150000}%
\pgfsetstrokecolor{currentstroke}%
\pgfsetdash{}{0pt}%
\pgfsys@defobject{currentmarker}{\pgfqpoint{0.000000in}{0.000000in}}{\pgfqpoint{0.000000in}{0.000000in}}{%
\pgfpathmoveto{\pgfqpoint{0.000000in}{0.000000in}}%
\pgfpathlineto{\pgfqpoint{0.000000in}{0.000000in}}%
\pgfusepath{stroke,fill}%
}%
\begin{pgfscope}%
\pgfsys@transformshift{1.705241in}{0.516222in}%
\pgfsys@useobject{currentmarker}{}%
\end{pgfscope}%
\end{pgfscope}%
\begin{pgfscope}%
\definecolor{textcolor}{rgb}{0.150000,0.150000,0.150000}%
\pgfsetstrokecolor{textcolor}%
\pgfsetfillcolor{textcolor}%
\pgftext[x=1.705241in,y=0.438444in,,top]{\color{textcolor}\sffamily\fontsize{8.000000}{9.600000}\selectfont 9}%
\end{pgfscope}%
\begin{pgfscope}%
\pgfpathrectangle{\pgfqpoint{0.556847in}{0.516222in}}{\pgfqpoint{1.722590in}{1.783528in}} %
\pgfusepath{clip}%
\pgfsetroundcap%
\pgfsetroundjoin%
\pgfsetlinewidth{0.803000pt}%
\definecolor{currentstroke}{rgb}{1.000000,1.000000,1.000000}%
\pgfsetstrokecolor{currentstroke}%
\pgfsetdash{}{0pt}%
\pgfpathmoveto{\pgfqpoint{1.992339in}{0.516222in}}%
\pgfpathlineto{\pgfqpoint{1.992339in}{2.299750in}}%
\pgfusepath{stroke}%
\end{pgfscope}%
\begin{pgfscope}%
\pgfsetbuttcap%
\pgfsetroundjoin%
\definecolor{currentfill}{rgb}{0.150000,0.150000,0.150000}%
\pgfsetfillcolor{currentfill}%
\pgfsetlinewidth{0.803000pt}%
\definecolor{currentstroke}{rgb}{0.150000,0.150000,0.150000}%
\pgfsetstrokecolor{currentstroke}%
\pgfsetdash{}{0pt}%
\pgfsys@defobject{currentmarker}{\pgfqpoint{0.000000in}{0.000000in}}{\pgfqpoint{0.000000in}{0.000000in}}{%
\pgfpathmoveto{\pgfqpoint{0.000000in}{0.000000in}}%
\pgfpathlineto{\pgfqpoint{0.000000in}{0.000000in}}%
\pgfusepath{stroke,fill}%
}%
\begin{pgfscope}%
\pgfsys@transformshift{1.992339in}{0.516222in}%
\pgfsys@useobject{currentmarker}{}%
\end{pgfscope}%
\end{pgfscope}%
\begin{pgfscope}%
\definecolor{textcolor}{rgb}{0.150000,0.150000,0.150000}%
\pgfsetstrokecolor{textcolor}%
\pgfsetfillcolor{textcolor}%
\pgftext[x=1.992339in,y=0.438444in,,top]{\color{textcolor}\sffamily\fontsize{8.000000}{9.600000}\selectfont 10}%
\end{pgfscope}%
\begin{pgfscope}%
\pgfpathrectangle{\pgfqpoint{0.556847in}{0.516222in}}{\pgfqpoint{1.722590in}{1.783528in}} %
\pgfusepath{clip}%
\pgfsetroundcap%
\pgfsetroundjoin%
\pgfsetlinewidth{0.803000pt}%
\definecolor{currentstroke}{rgb}{1.000000,1.000000,1.000000}%
\pgfsetstrokecolor{currentstroke}%
\pgfsetdash{}{0pt}%
\pgfpathmoveto{\pgfqpoint{2.279437in}{0.516222in}}%
\pgfpathlineto{\pgfqpoint{2.279437in}{2.299750in}}%
\pgfusepath{stroke}%
\end{pgfscope}%
\begin{pgfscope}%
\pgfsetbuttcap%
\pgfsetroundjoin%
\definecolor{currentfill}{rgb}{0.150000,0.150000,0.150000}%
\pgfsetfillcolor{currentfill}%
\pgfsetlinewidth{0.803000pt}%
\definecolor{currentstroke}{rgb}{0.150000,0.150000,0.150000}%
\pgfsetstrokecolor{currentstroke}%
\pgfsetdash{}{0pt}%
\pgfsys@defobject{currentmarker}{\pgfqpoint{0.000000in}{0.000000in}}{\pgfqpoint{0.000000in}{0.000000in}}{%
\pgfpathmoveto{\pgfqpoint{0.000000in}{0.000000in}}%
\pgfpathlineto{\pgfqpoint{0.000000in}{0.000000in}}%
\pgfusepath{stroke,fill}%
}%
\begin{pgfscope}%
\pgfsys@transformshift{2.279437in}{0.516222in}%
\pgfsys@useobject{currentmarker}{}%
\end{pgfscope}%
\end{pgfscope}%
\begin{pgfscope}%
\definecolor{textcolor}{rgb}{0.150000,0.150000,0.150000}%
\pgfsetstrokecolor{textcolor}%
\pgfsetfillcolor{textcolor}%
\pgftext[x=2.279437in,y=0.438444in,,top]{\color{textcolor}\sffamily\fontsize{8.000000}{9.600000}\selectfont 11}%
\end{pgfscope}%
\begin{pgfscope}%
\definecolor{textcolor}{rgb}{0.150000,0.150000,0.150000}%
\pgfsetstrokecolor{textcolor}%
\pgfsetfillcolor{textcolor}%
\pgftext[x=1.418142in,y=0.273321in,,top]{\color{textcolor}\sffamily\fontsize{8.800000}{10.560000}\selectfont Tail length}%
\end{pgfscope}%
\begin{pgfscope}%
\pgfpathrectangle{\pgfqpoint{0.556847in}{0.516222in}}{\pgfqpoint{1.722590in}{1.783528in}} %
\pgfusepath{clip}%
\pgfsetroundcap%
\pgfsetroundjoin%
\pgfsetlinewidth{0.803000pt}%
\definecolor{currentstroke}{rgb}{1.000000,1.000000,1.000000}%
\pgfsetstrokecolor{currentstroke}%
\pgfsetdash{}{0pt}%
\pgfpathmoveto{\pgfqpoint{0.556847in}{0.516222in}}%
\pgfpathlineto{\pgfqpoint{2.279437in}{0.516222in}}%
\pgfusepath{stroke}%
\end{pgfscope}%
\begin{pgfscope}%
\pgfsetbuttcap%
\pgfsetroundjoin%
\definecolor{currentfill}{rgb}{0.150000,0.150000,0.150000}%
\pgfsetfillcolor{currentfill}%
\pgfsetlinewidth{0.803000pt}%
\definecolor{currentstroke}{rgb}{0.150000,0.150000,0.150000}%
\pgfsetstrokecolor{currentstroke}%
\pgfsetdash{}{0pt}%
\pgfsys@defobject{currentmarker}{\pgfqpoint{0.000000in}{0.000000in}}{\pgfqpoint{0.000000in}{0.000000in}}{%
\pgfpathmoveto{\pgfqpoint{0.000000in}{0.000000in}}%
\pgfpathlineto{\pgfqpoint{0.000000in}{0.000000in}}%
\pgfusepath{stroke,fill}%
}%
\begin{pgfscope}%
\pgfsys@transformshift{0.556847in}{0.516222in}%
\pgfsys@useobject{currentmarker}{}%
\end{pgfscope}%
\end{pgfscope}%
\begin{pgfscope}%
\definecolor{textcolor}{rgb}{0.150000,0.150000,0.150000}%
\pgfsetstrokecolor{textcolor}%
\pgfsetfillcolor{textcolor}%
\pgftext[x=0.479069in,y=0.516222in,right,]{\color{textcolor}\sffamily\fontsize{8.000000}{9.600000}\selectfont 7}%
\end{pgfscope}%
\begin{pgfscope}%
\pgfpathrectangle{\pgfqpoint{0.556847in}{0.516222in}}{\pgfqpoint{1.722590in}{1.783528in}} %
\pgfusepath{clip}%
\pgfsetroundcap%
\pgfsetroundjoin%
\pgfsetlinewidth{0.803000pt}%
\definecolor{currentstroke}{rgb}{1.000000,1.000000,1.000000}%
\pgfsetstrokecolor{currentstroke}%
\pgfsetdash{}{0pt}%
\pgfpathmoveto{\pgfqpoint{0.556847in}{0.813477in}}%
\pgfpathlineto{\pgfqpoint{2.279437in}{0.813477in}}%
\pgfusepath{stroke}%
\end{pgfscope}%
\begin{pgfscope}%
\pgfsetbuttcap%
\pgfsetroundjoin%
\definecolor{currentfill}{rgb}{0.150000,0.150000,0.150000}%
\pgfsetfillcolor{currentfill}%
\pgfsetlinewidth{0.803000pt}%
\definecolor{currentstroke}{rgb}{0.150000,0.150000,0.150000}%
\pgfsetstrokecolor{currentstroke}%
\pgfsetdash{}{0pt}%
\pgfsys@defobject{currentmarker}{\pgfqpoint{0.000000in}{0.000000in}}{\pgfqpoint{0.000000in}{0.000000in}}{%
\pgfpathmoveto{\pgfqpoint{0.000000in}{0.000000in}}%
\pgfpathlineto{\pgfqpoint{0.000000in}{0.000000in}}%
\pgfusepath{stroke,fill}%
}%
\begin{pgfscope}%
\pgfsys@transformshift{0.556847in}{0.813477in}%
\pgfsys@useobject{currentmarker}{}%
\end{pgfscope}%
\end{pgfscope}%
\begin{pgfscope}%
\definecolor{textcolor}{rgb}{0.150000,0.150000,0.150000}%
\pgfsetstrokecolor{textcolor}%
\pgfsetfillcolor{textcolor}%
\pgftext[x=0.479069in,y=0.813477in,right,]{\color{textcolor}\sffamily\fontsize{8.000000}{9.600000}\selectfont 8}%
\end{pgfscope}%
\begin{pgfscope}%
\pgfpathrectangle{\pgfqpoint{0.556847in}{0.516222in}}{\pgfqpoint{1.722590in}{1.783528in}} %
\pgfusepath{clip}%
\pgfsetroundcap%
\pgfsetroundjoin%
\pgfsetlinewidth{0.803000pt}%
\definecolor{currentstroke}{rgb}{1.000000,1.000000,1.000000}%
\pgfsetstrokecolor{currentstroke}%
\pgfsetdash{}{0pt}%
\pgfpathmoveto{\pgfqpoint{0.556847in}{1.110731in}}%
\pgfpathlineto{\pgfqpoint{2.279437in}{1.110731in}}%
\pgfusepath{stroke}%
\end{pgfscope}%
\begin{pgfscope}%
\pgfsetbuttcap%
\pgfsetroundjoin%
\definecolor{currentfill}{rgb}{0.150000,0.150000,0.150000}%
\pgfsetfillcolor{currentfill}%
\pgfsetlinewidth{0.803000pt}%
\definecolor{currentstroke}{rgb}{0.150000,0.150000,0.150000}%
\pgfsetstrokecolor{currentstroke}%
\pgfsetdash{}{0pt}%
\pgfsys@defobject{currentmarker}{\pgfqpoint{0.000000in}{0.000000in}}{\pgfqpoint{0.000000in}{0.000000in}}{%
\pgfpathmoveto{\pgfqpoint{0.000000in}{0.000000in}}%
\pgfpathlineto{\pgfqpoint{0.000000in}{0.000000in}}%
\pgfusepath{stroke,fill}%
}%
\begin{pgfscope}%
\pgfsys@transformshift{0.556847in}{1.110731in}%
\pgfsys@useobject{currentmarker}{}%
\end{pgfscope}%
\end{pgfscope}%
\begin{pgfscope}%
\definecolor{textcolor}{rgb}{0.150000,0.150000,0.150000}%
\pgfsetstrokecolor{textcolor}%
\pgfsetfillcolor{textcolor}%
\pgftext[x=0.479069in,y=1.110731in,right,]{\color{textcolor}\sffamily\fontsize{8.000000}{9.600000}\selectfont 9}%
\end{pgfscope}%
\begin{pgfscope}%
\pgfpathrectangle{\pgfqpoint{0.556847in}{0.516222in}}{\pgfqpoint{1.722590in}{1.783528in}} %
\pgfusepath{clip}%
\pgfsetroundcap%
\pgfsetroundjoin%
\pgfsetlinewidth{0.803000pt}%
\definecolor{currentstroke}{rgb}{1.000000,1.000000,1.000000}%
\pgfsetstrokecolor{currentstroke}%
\pgfsetdash{}{0pt}%
\pgfpathmoveto{\pgfqpoint{0.556847in}{1.407986in}}%
\pgfpathlineto{\pgfqpoint{2.279437in}{1.407986in}}%
\pgfusepath{stroke}%
\end{pgfscope}%
\begin{pgfscope}%
\pgfsetbuttcap%
\pgfsetroundjoin%
\definecolor{currentfill}{rgb}{0.150000,0.150000,0.150000}%
\pgfsetfillcolor{currentfill}%
\pgfsetlinewidth{0.803000pt}%
\definecolor{currentstroke}{rgb}{0.150000,0.150000,0.150000}%
\pgfsetstrokecolor{currentstroke}%
\pgfsetdash{}{0pt}%
\pgfsys@defobject{currentmarker}{\pgfqpoint{0.000000in}{0.000000in}}{\pgfqpoint{0.000000in}{0.000000in}}{%
\pgfpathmoveto{\pgfqpoint{0.000000in}{0.000000in}}%
\pgfpathlineto{\pgfqpoint{0.000000in}{0.000000in}}%
\pgfusepath{stroke,fill}%
}%
\begin{pgfscope}%
\pgfsys@transformshift{0.556847in}{1.407986in}%
\pgfsys@useobject{currentmarker}{}%
\end{pgfscope}%
\end{pgfscope}%
\begin{pgfscope}%
\definecolor{textcolor}{rgb}{0.150000,0.150000,0.150000}%
\pgfsetstrokecolor{textcolor}%
\pgfsetfillcolor{textcolor}%
\pgftext[x=0.479069in,y=1.407986in,right,]{\color{textcolor}\sffamily\fontsize{8.000000}{9.600000}\selectfont 10}%
\end{pgfscope}%
\begin{pgfscope}%
\pgfpathrectangle{\pgfqpoint{0.556847in}{0.516222in}}{\pgfqpoint{1.722590in}{1.783528in}} %
\pgfusepath{clip}%
\pgfsetroundcap%
\pgfsetroundjoin%
\pgfsetlinewidth{0.803000pt}%
\definecolor{currentstroke}{rgb}{1.000000,1.000000,1.000000}%
\pgfsetstrokecolor{currentstroke}%
\pgfsetdash{}{0pt}%
\pgfpathmoveto{\pgfqpoint{0.556847in}{1.705241in}}%
\pgfpathlineto{\pgfqpoint{2.279437in}{1.705241in}}%
\pgfusepath{stroke}%
\end{pgfscope}%
\begin{pgfscope}%
\pgfsetbuttcap%
\pgfsetroundjoin%
\definecolor{currentfill}{rgb}{0.150000,0.150000,0.150000}%
\pgfsetfillcolor{currentfill}%
\pgfsetlinewidth{0.803000pt}%
\definecolor{currentstroke}{rgb}{0.150000,0.150000,0.150000}%
\pgfsetstrokecolor{currentstroke}%
\pgfsetdash{}{0pt}%
\pgfsys@defobject{currentmarker}{\pgfqpoint{0.000000in}{0.000000in}}{\pgfqpoint{0.000000in}{0.000000in}}{%
\pgfpathmoveto{\pgfqpoint{0.000000in}{0.000000in}}%
\pgfpathlineto{\pgfqpoint{0.000000in}{0.000000in}}%
\pgfusepath{stroke,fill}%
}%
\begin{pgfscope}%
\pgfsys@transformshift{0.556847in}{1.705241in}%
\pgfsys@useobject{currentmarker}{}%
\end{pgfscope}%
\end{pgfscope}%
\begin{pgfscope}%
\definecolor{textcolor}{rgb}{0.150000,0.150000,0.150000}%
\pgfsetstrokecolor{textcolor}%
\pgfsetfillcolor{textcolor}%
\pgftext[x=0.479069in,y=1.705241in,right,]{\color{textcolor}\sffamily\fontsize{8.000000}{9.600000}\selectfont 11}%
\end{pgfscope}%
\begin{pgfscope}%
\pgfpathrectangle{\pgfqpoint{0.556847in}{0.516222in}}{\pgfqpoint{1.722590in}{1.783528in}} %
\pgfusepath{clip}%
\pgfsetroundcap%
\pgfsetroundjoin%
\pgfsetlinewidth{0.803000pt}%
\definecolor{currentstroke}{rgb}{1.000000,1.000000,1.000000}%
\pgfsetstrokecolor{currentstroke}%
\pgfsetdash{}{0pt}%
\pgfpathmoveto{\pgfqpoint{0.556847in}{2.002495in}}%
\pgfpathlineto{\pgfqpoint{2.279437in}{2.002495in}}%
\pgfusepath{stroke}%
\end{pgfscope}%
\begin{pgfscope}%
\pgfsetbuttcap%
\pgfsetroundjoin%
\definecolor{currentfill}{rgb}{0.150000,0.150000,0.150000}%
\pgfsetfillcolor{currentfill}%
\pgfsetlinewidth{0.803000pt}%
\definecolor{currentstroke}{rgb}{0.150000,0.150000,0.150000}%
\pgfsetstrokecolor{currentstroke}%
\pgfsetdash{}{0pt}%
\pgfsys@defobject{currentmarker}{\pgfqpoint{0.000000in}{0.000000in}}{\pgfqpoint{0.000000in}{0.000000in}}{%
\pgfpathmoveto{\pgfqpoint{0.000000in}{0.000000in}}%
\pgfpathlineto{\pgfqpoint{0.000000in}{0.000000in}}%
\pgfusepath{stroke,fill}%
}%
\begin{pgfscope}%
\pgfsys@transformshift{0.556847in}{2.002495in}%
\pgfsys@useobject{currentmarker}{}%
\end{pgfscope}%
\end{pgfscope}%
\begin{pgfscope}%
\definecolor{textcolor}{rgb}{0.150000,0.150000,0.150000}%
\pgfsetstrokecolor{textcolor}%
\pgfsetfillcolor{textcolor}%
\pgftext[x=0.479069in,y=2.002495in,right,]{\color{textcolor}\sffamily\fontsize{8.000000}{9.600000}\selectfont 12}%
\end{pgfscope}%
\begin{pgfscope}%
\pgfpathrectangle{\pgfqpoint{0.556847in}{0.516222in}}{\pgfqpoint{1.722590in}{1.783528in}} %
\pgfusepath{clip}%
\pgfsetroundcap%
\pgfsetroundjoin%
\pgfsetlinewidth{0.803000pt}%
\definecolor{currentstroke}{rgb}{1.000000,1.000000,1.000000}%
\pgfsetstrokecolor{currentstroke}%
\pgfsetdash{}{0pt}%
\pgfpathmoveto{\pgfqpoint{0.556847in}{2.299750in}}%
\pgfpathlineto{\pgfqpoint{2.279437in}{2.299750in}}%
\pgfusepath{stroke}%
\end{pgfscope}%
\begin{pgfscope}%
\pgfsetbuttcap%
\pgfsetroundjoin%
\definecolor{currentfill}{rgb}{0.150000,0.150000,0.150000}%
\pgfsetfillcolor{currentfill}%
\pgfsetlinewidth{0.803000pt}%
\definecolor{currentstroke}{rgb}{0.150000,0.150000,0.150000}%
\pgfsetstrokecolor{currentstroke}%
\pgfsetdash{}{0pt}%
\pgfsys@defobject{currentmarker}{\pgfqpoint{0.000000in}{0.000000in}}{\pgfqpoint{0.000000in}{0.000000in}}{%
\pgfpathmoveto{\pgfqpoint{0.000000in}{0.000000in}}%
\pgfpathlineto{\pgfqpoint{0.000000in}{0.000000in}}%
\pgfusepath{stroke,fill}%
}%
\begin{pgfscope}%
\pgfsys@transformshift{0.556847in}{2.299750in}%
\pgfsys@useobject{currentmarker}{}%
\end{pgfscope}%
\end{pgfscope}%
\begin{pgfscope}%
\definecolor{textcolor}{rgb}{0.150000,0.150000,0.150000}%
\pgfsetstrokecolor{textcolor}%
\pgfsetfillcolor{textcolor}%
\pgftext[x=0.479069in,y=2.299750in,right,]{\color{textcolor}\sffamily\fontsize{8.000000}{9.600000}\selectfont 13}%
\end{pgfscope}%
\begin{pgfscope}%
\definecolor{textcolor}{rgb}{0.150000,0.150000,0.150000}%
\pgfsetstrokecolor{textcolor}%
\pgfsetfillcolor{textcolor}%
\pgftext[x=0.286014in,y=1.407986in,,bottom,rotate=90.000000]{\color{textcolor}\sffamily\fontsize{8.800000}{10.560000}\selectfont Arm length}%
\end{pgfscope}%
\begin{pgfscope}%
\pgfpathrectangle{\pgfqpoint{0.556847in}{0.516222in}}{\pgfqpoint{1.722590in}{1.783528in}} %
\pgfusepath{clip}%
\pgfsetbuttcap%
\pgfsetroundjoin%
\definecolor{currentfill}{rgb}{0.298039,0.447059,0.690196}%
\pgfsetfillcolor{currentfill}%
\pgfsetlinewidth{0.240900pt}%
\definecolor{currentstroke}{rgb}{1.000000,1.000000,1.000000}%
\pgfsetstrokecolor{currentstroke}%
\pgfsetdash{}{0pt}%
\pgfpathmoveto{\pgfqpoint{1.021947in}{1.287753in}}%
\pgfpathcurveto{\pgfqpoint{1.030183in}{1.287753in}}{\pgfqpoint{1.038083in}{1.291026in}}{\pgfqpoint{1.043907in}{1.296849in}}%
\pgfpathcurveto{\pgfqpoint{1.049731in}{1.302673in}}{\pgfqpoint{1.053003in}{1.310573in}}{\pgfqpoint{1.053003in}{1.318810in}}%
\pgfpathcurveto{\pgfqpoint{1.053003in}{1.327046in}}{\pgfqpoint{1.049731in}{1.334946in}}{\pgfqpoint{1.043907in}{1.340770in}}%
\pgfpathcurveto{\pgfqpoint{1.038083in}{1.346594in}}{\pgfqpoint{1.030183in}{1.349866in}}{\pgfqpoint{1.021947in}{1.349866in}}%
\pgfpathcurveto{\pgfqpoint{1.013710in}{1.349866in}}{\pgfqpoint{1.005810in}{1.346594in}}{\pgfqpoint{0.999986in}{1.340770in}}%
\pgfpathcurveto{\pgfqpoint{0.994162in}{1.334946in}}{\pgfqpoint{0.990890in}{1.327046in}}{\pgfqpoint{0.990890in}{1.318810in}}%
\pgfpathcurveto{\pgfqpoint{0.990890in}{1.310573in}}{\pgfqpoint{0.994162in}{1.302673in}}{\pgfqpoint{0.999986in}{1.296849in}}%
\pgfpathcurveto{\pgfqpoint{1.005810in}{1.291026in}}{\pgfqpoint{1.013710in}{1.287753in}}{\pgfqpoint{1.021947in}{1.287753in}}%
\pgfpathclose%
\pgfusepath{stroke,fill}%
\end{pgfscope}%
\begin{pgfscope}%
\pgfpathrectangle{\pgfqpoint{0.556847in}{0.516222in}}{\pgfqpoint{1.722590in}{1.783528in}} %
\pgfusepath{clip}%
\pgfsetbuttcap%
\pgfsetroundjoin%
\definecolor{currentfill}{rgb}{0.298039,0.447059,0.690196}%
\pgfsetfillcolor{currentfill}%
\pgfsetlinewidth{0.240900pt}%
\definecolor{currentstroke}{rgb}{1.000000,1.000000,1.000000}%
\pgfsetstrokecolor{currentstroke}%
\pgfsetdash{}{0pt}%
\pgfpathmoveto{\pgfqpoint{2.095695in}{1.763361in}}%
\pgfpathcurveto{\pgfqpoint{2.103931in}{1.763361in}}{\pgfqpoint{2.111831in}{1.766633in}}{\pgfqpoint{2.117655in}{1.772457in}}%
\pgfpathcurveto{\pgfqpoint{2.123479in}{1.778281in}}{\pgfqpoint{2.126751in}{1.786181in}}{\pgfqpoint{2.126751in}{1.794417in}}%
\pgfpathcurveto{\pgfqpoint{2.126751in}{1.802653in}}{\pgfqpoint{2.123479in}{1.810553in}}{\pgfqpoint{2.117655in}{1.816377in}}%
\pgfpathcurveto{\pgfqpoint{2.111831in}{1.822201in}}{\pgfqpoint{2.103931in}{1.825474in}}{\pgfqpoint{2.095695in}{1.825474in}}%
\pgfpathcurveto{\pgfqpoint{2.087458in}{1.825474in}}{\pgfqpoint{2.079558in}{1.822201in}}{\pgfqpoint{2.073734in}{1.816377in}}%
\pgfpathcurveto{\pgfqpoint{2.067910in}{1.810553in}}{\pgfqpoint{2.064638in}{1.802653in}}{\pgfqpoint{2.064638in}{1.794417in}}%
\pgfpathcurveto{\pgfqpoint{2.064638in}{1.786181in}}{\pgfqpoint{2.067910in}{1.778281in}}{\pgfqpoint{2.073734in}{1.772457in}}%
\pgfpathcurveto{\pgfqpoint{2.079558in}{1.766633in}}{\pgfqpoint{2.087458in}{1.763361in}}{\pgfqpoint{2.095695in}{1.763361in}}%
\pgfpathclose%
\pgfusepath{stroke,fill}%
\end{pgfscope}%
\begin{pgfscope}%
\pgfpathrectangle{\pgfqpoint{0.556847in}{0.516222in}}{\pgfqpoint{1.722590in}{1.783528in}} %
\pgfusepath{clip}%
\pgfsetbuttcap%
\pgfsetroundjoin%
\definecolor{currentfill}{rgb}{0.298039,0.447059,0.690196}%
\pgfsetfillcolor{currentfill}%
\pgfsetlinewidth{0.240900pt}%
\definecolor{currentstroke}{rgb}{1.000000,1.000000,1.000000}%
\pgfsetstrokecolor{currentstroke}%
\pgfsetdash{}{0pt}%
\pgfpathmoveto{\pgfqpoint{1.332013in}{1.436381in}}%
\pgfpathcurveto{\pgfqpoint{1.340249in}{1.436381in}}{\pgfqpoint{1.348149in}{1.439653in}}{\pgfqpoint{1.353973in}{1.445477in}}%
\pgfpathcurveto{\pgfqpoint{1.359797in}{1.451301in}}{\pgfqpoint{1.363069in}{1.459201in}}{\pgfqpoint{1.363069in}{1.467437in}}%
\pgfpathcurveto{\pgfqpoint{1.363069in}{1.475673in}}{\pgfqpoint{1.359797in}{1.483573in}}{\pgfqpoint{1.353973in}{1.489397in}}%
\pgfpathcurveto{\pgfqpoint{1.348149in}{1.495221in}}{\pgfqpoint{1.340249in}{1.498494in}}{\pgfqpoint{1.332013in}{1.498494in}}%
\pgfpathcurveto{\pgfqpoint{1.323777in}{1.498494in}}{\pgfqpoint{1.315877in}{1.495221in}}{\pgfqpoint{1.310053in}{1.489397in}}%
\pgfpathcurveto{\pgfqpoint{1.304229in}{1.483573in}}{\pgfqpoint{1.300956in}{1.475673in}}{\pgfqpoint{1.300956in}{1.467437in}}%
\pgfpathcurveto{\pgfqpoint{1.300956in}{1.459201in}}{\pgfqpoint{1.304229in}{1.451301in}}{\pgfqpoint{1.310053in}{1.445477in}}%
\pgfpathcurveto{\pgfqpoint{1.315877in}{1.439653in}}{\pgfqpoint{1.323777in}{1.436381in}}{\pgfqpoint{1.332013in}{1.436381in}}%
\pgfpathclose%
\pgfusepath{stroke,fill}%
\end{pgfscope}%
\begin{pgfscope}%
\pgfpathrectangle{\pgfqpoint{0.556847in}{0.516222in}}{\pgfqpoint{1.722590in}{1.783528in}} %
\pgfusepath{clip}%
\pgfsetbuttcap%
\pgfsetroundjoin%
\definecolor{currentfill}{rgb}{0.298039,0.447059,0.690196}%
\pgfsetfillcolor{currentfill}%
\pgfsetlinewidth{0.240900pt}%
\definecolor{currentstroke}{rgb}{1.000000,1.000000,1.000000}%
\pgfsetstrokecolor{currentstroke}%
\pgfsetdash{}{0pt}%
\pgfpathmoveto{\pgfqpoint{1.507143in}{1.822812in}}%
\pgfpathcurveto{\pgfqpoint{1.515379in}{1.822812in}}{\pgfqpoint{1.523279in}{1.826084in}}{\pgfqpoint{1.529103in}{1.831908in}}%
\pgfpathcurveto{\pgfqpoint{1.534927in}{1.837732in}}{\pgfqpoint{1.538199in}{1.845632in}}{\pgfqpoint{1.538199in}{1.853868in}}%
\pgfpathcurveto{\pgfqpoint{1.538199in}{1.862104in}}{\pgfqpoint{1.534927in}{1.870004in}}{\pgfqpoint{1.529103in}{1.875828in}}%
\pgfpathcurveto{\pgfqpoint{1.523279in}{1.881652in}}{\pgfqpoint{1.515379in}{1.884925in}}{\pgfqpoint{1.507143in}{1.884925in}}%
\pgfpathcurveto{\pgfqpoint{1.498907in}{1.884925in}}{\pgfqpoint{1.491007in}{1.881652in}}{\pgfqpoint{1.485183in}{1.875828in}}%
\pgfpathcurveto{\pgfqpoint{1.479359in}{1.870004in}}{\pgfqpoint{1.476086in}{1.862104in}}{\pgfqpoint{1.476086in}{1.853868in}}%
\pgfpathcurveto{\pgfqpoint{1.476086in}{1.845632in}}{\pgfqpoint{1.479359in}{1.837732in}}{\pgfqpoint{1.485183in}{1.831908in}}%
\pgfpathcurveto{\pgfqpoint{1.491007in}{1.826084in}}{\pgfqpoint{1.498907in}{1.822812in}}{\pgfqpoint{1.507143in}{1.822812in}}%
\pgfpathclose%
\pgfusepath{stroke,fill}%
\end{pgfscope}%
\begin{pgfscope}%
\pgfpathrectangle{\pgfqpoint{0.556847in}{0.516222in}}{\pgfqpoint{1.722590in}{1.783528in}} %
\pgfusepath{clip}%
\pgfsetbuttcap%
\pgfsetroundjoin%
\definecolor{currentfill}{rgb}{0.298039,0.447059,0.690196}%
\pgfsetfillcolor{currentfill}%
\pgfsetlinewidth{0.240900pt}%
\definecolor{currentstroke}{rgb}{1.000000,1.000000,1.000000}%
\pgfsetstrokecolor{currentstroke}%
\pgfsetdash{}{0pt}%
\pgfpathmoveto{\pgfqpoint{1.771273in}{1.733635in}}%
\pgfpathcurveto{\pgfqpoint{1.779510in}{1.733635in}}{\pgfqpoint{1.787410in}{1.736907in}}{\pgfqpoint{1.793234in}{1.742731in}}%
\pgfpathcurveto{\pgfqpoint{1.799058in}{1.748555in}}{\pgfqpoint{1.802330in}{1.756455in}}{\pgfqpoint{1.802330in}{1.764692in}}%
\pgfpathcurveto{\pgfqpoint{1.802330in}{1.772928in}}{\pgfqpoint{1.799058in}{1.780828in}}{\pgfqpoint{1.793234in}{1.786652in}}%
\pgfpathcurveto{\pgfqpoint{1.787410in}{1.792476in}}{\pgfqpoint{1.779510in}{1.795748in}}{\pgfqpoint{1.771273in}{1.795748in}}%
\pgfpathcurveto{\pgfqpoint{1.763037in}{1.795748in}}{\pgfqpoint{1.755137in}{1.792476in}}{\pgfqpoint{1.749313in}{1.786652in}}%
\pgfpathcurveto{\pgfqpoint{1.743489in}{1.780828in}}{\pgfqpoint{1.740217in}{1.772928in}}{\pgfqpoint{1.740217in}{1.764692in}}%
\pgfpathcurveto{\pgfqpoint{1.740217in}{1.756455in}}{\pgfqpoint{1.743489in}{1.748555in}}{\pgfqpoint{1.749313in}{1.742731in}}%
\pgfpathcurveto{\pgfqpoint{1.755137in}{1.736907in}}{\pgfqpoint{1.763037in}{1.733635in}}{\pgfqpoint{1.771273in}{1.733635in}}%
\pgfpathclose%
\pgfusepath{stroke,fill}%
\end{pgfscope}%
\begin{pgfscope}%
\pgfpathrectangle{\pgfqpoint{0.556847in}{0.516222in}}{\pgfqpoint{1.722590in}{1.783528in}} %
\pgfusepath{clip}%
\pgfsetbuttcap%
\pgfsetroundjoin%
\definecolor{currentfill}{rgb}{0.298039,0.447059,0.690196}%
\pgfsetfillcolor{currentfill}%
\pgfsetlinewidth{0.240900pt}%
\definecolor{currentstroke}{rgb}{1.000000,1.000000,1.000000}%
\pgfsetstrokecolor{currentstroke}%
\pgfsetdash{}{0pt}%
\pgfpathmoveto{\pgfqpoint{1.208561in}{1.911988in}}%
\pgfpathcurveto{\pgfqpoint{1.216797in}{1.911988in}}{\pgfqpoint{1.224697in}{1.915260in}}{\pgfqpoint{1.230521in}{1.921084in}}%
\pgfpathcurveto{\pgfqpoint{1.236345in}{1.926908in}}{\pgfqpoint{1.239617in}{1.934808in}}{\pgfqpoint{1.239617in}{1.943044in}}%
\pgfpathcurveto{\pgfqpoint{1.239617in}{1.951281in}}{\pgfqpoint{1.236345in}{1.959181in}}{\pgfqpoint{1.230521in}{1.965005in}}%
\pgfpathcurveto{\pgfqpoint{1.224697in}{1.970829in}}{\pgfqpoint{1.216797in}{1.974101in}}{\pgfqpoint{1.208561in}{1.974101in}}%
\pgfpathcurveto{\pgfqpoint{1.200324in}{1.974101in}}{\pgfqpoint{1.192424in}{1.970829in}}{\pgfqpoint{1.186600in}{1.965005in}}%
\pgfpathcurveto{\pgfqpoint{1.180776in}{1.959181in}}{\pgfqpoint{1.177504in}{1.951281in}}{\pgfqpoint{1.177504in}{1.943044in}}%
\pgfpathcurveto{\pgfqpoint{1.177504in}{1.934808in}}{\pgfqpoint{1.180776in}{1.926908in}}{\pgfqpoint{1.186600in}{1.921084in}}%
\pgfpathcurveto{\pgfqpoint{1.192424in}{1.915260in}}{\pgfqpoint{1.200324in}{1.911988in}}{\pgfqpoint{1.208561in}{1.911988in}}%
\pgfpathclose%
\pgfusepath{stroke,fill}%
\end{pgfscope}%
\begin{pgfscope}%
\pgfpathrectangle{\pgfqpoint{0.556847in}{0.516222in}}{\pgfqpoint{1.722590in}{1.783528in}} %
\pgfusepath{clip}%
\pgfsetbuttcap%
\pgfsetroundjoin%
\definecolor{currentfill}{rgb}{0.298039,0.447059,0.690196}%
\pgfsetfillcolor{currentfill}%
\pgfsetlinewidth{0.240900pt}%
\definecolor{currentstroke}{rgb}{1.000000,1.000000,1.000000}%
\pgfsetstrokecolor{currentstroke}%
\pgfsetdash{}{0pt}%
\pgfpathmoveto{\pgfqpoint{1.329142in}{1.168851in}}%
\pgfpathcurveto{\pgfqpoint{1.337378in}{1.168851in}}{\pgfqpoint{1.345278in}{1.172124in}}{\pgfqpoint{1.351102in}{1.177948in}}%
\pgfpathcurveto{\pgfqpoint{1.356926in}{1.183772in}}{\pgfqpoint{1.360198in}{1.191672in}}{\pgfqpoint{1.360198in}{1.199908in}}%
\pgfpathcurveto{\pgfqpoint{1.360198in}{1.208144in}}{\pgfqpoint{1.356926in}{1.216044in}}{\pgfqpoint{1.351102in}{1.221868in}}%
\pgfpathcurveto{\pgfqpoint{1.345278in}{1.227692in}}{\pgfqpoint{1.337378in}{1.230964in}}{\pgfqpoint{1.329142in}{1.230964in}}%
\pgfpathcurveto{\pgfqpoint{1.320906in}{1.230964in}}{\pgfqpoint{1.313006in}{1.227692in}}{\pgfqpoint{1.307182in}{1.221868in}}%
\pgfpathcurveto{\pgfqpoint{1.301358in}{1.216044in}}{\pgfqpoint{1.298085in}{1.208144in}}{\pgfqpoint{1.298085in}{1.199908in}}%
\pgfpathcurveto{\pgfqpoint{1.298085in}{1.191672in}}{\pgfqpoint{1.301358in}{1.183772in}}{\pgfqpoint{1.307182in}{1.177948in}}%
\pgfpathcurveto{\pgfqpoint{1.313006in}{1.172124in}}{\pgfqpoint{1.320906in}{1.168851in}}{\pgfqpoint{1.329142in}{1.168851in}}%
\pgfpathclose%
\pgfusepath{stroke,fill}%
\end{pgfscope}%
\begin{pgfscope}%
\pgfpathrectangle{\pgfqpoint{0.556847in}{0.516222in}}{\pgfqpoint{1.722590in}{1.783528in}} %
\pgfusepath{clip}%
\pgfsetbuttcap%
\pgfsetroundjoin%
\definecolor{currentfill}{rgb}{0.298039,0.447059,0.690196}%
\pgfsetfillcolor{currentfill}%
\pgfsetlinewidth{0.240900pt}%
\definecolor{currentstroke}{rgb}{1.000000,1.000000,1.000000}%
\pgfsetstrokecolor{currentstroke}%
\pgfsetdash{}{0pt}%
\pgfpathmoveto{\pgfqpoint{1.688015in}{1.674184in}}%
\pgfpathcurveto{\pgfqpoint{1.696251in}{1.674184in}}{\pgfqpoint{1.704151in}{1.677457in}}{\pgfqpoint{1.709975in}{1.683280in}}%
\pgfpathcurveto{\pgfqpoint{1.715799in}{1.689104in}}{\pgfqpoint{1.719071in}{1.697004in}}{\pgfqpoint{1.719071in}{1.705241in}}%
\pgfpathcurveto{\pgfqpoint{1.719071in}{1.713477in}}{\pgfqpoint{1.715799in}{1.721377in}}{\pgfqpoint{1.709975in}{1.727201in}}%
\pgfpathcurveto{\pgfqpoint{1.704151in}{1.733025in}}{\pgfqpoint{1.696251in}{1.736297in}}{\pgfqpoint{1.688015in}{1.736297in}}%
\pgfpathcurveto{\pgfqpoint{1.679779in}{1.736297in}}{\pgfqpoint{1.671879in}{1.733025in}}{\pgfqpoint{1.666055in}{1.727201in}}%
\pgfpathcurveto{\pgfqpoint{1.660231in}{1.721377in}}{\pgfqpoint{1.656958in}{1.713477in}}{\pgfqpoint{1.656958in}{1.705241in}}%
\pgfpathcurveto{\pgfqpoint{1.656958in}{1.697004in}}{\pgfqpoint{1.660231in}{1.689104in}}{\pgfqpoint{1.666055in}{1.683280in}}%
\pgfpathcurveto{\pgfqpoint{1.671879in}{1.677457in}}{\pgfqpoint{1.679779in}{1.674184in}}{\pgfqpoint{1.688015in}{1.674184in}}%
\pgfpathclose%
\pgfusepath{stroke,fill}%
\end{pgfscope}%
\begin{pgfscope}%
\pgfpathrectangle{\pgfqpoint{0.556847in}{0.516222in}}{\pgfqpoint{1.722590in}{1.783528in}} %
\pgfusepath{clip}%
\pgfsetbuttcap%
\pgfsetroundjoin%
\definecolor{currentfill}{rgb}{0.298039,0.447059,0.690196}%
\pgfsetfillcolor{currentfill}%
\pgfsetlinewidth{0.240900pt}%
\definecolor{currentstroke}{rgb}{1.000000,1.000000,1.000000}%
\pgfsetstrokecolor{currentstroke}%
\pgfsetdash{}{0pt}%
\pgfpathmoveto{\pgfqpoint{2.035404in}{1.585008in}}%
\pgfpathcurveto{\pgfqpoint{2.043640in}{1.585008in}}{\pgfqpoint{2.051540in}{1.588280in}}{\pgfqpoint{2.057364in}{1.594104in}}%
\pgfpathcurveto{\pgfqpoint{2.063188in}{1.599928in}}{\pgfqpoint{2.066460in}{1.607828in}}{\pgfqpoint{2.066460in}{1.616064in}}%
\pgfpathcurveto{\pgfqpoint{2.066460in}{1.624301in}}{\pgfqpoint{2.063188in}{1.632201in}}{\pgfqpoint{2.057364in}{1.638025in}}%
\pgfpathcurveto{\pgfqpoint{2.051540in}{1.643849in}}{\pgfqpoint{2.043640in}{1.647121in}}{\pgfqpoint{2.035404in}{1.647121in}}%
\pgfpathcurveto{\pgfqpoint{2.027168in}{1.647121in}}{\pgfqpoint{2.019268in}{1.643849in}}{\pgfqpoint{2.013444in}{1.638025in}}%
\pgfpathcurveto{\pgfqpoint{2.007620in}{1.632201in}}{\pgfqpoint{2.004347in}{1.624301in}}{\pgfqpoint{2.004347in}{1.616064in}}%
\pgfpathcurveto{\pgfqpoint{2.004347in}{1.607828in}}{\pgfqpoint{2.007620in}{1.599928in}}{\pgfqpoint{2.013444in}{1.594104in}}%
\pgfpathcurveto{\pgfqpoint{2.019268in}{1.588280in}}{\pgfqpoint{2.027168in}{1.585008in}}{\pgfqpoint{2.035404in}{1.585008in}}%
\pgfpathclose%
\pgfusepath{stroke,fill}%
\end{pgfscope}%
\begin{pgfscope}%
\pgfpathrectangle{\pgfqpoint{0.556847in}{0.516222in}}{\pgfqpoint{1.722590in}{1.783528in}} %
\pgfusepath{clip}%
\pgfsetbuttcap%
\pgfsetroundjoin%
\definecolor{currentfill}{rgb}{0.298039,0.447059,0.690196}%
\pgfsetfillcolor{currentfill}%
\pgfsetlinewidth{0.240900pt}%
\definecolor{currentstroke}{rgb}{1.000000,1.000000,1.000000}%
\pgfsetstrokecolor{currentstroke}%
\pgfsetdash{}{0pt}%
\pgfpathmoveto{\pgfqpoint{2.087082in}{1.436381in}}%
\pgfpathcurveto{\pgfqpoint{2.095318in}{1.436381in}}{\pgfqpoint{2.103218in}{1.439653in}}{\pgfqpoint{2.109042in}{1.445477in}}%
\pgfpathcurveto{\pgfqpoint{2.114866in}{1.451301in}}{\pgfqpoint{2.118138in}{1.459201in}}{\pgfqpoint{2.118138in}{1.467437in}}%
\pgfpathcurveto{\pgfqpoint{2.118138in}{1.475673in}}{\pgfqpoint{2.114866in}{1.483573in}}{\pgfqpoint{2.109042in}{1.489397in}}%
\pgfpathcurveto{\pgfqpoint{2.103218in}{1.495221in}}{\pgfqpoint{2.095318in}{1.498494in}}{\pgfqpoint{2.087082in}{1.498494in}}%
\pgfpathcurveto{\pgfqpoint{2.078845in}{1.498494in}}{\pgfqpoint{2.070945in}{1.495221in}}{\pgfqpoint{2.065121in}{1.489397in}}%
\pgfpathcurveto{\pgfqpoint{2.059297in}{1.483573in}}{\pgfqpoint{2.056025in}{1.475673in}}{\pgfqpoint{2.056025in}{1.467437in}}%
\pgfpathcurveto{\pgfqpoint{2.056025in}{1.459201in}}{\pgfqpoint{2.059297in}{1.451301in}}{\pgfqpoint{2.065121in}{1.445477in}}%
\pgfpathcurveto{\pgfqpoint{2.070945in}{1.439653in}}{\pgfqpoint{2.078845in}{1.436381in}}{\pgfqpoint{2.087082in}{1.436381in}}%
\pgfpathclose%
\pgfusepath{stroke,fill}%
\end{pgfscope}%
\begin{pgfscope}%
\pgfpathrectangle{\pgfqpoint{0.556847in}{0.516222in}}{\pgfqpoint{1.722590in}{1.783528in}} %
\pgfusepath{clip}%
\pgfsetbuttcap%
\pgfsetroundjoin%
\definecolor{currentfill}{rgb}{0.298039,0.447059,0.690196}%
\pgfsetfillcolor{currentfill}%
\pgfsetlinewidth{0.240900pt}%
\definecolor{currentstroke}{rgb}{1.000000,1.000000,1.000000}%
\pgfsetstrokecolor{currentstroke}%
\pgfsetdash{}{0pt}%
\pgfpathmoveto{\pgfqpoint{0.889881in}{0.633793in}}%
\pgfpathcurveto{\pgfqpoint{0.898118in}{0.633793in}}{\pgfqpoint{0.906018in}{0.637065in}}{\pgfqpoint{0.911842in}{0.642889in}}%
\pgfpathcurveto{\pgfqpoint{0.917666in}{0.648713in}}{\pgfqpoint{0.920938in}{0.656613in}}{\pgfqpoint{0.920938in}{0.664850in}}%
\pgfpathcurveto{\pgfqpoint{0.920938in}{0.673086in}}{\pgfqpoint{0.917666in}{0.680986in}}{\pgfqpoint{0.911842in}{0.686810in}}%
\pgfpathcurveto{\pgfqpoint{0.906018in}{0.692634in}}{\pgfqpoint{0.898118in}{0.695906in}}{\pgfqpoint{0.889881in}{0.695906in}}%
\pgfpathcurveto{\pgfqpoint{0.881645in}{0.695906in}}{\pgfqpoint{0.873745in}{0.692634in}}{\pgfqpoint{0.867921in}{0.686810in}}%
\pgfpathcurveto{\pgfqpoint{0.862097in}{0.680986in}}{\pgfqpoint{0.858825in}{0.673086in}}{\pgfqpoint{0.858825in}{0.664850in}}%
\pgfpathcurveto{\pgfqpoint{0.858825in}{0.656613in}}{\pgfqpoint{0.862097in}{0.648713in}}{\pgfqpoint{0.867921in}{0.642889in}}%
\pgfpathcurveto{\pgfqpoint{0.873745in}{0.637065in}}{\pgfqpoint{0.881645in}{0.633793in}}{\pgfqpoint{0.889881in}{0.633793in}}%
\pgfpathclose%
\pgfusepath{stroke,fill}%
\end{pgfscope}%
\begin{pgfscope}%
\pgfpathrectangle{\pgfqpoint{0.556847in}{0.516222in}}{\pgfqpoint{1.722590in}{1.783528in}} %
\pgfusepath{clip}%
\pgfsetbuttcap%
\pgfsetroundjoin%
\definecolor{currentfill}{rgb}{0.298039,0.447059,0.690196}%
\pgfsetfillcolor{currentfill}%
\pgfsetlinewidth{0.240900pt}%
\definecolor{currentstroke}{rgb}{1.000000,1.000000,1.000000}%
\pgfsetstrokecolor{currentstroke}%
\pgfsetdash{}{0pt}%
\pgfpathmoveto{\pgfqpoint{1.857403in}{1.793086in}}%
\pgfpathcurveto{\pgfqpoint{1.865639in}{1.793086in}}{\pgfqpoint{1.873539in}{1.796358in}}{\pgfqpoint{1.879363in}{1.802182in}}%
\pgfpathcurveto{\pgfqpoint{1.885187in}{1.808006in}}{\pgfqpoint{1.888459in}{1.815906in}}{\pgfqpoint{1.888459in}{1.824143in}}%
\pgfpathcurveto{\pgfqpoint{1.888459in}{1.832379in}}{\pgfqpoint{1.885187in}{1.840279in}}{\pgfqpoint{1.879363in}{1.846103in}}%
\pgfpathcurveto{\pgfqpoint{1.873539in}{1.851927in}}{\pgfqpoint{1.865639in}{1.855199in}}{\pgfqpoint{1.857403in}{1.855199in}}%
\pgfpathcurveto{\pgfqpoint{1.849167in}{1.855199in}}{\pgfqpoint{1.841267in}{1.851927in}}{\pgfqpoint{1.835443in}{1.846103in}}%
\pgfpathcurveto{\pgfqpoint{1.829619in}{1.840279in}}{\pgfqpoint{1.826346in}{1.832379in}}{\pgfqpoint{1.826346in}{1.824143in}}%
\pgfpathcurveto{\pgfqpoint{1.826346in}{1.815906in}}{\pgfqpoint{1.829619in}{1.808006in}}{\pgfqpoint{1.835443in}{1.802182in}}%
\pgfpathcurveto{\pgfqpoint{1.841267in}{1.796358in}}{\pgfqpoint{1.849167in}{1.793086in}}{\pgfqpoint{1.857403in}{1.793086in}}%
\pgfpathclose%
\pgfusepath{stroke,fill}%
\end{pgfscope}%
\begin{pgfscope}%
\pgfpathrectangle{\pgfqpoint{0.556847in}{0.516222in}}{\pgfqpoint{1.722590in}{1.783528in}} %
\pgfusepath{clip}%
\pgfsetbuttcap%
\pgfsetroundjoin%
\definecolor{currentfill}{rgb}{0.298039,0.447059,0.690196}%
\pgfsetfillcolor{currentfill}%
\pgfsetlinewidth{0.240900pt}%
\definecolor{currentstroke}{rgb}{1.000000,1.000000,1.000000}%
\pgfsetstrokecolor{currentstroke}%
\pgfsetdash{}{0pt}%
\pgfpathmoveto{\pgfqpoint{1.142528in}{1.376930in}}%
\pgfpathcurveto{\pgfqpoint{1.150764in}{1.376930in}}{\pgfqpoint{1.158664in}{1.380202in}}{\pgfqpoint{1.164488in}{1.386026in}}%
\pgfpathcurveto{\pgfqpoint{1.170312in}{1.391850in}}{\pgfqpoint{1.173584in}{1.399750in}}{\pgfqpoint{1.173584in}{1.407986in}}%
\pgfpathcurveto{\pgfqpoint{1.173584in}{1.416222in}}{\pgfqpoint{1.170312in}{1.424122in}}{\pgfqpoint{1.164488in}{1.429946in}}%
\pgfpathcurveto{\pgfqpoint{1.158664in}{1.435770in}}{\pgfqpoint{1.150764in}{1.439043in}}{\pgfqpoint{1.142528in}{1.439043in}}%
\pgfpathcurveto{\pgfqpoint{1.134292in}{1.439043in}}{\pgfqpoint{1.126392in}{1.435770in}}{\pgfqpoint{1.120568in}{1.429946in}}%
\pgfpathcurveto{\pgfqpoint{1.114744in}{1.424122in}}{\pgfqpoint{1.111471in}{1.416222in}}{\pgfqpoint{1.111471in}{1.407986in}}%
\pgfpathcurveto{\pgfqpoint{1.111471in}{1.399750in}}{\pgfqpoint{1.114744in}{1.391850in}}{\pgfqpoint{1.120568in}{1.386026in}}%
\pgfpathcurveto{\pgfqpoint{1.126392in}{1.380202in}}{\pgfqpoint{1.134292in}{1.376930in}}{\pgfqpoint{1.142528in}{1.376930in}}%
\pgfpathclose%
\pgfusepath{stroke,fill}%
\end{pgfscope}%
\begin{pgfscope}%
\pgfpathrectangle{\pgfqpoint{0.556847in}{0.516222in}}{\pgfqpoint{1.722590in}{1.783528in}} %
\pgfusepath{clip}%
\pgfsetbuttcap%
\pgfsetroundjoin%
\definecolor{currentfill}{rgb}{0.298039,0.447059,0.690196}%
\pgfsetfillcolor{currentfill}%
\pgfsetlinewidth{0.240900pt}%
\definecolor{currentstroke}{rgb}{1.000000,1.000000,1.000000}%
\pgfsetstrokecolor{currentstroke}%
\pgfsetdash{}{0pt}%
\pgfpathmoveto{\pgfqpoint{1.033431in}{0.693244in}}%
\pgfpathcurveto{\pgfqpoint{1.041667in}{0.693244in}}{\pgfqpoint{1.049567in}{0.696516in}}{\pgfqpoint{1.055391in}{0.702340in}}%
\pgfpathcurveto{\pgfqpoint{1.061215in}{0.708164in}}{\pgfqpoint{1.064487in}{0.716064in}}{\pgfqpoint{1.064487in}{0.724300in}}%
\pgfpathcurveto{\pgfqpoint{1.064487in}{0.732537in}}{\pgfqpoint{1.061215in}{0.740437in}}{\pgfqpoint{1.055391in}{0.746261in}}%
\pgfpathcurveto{\pgfqpoint{1.049567in}{0.752085in}}{\pgfqpoint{1.041667in}{0.755357in}}{\pgfqpoint{1.033431in}{0.755357in}}%
\pgfpathcurveto{\pgfqpoint{1.025194in}{0.755357in}}{\pgfqpoint{1.017294in}{0.752085in}}{\pgfqpoint{1.011470in}{0.746261in}}%
\pgfpathcurveto{\pgfqpoint{1.005646in}{0.740437in}}{\pgfqpoint{1.002374in}{0.732537in}}{\pgfqpoint{1.002374in}{0.724300in}}%
\pgfpathcurveto{\pgfqpoint{1.002374in}{0.716064in}}{\pgfqpoint{1.005646in}{0.708164in}}{\pgfqpoint{1.011470in}{0.702340in}}%
\pgfpathcurveto{\pgfqpoint{1.017294in}{0.696516in}}{\pgfqpoint{1.025194in}{0.693244in}}{\pgfqpoint{1.033431in}{0.693244in}}%
\pgfpathclose%
\pgfusepath{stroke,fill}%
\end{pgfscope}%
\begin{pgfscope}%
\pgfpathrectangle{\pgfqpoint{0.556847in}{0.516222in}}{\pgfqpoint{1.722590in}{1.783528in}} %
\pgfusepath{clip}%
\pgfsetbuttcap%
\pgfsetroundjoin%
\definecolor{currentfill}{rgb}{0.298039,0.447059,0.690196}%
\pgfsetfillcolor{currentfill}%
\pgfsetlinewidth{0.240900pt}%
\definecolor{currentstroke}{rgb}{1.000000,1.000000,1.000000}%
\pgfsetstrokecolor{currentstroke}%
\pgfsetdash{}{0pt}%
\pgfpathmoveto{\pgfqpoint{1.693757in}{2.001164in}}%
\pgfpathcurveto{\pgfqpoint{1.701993in}{2.001164in}}{\pgfqpoint{1.709893in}{2.004437in}}{\pgfqpoint{1.715717in}{2.010261in}}%
\pgfpathcurveto{\pgfqpoint{1.721541in}{2.016085in}}{\pgfqpoint{1.724813in}{2.023985in}}{\pgfqpoint{1.724813in}{2.032221in}}%
\pgfpathcurveto{\pgfqpoint{1.724813in}{2.040457in}}{\pgfqpoint{1.721541in}{2.048357in}}{\pgfqpoint{1.715717in}{2.054181in}}%
\pgfpathcurveto{\pgfqpoint{1.709893in}{2.060005in}}{\pgfqpoint{1.701993in}{2.063277in}}{\pgfqpoint{1.693757in}{2.063277in}}%
\pgfpathcurveto{\pgfqpoint{1.685521in}{2.063277in}}{\pgfqpoint{1.677620in}{2.060005in}}{\pgfqpoint{1.671797in}{2.054181in}}%
\pgfpathcurveto{\pgfqpoint{1.665973in}{2.048357in}}{\pgfqpoint{1.662700in}{2.040457in}}{\pgfqpoint{1.662700in}{2.032221in}}%
\pgfpathcurveto{\pgfqpoint{1.662700in}{2.023985in}}{\pgfqpoint{1.665973in}{2.016085in}}{\pgfqpoint{1.671797in}{2.010261in}}%
\pgfpathcurveto{\pgfqpoint{1.677620in}{2.004437in}}{\pgfqpoint{1.685521in}{2.001164in}}{\pgfqpoint{1.693757in}{2.001164in}}%
\pgfpathclose%
\pgfusepath{stroke,fill}%
\end{pgfscope}%
\begin{pgfscope}%
\pgfpathrectangle{\pgfqpoint{0.556847in}{0.516222in}}{\pgfqpoint{1.722590in}{1.783528in}} %
\pgfusepath{clip}%
\pgfsetbuttcap%
\pgfsetroundjoin%
\definecolor{currentfill}{rgb}{0.298039,0.447059,0.690196}%
\pgfsetfillcolor{currentfill}%
\pgfsetlinewidth{0.240900pt}%
\definecolor{currentstroke}{rgb}{1.000000,1.000000,1.000000}%
\pgfsetstrokecolor{currentstroke}%
\pgfsetdash{}{0pt}%
\pgfpathmoveto{\pgfqpoint{1.774144in}{1.466106in}}%
\pgfpathcurveto{\pgfqpoint{1.782381in}{1.466106in}}{\pgfqpoint{1.790281in}{1.469378in}}{\pgfqpoint{1.796105in}{1.475202in}}%
\pgfpathcurveto{\pgfqpoint{1.801929in}{1.481026in}}{\pgfqpoint{1.805201in}{1.488926in}}{\pgfqpoint{1.805201in}{1.497163in}}%
\pgfpathcurveto{\pgfqpoint{1.805201in}{1.505399in}}{\pgfqpoint{1.801929in}{1.513299in}}{\pgfqpoint{1.796105in}{1.519123in}}%
\pgfpathcurveto{\pgfqpoint{1.790281in}{1.524947in}}{\pgfqpoint{1.782381in}{1.528219in}}{\pgfqpoint{1.774144in}{1.528219in}}%
\pgfpathcurveto{\pgfqpoint{1.765908in}{1.528219in}}{\pgfqpoint{1.758008in}{1.524947in}}{\pgfqpoint{1.752184in}{1.519123in}}%
\pgfpathcurveto{\pgfqpoint{1.746360in}{1.513299in}}{\pgfqpoint{1.743088in}{1.505399in}}{\pgfqpoint{1.743088in}{1.497163in}}%
\pgfpathcurveto{\pgfqpoint{1.743088in}{1.488926in}}{\pgfqpoint{1.746360in}{1.481026in}}{\pgfqpoint{1.752184in}{1.475202in}}%
\pgfpathcurveto{\pgfqpoint{1.758008in}{1.469378in}}{\pgfqpoint{1.765908in}{1.466106in}}{\pgfqpoint{1.774144in}{1.466106in}}%
\pgfpathclose%
\pgfusepath{stroke,fill}%
\end{pgfscope}%
\begin{pgfscope}%
\pgfpathrectangle{\pgfqpoint{0.556847in}{0.516222in}}{\pgfqpoint{1.722590in}{1.783528in}} %
\pgfusepath{clip}%
\pgfsetbuttcap%
\pgfsetroundjoin%
\definecolor{currentfill}{rgb}{0.298039,0.447059,0.690196}%
\pgfsetfillcolor{currentfill}%
\pgfsetlinewidth{0.240900pt}%
\definecolor{currentstroke}{rgb}{1.000000,1.000000,1.000000}%
\pgfsetstrokecolor{currentstroke}%
\pgfsetdash{}{0pt}%
\pgfpathmoveto{\pgfqpoint{1.619111in}{1.258028in}}%
\pgfpathcurveto{\pgfqpoint{1.627348in}{1.258028in}}{\pgfqpoint{1.635248in}{1.261300in}}{\pgfqpoint{1.641071in}{1.267124in}}%
\pgfpathcurveto{\pgfqpoint{1.646895in}{1.272948in}}{\pgfqpoint{1.650168in}{1.280848in}}{\pgfqpoint{1.650168in}{1.289084in}}%
\pgfpathcurveto{\pgfqpoint{1.650168in}{1.297321in}}{\pgfqpoint{1.646895in}{1.305221in}}{\pgfqpoint{1.641071in}{1.311045in}}%
\pgfpathcurveto{\pgfqpoint{1.635248in}{1.316868in}}{\pgfqpoint{1.627348in}{1.320141in}}{\pgfqpoint{1.619111in}{1.320141in}}%
\pgfpathcurveto{\pgfqpoint{1.610875in}{1.320141in}}{\pgfqpoint{1.602975in}{1.316868in}}{\pgfqpoint{1.597151in}{1.311045in}}%
\pgfpathcurveto{\pgfqpoint{1.591327in}{1.305221in}}{\pgfqpoint{1.588055in}{1.297321in}}{\pgfqpoint{1.588055in}{1.289084in}}%
\pgfpathcurveto{\pgfqpoint{1.588055in}{1.280848in}}{\pgfqpoint{1.591327in}{1.272948in}}{\pgfqpoint{1.597151in}{1.267124in}}%
\pgfpathcurveto{\pgfqpoint{1.602975in}{1.261300in}}{\pgfqpoint{1.610875in}{1.258028in}}{\pgfqpoint{1.619111in}{1.258028in}}%
\pgfpathclose%
\pgfusepath{stroke,fill}%
\end{pgfscope}%
\begin{pgfscope}%
\pgfpathrectangle{\pgfqpoint{0.556847in}{0.516222in}}{\pgfqpoint{1.722590in}{1.783528in}} %
\pgfusepath{clip}%
\pgfsetbuttcap%
\pgfsetroundjoin%
\definecolor{currentfill}{rgb}{0.298039,0.447059,0.690196}%
\pgfsetfillcolor{currentfill}%
\pgfsetlinewidth{0.240900pt}%
\definecolor{currentstroke}{rgb}{1.000000,1.000000,1.000000}%
\pgfsetstrokecolor{currentstroke}%
\pgfsetdash{}{0pt}%
\pgfpathmoveto{\pgfqpoint{0.648719in}{0.633793in}}%
\pgfpathcurveto{\pgfqpoint{0.656955in}{0.633793in}}{\pgfqpoint{0.664855in}{0.637065in}}{\pgfqpoint{0.670679in}{0.642889in}}%
\pgfpathcurveto{\pgfqpoint{0.676503in}{0.648713in}}{\pgfqpoint{0.679775in}{0.656613in}}{\pgfqpoint{0.679775in}{0.664850in}}%
\pgfpathcurveto{\pgfqpoint{0.679775in}{0.673086in}}{\pgfqpoint{0.676503in}{0.680986in}}{\pgfqpoint{0.670679in}{0.686810in}}%
\pgfpathcurveto{\pgfqpoint{0.664855in}{0.692634in}}{\pgfqpoint{0.656955in}{0.695906in}}{\pgfqpoint{0.648719in}{0.695906in}}%
\pgfpathcurveto{\pgfqpoint{0.640482in}{0.695906in}}{\pgfqpoint{0.632582in}{0.692634in}}{\pgfqpoint{0.626758in}{0.686810in}}%
\pgfpathcurveto{\pgfqpoint{0.620935in}{0.680986in}}{\pgfqpoint{0.617662in}{0.673086in}}{\pgfqpoint{0.617662in}{0.664850in}}%
\pgfpathcurveto{\pgfqpoint{0.617662in}{0.656613in}}{\pgfqpoint{0.620935in}{0.648713in}}{\pgfqpoint{0.626758in}{0.642889in}}%
\pgfpathcurveto{\pgfqpoint{0.632582in}{0.637065in}}{\pgfqpoint{0.640482in}{0.633793in}}{\pgfqpoint{0.648719in}{0.633793in}}%
\pgfpathclose%
\pgfusepath{stroke,fill}%
\end{pgfscope}%
\begin{pgfscope}%
\pgfpathrectangle{\pgfqpoint{0.556847in}{0.516222in}}{\pgfqpoint{1.722590in}{1.783528in}} %
\pgfusepath{clip}%
\pgfsetbuttcap%
\pgfsetroundjoin%
\definecolor{currentfill}{rgb}{0.298039,0.447059,0.690196}%
\pgfsetfillcolor{currentfill}%
\pgfsetlinewidth{0.240900pt}%
\definecolor{currentstroke}{rgb}{1.000000,1.000000,1.000000}%
\pgfsetstrokecolor{currentstroke}%
\pgfsetdash{}{0pt}%
\pgfpathmoveto{\pgfqpoint{1.297561in}{1.049950in}}%
\pgfpathcurveto{\pgfqpoint{1.305797in}{1.049950in}}{\pgfqpoint{1.313697in}{1.053222in}}{\pgfqpoint{1.319521in}{1.059046in}}%
\pgfpathcurveto{\pgfqpoint{1.325345in}{1.064870in}}{\pgfqpoint{1.328618in}{1.072770in}}{\pgfqpoint{1.328618in}{1.081006in}}%
\pgfpathcurveto{\pgfqpoint{1.328618in}{1.089242in}}{\pgfqpoint{1.325345in}{1.097142in}}{\pgfqpoint{1.319521in}{1.102966in}}%
\pgfpathcurveto{\pgfqpoint{1.313697in}{1.108790in}}{\pgfqpoint{1.305797in}{1.112063in}}{\pgfqpoint{1.297561in}{1.112063in}}%
\pgfpathcurveto{\pgfqpoint{1.289325in}{1.112063in}}{\pgfqpoint{1.281425in}{1.108790in}}{\pgfqpoint{1.275601in}{1.102966in}}%
\pgfpathcurveto{\pgfqpoint{1.269777in}{1.097142in}}{\pgfqpoint{1.266505in}{1.089242in}}{\pgfqpoint{1.266505in}{1.081006in}}%
\pgfpathcurveto{\pgfqpoint{1.266505in}{1.072770in}}{\pgfqpoint{1.269777in}{1.064870in}}{\pgfqpoint{1.275601in}{1.059046in}}%
\pgfpathcurveto{\pgfqpoint{1.281425in}{1.053222in}}{\pgfqpoint{1.289325in}{1.049950in}}{\pgfqpoint{1.297561in}{1.049950in}}%
\pgfpathclose%
\pgfusepath{stroke,fill}%
\end{pgfscope}%
\begin{pgfscope}%
\pgfpathrectangle{\pgfqpoint{0.556847in}{0.516222in}}{\pgfqpoint{1.722590in}{1.783528in}} %
\pgfusepath{clip}%
\pgfsetbuttcap%
\pgfsetroundjoin%
\definecolor{currentfill}{rgb}{0.298039,0.447059,0.690196}%
\pgfsetfillcolor{currentfill}%
\pgfsetlinewidth{0.240900pt}%
\definecolor{currentstroke}{rgb}{1.000000,1.000000,1.000000}%
\pgfsetstrokecolor{currentstroke}%
\pgfsetdash{}{0pt}%
\pgfpathmoveto{\pgfqpoint{1.871758in}{1.139126in}}%
\pgfpathcurveto{\pgfqpoint{1.879994in}{1.139126in}}{\pgfqpoint{1.887894in}{1.142398in}}{\pgfqpoint{1.893718in}{1.148222in}}%
\pgfpathcurveto{\pgfqpoint{1.899542in}{1.154046in}}{\pgfqpoint{1.902814in}{1.161946in}}{\pgfqpoint{1.902814in}{1.170182in}}%
\pgfpathcurveto{\pgfqpoint{1.902814in}{1.178419in}}{\pgfqpoint{1.899542in}{1.186319in}}{\pgfqpoint{1.893718in}{1.192143in}}%
\pgfpathcurveto{\pgfqpoint{1.887894in}{1.197967in}}{\pgfqpoint{1.879994in}{1.201239in}}{\pgfqpoint{1.871758in}{1.201239in}}%
\pgfpathcurveto{\pgfqpoint{1.863522in}{1.201239in}}{\pgfqpoint{1.855621in}{1.197967in}}{\pgfqpoint{1.849798in}{1.192143in}}%
\pgfpathcurveto{\pgfqpoint{1.843974in}{1.186319in}}{\pgfqpoint{1.840701in}{1.178419in}}{\pgfqpoint{1.840701in}{1.170182in}}%
\pgfpathcurveto{\pgfqpoint{1.840701in}{1.161946in}}{\pgfqpoint{1.843974in}{1.154046in}}{\pgfqpoint{1.849798in}{1.148222in}}%
\pgfpathcurveto{\pgfqpoint{1.855621in}{1.142398in}}{\pgfqpoint{1.863522in}{1.139126in}}{\pgfqpoint{1.871758in}{1.139126in}}%
\pgfpathclose%
\pgfusepath{stroke,fill}%
\end{pgfscope}%
\begin{pgfscope}%
\pgfpathrectangle{\pgfqpoint{0.556847in}{0.516222in}}{\pgfqpoint{1.722590in}{1.783528in}} %
\pgfusepath{clip}%
\pgfsetbuttcap%
\pgfsetroundjoin%
\definecolor{currentfill}{rgb}{0.298039,0.447059,0.690196}%
\pgfsetfillcolor{currentfill}%
\pgfsetlinewidth{0.240900pt}%
\definecolor{currentstroke}{rgb}{1.000000,1.000000,1.000000}%
\pgfsetstrokecolor{currentstroke}%
\pgfsetdash{}{0pt}%
\pgfpathmoveto{\pgfqpoint{0.869784in}{1.495831in}}%
\pgfpathcurveto{\pgfqpoint{0.878021in}{1.495831in}}{\pgfqpoint{0.885921in}{1.499104in}}{\pgfqpoint{0.891745in}{1.504928in}}%
\pgfpathcurveto{\pgfqpoint{0.897569in}{1.510752in}}{\pgfqpoint{0.900841in}{1.518652in}}{\pgfqpoint{0.900841in}{1.526888in}}%
\pgfpathcurveto{\pgfqpoint{0.900841in}{1.535124in}}{\pgfqpoint{0.897569in}{1.543024in}}{\pgfqpoint{0.891745in}{1.548848in}}%
\pgfpathcurveto{\pgfqpoint{0.885921in}{1.554672in}}{\pgfqpoint{0.878021in}{1.557944in}}{\pgfqpoint{0.869784in}{1.557944in}}%
\pgfpathcurveto{\pgfqpoint{0.861548in}{1.557944in}}{\pgfqpoint{0.853648in}{1.554672in}}{\pgfqpoint{0.847824in}{1.548848in}}%
\pgfpathcurveto{\pgfqpoint{0.842000in}{1.543024in}}{\pgfqpoint{0.838728in}{1.535124in}}{\pgfqpoint{0.838728in}{1.526888in}}%
\pgfpathcurveto{\pgfqpoint{0.838728in}{1.518652in}}{\pgfqpoint{0.842000in}{1.510752in}}{\pgfqpoint{0.847824in}{1.504928in}}%
\pgfpathcurveto{\pgfqpoint{0.853648in}{1.499104in}}{\pgfqpoint{0.861548in}{1.495831in}}{\pgfqpoint{0.869784in}{1.495831in}}%
\pgfpathclose%
\pgfusepath{stroke,fill}%
\end{pgfscope}%
\begin{pgfscope}%
\pgfpathrectangle{\pgfqpoint{0.556847in}{0.516222in}}{\pgfqpoint{1.722590in}{1.783528in}} %
\pgfusepath{clip}%
\pgfsetbuttcap%
\pgfsetroundjoin%
\definecolor{currentfill}{rgb}{0.298039,0.447059,0.690196}%
\pgfsetfillcolor{currentfill}%
\pgfsetlinewidth{0.240900pt}%
\definecolor{currentstroke}{rgb}{1.000000,1.000000,1.000000}%
\pgfsetstrokecolor{currentstroke}%
\pgfsetdash{}{0pt}%
\pgfpathmoveto{\pgfqpoint{0.731977in}{0.901322in}}%
\pgfpathcurveto{\pgfqpoint{0.740214in}{0.901322in}}{\pgfqpoint{0.748114in}{0.904595in}}{\pgfqpoint{0.753937in}{0.910418in}}%
\pgfpathcurveto{\pgfqpoint{0.759761in}{0.916242in}}{\pgfqpoint{0.763034in}{0.924142in}}{\pgfqpoint{0.763034in}{0.932379in}}%
\pgfpathcurveto{\pgfqpoint{0.763034in}{0.940615in}}{\pgfqpoint{0.759761in}{0.948515in}}{\pgfqpoint{0.753937in}{0.954339in}}%
\pgfpathcurveto{\pgfqpoint{0.748114in}{0.960163in}}{\pgfqpoint{0.740214in}{0.963435in}}{\pgfqpoint{0.731977in}{0.963435in}}%
\pgfpathcurveto{\pgfqpoint{0.723741in}{0.963435in}}{\pgfqpoint{0.715841in}{0.960163in}}{\pgfqpoint{0.710017in}{0.954339in}}%
\pgfpathcurveto{\pgfqpoint{0.704193in}{0.948515in}}{\pgfqpoint{0.700921in}{0.940615in}}{\pgfqpoint{0.700921in}{0.932379in}}%
\pgfpathcurveto{\pgfqpoint{0.700921in}{0.924142in}}{\pgfqpoint{0.704193in}{0.916242in}}{\pgfqpoint{0.710017in}{0.910418in}}%
\pgfpathcurveto{\pgfqpoint{0.715841in}{0.904595in}}{\pgfqpoint{0.723741in}{0.901322in}}{\pgfqpoint{0.731977in}{0.901322in}}%
\pgfpathclose%
\pgfusepath{stroke,fill}%
\end{pgfscope}%
\begin{pgfscope}%
\pgfpathrectangle{\pgfqpoint{0.556847in}{0.516222in}}{\pgfqpoint{1.722590in}{1.783528in}} %
\pgfusepath{clip}%
\pgfsetbuttcap%
\pgfsetroundjoin%
\definecolor{currentfill}{rgb}{0.298039,0.447059,0.690196}%
\pgfsetfillcolor{currentfill}%
\pgfsetlinewidth{0.240900pt}%
\definecolor{currentstroke}{rgb}{1.000000,1.000000,1.000000}%
\pgfsetstrokecolor{currentstroke}%
\pgfsetdash{}{0pt}%
\pgfpathmoveto{\pgfqpoint{1.745435in}{1.644459in}}%
\pgfpathcurveto{\pgfqpoint{1.753671in}{1.644459in}}{\pgfqpoint{1.761571in}{1.647731in}}{\pgfqpoint{1.767395in}{1.653555in}}%
\pgfpathcurveto{\pgfqpoint{1.773219in}{1.659379in}}{\pgfqpoint{1.776491in}{1.667279in}}{\pgfqpoint{1.776491in}{1.675515in}}%
\pgfpathcurveto{\pgfqpoint{1.776491in}{1.683752in}}{\pgfqpoint{1.773219in}{1.691652in}}{\pgfqpoint{1.767395in}{1.697476in}}%
\pgfpathcurveto{\pgfqpoint{1.761571in}{1.703299in}}{\pgfqpoint{1.753671in}{1.706572in}}{\pgfqpoint{1.745435in}{1.706572in}}%
\pgfpathcurveto{\pgfqpoint{1.737198in}{1.706572in}}{\pgfqpoint{1.729298in}{1.703299in}}{\pgfqpoint{1.723474in}{1.697476in}}%
\pgfpathcurveto{\pgfqpoint{1.717650in}{1.691652in}}{\pgfqpoint{1.714378in}{1.683752in}}{\pgfqpoint{1.714378in}{1.675515in}}%
\pgfpathcurveto{\pgfqpoint{1.714378in}{1.667279in}}{\pgfqpoint{1.717650in}{1.659379in}}{\pgfqpoint{1.723474in}{1.653555in}}%
\pgfpathcurveto{\pgfqpoint{1.729298in}{1.647731in}}{\pgfqpoint{1.737198in}{1.644459in}}{\pgfqpoint{1.745435in}{1.644459in}}%
\pgfpathclose%
\pgfusepath{stroke,fill}%
\end{pgfscope}%
\begin{pgfscope}%
\pgfpathrectangle{\pgfqpoint{0.556847in}{0.516222in}}{\pgfqpoint{1.722590in}{1.783528in}} %
\pgfusepath{clip}%
\pgfsetbuttcap%
\pgfsetroundjoin%
\definecolor{currentfill}{rgb}{0.298039,0.447059,0.690196}%
\pgfsetfillcolor{currentfill}%
\pgfsetlinewidth{0.240900pt}%
\definecolor{currentstroke}{rgb}{1.000000,1.000000,1.000000}%
\pgfsetstrokecolor{currentstroke}%
\pgfsetdash{}{0pt}%
\pgfpathmoveto{\pgfqpoint{0.697525in}{0.871597in}}%
\pgfpathcurveto{\pgfqpoint{0.705762in}{0.871597in}}{\pgfqpoint{0.713662in}{0.874869in}}{\pgfqpoint{0.719486in}{0.880693in}}%
\pgfpathcurveto{\pgfqpoint{0.725310in}{0.886517in}}{\pgfqpoint{0.728582in}{0.894417in}}{\pgfqpoint{0.728582in}{0.902653in}}%
\pgfpathcurveto{\pgfqpoint{0.728582in}{0.910890in}}{\pgfqpoint{0.725310in}{0.918790in}}{\pgfqpoint{0.719486in}{0.924614in}}%
\pgfpathcurveto{\pgfqpoint{0.713662in}{0.930437in}}{\pgfqpoint{0.705762in}{0.933710in}}{\pgfqpoint{0.697525in}{0.933710in}}%
\pgfpathcurveto{\pgfqpoint{0.689289in}{0.933710in}}{\pgfqpoint{0.681389in}{0.930437in}}{\pgfqpoint{0.675565in}{0.924614in}}%
\pgfpathcurveto{\pgfqpoint{0.669741in}{0.918790in}}{\pgfqpoint{0.666469in}{0.910890in}}{\pgfqpoint{0.666469in}{0.902653in}}%
\pgfpathcurveto{\pgfqpoint{0.666469in}{0.894417in}}{\pgfqpoint{0.669741in}{0.886517in}}{\pgfqpoint{0.675565in}{0.880693in}}%
\pgfpathcurveto{\pgfqpoint{0.681389in}{0.874869in}}{\pgfqpoint{0.689289in}{0.871597in}}{\pgfqpoint{0.697525in}{0.871597in}}%
\pgfpathclose%
\pgfusepath{stroke,fill}%
\end{pgfscope}%
\begin{pgfscope}%
\pgfpathrectangle{\pgfqpoint{0.556847in}{0.516222in}}{\pgfqpoint{1.722590in}{1.783528in}} %
\pgfusepath{clip}%
\pgfsetbuttcap%
\pgfsetroundjoin%
\definecolor{currentfill}{rgb}{0.298039,0.447059,0.690196}%
\pgfsetfillcolor{currentfill}%
\pgfsetlinewidth{0.240900pt}%
\definecolor{currentstroke}{rgb}{1.000000,1.000000,1.000000}%
\pgfsetstrokecolor{currentstroke}%
\pgfsetdash{}{0pt}%
\pgfpathmoveto{\pgfqpoint{1.458336in}{1.198577in}}%
\pgfpathcurveto{\pgfqpoint{1.466572in}{1.198577in}}{\pgfqpoint{1.474472in}{1.201849in}}{\pgfqpoint{1.480296in}{1.207673in}}%
\pgfpathcurveto{\pgfqpoint{1.486120in}{1.213497in}}{\pgfqpoint{1.489393in}{1.221397in}}{\pgfqpoint{1.489393in}{1.229633in}}%
\pgfpathcurveto{\pgfqpoint{1.489393in}{1.237870in}}{\pgfqpoint{1.486120in}{1.245770in}}{\pgfqpoint{1.480296in}{1.251594in}}%
\pgfpathcurveto{\pgfqpoint{1.474472in}{1.257418in}}{\pgfqpoint{1.466572in}{1.260690in}}{\pgfqpoint{1.458336in}{1.260690in}}%
\pgfpathcurveto{\pgfqpoint{1.450100in}{1.260690in}}{\pgfqpoint{1.442200in}{1.257418in}}{\pgfqpoint{1.436376in}{1.251594in}}%
\pgfpathcurveto{\pgfqpoint{1.430552in}{1.245770in}}{\pgfqpoint{1.427280in}{1.237870in}}{\pgfqpoint{1.427280in}{1.229633in}}%
\pgfpathcurveto{\pgfqpoint{1.427280in}{1.221397in}}{\pgfqpoint{1.430552in}{1.213497in}}{\pgfqpoint{1.436376in}{1.207673in}}%
\pgfpathcurveto{\pgfqpoint{1.442200in}{1.201849in}}{\pgfqpoint{1.450100in}{1.198577in}}{\pgfqpoint{1.458336in}{1.198577in}}%
\pgfpathclose%
\pgfusepath{stroke,fill}%
\end{pgfscope}%
\begin{pgfscope}%
\pgfpathrectangle{\pgfqpoint{0.556847in}{0.516222in}}{\pgfqpoint{1.722590in}{1.783528in}} %
\pgfusepath{clip}%
\pgfsetbuttcap%
\pgfsetroundjoin%
\definecolor{currentfill}{rgb}{0.298039,0.447059,0.690196}%
\pgfsetfillcolor{currentfill}%
\pgfsetlinewidth{0.240900pt}%
\definecolor{currentstroke}{rgb}{1.000000,1.000000,1.000000}%
\pgfsetstrokecolor{currentstroke}%
\pgfsetdash{}{0pt}%
\pgfpathmoveto{\pgfqpoint{0.887010in}{1.406655in}}%
\pgfpathcurveto{\pgfqpoint{0.895247in}{1.406655in}}{\pgfqpoint{0.903147in}{1.409927in}}{\pgfqpoint{0.908971in}{1.415751in}}%
\pgfpathcurveto{\pgfqpoint{0.914795in}{1.421575in}}{\pgfqpoint{0.918067in}{1.429475in}}{\pgfqpoint{0.918067in}{1.437712in}}%
\pgfpathcurveto{\pgfqpoint{0.918067in}{1.445948in}}{\pgfqpoint{0.914795in}{1.453848in}}{\pgfqpoint{0.908971in}{1.459672in}}%
\pgfpathcurveto{\pgfqpoint{0.903147in}{1.465496in}}{\pgfqpoint{0.895247in}{1.468768in}}{\pgfqpoint{0.887010in}{1.468768in}}%
\pgfpathcurveto{\pgfqpoint{0.878774in}{1.468768in}}{\pgfqpoint{0.870874in}{1.465496in}}{\pgfqpoint{0.865050in}{1.459672in}}%
\pgfpathcurveto{\pgfqpoint{0.859226in}{1.453848in}}{\pgfqpoint{0.855954in}{1.445948in}}{\pgfqpoint{0.855954in}{1.437712in}}%
\pgfpathcurveto{\pgfqpoint{0.855954in}{1.429475in}}{\pgfqpoint{0.859226in}{1.421575in}}{\pgfqpoint{0.865050in}{1.415751in}}%
\pgfpathcurveto{\pgfqpoint{0.870874in}{1.409927in}}{\pgfqpoint{0.878774in}{1.406655in}}{\pgfqpoint{0.887010in}{1.406655in}}%
\pgfpathclose%
\pgfusepath{stroke,fill}%
\end{pgfscope}%
\begin{pgfscope}%
\pgfpathrectangle{\pgfqpoint{0.556847in}{0.516222in}}{\pgfqpoint{1.722590in}{1.783528in}} %
\pgfusepath{clip}%
\pgfsetbuttcap%
\pgfsetroundjoin%
\definecolor{currentfill}{rgb}{0.298039,0.447059,0.690196}%
\pgfsetfillcolor{currentfill}%
\pgfsetlinewidth{0.240900pt}%
\definecolor{currentstroke}{rgb}{1.000000,1.000000,1.000000}%
\pgfsetstrokecolor{currentstroke}%
\pgfsetdash{}{0pt}%
\pgfpathmoveto{\pgfqpoint{1.271722in}{1.139126in}}%
\pgfpathcurveto{\pgfqpoint{1.279958in}{1.139126in}}{\pgfqpoint{1.287859in}{1.142398in}}{\pgfqpoint{1.293682in}{1.148222in}}%
\pgfpathcurveto{\pgfqpoint{1.299506in}{1.154046in}}{\pgfqpoint{1.302779in}{1.161946in}}{\pgfqpoint{1.302779in}{1.170182in}}%
\pgfpathcurveto{\pgfqpoint{1.302779in}{1.178419in}}{\pgfqpoint{1.299506in}{1.186319in}}{\pgfqpoint{1.293682in}{1.192143in}}%
\pgfpathcurveto{\pgfqpoint{1.287859in}{1.197967in}}{\pgfqpoint{1.279958in}{1.201239in}}{\pgfqpoint{1.271722in}{1.201239in}}%
\pgfpathcurveto{\pgfqpoint{1.263486in}{1.201239in}}{\pgfqpoint{1.255586in}{1.197967in}}{\pgfqpoint{1.249762in}{1.192143in}}%
\pgfpathcurveto{\pgfqpoint{1.243938in}{1.186319in}}{\pgfqpoint{1.240666in}{1.178419in}}{\pgfqpoint{1.240666in}{1.170182in}}%
\pgfpathcurveto{\pgfqpoint{1.240666in}{1.161946in}}{\pgfqpoint{1.243938in}{1.154046in}}{\pgfqpoint{1.249762in}{1.148222in}}%
\pgfpathcurveto{\pgfqpoint{1.255586in}{1.142398in}}{\pgfqpoint{1.263486in}{1.139126in}}{\pgfqpoint{1.271722in}{1.139126in}}%
\pgfpathclose%
\pgfusepath{stroke,fill}%
\end{pgfscope}%
\begin{pgfscope}%
\pgfpathrectangle{\pgfqpoint{0.556847in}{0.516222in}}{\pgfqpoint{1.722590in}{1.783528in}} %
\pgfusepath{clip}%
\pgfsetbuttcap%
\pgfsetroundjoin%
\definecolor{currentfill}{rgb}{0.298039,0.447059,0.690196}%
\pgfsetfillcolor{currentfill}%
\pgfsetlinewidth{0.240900pt}%
\definecolor{currentstroke}{rgb}{1.000000,1.000000,1.000000}%
\pgfsetstrokecolor{currentstroke}%
\pgfsetdash{}{0pt}%
\pgfpathmoveto{\pgfqpoint{1.007592in}{1.079675in}}%
\pgfpathcurveto{\pgfqpoint{1.015828in}{1.079675in}}{\pgfqpoint{1.023728in}{1.082947in}}{\pgfqpoint{1.029552in}{1.088771in}}%
\pgfpathcurveto{\pgfqpoint{1.035376in}{1.094595in}}{\pgfqpoint{1.038648in}{1.102495in}}{\pgfqpoint{1.038648in}{1.110731in}}%
\pgfpathcurveto{\pgfqpoint{1.038648in}{1.118968in}}{\pgfqpoint{1.035376in}{1.126868in}}{\pgfqpoint{1.029552in}{1.132692in}}%
\pgfpathcurveto{\pgfqpoint{1.023728in}{1.138516in}}{\pgfqpoint{1.015828in}{1.141788in}}{\pgfqpoint{1.007592in}{1.141788in}}%
\pgfpathcurveto{\pgfqpoint{0.999355in}{1.141788in}}{\pgfqpoint{0.991455in}{1.138516in}}{\pgfqpoint{0.985631in}{1.132692in}}%
\pgfpathcurveto{\pgfqpoint{0.979807in}{1.126868in}}{\pgfqpoint{0.976535in}{1.118968in}}{\pgfqpoint{0.976535in}{1.110731in}}%
\pgfpathcurveto{\pgfqpoint{0.976535in}{1.102495in}}{\pgfqpoint{0.979807in}{1.094595in}}{\pgfqpoint{0.985631in}{1.088771in}}%
\pgfpathcurveto{\pgfqpoint{0.991455in}{1.082947in}}{\pgfqpoint{0.999355in}{1.079675in}}{\pgfqpoint{1.007592in}{1.079675in}}%
\pgfpathclose%
\pgfusepath{stroke,fill}%
\end{pgfscope}%
\begin{pgfscope}%
\pgfpathrectangle{\pgfqpoint{0.556847in}{0.516222in}}{\pgfqpoint{1.722590in}{1.783528in}} %
\pgfusepath{clip}%
\pgfsetbuttcap%
\pgfsetroundjoin%
\definecolor{currentfill}{rgb}{0.298039,0.447059,0.690196}%
\pgfsetfillcolor{currentfill}%
\pgfsetlinewidth{0.240900pt}%
\definecolor{currentstroke}{rgb}{1.000000,1.000000,1.000000}%
\pgfsetstrokecolor{currentstroke}%
\pgfsetdash{}{0pt}%
\pgfpathmoveto{\pgfqpoint{1.673660in}{1.168851in}}%
\pgfpathcurveto{\pgfqpoint{1.681896in}{1.168851in}}{\pgfqpoint{1.689796in}{1.172124in}}{\pgfqpoint{1.695620in}{1.177948in}}%
\pgfpathcurveto{\pgfqpoint{1.701444in}{1.183772in}}{\pgfqpoint{1.704716in}{1.191672in}}{\pgfqpoint{1.704716in}{1.199908in}}%
\pgfpathcurveto{\pgfqpoint{1.704716in}{1.208144in}}{\pgfqpoint{1.701444in}{1.216044in}}{\pgfqpoint{1.695620in}{1.221868in}}%
\pgfpathcurveto{\pgfqpoint{1.689796in}{1.227692in}}{\pgfqpoint{1.681896in}{1.230964in}}{\pgfqpoint{1.673660in}{1.230964in}}%
\pgfpathcurveto{\pgfqpoint{1.665424in}{1.230964in}}{\pgfqpoint{1.657524in}{1.227692in}}{\pgfqpoint{1.651700in}{1.221868in}}%
\pgfpathcurveto{\pgfqpoint{1.645876in}{1.216044in}}{\pgfqpoint{1.642603in}{1.208144in}}{\pgfqpoint{1.642603in}{1.199908in}}%
\pgfpathcurveto{\pgfqpoint{1.642603in}{1.191672in}}{\pgfqpoint{1.645876in}{1.183772in}}{\pgfqpoint{1.651700in}{1.177948in}}%
\pgfpathcurveto{\pgfqpoint{1.657524in}{1.172124in}}{\pgfqpoint{1.665424in}{1.168851in}}{\pgfqpoint{1.673660in}{1.168851in}}%
\pgfpathclose%
\pgfusepath{stroke,fill}%
\end{pgfscope}%
\begin{pgfscope}%
\pgfpathrectangle{\pgfqpoint{0.556847in}{0.516222in}}{\pgfqpoint{1.722590in}{1.783528in}} %
\pgfusepath{clip}%
\pgfsetbuttcap%
\pgfsetroundjoin%
\definecolor{currentfill}{rgb}{0.298039,0.447059,0.690196}%
\pgfsetfillcolor{currentfill}%
\pgfsetlinewidth{0.240900pt}%
\definecolor{currentstroke}{rgb}{1.000000,1.000000,1.000000}%
\pgfsetstrokecolor{currentstroke}%
\pgfsetdash{}{0pt}%
\pgfpathmoveto{\pgfqpoint{1.202819in}{1.763361in}}%
\pgfpathcurveto{\pgfqpoint{1.211055in}{1.763361in}}{\pgfqpoint{1.218955in}{1.766633in}}{\pgfqpoint{1.224779in}{1.772457in}}%
\pgfpathcurveto{\pgfqpoint{1.230603in}{1.778281in}}{\pgfqpoint{1.233875in}{1.786181in}}{\pgfqpoint{1.233875in}{1.794417in}}%
\pgfpathcurveto{\pgfqpoint{1.233875in}{1.802653in}}{\pgfqpoint{1.230603in}{1.810553in}}{\pgfqpoint{1.224779in}{1.816377in}}%
\pgfpathcurveto{\pgfqpoint{1.218955in}{1.822201in}}{\pgfqpoint{1.211055in}{1.825474in}}{\pgfqpoint{1.202819in}{1.825474in}}%
\pgfpathcurveto{\pgfqpoint{1.194582in}{1.825474in}}{\pgfqpoint{1.186682in}{1.822201in}}{\pgfqpoint{1.180858in}{1.816377in}}%
\pgfpathcurveto{\pgfqpoint{1.175034in}{1.810553in}}{\pgfqpoint{1.171762in}{1.802653in}}{\pgfqpoint{1.171762in}{1.794417in}}%
\pgfpathcurveto{\pgfqpoint{1.171762in}{1.786181in}}{\pgfqpoint{1.175034in}{1.778281in}}{\pgfqpoint{1.180858in}{1.772457in}}%
\pgfpathcurveto{\pgfqpoint{1.186682in}{1.766633in}}{\pgfqpoint{1.194582in}{1.763361in}}{\pgfqpoint{1.202819in}{1.763361in}}%
\pgfpathclose%
\pgfusepath{stroke,fill}%
\end{pgfscope}%
\begin{pgfscope}%
\pgfsetrectcap%
\pgfsetmiterjoin%
\pgfsetlinewidth{0.000000pt}%
\definecolor{currentstroke}{rgb}{1.000000,1.000000,1.000000}%
\pgfsetstrokecolor{currentstroke}%
\pgfsetdash{}{0pt}%
\pgfpathmoveto{\pgfqpoint{0.556847in}{0.516222in}}%
\pgfpathlineto{\pgfqpoint{0.556847in}{2.299750in}}%
\pgfusepath{}%
\end{pgfscope}%
\begin{pgfscope}%
\pgfsetrectcap%
\pgfsetmiterjoin%
\pgfsetlinewidth{0.000000pt}%
\definecolor{currentstroke}{rgb}{1.000000,1.000000,1.000000}%
\pgfsetstrokecolor{currentstroke}%
\pgfsetdash{}{0pt}%
\pgfpathmoveto{\pgfqpoint{0.556847in}{0.516222in}}%
\pgfpathlineto{\pgfqpoint{2.279437in}{0.516222in}}%
\pgfusepath{}%
\end{pgfscope}%
\end{pgfpicture}%
\makeatother%
\endgroup%

    \caption{Comparison between the tail and the arm lengths.}
    \label{fig_tl_al}
  \end{subfigure}
  \caption{}
\end{figure}

\paragraph{Relations in the input data.}
According to the provided data, we can appreciate some relation between
different variables. For instance, the wing length and the falling times seems
to be linearly correlated according to~\cref{fig_wl_times}. The same correlation
but this time less clear is present in~\cref{fig_al_times} with the arm length.
Finally, the arm and tail lengths might also be correlated as shown
in~\cref{fig_tl_al}.
\begin{figure}
  \centering
  %% Creator: Matplotlib, PGF backend
%%
%% To include the figure in your LaTeX document, write
%%   \input{<filename>.pgf}
%%
%% Make sure the required packages are loaded in your preamble
%%   \usepackage{pgf}
%%
%% Figures using additional raster images can only be included by \input if
%% they are in the same directory as the main LaTeX file. For loading figures
%% from other directories you can use the `import` package
%%   \usepackage{import}
%% and then include the figures with
%%   \import{<path to file>}{<filename>.pgf}
%%
%% Matplotlib used the following preamble
%%   \usepackage[utf8x]{inputenc}
%%   \usepackage[T1]{fontenc}
%%   \usepackage{cmbright}
%%
\begingroup%
\makeatletter%
\begin{pgfpicture}%
\pgfpathrectangle{\pgfpointorigin}{\pgfqpoint{5.000000in}{2.500000in}}%
\pgfusepath{use as bounding box, clip}%
\begin{pgfscope}%
\pgfsetbuttcap%
\pgfsetmiterjoin%
\definecolor{currentfill}{rgb}{1.000000,1.000000,1.000000}%
\pgfsetfillcolor{currentfill}%
\pgfsetlinewidth{0.000000pt}%
\definecolor{currentstroke}{rgb}{1.000000,1.000000,1.000000}%
\pgfsetstrokecolor{currentstroke}%
\pgfsetdash{}{0pt}%
\pgfpathmoveto{\pgfqpoint{0.000000in}{0.000000in}}%
\pgfpathlineto{\pgfqpoint{5.000000in}{0.000000in}}%
\pgfpathlineto{\pgfqpoint{5.000000in}{2.500000in}}%
\pgfpathlineto{\pgfqpoint{0.000000in}{2.500000in}}%
\pgfpathclose%
\pgfusepath{fill}%
\end{pgfscope}%
\begin{pgfscope}%
\pgfsetbuttcap%
\pgfsetmiterjoin%
\definecolor{currentfill}{rgb}{0.917647,0.917647,0.949020}%
\pgfsetfillcolor{currentfill}%
\pgfsetlinewidth{0.000000pt}%
\definecolor{currentstroke}{rgb}{0.000000,0.000000,0.000000}%
\pgfsetstrokecolor{currentstroke}%
\pgfsetstrokeopacity{0.000000}%
\pgfsetdash{}{0pt}%
\pgfpathmoveto{\pgfqpoint{0.556847in}{0.516222in}}%
\pgfpathlineto{\pgfqpoint{2.519580in}{0.516222in}}%
\pgfpathlineto{\pgfqpoint{2.519580in}{2.299750in}}%
\pgfpathlineto{\pgfqpoint{0.556847in}{2.299750in}}%
\pgfpathclose%
\pgfusepath{fill}%
\end{pgfscope}%
\begin{pgfscope}%
\pgfpathrectangle{\pgfqpoint{0.556847in}{0.516222in}}{\pgfqpoint{1.962733in}{1.783528in}} %
\pgfusepath{clip}%
\pgfsetroundcap%
\pgfsetroundjoin%
\pgfsetlinewidth{0.803000pt}%
\definecolor{currentstroke}{rgb}{1.000000,1.000000,1.000000}%
\pgfsetstrokecolor{currentstroke}%
\pgfsetdash{}{0pt}%
\pgfpathmoveto{\pgfqpoint{0.556847in}{0.516222in}}%
\pgfpathlineto{\pgfqpoint{0.556847in}{2.299750in}}%
\pgfusepath{stroke}%
\end{pgfscope}%
\begin{pgfscope}%
\pgfsetbuttcap%
\pgfsetroundjoin%
\definecolor{currentfill}{rgb}{0.150000,0.150000,0.150000}%
\pgfsetfillcolor{currentfill}%
\pgfsetlinewidth{0.803000pt}%
\definecolor{currentstroke}{rgb}{0.150000,0.150000,0.150000}%
\pgfsetstrokecolor{currentstroke}%
\pgfsetdash{}{0pt}%
\pgfsys@defobject{currentmarker}{\pgfqpoint{0.000000in}{0.000000in}}{\pgfqpoint{0.000000in}{0.000000in}}{%
\pgfpathmoveto{\pgfqpoint{0.000000in}{0.000000in}}%
\pgfpathlineto{\pgfqpoint{0.000000in}{0.000000in}}%
\pgfusepath{stroke,fill}%
}%
\begin{pgfscope}%
\pgfsys@transformshift{0.556847in}{0.516222in}%
\pgfsys@useobject{currentmarker}{}%
\end{pgfscope}%
\end{pgfscope}%
\begin{pgfscope}%
\definecolor{textcolor}{rgb}{0.150000,0.150000,0.150000}%
\pgfsetstrokecolor{textcolor}%
\pgfsetfillcolor{textcolor}%
\pgftext[x=0.556847in,y=0.438444in,,top]{\color{textcolor}\sffamily\fontsize{8.000000}{9.600000}\selectfont 1.5}%
\end{pgfscope}%
\begin{pgfscope}%
\pgfpathrectangle{\pgfqpoint{0.556847in}{0.516222in}}{\pgfqpoint{1.962733in}{1.783528in}} %
\pgfusepath{clip}%
\pgfsetroundcap%
\pgfsetroundjoin%
\pgfsetlinewidth{0.803000pt}%
\definecolor{currentstroke}{rgb}{1.000000,1.000000,1.000000}%
\pgfsetstrokecolor{currentstroke}%
\pgfsetdash{}{0pt}%
\pgfpathmoveto{\pgfqpoint{0.837238in}{0.516222in}}%
\pgfpathlineto{\pgfqpoint{0.837238in}{2.299750in}}%
\pgfusepath{stroke}%
\end{pgfscope}%
\begin{pgfscope}%
\pgfsetbuttcap%
\pgfsetroundjoin%
\definecolor{currentfill}{rgb}{0.150000,0.150000,0.150000}%
\pgfsetfillcolor{currentfill}%
\pgfsetlinewidth{0.803000pt}%
\definecolor{currentstroke}{rgb}{0.150000,0.150000,0.150000}%
\pgfsetstrokecolor{currentstroke}%
\pgfsetdash{}{0pt}%
\pgfsys@defobject{currentmarker}{\pgfqpoint{0.000000in}{0.000000in}}{\pgfqpoint{0.000000in}{0.000000in}}{%
\pgfpathmoveto{\pgfqpoint{0.000000in}{0.000000in}}%
\pgfpathlineto{\pgfqpoint{0.000000in}{0.000000in}}%
\pgfusepath{stroke,fill}%
}%
\begin{pgfscope}%
\pgfsys@transformshift{0.837238in}{0.516222in}%
\pgfsys@useobject{currentmarker}{}%
\end{pgfscope}%
\end{pgfscope}%
\begin{pgfscope}%
\definecolor{textcolor}{rgb}{0.150000,0.150000,0.150000}%
\pgfsetstrokecolor{textcolor}%
\pgfsetfillcolor{textcolor}%
\pgftext[x=0.837238in,y=0.438444in,,top]{\color{textcolor}\sffamily\fontsize{8.000000}{9.600000}\selectfont 2.0}%
\end{pgfscope}%
\begin{pgfscope}%
\pgfpathrectangle{\pgfqpoint{0.556847in}{0.516222in}}{\pgfqpoint{1.962733in}{1.783528in}} %
\pgfusepath{clip}%
\pgfsetroundcap%
\pgfsetroundjoin%
\pgfsetlinewidth{0.803000pt}%
\definecolor{currentstroke}{rgb}{1.000000,1.000000,1.000000}%
\pgfsetstrokecolor{currentstroke}%
\pgfsetdash{}{0pt}%
\pgfpathmoveto{\pgfqpoint{1.117628in}{0.516222in}}%
\pgfpathlineto{\pgfqpoint{1.117628in}{2.299750in}}%
\pgfusepath{stroke}%
\end{pgfscope}%
\begin{pgfscope}%
\pgfsetbuttcap%
\pgfsetroundjoin%
\definecolor{currentfill}{rgb}{0.150000,0.150000,0.150000}%
\pgfsetfillcolor{currentfill}%
\pgfsetlinewidth{0.803000pt}%
\definecolor{currentstroke}{rgb}{0.150000,0.150000,0.150000}%
\pgfsetstrokecolor{currentstroke}%
\pgfsetdash{}{0pt}%
\pgfsys@defobject{currentmarker}{\pgfqpoint{0.000000in}{0.000000in}}{\pgfqpoint{0.000000in}{0.000000in}}{%
\pgfpathmoveto{\pgfqpoint{0.000000in}{0.000000in}}%
\pgfpathlineto{\pgfqpoint{0.000000in}{0.000000in}}%
\pgfusepath{stroke,fill}%
}%
\begin{pgfscope}%
\pgfsys@transformshift{1.117628in}{0.516222in}%
\pgfsys@useobject{currentmarker}{}%
\end{pgfscope}%
\end{pgfscope}%
\begin{pgfscope}%
\definecolor{textcolor}{rgb}{0.150000,0.150000,0.150000}%
\pgfsetstrokecolor{textcolor}%
\pgfsetfillcolor{textcolor}%
\pgftext[x=1.117628in,y=0.438444in,,top]{\color{textcolor}\sffamily\fontsize{8.000000}{9.600000}\selectfont 2.5}%
\end{pgfscope}%
\begin{pgfscope}%
\pgfpathrectangle{\pgfqpoint{0.556847in}{0.516222in}}{\pgfqpoint{1.962733in}{1.783528in}} %
\pgfusepath{clip}%
\pgfsetroundcap%
\pgfsetroundjoin%
\pgfsetlinewidth{0.803000pt}%
\definecolor{currentstroke}{rgb}{1.000000,1.000000,1.000000}%
\pgfsetstrokecolor{currentstroke}%
\pgfsetdash{}{0pt}%
\pgfpathmoveto{\pgfqpoint{1.398018in}{0.516222in}}%
\pgfpathlineto{\pgfqpoint{1.398018in}{2.299750in}}%
\pgfusepath{stroke}%
\end{pgfscope}%
\begin{pgfscope}%
\pgfsetbuttcap%
\pgfsetroundjoin%
\definecolor{currentfill}{rgb}{0.150000,0.150000,0.150000}%
\pgfsetfillcolor{currentfill}%
\pgfsetlinewidth{0.803000pt}%
\definecolor{currentstroke}{rgb}{0.150000,0.150000,0.150000}%
\pgfsetstrokecolor{currentstroke}%
\pgfsetdash{}{0pt}%
\pgfsys@defobject{currentmarker}{\pgfqpoint{0.000000in}{0.000000in}}{\pgfqpoint{0.000000in}{0.000000in}}{%
\pgfpathmoveto{\pgfqpoint{0.000000in}{0.000000in}}%
\pgfpathlineto{\pgfqpoint{0.000000in}{0.000000in}}%
\pgfusepath{stroke,fill}%
}%
\begin{pgfscope}%
\pgfsys@transformshift{1.398018in}{0.516222in}%
\pgfsys@useobject{currentmarker}{}%
\end{pgfscope}%
\end{pgfscope}%
\begin{pgfscope}%
\definecolor{textcolor}{rgb}{0.150000,0.150000,0.150000}%
\pgfsetstrokecolor{textcolor}%
\pgfsetfillcolor{textcolor}%
\pgftext[x=1.398018in,y=0.438444in,,top]{\color{textcolor}\sffamily\fontsize{8.000000}{9.600000}\selectfont 3.0}%
\end{pgfscope}%
\begin{pgfscope}%
\pgfpathrectangle{\pgfqpoint{0.556847in}{0.516222in}}{\pgfqpoint{1.962733in}{1.783528in}} %
\pgfusepath{clip}%
\pgfsetroundcap%
\pgfsetroundjoin%
\pgfsetlinewidth{0.803000pt}%
\definecolor{currentstroke}{rgb}{1.000000,1.000000,1.000000}%
\pgfsetstrokecolor{currentstroke}%
\pgfsetdash{}{0pt}%
\pgfpathmoveto{\pgfqpoint{1.678409in}{0.516222in}}%
\pgfpathlineto{\pgfqpoint{1.678409in}{2.299750in}}%
\pgfusepath{stroke}%
\end{pgfscope}%
\begin{pgfscope}%
\pgfsetbuttcap%
\pgfsetroundjoin%
\definecolor{currentfill}{rgb}{0.150000,0.150000,0.150000}%
\pgfsetfillcolor{currentfill}%
\pgfsetlinewidth{0.803000pt}%
\definecolor{currentstroke}{rgb}{0.150000,0.150000,0.150000}%
\pgfsetstrokecolor{currentstroke}%
\pgfsetdash{}{0pt}%
\pgfsys@defobject{currentmarker}{\pgfqpoint{0.000000in}{0.000000in}}{\pgfqpoint{0.000000in}{0.000000in}}{%
\pgfpathmoveto{\pgfqpoint{0.000000in}{0.000000in}}%
\pgfpathlineto{\pgfqpoint{0.000000in}{0.000000in}}%
\pgfusepath{stroke,fill}%
}%
\begin{pgfscope}%
\pgfsys@transformshift{1.678409in}{0.516222in}%
\pgfsys@useobject{currentmarker}{}%
\end{pgfscope}%
\end{pgfscope}%
\begin{pgfscope}%
\definecolor{textcolor}{rgb}{0.150000,0.150000,0.150000}%
\pgfsetstrokecolor{textcolor}%
\pgfsetfillcolor{textcolor}%
\pgftext[x=1.678409in,y=0.438444in,,top]{\color{textcolor}\sffamily\fontsize{8.000000}{9.600000}\selectfont 3.5}%
\end{pgfscope}%
\begin{pgfscope}%
\pgfpathrectangle{\pgfqpoint{0.556847in}{0.516222in}}{\pgfqpoint{1.962733in}{1.783528in}} %
\pgfusepath{clip}%
\pgfsetroundcap%
\pgfsetroundjoin%
\pgfsetlinewidth{0.803000pt}%
\definecolor{currentstroke}{rgb}{1.000000,1.000000,1.000000}%
\pgfsetstrokecolor{currentstroke}%
\pgfsetdash{}{0pt}%
\pgfpathmoveto{\pgfqpoint{1.958799in}{0.516222in}}%
\pgfpathlineto{\pgfqpoint{1.958799in}{2.299750in}}%
\pgfusepath{stroke}%
\end{pgfscope}%
\begin{pgfscope}%
\pgfsetbuttcap%
\pgfsetroundjoin%
\definecolor{currentfill}{rgb}{0.150000,0.150000,0.150000}%
\pgfsetfillcolor{currentfill}%
\pgfsetlinewidth{0.803000pt}%
\definecolor{currentstroke}{rgb}{0.150000,0.150000,0.150000}%
\pgfsetstrokecolor{currentstroke}%
\pgfsetdash{}{0pt}%
\pgfsys@defobject{currentmarker}{\pgfqpoint{0.000000in}{0.000000in}}{\pgfqpoint{0.000000in}{0.000000in}}{%
\pgfpathmoveto{\pgfqpoint{0.000000in}{0.000000in}}%
\pgfpathlineto{\pgfqpoint{0.000000in}{0.000000in}}%
\pgfusepath{stroke,fill}%
}%
\begin{pgfscope}%
\pgfsys@transformshift{1.958799in}{0.516222in}%
\pgfsys@useobject{currentmarker}{}%
\end{pgfscope}%
\end{pgfscope}%
\begin{pgfscope}%
\definecolor{textcolor}{rgb}{0.150000,0.150000,0.150000}%
\pgfsetstrokecolor{textcolor}%
\pgfsetfillcolor{textcolor}%
\pgftext[x=1.958799in,y=0.438444in,,top]{\color{textcolor}\sffamily\fontsize{8.000000}{9.600000}\selectfont 4.0}%
\end{pgfscope}%
\begin{pgfscope}%
\pgfpathrectangle{\pgfqpoint{0.556847in}{0.516222in}}{\pgfqpoint{1.962733in}{1.783528in}} %
\pgfusepath{clip}%
\pgfsetroundcap%
\pgfsetroundjoin%
\pgfsetlinewidth{0.803000pt}%
\definecolor{currentstroke}{rgb}{1.000000,1.000000,1.000000}%
\pgfsetstrokecolor{currentstroke}%
\pgfsetdash{}{0pt}%
\pgfpathmoveto{\pgfqpoint{2.239189in}{0.516222in}}%
\pgfpathlineto{\pgfqpoint{2.239189in}{2.299750in}}%
\pgfusepath{stroke}%
\end{pgfscope}%
\begin{pgfscope}%
\pgfsetbuttcap%
\pgfsetroundjoin%
\definecolor{currentfill}{rgb}{0.150000,0.150000,0.150000}%
\pgfsetfillcolor{currentfill}%
\pgfsetlinewidth{0.803000pt}%
\definecolor{currentstroke}{rgb}{0.150000,0.150000,0.150000}%
\pgfsetstrokecolor{currentstroke}%
\pgfsetdash{}{0pt}%
\pgfsys@defobject{currentmarker}{\pgfqpoint{0.000000in}{0.000000in}}{\pgfqpoint{0.000000in}{0.000000in}}{%
\pgfpathmoveto{\pgfqpoint{0.000000in}{0.000000in}}%
\pgfpathlineto{\pgfqpoint{0.000000in}{0.000000in}}%
\pgfusepath{stroke,fill}%
}%
\begin{pgfscope}%
\pgfsys@transformshift{2.239189in}{0.516222in}%
\pgfsys@useobject{currentmarker}{}%
\end{pgfscope}%
\end{pgfscope}%
\begin{pgfscope}%
\definecolor{textcolor}{rgb}{0.150000,0.150000,0.150000}%
\pgfsetstrokecolor{textcolor}%
\pgfsetfillcolor{textcolor}%
\pgftext[x=2.239189in,y=0.438444in,,top]{\color{textcolor}\sffamily\fontsize{8.000000}{9.600000}\selectfont 4.5}%
\end{pgfscope}%
\begin{pgfscope}%
\pgfpathrectangle{\pgfqpoint{0.556847in}{0.516222in}}{\pgfqpoint{1.962733in}{1.783528in}} %
\pgfusepath{clip}%
\pgfsetroundcap%
\pgfsetroundjoin%
\pgfsetlinewidth{0.803000pt}%
\definecolor{currentstroke}{rgb}{1.000000,1.000000,1.000000}%
\pgfsetstrokecolor{currentstroke}%
\pgfsetdash{}{0pt}%
\pgfpathmoveto{\pgfqpoint{2.519580in}{0.516222in}}%
\pgfpathlineto{\pgfqpoint{2.519580in}{2.299750in}}%
\pgfusepath{stroke}%
\end{pgfscope}%
\begin{pgfscope}%
\pgfsetbuttcap%
\pgfsetroundjoin%
\definecolor{currentfill}{rgb}{0.150000,0.150000,0.150000}%
\pgfsetfillcolor{currentfill}%
\pgfsetlinewidth{0.803000pt}%
\definecolor{currentstroke}{rgb}{0.150000,0.150000,0.150000}%
\pgfsetstrokecolor{currentstroke}%
\pgfsetdash{}{0pt}%
\pgfsys@defobject{currentmarker}{\pgfqpoint{0.000000in}{0.000000in}}{\pgfqpoint{0.000000in}{0.000000in}}{%
\pgfpathmoveto{\pgfqpoint{0.000000in}{0.000000in}}%
\pgfpathlineto{\pgfqpoint{0.000000in}{0.000000in}}%
\pgfusepath{stroke,fill}%
}%
\begin{pgfscope}%
\pgfsys@transformshift{2.519580in}{0.516222in}%
\pgfsys@useobject{currentmarker}{}%
\end{pgfscope}%
\end{pgfscope}%
\begin{pgfscope}%
\definecolor{textcolor}{rgb}{0.150000,0.150000,0.150000}%
\pgfsetstrokecolor{textcolor}%
\pgfsetfillcolor{textcolor}%
\pgftext[x=2.519580in,y=0.438444in,,top]{\color{textcolor}\sffamily\fontsize{8.000000}{9.600000}\selectfont 5.0}%
\end{pgfscope}%
\begin{pgfscope}%
\definecolor{textcolor}{rgb}{0.150000,0.150000,0.150000}%
\pgfsetstrokecolor{textcolor}%
\pgfsetfillcolor{textcolor}%
\pgftext[x=1.538214in,y=0.273321in,,top]{\color{textcolor}\sffamily\fontsize{8.800000}{10.560000}\selectfont Falling time realization 1}%
\end{pgfscope}%
\begin{pgfscope}%
\pgfpathrectangle{\pgfqpoint{0.556847in}{0.516222in}}{\pgfqpoint{1.962733in}{1.783528in}} %
\pgfusepath{clip}%
\pgfsetroundcap%
\pgfsetroundjoin%
\pgfsetlinewidth{0.803000pt}%
\definecolor{currentstroke}{rgb}{1.000000,1.000000,1.000000}%
\pgfsetstrokecolor{currentstroke}%
\pgfsetdash{}{0pt}%
\pgfpathmoveto{\pgfqpoint{0.556847in}{0.516222in}}%
\pgfpathlineto{\pgfqpoint{2.519580in}{0.516222in}}%
\pgfusepath{stroke}%
\end{pgfscope}%
\begin{pgfscope}%
\pgfsetbuttcap%
\pgfsetroundjoin%
\definecolor{currentfill}{rgb}{0.150000,0.150000,0.150000}%
\pgfsetfillcolor{currentfill}%
\pgfsetlinewidth{0.803000pt}%
\definecolor{currentstroke}{rgb}{0.150000,0.150000,0.150000}%
\pgfsetstrokecolor{currentstroke}%
\pgfsetdash{}{0pt}%
\pgfsys@defobject{currentmarker}{\pgfqpoint{0.000000in}{0.000000in}}{\pgfqpoint{0.000000in}{0.000000in}}{%
\pgfpathmoveto{\pgfqpoint{0.000000in}{0.000000in}}%
\pgfpathlineto{\pgfqpoint{0.000000in}{0.000000in}}%
\pgfusepath{stroke,fill}%
}%
\begin{pgfscope}%
\pgfsys@transformshift{0.556847in}{0.516222in}%
\pgfsys@useobject{currentmarker}{}%
\end{pgfscope}%
\end{pgfscope}%
\begin{pgfscope}%
\definecolor{textcolor}{rgb}{0.150000,0.150000,0.150000}%
\pgfsetstrokecolor{textcolor}%
\pgfsetfillcolor{textcolor}%
\pgftext[x=0.479069in,y=0.516222in,right,]{\color{textcolor}\sffamily\fontsize{8.000000}{9.600000}\selectfont 3.0}%
\end{pgfscope}%
\begin{pgfscope}%
\pgfpathrectangle{\pgfqpoint{0.556847in}{0.516222in}}{\pgfqpoint{1.962733in}{1.783528in}} %
\pgfusepath{clip}%
\pgfsetroundcap%
\pgfsetroundjoin%
\pgfsetlinewidth{0.803000pt}%
\definecolor{currentstroke}{rgb}{1.000000,1.000000,1.000000}%
\pgfsetstrokecolor{currentstroke}%
\pgfsetdash{}{0pt}%
\pgfpathmoveto{\pgfqpoint{0.556847in}{0.739163in}}%
\pgfpathlineto{\pgfqpoint{2.519580in}{0.739163in}}%
\pgfusepath{stroke}%
\end{pgfscope}%
\begin{pgfscope}%
\pgfsetbuttcap%
\pgfsetroundjoin%
\definecolor{currentfill}{rgb}{0.150000,0.150000,0.150000}%
\pgfsetfillcolor{currentfill}%
\pgfsetlinewidth{0.803000pt}%
\definecolor{currentstroke}{rgb}{0.150000,0.150000,0.150000}%
\pgfsetstrokecolor{currentstroke}%
\pgfsetdash{}{0pt}%
\pgfsys@defobject{currentmarker}{\pgfqpoint{0.000000in}{0.000000in}}{\pgfqpoint{0.000000in}{0.000000in}}{%
\pgfpathmoveto{\pgfqpoint{0.000000in}{0.000000in}}%
\pgfpathlineto{\pgfqpoint{0.000000in}{0.000000in}}%
\pgfusepath{stroke,fill}%
}%
\begin{pgfscope}%
\pgfsys@transformshift{0.556847in}{0.739163in}%
\pgfsys@useobject{currentmarker}{}%
\end{pgfscope}%
\end{pgfscope}%
\begin{pgfscope}%
\definecolor{textcolor}{rgb}{0.150000,0.150000,0.150000}%
\pgfsetstrokecolor{textcolor}%
\pgfsetfillcolor{textcolor}%
\pgftext[x=0.479069in,y=0.739163in,right,]{\color{textcolor}\sffamily\fontsize{8.000000}{9.600000}\selectfont 3.5}%
\end{pgfscope}%
\begin{pgfscope}%
\pgfpathrectangle{\pgfqpoint{0.556847in}{0.516222in}}{\pgfqpoint{1.962733in}{1.783528in}} %
\pgfusepath{clip}%
\pgfsetroundcap%
\pgfsetroundjoin%
\pgfsetlinewidth{0.803000pt}%
\definecolor{currentstroke}{rgb}{1.000000,1.000000,1.000000}%
\pgfsetstrokecolor{currentstroke}%
\pgfsetdash{}{0pt}%
\pgfpathmoveto{\pgfqpoint{0.556847in}{0.962104in}}%
\pgfpathlineto{\pgfqpoint{2.519580in}{0.962104in}}%
\pgfusepath{stroke}%
\end{pgfscope}%
\begin{pgfscope}%
\pgfsetbuttcap%
\pgfsetroundjoin%
\definecolor{currentfill}{rgb}{0.150000,0.150000,0.150000}%
\pgfsetfillcolor{currentfill}%
\pgfsetlinewidth{0.803000pt}%
\definecolor{currentstroke}{rgb}{0.150000,0.150000,0.150000}%
\pgfsetstrokecolor{currentstroke}%
\pgfsetdash{}{0pt}%
\pgfsys@defobject{currentmarker}{\pgfqpoint{0.000000in}{0.000000in}}{\pgfqpoint{0.000000in}{0.000000in}}{%
\pgfpathmoveto{\pgfqpoint{0.000000in}{0.000000in}}%
\pgfpathlineto{\pgfqpoint{0.000000in}{0.000000in}}%
\pgfusepath{stroke,fill}%
}%
\begin{pgfscope}%
\pgfsys@transformshift{0.556847in}{0.962104in}%
\pgfsys@useobject{currentmarker}{}%
\end{pgfscope}%
\end{pgfscope}%
\begin{pgfscope}%
\definecolor{textcolor}{rgb}{0.150000,0.150000,0.150000}%
\pgfsetstrokecolor{textcolor}%
\pgfsetfillcolor{textcolor}%
\pgftext[x=0.479069in,y=0.962104in,right,]{\color{textcolor}\sffamily\fontsize{8.000000}{9.600000}\selectfont 4.0}%
\end{pgfscope}%
\begin{pgfscope}%
\pgfpathrectangle{\pgfqpoint{0.556847in}{0.516222in}}{\pgfqpoint{1.962733in}{1.783528in}} %
\pgfusepath{clip}%
\pgfsetroundcap%
\pgfsetroundjoin%
\pgfsetlinewidth{0.803000pt}%
\definecolor{currentstroke}{rgb}{1.000000,1.000000,1.000000}%
\pgfsetstrokecolor{currentstroke}%
\pgfsetdash{}{0pt}%
\pgfpathmoveto{\pgfqpoint{0.556847in}{1.185045in}}%
\pgfpathlineto{\pgfqpoint{2.519580in}{1.185045in}}%
\pgfusepath{stroke}%
\end{pgfscope}%
\begin{pgfscope}%
\pgfsetbuttcap%
\pgfsetroundjoin%
\definecolor{currentfill}{rgb}{0.150000,0.150000,0.150000}%
\pgfsetfillcolor{currentfill}%
\pgfsetlinewidth{0.803000pt}%
\definecolor{currentstroke}{rgb}{0.150000,0.150000,0.150000}%
\pgfsetstrokecolor{currentstroke}%
\pgfsetdash{}{0pt}%
\pgfsys@defobject{currentmarker}{\pgfqpoint{0.000000in}{0.000000in}}{\pgfqpoint{0.000000in}{0.000000in}}{%
\pgfpathmoveto{\pgfqpoint{0.000000in}{0.000000in}}%
\pgfpathlineto{\pgfqpoint{0.000000in}{0.000000in}}%
\pgfusepath{stroke,fill}%
}%
\begin{pgfscope}%
\pgfsys@transformshift{0.556847in}{1.185045in}%
\pgfsys@useobject{currentmarker}{}%
\end{pgfscope}%
\end{pgfscope}%
\begin{pgfscope}%
\definecolor{textcolor}{rgb}{0.150000,0.150000,0.150000}%
\pgfsetstrokecolor{textcolor}%
\pgfsetfillcolor{textcolor}%
\pgftext[x=0.479069in,y=1.185045in,right,]{\color{textcolor}\sffamily\fontsize{8.000000}{9.600000}\selectfont 4.5}%
\end{pgfscope}%
\begin{pgfscope}%
\pgfpathrectangle{\pgfqpoint{0.556847in}{0.516222in}}{\pgfqpoint{1.962733in}{1.783528in}} %
\pgfusepath{clip}%
\pgfsetroundcap%
\pgfsetroundjoin%
\pgfsetlinewidth{0.803000pt}%
\definecolor{currentstroke}{rgb}{1.000000,1.000000,1.000000}%
\pgfsetstrokecolor{currentstroke}%
\pgfsetdash{}{0pt}%
\pgfpathmoveto{\pgfqpoint{0.556847in}{1.407986in}}%
\pgfpathlineto{\pgfqpoint{2.519580in}{1.407986in}}%
\pgfusepath{stroke}%
\end{pgfscope}%
\begin{pgfscope}%
\pgfsetbuttcap%
\pgfsetroundjoin%
\definecolor{currentfill}{rgb}{0.150000,0.150000,0.150000}%
\pgfsetfillcolor{currentfill}%
\pgfsetlinewidth{0.803000pt}%
\definecolor{currentstroke}{rgb}{0.150000,0.150000,0.150000}%
\pgfsetstrokecolor{currentstroke}%
\pgfsetdash{}{0pt}%
\pgfsys@defobject{currentmarker}{\pgfqpoint{0.000000in}{0.000000in}}{\pgfqpoint{0.000000in}{0.000000in}}{%
\pgfpathmoveto{\pgfqpoint{0.000000in}{0.000000in}}%
\pgfpathlineto{\pgfqpoint{0.000000in}{0.000000in}}%
\pgfusepath{stroke,fill}%
}%
\begin{pgfscope}%
\pgfsys@transformshift{0.556847in}{1.407986in}%
\pgfsys@useobject{currentmarker}{}%
\end{pgfscope}%
\end{pgfscope}%
\begin{pgfscope}%
\definecolor{textcolor}{rgb}{0.150000,0.150000,0.150000}%
\pgfsetstrokecolor{textcolor}%
\pgfsetfillcolor{textcolor}%
\pgftext[x=0.479069in,y=1.407986in,right,]{\color{textcolor}\sffamily\fontsize{8.000000}{9.600000}\selectfont 5.0}%
\end{pgfscope}%
\begin{pgfscope}%
\pgfpathrectangle{\pgfqpoint{0.556847in}{0.516222in}}{\pgfqpoint{1.962733in}{1.783528in}} %
\pgfusepath{clip}%
\pgfsetroundcap%
\pgfsetroundjoin%
\pgfsetlinewidth{0.803000pt}%
\definecolor{currentstroke}{rgb}{1.000000,1.000000,1.000000}%
\pgfsetstrokecolor{currentstroke}%
\pgfsetdash{}{0pt}%
\pgfpathmoveto{\pgfqpoint{0.556847in}{1.630927in}}%
\pgfpathlineto{\pgfqpoint{2.519580in}{1.630927in}}%
\pgfusepath{stroke}%
\end{pgfscope}%
\begin{pgfscope}%
\pgfsetbuttcap%
\pgfsetroundjoin%
\definecolor{currentfill}{rgb}{0.150000,0.150000,0.150000}%
\pgfsetfillcolor{currentfill}%
\pgfsetlinewidth{0.803000pt}%
\definecolor{currentstroke}{rgb}{0.150000,0.150000,0.150000}%
\pgfsetstrokecolor{currentstroke}%
\pgfsetdash{}{0pt}%
\pgfsys@defobject{currentmarker}{\pgfqpoint{0.000000in}{0.000000in}}{\pgfqpoint{0.000000in}{0.000000in}}{%
\pgfpathmoveto{\pgfqpoint{0.000000in}{0.000000in}}%
\pgfpathlineto{\pgfqpoint{0.000000in}{0.000000in}}%
\pgfusepath{stroke,fill}%
}%
\begin{pgfscope}%
\pgfsys@transformshift{0.556847in}{1.630927in}%
\pgfsys@useobject{currentmarker}{}%
\end{pgfscope}%
\end{pgfscope}%
\begin{pgfscope}%
\definecolor{textcolor}{rgb}{0.150000,0.150000,0.150000}%
\pgfsetstrokecolor{textcolor}%
\pgfsetfillcolor{textcolor}%
\pgftext[x=0.479069in,y=1.630927in,right,]{\color{textcolor}\sffamily\fontsize{8.000000}{9.600000}\selectfont 5.5}%
\end{pgfscope}%
\begin{pgfscope}%
\pgfpathrectangle{\pgfqpoint{0.556847in}{0.516222in}}{\pgfqpoint{1.962733in}{1.783528in}} %
\pgfusepath{clip}%
\pgfsetroundcap%
\pgfsetroundjoin%
\pgfsetlinewidth{0.803000pt}%
\definecolor{currentstroke}{rgb}{1.000000,1.000000,1.000000}%
\pgfsetstrokecolor{currentstroke}%
\pgfsetdash{}{0pt}%
\pgfpathmoveto{\pgfqpoint{0.556847in}{1.853868in}}%
\pgfpathlineto{\pgfqpoint{2.519580in}{1.853868in}}%
\pgfusepath{stroke}%
\end{pgfscope}%
\begin{pgfscope}%
\pgfsetbuttcap%
\pgfsetroundjoin%
\definecolor{currentfill}{rgb}{0.150000,0.150000,0.150000}%
\pgfsetfillcolor{currentfill}%
\pgfsetlinewidth{0.803000pt}%
\definecolor{currentstroke}{rgb}{0.150000,0.150000,0.150000}%
\pgfsetstrokecolor{currentstroke}%
\pgfsetdash{}{0pt}%
\pgfsys@defobject{currentmarker}{\pgfqpoint{0.000000in}{0.000000in}}{\pgfqpoint{0.000000in}{0.000000in}}{%
\pgfpathmoveto{\pgfqpoint{0.000000in}{0.000000in}}%
\pgfpathlineto{\pgfqpoint{0.000000in}{0.000000in}}%
\pgfusepath{stroke,fill}%
}%
\begin{pgfscope}%
\pgfsys@transformshift{0.556847in}{1.853868in}%
\pgfsys@useobject{currentmarker}{}%
\end{pgfscope}%
\end{pgfscope}%
\begin{pgfscope}%
\definecolor{textcolor}{rgb}{0.150000,0.150000,0.150000}%
\pgfsetstrokecolor{textcolor}%
\pgfsetfillcolor{textcolor}%
\pgftext[x=0.479069in,y=1.853868in,right,]{\color{textcolor}\sffamily\fontsize{8.000000}{9.600000}\selectfont 6.0}%
\end{pgfscope}%
\begin{pgfscope}%
\pgfpathrectangle{\pgfqpoint{0.556847in}{0.516222in}}{\pgfqpoint{1.962733in}{1.783528in}} %
\pgfusepath{clip}%
\pgfsetroundcap%
\pgfsetroundjoin%
\pgfsetlinewidth{0.803000pt}%
\definecolor{currentstroke}{rgb}{1.000000,1.000000,1.000000}%
\pgfsetstrokecolor{currentstroke}%
\pgfsetdash{}{0pt}%
\pgfpathmoveto{\pgfqpoint{0.556847in}{2.076809in}}%
\pgfpathlineto{\pgfqpoint{2.519580in}{2.076809in}}%
\pgfusepath{stroke}%
\end{pgfscope}%
\begin{pgfscope}%
\pgfsetbuttcap%
\pgfsetroundjoin%
\definecolor{currentfill}{rgb}{0.150000,0.150000,0.150000}%
\pgfsetfillcolor{currentfill}%
\pgfsetlinewidth{0.803000pt}%
\definecolor{currentstroke}{rgb}{0.150000,0.150000,0.150000}%
\pgfsetstrokecolor{currentstroke}%
\pgfsetdash{}{0pt}%
\pgfsys@defobject{currentmarker}{\pgfqpoint{0.000000in}{0.000000in}}{\pgfqpoint{0.000000in}{0.000000in}}{%
\pgfpathmoveto{\pgfqpoint{0.000000in}{0.000000in}}%
\pgfpathlineto{\pgfqpoint{0.000000in}{0.000000in}}%
\pgfusepath{stroke,fill}%
}%
\begin{pgfscope}%
\pgfsys@transformshift{0.556847in}{2.076809in}%
\pgfsys@useobject{currentmarker}{}%
\end{pgfscope}%
\end{pgfscope}%
\begin{pgfscope}%
\definecolor{textcolor}{rgb}{0.150000,0.150000,0.150000}%
\pgfsetstrokecolor{textcolor}%
\pgfsetfillcolor{textcolor}%
\pgftext[x=0.479069in,y=2.076809in,right,]{\color{textcolor}\sffamily\fontsize{8.000000}{9.600000}\selectfont 6.5}%
\end{pgfscope}%
\begin{pgfscope}%
\pgfpathrectangle{\pgfqpoint{0.556847in}{0.516222in}}{\pgfqpoint{1.962733in}{1.783528in}} %
\pgfusepath{clip}%
\pgfsetroundcap%
\pgfsetroundjoin%
\pgfsetlinewidth{0.803000pt}%
\definecolor{currentstroke}{rgb}{1.000000,1.000000,1.000000}%
\pgfsetstrokecolor{currentstroke}%
\pgfsetdash{}{0pt}%
\pgfpathmoveto{\pgfqpoint{0.556847in}{2.299750in}}%
\pgfpathlineto{\pgfqpoint{2.519580in}{2.299750in}}%
\pgfusepath{stroke}%
\end{pgfscope}%
\begin{pgfscope}%
\pgfsetbuttcap%
\pgfsetroundjoin%
\definecolor{currentfill}{rgb}{0.150000,0.150000,0.150000}%
\pgfsetfillcolor{currentfill}%
\pgfsetlinewidth{0.803000pt}%
\definecolor{currentstroke}{rgb}{0.150000,0.150000,0.150000}%
\pgfsetstrokecolor{currentstroke}%
\pgfsetdash{}{0pt}%
\pgfsys@defobject{currentmarker}{\pgfqpoint{0.000000in}{0.000000in}}{\pgfqpoint{0.000000in}{0.000000in}}{%
\pgfpathmoveto{\pgfqpoint{0.000000in}{0.000000in}}%
\pgfpathlineto{\pgfqpoint{0.000000in}{0.000000in}}%
\pgfusepath{stroke,fill}%
}%
\begin{pgfscope}%
\pgfsys@transformshift{0.556847in}{2.299750in}%
\pgfsys@useobject{currentmarker}{}%
\end{pgfscope}%
\end{pgfscope}%
\begin{pgfscope}%
\definecolor{textcolor}{rgb}{0.150000,0.150000,0.150000}%
\pgfsetstrokecolor{textcolor}%
\pgfsetfillcolor{textcolor}%
\pgftext[x=0.479069in,y=2.299750in,right,]{\color{textcolor}\sffamily\fontsize{8.000000}{9.600000}\selectfont 7.0}%
\end{pgfscope}%
\begin{pgfscope}%
\definecolor{textcolor}{rgb}{0.150000,0.150000,0.150000}%
\pgfsetstrokecolor{textcolor}%
\pgfsetfillcolor{textcolor}%
\pgftext[x=0.251677in,y=1.407986in,,bottom,rotate=90.000000]{\color{textcolor}\sffamily\fontsize{8.800000}{10.560000}\selectfont Wing length}%
\end{pgfscope}%
\begin{pgfscope}%
\pgfpathrectangle{\pgfqpoint{0.556847in}{0.516222in}}{\pgfqpoint{1.962733in}{1.783528in}} %
\pgfusepath{clip}%
\pgfsetbuttcap%
\pgfsetroundjoin%
\definecolor{currentfill}{rgb}{0.298039,0.447059,0.690196}%
\pgfsetfillcolor{currentfill}%
\pgfsetlinewidth{0.240900pt}%
\definecolor{currentstroke}{rgb}{1.000000,1.000000,1.000000}%
\pgfsetstrokecolor{currentstroke}%
\pgfsetdash{}{0pt}%
\pgfpathmoveto{\pgfqpoint{1.902721in}{1.153989in}}%
\pgfpathcurveto{\pgfqpoint{1.910957in}{1.153989in}}{\pgfqpoint{1.918857in}{1.157261in}}{\pgfqpoint{1.924681in}{1.163085in}}%
\pgfpathcurveto{\pgfqpoint{1.930505in}{1.168909in}}{\pgfqpoint{1.933778in}{1.176809in}}{\pgfqpoint{1.933778in}{1.185045in}}%
\pgfpathcurveto{\pgfqpoint{1.933778in}{1.193281in}}{\pgfqpoint{1.930505in}{1.201181in}}{\pgfqpoint{1.924681in}{1.207005in}}%
\pgfpathcurveto{\pgfqpoint{1.918857in}{1.212829in}}{\pgfqpoint{1.910957in}{1.216102in}}{\pgfqpoint{1.902721in}{1.216102in}}%
\pgfpathcurveto{\pgfqpoint{1.894485in}{1.216102in}}{\pgfqpoint{1.886585in}{1.212829in}}{\pgfqpoint{1.880761in}{1.207005in}}%
\pgfpathcurveto{\pgfqpoint{1.874937in}{1.201181in}}{\pgfqpoint{1.871665in}{1.193281in}}{\pgfqpoint{1.871665in}{1.185045in}}%
\pgfpathcurveto{\pgfqpoint{1.871665in}{1.176809in}}{\pgfqpoint{1.874937in}{1.168909in}}{\pgfqpoint{1.880761in}{1.163085in}}%
\pgfpathcurveto{\pgfqpoint{1.886585in}{1.157261in}}{\pgfqpoint{1.894485in}{1.153989in}}{\pgfqpoint{1.902721in}{1.153989in}}%
\pgfpathclose%
\pgfusepath{stroke,fill}%
\end{pgfscope}%
\begin{pgfscope}%
\pgfpathrectangle{\pgfqpoint{0.556847in}{0.516222in}}{\pgfqpoint{1.962733in}{1.783528in}} %
\pgfusepath{clip}%
\pgfsetbuttcap%
\pgfsetroundjoin%
\definecolor{currentfill}{rgb}{0.298039,0.447059,0.690196}%
\pgfsetfillcolor{currentfill}%
\pgfsetlinewidth{0.240900pt}%
\definecolor{currentstroke}{rgb}{1.000000,1.000000,1.000000}%
\pgfsetstrokecolor{currentstroke}%
\pgfsetdash{}{0pt}%
\pgfpathmoveto{\pgfqpoint{1.622331in}{0.846330in}}%
\pgfpathcurveto{\pgfqpoint{1.630567in}{0.846330in}}{\pgfqpoint{1.638467in}{0.849602in}}{\pgfqpoint{1.644291in}{0.855426in}}%
\pgfpathcurveto{\pgfqpoint{1.650115in}{0.861250in}}{\pgfqpoint{1.653387in}{0.869150in}}{\pgfqpoint{1.653387in}{0.877387in}}%
\pgfpathcurveto{\pgfqpoint{1.653387in}{0.885623in}}{\pgfqpoint{1.650115in}{0.893523in}}{\pgfqpoint{1.644291in}{0.899347in}}%
\pgfpathcurveto{\pgfqpoint{1.638467in}{0.905171in}}{\pgfqpoint{1.630567in}{0.908443in}}{\pgfqpoint{1.622331in}{0.908443in}}%
\pgfpathcurveto{\pgfqpoint{1.614094in}{0.908443in}}{\pgfqpoint{1.606194in}{0.905171in}}{\pgfqpoint{1.600370in}{0.899347in}}%
\pgfpathcurveto{\pgfqpoint{1.594546in}{0.893523in}}{\pgfqpoint{1.591274in}{0.885623in}}{\pgfqpoint{1.591274in}{0.877387in}}%
\pgfpathcurveto{\pgfqpoint{1.591274in}{0.869150in}}{\pgfqpoint{1.594546in}{0.861250in}}{\pgfqpoint{1.600370in}{0.855426in}}%
\pgfpathcurveto{\pgfqpoint{1.606194in}{0.849602in}}{\pgfqpoint{1.614094in}{0.846330in}}{\pgfqpoint{1.622331in}{0.846330in}}%
\pgfpathclose%
\pgfusepath{stroke,fill}%
\end{pgfscope}%
\begin{pgfscope}%
\pgfpathrectangle{\pgfqpoint{0.556847in}{0.516222in}}{\pgfqpoint{1.962733in}{1.783528in}} %
\pgfusepath{clip}%
\pgfsetbuttcap%
\pgfsetroundjoin%
\definecolor{currentfill}{rgb}{0.298039,0.447059,0.690196}%
\pgfsetfillcolor{currentfill}%
\pgfsetlinewidth{0.240900pt}%
\definecolor{currentstroke}{rgb}{1.000000,1.000000,1.000000}%
\pgfsetstrokecolor{currentstroke}%
\pgfsetdash{}{0pt}%
\pgfpathmoveto{\pgfqpoint{1.958799in}{1.318965in}}%
\pgfpathcurveto{\pgfqpoint{1.967035in}{1.318965in}}{\pgfqpoint{1.974935in}{1.322237in}}{\pgfqpoint{1.980759in}{1.328061in}}%
\pgfpathcurveto{\pgfqpoint{1.986583in}{1.333885in}}{\pgfqpoint{1.989856in}{1.341785in}}{\pgfqpoint{1.989856in}{1.350021in}}%
\pgfpathcurveto{\pgfqpoint{1.989856in}{1.358258in}}{\pgfqpoint{1.986583in}{1.366158in}}{\pgfqpoint{1.980759in}{1.371982in}}%
\pgfpathcurveto{\pgfqpoint{1.974935in}{1.377806in}}{\pgfqpoint{1.967035in}{1.381078in}}{\pgfqpoint{1.958799in}{1.381078in}}%
\pgfpathcurveto{\pgfqpoint{1.950563in}{1.381078in}}{\pgfqpoint{1.942663in}{1.377806in}}{\pgfqpoint{1.936839in}{1.371982in}}%
\pgfpathcurveto{\pgfqpoint{1.931015in}{1.366158in}}{\pgfqpoint{1.927743in}{1.358258in}}{\pgfqpoint{1.927743in}{1.350021in}}%
\pgfpathcurveto{\pgfqpoint{1.927743in}{1.341785in}}{\pgfqpoint{1.931015in}{1.333885in}}{\pgfqpoint{1.936839in}{1.328061in}}%
\pgfpathcurveto{\pgfqpoint{1.942663in}{1.322237in}}{\pgfqpoint{1.950563in}{1.318965in}}{\pgfqpoint{1.958799in}{1.318965in}}%
\pgfpathclose%
\pgfusepath{stroke,fill}%
\end{pgfscope}%
\begin{pgfscope}%
\pgfpathrectangle{\pgfqpoint{0.556847in}{0.516222in}}{\pgfqpoint{1.962733in}{1.783528in}} %
\pgfusepath{clip}%
\pgfsetbuttcap%
\pgfsetroundjoin%
\definecolor{currentfill}{rgb}{0.298039,0.447059,0.690196}%
\pgfsetfillcolor{currentfill}%
\pgfsetlinewidth{0.240900pt}%
\definecolor{currentstroke}{rgb}{1.000000,1.000000,1.000000}%
\pgfsetstrokecolor{currentstroke}%
\pgfsetdash{}{0pt}%
\pgfpathmoveto{\pgfqpoint{2.239189in}{1.666753in}}%
\pgfpathcurveto{\pgfqpoint{2.247426in}{1.666753in}}{\pgfqpoint{2.255326in}{1.670025in}}{\pgfqpoint{2.261150in}{1.675849in}}%
\pgfpathcurveto{\pgfqpoint{2.266974in}{1.681673in}}{\pgfqpoint{2.270246in}{1.689573in}}{\pgfqpoint{2.270246in}{1.697809in}}%
\pgfpathcurveto{\pgfqpoint{2.270246in}{1.706046in}}{\pgfqpoint{2.266974in}{1.713946in}}{\pgfqpoint{2.261150in}{1.719770in}}%
\pgfpathcurveto{\pgfqpoint{2.255326in}{1.725594in}}{\pgfqpoint{2.247426in}{1.728866in}}{\pgfqpoint{2.239189in}{1.728866in}}%
\pgfpathcurveto{\pgfqpoint{2.230953in}{1.728866in}}{\pgfqpoint{2.223053in}{1.725594in}}{\pgfqpoint{2.217229in}{1.719770in}}%
\pgfpathcurveto{\pgfqpoint{2.211405in}{1.713946in}}{\pgfqpoint{2.208133in}{1.706046in}}{\pgfqpoint{2.208133in}{1.697809in}}%
\pgfpathcurveto{\pgfqpoint{2.208133in}{1.689573in}}{\pgfqpoint{2.211405in}{1.681673in}}{\pgfqpoint{2.217229in}{1.675849in}}%
\pgfpathcurveto{\pgfqpoint{2.223053in}{1.670025in}}{\pgfqpoint{2.230953in}{1.666753in}}{\pgfqpoint{2.239189in}{1.666753in}}%
\pgfpathclose%
\pgfusepath{stroke,fill}%
\end{pgfscope}%
\begin{pgfscope}%
\pgfpathrectangle{\pgfqpoint{0.556847in}{0.516222in}}{\pgfqpoint{1.962733in}{1.783528in}} %
\pgfusepath{clip}%
\pgfsetbuttcap%
\pgfsetroundjoin%
\definecolor{currentfill}{rgb}{0.298039,0.447059,0.690196}%
\pgfsetfillcolor{currentfill}%
\pgfsetlinewidth{0.240900pt}%
\definecolor{currentstroke}{rgb}{1.000000,1.000000,1.000000}%
\pgfsetstrokecolor{currentstroke}%
\pgfsetdash{}{0pt}%
\pgfpathmoveto{\pgfqpoint{1.622331in}{0.980095in}}%
\pgfpathcurveto{\pgfqpoint{1.630567in}{0.980095in}}{\pgfqpoint{1.638467in}{0.983367in}}{\pgfqpoint{1.644291in}{0.989191in}}%
\pgfpathcurveto{\pgfqpoint{1.650115in}{0.995015in}}{\pgfqpoint{1.653387in}{1.002915in}}{\pgfqpoint{1.653387in}{1.011151in}}%
\pgfpathcurveto{\pgfqpoint{1.653387in}{1.019387in}}{\pgfqpoint{1.650115in}{1.027288in}}{\pgfqpoint{1.644291in}{1.033111in}}%
\pgfpathcurveto{\pgfqpoint{1.638467in}{1.038935in}}{\pgfqpoint{1.630567in}{1.042208in}}{\pgfqpoint{1.622331in}{1.042208in}}%
\pgfpathcurveto{\pgfqpoint{1.614094in}{1.042208in}}{\pgfqpoint{1.606194in}{1.038935in}}{\pgfqpoint{1.600370in}{1.033111in}}%
\pgfpathcurveto{\pgfqpoint{1.594546in}{1.027288in}}{\pgfqpoint{1.591274in}{1.019387in}}{\pgfqpoint{1.591274in}{1.011151in}}%
\pgfpathcurveto{\pgfqpoint{1.591274in}{1.002915in}}{\pgfqpoint{1.594546in}{0.995015in}}{\pgfqpoint{1.600370in}{0.989191in}}%
\pgfpathcurveto{\pgfqpoint{1.606194in}{0.983367in}}{\pgfqpoint{1.614094in}{0.980095in}}{\pgfqpoint{1.622331in}{0.980095in}}%
\pgfpathclose%
\pgfusepath{stroke,fill}%
\end{pgfscope}%
\begin{pgfscope}%
\pgfpathrectangle{\pgfqpoint{0.556847in}{0.516222in}}{\pgfqpoint{1.962733in}{1.783528in}} %
\pgfusepath{clip}%
\pgfsetbuttcap%
\pgfsetroundjoin%
\definecolor{currentfill}{rgb}{0.298039,0.447059,0.690196}%
\pgfsetfillcolor{currentfill}%
\pgfsetlinewidth{0.240900pt}%
\definecolor{currentstroke}{rgb}{1.000000,1.000000,1.000000}%
\pgfsetstrokecolor{currentstroke}%
\pgfsetdash{}{0pt}%
\pgfpathmoveto{\pgfqpoint{2.407424in}{2.036835in}}%
\pgfpathcurveto{\pgfqpoint{2.415660in}{2.036835in}}{\pgfqpoint{2.423560in}{2.040107in}}{\pgfqpoint{2.429384in}{2.045931in}}%
\pgfpathcurveto{\pgfqpoint{2.435208in}{2.051755in}}{\pgfqpoint{2.438480in}{2.059655in}}{\pgfqpoint{2.438480in}{2.067891in}}%
\pgfpathcurveto{\pgfqpoint{2.438480in}{2.076128in}}{\pgfqpoint{2.435208in}{2.084028in}}{\pgfqpoint{2.429384in}{2.089852in}}%
\pgfpathcurveto{\pgfqpoint{2.423560in}{2.095676in}}{\pgfqpoint{2.415660in}{2.098948in}}{\pgfqpoint{2.407424in}{2.098948in}}%
\pgfpathcurveto{\pgfqpoint{2.399187in}{2.098948in}}{\pgfqpoint{2.391287in}{2.095676in}}{\pgfqpoint{2.385463in}{2.089852in}}%
\pgfpathcurveto{\pgfqpoint{2.379640in}{2.084028in}}{\pgfqpoint{2.376367in}{2.076128in}}{\pgfqpoint{2.376367in}{2.067891in}}%
\pgfpathcurveto{\pgfqpoint{2.376367in}{2.059655in}}{\pgfqpoint{2.379640in}{2.051755in}}{\pgfqpoint{2.385463in}{2.045931in}}%
\pgfpathcurveto{\pgfqpoint{2.391287in}{2.040107in}}{\pgfqpoint{2.399187in}{2.036835in}}{\pgfqpoint{2.407424in}{2.036835in}}%
\pgfpathclose%
\pgfusepath{stroke,fill}%
\end{pgfscope}%
\begin{pgfscope}%
\pgfpathrectangle{\pgfqpoint{0.556847in}{0.516222in}}{\pgfqpoint{1.962733in}{1.783528in}} %
\pgfusepath{clip}%
\pgfsetbuttcap%
\pgfsetroundjoin%
\definecolor{currentfill}{rgb}{0.298039,0.447059,0.690196}%
\pgfsetfillcolor{currentfill}%
\pgfsetlinewidth{0.240900pt}%
\definecolor{currentstroke}{rgb}{1.000000,1.000000,1.000000}%
\pgfsetstrokecolor{currentstroke}%
\pgfsetdash{}{0pt}%
\pgfpathmoveto{\pgfqpoint{1.622331in}{0.739318in}}%
\pgfpathcurveto{\pgfqpoint{1.630567in}{0.739318in}}{\pgfqpoint{1.638467in}{0.742591in}}{\pgfqpoint{1.644291in}{0.748415in}}%
\pgfpathcurveto{\pgfqpoint{1.650115in}{0.754239in}}{\pgfqpoint{1.653387in}{0.762139in}}{\pgfqpoint{1.653387in}{0.770375in}}%
\pgfpathcurveto{\pgfqpoint{1.653387in}{0.778611in}}{\pgfqpoint{1.650115in}{0.786511in}}{\pgfqpoint{1.644291in}{0.792335in}}%
\pgfpathcurveto{\pgfqpoint{1.638467in}{0.798159in}}{\pgfqpoint{1.630567in}{0.801431in}}{\pgfqpoint{1.622331in}{0.801431in}}%
\pgfpathcurveto{\pgfqpoint{1.614094in}{0.801431in}}{\pgfqpoint{1.606194in}{0.798159in}}{\pgfqpoint{1.600370in}{0.792335in}}%
\pgfpathcurveto{\pgfqpoint{1.594546in}{0.786511in}}{\pgfqpoint{1.591274in}{0.778611in}}{\pgfqpoint{1.591274in}{0.770375in}}%
\pgfpathcurveto{\pgfqpoint{1.591274in}{0.762139in}}{\pgfqpoint{1.594546in}{0.754239in}}{\pgfqpoint{1.600370in}{0.748415in}}%
\pgfpathcurveto{\pgfqpoint{1.606194in}{0.742591in}}{\pgfqpoint{1.614094in}{0.739318in}}{\pgfqpoint{1.622331in}{0.739318in}}%
\pgfpathclose%
\pgfusepath{stroke,fill}%
\end{pgfscope}%
\begin{pgfscope}%
\pgfpathrectangle{\pgfqpoint{0.556847in}{0.516222in}}{\pgfqpoint{1.962733in}{1.783528in}} %
\pgfusepath{clip}%
\pgfsetbuttcap%
\pgfsetroundjoin%
\definecolor{currentfill}{rgb}{0.298039,0.447059,0.690196}%
\pgfsetfillcolor{currentfill}%
\pgfsetlinewidth{0.240900pt}%
\definecolor{currentstroke}{rgb}{1.000000,1.000000,1.000000}%
\pgfsetstrokecolor{currentstroke}%
\pgfsetdash{}{0pt}%
\pgfpathmoveto{\pgfqpoint{2.407424in}{1.760388in}}%
\pgfpathcurveto{\pgfqpoint{2.415660in}{1.760388in}}{\pgfqpoint{2.423560in}{1.763660in}}{\pgfqpoint{2.429384in}{1.769484in}}%
\pgfpathcurveto{\pgfqpoint{2.435208in}{1.775308in}}{\pgfqpoint{2.438480in}{1.783208in}}{\pgfqpoint{2.438480in}{1.791445in}}%
\pgfpathcurveto{\pgfqpoint{2.438480in}{1.799681in}}{\pgfqpoint{2.435208in}{1.807581in}}{\pgfqpoint{2.429384in}{1.813405in}}%
\pgfpathcurveto{\pgfqpoint{2.423560in}{1.819229in}}{\pgfqpoint{2.415660in}{1.822501in}}{\pgfqpoint{2.407424in}{1.822501in}}%
\pgfpathcurveto{\pgfqpoint{2.399187in}{1.822501in}}{\pgfqpoint{2.391287in}{1.819229in}}{\pgfqpoint{2.385463in}{1.813405in}}%
\pgfpathcurveto{\pgfqpoint{2.379640in}{1.807581in}}{\pgfqpoint{2.376367in}{1.799681in}}{\pgfqpoint{2.376367in}{1.791445in}}%
\pgfpathcurveto{\pgfqpoint{2.376367in}{1.783208in}}{\pgfqpoint{2.379640in}{1.775308in}}{\pgfqpoint{2.385463in}{1.769484in}}%
\pgfpathcurveto{\pgfqpoint{2.391287in}{1.763660in}}{\pgfqpoint{2.399187in}{1.760388in}}{\pgfqpoint{2.407424in}{1.760388in}}%
\pgfpathclose%
\pgfusepath{stroke,fill}%
\end{pgfscope}%
\begin{pgfscope}%
\pgfpathrectangle{\pgfqpoint{0.556847in}{0.516222in}}{\pgfqpoint{1.962733in}{1.783528in}} %
\pgfusepath{clip}%
\pgfsetbuttcap%
\pgfsetroundjoin%
\definecolor{currentfill}{rgb}{0.298039,0.447059,0.690196}%
\pgfsetfillcolor{currentfill}%
\pgfsetlinewidth{0.240900pt}%
\definecolor{currentstroke}{rgb}{1.000000,1.000000,1.000000}%
\pgfsetstrokecolor{currentstroke}%
\pgfsetdash{}{0pt}%
\pgfpathmoveto{\pgfqpoint{1.173706in}{0.837412in}}%
\pgfpathcurveto{\pgfqpoint{1.181942in}{0.837412in}}{\pgfqpoint{1.189842in}{0.840685in}}{\pgfqpoint{1.195666in}{0.846509in}}%
\pgfpathcurveto{\pgfqpoint{1.201490in}{0.852333in}}{\pgfqpoint{1.204763in}{0.860233in}}{\pgfqpoint{1.204763in}{0.868469in}}%
\pgfpathcurveto{\pgfqpoint{1.204763in}{0.876705in}}{\pgfqpoint{1.201490in}{0.884605in}}{\pgfqpoint{1.195666in}{0.890429in}}%
\pgfpathcurveto{\pgfqpoint{1.189842in}{0.896253in}}{\pgfqpoint{1.181942in}{0.899525in}}{\pgfqpoint{1.173706in}{0.899525in}}%
\pgfpathcurveto{\pgfqpoint{1.165470in}{0.899525in}}{\pgfqpoint{1.157570in}{0.896253in}}{\pgfqpoint{1.151746in}{0.890429in}}%
\pgfpathcurveto{\pgfqpoint{1.145922in}{0.884605in}}{\pgfqpoint{1.142650in}{0.876705in}}{\pgfqpoint{1.142650in}{0.868469in}}%
\pgfpathcurveto{\pgfqpoint{1.142650in}{0.860233in}}{\pgfqpoint{1.145922in}{0.852333in}}{\pgfqpoint{1.151746in}{0.846509in}}%
\pgfpathcurveto{\pgfqpoint{1.157570in}{0.840685in}}{\pgfqpoint{1.165470in}{0.837412in}}{\pgfqpoint{1.173706in}{0.837412in}}%
\pgfpathclose%
\pgfusepath{stroke,fill}%
\end{pgfscope}%
\begin{pgfscope}%
\pgfpathrectangle{\pgfqpoint{0.556847in}{0.516222in}}{\pgfqpoint{1.962733in}{1.783528in}} %
\pgfusepath{clip}%
\pgfsetbuttcap%
\pgfsetroundjoin%
\definecolor{currentfill}{rgb}{0.298039,0.447059,0.690196}%
\pgfsetfillcolor{currentfill}%
\pgfsetlinewidth{0.240900pt}%
\definecolor{currentstroke}{rgb}{1.000000,1.000000,1.000000}%
\pgfsetstrokecolor{currentstroke}%
\pgfsetdash{}{0pt}%
\pgfpathmoveto{\pgfqpoint{1.678409in}{1.394765in}}%
\pgfpathcurveto{\pgfqpoint{1.686645in}{1.394765in}}{\pgfqpoint{1.694545in}{1.398037in}}{\pgfqpoint{1.700369in}{1.403861in}}%
\pgfpathcurveto{\pgfqpoint{1.706193in}{1.409685in}}{\pgfqpoint{1.709465in}{1.417585in}}{\pgfqpoint{1.709465in}{1.425821in}}%
\pgfpathcurveto{\pgfqpoint{1.709465in}{1.434058in}}{\pgfqpoint{1.706193in}{1.441958in}}{\pgfqpoint{1.700369in}{1.447782in}}%
\pgfpathcurveto{\pgfqpoint{1.694545in}{1.453606in}}{\pgfqpoint{1.686645in}{1.456878in}}{\pgfqpoint{1.678409in}{1.456878in}}%
\pgfpathcurveto{\pgfqpoint{1.670172in}{1.456878in}}{\pgfqpoint{1.662272in}{1.453606in}}{\pgfqpoint{1.656448in}{1.447782in}}%
\pgfpathcurveto{\pgfqpoint{1.650625in}{1.441958in}}{\pgfqpoint{1.647352in}{1.434058in}}{\pgfqpoint{1.647352in}{1.425821in}}%
\pgfpathcurveto{\pgfqpoint{1.647352in}{1.417585in}}{\pgfqpoint{1.650625in}{1.409685in}}{\pgfqpoint{1.656448in}{1.403861in}}%
\pgfpathcurveto{\pgfqpoint{1.662272in}{1.398037in}}{\pgfqpoint{1.670172in}{1.394765in}}{\pgfqpoint{1.678409in}{1.394765in}}%
\pgfpathclose%
\pgfusepath{stroke,fill}%
\end{pgfscope}%
\begin{pgfscope}%
\pgfpathrectangle{\pgfqpoint{0.556847in}{0.516222in}}{\pgfqpoint{1.962733in}{1.783528in}} %
\pgfusepath{clip}%
\pgfsetbuttcap%
\pgfsetroundjoin%
\definecolor{currentfill}{rgb}{0.298039,0.447059,0.690196}%
\pgfsetfillcolor{currentfill}%
\pgfsetlinewidth{0.240900pt}%
\definecolor{currentstroke}{rgb}{1.000000,1.000000,1.000000}%
\pgfsetstrokecolor{currentstroke}%
\pgfsetdash{}{0pt}%
\pgfpathmoveto{\pgfqpoint{1.510175in}{0.806201in}}%
\pgfpathcurveto{\pgfqpoint{1.518411in}{0.806201in}}{\pgfqpoint{1.526311in}{0.809473in}}{\pgfqpoint{1.532135in}{0.815297in}}%
\pgfpathcurveto{\pgfqpoint{1.537959in}{0.821121in}}{\pgfqpoint{1.541231in}{0.829021in}}{\pgfqpoint{1.541231in}{0.837257in}}%
\pgfpathcurveto{\pgfqpoint{1.541231in}{0.845494in}}{\pgfqpoint{1.537959in}{0.853394in}}{\pgfqpoint{1.532135in}{0.859217in}}%
\pgfpathcurveto{\pgfqpoint{1.526311in}{0.865041in}}{\pgfqpoint{1.518411in}{0.868314in}}{\pgfqpoint{1.510175in}{0.868314in}}%
\pgfpathcurveto{\pgfqpoint{1.501938in}{0.868314in}}{\pgfqpoint{1.494038in}{0.865041in}}{\pgfqpoint{1.488214in}{0.859217in}}%
\pgfpathcurveto{\pgfqpoint{1.482390in}{0.853394in}}{\pgfqpoint{1.479118in}{0.845494in}}{\pgfqpoint{1.479118in}{0.837257in}}%
\pgfpathcurveto{\pgfqpoint{1.479118in}{0.829021in}}{\pgfqpoint{1.482390in}{0.821121in}}{\pgfqpoint{1.488214in}{0.815297in}}%
\pgfpathcurveto{\pgfqpoint{1.494038in}{0.809473in}}{\pgfqpoint{1.501938in}{0.806201in}}{\pgfqpoint{1.510175in}{0.806201in}}%
\pgfpathclose%
\pgfusepath{stroke,fill}%
\end{pgfscope}%
\begin{pgfscope}%
\pgfpathrectangle{\pgfqpoint{0.556847in}{0.516222in}}{\pgfqpoint{1.962733in}{1.783528in}} %
\pgfusepath{clip}%
\pgfsetbuttcap%
\pgfsetroundjoin%
\definecolor{currentfill}{rgb}{0.298039,0.447059,0.690196}%
\pgfsetfillcolor{currentfill}%
\pgfsetlinewidth{0.240900pt}%
\definecolor{currentstroke}{rgb}{1.000000,1.000000,1.000000}%
\pgfsetstrokecolor{currentstroke}%
\pgfsetdash{}{0pt}%
\pgfpathmoveto{\pgfqpoint{1.678409in}{0.792824in}}%
\pgfpathcurveto{\pgfqpoint{1.686645in}{0.792824in}}{\pgfqpoint{1.694545in}{0.796097in}}{\pgfqpoint{1.700369in}{0.801921in}}%
\pgfpathcurveto{\pgfqpoint{1.706193in}{0.807744in}}{\pgfqpoint{1.709465in}{0.815644in}}{\pgfqpoint{1.709465in}{0.823881in}}%
\pgfpathcurveto{\pgfqpoint{1.709465in}{0.832117in}}{\pgfqpoint{1.706193in}{0.840017in}}{\pgfqpoint{1.700369in}{0.845841in}}%
\pgfpathcurveto{\pgfqpoint{1.694545in}{0.851665in}}{\pgfqpoint{1.686645in}{0.854937in}}{\pgfqpoint{1.678409in}{0.854937in}}%
\pgfpathcurveto{\pgfqpoint{1.670172in}{0.854937in}}{\pgfqpoint{1.662272in}{0.851665in}}{\pgfqpoint{1.656448in}{0.845841in}}%
\pgfpathcurveto{\pgfqpoint{1.650625in}{0.840017in}}{\pgfqpoint{1.647352in}{0.832117in}}{\pgfqpoint{1.647352in}{0.823881in}}%
\pgfpathcurveto{\pgfqpoint{1.647352in}{0.815644in}}{\pgfqpoint{1.650625in}{0.807744in}}{\pgfqpoint{1.656448in}{0.801921in}}%
\pgfpathcurveto{\pgfqpoint{1.662272in}{0.796097in}}{\pgfqpoint{1.670172in}{0.792824in}}{\pgfqpoint{1.678409in}{0.792824in}}%
\pgfpathclose%
\pgfusepath{stroke,fill}%
\end{pgfscope}%
\begin{pgfscope}%
\pgfpathrectangle{\pgfqpoint{0.556847in}{0.516222in}}{\pgfqpoint{1.962733in}{1.783528in}} %
\pgfusepath{clip}%
\pgfsetbuttcap%
\pgfsetroundjoin%
\definecolor{currentfill}{rgb}{0.298039,0.447059,0.690196}%
\pgfsetfillcolor{currentfill}%
\pgfsetlinewidth{0.240900pt}%
\definecolor{currentstroke}{rgb}{1.000000,1.000000,1.000000}%
\pgfsetstrokecolor{currentstroke}%
\pgfsetdash{}{0pt}%
\pgfpathmoveto{\pgfqpoint{1.790565in}{2.094800in}}%
\pgfpathcurveto{\pgfqpoint{1.798801in}{2.094800in}}{\pgfqpoint{1.806701in}{2.098072in}}{\pgfqpoint{1.812525in}{2.103896in}}%
\pgfpathcurveto{\pgfqpoint{1.818349in}{2.109720in}}{\pgfqpoint{1.821621in}{2.117620in}}{\pgfqpoint{1.821621in}{2.125856in}}%
\pgfpathcurveto{\pgfqpoint{1.821621in}{2.134092in}}{\pgfqpoint{1.818349in}{2.141992in}}{\pgfqpoint{1.812525in}{2.147816in}}%
\pgfpathcurveto{\pgfqpoint{1.806701in}{2.153640in}}{\pgfqpoint{1.798801in}{2.156913in}}{\pgfqpoint{1.790565in}{2.156913in}}%
\pgfpathcurveto{\pgfqpoint{1.782329in}{2.156913in}}{\pgfqpoint{1.774429in}{2.153640in}}{\pgfqpoint{1.768605in}{2.147816in}}%
\pgfpathcurveto{\pgfqpoint{1.762781in}{2.141992in}}{\pgfqpoint{1.759508in}{2.134092in}}{\pgfqpoint{1.759508in}{2.125856in}}%
\pgfpathcurveto{\pgfqpoint{1.759508in}{2.117620in}}{\pgfqpoint{1.762781in}{2.109720in}}{\pgfqpoint{1.768605in}{2.103896in}}%
\pgfpathcurveto{\pgfqpoint{1.774429in}{2.098072in}}{\pgfqpoint{1.782329in}{2.094800in}}{\pgfqpoint{1.790565in}{2.094800in}}%
\pgfpathclose%
\pgfusepath{stroke,fill}%
\end{pgfscope}%
\begin{pgfscope}%
\pgfpathrectangle{\pgfqpoint{0.556847in}{0.516222in}}{\pgfqpoint{1.962733in}{1.783528in}} %
\pgfusepath{clip}%
\pgfsetbuttcap%
\pgfsetroundjoin%
\definecolor{currentfill}{rgb}{0.298039,0.447059,0.690196}%
\pgfsetfillcolor{currentfill}%
\pgfsetlinewidth{0.240900pt}%
\definecolor{currentstroke}{rgb}{1.000000,1.000000,1.000000}%
\pgfsetstrokecolor{currentstroke}%
\pgfsetdash{}{0pt}%
\pgfpathmoveto{\pgfqpoint{1.061550in}{1.033601in}}%
\pgfpathcurveto{\pgfqpoint{1.069786in}{1.033601in}}{\pgfqpoint{1.077686in}{1.036873in}}{\pgfqpoint{1.083510in}{1.042697in}}%
\pgfpathcurveto{\pgfqpoint{1.089334in}{1.048521in}}{\pgfqpoint{1.092606in}{1.056421in}}{\pgfqpoint{1.092606in}{1.064657in}}%
\pgfpathcurveto{\pgfqpoint{1.092606in}{1.072893in}}{\pgfqpoint{1.089334in}{1.080793in}}{\pgfqpoint{1.083510in}{1.086617in}}%
\pgfpathcurveto{\pgfqpoint{1.077686in}{1.092441in}}{\pgfqpoint{1.069786in}{1.095714in}}{\pgfqpoint{1.061550in}{1.095714in}}%
\pgfpathcurveto{\pgfqpoint{1.053314in}{1.095714in}}{\pgfqpoint{1.045414in}{1.092441in}}{\pgfqpoint{1.039590in}{1.086617in}}%
\pgfpathcurveto{\pgfqpoint{1.033766in}{1.080793in}}{\pgfqpoint{1.030493in}{1.072893in}}{\pgfqpoint{1.030493in}{1.064657in}}%
\pgfpathcurveto{\pgfqpoint{1.030493in}{1.056421in}}{\pgfqpoint{1.033766in}{1.048521in}}{\pgfqpoint{1.039590in}{1.042697in}}%
\pgfpathcurveto{\pgfqpoint{1.045414in}{1.036873in}}{\pgfqpoint{1.053314in}{1.033601in}}{\pgfqpoint{1.061550in}{1.033601in}}%
\pgfpathclose%
\pgfusepath{stroke,fill}%
\end{pgfscope}%
\begin{pgfscope}%
\pgfpathrectangle{\pgfqpoint{0.556847in}{0.516222in}}{\pgfqpoint{1.962733in}{1.783528in}} %
\pgfusepath{clip}%
\pgfsetbuttcap%
\pgfsetroundjoin%
\definecolor{currentfill}{rgb}{0.298039,0.447059,0.690196}%
\pgfsetfillcolor{currentfill}%
\pgfsetlinewidth{0.240900pt}%
\definecolor{currentstroke}{rgb}{1.000000,1.000000,1.000000}%
\pgfsetstrokecolor{currentstroke}%
\pgfsetdash{}{0pt}%
\pgfpathmoveto{\pgfqpoint{2.070955in}{1.457188in}}%
\pgfpathcurveto{\pgfqpoint{2.079192in}{1.457188in}}{\pgfqpoint{2.087092in}{1.460461in}}{\pgfqpoint{2.092916in}{1.466285in}}%
\pgfpathcurveto{\pgfqpoint{2.098739in}{1.472109in}}{\pgfqpoint{2.102012in}{1.480009in}}{\pgfqpoint{2.102012in}{1.488245in}}%
\pgfpathcurveto{\pgfqpoint{2.102012in}{1.496481in}}{\pgfqpoint{2.098739in}{1.504381in}}{\pgfqpoint{2.092916in}{1.510205in}}%
\pgfpathcurveto{\pgfqpoint{2.087092in}{1.516029in}}{\pgfqpoint{2.079192in}{1.519301in}}{\pgfqpoint{2.070955in}{1.519301in}}%
\pgfpathcurveto{\pgfqpoint{2.062719in}{1.519301in}}{\pgfqpoint{2.054819in}{1.516029in}}{\pgfqpoint{2.048995in}{1.510205in}}%
\pgfpathcurveto{\pgfqpoint{2.043171in}{1.504381in}}{\pgfqpoint{2.039899in}{1.496481in}}{\pgfqpoint{2.039899in}{1.488245in}}%
\pgfpathcurveto{\pgfqpoint{2.039899in}{1.480009in}}{\pgfqpoint{2.043171in}{1.472109in}}{\pgfqpoint{2.048995in}{1.466285in}}%
\pgfpathcurveto{\pgfqpoint{2.054819in}{1.460461in}}{\pgfqpoint{2.062719in}{1.457188in}}{\pgfqpoint{2.070955in}{1.457188in}}%
\pgfpathclose%
\pgfusepath{stroke,fill}%
\end{pgfscope}%
\begin{pgfscope}%
\pgfpathrectangle{\pgfqpoint{0.556847in}{0.516222in}}{\pgfqpoint{1.962733in}{1.783528in}} %
\pgfusepath{clip}%
\pgfsetbuttcap%
\pgfsetroundjoin%
\definecolor{currentfill}{rgb}{0.298039,0.447059,0.690196}%
\pgfsetfillcolor{currentfill}%
\pgfsetlinewidth{0.240900pt}%
\definecolor{currentstroke}{rgb}{1.000000,1.000000,1.000000}%
\pgfsetstrokecolor{currentstroke}%
\pgfsetdash{}{0pt}%
\pgfpathmoveto{\pgfqpoint{1.958799in}{1.372471in}}%
\pgfpathcurveto{\pgfqpoint{1.967035in}{1.372471in}}{\pgfqpoint{1.974935in}{1.375743in}}{\pgfqpoint{1.980759in}{1.381567in}}%
\pgfpathcurveto{\pgfqpoint{1.986583in}{1.387391in}}{\pgfqpoint{1.989856in}{1.395291in}}{\pgfqpoint{1.989856in}{1.403527in}}%
\pgfpathcurveto{\pgfqpoint{1.989856in}{1.411764in}}{\pgfqpoint{1.986583in}{1.419664in}}{\pgfqpoint{1.980759in}{1.425488in}}%
\pgfpathcurveto{\pgfqpoint{1.974935in}{1.431311in}}{\pgfqpoint{1.967035in}{1.434584in}}{\pgfqpoint{1.958799in}{1.434584in}}%
\pgfpathcurveto{\pgfqpoint{1.950563in}{1.434584in}}{\pgfqpoint{1.942663in}{1.431311in}}{\pgfqpoint{1.936839in}{1.425488in}}%
\pgfpathcurveto{\pgfqpoint{1.931015in}{1.419664in}}{\pgfqpoint{1.927743in}{1.411764in}}{\pgfqpoint{1.927743in}{1.403527in}}%
\pgfpathcurveto{\pgfqpoint{1.927743in}{1.395291in}}{\pgfqpoint{1.931015in}{1.387391in}}{\pgfqpoint{1.936839in}{1.381567in}}%
\pgfpathcurveto{\pgfqpoint{1.942663in}{1.375743in}}{\pgfqpoint{1.950563in}{1.372471in}}{\pgfqpoint{1.958799in}{1.372471in}}%
\pgfpathclose%
\pgfusepath{stroke,fill}%
\end{pgfscope}%
\begin{pgfscope}%
\pgfpathrectangle{\pgfqpoint{0.556847in}{0.516222in}}{\pgfqpoint{1.962733in}{1.783528in}} %
\pgfusepath{clip}%
\pgfsetbuttcap%
\pgfsetroundjoin%
\definecolor{currentfill}{rgb}{0.298039,0.447059,0.690196}%
\pgfsetfillcolor{currentfill}%
\pgfsetlinewidth{0.240900pt}%
\definecolor{currentstroke}{rgb}{1.000000,1.000000,1.000000}%
\pgfsetstrokecolor{currentstroke}%
\pgfsetdash{}{0pt}%
\pgfpathmoveto{\pgfqpoint{1.510175in}{0.672436in}}%
\pgfpathcurveto{\pgfqpoint{1.518411in}{0.672436in}}{\pgfqpoint{1.526311in}{0.675708in}}{\pgfqpoint{1.532135in}{0.681532in}}%
\pgfpathcurveto{\pgfqpoint{1.537959in}{0.687356in}}{\pgfqpoint{1.541231in}{0.695256in}}{\pgfqpoint{1.541231in}{0.703493in}}%
\pgfpathcurveto{\pgfqpoint{1.541231in}{0.711729in}}{\pgfqpoint{1.537959in}{0.719629in}}{\pgfqpoint{1.532135in}{0.725453in}}%
\pgfpathcurveto{\pgfqpoint{1.526311in}{0.731277in}}{\pgfqpoint{1.518411in}{0.734549in}}{\pgfqpoint{1.510175in}{0.734549in}}%
\pgfpathcurveto{\pgfqpoint{1.501938in}{0.734549in}}{\pgfqpoint{1.494038in}{0.731277in}}{\pgfqpoint{1.488214in}{0.725453in}}%
\pgfpathcurveto{\pgfqpoint{1.482390in}{0.719629in}}{\pgfqpoint{1.479118in}{0.711729in}}{\pgfqpoint{1.479118in}{0.703493in}}%
\pgfpathcurveto{\pgfqpoint{1.479118in}{0.695256in}}{\pgfqpoint{1.482390in}{0.687356in}}{\pgfqpoint{1.488214in}{0.681532in}}%
\pgfpathcurveto{\pgfqpoint{1.494038in}{0.675708in}}{\pgfqpoint{1.501938in}{0.672436in}}{\pgfqpoint{1.510175in}{0.672436in}}%
\pgfpathclose%
\pgfusepath{stroke,fill}%
\end{pgfscope}%
\begin{pgfscope}%
\pgfpathrectangle{\pgfqpoint{0.556847in}{0.516222in}}{\pgfqpoint{1.962733in}{1.783528in}} %
\pgfusepath{clip}%
\pgfsetbuttcap%
\pgfsetroundjoin%
\definecolor{currentfill}{rgb}{0.298039,0.447059,0.690196}%
\pgfsetfillcolor{currentfill}%
\pgfsetlinewidth{0.240900pt}%
\definecolor{currentstroke}{rgb}{1.000000,1.000000,1.000000}%
\pgfsetstrokecolor{currentstroke}%
\pgfsetdash{}{0pt}%
\pgfpathmoveto{\pgfqpoint{1.678409in}{0.895377in}}%
\pgfpathcurveto{\pgfqpoint{1.686645in}{0.895377in}}{\pgfqpoint{1.694545in}{0.898649in}}{\pgfqpoint{1.700369in}{0.904473in}}%
\pgfpathcurveto{\pgfqpoint{1.706193in}{0.910297in}}{\pgfqpoint{1.709465in}{0.918197in}}{\pgfqpoint{1.709465in}{0.926434in}}%
\pgfpathcurveto{\pgfqpoint{1.709465in}{0.934670in}}{\pgfqpoint{1.706193in}{0.942570in}}{\pgfqpoint{1.700369in}{0.948394in}}%
\pgfpathcurveto{\pgfqpoint{1.694545in}{0.954218in}}{\pgfqpoint{1.686645in}{0.957490in}}{\pgfqpoint{1.678409in}{0.957490in}}%
\pgfpathcurveto{\pgfqpoint{1.670172in}{0.957490in}}{\pgfqpoint{1.662272in}{0.954218in}}{\pgfqpoint{1.656448in}{0.948394in}}%
\pgfpathcurveto{\pgfqpoint{1.650625in}{0.942570in}}{\pgfqpoint{1.647352in}{0.934670in}}{\pgfqpoint{1.647352in}{0.926434in}}%
\pgfpathcurveto{\pgfqpoint{1.647352in}{0.918197in}}{\pgfqpoint{1.650625in}{0.910297in}}{\pgfqpoint{1.656448in}{0.904473in}}%
\pgfpathcurveto{\pgfqpoint{1.662272in}{0.898649in}}{\pgfqpoint{1.670172in}{0.895377in}}{\pgfqpoint{1.678409in}{0.895377in}}%
\pgfpathclose%
\pgfusepath{stroke,fill}%
\end{pgfscope}%
\begin{pgfscope}%
\pgfpathrectangle{\pgfqpoint{0.556847in}{0.516222in}}{\pgfqpoint{1.962733in}{1.783528in}} %
\pgfusepath{clip}%
\pgfsetbuttcap%
\pgfsetroundjoin%
\definecolor{currentfill}{rgb}{0.298039,0.447059,0.690196}%
\pgfsetfillcolor{currentfill}%
\pgfsetlinewidth{0.240900pt}%
\definecolor{currentstroke}{rgb}{1.000000,1.000000,1.000000}%
\pgfsetstrokecolor{currentstroke}%
\pgfsetdash{}{0pt}%
\pgfpathmoveto{\pgfqpoint{1.285862in}{0.975636in}}%
\pgfpathcurveto{\pgfqpoint{1.294098in}{0.975636in}}{\pgfqpoint{1.301999in}{0.978908in}}{\pgfqpoint{1.307822in}{0.984732in}}%
\pgfpathcurveto{\pgfqpoint{1.313646in}{0.990556in}}{\pgfqpoint{1.316919in}{0.998456in}}{\pgfqpoint{1.316919in}{1.006692in}}%
\pgfpathcurveto{\pgfqpoint{1.316919in}{1.014929in}}{\pgfqpoint{1.313646in}{1.022829in}}{\pgfqpoint{1.307822in}{1.028653in}}%
\pgfpathcurveto{\pgfqpoint{1.301999in}{1.034477in}}{\pgfqpoint{1.294098in}{1.037749in}}{\pgfqpoint{1.285862in}{1.037749in}}%
\pgfpathcurveto{\pgfqpoint{1.277626in}{1.037749in}}{\pgfqpoint{1.269726in}{1.034477in}}{\pgfqpoint{1.263902in}{1.028653in}}%
\pgfpathcurveto{\pgfqpoint{1.258078in}{1.022829in}}{\pgfqpoint{1.254806in}{1.014929in}}{\pgfqpoint{1.254806in}{1.006692in}}%
\pgfpathcurveto{\pgfqpoint{1.254806in}{0.998456in}}{\pgfqpoint{1.258078in}{0.990556in}}{\pgfqpoint{1.263902in}{0.984732in}}%
\pgfpathcurveto{\pgfqpoint{1.269726in}{0.978908in}}{\pgfqpoint{1.277626in}{0.975636in}}{\pgfqpoint{1.285862in}{0.975636in}}%
\pgfpathclose%
\pgfusepath{stroke,fill}%
\end{pgfscope}%
\begin{pgfscope}%
\pgfpathrectangle{\pgfqpoint{0.556847in}{0.516222in}}{\pgfqpoint{1.962733in}{1.783528in}} %
\pgfusepath{clip}%
\pgfsetbuttcap%
\pgfsetroundjoin%
\definecolor{currentfill}{rgb}{0.298039,0.447059,0.690196}%
\pgfsetfillcolor{currentfill}%
\pgfsetlinewidth{0.240900pt}%
\definecolor{currentstroke}{rgb}{1.000000,1.000000,1.000000}%
\pgfsetstrokecolor{currentstroke}%
\pgfsetdash{}{0pt}%
\pgfpathmoveto{\pgfqpoint{1.454096in}{0.730401in}}%
\pgfpathcurveto{\pgfqpoint{1.462333in}{0.730401in}}{\pgfqpoint{1.470233in}{0.733673in}}{\pgfqpoint{1.476057in}{0.739497in}}%
\pgfpathcurveto{\pgfqpoint{1.481881in}{0.745321in}}{\pgfqpoint{1.485153in}{0.753221in}}{\pgfqpoint{1.485153in}{0.761457in}}%
\pgfpathcurveto{\pgfqpoint{1.485153in}{0.769694in}}{\pgfqpoint{1.481881in}{0.777594in}}{\pgfqpoint{1.476057in}{0.783418in}}%
\pgfpathcurveto{\pgfqpoint{1.470233in}{0.789241in}}{\pgfqpoint{1.462333in}{0.792514in}}{\pgfqpoint{1.454096in}{0.792514in}}%
\pgfpathcurveto{\pgfqpoint{1.445860in}{0.792514in}}{\pgfqpoint{1.437960in}{0.789241in}}{\pgfqpoint{1.432136in}{0.783418in}}%
\pgfpathcurveto{\pgfqpoint{1.426312in}{0.777594in}}{\pgfqpoint{1.423040in}{0.769694in}}{\pgfqpoint{1.423040in}{0.761457in}}%
\pgfpathcurveto{\pgfqpoint{1.423040in}{0.753221in}}{\pgfqpoint{1.426312in}{0.745321in}}{\pgfqpoint{1.432136in}{0.739497in}}%
\pgfpathcurveto{\pgfqpoint{1.437960in}{0.733673in}}{\pgfqpoint{1.445860in}{0.730401in}}{\pgfqpoint{1.454096in}{0.730401in}}%
\pgfpathclose%
\pgfusepath{stroke,fill}%
\end{pgfscope}%
\begin{pgfscope}%
\pgfpathrectangle{\pgfqpoint{0.556847in}{0.516222in}}{\pgfqpoint{1.962733in}{1.783528in}} %
\pgfusepath{clip}%
\pgfsetbuttcap%
\pgfsetroundjoin%
\definecolor{currentfill}{rgb}{0.298039,0.447059,0.690196}%
\pgfsetfillcolor{currentfill}%
\pgfsetlinewidth{0.240900pt}%
\definecolor{currentstroke}{rgb}{1.000000,1.000000,1.000000}%
\pgfsetstrokecolor{currentstroke}%
\pgfsetdash{}{0pt}%
\pgfpathmoveto{\pgfqpoint{1.398018in}{1.956576in}}%
\pgfpathcurveto{\pgfqpoint{1.406255in}{1.956576in}}{\pgfqpoint{1.414155in}{1.959848in}}{\pgfqpoint{1.419979in}{1.965672in}}%
\pgfpathcurveto{\pgfqpoint{1.425803in}{1.971496in}}{\pgfqpoint{1.429075in}{1.979396in}}{\pgfqpoint{1.429075in}{1.987633in}}%
\pgfpathcurveto{\pgfqpoint{1.429075in}{1.995869in}}{\pgfqpoint{1.425803in}{2.003769in}}{\pgfqpoint{1.419979in}{2.009593in}}%
\pgfpathcurveto{\pgfqpoint{1.414155in}{2.015417in}}{\pgfqpoint{1.406255in}{2.018689in}}{\pgfqpoint{1.398018in}{2.018689in}}%
\pgfpathcurveto{\pgfqpoint{1.389782in}{2.018689in}}{\pgfqpoint{1.381882in}{2.015417in}}{\pgfqpoint{1.376058in}{2.009593in}}%
\pgfpathcurveto{\pgfqpoint{1.370234in}{2.003769in}}{\pgfqpoint{1.366962in}{1.995869in}}{\pgfqpoint{1.366962in}{1.987633in}}%
\pgfpathcurveto{\pgfqpoint{1.366962in}{1.979396in}}{\pgfqpoint{1.370234in}{1.971496in}}{\pgfqpoint{1.376058in}{1.965672in}}%
\pgfpathcurveto{\pgfqpoint{1.381882in}{1.959848in}}{\pgfqpoint{1.389782in}{1.956576in}}{\pgfqpoint{1.398018in}{1.956576in}}%
\pgfpathclose%
\pgfusepath{stroke,fill}%
\end{pgfscope}%
\begin{pgfscope}%
\pgfpathrectangle{\pgfqpoint{0.556847in}{0.516222in}}{\pgfqpoint{1.962733in}{1.783528in}} %
\pgfusepath{clip}%
\pgfsetbuttcap%
\pgfsetroundjoin%
\definecolor{currentfill}{rgb}{0.298039,0.447059,0.690196}%
\pgfsetfillcolor{currentfill}%
\pgfsetlinewidth{0.240900pt}%
\definecolor{currentstroke}{rgb}{1.000000,1.000000,1.000000}%
\pgfsetstrokecolor{currentstroke}%
\pgfsetdash{}{0pt}%
\pgfpathmoveto{\pgfqpoint{2.014877in}{1.390306in}}%
\pgfpathcurveto{\pgfqpoint{2.023113in}{1.390306in}}{\pgfqpoint{2.031014in}{1.393578in}}{\pgfqpoint{2.036837in}{1.399402in}}%
\pgfpathcurveto{\pgfqpoint{2.042661in}{1.405226in}}{\pgfqpoint{2.045934in}{1.413126in}}{\pgfqpoint{2.045934in}{1.421363in}}%
\pgfpathcurveto{\pgfqpoint{2.045934in}{1.429599in}}{\pgfqpoint{2.042661in}{1.437499in}}{\pgfqpoint{2.036837in}{1.443323in}}%
\pgfpathcurveto{\pgfqpoint{2.031014in}{1.449147in}}{\pgfqpoint{2.023113in}{1.452419in}}{\pgfqpoint{2.014877in}{1.452419in}}%
\pgfpathcurveto{\pgfqpoint{2.006641in}{1.452419in}}{\pgfqpoint{1.998741in}{1.449147in}}{\pgfqpoint{1.992917in}{1.443323in}}%
\pgfpathcurveto{\pgfqpoint{1.987093in}{1.437499in}}{\pgfqpoint{1.983821in}{1.429599in}}{\pgfqpoint{1.983821in}{1.421363in}}%
\pgfpathcurveto{\pgfqpoint{1.983821in}{1.413126in}}{\pgfqpoint{1.987093in}{1.405226in}}{\pgfqpoint{1.992917in}{1.399402in}}%
\pgfpathcurveto{\pgfqpoint{1.998741in}{1.393578in}}{\pgfqpoint{2.006641in}{1.390306in}}{\pgfqpoint{2.014877in}{1.390306in}}%
\pgfpathclose%
\pgfusepath{stroke,fill}%
\end{pgfscope}%
\begin{pgfscope}%
\pgfpathrectangle{\pgfqpoint{0.556847in}{0.516222in}}{\pgfqpoint{1.962733in}{1.783528in}} %
\pgfusepath{clip}%
\pgfsetbuttcap%
\pgfsetroundjoin%
\definecolor{currentfill}{rgb}{0.298039,0.447059,0.690196}%
\pgfsetfillcolor{currentfill}%
\pgfsetlinewidth{0.240900pt}%
\definecolor{currentstroke}{rgb}{1.000000,1.000000,1.000000}%
\pgfsetstrokecolor{currentstroke}%
\pgfsetdash{}{0pt}%
\pgfpathmoveto{\pgfqpoint{1.846643in}{1.952117in}}%
\pgfpathcurveto{\pgfqpoint{1.854879in}{1.952117in}}{\pgfqpoint{1.862779in}{1.955390in}}{\pgfqpoint{1.868603in}{1.961214in}}%
\pgfpathcurveto{\pgfqpoint{1.874427in}{1.967037in}}{\pgfqpoint{1.877699in}{1.974938in}}{\pgfqpoint{1.877699in}{1.983174in}}%
\pgfpathcurveto{\pgfqpoint{1.877699in}{1.991410in}}{\pgfqpoint{1.874427in}{1.999310in}}{\pgfqpoint{1.868603in}{2.005134in}}%
\pgfpathcurveto{\pgfqpoint{1.862779in}{2.010958in}}{\pgfqpoint{1.854879in}{2.014230in}}{\pgfqpoint{1.846643in}{2.014230in}}%
\pgfpathcurveto{\pgfqpoint{1.838407in}{2.014230in}}{\pgfqpoint{1.830507in}{2.010958in}}{\pgfqpoint{1.824683in}{2.005134in}}%
\pgfpathcurveto{\pgfqpoint{1.818859in}{1.999310in}}{\pgfqpoint{1.815586in}{1.991410in}}{\pgfqpoint{1.815586in}{1.983174in}}%
\pgfpathcurveto{\pgfqpoint{1.815586in}{1.974938in}}{\pgfqpoint{1.818859in}{1.967037in}}{\pgfqpoint{1.824683in}{1.961214in}}%
\pgfpathcurveto{\pgfqpoint{1.830507in}{1.955390in}}{\pgfqpoint{1.838407in}{1.952117in}}{\pgfqpoint{1.846643in}{1.952117in}}%
\pgfpathclose%
\pgfusepath{stroke,fill}%
\end{pgfscope}%
\begin{pgfscope}%
\pgfpathrectangle{\pgfqpoint{0.556847in}{0.516222in}}{\pgfqpoint{1.962733in}{1.783528in}} %
\pgfusepath{clip}%
\pgfsetbuttcap%
\pgfsetroundjoin%
\definecolor{currentfill}{rgb}{0.298039,0.447059,0.690196}%
\pgfsetfillcolor{currentfill}%
\pgfsetlinewidth{0.240900pt}%
\definecolor{currentstroke}{rgb}{1.000000,1.000000,1.000000}%
\pgfsetstrokecolor{currentstroke}%
\pgfsetdash{}{0pt}%
\pgfpathmoveto{\pgfqpoint{1.229784in}{1.702423in}}%
\pgfpathcurveto{\pgfqpoint{1.238020in}{1.702423in}}{\pgfqpoint{1.245920in}{1.705696in}}{\pgfqpoint{1.251744in}{1.711520in}}%
\pgfpathcurveto{\pgfqpoint{1.257568in}{1.717344in}}{\pgfqpoint{1.260841in}{1.725244in}}{\pgfqpoint{1.260841in}{1.733480in}}%
\pgfpathcurveto{\pgfqpoint{1.260841in}{1.741716in}}{\pgfqpoint{1.257568in}{1.749616in}}{\pgfqpoint{1.251744in}{1.755440in}}%
\pgfpathcurveto{\pgfqpoint{1.245920in}{1.761264in}}{\pgfqpoint{1.238020in}{1.764536in}}{\pgfqpoint{1.229784in}{1.764536in}}%
\pgfpathcurveto{\pgfqpoint{1.221548in}{1.764536in}}{\pgfqpoint{1.213648in}{1.761264in}}{\pgfqpoint{1.207824in}{1.755440in}}%
\pgfpathcurveto{\pgfqpoint{1.202000in}{1.749616in}}{\pgfqpoint{1.198728in}{1.741716in}}{\pgfqpoint{1.198728in}{1.733480in}}%
\pgfpathcurveto{\pgfqpoint{1.198728in}{1.725244in}}{\pgfqpoint{1.202000in}{1.717344in}}{\pgfqpoint{1.207824in}{1.711520in}}%
\pgfpathcurveto{\pgfqpoint{1.213648in}{1.705696in}}{\pgfqpoint{1.221548in}{1.702423in}}{\pgfqpoint{1.229784in}{1.702423in}}%
\pgfpathclose%
\pgfusepath{stroke,fill}%
\end{pgfscope}%
\begin{pgfscope}%
\pgfpathrectangle{\pgfqpoint{0.556847in}{0.516222in}}{\pgfqpoint{1.962733in}{1.783528in}} %
\pgfusepath{clip}%
\pgfsetbuttcap%
\pgfsetroundjoin%
\definecolor{currentfill}{rgb}{0.298039,0.447059,0.690196}%
\pgfsetfillcolor{currentfill}%
\pgfsetlinewidth{0.240900pt}%
\definecolor{currentstroke}{rgb}{1.000000,1.000000,1.000000}%
\pgfsetstrokecolor{currentstroke}%
\pgfsetdash{}{0pt}%
\pgfpathmoveto{\pgfqpoint{2.183111in}{1.715800in}}%
\pgfpathcurveto{\pgfqpoint{2.191348in}{1.715800in}}{\pgfqpoint{2.199248in}{1.719072in}}{\pgfqpoint{2.205072in}{1.724896in}}%
\pgfpathcurveto{\pgfqpoint{2.210896in}{1.730720in}}{\pgfqpoint{2.214168in}{1.738620in}}{\pgfqpoint{2.214168in}{1.746856in}}%
\pgfpathcurveto{\pgfqpoint{2.214168in}{1.755093in}}{\pgfqpoint{2.210896in}{1.762993in}}{\pgfqpoint{2.205072in}{1.768817in}}%
\pgfpathcurveto{\pgfqpoint{2.199248in}{1.774641in}}{\pgfqpoint{2.191348in}{1.777913in}}{\pgfqpoint{2.183111in}{1.777913in}}%
\pgfpathcurveto{\pgfqpoint{2.174875in}{1.777913in}}{\pgfqpoint{2.166975in}{1.774641in}}{\pgfqpoint{2.161151in}{1.768817in}}%
\pgfpathcurveto{\pgfqpoint{2.155327in}{1.762993in}}{\pgfqpoint{2.152055in}{1.755093in}}{\pgfqpoint{2.152055in}{1.746856in}}%
\pgfpathcurveto{\pgfqpoint{2.152055in}{1.738620in}}{\pgfqpoint{2.155327in}{1.730720in}}{\pgfqpoint{2.161151in}{1.724896in}}%
\pgfpathcurveto{\pgfqpoint{2.166975in}{1.719072in}}{\pgfqpoint{2.174875in}{1.715800in}}{\pgfqpoint{2.183111in}{1.715800in}}%
\pgfpathclose%
\pgfusepath{stroke,fill}%
\end{pgfscope}%
\begin{pgfscope}%
\pgfpathrectangle{\pgfqpoint{0.556847in}{0.516222in}}{\pgfqpoint{1.962733in}{1.783528in}} %
\pgfusepath{clip}%
\pgfsetbuttcap%
\pgfsetroundjoin%
\definecolor{currentfill}{rgb}{0.298039,0.447059,0.690196}%
\pgfsetfillcolor{currentfill}%
\pgfsetlinewidth{0.240900pt}%
\definecolor{currentstroke}{rgb}{1.000000,1.000000,1.000000}%
\pgfsetstrokecolor{currentstroke}%
\pgfsetdash{}{0pt}%
\pgfpathmoveto{\pgfqpoint{2.295268in}{1.671212in}}%
\pgfpathcurveto{\pgfqpoint{2.303504in}{1.671212in}}{\pgfqpoint{2.311404in}{1.674484in}}{\pgfqpoint{2.317228in}{1.680308in}}%
\pgfpathcurveto{\pgfqpoint{2.323052in}{1.686132in}}{\pgfqpoint{2.326324in}{1.694032in}}{\pgfqpoint{2.326324in}{1.702268in}}%
\pgfpathcurveto{\pgfqpoint{2.326324in}{1.710504in}}{\pgfqpoint{2.323052in}{1.718405in}}{\pgfqpoint{2.317228in}{1.724228in}}%
\pgfpathcurveto{\pgfqpoint{2.311404in}{1.730052in}}{\pgfqpoint{2.303504in}{1.733325in}}{\pgfqpoint{2.295268in}{1.733325in}}%
\pgfpathcurveto{\pgfqpoint{2.287031in}{1.733325in}}{\pgfqpoint{2.279131in}{1.730052in}}{\pgfqpoint{2.273307in}{1.724228in}}%
\pgfpathcurveto{\pgfqpoint{2.267483in}{1.718405in}}{\pgfqpoint{2.264211in}{1.710504in}}{\pgfqpoint{2.264211in}{1.702268in}}%
\pgfpathcurveto{\pgfqpoint{2.264211in}{1.694032in}}{\pgfqpoint{2.267483in}{1.686132in}}{\pgfqpoint{2.273307in}{1.680308in}}%
\pgfpathcurveto{\pgfqpoint{2.279131in}{1.674484in}}{\pgfqpoint{2.287031in}{1.671212in}}{\pgfqpoint{2.295268in}{1.671212in}}%
\pgfpathclose%
\pgfusepath{stroke,fill}%
\end{pgfscope}%
\begin{pgfscope}%
\pgfpathrectangle{\pgfqpoint{0.556847in}{0.516222in}}{\pgfqpoint{1.962733in}{1.783528in}} %
\pgfusepath{clip}%
\pgfsetbuttcap%
\pgfsetroundjoin%
\definecolor{currentfill}{rgb}{0.298039,0.447059,0.690196}%
\pgfsetfillcolor{currentfill}%
\pgfsetlinewidth{0.240900pt}%
\definecolor{currentstroke}{rgb}{1.000000,1.000000,1.000000}%
\pgfsetstrokecolor{currentstroke}%
\pgfsetdash{}{0pt}%
\pgfpathmoveto{\pgfqpoint{1.341940in}{0.725942in}}%
\pgfpathcurveto{\pgfqpoint{1.350177in}{0.725942in}}{\pgfqpoint{1.358077in}{0.729214in}}{\pgfqpoint{1.363901in}{0.735038in}}%
\pgfpathcurveto{\pgfqpoint{1.369724in}{0.740862in}}{\pgfqpoint{1.372997in}{0.748762in}}{\pgfqpoint{1.372997in}{0.756998in}}%
\pgfpathcurveto{\pgfqpoint{1.372997in}{0.765235in}}{\pgfqpoint{1.369724in}{0.773135in}}{\pgfqpoint{1.363901in}{0.778959in}}%
\pgfpathcurveto{\pgfqpoint{1.358077in}{0.784783in}}{\pgfqpoint{1.350177in}{0.788055in}}{\pgfqpoint{1.341940in}{0.788055in}}%
\pgfpathcurveto{\pgfqpoint{1.333704in}{0.788055in}}{\pgfqpoint{1.325804in}{0.784783in}}{\pgfqpoint{1.319980in}{0.778959in}}%
\pgfpathcurveto{\pgfqpoint{1.314156in}{0.773135in}}{\pgfqpoint{1.310884in}{0.765235in}}{\pgfqpoint{1.310884in}{0.756998in}}%
\pgfpathcurveto{\pgfqpoint{1.310884in}{0.748762in}}{\pgfqpoint{1.314156in}{0.740862in}}{\pgfqpoint{1.319980in}{0.735038in}}%
\pgfpathcurveto{\pgfqpoint{1.325804in}{0.729214in}}{\pgfqpoint{1.333704in}{0.725942in}}{\pgfqpoint{1.341940in}{0.725942in}}%
\pgfpathclose%
\pgfusepath{stroke,fill}%
\end{pgfscope}%
\begin{pgfscope}%
\pgfpathrectangle{\pgfqpoint{0.556847in}{0.516222in}}{\pgfqpoint{1.962733in}{1.783528in}} %
\pgfusepath{clip}%
\pgfsetbuttcap%
\pgfsetroundjoin%
\definecolor{currentfill}{rgb}{0.298039,0.447059,0.690196}%
\pgfsetfillcolor{currentfill}%
\pgfsetlinewidth{0.240900pt}%
\definecolor{currentstroke}{rgb}{1.000000,1.000000,1.000000}%
\pgfsetstrokecolor{currentstroke}%
\pgfsetdash{}{0pt}%
\pgfpathmoveto{\pgfqpoint{0.837238in}{1.943200in}}%
\pgfpathcurveto{\pgfqpoint{0.845474in}{1.943200in}}{\pgfqpoint{0.853374in}{1.946472in}}{\pgfqpoint{0.859198in}{1.952296in}}%
\pgfpathcurveto{\pgfqpoint{0.865022in}{1.958120in}}{\pgfqpoint{0.868294in}{1.966020in}}{\pgfqpoint{0.868294in}{1.974256in}}%
\pgfpathcurveto{\pgfqpoint{0.868294in}{1.982492in}}{\pgfqpoint{0.865022in}{1.990393in}}{\pgfqpoint{0.859198in}{1.996216in}}%
\pgfpathcurveto{\pgfqpoint{0.853374in}{2.002040in}}{\pgfqpoint{0.845474in}{2.005313in}}{\pgfqpoint{0.837238in}{2.005313in}}%
\pgfpathcurveto{\pgfqpoint{0.829001in}{2.005313in}}{\pgfqpoint{0.821101in}{2.002040in}}{\pgfqpoint{0.815277in}{1.996216in}}%
\pgfpathcurveto{\pgfqpoint{0.809453in}{1.990393in}}{\pgfqpoint{0.806181in}{1.982492in}}{\pgfqpoint{0.806181in}{1.974256in}}%
\pgfpathcurveto{\pgfqpoint{0.806181in}{1.966020in}}{\pgfqpoint{0.809453in}{1.958120in}}{\pgfqpoint{0.815277in}{1.952296in}}%
\pgfpathcurveto{\pgfqpoint{0.821101in}{1.946472in}}{\pgfqpoint{0.829001in}{1.943200in}}{\pgfqpoint{0.837238in}{1.943200in}}%
\pgfpathclose%
\pgfusepath{stroke,fill}%
\end{pgfscope}%
\begin{pgfscope}%
\pgfpathrectangle{\pgfqpoint{0.556847in}{0.516222in}}{\pgfqpoint{1.962733in}{1.783528in}} %
\pgfusepath{clip}%
\pgfsetbuttcap%
\pgfsetroundjoin%
\definecolor{currentfill}{rgb}{0.298039,0.447059,0.690196}%
\pgfsetfillcolor{currentfill}%
\pgfsetlinewidth{0.240900pt}%
\definecolor{currentstroke}{rgb}{1.000000,1.000000,1.000000}%
\pgfsetstrokecolor{currentstroke}%
\pgfsetdash{}{0pt}%
\pgfpathmoveto{\pgfqpoint{1.285862in}{1.381388in}}%
\pgfpathcurveto{\pgfqpoint{1.294098in}{1.381388in}}{\pgfqpoint{1.301999in}{1.384661in}}{\pgfqpoint{1.307822in}{1.390485in}}%
\pgfpathcurveto{\pgfqpoint{1.313646in}{1.396309in}}{\pgfqpoint{1.316919in}{1.404209in}}{\pgfqpoint{1.316919in}{1.412445in}}%
\pgfpathcurveto{\pgfqpoint{1.316919in}{1.420681in}}{\pgfqpoint{1.313646in}{1.428581in}}{\pgfqpoint{1.307822in}{1.434405in}}%
\pgfpathcurveto{\pgfqpoint{1.301999in}{1.440229in}}{\pgfqpoint{1.294098in}{1.443501in}}{\pgfqpoint{1.285862in}{1.443501in}}%
\pgfpathcurveto{\pgfqpoint{1.277626in}{1.443501in}}{\pgfqpoint{1.269726in}{1.440229in}}{\pgfqpoint{1.263902in}{1.434405in}}%
\pgfpathcurveto{\pgfqpoint{1.258078in}{1.428581in}}{\pgfqpoint{1.254806in}{1.420681in}}{\pgfqpoint{1.254806in}{1.412445in}}%
\pgfpathcurveto{\pgfqpoint{1.254806in}{1.404209in}}{\pgfqpoint{1.258078in}{1.396309in}}{\pgfqpoint{1.263902in}{1.390485in}}%
\pgfpathcurveto{\pgfqpoint{1.269726in}{1.384661in}}{\pgfqpoint{1.277626in}{1.381388in}}{\pgfqpoint{1.285862in}{1.381388in}}%
\pgfpathclose%
\pgfusepath{stroke,fill}%
\end{pgfscope}%
\begin{pgfscope}%
\pgfpathrectangle{\pgfqpoint{0.556847in}{0.516222in}}{\pgfqpoint{1.962733in}{1.783528in}} %
\pgfusepath{clip}%
\pgfsetbuttcap%
\pgfsetroundjoin%
\definecolor{currentfill}{rgb}{0.298039,0.447059,0.690196}%
\pgfsetfillcolor{currentfill}%
\pgfsetlinewidth{0.240900pt}%
\definecolor{currentstroke}{rgb}{1.000000,1.000000,1.000000}%
\pgfsetstrokecolor{currentstroke}%
\pgfsetdash{}{0pt}%
\pgfpathmoveto{\pgfqpoint{2.183111in}{1.987788in}}%
\pgfpathcurveto{\pgfqpoint{2.191348in}{1.987788in}}{\pgfqpoint{2.199248in}{1.991060in}}{\pgfqpoint{2.205072in}{1.996884in}}%
\pgfpathcurveto{\pgfqpoint{2.210896in}{2.002708in}}{\pgfqpoint{2.214168in}{2.010608in}}{\pgfqpoint{2.214168in}{2.018844in}}%
\pgfpathcurveto{\pgfqpoint{2.214168in}{2.027081in}}{\pgfqpoint{2.210896in}{2.034981in}}{\pgfqpoint{2.205072in}{2.040805in}}%
\pgfpathcurveto{\pgfqpoint{2.199248in}{2.046629in}}{\pgfqpoint{2.191348in}{2.049901in}}{\pgfqpoint{2.183111in}{2.049901in}}%
\pgfpathcurveto{\pgfqpoint{2.174875in}{2.049901in}}{\pgfqpoint{2.166975in}{2.046629in}}{\pgfqpoint{2.161151in}{2.040805in}}%
\pgfpathcurveto{\pgfqpoint{2.155327in}{2.034981in}}{\pgfqpoint{2.152055in}{2.027081in}}{\pgfqpoint{2.152055in}{2.018844in}}%
\pgfpathcurveto{\pgfqpoint{2.152055in}{2.010608in}}{\pgfqpoint{2.155327in}{2.002708in}}{\pgfqpoint{2.161151in}{1.996884in}}%
\pgfpathcurveto{\pgfqpoint{2.166975in}{1.991060in}}{\pgfqpoint{2.174875in}{1.987788in}}{\pgfqpoint{2.183111in}{1.987788in}}%
\pgfpathclose%
\pgfusepath{stroke,fill}%
\end{pgfscope}%
\begin{pgfscope}%
\pgfsetrectcap%
\pgfsetmiterjoin%
\pgfsetlinewidth{0.000000pt}%
\definecolor{currentstroke}{rgb}{1.000000,1.000000,1.000000}%
\pgfsetstrokecolor{currentstroke}%
\pgfsetdash{}{0pt}%
\pgfpathmoveto{\pgfqpoint{0.556847in}{0.516222in}}%
\pgfpathlineto{\pgfqpoint{0.556847in}{2.299750in}}%
\pgfusepath{}%
\end{pgfscope}%
\begin{pgfscope}%
\pgfsetrectcap%
\pgfsetmiterjoin%
\pgfsetlinewidth{0.000000pt}%
\definecolor{currentstroke}{rgb}{1.000000,1.000000,1.000000}%
\pgfsetstrokecolor{currentstroke}%
\pgfsetdash{}{0pt}%
\pgfpathmoveto{\pgfqpoint{0.556847in}{0.516222in}}%
\pgfpathlineto{\pgfqpoint{2.519580in}{0.516222in}}%
\pgfusepath{}%
\end{pgfscope}%
\begin{pgfscope}%
\pgfsetbuttcap%
\pgfsetmiterjoin%
\definecolor{currentfill}{rgb}{0.917647,0.917647,0.949020}%
\pgfsetfillcolor{currentfill}%
\pgfsetlinewidth{0.000000pt}%
\definecolor{currentstroke}{rgb}{0.000000,0.000000,0.000000}%
\pgfsetstrokecolor{currentstroke}%
\pgfsetstrokeopacity{0.000000}%
\pgfsetdash{}{0pt}%
\pgfpathmoveto{\pgfqpoint{2.816705in}{0.516222in}}%
\pgfpathlineto{\pgfqpoint{4.779438in}{0.516222in}}%
\pgfpathlineto{\pgfqpoint{4.779438in}{2.299750in}}%
\pgfpathlineto{\pgfqpoint{2.816705in}{2.299750in}}%
\pgfpathclose%
\pgfusepath{fill}%
\end{pgfscope}%
\begin{pgfscope}%
\pgfpathrectangle{\pgfqpoint{2.816705in}{0.516222in}}{\pgfqpoint{1.962733in}{1.783528in}} %
\pgfusepath{clip}%
\pgfsetroundcap%
\pgfsetroundjoin%
\pgfsetlinewidth{0.803000pt}%
\definecolor{currentstroke}{rgb}{1.000000,1.000000,1.000000}%
\pgfsetstrokecolor{currentstroke}%
\pgfsetdash{}{0pt}%
\pgfpathmoveto{\pgfqpoint{2.816705in}{0.516222in}}%
\pgfpathlineto{\pgfqpoint{2.816705in}{2.299750in}}%
\pgfusepath{stroke}%
\end{pgfscope}%
\begin{pgfscope}%
\pgfsetbuttcap%
\pgfsetroundjoin%
\definecolor{currentfill}{rgb}{0.150000,0.150000,0.150000}%
\pgfsetfillcolor{currentfill}%
\pgfsetlinewidth{0.803000pt}%
\definecolor{currentstroke}{rgb}{0.150000,0.150000,0.150000}%
\pgfsetstrokecolor{currentstroke}%
\pgfsetdash{}{0pt}%
\pgfsys@defobject{currentmarker}{\pgfqpoint{0.000000in}{0.000000in}}{\pgfqpoint{0.000000in}{0.000000in}}{%
\pgfpathmoveto{\pgfqpoint{0.000000in}{0.000000in}}%
\pgfpathlineto{\pgfqpoint{0.000000in}{0.000000in}}%
\pgfusepath{stroke,fill}%
}%
\begin{pgfscope}%
\pgfsys@transformshift{2.816705in}{0.516222in}%
\pgfsys@useobject{currentmarker}{}%
\end{pgfscope}%
\end{pgfscope}%
\begin{pgfscope}%
\definecolor{textcolor}{rgb}{0.150000,0.150000,0.150000}%
\pgfsetstrokecolor{textcolor}%
\pgfsetfillcolor{textcolor}%
\pgftext[x=2.816705in,y=0.438444in,,top]{\color{textcolor}\sffamily\fontsize{8.000000}{9.600000}\selectfont 2.0}%
\end{pgfscope}%
\begin{pgfscope}%
\pgfpathrectangle{\pgfqpoint{2.816705in}{0.516222in}}{\pgfqpoint{1.962733in}{1.783528in}} %
\pgfusepath{clip}%
\pgfsetroundcap%
\pgfsetroundjoin%
\pgfsetlinewidth{0.803000pt}%
\definecolor{currentstroke}{rgb}{1.000000,1.000000,1.000000}%
\pgfsetstrokecolor{currentstroke}%
\pgfsetdash{}{0pt}%
\pgfpathmoveto{\pgfqpoint{3.097095in}{0.516222in}}%
\pgfpathlineto{\pgfqpoint{3.097095in}{2.299750in}}%
\pgfusepath{stroke}%
\end{pgfscope}%
\begin{pgfscope}%
\pgfsetbuttcap%
\pgfsetroundjoin%
\definecolor{currentfill}{rgb}{0.150000,0.150000,0.150000}%
\pgfsetfillcolor{currentfill}%
\pgfsetlinewidth{0.803000pt}%
\definecolor{currentstroke}{rgb}{0.150000,0.150000,0.150000}%
\pgfsetstrokecolor{currentstroke}%
\pgfsetdash{}{0pt}%
\pgfsys@defobject{currentmarker}{\pgfqpoint{0.000000in}{0.000000in}}{\pgfqpoint{0.000000in}{0.000000in}}{%
\pgfpathmoveto{\pgfqpoint{0.000000in}{0.000000in}}%
\pgfpathlineto{\pgfqpoint{0.000000in}{0.000000in}}%
\pgfusepath{stroke,fill}%
}%
\begin{pgfscope}%
\pgfsys@transformshift{3.097095in}{0.516222in}%
\pgfsys@useobject{currentmarker}{}%
\end{pgfscope}%
\end{pgfscope}%
\begin{pgfscope}%
\definecolor{textcolor}{rgb}{0.150000,0.150000,0.150000}%
\pgfsetstrokecolor{textcolor}%
\pgfsetfillcolor{textcolor}%
\pgftext[x=3.097095in,y=0.438444in,,top]{\color{textcolor}\sffamily\fontsize{8.000000}{9.600000}\selectfont 2.5}%
\end{pgfscope}%
\begin{pgfscope}%
\pgfpathrectangle{\pgfqpoint{2.816705in}{0.516222in}}{\pgfqpoint{1.962733in}{1.783528in}} %
\pgfusepath{clip}%
\pgfsetroundcap%
\pgfsetroundjoin%
\pgfsetlinewidth{0.803000pt}%
\definecolor{currentstroke}{rgb}{1.000000,1.000000,1.000000}%
\pgfsetstrokecolor{currentstroke}%
\pgfsetdash{}{0pt}%
\pgfpathmoveto{\pgfqpoint{3.377486in}{0.516222in}}%
\pgfpathlineto{\pgfqpoint{3.377486in}{2.299750in}}%
\pgfusepath{stroke}%
\end{pgfscope}%
\begin{pgfscope}%
\pgfsetbuttcap%
\pgfsetroundjoin%
\definecolor{currentfill}{rgb}{0.150000,0.150000,0.150000}%
\pgfsetfillcolor{currentfill}%
\pgfsetlinewidth{0.803000pt}%
\definecolor{currentstroke}{rgb}{0.150000,0.150000,0.150000}%
\pgfsetstrokecolor{currentstroke}%
\pgfsetdash{}{0pt}%
\pgfsys@defobject{currentmarker}{\pgfqpoint{0.000000in}{0.000000in}}{\pgfqpoint{0.000000in}{0.000000in}}{%
\pgfpathmoveto{\pgfqpoint{0.000000in}{0.000000in}}%
\pgfpathlineto{\pgfqpoint{0.000000in}{0.000000in}}%
\pgfusepath{stroke,fill}%
}%
\begin{pgfscope}%
\pgfsys@transformshift{3.377486in}{0.516222in}%
\pgfsys@useobject{currentmarker}{}%
\end{pgfscope}%
\end{pgfscope}%
\begin{pgfscope}%
\definecolor{textcolor}{rgb}{0.150000,0.150000,0.150000}%
\pgfsetstrokecolor{textcolor}%
\pgfsetfillcolor{textcolor}%
\pgftext[x=3.377486in,y=0.438444in,,top]{\color{textcolor}\sffamily\fontsize{8.000000}{9.600000}\selectfont 3.0}%
\end{pgfscope}%
\begin{pgfscope}%
\pgfpathrectangle{\pgfqpoint{2.816705in}{0.516222in}}{\pgfqpoint{1.962733in}{1.783528in}} %
\pgfusepath{clip}%
\pgfsetroundcap%
\pgfsetroundjoin%
\pgfsetlinewidth{0.803000pt}%
\definecolor{currentstroke}{rgb}{1.000000,1.000000,1.000000}%
\pgfsetstrokecolor{currentstroke}%
\pgfsetdash{}{0pt}%
\pgfpathmoveto{\pgfqpoint{3.657876in}{0.516222in}}%
\pgfpathlineto{\pgfqpoint{3.657876in}{2.299750in}}%
\pgfusepath{stroke}%
\end{pgfscope}%
\begin{pgfscope}%
\pgfsetbuttcap%
\pgfsetroundjoin%
\definecolor{currentfill}{rgb}{0.150000,0.150000,0.150000}%
\pgfsetfillcolor{currentfill}%
\pgfsetlinewidth{0.803000pt}%
\definecolor{currentstroke}{rgb}{0.150000,0.150000,0.150000}%
\pgfsetstrokecolor{currentstroke}%
\pgfsetdash{}{0pt}%
\pgfsys@defobject{currentmarker}{\pgfqpoint{0.000000in}{0.000000in}}{\pgfqpoint{0.000000in}{0.000000in}}{%
\pgfpathmoveto{\pgfqpoint{0.000000in}{0.000000in}}%
\pgfpathlineto{\pgfqpoint{0.000000in}{0.000000in}}%
\pgfusepath{stroke,fill}%
}%
\begin{pgfscope}%
\pgfsys@transformshift{3.657876in}{0.516222in}%
\pgfsys@useobject{currentmarker}{}%
\end{pgfscope}%
\end{pgfscope}%
\begin{pgfscope}%
\definecolor{textcolor}{rgb}{0.150000,0.150000,0.150000}%
\pgfsetstrokecolor{textcolor}%
\pgfsetfillcolor{textcolor}%
\pgftext[x=3.657876in,y=0.438444in,,top]{\color{textcolor}\sffamily\fontsize{8.000000}{9.600000}\selectfont 3.5}%
\end{pgfscope}%
\begin{pgfscope}%
\pgfpathrectangle{\pgfqpoint{2.816705in}{0.516222in}}{\pgfqpoint{1.962733in}{1.783528in}} %
\pgfusepath{clip}%
\pgfsetroundcap%
\pgfsetroundjoin%
\pgfsetlinewidth{0.803000pt}%
\definecolor{currentstroke}{rgb}{1.000000,1.000000,1.000000}%
\pgfsetstrokecolor{currentstroke}%
\pgfsetdash{}{0pt}%
\pgfpathmoveto{\pgfqpoint{3.938266in}{0.516222in}}%
\pgfpathlineto{\pgfqpoint{3.938266in}{2.299750in}}%
\pgfusepath{stroke}%
\end{pgfscope}%
\begin{pgfscope}%
\pgfsetbuttcap%
\pgfsetroundjoin%
\definecolor{currentfill}{rgb}{0.150000,0.150000,0.150000}%
\pgfsetfillcolor{currentfill}%
\pgfsetlinewidth{0.803000pt}%
\definecolor{currentstroke}{rgb}{0.150000,0.150000,0.150000}%
\pgfsetstrokecolor{currentstroke}%
\pgfsetdash{}{0pt}%
\pgfsys@defobject{currentmarker}{\pgfqpoint{0.000000in}{0.000000in}}{\pgfqpoint{0.000000in}{0.000000in}}{%
\pgfpathmoveto{\pgfqpoint{0.000000in}{0.000000in}}%
\pgfpathlineto{\pgfqpoint{0.000000in}{0.000000in}}%
\pgfusepath{stroke,fill}%
}%
\begin{pgfscope}%
\pgfsys@transformshift{3.938266in}{0.516222in}%
\pgfsys@useobject{currentmarker}{}%
\end{pgfscope}%
\end{pgfscope}%
\begin{pgfscope}%
\definecolor{textcolor}{rgb}{0.150000,0.150000,0.150000}%
\pgfsetstrokecolor{textcolor}%
\pgfsetfillcolor{textcolor}%
\pgftext[x=3.938266in,y=0.438444in,,top]{\color{textcolor}\sffamily\fontsize{8.000000}{9.600000}\selectfont 4.0}%
\end{pgfscope}%
\begin{pgfscope}%
\pgfpathrectangle{\pgfqpoint{2.816705in}{0.516222in}}{\pgfqpoint{1.962733in}{1.783528in}} %
\pgfusepath{clip}%
\pgfsetroundcap%
\pgfsetroundjoin%
\pgfsetlinewidth{0.803000pt}%
\definecolor{currentstroke}{rgb}{1.000000,1.000000,1.000000}%
\pgfsetstrokecolor{currentstroke}%
\pgfsetdash{}{0pt}%
\pgfpathmoveto{\pgfqpoint{4.218657in}{0.516222in}}%
\pgfpathlineto{\pgfqpoint{4.218657in}{2.299750in}}%
\pgfusepath{stroke}%
\end{pgfscope}%
\begin{pgfscope}%
\pgfsetbuttcap%
\pgfsetroundjoin%
\definecolor{currentfill}{rgb}{0.150000,0.150000,0.150000}%
\pgfsetfillcolor{currentfill}%
\pgfsetlinewidth{0.803000pt}%
\definecolor{currentstroke}{rgb}{0.150000,0.150000,0.150000}%
\pgfsetstrokecolor{currentstroke}%
\pgfsetdash{}{0pt}%
\pgfsys@defobject{currentmarker}{\pgfqpoint{0.000000in}{0.000000in}}{\pgfqpoint{0.000000in}{0.000000in}}{%
\pgfpathmoveto{\pgfqpoint{0.000000in}{0.000000in}}%
\pgfpathlineto{\pgfqpoint{0.000000in}{0.000000in}}%
\pgfusepath{stroke,fill}%
}%
\begin{pgfscope}%
\pgfsys@transformshift{4.218657in}{0.516222in}%
\pgfsys@useobject{currentmarker}{}%
\end{pgfscope}%
\end{pgfscope}%
\begin{pgfscope}%
\definecolor{textcolor}{rgb}{0.150000,0.150000,0.150000}%
\pgfsetstrokecolor{textcolor}%
\pgfsetfillcolor{textcolor}%
\pgftext[x=4.218657in,y=0.438444in,,top]{\color{textcolor}\sffamily\fontsize{8.000000}{9.600000}\selectfont 4.5}%
\end{pgfscope}%
\begin{pgfscope}%
\pgfpathrectangle{\pgfqpoint{2.816705in}{0.516222in}}{\pgfqpoint{1.962733in}{1.783528in}} %
\pgfusepath{clip}%
\pgfsetroundcap%
\pgfsetroundjoin%
\pgfsetlinewidth{0.803000pt}%
\definecolor{currentstroke}{rgb}{1.000000,1.000000,1.000000}%
\pgfsetstrokecolor{currentstroke}%
\pgfsetdash{}{0pt}%
\pgfpathmoveto{\pgfqpoint{4.499047in}{0.516222in}}%
\pgfpathlineto{\pgfqpoint{4.499047in}{2.299750in}}%
\pgfusepath{stroke}%
\end{pgfscope}%
\begin{pgfscope}%
\pgfsetbuttcap%
\pgfsetroundjoin%
\definecolor{currentfill}{rgb}{0.150000,0.150000,0.150000}%
\pgfsetfillcolor{currentfill}%
\pgfsetlinewidth{0.803000pt}%
\definecolor{currentstroke}{rgb}{0.150000,0.150000,0.150000}%
\pgfsetstrokecolor{currentstroke}%
\pgfsetdash{}{0pt}%
\pgfsys@defobject{currentmarker}{\pgfqpoint{0.000000in}{0.000000in}}{\pgfqpoint{0.000000in}{0.000000in}}{%
\pgfpathmoveto{\pgfqpoint{0.000000in}{0.000000in}}%
\pgfpathlineto{\pgfqpoint{0.000000in}{0.000000in}}%
\pgfusepath{stroke,fill}%
}%
\begin{pgfscope}%
\pgfsys@transformshift{4.499047in}{0.516222in}%
\pgfsys@useobject{currentmarker}{}%
\end{pgfscope}%
\end{pgfscope}%
\begin{pgfscope}%
\definecolor{textcolor}{rgb}{0.150000,0.150000,0.150000}%
\pgfsetstrokecolor{textcolor}%
\pgfsetfillcolor{textcolor}%
\pgftext[x=4.499047in,y=0.438444in,,top]{\color{textcolor}\sffamily\fontsize{8.000000}{9.600000}\selectfont 5.0}%
\end{pgfscope}%
\begin{pgfscope}%
\pgfpathrectangle{\pgfqpoint{2.816705in}{0.516222in}}{\pgfqpoint{1.962733in}{1.783528in}} %
\pgfusepath{clip}%
\pgfsetroundcap%
\pgfsetroundjoin%
\pgfsetlinewidth{0.803000pt}%
\definecolor{currentstroke}{rgb}{1.000000,1.000000,1.000000}%
\pgfsetstrokecolor{currentstroke}%
\pgfsetdash{}{0pt}%
\pgfpathmoveto{\pgfqpoint{4.779438in}{0.516222in}}%
\pgfpathlineto{\pgfqpoint{4.779438in}{2.299750in}}%
\pgfusepath{stroke}%
\end{pgfscope}%
\begin{pgfscope}%
\pgfsetbuttcap%
\pgfsetroundjoin%
\definecolor{currentfill}{rgb}{0.150000,0.150000,0.150000}%
\pgfsetfillcolor{currentfill}%
\pgfsetlinewidth{0.803000pt}%
\definecolor{currentstroke}{rgb}{0.150000,0.150000,0.150000}%
\pgfsetstrokecolor{currentstroke}%
\pgfsetdash{}{0pt}%
\pgfsys@defobject{currentmarker}{\pgfqpoint{0.000000in}{0.000000in}}{\pgfqpoint{0.000000in}{0.000000in}}{%
\pgfpathmoveto{\pgfqpoint{0.000000in}{0.000000in}}%
\pgfpathlineto{\pgfqpoint{0.000000in}{0.000000in}}%
\pgfusepath{stroke,fill}%
}%
\begin{pgfscope}%
\pgfsys@transformshift{4.779438in}{0.516222in}%
\pgfsys@useobject{currentmarker}{}%
\end{pgfscope}%
\end{pgfscope}%
\begin{pgfscope}%
\definecolor{textcolor}{rgb}{0.150000,0.150000,0.150000}%
\pgfsetstrokecolor{textcolor}%
\pgfsetfillcolor{textcolor}%
\pgftext[x=4.779438in,y=0.438444in,,top]{\color{textcolor}\sffamily\fontsize{8.000000}{9.600000}\selectfont 5.5}%
\end{pgfscope}%
\begin{pgfscope}%
\definecolor{textcolor}{rgb}{0.150000,0.150000,0.150000}%
\pgfsetstrokecolor{textcolor}%
\pgfsetfillcolor{textcolor}%
\pgftext[x=3.798071in,y=0.273321in,,top]{\color{textcolor}\sffamily\fontsize{8.800000}{10.560000}\selectfont Falling time realization 2}%
\end{pgfscope}%
\begin{pgfscope}%
\pgfpathrectangle{\pgfqpoint{2.816705in}{0.516222in}}{\pgfqpoint{1.962733in}{1.783528in}} %
\pgfusepath{clip}%
\pgfsetroundcap%
\pgfsetroundjoin%
\pgfsetlinewidth{0.803000pt}%
\definecolor{currentstroke}{rgb}{1.000000,1.000000,1.000000}%
\pgfsetstrokecolor{currentstroke}%
\pgfsetdash{}{0pt}%
\pgfpathmoveto{\pgfqpoint{2.816705in}{0.516222in}}%
\pgfpathlineto{\pgfqpoint{4.779438in}{0.516222in}}%
\pgfusepath{stroke}%
\end{pgfscope}%
\begin{pgfscope}%
\pgfsetbuttcap%
\pgfsetroundjoin%
\definecolor{currentfill}{rgb}{0.150000,0.150000,0.150000}%
\pgfsetfillcolor{currentfill}%
\pgfsetlinewidth{0.803000pt}%
\definecolor{currentstroke}{rgb}{0.150000,0.150000,0.150000}%
\pgfsetstrokecolor{currentstroke}%
\pgfsetdash{}{0pt}%
\pgfsys@defobject{currentmarker}{\pgfqpoint{0.000000in}{0.000000in}}{\pgfqpoint{0.000000in}{0.000000in}}{%
\pgfpathmoveto{\pgfqpoint{0.000000in}{0.000000in}}%
\pgfpathlineto{\pgfqpoint{0.000000in}{0.000000in}}%
\pgfusepath{stroke,fill}%
}%
\begin{pgfscope}%
\pgfsys@transformshift{2.816705in}{0.516222in}%
\pgfsys@useobject{currentmarker}{}%
\end{pgfscope}%
\end{pgfscope}%
\begin{pgfscope}%
\pgfpathrectangle{\pgfqpoint{2.816705in}{0.516222in}}{\pgfqpoint{1.962733in}{1.783528in}} %
\pgfusepath{clip}%
\pgfsetroundcap%
\pgfsetroundjoin%
\pgfsetlinewidth{0.803000pt}%
\definecolor{currentstroke}{rgb}{1.000000,1.000000,1.000000}%
\pgfsetstrokecolor{currentstroke}%
\pgfsetdash{}{0pt}%
\pgfpathmoveto{\pgfqpoint{2.816705in}{0.739163in}}%
\pgfpathlineto{\pgfqpoint{4.779438in}{0.739163in}}%
\pgfusepath{stroke}%
\end{pgfscope}%
\begin{pgfscope}%
\pgfsetbuttcap%
\pgfsetroundjoin%
\definecolor{currentfill}{rgb}{0.150000,0.150000,0.150000}%
\pgfsetfillcolor{currentfill}%
\pgfsetlinewidth{0.803000pt}%
\definecolor{currentstroke}{rgb}{0.150000,0.150000,0.150000}%
\pgfsetstrokecolor{currentstroke}%
\pgfsetdash{}{0pt}%
\pgfsys@defobject{currentmarker}{\pgfqpoint{0.000000in}{0.000000in}}{\pgfqpoint{0.000000in}{0.000000in}}{%
\pgfpathmoveto{\pgfqpoint{0.000000in}{0.000000in}}%
\pgfpathlineto{\pgfqpoint{0.000000in}{0.000000in}}%
\pgfusepath{stroke,fill}%
}%
\begin{pgfscope}%
\pgfsys@transformshift{2.816705in}{0.739163in}%
\pgfsys@useobject{currentmarker}{}%
\end{pgfscope}%
\end{pgfscope}%
\begin{pgfscope}%
\pgfpathrectangle{\pgfqpoint{2.816705in}{0.516222in}}{\pgfqpoint{1.962733in}{1.783528in}} %
\pgfusepath{clip}%
\pgfsetroundcap%
\pgfsetroundjoin%
\pgfsetlinewidth{0.803000pt}%
\definecolor{currentstroke}{rgb}{1.000000,1.000000,1.000000}%
\pgfsetstrokecolor{currentstroke}%
\pgfsetdash{}{0pt}%
\pgfpathmoveto{\pgfqpoint{2.816705in}{0.962104in}}%
\pgfpathlineto{\pgfqpoint{4.779438in}{0.962104in}}%
\pgfusepath{stroke}%
\end{pgfscope}%
\begin{pgfscope}%
\pgfsetbuttcap%
\pgfsetroundjoin%
\definecolor{currentfill}{rgb}{0.150000,0.150000,0.150000}%
\pgfsetfillcolor{currentfill}%
\pgfsetlinewidth{0.803000pt}%
\definecolor{currentstroke}{rgb}{0.150000,0.150000,0.150000}%
\pgfsetstrokecolor{currentstroke}%
\pgfsetdash{}{0pt}%
\pgfsys@defobject{currentmarker}{\pgfqpoint{0.000000in}{0.000000in}}{\pgfqpoint{0.000000in}{0.000000in}}{%
\pgfpathmoveto{\pgfqpoint{0.000000in}{0.000000in}}%
\pgfpathlineto{\pgfqpoint{0.000000in}{0.000000in}}%
\pgfusepath{stroke,fill}%
}%
\begin{pgfscope}%
\pgfsys@transformshift{2.816705in}{0.962104in}%
\pgfsys@useobject{currentmarker}{}%
\end{pgfscope}%
\end{pgfscope}%
\begin{pgfscope}%
\pgfpathrectangle{\pgfqpoint{2.816705in}{0.516222in}}{\pgfqpoint{1.962733in}{1.783528in}} %
\pgfusepath{clip}%
\pgfsetroundcap%
\pgfsetroundjoin%
\pgfsetlinewidth{0.803000pt}%
\definecolor{currentstroke}{rgb}{1.000000,1.000000,1.000000}%
\pgfsetstrokecolor{currentstroke}%
\pgfsetdash{}{0pt}%
\pgfpathmoveto{\pgfqpoint{2.816705in}{1.185045in}}%
\pgfpathlineto{\pgfqpoint{4.779438in}{1.185045in}}%
\pgfusepath{stroke}%
\end{pgfscope}%
\begin{pgfscope}%
\pgfsetbuttcap%
\pgfsetroundjoin%
\definecolor{currentfill}{rgb}{0.150000,0.150000,0.150000}%
\pgfsetfillcolor{currentfill}%
\pgfsetlinewidth{0.803000pt}%
\definecolor{currentstroke}{rgb}{0.150000,0.150000,0.150000}%
\pgfsetstrokecolor{currentstroke}%
\pgfsetdash{}{0pt}%
\pgfsys@defobject{currentmarker}{\pgfqpoint{0.000000in}{0.000000in}}{\pgfqpoint{0.000000in}{0.000000in}}{%
\pgfpathmoveto{\pgfqpoint{0.000000in}{0.000000in}}%
\pgfpathlineto{\pgfqpoint{0.000000in}{0.000000in}}%
\pgfusepath{stroke,fill}%
}%
\begin{pgfscope}%
\pgfsys@transformshift{2.816705in}{1.185045in}%
\pgfsys@useobject{currentmarker}{}%
\end{pgfscope}%
\end{pgfscope}%
\begin{pgfscope}%
\pgfpathrectangle{\pgfqpoint{2.816705in}{0.516222in}}{\pgfqpoint{1.962733in}{1.783528in}} %
\pgfusepath{clip}%
\pgfsetroundcap%
\pgfsetroundjoin%
\pgfsetlinewidth{0.803000pt}%
\definecolor{currentstroke}{rgb}{1.000000,1.000000,1.000000}%
\pgfsetstrokecolor{currentstroke}%
\pgfsetdash{}{0pt}%
\pgfpathmoveto{\pgfqpoint{2.816705in}{1.407986in}}%
\pgfpathlineto{\pgfqpoint{4.779438in}{1.407986in}}%
\pgfusepath{stroke}%
\end{pgfscope}%
\begin{pgfscope}%
\pgfsetbuttcap%
\pgfsetroundjoin%
\definecolor{currentfill}{rgb}{0.150000,0.150000,0.150000}%
\pgfsetfillcolor{currentfill}%
\pgfsetlinewidth{0.803000pt}%
\definecolor{currentstroke}{rgb}{0.150000,0.150000,0.150000}%
\pgfsetstrokecolor{currentstroke}%
\pgfsetdash{}{0pt}%
\pgfsys@defobject{currentmarker}{\pgfqpoint{0.000000in}{0.000000in}}{\pgfqpoint{0.000000in}{0.000000in}}{%
\pgfpathmoveto{\pgfqpoint{0.000000in}{0.000000in}}%
\pgfpathlineto{\pgfqpoint{0.000000in}{0.000000in}}%
\pgfusepath{stroke,fill}%
}%
\begin{pgfscope}%
\pgfsys@transformshift{2.816705in}{1.407986in}%
\pgfsys@useobject{currentmarker}{}%
\end{pgfscope}%
\end{pgfscope}%
\begin{pgfscope}%
\pgfpathrectangle{\pgfqpoint{2.816705in}{0.516222in}}{\pgfqpoint{1.962733in}{1.783528in}} %
\pgfusepath{clip}%
\pgfsetroundcap%
\pgfsetroundjoin%
\pgfsetlinewidth{0.803000pt}%
\definecolor{currentstroke}{rgb}{1.000000,1.000000,1.000000}%
\pgfsetstrokecolor{currentstroke}%
\pgfsetdash{}{0pt}%
\pgfpathmoveto{\pgfqpoint{2.816705in}{1.630927in}}%
\pgfpathlineto{\pgfqpoint{4.779438in}{1.630927in}}%
\pgfusepath{stroke}%
\end{pgfscope}%
\begin{pgfscope}%
\pgfsetbuttcap%
\pgfsetroundjoin%
\definecolor{currentfill}{rgb}{0.150000,0.150000,0.150000}%
\pgfsetfillcolor{currentfill}%
\pgfsetlinewidth{0.803000pt}%
\definecolor{currentstroke}{rgb}{0.150000,0.150000,0.150000}%
\pgfsetstrokecolor{currentstroke}%
\pgfsetdash{}{0pt}%
\pgfsys@defobject{currentmarker}{\pgfqpoint{0.000000in}{0.000000in}}{\pgfqpoint{0.000000in}{0.000000in}}{%
\pgfpathmoveto{\pgfqpoint{0.000000in}{0.000000in}}%
\pgfpathlineto{\pgfqpoint{0.000000in}{0.000000in}}%
\pgfusepath{stroke,fill}%
}%
\begin{pgfscope}%
\pgfsys@transformshift{2.816705in}{1.630927in}%
\pgfsys@useobject{currentmarker}{}%
\end{pgfscope}%
\end{pgfscope}%
\begin{pgfscope}%
\pgfpathrectangle{\pgfqpoint{2.816705in}{0.516222in}}{\pgfqpoint{1.962733in}{1.783528in}} %
\pgfusepath{clip}%
\pgfsetroundcap%
\pgfsetroundjoin%
\pgfsetlinewidth{0.803000pt}%
\definecolor{currentstroke}{rgb}{1.000000,1.000000,1.000000}%
\pgfsetstrokecolor{currentstroke}%
\pgfsetdash{}{0pt}%
\pgfpathmoveto{\pgfqpoint{2.816705in}{1.853868in}}%
\pgfpathlineto{\pgfqpoint{4.779438in}{1.853868in}}%
\pgfusepath{stroke}%
\end{pgfscope}%
\begin{pgfscope}%
\pgfsetbuttcap%
\pgfsetroundjoin%
\definecolor{currentfill}{rgb}{0.150000,0.150000,0.150000}%
\pgfsetfillcolor{currentfill}%
\pgfsetlinewidth{0.803000pt}%
\definecolor{currentstroke}{rgb}{0.150000,0.150000,0.150000}%
\pgfsetstrokecolor{currentstroke}%
\pgfsetdash{}{0pt}%
\pgfsys@defobject{currentmarker}{\pgfqpoint{0.000000in}{0.000000in}}{\pgfqpoint{0.000000in}{0.000000in}}{%
\pgfpathmoveto{\pgfqpoint{0.000000in}{0.000000in}}%
\pgfpathlineto{\pgfqpoint{0.000000in}{0.000000in}}%
\pgfusepath{stroke,fill}%
}%
\begin{pgfscope}%
\pgfsys@transformshift{2.816705in}{1.853868in}%
\pgfsys@useobject{currentmarker}{}%
\end{pgfscope}%
\end{pgfscope}%
\begin{pgfscope}%
\pgfpathrectangle{\pgfqpoint{2.816705in}{0.516222in}}{\pgfqpoint{1.962733in}{1.783528in}} %
\pgfusepath{clip}%
\pgfsetroundcap%
\pgfsetroundjoin%
\pgfsetlinewidth{0.803000pt}%
\definecolor{currentstroke}{rgb}{1.000000,1.000000,1.000000}%
\pgfsetstrokecolor{currentstroke}%
\pgfsetdash{}{0pt}%
\pgfpathmoveto{\pgfqpoint{2.816705in}{2.076809in}}%
\pgfpathlineto{\pgfqpoint{4.779438in}{2.076809in}}%
\pgfusepath{stroke}%
\end{pgfscope}%
\begin{pgfscope}%
\pgfsetbuttcap%
\pgfsetroundjoin%
\definecolor{currentfill}{rgb}{0.150000,0.150000,0.150000}%
\pgfsetfillcolor{currentfill}%
\pgfsetlinewidth{0.803000pt}%
\definecolor{currentstroke}{rgb}{0.150000,0.150000,0.150000}%
\pgfsetstrokecolor{currentstroke}%
\pgfsetdash{}{0pt}%
\pgfsys@defobject{currentmarker}{\pgfqpoint{0.000000in}{0.000000in}}{\pgfqpoint{0.000000in}{0.000000in}}{%
\pgfpathmoveto{\pgfqpoint{0.000000in}{0.000000in}}%
\pgfpathlineto{\pgfqpoint{0.000000in}{0.000000in}}%
\pgfusepath{stroke,fill}%
}%
\begin{pgfscope}%
\pgfsys@transformshift{2.816705in}{2.076809in}%
\pgfsys@useobject{currentmarker}{}%
\end{pgfscope}%
\end{pgfscope}%
\begin{pgfscope}%
\pgfpathrectangle{\pgfqpoint{2.816705in}{0.516222in}}{\pgfqpoint{1.962733in}{1.783528in}} %
\pgfusepath{clip}%
\pgfsetroundcap%
\pgfsetroundjoin%
\pgfsetlinewidth{0.803000pt}%
\definecolor{currentstroke}{rgb}{1.000000,1.000000,1.000000}%
\pgfsetstrokecolor{currentstroke}%
\pgfsetdash{}{0pt}%
\pgfpathmoveto{\pgfqpoint{2.816705in}{2.299750in}}%
\pgfpathlineto{\pgfqpoint{4.779438in}{2.299750in}}%
\pgfusepath{stroke}%
\end{pgfscope}%
\begin{pgfscope}%
\pgfsetbuttcap%
\pgfsetroundjoin%
\definecolor{currentfill}{rgb}{0.150000,0.150000,0.150000}%
\pgfsetfillcolor{currentfill}%
\pgfsetlinewidth{0.803000pt}%
\definecolor{currentstroke}{rgb}{0.150000,0.150000,0.150000}%
\pgfsetstrokecolor{currentstroke}%
\pgfsetdash{}{0pt}%
\pgfsys@defobject{currentmarker}{\pgfqpoint{0.000000in}{0.000000in}}{\pgfqpoint{0.000000in}{0.000000in}}{%
\pgfpathmoveto{\pgfqpoint{0.000000in}{0.000000in}}%
\pgfpathlineto{\pgfqpoint{0.000000in}{0.000000in}}%
\pgfusepath{stroke,fill}%
}%
\begin{pgfscope}%
\pgfsys@transformshift{2.816705in}{2.299750in}%
\pgfsys@useobject{currentmarker}{}%
\end{pgfscope}%
\end{pgfscope}%
\begin{pgfscope}%
\pgfpathrectangle{\pgfqpoint{2.816705in}{0.516222in}}{\pgfqpoint{1.962733in}{1.783528in}} %
\pgfusepath{clip}%
\pgfsetbuttcap%
\pgfsetroundjoin%
\definecolor{currentfill}{rgb}{0.298039,0.447059,0.690196}%
\pgfsetfillcolor{currentfill}%
\pgfsetlinewidth{0.240900pt}%
\definecolor{currentstroke}{rgb}{1.000000,1.000000,1.000000}%
\pgfsetstrokecolor{currentstroke}%
\pgfsetdash{}{0pt}%
\pgfpathmoveto{\pgfqpoint{3.826110in}{1.153989in}}%
\pgfpathcurveto{\pgfqpoint{3.834346in}{1.153989in}}{\pgfqpoint{3.842247in}{1.157261in}}{\pgfqpoint{3.848070in}{1.163085in}}%
\pgfpathcurveto{\pgfqpoint{3.853894in}{1.168909in}}{\pgfqpoint{3.857167in}{1.176809in}}{\pgfqpoint{3.857167in}{1.185045in}}%
\pgfpathcurveto{\pgfqpoint{3.857167in}{1.193281in}}{\pgfqpoint{3.853894in}{1.201181in}}{\pgfqpoint{3.848070in}{1.207005in}}%
\pgfpathcurveto{\pgfqpoint{3.842247in}{1.212829in}}{\pgfqpoint{3.834346in}{1.216102in}}{\pgfqpoint{3.826110in}{1.216102in}}%
\pgfpathcurveto{\pgfqpoint{3.817874in}{1.216102in}}{\pgfqpoint{3.809974in}{1.212829in}}{\pgfqpoint{3.804150in}{1.207005in}}%
\pgfpathcurveto{\pgfqpoint{3.798326in}{1.201181in}}{\pgfqpoint{3.795054in}{1.193281in}}{\pgfqpoint{3.795054in}{1.185045in}}%
\pgfpathcurveto{\pgfqpoint{3.795054in}{1.176809in}}{\pgfqpoint{3.798326in}{1.168909in}}{\pgfqpoint{3.804150in}{1.163085in}}%
\pgfpathcurveto{\pgfqpoint{3.809974in}{1.157261in}}{\pgfqpoint{3.817874in}{1.153989in}}{\pgfqpoint{3.826110in}{1.153989in}}%
\pgfpathclose%
\pgfusepath{stroke,fill}%
\end{pgfscope}%
\begin{pgfscope}%
\pgfpathrectangle{\pgfqpoint{2.816705in}{0.516222in}}{\pgfqpoint{1.962733in}{1.783528in}} %
\pgfusepath{clip}%
\pgfsetbuttcap%
\pgfsetroundjoin%
\definecolor{currentfill}{rgb}{0.298039,0.447059,0.690196}%
\pgfsetfillcolor{currentfill}%
\pgfsetlinewidth{0.240900pt}%
\definecolor{currentstroke}{rgb}{1.000000,1.000000,1.000000}%
\pgfsetstrokecolor{currentstroke}%
\pgfsetdash{}{0pt}%
\pgfpathmoveto{\pgfqpoint{3.657876in}{0.846330in}}%
\pgfpathcurveto{\pgfqpoint{3.666112in}{0.846330in}}{\pgfqpoint{3.674012in}{0.849602in}}{\pgfqpoint{3.679836in}{0.855426in}}%
\pgfpathcurveto{\pgfqpoint{3.685660in}{0.861250in}}{\pgfqpoint{3.688932in}{0.869150in}}{\pgfqpoint{3.688932in}{0.877387in}}%
\pgfpathcurveto{\pgfqpoint{3.688932in}{0.885623in}}{\pgfqpoint{3.685660in}{0.893523in}}{\pgfqpoint{3.679836in}{0.899347in}}%
\pgfpathcurveto{\pgfqpoint{3.674012in}{0.905171in}}{\pgfqpoint{3.666112in}{0.908443in}}{\pgfqpoint{3.657876in}{0.908443in}}%
\pgfpathcurveto{\pgfqpoint{3.649640in}{0.908443in}}{\pgfqpoint{3.641740in}{0.905171in}}{\pgfqpoint{3.635916in}{0.899347in}}%
\pgfpathcurveto{\pgfqpoint{3.630092in}{0.893523in}}{\pgfqpoint{3.626819in}{0.885623in}}{\pgfqpoint{3.626819in}{0.877387in}}%
\pgfpathcurveto{\pgfqpoint{3.626819in}{0.869150in}}{\pgfqpoint{3.630092in}{0.861250in}}{\pgfqpoint{3.635916in}{0.855426in}}%
\pgfpathcurveto{\pgfqpoint{3.641740in}{0.849602in}}{\pgfqpoint{3.649640in}{0.846330in}}{\pgfqpoint{3.657876in}{0.846330in}}%
\pgfpathclose%
\pgfusepath{stroke,fill}%
\end{pgfscope}%
\begin{pgfscope}%
\pgfpathrectangle{\pgfqpoint{2.816705in}{0.516222in}}{\pgfqpoint{1.962733in}{1.783528in}} %
\pgfusepath{clip}%
\pgfsetbuttcap%
\pgfsetroundjoin%
\definecolor{currentfill}{rgb}{0.298039,0.447059,0.690196}%
\pgfsetfillcolor{currentfill}%
\pgfsetlinewidth{0.240900pt}%
\definecolor{currentstroke}{rgb}{1.000000,1.000000,1.000000}%
\pgfsetstrokecolor{currentstroke}%
\pgfsetdash{}{0pt}%
\pgfpathmoveto{\pgfqpoint{3.713954in}{1.318965in}}%
\pgfpathcurveto{\pgfqpoint{3.722190in}{1.318965in}}{\pgfqpoint{3.730090in}{1.322237in}}{\pgfqpoint{3.735914in}{1.328061in}}%
\pgfpathcurveto{\pgfqpoint{3.741738in}{1.333885in}}{\pgfqpoint{3.745011in}{1.341785in}}{\pgfqpoint{3.745011in}{1.350021in}}%
\pgfpathcurveto{\pgfqpoint{3.745011in}{1.358258in}}{\pgfqpoint{3.741738in}{1.366158in}}{\pgfqpoint{3.735914in}{1.371982in}}%
\pgfpathcurveto{\pgfqpoint{3.730090in}{1.377806in}}{\pgfqpoint{3.722190in}{1.381078in}}{\pgfqpoint{3.713954in}{1.381078in}}%
\pgfpathcurveto{\pgfqpoint{3.705718in}{1.381078in}}{\pgfqpoint{3.697818in}{1.377806in}}{\pgfqpoint{3.691994in}{1.371982in}}%
\pgfpathcurveto{\pgfqpoint{3.686170in}{1.366158in}}{\pgfqpoint{3.682898in}{1.358258in}}{\pgfqpoint{3.682898in}{1.350021in}}%
\pgfpathcurveto{\pgfqpoint{3.682898in}{1.341785in}}{\pgfqpoint{3.686170in}{1.333885in}}{\pgfqpoint{3.691994in}{1.328061in}}%
\pgfpathcurveto{\pgfqpoint{3.697818in}{1.322237in}}{\pgfqpoint{3.705718in}{1.318965in}}{\pgfqpoint{3.713954in}{1.318965in}}%
\pgfpathclose%
\pgfusepath{stroke,fill}%
\end{pgfscope}%
\begin{pgfscope}%
\pgfpathrectangle{\pgfqpoint{2.816705in}{0.516222in}}{\pgfqpoint{1.962733in}{1.783528in}} %
\pgfusepath{clip}%
\pgfsetbuttcap%
\pgfsetroundjoin%
\definecolor{currentfill}{rgb}{0.298039,0.447059,0.690196}%
\pgfsetfillcolor{currentfill}%
\pgfsetlinewidth{0.240900pt}%
\definecolor{currentstroke}{rgb}{1.000000,1.000000,1.000000}%
\pgfsetstrokecolor{currentstroke}%
\pgfsetdash{}{0pt}%
\pgfpathmoveto{\pgfqpoint{4.162579in}{1.666753in}}%
\pgfpathcurveto{\pgfqpoint{4.170815in}{1.666753in}}{\pgfqpoint{4.178715in}{1.670025in}}{\pgfqpoint{4.184539in}{1.675849in}}%
\pgfpathcurveto{\pgfqpoint{4.190363in}{1.681673in}}{\pgfqpoint{4.193635in}{1.689573in}}{\pgfqpoint{4.193635in}{1.697809in}}%
\pgfpathcurveto{\pgfqpoint{4.193635in}{1.706046in}}{\pgfqpoint{4.190363in}{1.713946in}}{\pgfqpoint{4.184539in}{1.719770in}}%
\pgfpathcurveto{\pgfqpoint{4.178715in}{1.725594in}}{\pgfqpoint{4.170815in}{1.728866in}}{\pgfqpoint{4.162579in}{1.728866in}}%
\pgfpathcurveto{\pgfqpoint{4.154342in}{1.728866in}}{\pgfqpoint{4.146442in}{1.725594in}}{\pgfqpoint{4.140618in}{1.719770in}}%
\pgfpathcurveto{\pgfqpoint{4.134794in}{1.713946in}}{\pgfqpoint{4.131522in}{1.706046in}}{\pgfqpoint{4.131522in}{1.697809in}}%
\pgfpathcurveto{\pgfqpoint{4.131522in}{1.689573in}}{\pgfqpoint{4.134794in}{1.681673in}}{\pgfqpoint{4.140618in}{1.675849in}}%
\pgfpathcurveto{\pgfqpoint{4.146442in}{1.670025in}}{\pgfqpoint{4.154342in}{1.666753in}}{\pgfqpoint{4.162579in}{1.666753in}}%
\pgfpathclose%
\pgfusepath{stroke,fill}%
\end{pgfscope}%
\begin{pgfscope}%
\pgfpathrectangle{\pgfqpoint{2.816705in}{0.516222in}}{\pgfqpoint{1.962733in}{1.783528in}} %
\pgfusepath{clip}%
\pgfsetbuttcap%
\pgfsetroundjoin%
\definecolor{currentfill}{rgb}{0.298039,0.447059,0.690196}%
\pgfsetfillcolor{currentfill}%
\pgfsetlinewidth{0.240900pt}%
\definecolor{currentstroke}{rgb}{1.000000,1.000000,1.000000}%
\pgfsetstrokecolor{currentstroke}%
\pgfsetdash{}{0pt}%
\pgfpathmoveto{\pgfqpoint{3.601798in}{0.980095in}}%
\pgfpathcurveto{\pgfqpoint{3.610034in}{0.980095in}}{\pgfqpoint{3.617934in}{0.983367in}}{\pgfqpoint{3.623758in}{0.989191in}}%
\pgfpathcurveto{\pgfqpoint{3.629582in}{0.995015in}}{\pgfqpoint{3.632854in}{1.002915in}}{\pgfqpoint{3.632854in}{1.011151in}}%
\pgfpathcurveto{\pgfqpoint{3.632854in}{1.019387in}}{\pgfqpoint{3.629582in}{1.027288in}}{\pgfqpoint{3.623758in}{1.033111in}}%
\pgfpathcurveto{\pgfqpoint{3.617934in}{1.038935in}}{\pgfqpoint{3.610034in}{1.042208in}}{\pgfqpoint{3.601798in}{1.042208in}}%
\pgfpathcurveto{\pgfqpoint{3.593562in}{1.042208in}}{\pgfqpoint{3.585662in}{1.038935in}}{\pgfqpoint{3.579838in}{1.033111in}}%
\pgfpathcurveto{\pgfqpoint{3.574014in}{1.027288in}}{\pgfqpoint{3.570741in}{1.019387in}}{\pgfqpoint{3.570741in}{1.011151in}}%
\pgfpathcurveto{\pgfqpoint{3.570741in}{1.002915in}}{\pgfqpoint{3.574014in}{0.995015in}}{\pgfqpoint{3.579838in}{0.989191in}}%
\pgfpathcurveto{\pgfqpoint{3.585662in}{0.983367in}}{\pgfqpoint{3.593562in}{0.980095in}}{\pgfqpoint{3.601798in}{0.980095in}}%
\pgfpathclose%
\pgfusepath{stroke,fill}%
\end{pgfscope}%
\begin{pgfscope}%
\pgfpathrectangle{\pgfqpoint{2.816705in}{0.516222in}}{\pgfqpoint{1.962733in}{1.783528in}} %
\pgfusepath{clip}%
\pgfsetbuttcap%
\pgfsetroundjoin%
\definecolor{currentfill}{rgb}{0.298039,0.447059,0.690196}%
\pgfsetfillcolor{currentfill}%
\pgfsetlinewidth{0.240900pt}%
\definecolor{currentstroke}{rgb}{1.000000,1.000000,1.000000}%
\pgfsetstrokecolor{currentstroke}%
\pgfsetdash{}{0pt}%
\pgfpathmoveto{\pgfqpoint{4.218657in}{2.036835in}}%
\pgfpathcurveto{\pgfqpoint{4.226893in}{2.036835in}}{\pgfqpoint{4.234793in}{2.040107in}}{\pgfqpoint{4.240617in}{2.045931in}}%
\pgfpathcurveto{\pgfqpoint{4.246441in}{2.051755in}}{\pgfqpoint{4.249713in}{2.059655in}}{\pgfqpoint{4.249713in}{2.067891in}}%
\pgfpathcurveto{\pgfqpoint{4.249713in}{2.076128in}}{\pgfqpoint{4.246441in}{2.084028in}}{\pgfqpoint{4.240617in}{2.089852in}}%
\pgfpathcurveto{\pgfqpoint{4.234793in}{2.095676in}}{\pgfqpoint{4.226893in}{2.098948in}}{\pgfqpoint{4.218657in}{2.098948in}}%
\pgfpathcurveto{\pgfqpoint{4.210420in}{2.098948in}}{\pgfqpoint{4.202520in}{2.095676in}}{\pgfqpoint{4.196696in}{2.089852in}}%
\pgfpathcurveto{\pgfqpoint{4.190873in}{2.084028in}}{\pgfqpoint{4.187600in}{2.076128in}}{\pgfqpoint{4.187600in}{2.067891in}}%
\pgfpathcurveto{\pgfqpoint{4.187600in}{2.059655in}}{\pgfqpoint{4.190873in}{2.051755in}}{\pgfqpoint{4.196696in}{2.045931in}}%
\pgfpathcurveto{\pgfqpoint{4.202520in}{2.040107in}}{\pgfqpoint{4.210420in}{2.036835in}}{\pgfqpoint{4.218657in}{2.036835in}}%
\pgfpathclose%
\pgfusepath{stroke,fill}%
\end{pgfscope}%
\begin{pgfscope}%
\pgfpathrectangle{\pgfqpoint{2.816705in}{0.516222in}}{\pgfqpoint{1.962733in}{1.783528in}} %
\pgfusepath{clip}%
\pgfsetbuttcap%
\pgfsetroundjoin%
\definecolor{currentfill}{rgb}{0.298039,0.447059,0.690196}%
\pgfsetfillcolor{currentfill}%
\pgfsetlinewidth{0.240900pt}%
\definecolor{currentstroke}{rgb}{1.000000,1.000000,1.000000}%
\pgfsetstrokecolor{currentstroke}%
\pgfsetdash{}{0pt}%
\pgfpathmoveto{\pgfqpoint{3.601798in}{0.739318in}}%
\pgfpathcurveto{\pgfqpoint{3.610034in}{0.739318in}}{\pgfqpoint{3.617934in}{0.742591in}}{\pgfqpoint{3.623758in}{0.748415in}}%
\pgfpathcurveto{\pgfqpoint{3.629582in}{0.754239in}}{\pgfqpoint{3.632854in}{0.762139in}}{\pgfqpoint{3.632854in}{0.770375in}}%
\pgfpathcurveto{\pgfqpoint{3.632854in}{0.778611in}}{\pgfqpoint{3.629582in}{0.786511in}}{\pgfqpoint{3.623758in}{0.792335in}}%
\pgfpathcurveto{\pgfqpoint{3.617934in}{0.798159in}}{\pgfqpoint{3.610034in}{0.801431in}}{\pgfqpoint{3.601798in}{0.801431in}}%
\pgfpathcurveto{\pgfqpoint{3.593562in}{0.801431in}}{\pgfqpoint{3.585662in}{0.798159in}}{\pgfqpoint{3.579838in}{0.792335in}}%
\pgfpathcurveto{\pgfqpoint{3.574014in}{0.786511in}}{\pgfqpoint{3.570741in}{0.778611in}}{\pgfqpoint{3.570741in}{0.770375in}}%
\pgfpathcurveto{\pgfqpoint{3.570741in}{0.762139in}}{\pgfqpoint{3.574014in}{0.754239in}}{\pgfqpoint{3.579838in}{0.748415in}}%
\pgfpathcurveto{\pgfqpoint{3.585662in}{0.742591in}}{\pgfqpoint{3.593562in}{0.739318in}}{\pgfqpoint{3.601798in}{0.739318in}}%
\pgfpathclose%
\pgfusepath{stroke,fill}%
\end{pgfscope}%
\begin{pgfscope}%
\pgfpathrectangle{\pgfqpoint{2.816705in}{0.516222in}}{\pgfqpoint{1.962733in}{1.783528in}} %
\pgfusepath{clip}%
\pgfsetbuttcap%
\pgfsetroundjoin%
\definecolor{currentfill}{rgb}{0.298039,0.447059,0.690196}%
\pgfsetfillcolor{currentfill}%
\pgfsetlinewidth{0.240900pt}%
\definecolor{currentstroke}{rgb}{1.000000,1.000000,1.000000}%
\pgfsetstrokecolor{currentstroke}%
\pgfsetdash{}{0pt}%
\pgfpathmoveto{\pgfqpoint{4.442969in}{1.760388in}}%
\pgfpathcurveto{\pgfqpoint{4.451205in}{1.760388in}}{\pgfqpoint{4.459105in}{1.763660in}}{\pgfqpoint{4.464929in}{1.769484in}}%
\pgfpathcurveto{\pgfqpoint{4.470753in}{1.775308in}}{\pgfqpoint{4.474026in}{1.783208in}}{\pgfqpoint{4.474026in}{1.791445in}}%
\pgfpathcurveto{\pgfqpoint{4.474026in}{1.799681in}}{\pgfqpoint{4.470753in}{1.807581in}}{\pgfqpoint{4.464929in}{1.813405in}}%
\pgfpathcurveto{\pgfqpoint{4.459105in}{1.819229in}}{\pgfqpoint{4.451205in}{1.822501in}}{\pgfqpoint{4.442969in}{1.822501in}}%
\pgfpathcurveto{\pgfqpoint{4.434733in}{1.822501in}}{\pgfqpoint{4.426833in}{1.819229in}}{\pgfqpoint{4.421009in}{1.813405in}}%
\pgfpathcurveto{\pgfqpoint{4.415185in}{1.807581in}}{\pgfqpoint{4.411913in}{1.799681in}}{\pgfqpoint{4.411913in}{1.791445in}}%
\pgfpathcurveto{\pgfqpoint{4.411913in}{1.783208in}}{\pgfqpoint{4.415185in}{1.775308in}}{\pgfqpoint{4.421009in}{1.769484in}}%
\pgfpathcurveto{\pgfqpoint{4.426833in}{1.763660in}}{\pgfqpoint{4.434733in}{1.760388in}}{\pgfqpoint{4.442969in}{1.760388in}}%
\pgfpathclose%
\pgfusepath{stroke,fill}%
\end{pgfscope}%
\begin{pgfscope}%
\pgfpathrectangle{\pgfqpoint{2.816705in}{0.516222in}}{\pgfqpoint{1.962733in}{1.783528in}} %
\pgfusepath{clip}%
\pgfsetbuttcap%
\pgfsetroundjoin%
\definecolor{currentfill}{rgb}{0.298039,0.447059,0.690196}%
\pgfsetfillcolor{currentfill}%
\pgfsetlinewidth{0.240900pt}%
\definecolor{currentstroke}{rgb}{1.000000,1.000000,1.000000}%
\pgfsetstrokecolor{currentstroke}%
\pgfsetdash{}{0pt}%
\pgfpathmoveto{\pgfqpoint{3.097095in}{0.837412in}}%
\pgfpathcurveto{\pgfqpoint{3.105332in}{0.837412in}}{\pgfqpoint{3.113232in}{0.840685in}}{\pgfqpoint{3.119055in}{0.846509in}}%
\pgfpathcurveto{\pgfqpoint{3.124879in}{0.852333in}}{\pgfqpoint{3.128152in}{0.860233in}}{\pgfqpoint{3.128152in}{0.868469in}}%
\pgfpathcurveto{\pgfqpoint{3.128152in}{0.876705in}}{\pgfqpoint{3.124879in}{0.884605in}}{\pgfqpoint{3.119055in}{0.890429in}}%
\pgfpathcurveto{\pgfqpoint{3.113232in}{0.896253in}}{\pgfqpoint{3.105332in}{0.899525in}}{\pgfqpoint{3.097095in}{0.899525in}}%
\pgfpathcurveto{\pgfqpoint{3.088859in}{0.899525in}}{\pgfqpoint{3.080959in}{0.896253in}}{\pgfqpoint{3.075135in}{0.890429in}}%
\pgfpathcurveto{\pgfqpoint{3.069311in}{0.884605in}}{\pgfqpoint{3.066039in}{0.876705in}}{\pgfqpoint{3.066039in}{0.868469in}}%
\pgfpathcurveto{\pgfqpoint{3.066039in}{0.860233in}}{\pgfqpoint{3.069311in}{0.852333in}}{\pgfqpoint{3.075135in}{0.846509in}}%
\pgfpathcurveto{\pgfqpoint{3.080959in}{0.840685in}}{\pgfqpoint{3.088859in}{0.837412in}}{\pgfqpoint{3.097095in}{0.837412in}}%
\pgfpathclose%
\pgfusepath{stroke,fill}%
\end{pgfscope}%
\begin{pgfscope}%
\pgfpathrectangle{\pgfqpoint{2.816705in}{0.516222in}}{\pgfqpoint{1.962733in}{1.783528in}} %
\pgfusepath{clip}%
\pgfsetbuttcap%
\pgfsetroundjoin%
\definecolor{currentfill}{rgb}{0.298039,0.447059,0.690196}%
\pgfsetfillcolor{currentfill}%
\pgfsetlinewidth{0.240900pt}%
\definecolor{currentstroke}{rgb}{1.000000,1.000000,1.000000}%
\pgfsetstrokecolor{currentstroke}%
\pgfsetdash{}{0pt}%
\pgfpathmoveto{\pgfqpoint{3.657876in}{1.394765in}}%
\pgfpathcurveto{\pgfqpoint{3.666112in}{1.394765in}}{\pgfqpoint{3.674012in}{1.398037in}}{\pgfqpoint{3.679836in}{1.403861in}}%
\pgfpathcurveto{\pgfqpoint{3.685660in}{1.409685in}}{\pgfqpoint{3.688932in}{1.417585in}}{\pgfqpoint{3.688932in}{1.425821in}}%
\pgfpathcurveto{\pgfqpoint{3.688932in}{1.434058in}}{\pgfqpoint{3.685660in}{1.441958in}}{\pgfqpoint{3.679836in}{1.447782in}}%
\pgfpathcurveto{\pgfqpoint{3.674012in}{1.453606in}}{\pgfqpoint{3.666112in}{1.456878in}}{\pgfqpoint{3.657876in}{1.456878in}}%
\pgfpathcurveto{\pgfqpoint{3.649640in}{1.456878in}}{\pgfqpoint{3.641740in}{1.453606in}}{\pgfqpoint{3.635916in}{1.447782in}}%
\pgfpathcurveto{\pgfqpoint{3.630092in}{1.441958in}}{\pgfqpoint{3.626819in}{1.434058in}}{\pgfqpoint{3.626819in}{1.425821in}}%
\pgfpathcurveto{\pgfqpoint{3.626819in}{1.417585in}}{\pgfqpoint{3.630092in}{1.409685in}}{\pgfqpoint{3.635916in}{1.403861in}}%
\pgfpathcurveto{\pgfqpoint{3.641740in}{1.398037in}}{\pgfqpoint{3.649640in}{1.394765in}}{\pgfqpoint{3.657876in}{1.394765in}}%
\pgfpathclose%
\pgfusepath{stroke,fill}%
\end{pgfscope}%
\begin{pgfscope}%
\pgfpathrectangle{\pgfqpoint{2.816705in}{0.516222in}}{\pgfqpoint{1.962733in}{1.783528in}} %
\pgfusepath{clip}%
\pgfsetbuttcap%
\pgfsetroundjoin%
\definecolor{currentfill}{rgb}{0.298039,0.447059,0.690196}%
\pgfsetfillcolor{currentfill}%
\pgfsetlinewidth{0.240900pt}%
\definecolor{currentstroke}{rgb}{1.000000,1.000000,1.000000}%
\pgfsetstrokecolor{currentstroke}%
\pgfsetdash{}{0pt}%
\pgfpathmoveto{\pgfqpoint{3.657876in}{0.806201in}}%
\pgfpathcurveto{\pgfqpoint{3.666112in}{0.806201in}}{\pgfqpoint{3.674012in}{0.809473in}}{\pgfqpoint{3.679836in}{0.815297in}}%
\pgfpathcurveto{\pgfqpoint{3.685660in}{0.821121in}}{\pgfqpoint{3.688932in}{0.829021in}}{\pgfqpoint{3.688932in}{0.837257in}}%
\pgfpathcurveto{\pgfqpoint{3.688932in}{0.845494in}}{\pgfqpoint{3.685660in}{0.853394in}}{\pgfqpoint{3.679836in}{0.859217in}}%
\pgfpathcurveto{\pgfqpoint{3.674012in}{0.865041in}}{\pgfqpoint{3.666112in}{0.868314in}}{\pgfqpoint{3.657876in}{0.868314in}}%
\pgfpathcurveto{\pgfqpoint{3.649640in}{0.868314in}}{\pgfqpoint{3.641740in}{0.865041in}}{\pgfqpoint{3.635916in}{0.859217in}}%
\pgfpathcurveto{\pgfqpoint{3.630092in}{0.853394in}}{\pgfqpoint{3.626819in}{0.845494in}}{\pgfqpoint{3.626819in}{0.837257in}}%
\pgfpathcurveto{\pgfqpoint{3.626819in}{0.829021in}}{\pgfqpoint{3.630092in}{0.821121in}}{\pgfqpoint{3.635916in}{0.815297in}}%
\pgfpathcurveto{\pgfqpoint{3.641740in}{0.809473in}}{\pgfqpoint{3.649640in}{0.806201in}}{\pgfqpoint{3.657876in}{0.806201in}}%
\pgfpathclose%
\pgfusepath{stroke,fill}%
\end{pgfscope}%
\begin{pgfscope}%
\pgfpathrectangle{\pgfqpoint{2.816705in}{0.516222in}}{\pgfqpoint{1.962733in}{1.783528in}} %
\pgfusepath{clip}%
\pgfsetbuttcap%
\pgfsetroundjoin%
\definecolor{currentfill}{rgb}{0.298039,0.447059,0.690196}%
\pgfsetfillcolor{currentfill}%
\pgfsetlinewidth{0.240900pt}%
\definecolor{currentstroke}{rgb}{1.000000,1.000000,1.000000}%
\pgfsetstrokecolor{currentstroke}%
\pgfsetdash{}{0pt}%
\pgfpathmoveto{\pgfqpoint{3.545720in}{0.792824in}}%
\pgfpathcurveto{\pgfqpoint{3.553956in}{0.792824in}}{\pgfqpoint{3.561856in}{0.796097in}}{\pgfqpoint{3.567680in}{0.801921in}}%
\pgfpathcurveto{\pgfqpoint{3.573504in}{0.807744in}}{\pgfqpoint{3.576776in}{0.815644in}}{\pgfqpoint{3.576776in}{0.823881in}}%
\pgfpathcurveto{\pgfqpoint{3.576776in}{0.832117in}}{\pgfqpoint{3.573504in}{0.840017in}}{\pgfqpoint{3.567680in}{0.845841in}}%
\pgfpathcurveto{\pgfqpoint{3.561856in}{0.851665in}}{\pgfqpoint{3.553956in}{0.854937in}}{\pgfqpoint{3.545720in}{0.854937in}}%
\pgfpathcurveto{\pgfqpoint{3.537484in}{0.854937in}}{\pgfqpoint{3.529584in}{0.851665in}}{\pgfqpoint{3.523760in}{0.845841in}}%
\pgfpathcurveto{\pgfqpoint{3.517936in}{0.840017in}}{\pgfqpoint{3.514663in}{0.832117in}}{\pgfqpoint{3.514663in}{0.823881in}}%
\pgfpathcurveto{\pgfqpoint{3.514663in}{0.815644in}}{\pgfqpoint{3.517936in}{0.807744in}}{\pgfqpoint{3.523760in}{0.801921in}}%
\pgfpathcurveto{\pgfqpoint{3.529584in}{0.796097in}}{\pgfqpoint{3.537484in}{0.792824in}}{\pgfqpoint{3.545720in}{0.792824in}}%
\pgfpathclose%
\pgfusepath{stroke,fill}%
\end{pgfscope}%
\begin{pgfscope}%
\pgfpathrectangle{\pgfqpoint{2.816705in}{0.516222in}}{\pgfqpoint{1.962733in}{1.783528in}} %
\pgfusepath{clip}%
\pgfsetbuttcap%
\pgfsetroundjoin%
\definecolor{currentfill}{rgb}{0.298039,0.447059,0.690196}%
\pgfsetfillcolor{currentfill}%
\pgfsetlinewidth{0.240900pt}%
\definecolor{currentstroke}{rgb}{1.000000,1.000000,1.000000}%
\pgfsetstrokecolor{currentstroke}%
\pgfsetdash{}{0pt}%
\pgfpathmoveto{\pgfqpoint{3.601798in}{2.094800in}}%
\pgfpathcurveto{\pgfqpoint{3.610034in}{2.094800in}}{\pgfqpoint{3.617934in}{2.098072in}}{\pgfqpoint{3.623758in}{2.103896in}}%
\pgfpathcurveto{\pgfqpoint{3.629582in}{2.109720in}}{\pgfqpoint{3.632854in}{2.117620in}}{\pgfqpoint{3.632854in}{2.125856in}}%
\pgfpathcurveto{\pgfqpoint{3.632854in}{2.134092in}}{\pgfqpoint{3.629582in}{2.141992in}}{\pgfqpoint{3.623758in}{2.147816in}}%
\pgfpathcurveto{\pgfqpoint{3.617934in}{2.153640in}}{\pgfqpoint{3.610034in}{2.156913in}}{\pgfqpoint{3.601798in}{2.156913in}}%
\pgfpathcurveto{\pgfqpoint{3.593562in}{2.156913in}}{\pgfqpoint{3.585662in}{2.153640in}}{\pgfqpoint{3.579838in}{2.147816in}}%
\pgfpathcurveto{\pgfqpoint{3.574014in}{2.141992in}}{\pgfqpoint{3.570741in}{2.134092in}}{\pgfqpoint{3.570741in}{2.125856in}}%
\pgfpathcurveto{\pgfqpoint{3.570741in}{2.117620in}}{\pgfqpoint{3.574014in}{2.109720in}}{\pgfqpoint{3.579838in}{2.103896in}}%
\pgfpathcurveto{\pgfqpoint{3.585662in}{2.098072in}}{\pgfqpoint{3.593562in}{2.094800in}}{\pgfqpoint{3.601798in}{2.094800in}}%
\pgfpathclose%
\pgfusepath{stroke,fill}%
\end{pgfscope}%
\begin{pgfscope}%
\pgfpathrectangle{\pgfqpoint{2.816705in}{0.516222in}}{\pgfqpoint{1.962733in}{1.783528in}} %
\pgfusepath{clip}%
\pgfsetbuttcap%
\pgfsetroundjoin%
\definecolor{currentfill}{rgb}{0.298039,0.447059,0.690196}%
\pgfsetfillcolor{currentfill}%
\pgfsetlinewidth{0.240900pt}%
\definecolor{currentstroke}{rgb}{1.000000,1.000000,1.000000}%
\pgfsetstrokecolor{currentstroke}%
\pgfsetdash{}{0pt}%
\pgfpathmoveto{\pgfqpoint{2.984939in}{1.033601in}}%
\pgfpathcurveto{\pgfqpoint{2.993175in}{1.033601in}}{\pgfqpoint{3.001075in}{1.036873in}}{\pgfqpoint{3.006899in}{1.042697in}}%
\pgfpathcurveto{\pgfqpoint{3.012723in}{1.048521in}}{\pgfqpoint{3.015996in}{1.056421in}}{\pgfqpoint{3.015996in}{1.064657in}}%
\pgfpathcurveto{\pgfqpoint{3.015996in}{1.072893in}}{\pgfqpoint{3.012723in}{1.080793in}}{\pgfqpoint{3.006899in}{1.086617in}}%
\pgfpathcurveto{\pgfqpoint{3.001075in}{1.092441in}}{\pgfqpoint{2.993175in}{1.095714in}}{\pgfqpoint{2.984939in}{1.095714in}}%
\pgfpathcurveto{\pgfqpoint{2.976703in}{1.095714in}}{\pgfqpoint{2.968803in}{1.092441in}}{\pgfqpoint{2.962979in}{1.086617in}}%
\pgfpathcurveto{\pgfqpoint{2.957155in}{1.080793in}}{\pgfqpoint{2.953883in}{1.072893in}}{\pgfqpoint{2.953883in}{1.064657in}}%
\pgfpathcurveto{\pgfqpoint{2.953883in}{1.056421in}}{\pgfqpoint{2.957155in}{1.048521in}}{\pgfqpoint{2.962979in}{1.042697in}}%
\pgfpathcurveto{\pgfqpoint{2.968803in}{1.036873in}}{\pgfqpoint{2.976703in}{1.033601in}}{\pgfqpoint{2.984939in}{1.033601in}}%
\pgfpathclose%
\pgfusepath{stroke,fill}%
\end{pgfscope}%
\begin{pgfscope}%
\pgfpathrectangle{\pgfqpoint{2.816705in}{0.516222in}}{\pgfqpoint{1.962733in}{1.783528in}} %
\pgfusepath{clip}%
\pgfsetbuttcap%
\pgfsetroundjoin%
\definecolor{currentfill}{rgb}{0.298039,0.447059,0.690196}%
\pgfsetfillcolor{currentfill}%
\pgfsetlinewidth{0.240900pt}%
\definecolor{currentstroke}{rgb}{1.000000,1.000000,1.000000}%
\pgfsetstrokecolor{currentstroke}%
\pgfsetdash{}{0pt}%
\pgfpathmoveto{\pgfqpoint{3.938266in}{1.457188in}}%
\pgfpathcurveto{\pgfqpoint{3.946503in}{1.457188in}}{\pgfqpoint{3.954403in}{1.460461in}}{\pgfqpoint{3.960227in}{1.466285in}}%
\pgfpathcurveto{\pgfqpoint{3.966051in}{1.472109in}}{\pgfqpoint{3.969323in}{1.480009in}}{\pgfqpoint{3.969323in}{1.488245in}}%
\pgfpathcurveto{\pgfqpoint{3.969323in}{1.496481in}}{\pgfqpoint{3.966051in}{1.504381in}}{\pgfqpoint{3.960227in}{1.510205in}}%
\pgfpathcurveto{\pgfqpoint{3.954403in}{1.516029in}}{\pgfqpoint{3.946503in}{1.519301in}}{\pgfqpoint{3.938266in}{1.519301in}}%
\pgfpathcurveto{\pgfqpoint{3.930030in}{1.519301in}}{\pgfqpoint{3.922130in}{1.516029in}}{\pgfqpoint{3.916306in}{1.510205in}}%
\pgfpathcurveto{\pgfqpoint{3.910482in}{1.504381in}}{\pgfqpoint{3.907210in}{1.496481in}}{\pgfqpoint{3.907210in}{1.488245in}}%
\pgfpathcurveto{\pgfqpoint{3.907210in}{1.480009in}}{\pgfqpoint{3.910482in}{1.472109in}}{\pgfqpoint{3.916306in}{1.466285in}}%
\pgfpathcurveto{\pgfqpoint{3.922130in}{1.460461in}}{\pgfqpoint{3.930030in}{1.457188in}}{\pgfqpoint{3.938266in}{1.457188in}}%
\pgfpathclose%
\pgfusepath{stroke,fill}%
\end{pgfscope}%
\begin{pgfscope}%
\pgfpathrectangle{\pgfqpoint{2.816705in}{0.516222in}}{\pgfqpoint{1.962733in}{1.783528in}} %
\pgfusepath{clip}%
\pgfsetbuttcap%
\pgfsetroundjoin%
\definecolor{currentfill}{rgb}{0.298039,0.447059,0.690196}%
\pgfsetfillcolor{currentfill}%
\pgfsetlinewidth{0.240900pt}%
\definecolor{currentstroke}{rgb}{1.000000,1.000000,1.000000}%
\pgfsetstrokecolor{currentstroke}%
\pgfsetdash{}{0pt}%
\pgfpathmoveto{\pgfqpoint{3.882188in}{1.372471in}}%
\pgfpathcurveto{\pgfqpoint{3.890425in}{1.372471in}}{\pgfqpoint{3.898325in}{1.375743in}}{\pgfqpoint{3.904149in}{1.381567in}}%
\pgfpathcurveto{\pgfqpoint{3.909972in}{1.387391in}}{\pgfqpoint{3.913245in}{1.395291in}}{\pgfqpoint{3.913245in}{1.403527in}}%
\pgfpathcurveto{\pgfqpoint{3.913245in}{1.411764in}}{\pgfqpoint{3.909972in}{1.419664in}}{\pgfqpoint{3.904149in}{1.425488in}}%
\pgfpathcurveto{\pgfqpoint{3.898325in}{1.431311in}}{\pgfqpoint{3.890425in}{1.434584in}}{\pgfqpoint{3.882188in}{1.434584in}}%
\pgfpathcurveto{\pgfqpoint{3.873952in}{1.434584in}}{\pgfqpoint{3.866052in}{1.431311in}}{\pgfqpoint{3.860228in}{1.425488in}}%
\pgfpathcurveto{\pgfqpoint{3.854404in}{1.419664in}}{\pgfqpoint{3.851132in}{1.411764in}}{\pgfqpoint{3.851132in}{1.403527in}}%
\pgfpathcurveto{\pgfqpoint{3.851132in}{1.395291in}}{\pgfqpoint{3.854404in}{1.387391in}}{\pgfqpoint{3.860228in}{1.381567in}}%
\pgfpathcurveto{\pgfqpoint{3.866052in}{1.375743in}}{\pgfqpoint{3.873952in}{1.372471in}}{\pgfqpoint{3.882188in}{1.372471in}}%
\pgfpathclose%
\pgfusepath{stroke,fill}%
\end{pgfscope}%
\begin{pgfscope}%
\pgfpathrectangle{\pgfqpoint{2.816705in}{0.516222in}}{\pgfqpoint{1.962733in}{1.783528in}} %
\pgfusepath{clip}%
\pgfsetbuttcap%
\pgfsetroundjoin%
\definecolor{currentfill}{rgb}{0.298039,0.447059,0.690196}%
\pgfsetfillcolor{currentfill}%
\pgfsetlinewidth{0.240900pt}%
\definecolor{currentstroke}{rgb}{1.000000,1.000000,1.000000}%
\pgfsetstrokecolor{currentstroke}%
\pgfsetdash{}{0pt}%
\pgfpathmoveto{\pgfqpoint{3.433564in}{0.672436in}}%
\pgfpathcurveto{\pgfqpoint{3.441800in}{0.672436in}}{\pgfqpoint{3.449700in}{0.675708in}}{\pgfqpoint{3.455524in}{0.681532in}}%
\pgfpathcurveto{\pgfqpoint{3.461348in}{0.687356in}}{\pgfqpoint{3.464620in}{0.695256in}}{\pgfqpoint{3.464620in}{0.703493in}}%
\pgfpathcurveto{\pgfqpoint{3.464620in}{0.711729in}}{\pgfqpoint{3.461348in}{0.719629in}}{\pgfqpoint{3.455524in}{0.725453in}}%
\pgfpathcurveto{\pgfqpoint{3.449700in}{0.731277in}}{\pgfqpoint{3.441800in}{0.734549in}}{\pgfqpoint{3.433564in}{0.734549in}}%
\pgfpathcurveto{\pgfqpoint{3.425327in}{0.734549in}}{\pgfqpoint{3.417427in}{0.731277in}}{\pgfqpoint{3.411603in}{0.725453in}}%
\pgfpathcurveto{\pgfqpoint{3.405779in}{0.719629in}}{\pgfqpoint{3.402507in}{0.711729in}}{\pgfqpoint{3.402507in}{0.703493in}}%
\pgfpathcurveto{\pgfqpoint{3.402507in}{0.695256in}}{\pgfqpoint{3.405779in}{0.687356in}}{\pgfqpoint{3.411603in}{0.681532in}}%
\pgfpathcurveto{\pgfqpoint{3.417427in}{0.675708in}}{\pgfqpoint{3.425327in}{0.672436in}}{\pgfqpoint{3.433564in}{0.672436in}}%
\pgfpathclose%
\pgfusepath{stroke,fill}%
\end{pgfscope}%
\begin{pgfscope}%
\pgfpathrectangle{\pgfqpoint{2.816705in}{0.516222in}}{\pgfqpoint{1.962733in}{1.783528in}} %
\pgfusepath{clip}%
\pgfsetbuttcap%
\pgfsetroundjoin%
\definecolor{currentfill}{rgb}{0.298039,0.447059,0.690196}%
\pgfsetfillcolor{currentfill}%
\pgfsetlinewidth{0.240900pt}%
\definecolor{currentstroke}{rgb}{1.000000,1.000000,1.000000}%
\pgfsetstrokecolor{currentstroke}%
\pgfsetdash{}{0pt}%
\pgfpathmoveto{\pgfqpoint{3.601798in}{0.895377in}}%
\pgfpathcurveto{\pgfqpoint{3.610034in}{0.895377in}}{\pgfqpoint{3.617934in}{0.898649in}}{\pgfqpoint{3.623758in}{0.904473in}}%
\pgfpathcurveto{\pgfqpoint{3.629582in}{0.910297in}}{\pgfqpoint{3.632854in}{0.918197in}}{\pgfqpoint{3.632854in}{0.926434in}}%
\pgfpathcurveto{\pgfqpoint{3.632854in}{0.934670in}}{\pgfqpoint{3.629582in}{0.942570in}}{\pgfqpoint{3.623758in}{0.948394in}}%
\pgfpathcurveto{\pgfqpoint{3.617934in}{0.954218in}}{\pgfqpoint{3.610034in}{0.957490in}}{\pgfqpoint{3.601798in}{0.957490in}}%
\pgfpathcurveto{\pgfqpoint{3.593562in}{0.957490in}}{\pgfqpoint{3.585662in}{0.954218in}}{\pgfqpoint{3.579838in}{0.948394in}}%
\pgfpathcurveto{\pgfqpoint{3.574014in}{0.942570in}}{\pgfqpoint{3.570741in}{0.934670in}}{\pgfqpoint{3.570741in}{0.926434in}}%
\pgfpathcurveto{\pgfqpoint{3.570741in}{0.918197in}}{\pgfqpoint{3.574014in}{0.910297in}}{\pgfqpoint{3.579838in}{0.904473in}}%
\pgfpathcurveto{\pgfqpoint{3.585662in}{0.898649in}}{\pgfqpoint{3.593562in}{0.895377in}}{\pgfqpoint{3.601798in}{0.895377in}}%
\pgfpathclose%
\pgfusepath{stroke,fill}%
\end{pgfscope}%
\begin{pgfscope}%
\pgfpathrectangle{\pgfqpoint{2.816705in}{0.516222in}}{\pgfqpoint{1.962733in}{1.783528in}} %
\pgfusepath{clip}%
\pgfsetbuttcap%
\pgfsetroundjoin%
\definecolor{currentfill}{rgb}{0.298039,0.447059,0.690196}%
\pgfsetfillcolor{currentfill}%
\pgfsetlinewidth{0.240900pt}%
\definecolor{currentstroke}{rgb}{1.000000,1.000000,1.000000}%
\pgfsetstrokecolor{currentstroke}%
\pgfsetdash{}{0pt}%
\pgfpathmoveto{\pgfqpoint{3.209251in}{0.975636in}}%
\pgfpathcurveto{\pgfqpoint{3.217488in}{0.975636in}}{\pgfqpoint{3.225388in}{0.978908in}}{\pgfqpoint{3.231212in}{0.984732in}}%
\pgfpathcurveto{\pgfqpoint{3.237036in}{0.990556in}}{\pgfqpoint{3.240308in}{0.998456in}}{\pgfqpoint{3.240308in}{1.006692in}}%
\pgfpathcurveto{\pgfqpoint{3.240308in}{1.014929in}}{\pgfqpoint{3.237036in}{1.022829in}}{\pgfqpoint{3.231212in}{1.028653in}}%
\pgfpathcurveto{\pgfqpoint{3.225388in}{1.034477in}}{\pgfqpoint{3.217488in}{1.037749in}}{\pgfqpoint{3.209251in}{1.037749in}}%
\pgfpathcurveto{\pgfqpoint{3.201015in}{1.037749in}}{\pgfqpoint{3.193115in}{1.034477in}}{\pgfqpoint{3.187291in}{1.028653in}}%
\pgfpathcurveto{\pgfqpoint{3.181467in}{1.022829in}}{\pgfqpoint{3.178195in}{1.014929in}}{\pgfqpoint{3.178195in}{1.006692in}}%
\pgfpathcurveto{\pgfqpoint{3.178195in}{0.998456in}}{\pgfqpoint{3.181467in}{0.990556in}}{\pgfqpoint{3.187291in}{0.984732in}}%
\pgfpathcurveto{\pgfqpoint{3.193115in}{0.978908in}}{\pgfqpoint{3.201015in}{0.975636in}}{\pgfqpoint{3.209251in}{0.975636in}}%
\pgfpathclose%
\pgfusepath{stroke,fill}%
\end{pgfscope}%
\begin{pgfscope}%
\pgfpathrectangle{\pgfqpoint{2.816705in}{0.516222in}}{\pgfqpoint{1.962733in}{1.783528in}} %
\pgfusepath{clip}%
\pgfsetbuttcap%
\pgfsetroundjoin%
\definecolor{currentfill}{rgb}{0.298039,0.447059,0.690196}%
\pgfsetfillcolor{currentfill}%
\pgfsetlinewidth{0.240900pt}%
\definecolor{currentstroke}{rgb}{1.000000,1.000000,1.000000}%
\pgfsetstrokecolor{currentstroke}%
\pgfsetdash{}{0pt}%
\pgfpathmoveto{\pgfqpoint{3.545720in}{0.730401in}}%
\pgfpathcurveto{\pgfqpoint{3.553956in}{0.730401in}}{\pgfqpoint{3.561856in}{0.733673in}}{\pgfqpoint{3.567680in}{0.739497in}}%
\pgfpathcurveto{\pgfqpoint{3.573504in}{0.745321in}}{\pgfqpoint{3.576776in}{0.753221in}}{\pgfqpoint{3.576776in}{0.761457in}}%
\pgfpathcurveto{\pgfqpoint{3.576776in}{0.769694in}}{\pgfqpoint{3.573504in}{0.777594in}}{\pgfqpoint{3.567680in}{0.783418in}}%
\pgfpathcurveto{\pgfqpoint{3.561856in}{0.789241in}}{\pgfqpoint{3.553956in}{0.792514in}}{\pgfqpoint{3.545720in}{0.792514in}}%
\pgfpathcurveto{\pgfqpoint{3.537484in}{0.792514in}}{\pgfqpoint{3.529584in}{0.789241in}}{\pgfqpoint{3.523760in}{0.783418in}}%
\pgfpathcurveto{\pgfqpoint{3.517936in}{0.777594in}}{\pgfqpoint{3.514663in}{0.769694in}}{\pgfqpoint{3.514663in}{0.761457in}}%
\pgfpathcurveto{\pgfqpoint{3.514663in}{0.753221in}}{\pgfqpoint{3.517936in}{0.745321in}}{\pgfqpoint{3.523760in}{0.739497in}}%
\pgfpathcurveto{\pgfqpoint{3.529584in}{0.733673in}}{\pgfqpoint{3.537484in}{0.730401in}}{\pgfqpoint{3.545720in}{0.730401in}}%
\pgfpathclose%
\pgfusepath{stroke,fill}%
\end{pgfscope}%
\begin{pgfscope}%
\pgfpathrectangle{\pgfqpoint{2.816705in}{0.516222in}}{\pgfqpoint{1.962733in}{1.783528in}} %
\pgfusepath{clip}%
\pgfsetbuttcap%
\pgfsetroundjoin%
\definecolor{currentfill}{rgb}{0.298039,0.447059,0.690196}%
\pgfsetfillcolor{currentfill}%
\pgfsetlinewidth{0.240900pt}%
\definecolor{currentstroke}{rgb}{1.000000,1.000000,1.000000}%
\pgfsetstrokecolor{currentstroke}%
\pgfsetdash{}{0pt}%
\pgfpathmoveto{\pgfqpoint{3.882188in}{1.956576in}}%
\pgfpathcurveto{\pgfqpoint{3.890425in}{1.956576in}}{\pgfqpoint{3.898325in}{1.959848in}}{\pgfqpoint{3.904149in}{1.965672in}}%
\pgfpathcurveto{\pgfqpoint{3.909972in}{1.971496in}}{\pgfqpoint{3.913245in}{1.979396in}}{\pgfqpoint{3.913245in}{1.987633in}}%
\pgfpathcurveto{\pgfqpoint{3.913245in}{1.995869in}}{\pgfqpoint{3.909972in}{2.003769in}}{\pgfqpoint{3.904149in}{2.009593in}}%
\pgfpathcurveto{\pgfqpoint{3.898325in}{2.015417in}}{\pgfqpoint{3.890425in}{2.018689in}}{\pgfqpoint{3.882188in}{2.018689in}}%
\pgfpathcurveto{\pgfqpoint{3.873952in}{2.018689in}}{\pgfqpoint{3.866052in}{2.015417in}}{\pgfqpoint{3.860228in}{2.009593in}}%
\pgfpathcurveto{\pgfqpoint{3.854404in}{2.003769in}}{\pgfqpoint{3.851132in}{1.995869in}}{\pgfqpoint{3.851132in}{1.987633in}}%
\pgfpathcurveto{\pgfqpoint{3.851132in}{1.979396in}}{\pgfqpoint{3.854404in}{1.971496in}}{\pgfqpoint{3.860228in}{1.965672in}}%
\pgfpathcurveto{\pgfqpoint{3.866052in}{1.959848in}}{\pgfqpoint{3.873952in}{1.956576in}}{\pgfqpoint{3.882188in}{1.956576in}}%
\pgfpathclose%
\pgfusepath{stroke,fill}%
\end{pgfscope}%
\begin{pgfscope}%
\pgfpathrectangle{\pgfqpoint{2.816705in}{0.516222in}}{\pgfqpoint{1.962733in}{1.783528in}} %
\pgfusepath{clip}%
\pgfsetbuttcap%
\pgfsetroundjoin%
\definecolor{currentfill}{rgb}{0.298039,0.447059,0.690196}%
\pgfsetfillcolor{currentfill}%
\pgfsetlinewidth{0.240900pt}%
\definecolor{currentstroke}{rgb}{1.000000,1.000000,1.000000}%
\pgfsetstrokecolor{currentstroke}%
\pgfsetdash{}{0pt}%
\pgfpathmoveto{\pgfqpoint{4.050423in}{1.390306in}}%
\pgfpathcurveto{\pgfqpoint{4.058659in}{1.390306in}}{\pgfqpoint{4.066559in}{1.393578in}}{\pgfqpoint{4.072383in}{1.399402in}}%
\pgfpathcurveto{\pgfqpoint{4.078207in}{1.405226in}}{\pgfqpoint{4.081479in}{1.413126in}}{\pgfqpoint{4.081479in}{1.421363in}}%
\pgfpathcurveto{\pgfqpoint{4.081479in}{1.429599in}}{\pgfqpoint{4.078207in}{1.437499in}}{\pgfqpoint{4.072383in}{1.443323in}}%
\pgfpathcurveto{\pgfqpoint{4.066559in}{1.449147in}}{\pgfqpoint{4.058659in}{1.452419in}}{\pgfqpoint{4.050423in}{1.452419in}}%
\pgfpathcurveto{\pgfqpoint{4.042186in}{1.452419in}}{\pgfqpoint{4.034286in}{1.449147in}}{\pgfqpoint{4.028462in}{1.443323in}}%
\pgfpathcurveto{\pgfqpoint{4.022638in}{1.437499in}}{\pgfqpoint{4.019366in}{1.429599in}}{\pgfqpoint{4.019366in}{1.421363in}}%
\pgfpathcurveto{\pgfqpoint{4.019366in}{1.413126in}}{\pgfqpoint{4.022638in}{1.405226in}}{\pgfqpoint{4.028462in}{1.399402in}}%
\pgfpathcurveto{\pgfqpoint{4.034286in}{1.393578in}}{\pgfqpoint{4.042186in}{1.390306in}}{\pgfqpoint{4.050423in}{1.390306in}}%
\pgfpathclose%
\pgfusepath{stroke,fill}%
\end{pgfscope}%
\begin{pgfscope}%
\pgfpathrectangle{\pgfqpoint{2.816705in}{0.516222in}}{\pgfqpoint{1.962733in}{1.783528in}} %
\pgfusepath{clip}%
\pgfsetbuttcap%
\pgfsetroundjoin%
\definecolor{currentfill}{rgb}{0.298039,0.447059,0.690196}%
\pgfsetfillcolor{currentfill}%
\pgfsetlinewidth{0.240900pt}%
\definecolor{currentstroke}{rgb}{1.000000,1.000000,1.000000}%
\pgfsetstrokecolor{currentstroke}%
\pgfsetdash{}{0pt}%
\pgfpathmoveto{\pgfqpoint{3.601798in}{1.952117in}}%
\pgfpathcurveto{\pgfqpoint{3.610034in}{1.952117in}}{\pgfqpoint{3.617934in}{1.955390in}}{\pgfqpoint{3.623758in}{1.961214in}}%
\pgfpathcurveto{\pgfqpoint{3.629582in}{1.967037in}}{\pgfqpoint{3.632854in}{1.974938in}}{\pgfqpoint{3.632854in}{1.983174in}}%
\pgfpathcurveto{\pgfqpoint{3.632854in}{1.991410in}}{\pgfqpoint{3.629582in}{1.999310in}}{\pgfqpoint{3.623758in}{2.005134in}}%
\pgfpathcurveto{\pgfqpoint{3.617934in}{2.010958in}}{\pgfqpoint{3.610034in}{2.014230in}}{\pgfqpoint{3.601798in}{2.014230in}}%
\pgfpathcurveto{\pgfqpoint{3.593562in}{2.014230in}}{\pgfqpoint{3.585662in}{2.010958in}}{\pgfqpoint{3.579838in}{2.005134in}}%
\pgfpathcurveto{\pgfqpoint{3.574014in}{1.999310in}}{\pgfqpoint{3.570741in}{1.991410in}}{\pgfqpoint{3.570741in}{1.983174in}}%
\pgfpathcurveto{\pgfqpoint{3.570741in}{1.974938in}}{\pgfqpoint{3.574014in}{1.967037in}}{\pgfqpoint{3.579838in}{1.961214in}}%
\pgfpathcurveto{\pgfqpoint{3.585662in}{1.955390in}}{\pgfqpoint{3.593562in}{1.952117in}}{\pgfqpoint{3.601798in}{1.952117in}}%
\pgfpathclose%
\pgfusepath{stroke,fill}%
\end{pgfscope}%
\begin{pgfscope}%
\pgfpathrectangle{\pgfqpoint{2.816705in}{0.516222in}}{\pgfqpoint{1.962733in}{1.783528in}} %
\pgfusepath{clip}%
\pgfsetbuttcap%
\pgfsetroundjoin%
\definecolor{currentfill}{rgb}{0.298039,0.447059,0.690196}%
\pgfsetfillcolor{currentfill}%
\pgfsetlinewidth{0.240900pt}%
\definecolor{currentstroke}{rgb}{1.000000,1.000000,1.000000}%
\pgfsetstrokecolor{currentstroke}%
\pgfsetdash{}{0pt}%
\pgfpathmoveto{\pgfqpoint{3.433564in}{1.702423in}}%
\pgfpathcurveto{\pgfqpoint{3.441800in}{1.702423in}}{\pgfqpoint{3.449700in}{1.705696in}}{\pgfqpoint{3.455524in}{1.711520in}}%
\pgfpathcurveto{\pgfqpoint{3.461348in}{1.717344in}}{\pgfqpoint{3.464620in}{1.725244in}}{\pgfqpoint{3.464620in}{1.733480in}}%
\pgfpathcurveto{\pgfqpoint{3.464620in}{1.741716in}}{\pgfqpoint{3.461348in}{1.749616in}}{\pgfqpoint{3.455524in}{1.755440in}}%
\pgfpathcurveto{\pgfqpoint{3.449700in}{1.761264in}}{\pgfqpoint{3.441800in}{1.764536in}}{\pgfqpoint{3.433564in}{1.764536in}}%
\pgfpathcurveto{\pgfqpoint{3.425327in}{1.764536in}}{\pgfqpoint{3.417427in}{1.761264in}}{\pgfqpoint{3.411603in}{1.755440in}}%
\pgfpathcurveto{\pgfqpoint{3.405779in}{1.749616in}}{\pgfqpoint{3.402507in}{1.741716in}}{\pgfqpoint{3.402507in}{1.733480in}}%
\pgfpathcurveto{\pgfqpoint{3.402507in}{1.725244in}}{\pgfqpoint{3.405779in}{1.717344in}}{\pgfqpoint{3.411603in}{1.711520in}}%
\pgfpathcurveto{\pgfqpoint{3.417427in}{1.705696in}}{\pgfqpoint{3.425327in}{1.702423in}}{\pgfqpoint{3.433564in}{1.702423in}}%
\pgfpathclose%
\pgfusepath{stroke,fill}%
\end{pgfscope}%
\begin{pgfscope}%
\pgfpathrectangle{\pgfqpoint{2.816705in}{0.516222in}}{\pgfqpoint{1.962733in}{1.783528in}} %
\pgfusepath{clip}%
\pgfsetbuttcap%
\pgfsetroundjoin%
\definecolor{currentfill}{rgb}{0.298039,0.447059,0.690196}%
\pgfsetfillcolor{currentfill}%
\pgfsetlinewidth{0.240900pt}%
\definecolor{currentstroke}{rgb}{1.000000,1.000000,1.000000}%
\pgfsetstrokecolor{currentstroke}%
\pgfsetdash{}{0pt}%
\pgfpathmoveto{\pgfqpoint{3.433564in}{1.715800in}}%
\pgfpathcurveto{\pgfqpoint{3.441800in}{1.715800in}}{\pgfqpoint{3.449700in}{1.719072in}}{\pgfqpoint{3.455524in}{1.724896in}}%
\pgfpathcurveto{\pgfqpoint{3.461348in}{1.730720in}}{\pgfqpoint{3.464620in}{1.738620in}}{\pgfqpoint{3.464620in}{1.746856in}}%
\pgfpathcurveto{\pgfqpoint{3.464620in}{1.755093in}}{\pgfqpoint{3.461348in}{1.762993in}}{\pgfqpoint{3.455524in}{1.768817in}}%
\pgfpathcurveto{\pgfqpoint{3.449700in}{1.774641in}}{\pgfqpoint{3.441800in}{1.777913in}}{\pgfqpoint{3.433564in}{1.777913in}}%
\pgfpathcurveto{\pgfqpoint{3.425327in}{1.777913in}}{\pgfqpoint{3.417427in}{1.774641in}}{\pgfqpoint{3.411603in}{1.768817in}}%
\pgfpathcurveto{\pgfqpoint{3.405779in}{1.762993in}}{\pgfqpoint{3.402507in}{1.755093in}}{\pgfqpoint{3.402507in}{1.746856in}}%
\pgfpathcurveto{\pgfqpoint{3.402507in}{1.738620in}}{\pgfqpoint{3.405779in}{1.730720in}}{\pgfqpoint{3.411603in}{1.724896in}}%
\pgfpathcurveto{\pgfqpoint{3.417427in}{1.719072in}}{\pgfqpoint{3.425327in}{1.715800in}}{\pgfqpoint{3.433564in}{1.715800in}}%
\pgfpathclose%
\pgfusepath{stroke,fill}%
\end{pgfscope}%
\begin{pgfscope}%
\pgfpathrectangle{\pgfqpoint{2.816705in}{0.516222in}}{\pgfqpoint{1.962733in}{1.783528in}} %
\pgfusepath{clip}%
\pgfsetbuttcap%
\pgfsetroundjoin%
\definecolor{currentfill}{rgb}{0.298039,0.447059,0.690196}%
\pgfsetfillcolor{currentfill}%
\pgfsetlinewidth{0.240900pt}%
\definecolor{currentstroke}{rgb}{1.000000,1.000000,1.000000}%
\pgfsetstrokecolor{currentstroke}%
\pgfsetdash{}{0pt}%
\pgfpathmoveto{\pgfqpoint{4.050423in}{1.671212in}}%
\pgfpathcurveto{\pgfqpoint{4.058659in}{1.671212in}}{\pgfqpoint{4.066559in}{1.674484in}}{\pgfqpoint{4.072383in}{1.680308in}}%
\pgfpathcurveto{\pgfqpoint{4.078207in}{1.686132in}}{\pgfqpoint{4.081479in}{1.694032in}}{\pgfqpoint{4.081479in}{1.702268in}}%
\pgfpathcurveto{\pgfqpoint{4.081479in}{1.710504in}}{\pgfqpoint{4.078207in}{1.718405in}}{\pgfqpoint{4.072383in}{1.724228in}}%
\pgfpathcurveto{\pgfqpoint{4.066559in}{1.730052in}}{\pgfqpoint{4.058659in}{1.733325in}}{\pgfqpoint{4.050423in}{1.733325in}}%
\pgfpathcurveto{\pgfqpoint{4.042186in}{1.733325in}}{\pgfqpoint{4.034286in}{1.730052in}}{\pgfqpoint{4.028462in}{1.724228in}}%
\pgfpathcurveto{\pgfqpoint{4.022638in}{1.718405in}}{\pgfqpoint{4.019366in}{1.710504in}}{\pgfqpoint{4.019366in}{1.702268in}}%
\pgfpathcurveto{\pgfqpoint{4.019366in}{1.694032in}}{\pgfqpoint{4.022638in}{1.686132in}}{\pgfqpoint{4.028462in}{1.680308in}}%
\pgfpathcurveto{\pgfqpoint{4.034286in}{1.674484in}}{\pgfqpoint{4.042186in}{1.671212in}}{\pgfqpoint{4.050423in}{1.671212in}}%
\pgfpathclose%
\pgfusepath{stroke,fill}%
\end{pgfscope}%
\begin{pgfscope}%
\pgfpathrectangle{\pgfqpoint{2.816705in}{0.516222in}}{\pgfqpoint{1.962733in}{1.783528in}} %
\pgfusepath{clip}%
\pgfsetbuttcap%
\pgfsetroundjoin%
\definecolor{currentfill}{rgb}{0.298039,0.447059,0.690196}%
\pgfsetfillcolor{currentfill}%
\pgfsetlinewidth{0.240900pt}%
\definecolor{currentstroke}{rgb}{1.000000,1.000000,1.000000}%
\pgfsetstrokecolor{currentstroke}%
\pgfsetdash{}{0pt}%
\pgfpathmoveto{\pgfqpoint{3.377486in}{0.725942in}}%
\pgfpathcurveto{\pgfqpoint{3.385722in}{0.725942in}}{\pgfqpoint{3.393622in}{0.729214in}}{\pgfqpoint{3.399446in}{0.735038in}}%
\pgfpathcurveto{\pgfqpoint{3.405270in}{0.740862in}}{\pgfqpoint{3.408542in}{0.748762in}}{\pgfqpoint{3.408542in}{0.756998in}}%
\pgfpathcurveto{\pgfqpoint{3.408542in}{0.765235in}}{\pgfqpoint{3.405270in}{0.773135in}}{\pgfqpoint{3.399446in}{0.778959in}}%
\pgfpathcurveto{\pgfqpoint{3.393622in}{0.784783in}}{\pgfqpoint{3.385722in}{0.788055in}}{\pgfqpoint{3.377486in}{0.788055in}}%
\pgfpathcurveto{\pgfqpoint{3.369249in}{0.788055in}}{\pgfqpoint{3.361349in}{0.784783in}}{\pgfqpoint{3.355525in}{0.778959in}}%
\pgfpathcurveto{\pgfqpoint{3.349701in}{0.773135in}}{\pgfqpoint{3.346429in}{0.765235in}}{\pgfqpoint{3.346429in}{0.756998in}}%
\pgfpathcurveto{\pgfqpoint{3.346429in}{0.748762in}}{\pgfqpoint{3.349701in}{0.740862in}}{\pgfqpoint{3.355525in}{0.735038in}}%
\pgfpathcurveto{\pgfqpoint{3.361349in}{0.729214in}}{\pgfqpoint{3.369249in}{0.725942in}}{\pgfqpoint{3.377486in}{0.725942in}}%
\pgfpathclose%
\pgfusepath{stroke,fill}%
\end{pgfscope}%
\begin{pgfscope}%
\pgfpathrectangle{\pgfqpoint{2.816705in}{0.516222in}}{\pgfqpoint{1.962733in}{1.783528in}} %
\pgfusepath{clip}%
\pgfsetbuttcap%
\pgfsetroundjoin%
\definecolor{currentfill}{rgb}{0.298039,0.447059,0.690196}%
\pgfsetfillcolor{currentfill}%
\pgfsetlinewidth{0.240900pt}%
\definecolor{currentstroke}{rgb}{1.000000,1.000000,1.000000}%
\pgfsetstrokecolor{currentstroke}%
\pgfsetdash{}{0pt}%
\pgfpathmoveto{\pgfqpoint{3.489642in}{1.943200in}}%
\pgfpathcurveto{\pgfqpoint{3.497878in}{1.943200in}}{\pgfqpoint{3.505778in}{1.946472in}}{\pgfqpoint{3.511602in}{1.952296in}}%
\pgfpathcurveto{\pgfqpoint{3.517426in}{1.958120in}}{\pgfqpoint{3.520698in}{1.966020in}}{\pgfqpoint{3.520698in}{1.974256in}}%
\pgfpathcurveto{\pgfqpoint{3.520698in}{1.982492in}}{\pgfqpoint{3.517426in}{1.990393in}}{\pgfqpoint{3.511602in}{1.996216in}}%
\pgfpathcurveto{\pgfqpoint{3.505778in}{2.002040in}}{\pgfqpoint{3.497878in}{2.005313in}}{\pgfqpoint{3.489642in}{2.005313in}}%
\pgfpathcurveto{\pgfqpoint{3.481405in}{2.005313in}}{\pgfqpoint{3.473505in}{2.002040in}}{\pgfqpoint{3.467682in}{1.996216in}}%
\pgfpathcurveto{\pgfqpoint{3.461858in}{1.990393in}}{\pgfqpoint{3.458585in}{1.982492in}}{\pgfqpoint{3.458585in}{1.974256in}}%
\pgfpathcurveto{\pgfqpoint{3.458585in}{1.966020in}}{\pgfqpoint{3.461858in}{1.958120in}}{\pgfqpoint{3.467682in}{1.952296in}}%
\pgfpathcurveto{\pgfqpoint{3.473505in}{1.946472in}}{\pgfqpoint{3.481405in}{1.943200in}}{\pgfqpoint{3.489642in}{1.943200in}}%
\pgfpathclose%
\pgfusepath{stroke,fill}%
\end{pgfscope}%
\begin{pgfscope}%
\pgfpathrectangle{\pgfqpoint{2.816705in}{0.516222in}}{\pgfqpoint{1.962733in}{1.783528in}} %
\pgfusepath{clip}%
\pgfsetbuttcap%
\pgfsetroundjoin%
\definecolor{currentfill}{rgb}{0.298039,0.447059,0.690196}%
\pgfsetfillcolor{currentfill}%
\pgfsetlinewidth{0.240900pt}%
\definecolor{currentstroke}{rgb}{1.000000,1.000000,1.000000}%
\pgfsetstrokecolor{currentstroke}%
\pgfsetdash{}{0pt}%
\pgfpathmoveto{\pgfqpoint{3.265329in}{1.381388in}}%
\pgfpathcurveto{\pgfqpoint{3.273566in}{1.381388in}}{\pgfqpoint{3.281466in}{1.384661in}}{\pgfqpoint{3.287290in}{1.390485in}}%
\pgfpathcurveto{\pgfqpoint{3.293114in}{1.396309in}}{\pgfqpoint{3.296386in}{1.404209in}}{\pgfqpoint{3.296386in}{1.412445in}}%
\pgfpathcurveto{\pgfqpoint{3.296386in}{1.420681in}}{\pgfqpoint{3.293114in}{1.428581in}}{\pgfqpoint{3.287290in}{1.434405in}}%
\pgfpathcurveto{\pgfqpoint{3.281466in}{1.440229in}}{\pgfqpoint{3.273566in}{1.443501in}}{\pgfqpoint{3.265329in}{1.443501in}}%
\pgfpathcurveto{\pgfqpoint{3.257093in}{1.443501in}}{\pgfqpoint{3.249193in}{1.440229in}}{\pgfqpoint{3.243369in}{1.434405in}}%
\pgfpathcurveto{\pgfqpoint{3.237545in}{1.428581in}}{\pgfqpoint{3.234273in}{1.420681in}}{\pgfqpoint{3.234273in}{1.412445in}}%
\pgfpathcurveto{\pgfqpoint{3.234273in}{1.404209in}}{\pgfqpoint{3.237545in}{1.396309in}}{\pgfqpoint{3.243369in}{1.390485in}}%
\pgfpathcurveto{\pgfqpoint{3.249193in}{1.384661in}}{\pgfqpoint{3.257093in}{1.381388in}}{\pgfqpoint{3.265329in}{1.381388in}}%
\pgfpathclose%
\pgfusepath{stroke,fill}%
\end{pgfscope}%
\begin{pgfscope}%
\pgfpathrectangle{\pgfqpoint{2.816705in}{0.516222in}}{\pgfqpoint{1.962733in}{1.783528in}} %
\pgfusepath{clip}%
\pgfsetbuttcap%
\pgfsetroundjoin%
\definecolor{currentfill}{rgb}{0.298039,0.447059,0.690196}%
\pgfsetfillcolor{currentfill}%
\pgfsetlinewidth{0.240900pt}%
\definecolor{currentstroke}{rgb}{1.000000,1.000000,1.000000}%
\pgfsetstrokecolor{currentstroke}%
\pgfsetdash{}{0pt}%
\pgfpathmoveto{\pgfqpoint{4.218657in}{1.987788in}}%
\pgfpathcurveto{\pgfqpoint{4.226893in}{1.987788in}}{\pgfqpoint{4.234793in}{1.991060in}}{\pgfqpoint{4.240617in}{1.996884in}}%
\pgfpathcurveto{\pgfqpoint{4.246441in}{2.002708in}}{\pgfqpoint{4.249713in}{2.010608in}}{\pgfqpoint{4.249713in}{2.018844in}}%
\pgfpathcurveto{\pgfqpoint{4.249713in}{2.027081in}}{\pgfqpoint{4.246441in}{2.034981in}}{\pgfqpoint{4.240617in}{2.040805in}}%
\pgfpathcurveto{\pgfqpoint{4.234793in}{2.046629in}}{\pgfqpoint{4.226893in}{2.049901in}}{\pgfqpoint{4.218657in}{2.049901in}}%
\pgfpathcurveto{\pgfqpoint{4.210420in}{2.049901in}}{\pgfqpoint{4.202520in}{2.046629in}}{\pgfqpoint{4.196696in}{2.040805in}}%
\pgfpathcurveto{\pgfqpoint{4.190873in}{2.034981in}}{\pgfqpoint{4.187600in}{2.027081in}}{\pgfqpoint{4.187600in}{2.018844in}}%
\pgfpathcurveto{\pgfqpoint{4.187600in}{2.010608in}}{\pgfqpoint{4.190873in}{2.002708in}}{\pgfqpoint{4.196696in}{1.996884in}}%
\pgfpathcurveto{\pgfqpoint{4.202520in}{1.991060in}}{\pgfqpoint{4.210420in}{1.987788in}}{\pgfqpoint{4.218657in}{1.987788in}}%
\pgfpathclose%
\pgfusepath{stroke,fill}%
\end{pgfscope}%
\begin{pgfscope}%
\pgfsetrectcap%
\pgfsetmiterjoin%
\pgfsetlinewidth{0.000000pt}%
\definecolor{currentstroke}{rgb}{1.000000,1.000000,1.000000}%
\pgfsetstrokecolor{currentstroke}%
\pgfsetdash{}{0pt}%
\pgfpathmoveto{\pgfqpoint{2.816705in}{0.516222in}}%
\pgfpathlineto{\pgfqpoint{2.816705in}{2.299750in}}%
\pgfusepath{}%
\end{pgfscope}%
\begin{pgfscope}%
\pgfsetrectcap%
\pgfsetmiterjoin%
\pgfsetlinewidth{0.000000pt}%
\definecolor{currentstroke}{rgb}{1.000000,1.000000,1.000000}%
\pgfsetstrokecolor{currentstroke}%
\pgfsetdash{}{0pt}%
\pgfpathmoveto{\pgfqpoint{2.816705in}{0.516222in}}%
\pgfpathlineto{\pgfqpoint{4.779438in}{0.516222in}}%
\pgfusepath{}%
\end{pgfscope}%
\end{pgfpicture}%
\makeatother%
\endgroup%

  \caption{Correlation between the wing length (in centimeters) and the falling
  times (in seconds) of the two realizations.}
  \label{fig_wl_times}
\end{figure}

\begin{figure}
  \centering
  %% Creator: Matplotlib, PGF backend
%%
%% To include the figure in your LaTeX document, write
%%   \input{<filename>.pgf}
%%
%% Make sure the required packages are loaded in your preamble
%%   \usepackage{pgf}
%%
%% Figures using additional raster images can only be included by \input if
%% they are in the same directory as the main LaTeX file. For loading figures
%% from other directories you can use the `import` package
%%   \usepackage{import}
%% and then include the figures with
%%   \import{<path to file>}{<filename>.pgf}
%%
%% Matplotlib used the following preamble
%%   \usepackage[utf8x]{inputenc}
%%   \usepackage[T1]{fontenc}
%%   \usepackage{cmbright}
%%
\begingroup%
\makeatletter%
\begin{pgfpicture}%
\pgfpathrectangle{\pgfpointorigin}{\pgfqpoint{5.000000in}{2.500000in}}%
\pgfusepath{use as bounding box, clip}%
\begin{pgfscope}%
\pgfsetbuttcap%
\pgfsetmiterjoin%
\definecolor{currentfill}{rgb}{1.000000,1.000000,1.000000}%
\pgfsetfillcolor{currentfill}%
\pgfsetlinewidth{0.000000pt}%
\definecolor{currentstroke}{rgb}{1.000000,1.000000,1.000000}%
\pgfsetstrokecolor{currentstroke}%
\pgfsetdash{}{0pt}%
\pgfpathmoveto{\pgfqpoint{0.000000in}{0.000000in}}%
\pgfpathlineto{\pgfqpoint{5.000000in}{0.000000in}}%
\pgfpathlineto{\pgfqpoint{5.000000in}{2.500000in}}%
\pgfpathlineto{\pgfqpoint{0.000000in}{2.500000in}}%
\pgfpathclose%
\pgfusepath{fill}%
\end{pgfscope}%
\begin{pgfscope}%
\pgfsetbuttcap%
\pgfsetmiterjoin%
\definecolor{currentfill}{rgb}{0.917647,0.917647,0.949020}%
\pgfsetfillcolor{currentfill}%
\pgfsetlinewidth{0.000000pt}%
\definecolor{currentstroke}{rgb}{0.000000,0.000000,0.000000}%
\pgfsetstrokecolor{currentstroke}%
\pgfsetstrokeopacity{0.000000}%
\pgfsetdash{}{0pt}%
\pgfpathmoveto{\pgfqpoint{0.556847in}{0.516222in}}%
\pgfpathlineto{\pgfqpoint{2.519580in}{0.516222in}}%
\pgfpathlineto{\pgfqpoint{2.519580in}{2.299750in}}%
\pgfpathlineto{\pgfqpoint{0.556847in}{2.299750in}}%
\pgfpathclose%
\pgfusepath{fill}%
\end{pgfscope}%
\begin{pgfscope}%
\pgfpathrectangle{\pgfqpoint{0.556847in}{0.516222in}}{\pgfqpoint{1.962733in}{1.783528in}} %
\pgfusepath{clip}%
\pgfsetroundcap%
\pgfsetroundjoin%
\pgfsetlinewidth{0.803000pt}%
\definecolor{currentstroke}{rgb}{1.000000,1.000000,1.000000}%
\pgfsetstrokecolor{currentstroke}%
\pgfsetdash{}{0pt}%
\pgfpathmoveto{\pgfqpoint{0.556847in}{0.516222in}}%
\pgfpathlineto{\pgfqpoint{0.556847in}{2.299750in}}%
\pgfusepath{stroke}%
\end{pgfscope}%
\begin{pgfscope}%
\pgfsetbuttcap%
\pgfsetroundjoin%
\definecolor{currentfill}{rgb}{0.150000,0.150000,0.150000}%
\pgfsetfillcolor{currentfill}%
\pgfsetlinewidth{0.803000pt}%
\definecolor{currentstroke}{rgb}{0.150000,0.150000,0.150000}%
\pgfsetstrokecolor{currentstroke}%
\pgfsetdash{}{0pt}%
\pgfsys@defobject{currentmarker}{\pgfqpoint{0.000000in}{0.000000in}}{\pgfqpoint{0.000000in}{0.000000in}}{%
\pgfpathmoveto{\pgfqpoint{0.000000in}{0.000000in}}%
\pgfpathlineto{\pgfqpoint{0.000000in}{0.000000in}}%
\pgfusepath{stroke,fill}%
}%
\begin{pgfscope}%
\pgfsys@transformshift{0.556847in}{0.516222in}%
\pgfsys@useobject{currentmarker}{}%
\end{pgfscope}%
\end{pgfscope}%
\begin{pgfscope}%
\definecolor{textcolor}{rgb}{0.150000,0.150000,0.150000}%
\pgfsetstrokecolor{textcolor}%
\pgfsetfillcolor{textcolor}%
\pgftext[x=0.556847in,y=0.438444in,,top]{\color{textcolor}\sffamily\fontsize{8.000000}{9.600000}\selectfont 1.5}%
\end{pgfscope}%
\begin{pgfscope}%
\pgfpathrectangle{\pgfqpoint{0.556847in}{0.516222in}}{\pgfqpoint{1.962733in}{1.783528in}} %
\pgfusepath{clip}%
\pgfsetroundcap%
\pgfsetroundjoin%
\pgfsetlinewidth{0.803000pt}%
\definecolor{currentstroke}{rgb}{1.000000,1.000000,1.000000}%
\pgfsetstrokecolor{currentstroke}%
\pgfsetdash{}{0pt}%
\pgfpathmoveto{\pgfqpoint{0.837238in}{0.516222in}}%
\pgfpathlineto{\pgfqpoint{0.837238in}{2.299750in}}%
\pgfusepath{stroke}%
\end{pgfscope}%
\begin{pgfscope}%
\pgfsetbuttcap%
\pgfsetroundjoin%
\definecolor{currentfill}{rgb}{0.150000,0.150000,0.150000}%
\pgfsetfillcolor{currentfill}%
\pgfsetlinewidth{0.803000pt}%
\definecolor{currentstroke}{rgb}{0.150000,0.150000,0.150000}%
\pgfsetstrokecolor{currentstroke}%
\pgfsetdash{}{0pt}%
\pgfsys@defobject{currentmarker}{\pgfqpoint{0.000000in}{0.000000in}}{\pgfqpoint{0.000000in}{0.000000in}}{%
\pgfpathmoveto{\pgfqpoint{0.000000in}{0.000000in}}%
\pgfpathlineto{\pgfqpoint{0.000000in}{0.000000in}}%
\pgfusepath{stroke,fill}%
}%
\begin{pgfscope}%
\pgfsys@transformshift{0.837238in}{0.516222in}%
\pgfsys@useobject{currentmarker}{}%
\end{pgfscope}%
\end{pgfscope}%
\begin{pgfscope}%
\definecolor{textcolor}{rgb}{0.150000,0.150000,0.150000}%
\pgfsetstrokecolor{textcolor}%
\pgfsetfillcolor{textcolor}%
\pgftext[x=0.837238in,y=0.438444in,,top]{\color{textcolor}\sffamily\fontsize{8.000000}{9.600000}\selectfont 2.0}%
\end{pgfscope}%
\begin{pgfscope}%
\pgfpathrectangle{\pgfqpoint{0.556847in}{0.516222in}}{\pgfqpoint{1.962733in}{1.783528in}} %
\pgfusepath{clip}%
\pgfsetroundcap%
\pgfsetroundjoin%
\pgfsetlinewidth{0.803000pt}%
\definecolor{currentstroke}{rgb}{1.000000,1.000000,1.000000}%
\pgfsetstrokecolor{currentstroke}%
\pgfsetdash{}{0pt}%
\pgfpathmoveto{\pgfqpoint{1.117628in}{0.516222in}}%
\pgfpathlineto{\pgfqpoint{1.117628in}{2.299750in}}%
\pgfusepath{stroke}%
\end{pgfscope}%
\begin{pgfscope}%
\pgfsetbuttcap%
\pgfsetroundjoin%
\definecolor{currentfill}{rgb}{0.150000,0.150000,0.150000}%
\pgfsetfillcolor{currentfill}%
\pgfsetlinewidth{0.803000pt}%
\definecolor{currentstroke}{rgb}{0.150000,0.150000,0.150000}%
\pgfsetstrokecolor{currentstroke}%
\pgfsetdash{}{0pt}%
\pgfsys@defobject{currentmarker}{\pgfqpoint{0.000000in}{0.000000in}}{\pgfqpoint{0.000000in}{0.000000in}}{%
\pgfpathmoveto{\pgfqpoint{0.000000in}{0.000000in}}%
\pgfpathlineto{\pgfqpoint{0.000000in}{0.000000in}}%
\pgfusepath{stroke,fill}%
}%
\begin{pgfscope}%
\pgfsys@transformshift{1.117628in}{0.516222in}%
\pgfsys@useobject{currentmarker}{}%
\end{pgfscope}%
\end{pgfscope}%
\begin{pgfscope}%
\definecolor{textcolor}{rgb}{0.150000,0.150000,0.150000}%
\pgfsetstrokecolor{textcolor}%
\pgfsetfillcolor{textcolor}%
\pgftext[x=1.117628in,y=0.438444in,,top]{\color{textcolor}\sffamily\fontsize{8.000000}{9.600000}\selectfont 2.5}%
\end{pgfscope}%
\begin{pgfscope}%
\pgfpathrectangle{\pgfqpoint{0.556847in}{0.516222in}}{\pgfqpoint{1.962733in}{1.783528in}} %
\pgfusepath{clip}%
\pgfsetroundcap%
\pgfsetroundjoin%
\pgfsetlinewidth{0.803000pt}%
\definecolor{currentstroke}{rgb}{1.000000,1.000000,1.000000}%
\pgfsetstrokecolor{currentstroke}%
\pgfsetdash{}{0pt}%
\pgfpathmoveto{\pgfqpoint{1.398018in}{0.516222in}}%
\pgfpathlineto{\pgfqpoint{1.398018in}{2.299750in}}%
\pgfusepath{stroke}%
\end{pgfscope}%
\begin{pgfscope}%
\pgfsetbuttcap%
\pgfsetroundjoin%
\definecolor{currentfill}{rgb}{0.150000,0.150000,0.150000}%
\pgfsetfillcolor{currentfill}%
\pgfsetlinewidth{0.803000pt}%
\definecolor{currentstroke}{rgb}{0.150000,0.150000,0.150000}%
\pgfsetstrokecolor{currentstroke}%
\pgfsetdash{}{0pt}%
\pgfsys@defobject{currentmarker}{\pgfqpoint{0.000000in}{0.000000in}}{\pgfqpoint{0.000000in}{0.000000in}}{%
\pgfpathmoveto{\pgfqpoint{0.000000in}{0.000000in}}%
\pgfpathlineto{\pgfqpoint{0.000000in}{0.000000in}}%
\pgfusepath{stroke,fill}%
}%
\begin{pgfscope}%
\pgfsys@transformshift{1.398018in}{0.516222in}%
\pgfsys@useobject{currentmarker}{}%
\end{pgfscope}%
\end{pgfscope}%
\begin{pgfscope}%
\definecolor{textcolor}{rgb}{0.150000,0.150000,0.150000}%
\pgfsetstrokecolor{textcolor}%
\pgfsetfillcolor{textcolor}%
\pgftext[x=1.398018in,y=0.438444in,,top]{\color{textcolor}\sffamily\fontsize{8.000000}{9.600000}\selectfont 3.0}%
\end{pgfscope}%
\begin{pgfscope}%
\pgfpathrectangle{\pgfqpoint{0.556847in}{0.516222in}}{\pgfqpoint{1.962733in}{1.783528in}} %
\pgfusepath{clip}%
\pgfsetroundcap%
\pgfsetroundjoin%
\pgfsetlinewidth{0.803000pt}%
\definecolor{currentstroke}{rgb}{1.000000,1.000000,1.000000}%
\pgfsetstrokecolor{currentstroke}%
\pgfsetdash{}{0pt}%
\pgfpathmoveto{\pgfqpoint{1.678409in}{0.516222in}}%
\pgfpathlineto{\pgfqpoint{1.678409in}{2.299750in}}%
\pgfusepath{stroke}%
\end{pgfscope}%
\begin{pgfscope}%
\pgfsetbuttcap%
\pgfsetroundjoin%
\definecolor{currentfill}{rgb}{0.150000,0.150000,0.150000}%
\pgfsetfillcolor{currentfill}%
\pgfsetlinewidth{0.803000pt}%
\definecolor{currentstroke}{rgb}{0.150000,0.150000,0.150000}%
\pgfsetstrokecolor{currentstroke}%
\pgfsetdash{}{0pt}%
\pgfsys@defobject{currentmarker}{\pgfqpoint{0.000000in}{0.000000in}}{\pgfqpoint{0.000000in}{0.000000in}}{%
\pgfpathmoveto{\pgfqpoint{0.000000in}{0.000000in}}%
\pgfpathlineto{\pgfqpoint{0.000000in}{0.000000in}}%
\pgfusepath{stroke,fill}%
}%
\begin{pgfscope}%
\pgfsys@transformshift{1.678409in}{0.516222in}%
\pgfsys@useobject{currentmarker}{}%
\end{pgfscope}%
\end{pgfscope}%
\begin{pgfscope}%
\definecolor{textcolor}{rgb}{0.150000,0.150000,0.150000}%
\pgfsetstrokecolor{textcolor}%
\pgfsetfillcolor{textcolor}%
\pgftext[x=1.678409in,y=0.438444in,,top]{\color{textcolor}\sffamily\fontsize{8.000000}{9.600000}\selectfont 3.5}%
\end{pgfscope}%
\begin{pgfscope}%
\pgfpathrectangle{\pgfqpoint{0.556847in}{0.516222in}}{\pgfqpoint{1.962733in}{1.783528in}} %
\pgfusepath{clip}%
\pgfsetroundcap%
\pgfsetroundjoin%
\pgfsetlinewidth{0.803000pt}%
\definecolor{currentstroke}{rgb}{1.000000,1.000000,1.000000}%
\pgfsetstrokecolor{currentstroke}%
\pgfsetdash{}{0pt}%
\pgfpathmoveto{\pgfqpoint{1.958799in}{0.516222in}}%
\pgfpathlineto{\pgfqpoint{1.958799in}{2.299750in}}%
\pgfusepath{stroke}%
\end{pgfscope}%
\begin{pgfscope}%
\pgfsetbuttcap%
\pgfsetroundjoin%
\definecolor{currentfill}{rgb}{0.150000,0.150000,0.150000}%
\pgfsetfillcolor{currentfill}%
\pgfsetlinewidth{0.803000pt}%
\definecolor{currentstroke}{rgb}{0.150000,0.150000,0.150000}%
\pgfsetstrokecolor{currentstroke}%
\pgfsetdash{}{0pt}%
\pgfsys@defobject{currentmarker}{\pgfqpoint{0.000000in}{0.000000in}}{\pgfqpoint{0.000000in}{0.000000in}}{%
\pgfpathmoveto{\pgfqpoint{0.000000in}{0.000000in}}%
\pgfpathlineto{\pgfqpoint{0.000000in}{0.000000in}}%
\pgfusepath{stroke,fill}%
}%
\begin{pgfscope}%
\pgfsys@transformshift{1.958799in}{0.516222in}%
\pgfsys@useobject{currentmarker}{}%
\end{pgfscope}%
\end{pgfscope}%
\begin{pgfscope}%
\definecolor{textcolor}{rgb}{0.150000,0.150000,0.150000}%
\pgfsetstrokecolor{textcolor}%
\pgfsetfillcolor{textcolor}%
\pgftext[x=1.958799in,y=0.438444in,,top]{\color{textcolor}\sffamily\fontsize{8.000000}{9.600000}\selectfont 4.0}%
\end{pgfscope}%
\begin{pgfscope}%
\pgfpathrectangle{\pgfqpoint{0.556847in}{0.516222in}}{\pgfqpoint{1.962733in}{1.783528in}} %
\pgfusepath{clip}%
\pgfsetroundcap%
\pgfsetroundjoin%
\pgfsetlinewidth{0.803000pt}%
\definecolor{currentstroke}{rgb}{1.000000,1.000000,1.000000}%
\pgfsetstrokecolor{currentstroke}%
\pgfsetdash{}{0pt}%
\pgfpathmoveto{\pgfqpoint{2.239189in}{0.516222in}}%
\pgfpathlineto{\pgfqpoint{2.239189in}{2.299750in}}%
\pgfusepath{stroke}%
\end{pgfscope}%
\begin{pgfscope}%
\pgfsetbuttcap%
\pgfsetroundjoin%
\definecolor{currentfill}{rgb}{0.150000,0.150000,0.150000}%
\pgfsetfillcolor{currentfill}%
\pgfsetlinewidth{0.803000pt}%
\definecolor{currentstroke}{rgb}{0.150000,0.150000,0.150000}%
\pgfsetstrokecolor{currentstroke}%
\pgfsetdash{}{0pt}%
\pgfsys@defobject{currentmarker}{\pgfqpoint{0.000000in}{0.000000in}}{\pgfqpoint{0.000000in}{0.000000in}}{%
\pgfpathmoveto{\pgfqpoint{0.000000in}{0.000000in}}%
\pgfpathlineto{\pgfqpoint{0.000000in}{0.000000in}}%
\pgfusepath{stroke,fill}%
}%
\begin{pgfscope}%
\pgfsys@transformshift{2.239189in}{0.516222in}%
\pgfsys@useobject{currentmarker}{}%
\end{pgfscope}%
\end{pgfscope}%
\begin{pgfscope}%
\definecolor{textcolor}{rgb}{0.150000,0.150000,0.150000}%
\pgfsetstrokecolor{textcolor}%
\pgfsetfillcolor{textcolor}%
\pgftext[x=2.239189in,y=0.438444in,,top]{\color{textcolor}\sffamily\fontsize{8.000000}{9.600000}\selectfont 4.5}%
\end{pgfscope}%
\begin{pgfscope}%
\pgfpathrectangle{\pgfqpoint{0.556847in}{0.516222in}}{\pgfqpoint{1.962733in}{1.783528in}} %
\pgfusepath{clip}%
\pgfsetroundcap%
\pgfsetroundjoin%
\pgfsetlinewidth{0.803000pt}%
\definecolor{currentstroke}{rgb}{1.000000,1.000000,1.000000}%
\pgfsetstrokecolor{currentstroke}%
\pgfsetdash{}{0pt}%
\pgfpathmoveto{\pgfqpoint{2.519580in}{0.516222in}}%
\pgfpathlineto{\pgfqpoint{2.519580in}{2.299750in}}%
\pgfusepath{stroke}%
\end{pgfscope}%
\begin{pgfscope}%
\pgfsetbuttcap%
\pgfsetroundjoin%
\definecolor{currentfill}{rgb}{0.150000,0.150000,0.150000}%
\pgfsetfillcolor{currentfill}%
\pgfsetlinewidth{0.803000pt}%
\definecolor{currentstroke}{rgb}{0.150000,0.150000,0.150000}%
\pgfsetstrokecolor{currentstroke}%
\pgfsetdash{}{0pt}%
\pgfsys@defobject{currentmarker}{\pgfqpoint{0.000000in}{0.000000in}}{\pgfqpoint{0.000000in}{0.000000in}}{%
\pgfpathmoveto{\pgfqpoint{0.000000in}{0.000000in}}%
\pgfpathlineto{\pgfqpoint{0.000000in}{0.000000in}}%
\pgfusepath{stroke,fill}%
}%
\begin{pgfscope}%
\pgfsys@transformshift{2.519580in}{0.516222in}%
\pgfsys@useobject{currentmarker}{}%
\end{pgfscope}%
\end{pgfscope}%
\begin{pgfscope}%
\definecolor{textcolor}{rgb}{0.150000,0.150000,0.150000}%
\pgfsetstrokecolor{textcolor}%
\pgfsetfillcolor{textcolor}%
\pgftext[x=2.519580in,y=0.438444in,,top]{\color{textcolor}\sffamily\fontsize{8.000000}{9.600000}\selectfont 5.0}%
\end{pgfscope}%
\begin{pgfscope}%
\definecolor{textcolor}{rgb}{0.150000,0.150000,0.150000}%
\pgfsetstrokecolor{textcolor}%
\pgfsetfillcolor{textcolor}%
\pgftext[x=1.538214in,y=0.273321in,,top]{\color{textcolor}\sffamily\fontsize{8.800000}{10.560000}\selectfont Falling time realization 1}%
\end{pgfscope}%
\begin{pgfscope}%
\pgfpathrectangle{\pgfqpoint{0.556847in}{0.516222in}}{\pgfqpoint{1.962733in}{1.783528in}} %
\pgfusepath{clip}%
\pgfsetroundcap%
\pgfsetroundjoin%
\pgfsetlinewidth{0.803000pt}%
\definecolor{currentstroke}{rgb}{1.000000,1.000000,1.000000}%
\pgfsetstrokecolor{currentstroke}%
\pgfsetdash{}{0pt}%
\pgfpathmoveto{\pgfqpoint{0.556847in}{0.516222in}}%
\pgfpathlineto{\pgfqpoint{2.519580in}{0.516222in}}%
\pgfusepath{stroke}%
\end{pgfscope}%
\begin{pgfscope}%
\pgfsetbuttcap%
\pgfsetroundjoin%
\definecolor{currentfill}{rgb}{0.150000,0.150000,0.150000}%
\pgfsetfillcolor{currentfill}%
\pgfsetlinewidth{0.803000pt}%
\definecolor{currentstroke}{rgb}{0.150000,0.150000,0.150000}%
\pgfsetstrokecolor{currentstroke}%
\pgfsetdash{}{0pt}%
\pgfsys@defobject{currentmarker}{\pgfqpoint{0.000000in}{0.000000in}}{\pgfqpoint{0.000000in}{0.000000in}}{%
\pgfpathmoveto{\pgfqpoint{0.000000in}{0.000000in}}%
\pgfpathlineto{\pgfqpoint{0.000000in}{0.000000in}}%
\pgfusepath{stroke,fill}%
}%
\begin{pgfscope}%
\pgfsys@transformshift{0.556847in}{0.516222in}%
\pgfsys@useobject{currentmarker}{}%
\end{pgfscope}%
\end{pgfscope}%
\begin{pgfscope}%
\definecolor{textcolor}{rgb}{0.150000,0.150000,0.150000}%
\pgfsetstrokecolor{textcolor}%
\pgfsetfillcolor{textcolor}%
\pgftext[x=0.479069in,y=0.516222in,right,]{\color{textcolor}\sffamily\fontsize{8.000000}{9.600000}\selectfont 7}%
\end{pgfscope}%
\begin{pgfscope}%
\pgfpathrectangle{\pgfqpoint{0.556847in}{0.516222in}}{\pgfqpoint{1.962733in}{1.783528in}} %
\pgfusepath{clip}%
\pgfsetroundcap%
\pgfsetroundjoin%
\pgfsetlinewidth{0.803000pt}%
\definecolor{currentstroke}{rgb}{1.000000,1.000000,1.000000}%
\pgfsetstrokecolor{currentstroke}%
\pgfsetdash{}{0pt}%
\pgfpathmoveto{\pgfqpoint{0.556847in}{0.813477in}}%
\pgfpathlineto{\pgfqpoint{2.519580in}{0.813477in}}%
\pgfusepath{stroke}%
\end{pgfscope}%
\begin{pgfscope}%
\pgfsetbuttcap%
\pgfsetroundjoin%
\definecolor{currentfill}{rgb}{0.150000,0.150000,0.150000}%
\pgfsetfillcolor{currentfill}%
\pgfsetlinewidth{0.803000pt}%
\definecolor{currentstroke}{rgb}{0.150000,0.150000,0.150000}%
\pgfsetstrokecolor{currentstroke}%
\pgfsetdash{}{0pt}%
\pgfsys@defobject{currentmarker}{\pgfqpoint{0.000000in}{0.000000in}}{\pgfqpoint{0.000000in}{0.000000in}}{%
\pgfpathmoveto{\pgfqpoint{0.000000in}{0.000000in}}%
\pgfpathlineto{\pgfqpoint{0.000000in}{0.000000in}}%
\pgfusepath{stroke,fill}%
}%
\begin{pgfscope}%
\pgfsys@transformshift{0.556847in}{0.813477in}%
\pgfsys@useobject{currentmarker}{}%
\end{pgfscope}%
\end{pgfscope}%
\begin{pgfscope}%
\definecolor{textcolor}{rgb}{0.150000,0.150000,0.150000}%
\pgfsetstrokecolor{textcolor}%
\pgfsetfillcolor{textcolor}%
\pgftext[x=0.479069in,y=0.813477in,right,]{\color{textcolor}\sffamily\fontsize{8.000000}{9.600000}\selectfont 8}%
\end{pgfscope}%
\begin{pgfscope}%
\pgfpathrectangle{\pgfqpoint{0.556847in}{0.516222in}}{\pgfqpoint{1.962733in}{1.783528in}} %
\pgfusepath{clip}%
\pgfsetroundcap%
\pgfsetroundjoin%
\pgfsetlinewidth{0.803000pt}%
\definecolor{currentstroke}{rgb}{1.000000,1.000000,1.000000}%
\pgfsetstrokecolor{currentstroke}%
\pgfsetdash{}{0pt}%
\pgfpathmoveto{\pgfqpoint{0.556847in}{1.110731in}}%
\pgfpathlineto{\pgfqpoint{2.519580in}{1.110731in}}%
\pgfusepath{stroke}%
\end{pgfscope}%
\begin{pgfscope}%
\pgfsetbuttcap%
\pgfsetroundjoin%
\definecolor{currentfill}{rgb}{0.150000,0.150000,0.150000}%
\pgfsetfillcolor{currentfill}%
\pgfsetlinewidth{0.803000pt}%
\definecolor{currentstroke}{rgb}{0.150000,0.150000,0.150000}%
\pgfsetstrokecolor{currentstroke}%
\pgfsetdash{}{0pt}%
\pgfsys@defobject{currentmarker}{\pgfqpoint{0.000000in}{0.000000in}}{\pgfqpoint{0.000000in}{0.000000in}}{%
\pgfpathmoveto{\pgfqpoint{0.000000in}{0.000000in}}%
\pgfpathlineto{\pgfqpoint{0.000000in}{0.000000in}}%
\pgfusepath{stroke,fill}%
}%
\begin{pgfscope}%
\pgfsys@transformshift{0.556847in}{1.110731in}%
\pgfsys@useobject{currentmarker}{}%
\end{pgfscope}%
\end{pgfscope}%
\begin{pgfscope}%
\definecolor{textcolor}{rgb}{0.150000,0.150000,0.150000}%
\pgfsetstrokecolor{textcolor}%
\pgfsetfillcolor{textcolor}%
\pgftext[x=0.479069in,y=1.110731in,right,]{\color{textcolor}\sffamily\fontsize{8.000000}{9.600000}\selectfont 9}%
\end{pgfscope}%
\begin{pgfscope}%
\pgfpathrectangle{\pgfqpoint{0.556847in}{0.516222in}}{\pgfqpoint{1.962733in}{1.783528in}} %
\pgfusepath{clip}%
\pgfsetroundcap%
\pgfsetroundjoin%
\pgfsetlinewidth{0.803000pt}%
\definecolor{currentstroke}{rgb}{1.000000,1.000000,1.000000}%
\pgfsetstrokecolor{currentstroke}%
\pgfsetdash{}{0pt}%
\pgfpathmoveto{\pgfqpoint{0.556847in}{1.407986in}}%
\pgfpathlineto{\pgfqpoint{2.519580in}{1.407986in}}%
\pgfusepath{stroke}%
\end{pgfscope}%
\begin{pgfscope}%
\pgfsetbuttcap%
\pgfsetroundjoin%
\definecolor{currentfill}{rgb}{0.150000,0.150000,0.150000}%
\pgfsetfillcolor{currentfill}%
\pgfsetlinewidth{0.803000pt}%
\definecolor{currentstroke}{rgb}{0.150000,0.150000,0.150000}%
\pgfsetstrokecolor{currentstroke}%
\pgfsetdash{}{0pt}%
\pgfsys@defobject{currentmarker}{\pgfqpoint{0.000000in}{0.000000in}}{\pgfqpoint{0.000000in}{0.000000in}}{%
\pgfpathmoveto{\pgfqpoint{0.000000in}{0.000000in}}%
\pgfpathlineto{\pgfqpoint{0.000000in}{0.000000in}}%
\pgfusepath{stroke,fill}%
}%
\begin{pgfscope}%
\pgfsys@transformshift{0.556847in}{1.407986in}%
\pgfsys@useobject{currentmarker}{}%
\end{pgfscope}%
\end{pgfscope}%
\begin{pgfscope}%
\definecolor{textcolor}{rgb}{0.150000,0.150000,0.150000}%
\pgfsetstrokecolor{textcolor}%
\pgfsetfillcolor{textcolor}%
\pgftext[x=0.479069in,y=1.407986in,right,]{\color{textcolor}\sffamily\fontsize{8.000000}{9.600000}\selectfont 10}%
\end{pgfscope}%
\begin{pgfscope}%
\pgfpathrectangle{\pgfqpoint{0.556847in}{0.516222in}}{\pgfqpoint{1.962733in}{1.783528in}} %
\pgfusepath{clip}%
\pgfsetroundcap%
\pgfsetroundjoin%
\pgfsetlinewidth{0.803000pt}%
\definecolor{currentstroke}{rgb}{1.000000,1.000000,1.000000}%
\pgfsetstrokecolor{currentstroke}%
\pgfsetdash{}{0pt}%
\pgfpathmoveto{\pgfqpoint{0.556847in}{1.705241in}}%
\pgfpathlineto{\pgfqpoint{2.519580in}{1.705241in}}%
\pgfusepath{stroke}%
\end{pgfscope}%
\begin{pgfscope}%
\pgfsetbuttcap%
\pgfsetroundjoin%
\definecolor{currentfill}{rgb}{0.150000,0.150000,0.150000}%
\pgfsetfillcolor{currentfill}%
\pgfsetlinewidth{0.803000pt}%
\definecolor{currentstroke}{rgb}{0.150000,0.150000,0.150000}%
\pgfsetstrokecolor{currentstroke}%
\pgfsetdash{}{0pt}%
\pgfsys@defobject{currentmarker}{\pgfqpoint{0.000000in}{0.000000in}}{\pgfqpoint{0.000000in}{0.000000in}}{%
\pgfpathmoveto{\pgfqpoint{0.000000in}{0.000000in}}%
\pgfpathlineto{\pgfqpoint{0.000000in}{0.000000in}}%
\pgfusepath{stroke,fill}%
}%
\begin{pgfscope}%
\pgfsys@transformshift{0.556847in}{1.705241in}%
\pgfsys@useobject{currentmarker}{}%
\end{pgfscope}%
\end{pgfscope}%
\begin{pgfscope}%
\definecolor{textcolor}{rgb}{0.150000,0.150000,0.150000}%
\pgfsetstrokecolor{textcolor}%
\pgfsetfillcolor{textcolor}%
\pgftext[x=0.479069in,y=1.705241in,right,]{\color{textcolor}\sffamily\fontsize{8.000000}{9.600000}\selectfont 11}%
\end{pgfscope}%
\begin{pgfscope}%
\pgfpathrectangle{\pgfqpoint{0.556847in}{0.516222in}}{\pgfqpoint{1.962733in}{1.783528in}} %
\pgfusepath{clip}%
\pgfsetroundcap%
\pgfsetroundjoin%
\pgfsetlinewidth{0.803000pt}%
\definecolor{currentstroke}{rgb}{1.000000,1.000000,1.000000}%
\pgfsetstrokecolor{currentstroke}%
\pgfsetdash{}{0pt}%
\pgfpathmoveto{\pgfqpoint{0.556847in}{2.002495in}}%
\pgfpathlineto{\pgfqpoint{2.519580in}{2.002495in}}%
\pgfusepath{stroke}%
\end{pgfscope}%
\begin{pgfscope}%
\pgfsetbuttcap%
\pgfsetroundjoin%
\definecolor{currentfill}{rgb}{0.150000,0.150000,0.150000}%
\pgfsetfillcolor{currentfill}%
\pgfsetlinewidth{0.803000pt}%
\definecolor{currentstroke}{rgb}{0.150000,0.150000,0.150000}%
\pgfsetstrokecolor{currentstroke}%
\pgfsetdash{}{0pt}%
\pgfsys@defobject{currentmarker}{\pgfqpoint{0.000000in}{0.000000in}}{\pgfqpoint{0.000000in}{0.000000in}}{%
\pgfpathmoveto{\pgfqpoint{0.000000in}{0.000000in}}%
\pgfpathlineto{\pgfqpoint{0.000000in}{0.000000in}}%
\pgfusepath{stroke,fill}%
}%
\begin{pgfscope}%
\pgfsys@transformshift{0.556847in}{2.002495in}%
\pgfsys@useobject{currentmarker}{}%
\end{pgfscope}%
\end{pgfscope}%
\begin{pgfscope}%
\definecolor{textcolor}{rgb}{0.150000,0.150000,0.150000}%
\pgfsetstrokecolor{textcolor}%
\pgfsetfillcolor{textcolor}%
\pgftext[x=0.479069in,y=2.002495in,right,]{\color{textcolor}\sffamily\fontsize{8.000000}{9.600000}\selectfont 12}%
\end{pgfscope}%
\begin{pgfscope}%
\pgfpathrectangle{\pgfqpoint{0.556847in}{0.516222in}}{\pgfqpoint{1.962733in}{1.783528in}} %
\pgfusepath{clip}%
\pgfsetroundcap%
\pgfsetroundjoin%
\pgfsetlinewidth{0.803000pt}%
\definecolor{currentstroke}{rgb}{1.000000,1.000000,1.000000}%
\pgfsetstrokecolor{currentstroke}%
\pgfsetdash{}{0pt}%
\pgfpathmoveto{\pgfqpoint{0.556847in}{2.299750in}}%
\pgfpathlineto{\pgfqpoint{2.519580in}{2.299750in}}%
\pgfusepath{stroke}%
\end{pgfscope}%
\begin{pgfscope}%
\pgfsetbuttcap%
\pgfsetroundjoin%
\definecolor{currentfill}{rgb}{0.150000,0.150000,0.150000}%
\pgfsetfillcolor{currentfill}%
\pgfsetlinewidth{0.803000pt}%
\definecolor{currentstroke}{rgb}{0.150000,0.150000,0.150000}%
\pgfsetstrokecolor{currentstroke}%
\pgfsetdash{}{0pt}%
\pgfsys@defobject{currentmarker}{\pgfqpoint{0.000000in}{0.000000in}}{\pgfqpoint{0.000000in}{0.000000in}}{%
\pgfpathmoveto{\pgfqpoint{0.000000in}{0.000000in}}%
\pgfpathlineto{\pgfqpoint{0.000000in}{0.000000in}}%
\pgfusepath{stroke,fill}%
}%
\begin{pgfscope}%
\pgfsys@transformshift{0.556847in}{2.299750in}%
\pgfsys@useobject{currentmarker}{}%
\end{pgfscope}%
\end{pgfscope}%
\begin{pgfscope}%
\definecolor{textcolor}{rgb}{0.150000,0.150000,0.150000}%
\pgfsetstrokecolor{textcolor}%
\pgfsetfillcolor{textcolor}%
\pgftext[x=0.479069in,y=2.299750in,right,]{\color{textcolor}\sffamily\fontsize{8.000000}{9.600000}\selectfont 13}%
\end{pgfscope}%
\begin{pgfscope}%
\definecolor{textcolor}{rgb}{0.150000,0.150000,0.150000}%
\pgfsetstrokecolor{textcolor}%
\pgfsetfillcolor{textcolor}%
\pgftext[x=0.286014in,y=1.407986in,,bottom,rotate=90.000000]{\color{textcolor}\sffamily\fontsize{8.800000}{10.560000}\selectfont Arm length}%
\end{pgfscope}%
\begin{pgfscope}%
\pgfpathrectangle{\pgfqpoint{0.556847in}{0.516222in}}{\pgfqpoint{1.962733in}{1.783528in}} %
\pgfusepath{clip}%
\pgfsetbuttcap%
\pgfsetroundjoin%
\definecolor{currentfill}{rgb}{0.298039,0.447059,0.690196}%
\pgfsetfillcolor{currentfill}%
\pgfsetlinewidth{0.240900pt}%
\definecolor{currentstroke}{rgb}{1.000000,1.000000,1.000000}%
\pgfsetstrokecolor{currentstroke}%
\pgfsetdash{}{0pt}%
\pgfpathmoveto{\pgfqpoint{1.902721in}{1.287753in}}%
\pgfpathcurveto{\pgfqpoint{1.910957in}{1.287753in}}{\pgfqpoint{1.918857in}{1.291026in}}{\pgfqpoint{1.924681in}{1.296849in}}%
\pgfpathcurveto{\pgfqpoint{1.930505in}{1.302673in}}{\pgfqpoint{1.933778in}{1.310573in}}{\pgfqpoint{1.933778in}{1.318810in}}%
\pgfpathcurveto{\pgfqpoint{1.933778in}{1.327046in}}{\pgfqpoint{1.930505in}{1.334946in}}{\pgfqpoint{1.924681in}{1.340770in}}%
\pgfpathcurveto{\pgfqpoint{1.918857in}{1.346594in}}{\pgfqpoint{1.910957in}{1.349866in}}{\pgfqpoint{1.902721in}{1.349866in}}%
\pgfpathcurveto{\pgfqpoint{1.894485in}{1.349866in}}{\pgfqpoint{1.886585in}{1.346594in}}{\pgfqpoint{1.880761in}{1.340770in}}%
\pgfpathcurveto{\pgfqpoint{1.874937in}{1.334946in}}{\pgfqpoint{1.871665in}{1.327046in}}{\pgfqpoint{1.871665in}{1.318810in}}%
\pgfpathcurveto{\pgfqpoint{1.871665in}{1.310573in}}{\pgfqpoint{1.874937in}{1.302673in}}{\pgfqpoint{1.880761in}{1.296849in}}%
\pgfpathcurveto{\pgfqpoint{1.886585in}{1.291026in}}{\pgfqpoint{1.894485in}{1.287753in}}{\pgfqpoint{1.902721in}{1.287753in}}%
\pgfpathclose%
\pgfusepath{stroke,fill}%
\end{pgfscope}%
\begin{pgfscope}%
\pgfpathrectangle{\pgfqpoint{0.556847in}{0.516222in}}{\pgfqpoint{1.962733in}{1.783528in}} %
\pgfusepath{clip}%
\pgfsetbuttcap%
\pgfsetroundjoin%
\definecolor{currentfill}{rgb}{0.298039,0.447059,0.690196}%
\pgfsetfillcolor{currentfill}%
\pgfsetlinewidth{0.240900pt}%
\definecolor{currentstroke}{rgb}{1.000000,1.000000,1.000000}%
\pgfsetstrokecolor{currentstroke}%
\pgfsetdash{}{0pt}%
\pgfpathmoveto{\pgfqpoint{1.622331in}{1.763361in}}%
\pgfpathcurveto{\pgfqpoint{1.630567in}{1.763361in}}{\pgfqpoint{1.638467in}{1.766633in}}{\pgfqpoint{1.644291in}{1.772457in}}%
\pgfpathcurveto{\pgfqpoint{1.650115in}{1.778281in}}{\pgfqpoint{1.653387in}{1.786181in}}{\pgfqpoint{1.653387in}{1.794417in}}%
\pgfpathcurveto{\pgfqpoint{1.653387in}{1.802653in}}{\pgfqpoint{1.650115in}{1.810553in}}{\pgfqpoint{1.644291in}{1.816377in}}%
\pgfpathcurveto{\pgfqpoint{1.638467in}{1.822201in}}{\pgfqpoint{1.630567in}{1.825474in}}{\pgfqpoint{1.622331in}{1.825474in}}%
\pgfpathcurveto{\pgfqpoint{1.614094in}{1.825474in}}{\pgfqpoint{1.606194in}{1.822201in}}{\pgfqpoint{1.600370in}{1.816377in}}%
\pgfpathcurveto{\pgfqpoint{1.594546in}{1.810553in}}{\pgfqpoint{1.591274in}{1.802653in}}{\pgfqpoint{1.591274in}{1.794417in}}%
\pgfpathcurveto{\pgfqpoint{1.591274in}{1.786181in}}{\pgfqpoint{1.594546in}{1.778281in}}{\pgfqpoint{1.600370in}{1.772457in}}%
\pgfpathcurveto{\pgfqpoint{1.606194in}{1.766633in}}{\pgfqpoint{1.614094in}{1.763361in}}{\pgfqpoint{1.622331in}{1.763361in}}%
\pgfpathclose%
\pgfusepath{stroke,fill}%
\end{pgfscope}%
\begin{pgfscope}%
\pgfpathrectangle{\pgfqpoint{0.556847in}{0.516222in}}{\pgfqpoint{1.962733in}{1.783528in}} %
\pgfusepath{clip}%
\pgfsetbuttcap%
\pgfsetroundjoin%
\definecolor{currentfill}{rgb}{0.298039,0.447059,0.690196}%
\pgfsetfillcolor{currentfill}%
\pgfsetlinewidth{0.240900pt}%
\definecolor{currentstroke}{rgb}{1.000000,1.000000,1.000000}%
\pgfsetstrokecolor{currentstroke}%
\pgfsetdash{}{0pt}%
\pgfpathmoveto{\pgfqpoint{1.958799in}{1.436381in}}%
\pgfpathcurveto{\pgfqpoint{1.967035in}{1.436381in}}{\pgfqpoint{1.974935in}{1.439653in}}{\pgfqpoint{1.980759in}{1.445477in}}%
\pgfpathcurveto{\pgfqpoint{1.986583in}{1.451301in}}{\pgfqpoint{1.989856in}{1.459201in}}{\pgfqpoint{1.989856in}{1.467437in}}%
\pgfpathcurveto{\pgfqpoint{1.989856in}{1.475673in}}{\pgfqpoint{1.986583in}{1.483573in}}{\pgfqpoint{1.980759in}{1.489397in}}%
\pgfpathcurveto{\pgfqpoint{1.974935in}{1.495221in}}{\pgfqpoint{1.967035in}{1.498494in}}{\pgfqpoint{1.958799in}{1.498494in}}%
\pgfpathcurveto{\pgfqpoint{1.950563in}{1.498494in}}{\pgfqpoint{1.942663in}{1.495221in}}{\pgfqpoint{1.936839in}{1.489397in}}%
\pgfpathcurveto{\pgfqpoint{1.931015in}{1.483573in}}{\pgfqpoint{1.927743in}{1.475673in}}{\pgfqpoint{1.927743in}{1.467437in}}%
\pgfpathcurveto{\pgfqpoint{1.927743in}{1.459201in}}{\pgfqpoint{1.931015in}{1.451301in}}{\pgfqpoint{1.936839in}{1.445477in}}%
\pgfpathcurveto{\pgfqpoint{1.942663in}{1.439653in}}{\pgfqpoint{1.950563in}{1.436381in}}{\pgfqpoint{1.958799in}{1.436381in}}%
\pgfpathclose%
\pgfusepath{stroke,fill}%
\end{pgfscope}%
\begin{pgfscope}%
\pgfpathrectangle{\pgfqpoint{0.556847in}{0.516222in}}{\pgfqpoint{1.962733in}{1.783528in}} %
\pgfusepath{clip}%
\pgfsetbuttcap%
\pgfsetroundjoin%
\definecolor{currentfill}{rgb}{0.298039,0.447059,0.690196}%
\pgfsetfillcolor{currentfill}%
\pgfsetlinewidth{0.240900pt}%
\definecolor{currentstroke}{rgb}{1.000000,1.000000,1.000000}%
\pgfsetstrokecolor{currentstroke}%
\pgfsetdash{}{0pt}%
\pgfpathmoveto{\pgfqpoint{2.239189in}{1.822812in}}%
\pgfpathcurveto{\pgfqpoint{2.247426in}{1.822812in}}{\pgfqpoint{2.255326in}{1.826084in}}{\pgfqpoint{2.261150in}{1.831908in}}%
\pgfpathcurveto{\pgfqpoint{2.266974in}{1.837732in}}{\pgfqpoint{2.270246in}{1.845632in}}{\pgfqpoint{2.270246in}{1.853868in}}%
\pgfpathcurveto{\pgfqpoint{2.270246in}{1.862104in}}{\pgfqpoint{2.266974in}{1.870004in}}{\pgfqpoint{2.261150in}{1.875828in}}%
\pgfpathcurveto{\pgfqpoint{2.255326in}{1.881652in}}{\pgfqpoint{2.247426in}{1.884925in}}{\pgfqpoint{2.239189in}{1.884925in}}%
\pgfpathcurveto{\pgfqpoint{2.230953in}{1.884925in}}{\pgfqpoint{2.223053in}{1.881652in}}{\pgfqpoint{2.217229in}{1.875828in}}%
\pgfpathcurveto{\pgfqpoint{2.211405in}{1.870004in}}{\pgfqpoint{2.208133in}{1.862104in}}{\pgfqpoint{2.208133in}{1.853868in}}%
\pgfpathcurveto{\pgfqpoint{2.208133in}{1.845632in}}{\pgfqpoint{2.211405in}{1.837732in}}{\pgfqpoint{2.217229in}{1.831908in}}%
\pgfpathcurveto{\pgfqpoint{2.223053in}{1.826084in}}{\pgfqpoint{2.230953in}{1.822812in}}{\pgfqpoint{2.239189in}{1.822812in}}%
\pgfpathclose%
\pgfusepath{stroke,fill}%
\end{pgfscope}%
\begin{pgfscope}%
\pgfpathrectangle{\pgfqpoint{0.556847in}{0.516222in}}{\pgfqpoint{1.962733in}{1.783528in}} %
\pgfusepath{clip}%
\pgfsetbuttcap%
\pgfsetroundjoin%
\definecolor{currentfill}{rgb}{0.298039,0.447059,0.690196}%
\pgfsetfillcolor{currentfill}%
\pgfsetlinewidth{0.240900pt}%
\definecolor{currentstroke}{rgb}{1.000000,1.000000,1.000000}%
\pgfsetstrokecolor{currentstroke}%
\pgfsetdash{}{0pt}%
\pgfpathmoveto{\pgfqpoint{1.622331in}{1.733635in}}%
\pgfpathcurveto{\pgfqpoint{1.630567in}{1.733635in}}{\pgfqpoint{1.638467in}{1.736907in}}{\pgfqpoint{1.644291in}{1.742731in}}%
\pgfpathcurveto{\pgfqpoint{1.650115in}{1.748555in}}{\pgfqpoint{1.653387in}{1.756455in}}{\pgfqpoint{1.653387in}{1.764692in}}%
\pgfpathcurveto{\pgfqpoint{1.653387in}{1.772928in}}{\pgfqpoint{1.650115in}{1.780828in}}{\pgfqpoint{1.644291in}{1.786652in}}%
\pgfpathcurveto{\pgfqpoint{1.638467in}{1.792476in}}{\pgfqpoint{1.630567in}{1.795748in}}{\pgfqpoint{1.622331in}{1.795748in}}%
\pgfpathcurveto{\pgfqpoint{1.614094in}{1.795748in}}{\pgfqpoint{1.606194in}{1.792476in}}{\pgfqpoint{1.600370in}{1.786652in}}%
\pgfpathcurveto{\pgfqpoint{1.594546in}{1.780828in}}{\pgfqpoint{1.591274in}{1.772928in}}{\pgfqpoint{1.591274in}{1.764692in}}%
\pgfpathcurveto{\pgfqpoint{1.591274in}{1.756455in}}{\pgfqpoint{1.594546in}{1.748555in}}{\pgfqpoint{1.600370in}{1.742731in}}%
\pgfpathcurveto{\pgfqpoint{1.606194in}{1.736907in}}{\pgfqpoint{1.614094in}{1.733635in}}{\pgfqpoint{1.622331in}{1.733635in}}%
\pgfpathclose%
\pgfusepath{stroke,fill}%
\end{pgfscope}%
\begin{pgfscope}%
\pgfpathrectangle{\pgfqpoint{0.556847in}{0.516222in}}{\pgfqpoint{1.962733in}{1.783528in}} %
\pgfusepath{clip}%
\pgfsetbuttcap%
\pgfsetroundjoin%
\definecolor{currentfill}{rgb}{0.298039,0.447059,0.690196}%
\pgfsetfillcolor{currentfill}%
\pgfsetlinewidth{0.240900pt}%
\definecolor{currentstroke}{rgb}{1.000000,1.000000,1.000000}%
\pgfsetstrokecolor{currentstroke}%
\pgfsetdash{}{0pt}%
\pgfpathmoveto{\pgfqpoint{2.407424in}{1.911988in}}%
\pgfpathcurveto{\pgfqpoint{2.415660in}{1.911988in}}{\pgfqpoint{2.423560in}{1.915260in}}{\pgfqpoint{2.429384in}{1.921084in}}%
\pgfpathcurveto{\pgfqpoint{2.435208in}{1.926908in}}{\pgfqpoint{2.438480in}{1.934808in}}{\pgfqpoint{2.438480in}{1.943044in}}%
\pgfpathcurveto{\pgfqpoint{2.438480in}{1.951281in}}{\pgfqpoint{2.435208in}{1.959181in}}{\pgfqpoint{2.429384in}{1.965005in}}%
\pgfpathcurveto{\pgfqpoint{2.423560in}{1.970829in}}{\pgfqpoint{2.415660in}{1.974101in}}{\pgfqpoint{2.407424in}{1.974101in}}%
\pgfpathcurveto{\pgfqpoint{2.399187in}{1.974101in}}{\pgfqpoint{2.391287in}{1.970829in}}{\pgfqpoint{2.385463in}{1.965005in}}%
\pgfpathcurveto{\pgfqpoint{2.379640in}{1.959181in}}{\pgfqpoint{2.376367in}{1.951281in}}{\pgfqpoint{2.376367in}{1.943044in}}%
\pgfpathcurveto{\pgfqpoint{2.376367in}{1.934808in}}{\pgfqpoint{2.379640in}{1.926908in}}{\pgfqpoint{2.385463in}{1.921084in}}%
\pgfpathcurveto{\pgfqpoint{2.391287in}{1.915260in}}{\pgfqpoint{2.399187in}{1.911988in}}{\pgfqpoint{2.407424in}{1.911988in}}%
\pgfpathclose%
\pgfusepath{stroke,fill}%
\end{pgfscope}%
\begin{pgfscope}%
\pgfpathrectangle{\pgfqpoint{0.556847in}{0.516222in}}{\pgfqpoint{1.962733in}{1.783528in}} %
\pgfusepath{clip}%
\pgfsetbuttcap%
\pgfsetroundjoin%
\definecolor{currentfill}{rgb}{0.298039,0.447059,0.690196}%
\pgfsetfillcolor{currentfill}%
\pgfsetlinewidth{0.240900pt}%
\definecolor{currentstroke}{rgb}{1.000000,1.000000,1.000000}%
\pgfsetstrokecolor{currentstroke}%
\pgfsetdash{}{0pt}%
\pgfpathmoveto{\pgfqpoint{1.622331in}{1.168851in}}%
\pgfpathcurveto{\pgfqpoint{1.630567in}{1.168851in}}{\pgfqpoint{1.638467in}{1.172124in}}{\pgfqpoint{1.644291in}{1.177948in}}%
\pgfpathcurveto{\pgfqpoint{1.650115in}{1.183772in}}{\pgfqpoint{1.653387in}{1.191672in}}{\pgfqpoint{1.653387in}{1.199908in}}%
\pgfpathcurveto{\pgfqpoint{1.653387in}{1.208144in}}{\pgfqpoint{1.650115in}{1.216044in}}{\pgfqpoint{1.644291in}{1.221868in}}%
\pgfpathcurveto{\pgfqpoint{1.638467in}{1.227692in}}{\pgfqpoint{1.630567in}{1.230964in}}{\pgfqpoint{1.622331in}{1.230964in}}%
\pgfpathcurveto{\pgfqpoint{1.614094in}{1.230964in}}{\pgfqpoint{1.606194in}{1.227692in}}{\pgfqpoint{1.600370in}{1.221868in}}%
\pgfpathcurveto{\pgfqpoint{1.594546in}{1.216044in}}{\pgfqpoint{1.591274in}{1.208144in}}{\pgfqpoint{1.591274in}{1.199908in}}%
\pgfpathcurveto{\pgfqpoint{1.591274in}{1.191672in}}{\pgfqpoint{1.594546in}{1.183772in}}{\pgfqpoint{1.600370in}{1.177948in}}%
\pgfpathcurveto{\pgfqpoint{1.606194in}{1.172124in}}{\pgfqpoint{1.614094in}{1.168851in}}{\pgfqpoint{1.622331in}{1.168851in}}%
\pgfpathclose%
\pgfusepath{stroke,fill}%
\end{pgfscope}%
\begin{pgfscope}%
\pgfpathrectangle{\pgfqpoint{0.556847in}{0.516222in}}{\pgfqpoint{1.962733in}{1.783528in}} %
\pgfusepath{clip}%
\pgfsetbuttcap%
\pgfsetroundjoin%
\definecolor{currentfill}{rgb}{0.298039,0.447059,0.690196}%
\pgfsetfillcolor{currentfill}%
\pgfsetlinewidth{0.240900pt}%
\definecolor{currentstroke}{rgb}{1.000000,1.000000,1.000000}%
\pgfsetstrokecolor{currentstroke}%
\pgfsetdash{}{0pt}%
\pgfpathmoveto{\pgfqpoint{2.407424in}{1.674184in}}%
\pgfpathcurveto{\pgfqpoint{2.415660in}{1.674184in}}{\pgfqpoint{2.423560in}{1.677457in}}{\pgfqpoint{2.429384in}{1.683280in}}%
\pgfpathcurveto{\pgfqpoint{2.435208in}{1.689104in}}{\pgfqpoint{2.438480in}{1.697004in}}{\pgfqpoint{2.438480in}{1.705241in}}%
\pgfpathcurveto{\pgfqpoint{2.438480in}{1.713477in}}{\pgfqpoint{2.435208in}{1.721377in}}{\pgfqpoint{2.429384in}{1.727201in}}%
\pgfpathcurveto{\pgfqpoint{2.423560in}{1.733025in}}{\pgfqpoint{2.415660in}{1.736297in}}{\pgfqpoint{2.407424in}{1.736297in}}%
\pgfpathcurveto{\pgfqpoint{2.399187in}{1.736297in}}{\pgfqpoint{2.391287in}{1.733025in}}{\pgfqpoint{2.385463in}{1.727201in}}%
\pgfpathcurveto{\pgfqpoint{2.379640in}{1.721377in}}{\pgfqpoint{2.376367in}{1.713477in}}{\pgfqpoint{2.376367in}{1.705241in}}%
\pgfpathcurveto{\pgfqpoint{2.376367in}{1.697004in}}{\pgfqpoint{2.379640in}{1.689104in}}{\pgfqpoint{2.385463in}{1.683280in}}%
\pgfpathcurveto{\pgfqpoint{2.391287in}{1.677457in}}{\pgfqpoint{2.399187in}{1.674184in}}{\pgfqpoint{2.407424in}{1.674184in}}%
\pgfpathclose%
\pgfusepath{stroke,fill}%
\end{pgfscope}%
\begin{pgfscope}%
\pgfpathrectangle{\pgfqpoint{0.556847in}{0.516222in}}{\pgfqpoint{1.962733in}{1.783528in}} %
\pgfusepath{clip}%
\pgfsetbuttcap%
\pgfsetroundjoin%
\definecolor{currentfill}{rgb}{0.298039,0.447059,0.690196}%
\pgfsetfillcolor{currentfill}%
\pgfsetlinewidth{0.240900pt}%
\definecolor{currentstroke}{rgb}{1.000000,1.000000,1.000000}%
\pgfsetstrokecolor{currentstroke}%
\pgfsetdash{}{0pt}%
\pgfpathmoveto{\pgfqpoint{1.173706in}{1.585008in}}%
\pgfpathcurveto{\pgfqpoint{1.181942in}{1.585008in}}{\pgfqpoint{1.189842in}{1.588280in}}{\pgfqpoint{1.195666in}{1.594104in}}%
\pgfpathcurveto{\pgfqpoint{1.201490in}{1.599928in}}{\pgfqpoint{1.204763in}{1.607828in}}{\pgfqpoint{1.204763in}{1.616064in}}%
\pgfpathcurveto{\pgfqpoint{1.204763in}{1.624301in}}{\pgfqpoint{1.201490in}{1.632201in}}{\pgfqpoint{1.195666in}{1.638025in}}%
\pgfpathcurveto{\pgfqpoint{1.189842in}{1.643849in}}{\pgfqpoint{1.181942in}{1.647121in}}{\pgfqpoint{1.173706in}{1.647121in}}%
\pgfpathcurveto{\pgfqpoint{1.165470in}{1.647121in}}{\pgfqpoint{1.157570in}{1.643849in}}{\pgfqpoint{1.151746in}{1.638025in}}%
\pgfpathcurveto{\pgfqpoint{1.145922in}{1.632201in}}{\pgfqpoint{1.142650in}{1.624301in}}{\pgfqpoint{1.142650in}{1.616064in}}%
\pgfpathcurveto{\pgfqpoint{1.142650in}{1.607828in}}{\pgfqpoint{1.145922in}{1.599928in}}{\pgfqpoint{1.151746in}{1.594104in}}%
\pgfpathcurveto{\pgfqpoint{1.157570in}{1.588280in}}{\pgfqpoint{1.165470in}{1.585008in}}{\pgfqpoint{1.173706in}{1.585008in}}%
\pgfpathclose%
\pgfusepath{stroke,fill}%
\end{pgfscope}%
\begin{pgfscope}%
\pgfpathrectangle{\pgfqpoint{0.556847in}{0.516222in}}{\pgfqpoint{1.962733in}{1.783528in}} %
\pgfusepath{clip}%
\pgfsetbuttcap%
\pgfsetroundjoin%
\definecolor{currentfill}{rgb}{0.298039,0.447059,0.690196}%
\pgfsetfillcolor{currentfill}%
\pgfsetlinewidth{0.240900pt}%
\definecolor{currentstroke}{rgb}{1.000000,1.000000,1.000000}%
\pgfsetstrokecolor{currentstroke}%
\pgfsetdash{}{0pt}%
\pgfpathmoveto{\pgfqpoint{1.678409in}{1.436381in}}%
\pgfpathcurveto{\pgfqpoint{1.686645in}{1.436381in}}{\pgfqpoint{1.694545in}{1.439653in}}{\pgfqpoint{1.700369in}{1.445477in}}%
\pgfpathcurveto{\pgfqpoint{1.706193in}{1.451301in}}{\pgfqpoint{1.709465in}{1.459201in}}{\pgfqpoint{1.709465in}{1.467437in}}%
\pgfpathcurveto{\pgfqpoint{1.709465in}{1.475673in}}{\pgfqpoint{1.706193in}{1.483573in}}{\pgfqpoint{1.700369in}{1.489397in}}%
\pgfpathcurveto{\pgfqpoint{1.694545in}{1.495221in}}{\pgfqpoint{1.686645in}{1.498494in}}{\pgfqpoint{1.678409in}{1.498494in}}%
\pgfpathcurveto{\pgfqpoint{1.670172in}{1.498494in}}{\pgfqpoint{1.662272in}{1.495221in}}{\pgfqpoint{1.656448in}{1.489397in}}%
\pgfpathcurveto{\pgfqpoint{1.650625in}{1.483573in}}{\pgfqpoint{1.647352in}{1.475673in}}{\pgfqpoint{1.647352in}{1.467437in}}%
\pgfpathcurveto{\pgfqpoint{1.647352in}{1.459201in}}{\pgfqpoint{1.650625in}{1.451301in}}{\pgfqpoint{1.656448in}{1.445477in}}%
\pgfpathcurveto{\pgfqpoint{1.662272in}{1.439653in}}{\pgfqpoint{1.670172in}{1.436381in}}{\pgfqpoint{1.678409in}{1.436381in}}%
\pgfpathclose%
\pgfusepath{stroke,fill}%
\end{pgfscope}%
\begin{pgfscope}%
\pgfpathrectangle{\pgfqpoint{0.556847in}{0.516222in}}{\pgfqpoint{1.962733in}{1.783528in}} %
\pgfusepath{clip}%
\pgfsetbuttcap%
\pgfsetroundjoin%
\definecolor{currentfill}{rgb}{0.298039,0.447059,0.690196}%
\pgfsetfillcolor{currentfill}%
\pgfsetlinewidth{0.240900pt}%
\definecolor{currentstroke}{rgb}{1.000000,1.000000,1.000000}%
\pgfsetstrokecolor{currentstroke}%
\pgfsetdash{}{0pt}%
\pgfpathmoveto{\pgfqpoint{1.510175in}{0.633793in}}%
\pgfpathcurveto{\pgfqpoint{1.518411in}{0.633793in}}{\pgfqpoint{1.526311in}{0.637065in}}{\pgfqpoint{1.532135in}{0.642889in}}%
\pgfpathcurveto{\pgfqpoint{1.537959in}{0.648713in}}{\pgfqpoint{1.541231in}{0.656613in}}{\pgfqpoint{1.541231in}{0.664850in}}%
\pgfpathcurveto{\pgfqpoint{1.541231in}{0.673086in}}{\pgfqpoint{1.537959in}{0.680986in}}{\pgfqpoint{1.532135in}{0.686810in}}%
\pgfpathcurveto{\pgfqpoint{1.526311in}{0.692634in}}{\pgfqpoint{1.518411in}{0.695906in}}{\pgfqpoint{1.510175in}{0.695906in}}%
\pgfpathcurveto{\pgfqpoint{1.501938in}{0.695906in}}{\pgfqpoint{1.494038in}{0.692634in}}{\pgfqpoint{1.488214in}{0.686810in}}%
\pgfpathcurveto{\pgfqpoint{1.482390in}{0.680986in}}{\pgfqpoint{1.479118in}{0.673086in}}{\pgfqpoint{1.479118in}{0.664850in}}%
\pgfpathcurveto{\pgfqpoint{1.479118in}{0.656613in}}{\pgfqpoint{1.482390in}{0.648713in}}{\pgfqpoint{1.488214in}{0.642889in}}%
\pgfpathcurveto{\pgfqpoint{1.494038in}{0.637065in}}{\pgfqpoint{1.501938in}{0.633793in}}{\pgfqpoint{1.510175in}{0.633793in}}%
\pgfpathclose%
\pgfusepath{stroke,fill}%
\end{pgfscope}%
\begin{pgfscope}%
\pgfpathrectangle{\pgfqpoint{0.556847in}{0.516222in}}{\pgfqpoint{1.962733in}{1.783528in}} %
\pgfusepath{clip}%
\pgfsetbuttcap%
\pgfsetroundjoin%
\definecolor{currentfill}{rgb}{0.298039,0.447059,0.690196}%
\pgfsetfillcolor{currentfill}%
\pgfsetlinewidth{0.240900pt}%
\definecolor{currentstroke}{rgb}{1.000000,1.000000,1.000000}%
\pgfsetstrokecolor{currentstroke}%
\pgfsetdash{}{0pt}%
\pgfpathmoveto{\pgfqpoint{1.678409in}{1.793086in}}%
\pgfpathcurveto{\pgfqpoint{1.686645in}{1.793086in}}{\pgfqpoint{1.694545in}{1.796358in}}{\pgfqpoint{1.700369in}{1.802182in}}%
\pgfpathcurveto{\pgfqpoint{1.706193in}{1.808006in}}{\pgfqpoint{1.709465in}{1.815906in}}{\pgfqpoint{1.709465in}{1.824143in}}%
\pgfpathcurveto{\pgfqpoint{1.709465in}{1.832379in}}{\pgfqpoint{1.706193in}{1.840279in}}{\pgfqpoint{1.700369in}{1.846103in}}%
\pgfpathcurveto{\pgfqpoint{1.694545in}{1.851927in}}{\pgfqpoint{1.686645in}{1.855199in}}{\pgfqpoint{1.678409in}{1.855199in}}%
\pgfpathcurveto{\pgfqpoint{1.670172in}{1.855199in}}{\pgfqpoint{1.662272in}{1.851927in}}{\pgfqpoint{1.656448in}{1.846103in}}%
\pgfpathcurveto{\pgfqpoint{1.650625in}{1.840279in}}{\pgfqpoint{1.647352in}{1.832379in}}{\pgfqpoint{1.647352in}{1.824143in}}%
\pgfpathcurveto{\pgfqpoint{1.647352in}{1.815906in}}{\pgfqpoint{1.650625in}{1.808006in}}{\pgfqpoint{1.656448in}{1.802182in}}%
\pgfpathcurveto{\pgfqpoint{1.662272in}{1.796358in}}{\pgfqpoint{1.670172in}{1.793086in}}{\pgfqpoint{1.678409in}{1.793086in}}%
\pgfpathclose%
\pgfusepath{stroke,fill}%
\end{pgfscope}%
\begin{pgfscope}%
\pgfpathrectangle{\pgfqpoint{0.556847in}{0.516222in}}{\pgfqpoint{1.962733in}{1.783528in}} %
\pgfusepath{clip}%
\pgfsetbuttcap%
\pgfsetroundjoin%
\definecolor{currentfill}{rgb}{0.298039,0.447059,0.690196}%
\pgfsetfillcolor{currentfill}%
\pgfsetlinewidth{0.240900pt}%
\definecolor{currentstroke}{rgb}{1.000000,1.000000,1.000000}%
\pgfsetstrokecolor{currentstroke}%
\pgfsetdash{}{0pt}%
\pgfpathmoveto{\pgfqpoint{1.790565in}{1.376930in}}%
\pgfpathcurveto{\pgfqpoint{1.798801in}{1.376930in}}{\pgfqpoint{1.806701in}{1.380202in}}{\pgfqpoint{1.812525in}{1.386026in}}%
\pgfpathcurveto{\pgfqpoint{1.818349in}{1.391850in}}{\pgfqpoint{1.821621in}{1.399750in}}{\pgfqpoint{1.821621in}{1.407986in}}%
\pgfpathcurveto{\pgfqpoint{1.821621in}{1.416222in}}{\pgfqpoint{1.818349in}{1.424122in}}{\pgfqpoint{1.812525in}{1.429946in}}%
\pgfpathcurveto{\pgfqpoint{1.806701in}{1.435770in}}{\pgfqpoint{1.798801in}{1.439043in}}{\pgfqpoint{1.790565in}{1.439043in}}%
\pgfpathcurveto{\pgfqpoint{1.782329in}{1.439043in}}{\pgfqpoint{1.774429in}{1.435770in}}{\pgfqpoint{1.768605in}{1.429946in}}%
\pgfpathcurveto{\pgfqpoint{1.762781in}{1.424122in}}{\pgfqpoint{1.759508in}{1.416222in}}{\pgfqpoint{1.759508in}{1.407986in}}%
\pgfpathcurveto{\pgfqpoint{1.759508in}{1.399750in}}{\pgfqpoint{1.762781in}{1.391850in}}{\pgfqpoint{1.768605in}{1.386026in}}%
\pgfpathcurveto{\pgfqpoint{1.774429in}{1.380202in}}{\pgfqpoint{1.782329in}{1.376930in}}{\pgfqpoint{1.790565in}{1.376930in}}%
\pgfpathclose%
\pgfusepath{stroke,fill}%
\end{pgfscope}%
\begin{pgfscope}%
\pgfpathrectangle{\pgfqpoint{0.556847in}{0.516222in}}{\pgfqpoint{1.962733in}{1.783528in}} %
\pgfusepath{clip}%
\pgfsetbuttcap%
\pgfsetroundjoin%
\definecolor{currentfill}{rgb}{0.298039,0.447059,0.690196}%
\pgfsetfillcolor{currentfill}%
\pgfsetlinewidth{0.240900pt}%
\definecolor{currentstroke}{rgb}{1.000000,1.000000,1.000000}%
\pgfsetstrokecolor{currentstroke}%
\pgfsetdash{}{0pt}%
\pgfpathmoveto{\pgfqpoint{1.061550in}{0.693244in}}%
\pgfpathcurveto{\pgfqpoint{1.069786in}{0.693244in}}{\pgfqpoint{1.077686in}{0.696516in}}{\pgfqpoint{1.083510in}{0.702340in}}%
\pgfpathcurveto{\pgfqpoint{1.089334in}{0.708164in}}{\pgfqpoint{1.092606in}{0.716064in}}{\pgfqpoint{1.092606in}{0.724300in}}%
\pgfpathcurveto{\pgfqpoint{1.092606in}{0.732537in}}{\pgfqpoint{1.089334in}{0.740437in}}{\pgfqpoint{1.083510in}{0.746261in}}%
\pgfpathcurveto{\pgfqpoint{1.077686in}{0.752085in}}{\pgfqpoint{1.069786in}{0.755357in}}{\pgfqpoint{1.061550in}{0.755357in}}%
\pgfpathcurveto{\pgfqpoint{1.053314in}{0.755357in}}{\pgfqpoint{1.045414in}{0.752085in}}{\pgfqpoint{1.039590in}{0.746261in}}%
\pgfpathcurveto{\pgfqpoint{1.033766in}{0.740437in}}{\pgfqpoint{1.030493in}{0.732537in}}{\pgfqpoint{1.030493in}{0.724300in}}%
\pgfpathcurveto{\pgfqpoint{1.030493in}{0.716064in}}{\pgfqpoint{1.033766in}{0.708164in}}{\pgfqpoint{1.039590in}{0.702340in}}%
\pgfpathcurveto{\pgfqpoint{1.045414in}{0.696516in}}{\pgfqpoint{1.053314in}{0.693244in}}{\pgfqpoint{1.061550in}{0.693244in}}%
\pgfpathclose%
\pgfusepath{stroke,fill}%
\end{pgfscope}%
\begin{pgfscope}%
\pgfpathrectangle{\pgfqpoint{0.556847in}{0.516222in}}{\pgfqpoint{1.962733in}{1.783528in}} %
\pgfusepath{clip}%
\pgfsetbuttcap%
\pgfsetroundjoin%
\definecolor{currentfill}{rgb}{0.298039,0.447059,0.690196}%
\pgfsetfillcolor{currentfill}%
\pgfsetlinewidth{0.240900pt}%
\definecolor{currentstroke}{rgb}{1.000000,1.000000,1.000000}%
\pgfsetstrokecolor{currentstroke}%
\pgfsetdash{}{0pt}%
\pgfpathmoveto{\pgfqpoint{2.070955in}{2.001164in}}%
\pgfpathcurveto{\pgfqpoint{2.079192in}{2.001164in}}{\pgfqpoint{2.087092in}{2.004437in}}{\pgfqpoint{2.092916in}{2.010261in}}%
\pgfpathcurveto{\pgfqpoint{2.098739in}{2.016085in}}{\pgfqpoint{2.102012in}{2.023985in}}{\pgfqpoint{2.102012in}{2.032221in}}%
\pgfpathcurveto{\pgfqpoint{2.102012in}{2.040457in}}{\pgfqpoint{2.098739in}{2.048357in}}{\pgfqpoint{2.092916in}{2.054181in}}%
\pgfpathcurveto{\pgfqpoint{2.087092in}{2.060005in}}{\pgfqpoint{2.079192in}{2.063277in}}{\pgfqpoint{2.070955in}{2.063277in}}%
\pgfpathcurveto{\pgfqpoint{2.062719in}{2.063277in}}{\pgfqpoint{2.054819in}{2.060005in}}{\pgfqpoint{2.048995in}{2.054181in}}%
\pgfpathcurveto{\pgfqpoint{2.043171in}{2.048357in}}{\pgfqpoint{2.039899in}{2.040457in}}{\pgfqpoint{2.039899in}{2.032221in}}%
\pgfpathcurveto{\pgfqpoint{2.039899in}{2.023985in}}{\pgfqpoint{2.043171in}{2.016085in}}{\pgfqpoint{2.048995in}{2.010261in}}%
\pgfpathcurveto{\pgfqpoint{2.054819in}{2.004437in}}{\pgfqpoint{2.062719in}{2.001164in}}{\pgfqpoint{2.070955in}{2.001164in}}%
\pgfpathclose%
\pgfusepath{stroke,fill}%
\end{pgfscope}%
\begin{pgfscope}%
\pgfpathrectangle{\pgfqpoint{0.556847in}{0.516222in}}{\pgfqpoint{1.962733in}{1.783528in}} %
\pgfusepath{clip}%
\pgfsetbuttcap%
\pgfsetroundjoin%
\definecolor{currentfill}{rgb}{0.298039,0.447059,0.690196}%
\pgfsetfillcolor{currentfill}%
\pgfsetlinewidth{0.240900pt}%
\definecolor{currentstroke}{rgb}{1.000000,1.000000,1.000000}%
\pgfsetstrokecolor{currentstroke}%
\pgfsetdash{}{0pt}%
\pgfpathmoveto{\pgfqpoint{1.958799in}{1.466106in}}%
\pgfpathcurveto{\pgfqpoint{1.967035in}{1.466106in}}{\pgfqpoint{1.974935in}{1.469378in}}{\pgfqpoint{1.980759in}{1.475202in}}%
\pgfpathcurveto{\pgfqpoint{1.986583in}{1.481026in}}{\pgfqpoint{1.989856in}{1.488926in}}{\pgfqpoint{1.989856in}{1.497163in}}%
\pgfpathcurveto{\pgfqpoint{1.989856in}{1.505399in}}{\pgfqpoint{1.986583in}{1.513299in}}{\pgfqpoint{1.980759in}{1.519123in}}%
\pgfpathcurveto{\pgfqpoint{1.974935in}{1.524947in}}{\pgfqpoint{1.967035in}{1.528219in}}{\pgfqpoint{1.958799in}{1.528219in}}%
\pgfpathcurveto{\pgfqpoint{1.950563in}{1.528219in}}{\pgfqpoint{1.942663in}{1.524947in}}{\pgfqpoint{1.936839in}{1.519123in}}%
\pgfpathcurveto{\pgfqpoint{1.931015in}{1.513299in}}{\pgfqpoint{1.927743in}{1.505399in}}{\pgfqpoint{1.927743in}{1.497163in}}%
\pgfpathcurveto{\pgfqpoint{1.927743in}{1.488926in}}{\pgfqpoint{1.931015in}{1.481026in}}{\pgfqpoint{1.936839in}{1.475202in}}%
\pgfpathcurveto{\pgfqpoint{1.942663in}{1.469378in}}{\pgfqpoint{1.950563in}{1.466106in}}{\pgfqpoint{1.958799in}{1.466106in}}%
\pgfpathclose%
\pgfusepath{stroke,fill}%
\end{pgfscope}%
\begin{pgfscope}%
\pgfpathrectangle{\pgfqpoint{0.556847in}{0.516222in}}{\pgfqpoint{1.962733in}{1.783528in}} %
\pgfusepath{clip}%
\pgfsetbuttcap%
\pgfsetroundjoin%
\definecolor{currentfill}{rgb}{0.298039,0.447059,0.690196}%
\pgfsetfillcolor{currentfill}%
\pgfsetlinewidth{0.240900pt}%
\definecolor{currentstroke}{rgb}{1.000000,1.000000,1.000000}%
\pgfsetstrokecolor{currentstroke}%
\pgfsetdash{}{0pt}%
\pgfpathmoveto{\pgfqpoint{1.510175in}{1.258028in}}%
\pgfpathcurveto{\pgfqpoint{1.518411in}{1.258028in}}{\pgfqpoint{1.526311in}{1.261300in}}{\pgfqpoint{1.532135in}{1.267124in}}%
\pgfpathcurveto{\pgfqpoint{1.537959in}{1.272948in}}{\pgfqpoint{1.541231in}{1.280848in}}{\pgfqpoint{1.541231in}{1.289084in}}%
\pgfpathcurveto{\pgfqpoint{1.541231in}{1.297321in}}{\pgfqpoint{1.537959in}{1.305221in}}{\pgfqpoint{1.532135in}{1.311045in}}%
\pgfpathcurveto{\pgfqpoint{1.526311in}{1.316868in}}{\pgfqpoint{1.518411in}{1.320141in}}{\pgfqpoint{1.510175in}{1.320141in}}%
\pgfpathcurveto{\pgfqpoint{1.501938in}{1.320141in}}{\pgfqpoint{1.494038in}{1.316868in}}{\pgfqpoint{1.488214in}{1.311045in}}%
\pgfpathcurveto{\pgfqpoint{1.482390in}{1.305221in}}{\pgfqpoint{1.479118in}{1.297321in}}{\pgfqpoint{1.479118in}{1.289084in}}%
\pgfpathcurveto{\pgfqpoint{1.479118in}{1.280848in}}{\pgfqpoint{1.482390in}{1.272948in}}{\pgfqpoint{1.488214in}{1.267124in}}%
\pgfpathcurveto{\pgfqpoint{1.494038in}{1.261300in}}{\pgfqpoint{1.501938in}{1.258028in}}{\pgfqpoint{1.510175in}{1.258028in}}%
\pgfpathclose%
\pgfusepath{stroke,fill}%
\end{pgfscope}%
\begin{pgfscope}%
\pgfpathrectangle{\pgfqpoint{0.556847in}{0.516222in}}{\pgfqpoint{1.962733in}{1.783528in}} %
\pgfusepath{clip}%
\pgfsetbuttcap%
\pgfsetroundjoin%
\definecolor{currentfill}{rgb}{0.298039,0.447059,0.690196}%
\pgfsetfillcolor{currentfill}%
\pgfsetlinewidth{0.240900pt}%
\definecolor{currentstroke}{rgb}{1.000000,1.000000,1.000000}%
\pgfsetstrokecolor{currentstroke}%
\pgfsetdash{}{0pt}%
\pgfpathmoveto{\pgfqpoint{1.678409in}{0.633793in}}%
\pgfpathcurveto{\pgfqpoint{1.686645in}{0.633793in}}{\pgfqpoint{1.694545in}{0.637065in}}{\pgfqpoint{1.700369in}{0.642889in}}%
\pgfpathcurveto{\pgfqpoint{1.706193in}{0.648713in}}{\pgfqpoint{1.709465in}{0.656613in}}{\pgfqpoint{1.709465in}{0.664850in}}%
\pgfpathcurveto{\pgfqpoint{1.709465in}{0.673086in}}{\pgfqpoint{1.706193in}{0.680986in}}{\pgfqpoint{1.700369in}{0.686810in}}%
\pgfpathcurveto{\pgfqpoint{1.694545in}{0.692634in}}{\pgfqpoint{1.686645in}{0.695906in}}{\pgfqpoint{1.678409in}{0.695906in}}%
\pgfpathcurveto{\pgfqpoint{1.670172in}{0.695906in}}{\pgfqpoint{1.662272in}{0.692634in}}{\pgfqpoint{1.656448in}{0.686810in}}%
\pgfpathcurveto{\pgfqpoint{1.650625in}{0.680986in}}{\pgfqpoint{1.647352in}{0.673086in}}{\pgfqpoint{1.647352in}{0.664850in}}%
\pgfpathcurveto{\pgfqpoint{1.647352in}{0.656613in}}{\pgfqpoint{1.650625in}{0.648713in}}{\pgfqpoint{1.656448in}{0.642889in}}%
\pgfpathcurveto{\pgfqpoint{1.662272in}{0.637065in}}{\pgfqpoint{1.670172in}{0.633793in}}{\pgfqpoint{1.678409in}{0.633793in}}%
\pgfpathclose%
\pgfusepath{stroke,fill}%
\end{pgfscope}%
\begin{pgfscope}%
\pgfpathrectangle{\pgfqpoint{0.556847in}{0.516222in}}{\pgfqpoint{1.962733in}{1.783528in}} %
\pgfusepath{clip}%
\pgfsetbuttcap%
\pgfsetroundjoin%
\definecolor{currentfill}{rgb}{0.298039,0.447059,0.690196}%
\pgfsetfillcolor{currentfill}%
\pgfsetlinewidth{0.240900pt}%
\definecolor{currentstroke}{rgb}{1.000000,1.000000,1.000000}%
\pgfsetstrokecolor{currentstroke}%
\pgfsetdash{}{0pt}%
\pgfpathmoveto{\pgfqpoint{1.285862in}{1.049950in}}%
\pgfpathcurveto{\pgfqpoint{1.294098in}{1.049950in}}{\pgfqpoint{1.301999in}{1.053222in}}{\pgfqpoint{1.307822in}{1.059046in}}%
\pgfpathcurveto{\pgfqpoint{1.313646in}{1.064870in}}{\pgfqpoint{1.316919in}{1.072770in}}{\pgfqpoint{1.316919in}{1.081006in}}%
\pgfpathcurveto{\pgfqpoint{1.316919in}{1.089242in}}{\pgfqpoint{1.313646in}{1.097142in}}{\pgfqpoint{1.307822in}{1.102966in}}%
\pgfpathcurveto{\pgfqpoint{1.301999in}{1.108790in}}{\pgfqpoint{1.294098in}{1.112063in}}{\pgfqpoint{1.285862in}{1.112063in}}%
\pgfpathcurveto{\pgfqpoint{1.277626in}{1.112063in}}{\pgfqpoint{1.269726in}{1.108790in}}{\pgfqpoint{1.263902in}{1.102966in}}%
\pgfpathcurveto{\pgfqpoint{1.258078in}{1.097142in}}{\pgfqpoint{1.254806in}{1.089242in}}{\pgfqpoint{1.254806in}{1.081006in}}%
\pgfpathcurveto{\pgfqpoint{1.254806in}{1.072770in}}{\pgfqpoint{1.258078in}{1.064870in}}{\pgfqpoint{1.263902in}{1.059046in}}%
\pgfpathcurveto{\pgfqpoint{1.269726in}{1.053222in}}{\pgfqpoint{1.277626in}{1.049950in}}{\pgfqpoint{1.285862in}{1.049950in}}%
\pgfpathclose%
\pgfusepath{stroke,fill}%
\end{pgfscope}%
\begin{pgfscope}%
\pgfpathrectangle{\pgfqpoint{0.556847in}{0.516222in}}{\pgfqpoint{1.962733in}{1.783528in}} %
\pgfusepath{clip}%
\pgfsetbuttcap%
\pgfsetroundjoin%
\definecolor{currentfill}{rgb}{0.298039,0.447059,0.690196}%
\pgfsetfillcolor{currentfill}%
\pgfsetlinewidth{0.240900pt}%
\definecolor{currentstroke}{rgb}{1.000000,1.000000,1.000000}%
\pgfsetstrokecolor{currentstroke}%
\pgfsetdash{}{0pt}%
\pgfpathmoveto{\pgfqpoint{1.454096in}{1.139126in}}%
\pgfpathcurveto{\pgfqpoint{1.462333in}{1.139126in}}{\pgfqpoint{1.470233in}{1.142398in}}{\pgfqpoint{1.476057in}{1.148222in}}%
\pgfpathcurveto{\pgfqpoint{1.481881in}{1.154046in}}{\pgfqpoint{1.485153in}{1.161946in}}{\pgfqpoint{1.485153in}{1.170182in}}%
\pgfpathcurveto{\pgfqpoint{1.485153in}{1.178419in}}{\pgfqpoint{1.481881in}{1.186319in}}{\pgfqpoint{1.476057in}{1.192143in}}%
\pgfpathcurveto{\pgfqpoint{1.470233in}{1.197967in}}{\pgfqpoint{1.462333in}{1.201239in}}{\pgfqpoint{1.454096in}{1.201239in}}%
\pgfpathcurveto{\pgfqpoint{1.445860in}{1.201239in}}{\pgfqpoint{1.437960in}{1.197967in}}{\pgfqpoint{1.432136in}{1.192143in}}%
\pgfpathcurveto{\pgfqpoint{1.426312in}{1.186319in}}{\pgfqpoint{1.423040in}{1.178419in}}{\pgfqpoint{1.423040in}{1.170182in}}%
\pgfpathcurveto{\pgfqpoint{1.423040in}{1.161946in}}{\pgfqpoint{1.426312in}{1.154046in}}{\pgfqpoint{1.432136in}{1.148222in}}%
\pgfpathcurveto{\pgfqpoint{1.437960in}{1.142398in}}{\pgfqpoint{1.445860in}{1.139126in}}{\pgfqpoint{1.454096in}{1.139126in}}%
\pgfpathclose%
\pgfusepath{stroke,fill}%
\end{pgfscope}%
\begin{pgfscope}%
\pgfpathrectangle{\pgfqpoint{0.556847in}{0.516222in}}{\pgfqpoint{1.962733in}{1.783528in}} %
\pgfusepath{clip}%
\pgfsetbuttcap%
\pgfsetroundjoin%
\definecolor{currentfill}{rgb}{0.298039,0.447059,0.690196}%
\pgfsetfillcolor{currentfill}%
\pgfsetlinewidth{0.240900pt}%
\definecolor{currentstroke}{rgb}{1.000000,1.000000,1.000000}%
\pgfsetstrokecolor{currentstroke}%
\pgfsetdash{}{0pt}%
\pgfpathmoveto{\pgfqpoint{1.398018in}{1.495831in}}%
\pgfpathcurveto{\pgfqpoint{1.406255in}{1.495831in}}{\pgfqpoint{1.414155in}{1.499104in}}{\pgfqpoint{1.419979in}{1.504928in}}%
\pgfpathcurveto{\pgfqpoint{1.425803in}{1.510752in}}{\pgfqpoint{1.429075in}{1.518652in}}{\pgfqpoint{1.429075in}{1.526888in}}%
\pgfpathcurveto{\pgfqpoint{1.429075in}{1.535124in}}{\pgfqpoint{1.425803in}{1.543024in}}{\pgfqpoint{1.419979in}{1.548848in}}%
\pgfpathcurveto{\pgfqpoint{1.414155in}{1.554672in}}{\pgfqpoint{1.406255in}{1.557944in}}{\pgfqpoint{1.398018in}{1.557944in}}%
\pgfpathcurveto{\pgfqpoint{1.389782in}{1.557944in}}{\pgfqpoint{1.381882in}{1.554672in}}{\pgfqpoint{1.376058in}{1.548848in}}%
\pgfpathcurveto{\pgfqpoint{1.370234in}{1.543024in}}{\pgfqpoint{1.366962in}{1.535124in}}{\pgfqpoint{1.366962in}{1.526888in}}%
\pgfpathcurveto{\pgfqpoint{1.366962in}{1.518652in}}{\pgfqpoint{1.370234in}{1.510752in}}{\pgfqpoint{1.376058in}{1.504928in}}%
\pgfpathcurveto{\pgfqpoint{1.381882in}{1.499104in}}{\pgfqpoint{1.389782in}{1.495831in}}{\pgfqpoint{1.398018in}{1.495831in}}%
\pgfpathclose%
\pgfusepath{stroke,fill}%
\end{pgfscope}%
\begin{pgfscope}%
\pgfpathrectangle{\pgfqpoint{0.556847in}{0.516222in}}{\pgfqpoint{1.962733in}{1.783528in}} %
\pgfusepath{clip}%
\pgfsetbuttcap%
\pgfsetroundjoin%
\definecolor{currentfill}{rgb}{0.298039,0.447059,0.690196}%
\pgfsetfillcolor{currentfill}%
\pgfsetlinewidth{0.240900pt}%
\definecolor{currentstroke}{rgb}{1.000000,1.000000,1.000000}%
\pgfsetstrokecolor{currentstroke}%
\pgfsetdash{}{0pt}%
\pgfpathmoveto{\pgfqpoint{2.014877in}{0.901322in}}%
\pgfpathcurveto{\pgfqpoint{2.023113in}{0.901322in}}{\pgfqpoint{2.031014in}{0.904595in}}{\pgfqpoint{2.036837in}{0.910418in}}%
\pgfpathcurveto{\pgfqpoint{2.042661in}{0.916242in}}{\pgfqpoint{2.045934in}{0.924142in}}{\pgfqpoint{2.045934in}{0.932379in}}%
\pgfpathcurveto{\pgfqpoint{2.045934in}{0.940615in}}{\pgfqpoint{2.042661in}{0.948515in}}{\pgfqpoint{2.036837in}{0.954339in}}%
\pgfpathcurveto{\pgfqpoint{2.031014in}{0.960163in}}{\pgfqpoint{2.023113in}{0.963435in}}{\pgfqpoint{2.014877in}{0.963435in}}%
\pgfpathcurveto{\pgfqpoint{2.006641in}{0.963435in}}{\pgfqpoint{1.998741in}{0.960163in}}{\pgfqpoint{1.992917in}{0.954339in}}%
\pgfpathcurveto{\pgfqpoint{1.987093in}{0.948515in}}{\pgfqpoint{1.983821in}{0.940615in}}{\pgfqpoint{1.983821in}{0.932379in}}%
\pgfpathcurveto{\pgfqpoint{1.983821in}{0.924142in}}{\pgfqpoint{1.987093in}{0.916242in}}{\pgfqpoint{1.992917in}{0.910418in}}%
\pgfpathcurveto{\pgfqpoint{1.998741in}{0.904595in}}{\pgfqpoint{2.006641in}{0.901322in}}{\pgfqpoint{2.014877in}{0.901322in}}%
\pgfpathclose%
\pgfusepath{stroke,fill}%
\end{pgfscope}%
\begin{pgfscope}%
\pgfpathrectangle{\pgfqpoint{0.556847in}{0.516222in}}{\pgfqpoint{1.962733in}{1.783528in}} %
\pgfusepath{clip}%
\pgfsetbuttcap%
\pgfsetroundjoin%
\definecolor{currentfill}{rgb}{0.298039,0.447059,0.690196}%
\pgfsetfillcolor{currentfill}%
\pgfsetlinewidth{0.240900pt}%
\definecolor{currentstroke}{rgb}{1.000000,1.000000,1.000000}%
\pgfsetstrokecolor{currentstroke}%
\pgfsetdash{}{0pt}%
\pgfpathmoveto{\pgfqpoint{1.846643in}{1.644459in}}%
\pgfpathcurveto{\pgfqpoint{1.854879in}{1.644459in}}{\pgfqpoint{1.862779in}{1.647731in}}{\pgfqpoint{1.868603in}{1.653555in}}%
\pgfpathcurveto{\pgfqpoint{1.874427in}{1.659379in}}{\pgfqpoint{1.877699in}{1.667279in}}{\pgfqpoint{1.877699in}{1.675515in}}%
\pgfpathcurveto{\pgfqpoint{1.877699in}{1.683752in}}{\pgfqpoint{1.874427in}{1.691652in}}{\pgfqpoint{1.868603in}{1.697476in}}%
\pgfpathcurveto{\pgfqpoint{1.862779in}{1.703299in}}{\pgfqpoint{1.854879in}{1.706572in}}{\pgfqpoint{1.846643in}{1.706572in}}%
\pgfpathcurveto{\pgfqpoint{1.838407in}{1.706572in}}{\pgfqpoint{1.830507in}{1.703299in}}{\pgfqpoint{1.824683in}{1.697476in}}%
\pgfpathcurveto{\pgfqpoint{1.818859in}{1.691652in}}{\pgfqpoint{1.815586in}{1.683752in}}{\pgfqpoint{1.815586in}{1.675515in}}%
\pgfpathcurveto{\pgfqpoint{1.815586in}{1.667279in}}{\pgfqpoint{1.818859in}{1.659379in}}{\pgfqpoint{1.824683in}{1.653555in}}%
\pgfpathcurveto{\pgfqpoint{1.830507in}{1.647731in}}{\pgfqpoint{1.838407in}{1.644459in}}{\pgfqpoint{1.846643in}{1.644459in}}%
\pgfpathclose%
\pgfusepath{stroke,fill}%
\end{pgfscope}%
\begin{pgfscope}%
\pgfpathrectangle{\pgfqpoint{0.556847in}{0.516222in}}{\pgfqpoint{1.962733in}{1.783528in}} %
\pgfusepath{clip}%
\pgfsetbuttcap%
\pgfsetroundjoin%
\definecolor{currentfill}{rgb}{0.298039,0.447059,0.690196}%
\pgfsetfillcolor{currentfill}%
\pgfsetlinewidth{0.240900pt}%
\definecolor{currentstroke}{rgb}{1.000000,1.000000,1.000000}%
\pgfsetstrokecolor{currentstroke}%
\pgfsetdash{}{0pt}%
\pgfpathmoveto{\pgfqpoint{1.229784in}{0.871597in}}%
\pgfpathcurveto{\pgfqpoint{1.238020in}{0.871597in}}{\pgfqpoint{1.245920in}{0.874869in}}{\pgfqpoint{1.251744in}{0.880693in}}%
\pgfpathcurveto{\pgfqpoint{1.257568in}{0.886517in}}{\pgfqpoint{1.260841in}{0.894417in}}{\pgfqpoint{1.260841in}{0.902653in}}%
\pgfpathcurveto{\pgfqpoint{1.260841in}{0.910890in}}{\pgfqpoint{1.257568in}{0.918790in}}{\pgfqpoint{1.251744in}{0.924614in}}%
\pgfpathcurveto{\pgfqpoint{1.245920in}{0.930437in}}{\pgfqpoint{1.238020in}{0.933710in}}{\pgfqpoint{1.229784in}{0.933710in}}%
\pgfpathcurveto{\pgfqpoint{1.221548in}{0.933710in}}{\pgfqpoint{1.213648in}{0.930437in}}{\pgfqpoint{1.207824in}{0.924614in}}%
\pgfpathcurveto{\pgfqpoint{1.202000in}{0.918790in}}{\pgfqpoint{1.198728in}{0.910890in}}{\pgfqpoint{1.198728in}{0.902653in}}%
\pgfpathcurveto{\pgfqpoint{1.198728in}{0.894417in}}{\pgfqpoint{1.202000in}{0.886517in}}{\pgfqpoint{1.207824in}{0.880693in}}%
\pgfpathcurveto{\pgfqpoint{1.213648in}{0.874869in}}{\pgfqpoint{1.221548in}{0.871597in}}{\pgfqpoint{1.229784in}{0.871597in}}%
\pgfpathclose%
\pgfusepath{stroke,fill}%
\end{pgfscope}%
\begin{pgfscope}%
\pgfpathrectangle{\pgfqpoint{0.556847in}{0.516222in}}{\pgfqpoint{1.962733in}{1.783528in}} %
\pgfusepath{clip}%
\pgfsetbuttcap%
\pgfsetroundjoin%
\definecolor{currentfill}{rgb}{0.298039,0.447059,0.690196}%
\pgfsetfillcolor{currentfill}%
\pgfsetlinewidth{0.240900pt}%
\definecolor{currentstroke}{rgb}{1.000000,1.000000,1.000000}%
\pgfsetstrokecolor{currentstroke}%
\pgfsetdash{}{0pt}%
\pgfpathmoveto{\pgfqpoint{2.183111in}{1.198577in}}%
\pgfpathcurveto{\pgfqpoint{2.191348in}{1.198577in}}{\pgfqpoint{2.199248in}{1.201849in}}{\pgfqpoint{2.205072in}{1.207673in}}%
\pgfpathcurveto{\pgfqpoint{2.210896in}{1.213497in}}{\pgfqpoint{2.214168in}{1.221397in}}{\pgfqpoint{2.214168in}{1.229633in}}%
\pgfpathcurveto{\pgfqpoint{2.214168in}{1.237870in}}{\pgfqpoint{2.210896in}{1.245770in}}{\pgfqpoint{2.205072in}{1.251594in}}%
\pgfpathcurveto{\pgfqpoint{2.199248in}{1.257418in}}{\pgfqpoint{2.191348in}{1.260690in}}{\pgfqpoint{2.183111in}{1.260690in}}%
\pgfpathcurveto{\pgfqpoint{2.174875in}{1.260690in}}{\pgfqpoint{2.166975in}{1.257418in}}{\pgfqpoint{2.161151in}{1.251594in}}%
\pgfpathcurveto{\pgfqpoint{2.155327in}{1.245770in}}{\pgfqpoint{2.152055in}{1.237870in}}{\pgfqpoint{2.152055in}{1.229633in}}%
\pgfpathcurveto{\pgfqpoint{2.152055in}{1.221397in}}{\pgfqpoint{2.155327in}{1.213497in}}{\pgfqpoint{2.161151in}{1.207673in}}%
\pgfpathcurveto{\pgfqpoint{2.166975in}{1.201849in}}{\pgfqpoint{2.174875in}{1.198577in}}{\pgfqpoint{2.183111in}{1.198577in}}%
\pgfpathclose%
\pgfusepath{stroke,fill}%
\end{pgfscope}%
\begin{pgfscope}%
\pgfpathrectangle{\pgfqpoint{0.556847in}{0.516222in}}{\pgfqpoint{1.962733in}{1.783528in}} %
\pgfusepath{clip}%
\pgfsetbuttcap%
\pgfsetroundjoin%
\definecolor{currentfill}{rgb}{0.298039,0.447059,0.690196}%
\pgfsetfillcolor{currentfill}%
\pgfsetlinewidth{0.240900pt}%
\definecolor{currentstroke}{rgb}{1.000000,1.000000,1.000000}%
\pgfsetstrokecolor{currentstroke}%
\pgfsetdash{}{0pt}%
\pgfpathmoveto{\pgfqpoint{2.295268in}{1.406655in}}%
\pgfpathcurveto{\pgfqpoint{2.303504in}{1.406655in}}{\pgfqpoint{2.311404in}{1.409927in}}{\pgfqpoint{2.317228in}{1.415751in}}%
\pgfpathcurveto{\pgfqpoint{2.323052in}{1.421575in}}{\pgfqpoint{2.326324in}{1.429475in}}{\pgfqpoint{2.326324in}{1.437712in}}%
\pgfpathcurveto{\pgfqpoint{2.326324in}{1.445948in}}{\pgfqpoint{2.323052in}{1.453848in}}{\pgfqpoint{2.317228in}{1.459672in}}%
\pgfpathcurveto{\pgfqpoint{2.311404in}{1.465496in}}{\pgfqpoint{2.303504in}{1.468768in}}{\pgfqpoint{2.295268in}{1.468768in}}%
\pgfpathcurveto{\pgfqpoint{2.287031in}{1.468768in}}{\pgfqpoint{2.279131in}{1.465496in}}{\pgfqpoint{2.273307in}{1.459672in}}%
\pgfpathcurveto{\pgfqpoint{2.267483in}{1.453848in}}{\pgfqpoint{2.264211in}{1.445948in}}{\pgfqpoint{2.264211in}{1.437712in}}%
\pgfpathcurveto{\pgfqpoint{2.264211in}{1.429475in}}{\pgfqpoint{2.267483in}{1.421575in}}{\pgfqpoint{2.273307in}{1.415751in}}%
\pgfpathcurveto{\pgfqpoint{2.279131in}{1.409927in}}{\pgfqpoint{2.287031in}{1.406655in}}{\pgfqpoint{2.295268in}{1.406655in}}%
\pgfpathclose%
\pgfusepath{stroke,fill}%
\end{pgfscope}%
\begin{pgfscope}%
\pgfpathrectangle{\pgfqpoint{0.556847in}{0.516222in}}{\pgfqpoint{1.962733in}{1.783528in}} %
\pgfusepath{clip}%
\pgfsetbuttcap%
\pgfsetroundjoin%
\definecolor{currentfill}{rgb}{0.298039,0.447059,0.690196}%
\pgfsetfillcolor{currentfill}%
\pgfsetlinewidth{0.240900pt}%
\definecolor{currentstroke}{rgb}{1.000000,1.000000,1.000000}%
\pgfsetstrokecolor{currentstroke}%
\pgfsetdash{}{0pt}%
\pgfpathmoveto{\pgfqpoint{1.341940in}{1.139126in}}%
\pgfpathcurveto{\pgfqpoint{1.350177in}{1.139126in}}{\pgfqpoint{1.358077in}{1.142398in}}{\pgfqpoint{1.363901in}{1.148222in}}%
\pgfpathcurveto{\pgfqpoint{1.369724in}{1.154046in}}{\pgfqpoint{1.372997in}{1.161946in}}{\pgfqpoint{1.372997in}{1.170182in}}%
\pgfpathcurveto{\pgfqpoint{1.372997in}{1.178419in}}{\pgfqpoint{1.369724in}{1.186319in}}{\pgfqpoint{1.363901in}{1.192143in}}%
\pgfpathcurveto{\pgfqpoint{1.358077in}{1.197967in}}{\pgfqpoint{1.350177in}{1.201239in}}{\pgfqpoint{1.341940in}{1.201239in}}%
\pgfpathcurveto{\pgfqpoint{1.333704in}{1.201239in}}{\pgfqpoint{1.325804in}{1.197967in}}{\pgfqpoint{1.319980in}{1.192143in}}%
\pgfpathcurveto{\pgfqpoint{1.314156in}{1.186319in}}{\pgfqpoint{1.310884in}{1.178419in}}{\pgfqpoint{1.310884in}{1.170182in}}%
\pgfpathcurveto{\pgfqpoint{1.310884in}{1.161946in}}{\pgfqpoint{1.314156in}{1.154046in}}{\pgfqpoint{1.319980in}{1.148222in}}%
\pgfpathcurveto{\pgfqpoint{1.325804in}{1.142398in}}{\pgfqpoint{1.333704in}{1.139126in}}{\pgfqpoint{1.341940in}{1.139126in}}%
\pgfpathclose%
\pgfusepath{stroke,fill}%
\end{pgfscope}%
\begin{pgfscope}%
\pgfpathrectangle{\pgfqpoint{0.556847in}{0.516222in}}{\pgfqpoint{1.962733in}{1.783528in}} %
\pgfusepath{clip}%
\pgfsetbuttcap%
\pgfsetroundjoin%
\definecolor{currentfill}{rgb}{0.298039,0.447059,0.690196}%
\pgfsetfillcolor{currentfill}%
\pgfsetlinewidth{0.240900pt}%
\definecolor{currentstroke}{rgb}{1.000000,1.000000,1.000000}%
\pgfsetstrokecolor{currentstroke}%
\pgfsetdash{}{0pt}%
\pgfpathmoveto{\pgfqpoint{0.837238in}{1.079675in}}%
\pgfpathcurveto{\pgfqpoint{0.845474in}{1.079675in}}{\pgfqpoint{0.853374in}{1.082947in}}{\pgfqpoint{0.859198in}{1.088771in}}%
\pgfpathcurveto{\pgfqpoint{0.865022in}{1.094595in}}{\pgfqpoint{0.868294in}{1.102495in}}{\pgfqpoint{0.868294in}{1.110731in}}%
\pgfpathcurveto{\pgfqpoint{0.868294in}{1.118968in}}{\pgfqpoint{0.865022in}{1.126868in}}{\pgfqpoint{0.859198in}{1.132692in}}%
\pgfpathcurveto{\pgfqpoint{0.853374in}{1.138516in}}{\pgfqpoint{0.845474in}{1.141788in}}{\pgfqpoint{0.837238in}{1.141788in}}%
\pgfpathcurveto{\pgfqpoint{0.829001in}{1.141788in}}{\pgfqpoint{0.821101in}{1.138516in}}{\pgfqpoint{0.815277in}{1.132692in}}%
\pgfpathcurveto{\pgfqpoint{0.809453in}{1.126868in}}{\pgfqpoint{0.806181in}{1.118968in}}{\pgfqpoint{0.806181in}{1.110731in}}%
\pgfpathcurveto{\pgfqpoint{0.806181in}{1.102495in}}{\pgfqpoint{0.809453in}{1.094595in}}{\pgfqpoint{0.815277in}{1.088771in}}%
\pgfpathcurveto{\pgfqpoint{0.821101in}{1.082947in}}{\pgfqpoint{0.829001in}{1.079675in}}{\pgfqpoint{0.837238in}{1.079675in}}%
\pgfpathclose%
\pgfusepath{stroke,fill}%
\end{pgfscope}%
\begin{pgfscope}%
\pgfpathrectangle{\pgfqpoint{0.556847in}{0.516222in}}{\pgfqpoint{1.962733in}{1.783528in}} %
\pgfusepath{clip}%
\pgfsetbuttcap%
\pgfsetroundjoin%
\definecolor{currentfill}{rgb}{0.298039,0.447059,0.690196}%
\pgfsetfillcolor{currentfill}%
\pgfsetlinewidth{0.240900pt}%
\definecolor{currentstroke}{rgb}{1.000000,1.000000,1.000000}%
\pgfsetstrokecolor{currentstroke}%
\pgfsetdash{}{0pt}%
\pgfpathmoveto{\pgfqpoint{1.285862in}{1.168851in}}%
\pgfpathcurveto{\pgfqpoint{1.294098in}{1.168851in}}{\pgfqpoint{1.301999in}{1.172124in}}{\pgfqpoint{1.307822in}{1.177948in}}%
\pgfpathcurveto{\pgfqpoint{1.313646in}{1.183772in}}{\pgfqpoint{1.316919in}{1.191672in}}{\pgfqpoint{1.316919in}{1.199908in}}%
\pgfpathcurveto{\pgfqpoint{1.316919in}{1.208144in}}{\pgfqpoint{1.313646in}{1.216044in}}{\pgfqpoint{1.307822in}{1.221868in}}%
\pgfpathcurveto{\pgfqpoint{1.301999in}{1.227692in}}{\pgfqpoint{1.294098in}{1.230964in}}{\pgfqpoint{1.285862in}{1.230964in}}%
\pgfpathcurveto{\pgfqpoint{1.277626in}{1.230964in}}{\pgfqpoint{1.269726in}{1.227692in}}{\pgfqpoint{1.263902in}{1.221868in}}%
\pgfpathcurveto{\pgfqpoint{1.258078in}{1.216044in}}{\pgfqpoint{1.254806in}{1.208144in}}{\pgfqpoint{1.254806in}{1.199908in}}%
\pgfpathcurveto{\pgfqpoint{1.254806in}{1.191672in}}{\pgfqpoint{1.258078in}{1.183772in}}{\pgfqpoint{1.263902in}{1.177948in}}%
\pgfpathcurveto{\pgfqpoint{1.269726in}{1.172124in}}{\pgfqpoint{1.277626in}{1.168851in}}{\pgfqpoint{1.285862in}{1.168851in}}%
\pgfpathclose%
\pgfusepath{stroke,fill}%
\end{pgfscope}%
\begin{pgfscope}%
\pgfpathrectangle{\pgfqpoint{0.556847in}{0.516222in}}{\pgfqpoint{1.962733in}{1.783528in}} %
\pgfusepath{clip}%
\pgfsetbuttcap%
\pgfsetroundjoin%
\definecolor{currentfill}{rgb}{0.298039,0.447059,0.690196}%
\pgfsetfillcolor{currentfill}%
\pgfsetlinewidth{0.240900pt}%
\definecolor{currentstroke}{rgb}{1.000000,1.000000,1.000000}%
\pgfsetstrokecolor{currentstroke}%
\pgfsetdash{}{0pt}%
\pgfpathmoveto{\pgfqpoint{2.183111in}{1.763361in}}%
\pgfpathcurveto{\pgfqpoint{2.191348in}{1.763361in}}{\pgfqpoint{2.199248in}{1.766633in}}{\pgfqpoint{2.205072in}{1.772457in}}%
\pgfpathcurveto{\pgfqpoint{2.210896in}{1.778281in}}{\pgfqpoint{2.214168in}{1.786181in}}{\pgfqpoint{2.214168in}{1.794417in}}%
\pgfpathcurveto{\pgfqpoint{2.214168in}{1.802653in}}{\pgfqpoint{2.210896in}{1.810553in}}{\pgfqpoint{2.205072in}{1.816377in}}%
\pgfpathcurveto{\pgfqpoint{2.199248in}{1.822201in}}{\pgfqpoint{2.191348in}{1.825474in}}{\pgfqpoint{2.183111in}{1.825474in}}%
\pgfpathcurveto{\pgfqpoint{2.174875in}{1.825474in}}{\pgfqpoint{2.166975in}{1.822201in}}{\pgfqpoint{2.161151in}{1.816377in}}%
\pgfpathcurveto{\pgfqpoint{2.155327in}{1.810553in}}{\pgfqpoint{2.152055in}{1.802653in}}{\pgfqpoint{2.152055in}{1.794417in}}%
\pgfpathcurveto{\pgfqpoint{2.152055in}{1.786181in}}{\pgfqpoint{2.155327in}{1.778281in}}{\pgfqpoint{2.161151in}{1.772457in}}%
\pgfpathcurveto{\pgfqpoint{2.166975in}{1.766633in}}{\pgfqpoint{2.174875in}{1.763361in}}{\pgfqpoint{2.183111in}{1.763361in}}%
\pgfpathclose%
\pgfusepath{stroke,fill}%
\end{pgfscope}%
\begin{pgfscope}%
\pgfsetrectcap%
\pgfsetmiterjoin%
\pgfsetlinewidth{0.000000pt}%
\definecolor{currentstroke}{rgb}{1.000000,1.000000,1.000000}%
\pgfsetstrokecolor{currentstroke}%
\pgfsetdash{}{0pt}%
\pgfpathmoveto{\pgfqpoint{0.556847in}{0.516222in}}%
\pgfpathlineto{\pgfqpoint{0.556847in}{2.299750in}}%
\pgfusepath{}%
\end{pgfscope}%
\begin{pgfscope}%
\pgfsetrectcap%
\pgfsetmiterjoin%
\pgfsetlinewidth{0.000000pt}%
\definecolor{currentstroke}{rgb}{1.000000,1.000000,1.000000}%
\pgfsetstrokecolor{currentstroke}%
\pgfsetdash{}{0pt}%
\pgfpathmoveto{\pgfqpoint{0.556847in}{0.516222in}}%
\pgfpathlineto{\pgfqpoint{2.519580in}{0.516222in}}%
\pgfusepath{}%
\end{pgfscope}%
\begin{pgfscope}%
\pgfsetbuttcap%
\pgfsetmiterjoin%
\definecolor{currentfill}{rgb}{0.917647,0.917647,0.949020}%
\pgfsetfillcolor{currentfill}%
\pgfsetlinewidth{0.000000pt}%
\definecolor{currentstroke}{rgb}{0.000000,0.000000,0.000000}%
\pgfsetstrokecolor{currentstroke}%
\pgfsetstrokeopacity{0.000000}%
\pgfsetdash{}{0pt}%
\pgfpathmoveto{\pgfqpoint{2.816705in}{0.516222in}}%
\pgfpathlineto{\pgfqpoint{4.779438in}{0.516222in}}%
\pgfpathlineto{\pgfqpoint{4.779438in}{2.299750in}}%
\pgfpathlineto{\pgfqpoint{2.816705in}{2.299750in}}%
\pgfpathclose%
\pgfusepath{fill}%
\end{pgfscope}%
\begin{pgfscope}%
\pgfpathrectangle{\pgfqpoint{2.816705in}{0.516222in}}{\pgfqpoint{1.962733in}{1.783528in}} %
\pgfusepath{clip}%
\pgfsetroundcap%
\pgfsetroundjoin%
\pgfsetlinewidth{0.803000pt}%
\definecolor{currentstroke}{rgb}{1.000000,1.000000,1.000000}%
\pgfsetstrokecolor{currentstroke}%
\pgfsetdash{}{0pt}%
\pgfpathmoveto{\pgfqpoint{2.816705in}{0.516222in}}%
\pgfpathlineto{\pgfqpoint{2.816705in}{2.299750in}}%
\pgfusepath{stroke}%
\end{pgfscope}%
\begin{pgfscope}%
\pgfsetbuttcap%
\pgfsetroundjoin%
\definecolor{currentfill}{rgb}{0.150000,0.150000,0.150000}%
\pgfsetfillcolor{currentfill}%
\pgfsetlinewidth{0.803000pt}%
\definecolor{currentstroke}{rgb}{0.150000,0.150000,0.150000}%
\pgfsetstrokecolor{currentstroke}%
\pgfsetdash{}{0pt}%
\pgfsys@defobject{currentmarker}{\pgfqpoint{0.000000in}{0.000000in}}{\pgfqpoint{0.000000in}{0.000000in}}{%
\pgfpathmoveto{\pgfqpoint{0.000000in}{0.000000in}}%
\pgfpathlineto{\pgfqpoint{0.000000in}{0.000000in}}%
\pgfusepath{stroke,fill}%
}%
\begin{pgfscope}%
\pgfsys@transformshift{2.816705in}{0.516222in}%
\pgfsys@useobject{currentmarker}{}%
\end{pgfscope}%
\end{pgfscope}%
\begin{pgfscope}%
\definecolor{textcolor}{rgb}{0.150000,0.150000,0.150000}%
\pgfsetstrokecolor{textcolor}%
\pgfsetfillcolor{textcolor}%
\pgftext[x=2.816705in,y=0.438444in,,top]{\color{textcolor}\sffamily\fontsize{8.000000}{9.600000}\selectfont 2.0}%
\end{pgfscope}%
\begin{pgfscope}%
\pgfpathrectangle{\pgfqpoint{2.816705in}{0.516222in}}{\pgfqpoint{1.962733in}{1.783528in}} %
\pgfusepath{clip}%
\pgfsetroundcap%
\pgfsetroundjoin%
\pgfsetlinewidth{0.803000pt}%
\definecolor{currentstroke}{rgb}{1.000000,1.000000,1.000000}%
\pgfsetstrokecolor{currentstroke}%
\pgfsetdash{}{0pt}%
\pgfpathmoveto{\pgfqpoint{3.097095in}{0.516222in}}%
\pgfpathlineto{\pgfqpoint{3.097095in}{2.299750in}}%
\pgfusepath{stroke}%
\end{pgfscope}%
\begin{pgfscope}%
\pgfsetbuttcap%
\pgfsetroundjoin%
\definecolor{currentfill}{rgb}{0.150000,0.150000,0.150000}%
\pgfsetfillcolor{currentfill}%
\pgfsetlinewidth{0.803000pt}%
\definecolor{currentstroke}{rgb}{0.150000,0.150000,0.150000}%
\pgfsetstrokecolor{currentstroke}%
\pgfsetdash{}{0pt}%
\pgfsys@defobject{currentmarker}{\pgfqpoint{0.000000in}{0.000000in}}{\pgfqpoint{0.000000in}{0.000000in}}{%
\pgfpathmoveto{\pgfqpoint{0.000000in}{0.000000in}}%
\pgfpathlineto{\pgfqpoint{0.000000in}{0.000000in}}%
\pgfusepath{stroke,fill}%
}%
\begin{pgfscope}%
\pgfsys@transformshift{3.097095in}{0.516222in}%
\pgfsys@useobject{currentmarker}{}%
\end{pgfscope}%
\end{pgfscope}%
\begin{pgfscope}%
\definecolor{textcolor}{rgb}{0.150000,0.150000,0.150000}%
\pgfsetstrokecolor{textcolor}%
\pgfsetfillcolor{textcolor}%
\pgftext[x=3.097095in,y=0.438444in,,top]{\color{textcolor}\sffamily\fontsize{8.000000}{9.600000}\selectfont 2.5}%
\end{pgfscope}%
\begin{pgfscope}%
\pgfpathrectangle{\pgfqpoint{2.816705in}{0.516222in}}{\pgfqpoint{1.962733in}{1.783528in}} %
\pgfusepath{clip}%
\pgfsetroundcap%
\pgfsetroundjoin%
\pgfsetlinewidth{0.803000pt}%
\definecolor{currentstroke}{rgb}{1.000000,1.000000,1.000000}%
\pgfsetstrokecolor{currentstroke}%
\pgfsetdash{}{0pt}%
\pgfpathmoveto{\pgfqpoint{3.377486in}{0.516222in}}%
\pgfpathlineto{\pgfqpoint{3.377486in}{2.299750in}}%
\pgfusepath{stroke}%
\end{pgfscope}%
\begin{pgfscope}%
\pgfsetbuttcap%
\pgfsetroundjoin%
\definecolor{currentfill}{rgb}{0.150000,0.150000,0.150000}%
\pgfsetfillcolor{currentfill}%
\pgfsetlinewidth{0.803000pt}%
\definecolor{currentstroke}{rgb}{0.150000,0.150000,0.150000}%
\pgfsetstrokecolor{currentstroke}%
\pgfsetdash{}{0pt}%
\pgfsys@defobject{currentmarker}{\pgfqpoint{0.000000in}{0.000000in}}{\pgfqpoint{0.000000in}{0.000000in}}{%
\pgfpathmoveto{\pgfqpoint{0.000000in}{0.000000in}}%
\pgfpathlineto{\pgfqpoint{0.000000in}{0.000000in}}%
\pgfusepath{stroke,fill}%
}%
\begin{pgfscope}%
\pgfsys@transformshift{3.377486in}{0.516222in}%
\pgfsys@useobject{currentmarker}{}%
\end{pgfscope}%
\end{pgfscope}%
\begin{pgfscope}%
\definecolor{textcolor}{rgb}{0.150000,0.150000,0.150000}%
\pgfsetstrokecolor{textcolor}%
\pgfsetfillcolor{textcolor}%
\pgftext[x=3.377486in,y=0.438444in,,top]{\color{textcolor}\sffamily\fontsize{8.000000}{9.600000}\selectfont 3.0}%
\end{pgfscope}%
\begin{pgfscope}%
\pgfpathrectangle{\pgfqpoint{2.816705in}{0.516222in}}{\pgfqpoint{1.962733in}{1.783528in}} %
\pgfusepath{clip}%
\pgfsetroundcap%
\pgfsetroundjoin%
\pgfsetlinewidth{0.803000pt}%
\definecolor{currentstroke}{rgb}{1.000000,1.000000,1.000000}%
\pgfsetstrokecolor{currentstroke}%
\pgfsetdash{}{0pt}%
\pgfpathmoveto{\pgfqpoint{3.657876in}{0.516222in}}%
\pgfpathlineto{\pgfqpoint{3.657876in}{2.299750in}}%
\pgfusepath{stroke}%
\end{pgfscope}%
\begin{pgfscope}%
\pgfsetbuttcap%
\pgfsetroundjoin%
\definecolor{currentfill}{rgb}{0.150000,0.150000,0.150000}%
\pgfsetfillcolor{currentfill}%
\pgfsetlinewidth{0.803000pt}%
\definecolor{currentstroke}{rgb}{0.150000,0.150000,0.150000}%
\pgfsetstrokecolor{currentstroke}%
\pgfsetdash{}{0pt}%
\pgfsys@defobject{currentmarker}{\pgfqpoint{0.000000in}{0.000000in}}{\pgfqpoint{0.000000in}{0.000000in}}{%
\pgfpathmoveto{\pgfqpoint{0.000000in}{0.000000in}}%
\pgfpathlineto{\pgfqpoint{0.000000in}{0.000000in}}%
\pgfusepath{stroke,fill}%
}%
\begin{pgfscope}%
\pgfsys@transformshift{3.657876in}{0.516222in}%
\pgfsys@useobject{currentmarker}{}%
\end{pgfscope}%
\end{pgfscope}%
\begin{pgfscope}%
\definecolor{textcolor}{rgb}{0.150000,0.150000,0.150000}%
\pgfsetstrokecolor{textcolor}%
\pgfsetfillcolor{textcolor}%
\pgftext[x=3.657876in,y=0.438444in,,top]{\color{textcolor}\sffamily\fontsize{8.000000}{9.600000}\selectfont 3.5}%
\end{pgfscope}%
\begin{pgfscope}%
\pgfpathrectangle{\pgfqpoint{2.816705in}{0.516222in}}{\pgfqpoint{1.962733in}{1.783528in}} %
\pgfusepath{clip}%
\pgfsetroundcap%
\pgfsetroundjoin%
\pgfsetlinewidth{0.803000pt}%
\definecolor{currentstroke}{rgb}{1.000000,1.000000,1.000000}%
\pgfsetstrokecolor{currentstroke}%
\pgfsetdash{}{0pt}%
\pgfpathmoveto{\pgfqpoint{3.938266in}{0.516222in}}%
\pgfpathlineto{\pgfqpoint{3.938266in}{2.299750in}}%
\pgfusepath{stroke}%
\end{pgfscope}%
\begin{pgfscope}%
\pgfsetbuttcap%
\pgfsetroundjoin%
\definecolor{currentfill}{rgb}{0.150000,0.150000,0.150000}%
\pgfsetfillcolor{currentfill}%
\pgfsetlinewidth{0.803000pt}%
\definecolor{currentstroke}{rgb}{0.150000,0.150000,0.150000}%
\pgfsetstrokecolor{currentstroke}%
\pgfsetdash{}{0pt}%
\pgfsys@defobject{currentmarker}{\pgfqpoint{0.000000in}{0.000000in}}{\pgfqpoint{0.000000in}{0.000000in}}{%
\pgfpathmoveto{\pgfqpoint{0.000000in}{0.000000in}}%
\pgfpathlineto{\pgfqpoint{0.000000in}{0.000000in}}%
\pgfusepath{stroke,fill}%
}%
\begin{pgfscope}%
\pgfsys@transformshift{3.938266in}{0.516222in}%
\pgfsys@useobject{currentmarker}{}%
\end{pgfscope}%
\end{pgfscope}%
\begin{pgfscope}%
\definecolor{textcolor}{rgb}{0.150000,0.150000,0.150000}%
\pgfsetstrokecolor{textcolor}%
\pgfsetfillcolor{textcolor}%
\pgftext[x=3.938266in,y=0.438444in,,top]{\color{textcolor}\sffamily\fontsize{8.000000}{9.600000}\selectfont 4.0}%
\end{pgfscope}%
\begin{pgfscope}%
\pgfpathrectangle{\pgfqpoint{2.816705in}{0.516222in}}{\pgfqpoint{1.962733in}{1.783528in}} %
\pgfusepath{clip}%
\pgfsetroundcap%
\pgfsetroundjoin%
\pgfsetlinewidth{0.803000pt}%
\definecolor{currentstroke}{rgb}{1.000000,1.000000,1.000000}%
\pgfsetstrokecolor{currentstroke}%
\pgfsetdash{}{0pt}%
\pgfpathmoveto{\pgfqpoint{4.218657in}{0.516222in}}%
\pgfpathlineto{\pgfqpoint{4.218657in}{2.299750in}}%
\pgfusepath{stroke}%
\end{pgfscope}%
\begin{pgfscope}%
\pgfsetbuttcap%
\pgfsetroundjoin%
\definecolor{currentfill}{rgb}{0.150000,0.150000,0.150000}%
\pgfsetfillcolor{currentfill}%
\pgfsetlinewidth{0.803000pt}%
\definecolor{currentstroke}{rgb}{0.150000,0.150000,0.150000}%
\pgfsetstrokecolor{currentstroke}%
\pgfsetdash{}{0pt}%
\pgfsys@defobject{currentmarker}{\pgfqpoint{0.000000in}{0.000000in}}{\pgfqpoint{0.000000in}{0.000000in}}{%
\pgfpathmoveto{\pgfqpoint{0.000000in}{0.000000in}}%
\pgfpathlineto{\pgfqpoint{0.000000in}{0.000000in}}%
\pgfusepath{stroke,fill}%
}%
\begin{pgfscope}%
\pgfsys@transformshift{4.218657in}{0.516222in}%
\pgfsys@useobject{currentmarker}{}%
\end{pgfscope}%
\end{pgfscope}%
\begin{pgfscope}%
\definecolor{textcolor}{rgb}{0.150000,0.150000,0.150000}%
\pgfsetstrokecolor{textcolor}%
\pgfsetfillcolor{textcolor}%
\pgftext[x=4.218657in,y=0.438444in,,top]{\color{textcolor}\sffamily\fontsize{8.000000}{9.600000}\selectfont 4.5}%
\end{pgfscope}%
\begin{pgfscope}%
\pgfpathrectangle{\pgfqpoint{2.816705in}{0.516222in}}{\pgfqpoint{1.962733in}{1.783528in}} %
\pgfusepath{clip}%
\pgfsetroundcap%
\pgfsetroundjoin%
\pgfsetlinewidth{0.803000pt}%
\definecolor{currentstroke}{rgb}{1.000000,1.000000,1.000000}%
\pgfsetstrokecolor{currentstroke}%
\pgfsetdash{}{0pt}%
\pgfpathmoveto{\pgfqpoint{4.499047in}{0.516222in}}%
\pgfpathlineto{\pgfqpoint{4.499047in}{2.299750in}}%
\pgfusepath{stroke}%
\end{pgfscope}%
\begin{pgfscope}%
\pgfsetbuttcap%
\pgfsetroundjoin%
\definecolor{currentfill}{rgb}{0.150000,0.150000,0.150000}%
\pgfsetfillcolor{currentfill}%
\pgfsetlinewidth{0.803000pt}%
\definecolor{currentstroke}{rgb}{0.150000,0.150000,0.150000}%
\pgfsetstrokecolor{currentstroke}%
\pgfsetdash{}{0pt}%
\pgfsys@defobject{currentmarker}{\pgfqpoint{0.000000in}{0.000000in}}{\pgfqpoint{0.000000in}{0.000000in}}{%
\pgfpathmoveto{\pgfqpoint{0.000000in}{0.000000in}}%
\pgfpathlineto{\pgfqpoint{0.000000in}{0.000000in}}%
\pgfusepath{stroke,fill}%
}%
\begin{pgfscope}%
\pgfsys@transformshift{4.499047in}{0.516222in}%
\pgfsys@useobject{currentmarker}{}%
\end{pgfscope}%
\end{pgfscope}%
\begin{pgfscope}%
\definecolor{textcolor}{rgb}{0.150000,0.150000,0.150000}%
\pgfsetstrokecolor{textcolor}%
\pgfsetfillcolor{textcolor}%
\pgftext[x=4.499047in,y=0.438444in,,top]{\color{textcolor}\sffamily\fontsize{8.000000}{9.600000}\selectfont 5.0}%
\end{pgfscope}%
\begin{pgfscope}%
\pgfpathrectangle{\pgfqpoint{2.816705in}{0.516222in}}{\pgfqpoint{1.962733in}{1.783528in}} %
\pgfusepath{clip}%
\pgfsetroundcap%
\pgfsetroundjoin%
\pgfsetlinewidth{0.803000pt}%
\definecolor{currentstroke}{rgb}{1.000000,1.000000,1.000000}%
\pgfsetstrokecolor{currentstroke}%
\pgfsetdash{}{0pt}%
\pgfpathmoveto{\pgfqpoint{4.779438in}{0.516222in}}%
\pgfpathlineto{\pgfqpoint{4.779438in}{2.299750in}}%
\pgfusepath{stroke}%
\end{pgfscope}%
\begin{pgfscope}%
\pgfsetbuttcap%
\pgfsetroundjoin%
\definecolor{currentfill}{rgb}{0.150000,0.150000,0.150000}%
\pgfsetfillcolor{currentfill}%
\pgfsetlinewidth{0.803000pt}%
\definecolor{currentstroke}{rgb}{0.150000,0.150000,0.150000}%
\pgfsetstrokecolor{currentstroke}%
\pgfsetdash{}{0pt}%
\pgfsys@defobject{currentmarker}{\pgfqpoint{0.000000in}{0.000000in}}{\pgfqpoint{0.000000in}{0.000000in}}{%
\pgfpathmoveto{\pgfqpoint{0.000000in}{0.000000in}}%
\pgfpathlineto{\pgfqpoint{0.000000in}{0.000000in}}%
\pgfusepath{stroke,fill}%
}%
\begin{pgfscope}%
\pgfsys@transformshift{4.779438in}{0.516222in}%
\pgfsys@useobject{currentmarker}{}%
\end{pgfscope}%
\end{pgfscope}%
\begin{pgfscope}%
\definecolor{textcolor}{rgb}{0.150000,0.150000,0.150000}%
\pgfsetstrokecolor{textcolor}%
\pgfsetfillcolor{textcolor}%
\pgftext[x=4.779438in,y=0.438444in,,top]{\color{textcolor}\sffamily\fontsize{8.000000}{9.600000}\selectfont 5.5}%
\end{pgfscope}%
\begin{pgfscope}%
\definecolor{textcolor}{rgb}{0.150000,0.150000,0.150000}%
\pgfsetstrokecolor{textcolor}%
\pgfsetfillcolor{textcolor}%
\pgftext[x=3.798071in,y=0.273321in,,top]{\color{textcolor}\sffamily\fontsize{8.800000}{10.560000}\selectfont Falling time realization 2}%
\end{pgfscope}%
\begin{pgfscope}%
\pgfpathrectangle{\pgfqpoint{2.816705in}{0.516222in}}{\pgfqpoint{1.962733in}{1.783528in}} %
\pgfusepath{clip}%
\pgfsetroundcap%
\pgfsetroundjoin%
\pgfsetlinewidth{0.803000pt}%
\definecolor{currentstroke}{rgb}{1.000000,1.000000,1.000000}%
\pgfsetstrokecolor{currentstroke}%
\pgfsetdash{}{0pt}%
\pgfpathmoveto{\pgfqpoint{2.816705in}{0.516222in}}%
\pgfpathlineto{\pgfqpoint{4.779438in}{0.516222in}}%
\pgfusepath{stroke}%
\end{pgfscope}%
\begin{pgfscope}%
\pgfsetbuttcap%
\pgfsetroundjoin%
\definecolor{currentfill}{rgb}{0.150000,0.150000,0.150000}%
\pgfsetfillcolor{currentfill}%
\pgfsetlinewidth{0.803000pt}%
\definecolor{currentstroke}{rgb}{0.150000,0.150000,0.150000}%
\pgfsetstrokecolor{currentstroke}%
\pgfsetdash{}{0pt}%
\pgfsys@defobject{currentmarker}{\pgfqpoint{0.000000in}{0.000000in}}{\pgfqpoint{0.000000in}{0.000000in}}{%
\pgfpathmoveto{\pgfqpoint{0.000000in}{0.000000in}}%
\pgfpathlineto{\pgfqpoint{0.000000in}{0.000000in}}%
\pgfusepath{stroke,fill}%
}%
\begin{pgfscope}%
\pgfsys@transformshift{2.816705in}{0.516222in}%
\pgfsys@useobject{currentmarker}{}%
\end{pgfscope}%
\end{pgfscope}%
\begin{pgfscope}%
\pgfpathrectangle{\pgfqpoint{2.816705in}{0.516222in}}{\pgfqpoint{1.962733in}{1.783528in}} %
\pgfusepath{clip}%
\pgfsetroundcap%
\pgfsetroundjoin%
\pgfsetlinewidth{0.803000pt}%
\definecolor{currentstroke}{rgb}{1.000000,1.000000,1.000000}%
\pgfsetstrokecolor{currentstroke}%
\pgfsetdash{}{0pt}%
\pgfpathmoveto{\pgfqpoint{2.816705in}{0.813477in}}%
\pgfpathlineto{\pgfqpoint{4.779438in}{0.813477in}}%
\pgfusepath{stroke}%
\end{pgfscope}%
\begin{pgfscope}%
\pgfsetbuttcap%
\pgfsetroundjoin%
\definecolor{currentfill}{rgb}{0.150000,0.150000,0.150000}%
\pgfsetfillcolor{currentfill}%
\pgfsetlinewidth{0.803000pt}%
\definecolor{currentstroke}{rgb}{0.150000,0.150000,0.150000}%
\pgfsetstrokecolor{currentstroke}%
\pgfsetdash{}{0pt}%
\pgfsys@defobject{currentmarker}{\pgfqpoint{0.000000in}{0.000000in}}{\pgfqpoint{0.000000in}{0.000000in}}{%
\pgfpathmoveto{\pgfqpoint{0.000000in}{0.000000in}}%
\pgfpathlineto{\pgfqpoint{0.000000in}{0.000000in}}%
\pgfusepath{stroke,fill}%
}%
\begin{pgfscope}%
\pgfsys@transformshift{2.816705in}{0.813477in}%
\pgfsys@useobject{currentmarker}{}%
\end{pgfscope}%
\end{pgfscope}%
\begin{pgfscope}%
\pgfpathrectangle{\pgfqpoint{2.816705in}{0.516222in}}{\pgfqpoint{1.962733in}{1.783528in}} %
\pgfusepath{clip}%
\pgfsetroundcap%
\pgfsetroundjoin%
\pgfsetlinewidth{0.803000pt}%
\definecolor{currentstroke}{rgb}{1.000000,1.000000,1.000000}%
\pgfsetstrokecolor{currentstroke}%
\pgfsetdash{}{0pt}%
\pgfpathmoveto{\pgfqpoint{2.816705in}{1.110731in}}%
\pgfpathlineto{\pgfqpoint{4.779438in}{1.110731in}}%
\pgfusepath{stroke}%
\end{pgfscope}%
\begin{pgfscope}%
\pgfsetbuttcap%
\pgfsetroundjoin%
\definecolor{currentfill}{rgb}{0.150000,0.150000,0.150000}%
\pgfsetfillcolor{currentfill}%
\pgfsetlinewidth{0.803000pt}%
\definecolor{currentstroke}{rgb}{0.150000,0.150000,0.150000}%
\pgfsetstrokecolor{currentstroke}%
\pgfsetdash{}{0pt}%
\pgfsys@defobject{currentmarker}{\pgfqpoint{0.000000in}{0.000000in}}{\pgfqpoint{0.000000in}{0.000000in}}{%
\pgfpathmoveto{\pgfqpoint{0.000000in}{0.000000in}}%
\pgfpathlineto{\pgfqpoint{0.000000in}{0.000000in}}%
\pgfusepath{stroke,fill}%
}%
\begin{pgfscope}%
\pgfsys@transformshift{2.816705in}{1.110731in}%
\pgfsys@useobject{currentmarker}{}%
\end{pgfscope}%
\end{pgfscope}%
\begin{pgfscope}%
\pgfpathrectangle{\pgfqpoint{2.816705in}{0.516222in}}{\pgfqpoint{1.962733in}{1.783528in}} %
\pgfusepath{clip}%
\pgfsetroundcap%
\pgfsetroundjoin%
\pgfsetlinewidth{0.803000pt}%
\definecolor{currentstroke}{rgb}{1.000000,1.000000,1.000000}%
\pgfsetstrokecolor{currentstroke}%
\pgfsetdash{}{0pt}%
\pgfpathmoveto{\pgfqpoint{2.816705in}{1.407986in}}%
\pgfpathlineto{\pgfqpoint{4.779438in}{1.407986in}}%
\pgfusepath{stroke}%
\end{pgfscope}%
\begin{pgfscope}%
\pgfsetbuttcap%
\pgfsetroundjoin%
\definecolor{currentfill}{rgb}{0.150000,0.150000,0.150000}%
\pgfsetfillcolor{currentfill}%
\pgfsetlinewidth{0.803000pt}%
\definecolor{currentstroke}{rgb}{0.150000,0.150000,0.150000}%
\pgfsetstrokecolor{currentstroke}%
\pgfsetdash{}{0pt}%
\pgfsys@defobject{currentmarker}{\pgfqpoint{0.000000in}{0.000000in}}{\pgfqpoint{0.000000in}{0.000000in}}{%
\pgfpathmoveto{\pgfqpoint{0.000000in}{0.000000in}}%
\pgfpathlineto{\pgfqpoint{0.000000in}{0.000000in}}%
\pgfusepath{stroke,fill}%
}%
\begin{pgfscope}%
\pgfsys@transformshift{2.816705in}{1.407986in}%
\pgfsys@useobject{currentmarker}{}%
\end{pgfscope}%
\end{pgfscope}%
\begin{pgfscope}%
\pgfpathrectangle{\pgfqpoint{2.816705in}{0.516222in}}{\pgfqpoint{1.962733in}{1.783528in}} %
\pgfusepath{clip}%
\pgfsetroundcap%
\pgfsetroundjoin%
\pgfsetlinewidth{0.803000pt}%
\definecolor{currentstroke}{rgb}{1.000000,1.000000,1.000000}%
\pgfsetstrokecolor{currentstroke}%
\pgfsetdash{}{0pt}%
\pgfpathmoveto{\pgfqpoint{2.816705in}{1.705241in}}%
\pgfpathlineto{\pgfqpoint{4.779438in}{1.705241in}}%
\pgfusepath{stroke}%
\end{pgfscope}%
\begin{pgfscope}%
\pgfsetbuttcap%
\pgfsetroundjoin%
\definecolor{currentfill}{rgb}{0.150000,0.150000,0.150000}%
\pgfsetfillcolor{currentfill}%
\pgfsetlinewidth{0.803000pt}%
\definecolor{currentstroke}{rgb}{0.150000,0.150000,0.150000}%
\pgfsetstrokecolor{currentstroke}%
\pgfsetdash{}{0pt}%
\pgfsys@defobject{currentmarker}{\pgfqpoint{0.000000in}{0.000000in}}{\pgfqpoint{0.000000in}{0.000000in}}{%
\pgfpathmoveto{\pgfqpoint{0.000000in}{0.000000in}}%
\pgfpathlineto{\pgfqpoint{0.000000in}{0.000000in}}%
\pgfusepath{stroke,fill}%
}%
\begin{pgfscope}%
\pgfsys@transformshift{2.816705in}{1.705241in}%
\pgfsys@useobject{currentmarker}{}%
\end{pgfscope}%
\end{pgfscope}%
\begin{pgfscope}%
\pgfpathrectangle{\pgfqpoint{2.816705in}{0.516222in}}{\pgfqpoint{1.962733in}{1.783528in}} %
\pgfusepath{clip}%
\pgfsetroundcap%
\pgfsetroundjoin%
\pgfsetlinewidth{0.803000pt}%
\definecolor{currentstroke}{rgb}{1.000000,1.000000,1.000000}%
\pgfsetstrokecolor{currentstroke}%
\pgfsetdash{}{0pt}%
\pgfpathmoveto{\pgfqpoint{2.816705in}{2.002495in}}%
\pgfpathlineto{\pgfqpoint{4.779438in}{2.002495in}}%
\pgfusepath{stroke}%
\end{pgfscope}%
\begin{pgfscope}%
\pgfsetbuttcap%
\pgfsetroundjoin%
\definecolor{currentfill}{rgb}{0.150000,0.150000,0.150000}%
\pgfsetfillcolor{currentfill}%
\pgfsetlinewidth{0.803000pt}%
\definecolor{currentstroke}{rgb}{0.150000,0.150000,0.150000}%
\pgfsetstrokecolor{currentstroke}%
\pgfsetdash{}{0pt}%
\pgfsys@defobject{currentmarker}{\pgfqpoint{0.000000in}{0.000000in}}{\pgfqpoint{0.000000in}{0.000000in}}{%
\pgfpathmoveto{\pgfqpoint{0.000000in}{0.000000in}}%
\pgfpathlineto{\pgfqpoint{0.000000in}{0.000000in}}%
\pgfusepath{stroke,fill}%
}%
\begin{pgfscope}%
\pgfsys@transformshift{2.816705in}{2.002495in}%
\pgfsys@useobject{currentmarker}{}%
\end{pgfscope}%
\end{pgfscope}%
\begin{pgfscope}%
\pgfpathrectangle{\pgfqpoint{2.816705in}{0.516222in}}{\pgfqpoint{1.962733in}{1.783528in}} %
\pgfusepath{clip}%
\pgfsetroundcap%
\pgfsetroundjoin%
\pgfsetlinewidth{0.803000pt}%
\definecolor{currentstroke}{rgb}{1.000000,1.000000,1.000000}%
\pgfsetstrokecolor{currentstroke}%
\pgfsetdash{}{0pt}%
\pgfpathmoveto{\pgfqpoint{2.816705in}{2.299750in}}%
\pgfpathlineto{\pgfqpoint{4.779438in}{2.299750in}}%
\pgfusepath{stroke}%
\end{pgfscope}%
\begin{pgfscope}%
\pgfsetbuttcap%
\pgfsetroundjoin%
\definecolor{currentfill}{rgb}{0.150000,0.150000,0.150000}%
\pgfsetfillcolor{currentfill}%
\pgfsetlinewidth{0.803000pt}%
\definecolor{currentstroke}{rgb}{0.150000,0.150000,0.150000}%
\pgfsetstrokecolor{currentstroke}%
\pgfsetdash{}{0pt}%
\pgfsys@defobject{currentmarker}{\pgfqpoint{0.000000in}{0.000000in}}{\pgfqpoint{0.000000in}{0.000000in}}{%
\pgfpathmoveto{\pgfqpoint{0.000000in}{0.000000in}}%
\pgfpathlineto{\pgfqpoint{0.000000in}{0.000000in}}%
\pgfusepath{stroke,fill}%
}%
\begin{pgfscope}%
\pgfsys@transformshift{2.816705in}{2.299750in}%
\pgfsys@useobject{currentmarker}{}%
\end{pgfscope}%
\end{pgfscope}%
\begin{pgfscope}%
\pgfpathrectangle{\pgfqpoint{2.816705in}{0.516222in}}{\pgfqpoint{1.962733in}{1.783528in}} %
\pgfusepath{clip}%
\pgfsetbuttcap%
\pgfsetroundjoin%
\definecolor{currentfill}{rgb}{0.298039,0.447059,0.690196}%
\pgfsetfillcolor{currentfill}%
\pgfsetlinewidth{0.240900pt}%
\definecolor{currentstroke}{rgb}{1.000000,1.000000,1.000000}%
\pgfsetstrokecolor{currentstroke}%
\pgfsetdash{}{0pt}%
\pgfpathmoveto{\pgfqpoint{3.826110in}{1.287753in}}%
\pgfpathcurveto{\pgfqpoint{3.834346in}{1.287753in}}{\pgfqpoint{3.842247in}{1.291026in}}{\pgfqpoint{3.848070in}{1.296849in}}%
\pgfpathcurveto{\pgfqpoint{3.853894in}{1.302673in}}{\pgfqpoint{3.857167in}{1.310573in}}{\pgfqpoint{3.857167in}{1.318810in}}%
\pgfpathcurveto{\pgfqpoint{3.857167in}{1.327046in}}{\pgfqpoint{3.853894in}{1.334946in}}{\pgfqpoint{3.848070in}{1.340770in}}%
\pgfpathcurveto{\pgfqpoint{3.842247in}{1.346594in}}{\pgfqpoint{3.834346in}{1.349866in}}{\pgfqpoint{3.826110in}{1.349866in}}%
\pgfpathcurveto{\pgfqpoint{3.817874in}{1.349866in}}{\pgfqpoint{3.809974in}{1.346594in}}{\pgfqpoint{3.804150in}{1.340770in}}%
\pgfpathcurveto{\pgfqpoint{3.798326in}{1.334946in}}{\pgfqpoint{3.795054in}{1.327046in}}{\pgfqpoint{3.795054in}{1.318810in}}%
\pgfpathcurveto{\pgfqpoint{3.795054in}{1.310573in}}{\pgfqpoint{3.798326in}{1.302673in}}{\pgfqpoint{3.804150in}{1.296849in}}%
\pgfpathcurveto{\pgfqpoint{3.809974in}{1.291026in}}{\pgfqpoint{3.817874in}{1.287753in}}{\pgfqpoint{3.826110in}{1.287753in}}%
\pgfpathclose%
\pgfusepath{stroke,fill}%
\end{pgfscope}%
\begin{pgfscope}%
\pgfpathrectangle{\pgfqpoint{2.816705in}{0.516222in}}{\pgfqpoint{1.962733in}{1.783528in}} %
\pgfusepath{clip}%
\pgfsetbuttcap%
\pgfsetroundjoin%
\definecolor{currentfill}{rgb}{0.298039,0.447059,0.690196}%
\pgfsetfillcolor{currentfill}%
\pgfsetlinewidth{0.240900pt}%
\definecolor{currentstroke}{rgb}{1.000000,1.000000,1.000000}%
\pgfsetstrokecolor{currentstroke}%
\pgfsetdash{}{0pt}%
\pgfpathmoveto{\pgfqpoint{3.657876in}{1.763361in}}%
\pgfpathcurveto{\pgfqpoint{3.666112in}{1.763361in}}{\pgfqpoint{3.674012in}{1.766633in}}{\pgfqpoint{3.679836in}{1.772457in}}%
\pgfpathcurveto{\pgfqpoint{3.685660in}{1.778281in}}{\pgfqpoint{3.688932in}{1.786181in}}{\pgfqpoint{3.688932in}{1.794417in}}%
\pgfpathcurveto{\pgfqpoint{3.688932in}{1.802653in}}{\pgfqpoint{3.685660in}{1.810553in}}{\pgfqpoint{3.679836in}{1.816377in}}%
\pgfpathcurveto{\pgfqpoint{3.674012in}{1.822201in}}{\pgfqpoint{3.666112in}{1.825474in}}{\pgfqpoint{3.657876in}{1.825474in}}%
\pgfpathcurveto{\pgfqpoint{3.649640in}{1.825474in}}{\pgfqpoint{3.641740in}{1.822201in}}{\pgfqpoint{3.635916in}{1.816377in}}%
\pgfpathcurveto{\pgfqpoint{3.630092in}{1.810553in}}{\pgfqpoint{3.626819in}{1.802653in}}{\pgfqpoint{3.626819in}{1.794417in}}%
\pgfpathcurveto{\pgfqpoint{3.626819in}{1.786181in}}{\pgfqpoint{3.630092in}{1.778281in}}{\pgfqpoint{3.635916in}{1.772457in}}%
\pgfpathcurveto{\pgfqpoint{3.641740in}{1.766633in}}{\pgfqpoint{3.649640in}{1.763361in}}{\pgfqpoint{3.657876in}{1.763361in}}%
\pgfpathclose%
\pgfusepath{stroke,fill}%
\end{pgfscope}%
\begin{pgfscope}%
\pgfpathrectangle{\pgfqpoint{2.816705in}{0.516222in}}{\pgfqpoint{1.962733in}{1.783528in}} %
\pgfusepath{clip}%
\pgfsetbuttcap%
\pgfsetroundjoin%
\definecolor{currentfill}{rgb}{0.298039,0.447059,0.690196}%
\pgfsetfillcolor{currentfill}%
\pgfsetlinewidth{0.240900pt}%
\definecolor{currentstroke}{rgb}{1.000000,1.000000,1.000000}%
\pgfsetstrokecolor{currentstroke}%
\pgfsetdash{}{0pt}%
\pgfpathmoveto{\pgfqpoint{3.713954in}{1.436381in}}%
\pgfpathcurveto{\pgfqpoint{3.722190in}{1.436381in}}{\pgfqpoint{3.730090in}{1.439653in}}{\pgfqpoint{3.735914in}{1.445477in}}%
\pgfpathcurveto{\pgfqpoint{3.741738in}{1.451301in}}{\pgfqpoint{3.745011in}{1.459201in}}{\pgfqpoint{3.745011in}{1.467437in}}%
\pgfpathcurveto{\pgfqpoint{3.745011in}{1.475673in}}{\pgfqpoint{3.741738in}{1.483573in}}{\pgfqpoint{3.735914in}{1.489397in}}%
\pgfpathcurveto{\pgfqpoint{3.730090in}{1.495221in}}{\pgfqpoint{3.722190in}{1.498494in}}{\pgfqpoint{3.713954in}{1.498494in}}%
\pgfpathcurveto{\pgfqpoint{3.705718in}{1.498494in}}{\pgfqpoint{3.697818in}{1.495221in}}{\pgfqpoint{3.691994in}{1.489397in}}%
\pgfpathcurveto{\pgfqpoint{3.686170in}{1.483573in}}{\pgfqpoint{3.682898in}{1.475673in}}{\pgfqpoint{3.682898in}{1.467437in}}%
\pgfpathcurveto{\pgfqpoint{3.682898in}{1.459201in}}{\pgfqpoint{3.686170in}{1.451301in}}{\pgfqpoint{3.691994in}{1.445477in}}%
\pgfpathcurveto{\pgfqpoint{3.697818in}{1.439653in}}{\pgfqpoint{3.705718in}{1.436381in}}{\pgfqpoint{3.713954in}{1.436381in}}%
\pgfpathclose%
\pgfusepath{stroke,fill}%
\end{pgfscope}%
\begin{pgfscope}%
\pgfpathrectangle{\pgfqpoint{2.816705in}{0.516222in}}{\pgfqpoint{1.962733in}{1.783528in}} %
\pgfusepath{clip}%
\pgfsetbuttcap%
\pgfsetroundjoin%
\definecolor{currentfill}{rgb}{0.298039,0.447059,0.690196}%
\pgfsetfillcolor{currentfill}%
\pgfsetlinewidth{0.240900pt}%
\definecolor{currentstroke}{rgb}{1.000000,1.000000,1.000000}%
\pgfsetstrokecolor{currentstroke}%
\pgfsetdash{}{0pt}%
\pgfpathmoveto{\pgfqpoint{4.162579in}{1.822812in}}%
\pgfpathcurveto{\pgfqpoint{4.170815in}{1.822812in}}{\pgfqpoint{4.178715in}{1.826084in}}{\pgfqpoint{4.184539in}{1.831908in}}%
\pgfpathcurveto{\pgfqpoint{4.190363in}{1.837732in}}{\pgfqpoint{4.193635in}{1.845632in}}{\pgfqpoint{4.193635in}{1.853868in}}%
\pgfpathcurveto{\pgfqpoint{4.193635in}{1.862104in}}{\pgfqpoint{4.190363in}{1.870004in}}{\pgfqpoint{4.184539in}{1.875828in}}%
\pgfpathcurveto{\pgfqpoint{4.178715in}{1.881652in}}{\pgfqpoint{4.170815in}{1.884925in}}{\pgfqpoint{4.162579in}{1.884925in}}%
\pgfpathcurveto{\pgfqpoint{4.154342in}{1.884925in}}{\pgfqpoint{4.146442in}{1.881652in}}{\pgfqpoint{4.140618in}{1.875828in}}%
\pgfpathcurveto{\pgfqpoint{4.134794in}{1.870004in}}{\pgfqpoint{4.131522in}{1.862104in}}{\pgfqpoint{4.131522in}{1.853868in}}%
\pgfpathcurveto{\pgfqpoint{4.131522in}{1.845632in}}{\pgfqpoint{4.134794in}{1.837732in}}{\pgfqpoint{4.140618in}{1.831908in}}%
\pgfpathcurveto{\pgfqpoint{4.146442in}{1.826084in}}{\pgfqpoint{4.154342in}{1.822812in}}{\pgfqpoint{4.162579in}{1.822812in}}%
\pgfpathclose%
\pgfusepath{stroke,fill}%
\end{pgfscope}%
\begin{pgfscope}%
\pgfpathrectangle{\pgfqpoint{2.816705in}{0.516222in}}{\pgfqpoint{1.962733in}{1.783528in}} %
\pgfusepath{clip}%
\pgfsetbuttcap%
\pgfsetroundjoin%
\definecolor{currentfill}{rgb}{0.298039,0.447059,0.690196}%
\pgfsetfillcolor{currentfill}%
\pgfsetlinewidth{0.240900pt}%
\definecolor{currentstroke}{rgb}{1.000000,1.000000,1.000000}%
\pgfsetstrokecolor{currentstroke}%
\pgfsetdash{}{0pt}%
\pgfpathmoveto{\pgfqpoint{3.601798in}{1.733635in}}%
\pgfpathcurveto{\pgfqpoint{3.610034in}{1.733635in}}{\pgfqpoint{3.617934in}{1.736907in}}{\pgfqpoint{3.623758in}{1.742731in}}%
\pgfpathcurveto{\pgfqpoint{3.629582in}{1.748555in}}{\pgfqpoint{3.632854in}{1.756455in}}{\pgfqpoint{3.632854in}{1.764692in}}%
\pgfpathcurveto{\pgfqpoint{3.632854in}{1.772928in}}{\pgfqpoint{3.629582in}{1.780828in}}{\pgfqpoint{3.623758in}{1.786652in}}%
\pgfpathcurveto{\pgfqpoint{3.617934in}{1.792476in}}{\pgfqpoint{3.610034in}{1.795748in}}{\pgfqpoint{3.601798in}{1.795748in}}%
\pgfpathcurveto{\pgfqpoint{3.593562in}{1.795748in}}{\pgfqpoint{3.585662in}{1.792476in}}{\pgfqpoint{3.579838in}{1.786652in}}%
\pgfpathcurveto{\pgfqpoint{3.574014in}{1.780828in}}{\pgfqpoint{3.570741in}{1.772928in}}{\pgfqpoint{3.570741in}{1.764692in}}%
\pgfpathcurveto{\pgfqpoint{3.570741in}{1.756455in}}{\pgfqpoint{3.574014in}{1.748555in}}{\pgfqpoint{3.579838in}{1.742731in}}%
\pgfpathcurveto{\pgfqpoint{3.585662in}{1.736907in}}{\pgfqpoint{3.593562in}{1.733635in}}{\pgfqpoint{3.601798in}{1.733635in}}%
\pgfpathclose%
\pgfusepath{stroke,fill}%
\end{pgfscope}%
\begin{pgfscope}%
\pgfpathrectangle{\pgfqpoint{2.816705in}{0.516222in}}{\pgfqpoint{1.962733in}{1.783528in}} %
\pgfusepath{clip}%
\pgfsetbuttcap%
\pgfsetroundjoin%
\definecolor{currentfill}{rgb}{0.298039,0.447059,0.690196}%
\pgfsetfillcolor{currentfill}%
\pgfsetlinewidth{0.240900pt}%
\definecolor{currentstroke}{rgb}{1.000000,1.000000,1.000000}%
\pgfsetstrokecolor{currentstroke}%
\pgfsetdash{}{0pt}%
\pgfpathmoveto{\pgfqpoint{4.218657in}{1.911988in}}%
\pgfpathcurveto{\pgfqpoint{4.226893in}{1.911988in}}{\pgfqpoint{4.234793in}{1.915260in}}{\pgfqpoint{4.240617in}{1.921084in}}%
\pgfpathcurveto{\pgfqpoint{4.246441in}{1.926908in}}{\pgfqpoint{4.249713in}{1.934808in}}{\pgfqpoint{4.249713in}{1.943044in}}%
\pgfpathcurveto{\pgfqpoint{4.249713in}{1.951281in}}{\pgfqpoint{4.246441in}{1.959181in}}{\pgfqpoint{4.240617in}{1.965005in}}%
\pgfpathcurveto{\pgfqpoint{4.234793in}{1.970829in}}{\pgfqpoint{4.226893in}{1.974101in}}{\pgfqpoint{4.218657in}{1.974101in}}%
\pgfpathcurveto{\pgfqpoint{4.210420in}{1.974101in}}{\pgfqpoint{4.202520in}{1.970829in}}{\pgfqpoint{4.196696in}{1.965005in}}%
\pgfpathcurveto{\pgfqpoint{4.190873in}{1.959181in}}{\pgfqpoint{4.187600in}{1.951281in}}{\pgfqpoint{4.187600in}{1.943044in}}%
\pgfpathcurveto{\pgfqpoint{4.187600in}{1.934808in}}{\pgfqpoint{4.190873in}{1.926908in}}{\pgfqpoint{4.196696in}{1.921084in}}%
\pgfpathcurveto{\pgfqpoint{4.202520in}{1.915260in}}{\pgfqpoint{4.210420in}{1.911988in}}{\pgfqpoint{4.218657in}{1.911988in}}%
\pgfpathclose%
\pgfusepath{stroke,fill}%
\end{pgfscope}%
\begin{pgfscope}%
\pgfpathrectangle{\pgfqpoint{2.816705in}{0.516222in}}{\pgfqpoint{1.962733in}{1.783528in}} %
\pgfusepath{clip}%
\pgfsetbuttcap%
\pgfsetroundjoin%
\definecolor{currentfill}{rgb}{0.298039,0.447059,0.690196}%
\pgfsetfillcolor{currentfill}%
\pgfsetlinewidth{0.240900pt}%
\definecolor{currentstroke}{rgb}{1.000000,1.000000,1.000000}%
\pgfsetstrokecolor{currentstroke}%
\pgfsetdash{}{0pt}%
\pgfpathmoveto{\pgfqpoint{3.601798in}{1.168851in}}%
\pgfpathcurveto{\pgfqpoint{3.610034in}{1.168851in}}{\pgfqpoint{3.617934in}{1.172124in}}{\pgfqpoint{3.623758in}{1.177948in}}%
\pgfpathcurveto{\pgfqpoint{3.629582in}{1.183772in}}{\pgfqpoint{3.632854in}{1.191672in}}{\pgfqpoint{3.632854in}{1.199908in}}%
\pgfpathcurveto{\pgfqpoint{3.632854in}{1.208144in}}{\pgfqpoint{3.629582in}{1.216044in}}{\pgfqpoint{3.623758in}{1.221868in}}%
\pgfpathcurveto{\pgfqpoint{3.617934in}{1.227692in}}{\pgfqpoint{3.610034in}{1.230964in}}{\pgfqpoint{3.601798in}{1.230964in}}%
\pgfpathcurveto{\pgfqpoint{3.593562in}{1.230964in}}{\pgfqpoint{3.585662in}{1.227692in}}{\pgfqpoint{3.579838in}{1.221868in}}%
\pgfpathcurveto{\pgfqpoint{3.574014in}{1.216044in}}{\pgfqpoint{3.570741in}{1.208144in}}{\pgfqpoint{3.570741in}{1.199908in}}%
\pgfpathcurveto{\pgfqpoint{3.570741in}{1.191672in}}{\pgfqpoint{3.574014in}{1.183772in}}{\pgfqpoint{3.579838in}{1.177948in}}%
\pgfpathcurveto{\pgfqpoint{3.585662in}{1.172124in}}{\pgfqpoint{3.593562in}{1.168851in}}{\pgfqpoint{3.601798in}{1.168851in}}%
\pgfpathclose%
\pgfusepath{stroke,fill}%
\end{pgfscope}%
\begin{pgfscope}%
\pgfpathrectangle{\pgfqpoint{2.816705in}{0.516222in}}{\pgfqpoint{1.962733in}{1.783528in}} %
\pgfusepath{clip}%
\pgfsetbuttcap%
\pgfsetroundjoin%
\definecolor{currentfill}{rgb}{0.298039,0.447059,0.690196}%
\pgfsetfillcolor{currentfill}%
\pgfsetlinewidth{0.240900pt}%
\definecolor{currentstroke}{rgb}{1.000000,1.000000,1.000000}%
\pgfsetstrokecolor{currentstroke}%
\pgfsetdash{}{0pt}%
\pgfpathmoveto{\pgfqpoint{4.442969in}{1.674184in}}%
\pgfpathcurveto{\pgfqpoint{4.451205in}{1.674184in}}{\pgfqpoint{4.459105in}{1.677457in}}{\pgfqpoint{4.464929in}{1.683280in}}%
\pgfpathcurveto{\pgfqpoint{4.470753in}{1.689104in}}{\pgfqpoint{4.474026in}{1.697004in}}{\pgfqpoint{4.474026in}{1.705241in}}%
\pgfpathcurveto{\pgfqpoint{4.474026in}{1.713477in}}{\pgfqpoint{4.470753in}{1.721377in}}{\pgfqpoint{4.464929in}{1.727201in}}%
\pgfpathcurveto{\pgfqpoint{4.459105in}{1.733025in}}{\pgfqpoint{4.451205in}{1.736297in}}{\pgfqpoint{4.442969in}{1.736297in}}%
\pgfpathcurveto{\pgfqpoint{4.434733in}{1.736297in}}{\pgfqpoint{4.426833in}{1.733025in}}{\pgfqpoint{4.421009in}{1.727201in}}%
\pgfpathcurveto{\pgfqpoint{4.415185in}{1.721377in}}{\pgfqpoint{4.411913in}{1.713477in}}{\pgfqpoint{4.411913in}{1.705241in}}%
\pgfpathcurveto{\pgfqpoint{4.411913in}{1.697004in}}{\pgfqpoint{4.415185in}{1.689104in}}{\pgfqpoint{4.421009in}{1.683280in}}%
\pgfpathcurveto{\pgfqpoint{4.426833in}{1.677457in}}{\pgfqpoint{4.434733in}{1.674184in}}{\pgfqpoint{4.442969in}{1.674184in}}%
\pgfpathclose%
\pgfusepath{stroke,fill}%
\end{pgfscope}%
\begin{pgfscope}%
\pgfpathrectangle{\pgfqpoint{2.816705in}{0.516222in}}{\pgfqpoint{1.962733in}{1.783528in}} %
\pgfusepath{clip}%
\pgfsetbuttcap%
\pgfsetroundjoin%
\definecolor{currentfill}{rgb}{0.298039,0.447059,0.690196}%
\pgfsetfillcolor{currentfill}%
\pgfsetlinewidth{0.240900pt}%
\definecolor{currentstroke}{rgb}{1.000000,1.000000,1.000000}%
\pgfsetstrokecolor{currentstroke}%
\pgfsetdash{}{0pt}%
\pgfpathmoveto{\pgfqpoint{3.097095in}{1.585008in}}%
\pgfpathcurveto{\pgfqpoint{3.105332in}{1.585008in}}{\pgfqpoint{3.113232in}{1.588280in}}{\pgfqpoint{3.119055in}{1.594104in}}%
\pgfpathcurveto{\pgfqpoint{3.124879in}{1.599928in}}{\pgfqpoint{3.128152in}{1.607828in}}{\pgfqpoint{3.128152in}{1.616064in}}%
\pgfpathcurveto{\pgfqpoint{3.128152in}{1.624301in}}{\pgfqpoint{3.124879in}{1.632201in}}{\pgfqpoint{3.119055in}{1.638025in}}%
\pgfpathcurveto{\pgfqpoint{3.113232in}{1.643849in}}{\pgfqpoint{3.105332in}{1.647121in}}{\pgfqpoint{3.097095in}{1.647121in}}%
\pgfpathcurveto{\pgfqpoint{3.088859in}{1.647121in}}{\pgfqpoint{3.080959in}{1.643849in}}{\pgfqpoint{3.075135in}{1.638025in}}%
\pgfpathcurveto{\pgfqpoint{3.069311in}{1.632201in}}{\pgfqpoint{3.066039in}{1.624301in}}{\pgfqpoint{3.066039in}{1.616064in}}%
\pgfpathcurveto{\pgfqpoint{3.066039in}{1.607828in}}{\pgfqpoint{3.069311in}{1.599928in}}{\pgfqpoint{3.075135in}{1.594104in}}%
\pgfpathcurveto{\pgfqpoint{3.080959in}{1.588280in}}{\pgfqpoint{3.088859in}{1.585008in}}{\pgfqpoint{3.097095in}{1.585008in}}%
\pgfpathclose%
\pgfusepath{stroke,fill}%
\end{pgfscope}%
\begin{pgfscope}%
\pgfpathrectangle{\pgfqpoint{2.816705in}{0.516222in}}{\pgfqpoint{1.962733in}{1.783528in}} %
\pgfusepath{clip}%
\pgfsetbuttcap%
\pgfsetroundjoin%
\definecolor{currentfill}{rgb}{0.298039,0.447059,0.690196}%
\pgfsetfillcolor{currentfill}%
\pgfsetlinewidth{0.240900pt}%
\definecolor{currentstroke}{rgb}{1.000000,1.000000,1.000000}%
\pgfsetstrokecolor{currentstroke}%
\pgfsetdash{}{0pt}%
\pgfpathmoveto{\pgfqpoint{3.657876in}{1.436381in}}%
\pgfpathcurveto{\pgfqpoint{3.666112in}{1.436381in}}{\pgfqpoint{3.674012in}{1.439653in}}{\pgfqpoint{3.679836in}{1.445477in}}%
\pgfpathcurveto{\pgfqpoint{3.685660in}{1.451301in}}{\pgfqpoint{3.688932in}{1.459201in}}{\pgfqpoint{3.688932in}{1.467437in}}%
\pgfpathcurveto{\pgfqpoint{3.688932in}{1.475673in}}{\pgfqpoint{3.685660in}{1.483573in}}{\pgfqpoint{3.679836in}{1.489397in}}%
\pgfpathcurveto{\pgfqpoint{3.674012in}{1.495221in}}{\pgfqpoint{3.666112in}{1.498494in}}{\pgfqpoint{3.657876in}{1.498494in}}%
\pgfpathcurveto{\pgfqpoint{3.649640in}{1.498494in}}{\pgfqpoint{3.641740in}{1.495221in}}{\pgfqpoint{3.635916in}{1.489397in}}%
\pgfpathcurveto{\pgfqpoint{3.630092in}{1.483573in}}{\pgfqpoint{3.626819in}{1.475673in}}{\pgfqpoint{3.626819in}{1.467437in}}%
\pgfpathcurveto{\pgfqpoint{3.626819in}{1.459201in}}{\pgfqpoint{3.630092in}{1.451301in}}{\pgfqpoint{3.635916in}{1.445477in}}%
\pgfpathcurveto{\pgfqpoint{3.641740in}{1.439653in}}{\pgfqpoint{3.649640in}{1.436381in}}{\pgfqpoint{3.657876in}{1.436381in}}%
\pgfpathclose%
\pgfusepath{stroke,fill}%
\end{pgfscope}%
\begin{pgfscope}%
\pgfpathrectangle{\pgfqpoint{2.816705in}{0.516222in}}{\pgfqpoint{1.962733in}{1.783528in}} %
\pgfusepath{clip}%
\pgfsetbuttcap%
\pgfsetroundjoin%
\definecolor{currentfill}{rgb}{0.298039,0.447059,0.690196}%
\pgfsetfillcolor{currentfill}%
\pgfsetlinewidth{0.240900pt}%
\definecolor{currentstroke}{rgb}{1.000000,1.000000,1.000000}%
\pgfsetstrokecolor{currentstroke}%
\pgfsetdash{}{0pt}%
\pgfpathmoveto{\pgfqpoint{3.657876in}{0.633793in}}%
\pgfpathcurveto{\pgfqpoint{3.666112in}{0.633793in}}{\pgfqpoint{3.674012in}{0.637065in}}{\pgfqpoint{3.679836in}{0.642889in}}%
\pgfpathcurveto{\pgfqpoint{3.685660in}{0.648713in}}{\pgfqpoint{3.688932in}{0.656613in}}{\pgfqpoint{3.688932in}{0.664850in}}%
\pgfpathcurveto{\pgfqpoint{3.688932in}{0.673086in}}{\pgfqpoint{3.685660in}{0.680986in}}{\pgfqpoint{3.679836in}{0.686810in}}%
\pgfpathcurveto{\pgfqpoint{3.674012in}{0.692634in}}{\pgfqpoint{3.666112in}{0.695906in}}{\pgfqpoint{3.657876in}{0.695906in}}%
\pgfpathcurveto{\pgfqpoint{3.649640in}{0.695906in}}{\pgfqpoint{3.641740in}{0.692634in}}{\pgfqpoint{3.635916in}{0.686810in}}%
\pgfpathcurveto{\pgfqpoint{3.630092in}{0.680986in}}{\pgfqpoint{3.626819in}{0.673086in}}{\pgfqpoint{3.626819in}{0.664850in}}%
\pgfpathcurveto{\pgfqpoint{3.626819in}{0.656613in}}{\pgfqpoint{3.630092in}{0.648713in}}{\pgfqpoint{3.635916in}{0.642889in}}%
\pgfpathcurveto{\pgfqpoint{3.641740in}{0.637065in}}{\pgfqpoint{3.649640in}{0.633793in}}{\pgfqpoint{3.657876in}{0.633793in}}%
\pgfpathclose%
\pgfusepath{stroke,fill}%
\end{pgfscope}%
\begin{pgfscope}%
\pgfpathrectangle{\pgfqpoint{2.816705in}{0.516222in}}{\pgfqpoint{1.962733in}{1.783528in}} %
\pgfusepath{clip}%
\pgfsetbuttcap%
\pgfsetroundjoin%
\definecolor{currentfill}{rgb}{0.298039,0.447059,0.690196}%
\pgfsetfillcolor{currentfill}%
\pgfsetlinewidth{0.240900pt}%
\definecolor{currentstroke}{rgb}{1.000000,1.000000,1.000000}%
\pgfsetstrokecolor{currentstroke}%
\pgfsetdash{}{0pt}%
\pgfpathmoveto{\pgfqpoint{3.545720in}{1.793086in}}%
\pgfpathcurveto{\pgfqpoint{3.553956in}{1.793086in}}{\pgfqpoint{3.561856in}{1.796358in}}{\pgfqpoint{3.567680in}{1.802182in}}%
\pgfpathcurveto{\pgfqpoint{3.573504in}{1.808006in}}{\pgfqpoint{3.576776in}{1.815906in}}{\pgfqpoint{3.576776in}{1.824143in}}%
\pgfpathcurveto{\pgfqpoint{3.576776in}{1.832379in}}{\pgfqpoint{3.573504in}{1.840279in}}{\pgfqpoint{3.567680in}{1.846103in}}%
\pgfpathcurveto{\pgfqpoint{3.561856in}{1.851927in}}{\pgfqpoint{3.553956in}{1.855199in}}{\pgfqpoint{3.545720in}{1.855199in}}%
\pgfpathcurveto{\pgfqpoint{3.537484in}{1.855199in}}{\pgfqpoint{3.529584in}{1.851927in}}{\pgfqpoint{3.523760in}{1.846103in}}%
\pgfpathcurveto{\pgfqpoint{3.517936in}{1.840279in}}{\pgfqpoint{3.514663in}{1.832379in}}{\pgfqpoint{3.514663in}{1.824143in}}%
\pgfpathcurveto{\pgfqpoint{3.514663in}{1.815906in}}{\pgfqpoint{3.517936in}{1.808006in}}{\pgfqpoint{3.523760in}{1.802182in}}%
\pgfpathcurveto{\pgfqpoint{3.529584in}{1.796358in}}{\pgfqpoint{3.537484in}{1.793086in}}{\pgfqpoint{3.545720in}{1.793086in}}%
\pgfpathclose%
\pgfusepath{stroke,fill}%
\end{pgfscope}%
\begin{pgfscope}%
\pgfpathrectangle{\pgfqpoint{2.816705in}{0.516222in}}{\pgfqpoint{1.962733in}{1.783528in}} %
\pgfusepath{clip}%
\pgfsetbuttcap%
\pgfsetroundjoin%
\definecolor{currentfill}{rgb}{0.298039,0.447059,0.690196}%
\pgfsetfillcolor{currentfill}%
\pgfsetlinewidth{0.240900pt}%
\definecolor{currentstroke}{rgb}{1.000000,1.000000,1.000000}%
\pgfsetstrokecolor{currentstroke}%
\pgfsetdash{}{0pt}%
\pgfpathmoveto{\pgfqpoint{3.601798in}{1.376930in}}%
\pgfpathcurveto{\pgfqpoint{3.610034in}{1.376930in}}{\pgfqpoint{3.617934in}{1.380202in}}{\pgfqpoint{3.623758in}{1.386026in}}%
\pgfpathcurveto{\pgfqpoint{3.629582in}{1.391850in}}{\pgfqpoint{3.632854in}{1.399750in}}{\pgfqpoint{3.632854in}{1.407986in}}%
\pgfpathcurveto{\pgfqpoint{3.632854in}{1.416222in}}{\pgfqpoint{3.629582in}{1.424122in}}{\pgfqpoint{3.623758in}{1.429946in}}%
\pgfpathcurveto{\pgfqpoint{3.617934in}{1.435770in}}{\pgfqpoint{3.610034in}{1.439043in}}{\pgfqpoint{3.601798in}{1.439043in}}%
\pgfpathcurveto{\pgfqpoint{3.593562in}{1.439043in}}{\pgfqpoint{3.585662in}{1.435770in}}{\pgfqpoint{3.579838in}{1.429946in}}%
\pgfpathcurveto{\pgfqpoint{3.574014in}{1.424122in}}{\pgfqpoint{3.570741in}{1.416222in}}{\pgfqpoint{3.570741in}{1.407986in}}%
\pgfpathcurveto{\pgfqpoint{3.570741in}{1.399750in}}{\pgfqpoint{3.574014in}{1.391850in}}{\pgfqpoint{3.579838in}{1.386026in}}%
\pgfpathcurveto{\pgfqpoint{3.585662in}{1.380202in}}{\pgfqpoint{3.593562in}{1.376930in}}{\pgfqpoint{3.601798in}{1.376930in}}%
\pgfpathclose%
\pgfusepath{stroke,fill}%
\end{pgfscope}%
\begin{pgfscope}%
\pgfpathrectangle{\pgfqpoint{2.816705in}{0.516222in}}{\pgfqpoint{1.962733in}{1.783528in}} %
\pgfusepath{clip}%
\pgfsetbuttcap%
\pgfsetroundjoin%
\definecolor{currentfill}{rgb}{0.298039,0.447059,0.690196}%
\pgfsetfillcolor{currentfill}%
\pgfsetlinewidth{0.240900pt}%
\definecolor{currentstroke}{rgb}{1.000000,1.000000,1.000000}%
\pgfsetstrokecolor{currentstroke}%
\pgfsetdash{}{0pt}%
\pgfpathmoveto{\pgfqpoint{2.984939in}{0.693244in}}%
\pgfpathcurveto{\pgfqpoint{2.993175in}{0.693244in}}{\pgfqpoint{3.001075in}{0.696516in}}{\pgfqpoint{3.006899in}{0.702340in}}%
\pgfpathcurveto{\pgfqpoint{3.012723in}{0.708164in}}{\pgfqpoint{3.015996in}{0.716064in}}{\pgfqpoint{3.015996in}{0.724300in}}%
\pgfpathcurveto{\pgfqpoint{3.015996in}{0.732537in}}{\pgfqpoint{3.012723in}{0.740437in}}{\pgfqpoint{3.006899in}{0.746261in}}%
\pgfpathcurveto{\pgfqpoint{3.001075in}{0.752085in}}{\pgfqpoint{2.993175in}{0.755357in}}{\pgfqpoint{2.984939in}{0.755357in}}%
\pgfpathcurveto{\pgfqpoint{2.976703in}{0.755357in}}{\pgfqpoint{2.968803in}{0.752085in}}{\pgfqpoint{2.962979in}{0.746261in}}%
\pgfpathcurveto{\pgfqpoint{2.957155in}{0.740437in}}{\pgfqpoint{2.953883in}{0.732537in}}{\pgfqpoint{2.953883in}{0.724300in}}%
\pgfpathcurveto{\pgfqpoint{2.953883in}{0.716064in}}{\pgfqpoint{2.957155in}{0.708164in}}{\pgfqpoint{2.962979in}{0.702340in}}%
\pgfpathcurveto{\pgfqpoint{2.968803in}{0.696516in}}{\pgfqpoint{2.976703in}{0.693244in}}{\pgfqpoint{2.984939in}{0.693244in}}%
\pgfpathclose%
\pgfusepath{stroke,fill}%
\end{pgfscope}%
\begin{pgfscope}%
\pgfpathrectangle{\pgfqpoint{2.816705in}{0.516222in}}{\pgfqpoint{1.962733in}{1.783528in}} %
\pgfusepath{clip}%
\pgfsetbuttcap%
\pgfsetroundjoin%
\definecolor{currentfill}{rgb}{0.298039,0.447059,0.690196}%
\pgfsetfillcolor{currentfill}%
\pgfsetlinewidth{0.240900pt}%
\definecolor{currentstroke}{rgb}{1.000000,1.000000,1.000000}%
\pgfsetstrokecolor{currentstroke}%
\pgfsetdash{}{0pt}%
\pgfpathmoveto{\pgfqpoint{3.938266in}{2.001164in}}%
\pgfpathcurveto{\pgfqpoint{3.946503in}{2.001164in}}{\pgfqpoint{3.954403in}{2.004437in}}{\pgfqpoint{3.960227in}{2.010261in}}%
\pgfpathcurveto{\pgfqpoint{3.966051in}{2.016085in}}{\pgfqpoint{3.969323in}{2.023985in}}{\pgfqpoint{3.969323in}{2.032221in}}%
\pgfpathcurveto{\pgfqpoint{3.969323in}{2.040457in}}{\pgfqpoint{3.966051in}{2.048357in}}{\pgfqpoint{3.960227in}{2.054181in}}%
\pgfpathcurveto{\pgfqpoint{3.954403in}{2.060005in}}{\pgfqpoint{3.946503in}{2.063277in}}{\pgfqpoint{3.938266in}{2.063277in}}%
\pgfpathcurveto{\pgfqpoint{3.930030in}{2.063277in}}{\pgfqpoint{3.922130in}{2.060005in}}{\pgfqpoint{3.916306in}{2.054181in}}%
\pgfpathcurveto{\pgfqpoint{3.910482in}{2.048357in}}{\pgfqpoint{3.907210in}{2.040457in}}{\pgfqpoint{3.907210in}{2.032221in}}%
\pgfpathcurveto{\pgfqpoint{3.907210in}{2.023985in}}{\pgfqpoint{3.910482in}{2.016085in}}{\pgfqpoint{3.916306in}{2.010261in}}%
\pgfpathcurveto{\pgfqpoint{3.922130in}{2.004437in}}{\pgfqpoint{3.930030in}{2.001164in}}{\pgfqpoint{3.938266in}{2.001164in}}%
\pgfpathclose%
\pgfusepath{stroke,fill}%
\end{pgfscope}%
\begin{pgfscope}%
\pgfpathrectangle{\pgfqpoint{2.816705in}{0.516222in}}{\pgfqpoint{1.962733in}{1.783528in}} %
\pgfusepath{clip}%
\pgfsetbuttcap%
\pgfsetroundjoin%
\definecolor{currentfill}{rgb}{0.298039,0.447059,0.690196}%
\pgfsetfillcolor{currentfill}%
\pgfsetlinewidth{0.240900pt}%
\definecolor{currentstroke}{rgb}{1.000000,1.000000,1.000000}%
\pgfsetstrokecolor{currentstroke}%
\pgfsetdash{}{0pt}%
\pgfpathmoveto{\pgfqpoint{3.882188in}{1.466106in}}%
\pgfpathcurveto{\pgfqpoint{3.890425in}{1.466106in}}{\pgfqpoint{3.898325in}{1.469378in}}{\pgfqpoint{3.904149in}{1.475202in}}%
\pgfpathcurveto{\pgfqpoint{3.909972in}{1.481026in}}{\pgfqpoint{3.913245in}{1.488926in}}{\pgfqpoint{3.913245in}{1.497163in}}%
\pgfpathcurveto{\pgfqpoint{3.913245in}{1.505399in}}{\pgfqpoint{3.909972in}{1.513299in}}{\pgfqpoint{3.904149in}{1.519123in}}%
\pgfpathcurveto{\pgfqpoint{3.898325in}{1.524947in}}{\pgfqpoint{3.890425in}{1.528219in}}{\pgfqpoint{3.882188in}{1.528219in}}%
\pgfpathcurveto{\pgfqpoint{3.873952in}{1.528219in}}{\pgfqpoint{3.866052in}{1.524947in}}{\pgfqpoint{3.860228in}{1.519123in}}%
\pgfpathcurveto{\pgfqpoint{3.854404in}{1.513299in}}{\pgfqpoint{3.851132in}{1.505399in}}{\pgfqpoint{3.851132in}{1.497163in}}%
\pgfpathcurveto{\pgfqpoint{3.851132in}{1.488926in}}{\pgfqpoint{3.854404in}{1.481026in}}{\pgfqpoint{3.860228in}{1.475202in}}%
\pgfpathcurveto{\pgfqpoint{3.866052in}{1.469378in}}{\pgfqpoint{3.873952in}{1.466106in}}{\pgfqpoint{3.882188in}{1.466106in}}%
\pgfpathclose%
\pgfusepath{stroke,fill}%
\end{pgfscope}%
\begin{pgfscope}%
\pgfpathrectangle{\pgfqpoint{2.816705in}{0.516222in}}{\pgfqpoint{1.962733in}{1.783528in}} %
\pgfusepath{clip}%
\pgfsetbuttcap%
\pgfsetroundjoin%
\definecolor{currentfill}{rgb}{0.298039,0.447059,0.690196}%
\pgfsetfillcolor{currentfill}%
\pgfsetlinewidth{0.240900pt}%
\definecolor{currentstroke}{rgb}{1.000000,1.000000,1.000000}%
\pgfsetstrokecolor{currentstroke}%
\pgfsetdash{}{0pt}%
\pgfpathmoveto{\pgfqpoint{3.433564in}{1.258028in}}%
\pgfpathcurveto{\pgfqpoint{3.441800in}{1.258028in}}{\pgfqpoint{3.449700in}{1.261300in}}{\pgfqpoint{3.455524in}{1.267124in}}%
\pgfpathcurveto{\pgfqpoint{3.461348in}{1.272948in}}{\pgfqpoint{3.464620in}{1.280848in}}{\pgfqpoint{3.464620in}{1.289084in}}%
\pgfpathcurveto{\pgfqpoint{3.464620in}{1.297321in}}{\pgfqpoint{3.461348in}{1.305221in}}{\pgfqpoint{3.455524in}{1.311045in}}%
\pgfpathcurveto{\pgfqpoint{3.449700in}{1.316868in}}{\pgfqpoint{3.441800in}{1.320141in}}{\pgfqpoint{3.433564in}{1.320141in}}%
\pgfpathcurveto{\pgfqpoint{3.425327in}{1.320141in}}{\pgfqpoint{3.417427in}{1.316868in}}{\pgfqpoint{3.411603in}{1.311045in}}%
\pgfpathcurveto{\pgfqpoint{3.405779in}{1.305221in}}{\pgfqpoint{3.402507in}{1.297321in}}{\pgfqpoint{3.402507in}{1.289084in}}%
\pgfpathcurveto{\pgfqpoint{3.402507in}{1.280848in}}{\pgfqpoint{3.405779in}{1.272948in}}{\pgfqpoint{3.411603in}{1.267124in}}%
\pgfpathcurveto{\pgfqpoint{3.417427in}{1.261300in}}{\pgfqpoint{3.425327in}{1.258028in}}{\pgfqpoint{3.433564in}{1.258028in}}%
\pgfpathclose%
\pgfusepath{stroke,fill}%
\end{pgfscope}%
\begin{pgfscope}%
\pgfpathrectangle{\pgfqpoint{2.816705in}{0.516222in}}{\pgfqpoint{1.962733in}{1.783528in}} %
\pgfusepath{clip}%
\pgfsetbuttcap%
\pgfsetroundjoin%
\definecolor{currentfill}{rgb}{0.298039,0.447059,0.690196}%
\pgfsetfillcolor{currentfill}%
\pgfsetlinewidth{0.240900pt}%
\definecolor{currentstroke}{rgb}{1.000000,1.000000,1.000000}%
\pgfsetstrokecolor{currentstroke}%
\pgfsetdash{}{0pt}%
\pgfpathmoveto{\pgfqpoint{3.601798in}{0.633793in}}%
\pgfpathcurveto{\pgfqpoint{3.610034in}{0.633793in}}{\pgfqpoint{3.617934in}{0.637065in}}{\pgfqpoint{3.623758in}{0.642889in}}%
\pgfpathcurveto{\pgfqpoint{3.629582in}{0.648713in}}{\pgfqpoint{3.632854in}{0.656613in}}{\pgfqpoint{3.632854in}{0.664850in}}%
\pgfpathcurveto{\pgfqpoint{3.632854in}{0.673086in}}{\pgfqpoint{3.629582in}{0.680986in}}{\pgfqpoint{3.623758in}{0.686810in}}%
\pgfpathcurveto{\pgfqpoint{3.617934in}{0.692634in}}{\pgfqpoint{3.610034in}{0.695906in}}{\pgfqpoint{3.601798in}{0.695906in}}%
\pgfpathcurveto{\pgfqpoint{3.593562in}{0.695906in}}{\pgfqpoint{3.585662in}{0.692634in}}{\pgfqpoint{3.579838in}{0.686810in}}%
\pgfpathcurveto{\pgfqpoint{3.574014in}{0.680986in}}{\pgfqpoint{3.570741in}{0.673086in}}{\pgfqpoint{3.570741in}{0.664850in}}%
\pgfpathcurveto{\pgfqpoint{3.570741in}{0.656613in}}{\pgfqpoint{3.574014in}{0.648713in}}{\pgfqpoint{3.579838in}{0.642889in}}%
\pgfpathcurveto{\pgfqpoint{3.585662in}{0.637065in}}{\pgfqpoint{3.593562in}{0.633793in}}{\pgfqpoint{3.601798in}{0.633793in}}%
\pgfpathclose%
\pgfusepath{stroke,fill}%
\end{pgfscope}%
\begin{pgfscope}%
\pgfpathrectangle{\pgfqpoint{2.816705in}{0.516222in}}{\pgfqpoint{1.962733in}{1.783528in}} %
\pgfusepath{clip}%
\pgfsetbuttcap%
\pgfsetroundjoin%
\definecolor{currentfill}{rgb}{0.298039,0.447059,0.690196}%
\pgfsetfillcolor{currentfill}%
\pgfsetlinewidth{0.240900pt}%
\definecolor{currentstroke}{rgb}{1.000000,1.000000,1.000000}%
\pgfsetstrokecolor{currentstroke}%
\pgfsetdash{}{0pt}%
\pgfpathmoveto{\pgfqpoint{3.209251in}{1.049950in}}%
\pgfpathcurveto{\pgfqpoint{3.217488in}{1.049950in}}{\pgfqpoint{3.225388in}{1.053222in}}{\pgfqpoint{3.231212in}{1.059046in}}%
\pgfpathcurveto{\pgfqpoint{3.237036in}{1.064870in}}{\pgfqpoint{3.240308in}{1.072770in}}{\pgfqpoint{3.240308in}{1.081006in}}%
\pgfpathcurveto{\pgfqpoint{3.240308in}{1.089242in}}{\pgfqpoint{3.237036in}{1.097142in}}{\pgfqpoint{3.231212in}{1.102966in}}%
\pgfpathcurveto{\pgfqpoint{3.225388in}{1.108790in}}{\pgfqpoint{3.217488in}{1.112063in}}{\pgfqpoint{3.209251in}{1.112063in}}%
\pgfpathcurveto{\pgfqpoint{3.201015in}{1.112063in}}{\pgfqpoint{3.193115in}{1.108790in}}{\pgfqpoint{3.187291in}{1.102966in}}%
\pgfpathcurveto{\pgfqpoint{3.181467in}{1.097142in}}{\pgfqpoint{3.178195in}{1.089242in}}{\pgfqpoint{3.178195in}{1.081006in}}%
\pgfpathcurveto{\pgfqpoint{3.178195in}{1.072770in}}{\pgfqpoint{3.181467in}{1.064870in}}{\pgfqpoint{3.187291in}{1.059046in}}%
\pgfpathcurveto{\pgfqpoint{3.193115in}{1.053222in}}{\pgfqpoint{3.201015in}{1.049950in}}{\pgfqpoint{3.209251in}{1.049950in}}%
\pgfpathclose%
\pgfusepath{stroke,fill}%
\end{pgfscope}%
\begin{pgfscope}%
\pgfpathrectangle{\pgfqpoint{2.816705in}{0.516222in}}{\pgfqpoint{1.962733in}{1.783528in}} %
\pgfusepath{clip}%
\pgfsetbuttcap%
\pgfsetroundjoin%
\definecolor{currentfill}{rgb}{0.298039,0.447059,0.690196}%
\pgfsetfillcolor{currentfill}%
\pgfsetlinewidth{0.240900pt}%
\definecolor{currentstroke}{rgb}{1.000000,1.000000,1.000000}%
\pgfsetstrokecolor{currentstroke}%
\pgfsetdash{}{0pt}%
\pgfpathmoveto{\pgfqpoint{3.545720in}{1.139126in}}%
\pgfpathcurveto{\pgfqpoint{3.553956in}{1.139126in}}{\pgfqpoint{3.561856in}{1.142398in}}{\pgfqpoint{3.567680in}{1.148222in}}%
\pgfpathcurveto{\pgfqpoint{3.573504in}{1.154046in}}{\pgfqpoint{3.576776in}{1.161946in}}{\pgfqpoint{3.576776in}{1.170182in}}%
\pgfpathcurveto{\pgfqpoint{3.576776in}{1.178419in}}{\pgfqpoint{3.573504in}{1.186319in}}{\pgfqpoint{3.567680in}{1.192143in}}%
\pgfpathcurveto{\pgfqpoint{3.561856in}{1.197967in}}{\pgfqpoint{3.553956in}{1.201239in}}{\pgfqpoint{3.545720in}{1.201239in}}%
\pgfpathcurveto{\pgfqpoint{3.537484in}{1.201239in}}{\pgfqpoint{3.529584in}{1.197967in}}{\pgfqpoint{3.523760in}{1.192143in}}%
\pgfpathcurveto{\pgfqpoint{3.517936in}{1.186319in}}{\pgfqpoint{3.514663in}{1.178419in}}{\pgfqpoint{3.514663in}{1.170182in}}%
\pgfpathcurveto{\pgfqpoint{3.514663in}{1.161946in}}{\pgfqpoint{3.517936in}{1.154046in}}{\pgfqpoint{3.523760in}{1.148222in}}%
\pgfpathcurveto{\pgfqpoint{3.529584in}{1.142398in}}{\pgfqpoint{3.537484in}{1.139126in}}{\pgfqpoint{3.545720in}{1.139126in}}%
\pgfpathclose%
\pgfusepath{stroke,fill}%
\end{pgfscope}%
\begin{pgfscope}%
\pgfpathrectangle{\pgfqpoint{2.816705in}{0.516222in}}{\pgfqpoint{1.962733in}{1.783528in}} %
\pgfusepath{clip}%
\pgfsetbuttcap%
\pgfsetroundjoin%
\definecolor{currentfill}{rgb}{0.298039,0.447059,0.690196}%
\pgfsetfillcolor{currentfill}%
\pgfsetlinewidth{0.240900pt}%
\definecolor{currentstroke}{rgb}{1.000000,1.000000,1.000000}%
\pgfsetstrokecolor{currentstroke}%
\pgfsetdash{}{0pt}%
\pgfpathmoveto{\pgfqpoint{3.882188in}{1.495831in}}%
\pgfpathcurveto{\pgfqpoint{3.890425in}{1.495831in}}{\pgfqpoint{3.898325in}{1.499104in}}{\pgfqpoint{3.904149in}{1.504928in}}%
\pgfpathcurveto{\pgfqpoint{3.909972in}{1.510752in}}{\pgfqpoint{3.913245in}{1.518652in}}{\pgfqpoint{3.913245in}{1.526888in}}%
\pgfpathcurveto{\pgfqpoint{3.913245in}{1.535124in}}{\pgfqpoint{3.909972in}{1.543024in}}{\pgfqpoint{3.904149in}{1.548848in}}%
\pgfpathcurveto{\pgfqpoint{3.898325in}{1.554672in}}{\pgfqpoint{3.890425in}{1.557944in}}{\pgfqpoint{3.882188in}{1.557944in}}%
\pgfpathcurveto{\pgfqpoint{3.873952in}{1.557944in}}{\pgfqpoint{3.866052in}{1.554672in}}{\pgfqpoint{3.860228in}{1.548848in}}%
\pgfpathcurveto{\pgfqpoint{3.854404in}{1.543024in}}{\pgfqpoint{3.851132in}{1.535124in}}{\pgfqpoint{3.851132in}{1.526888in}}%
\pgfpathcurveto{\pgfqpoint{3.851132in}{1.518652in}}{\pgfqpoint{3.854404in}{1.510752in}}{\pgfqpoint{3.860228in}{1.504928in}}%
\pgfpathcurveto{\pgfqpoint{3.866052in}{1.499104in}}{\pgfqpoint{3.873952in}{1.495831in}}{\pgfqpoint{3.882188in}{1.495831in}}%
\pgfpathclose%
\pgfusepath{stroke,fill}%
\end{pgfscope}%
\begin{pgfscope}%
\pgfpathrectangle{\pgfqpoint{2.816705in}{0.516222in}}{\pgfqpoint{1.962733in}{1.783528in}} %
\pgfusepath{clip}%
\pgfsetbuttcap%
\pgfsetroundjoin%
\definecolor{currentfill}{rgb}{0.298039,0.447059,0.690196}%
\pgfsetfillcolor{currentfill}%
\pgfsetlinewidth{0.240900pt}%
\definecolor{currentstroke}{rgb}{1.000000,1.000000,1.000000}%
\pgfsetstrokecolor{currentstroke}%
\pgfsetdash{}{0pt}%
\pgfpathmoveto{\pgfqpoint{4.050423in}{0.901322in}}%
\pgfpathcurveto{\pgfqpoint{4.058659in}{0.901322in}}{\pgfqpoint{4.066559in}{0.904595in}}{\pgfqpoint{4.072383in}{0.910418in}}%
\pgfpathcurveto{\pgfqpoint{4.078207in}{0.916242in}}{\pgfqpoint{4.081479in}{0.924142in}}{\pgfqpoint{4.081479in}{0.932379in}}%
\pgfpathcurveto{\pgfqpoint{4.081479in}{0.940615in}}{\pgfqpoint{4.078207in}{0.948515in}}{\pgfqpoint{4.072383in}{0.954339in}}%
\pgfpathcurveto{\pgfqpoint{4.066559in}{0.960163in}}{\pgfqpoint{4.058659in}{0.963435in}}{\pgfqpoint{4.050423in}{0.963435in}}%
\pgfpathcurveto{\pgfqpoint{4.042186in}{0.963435in}}{\pgfqpoint{4.034286in}{0.960163in}}{\pgfqpoint{4.028462in}{0.954339in}}%
\pgfpathcurveto{\pgfqpoint{4.022638in}{0.948515in}}{\pgfqpoint{4.019366in}{0.940615in}}{\pgfqpoint{4.019366in}{0.932379in}}%
\pgfpathcurveto{\pgfqpoint{4.019366in}{0.924142in}}{\pgfqpoint{4.022638in}{0.916242in}}{\pgfqpoint{4.028462in}{0.910418in}}%
\pgfpathcurveto{\pgfqpoint{4.034286in}{0.904595in}}{\pgfqpoint{4.042186in}{0.901322in}}{\pgfqpoint{4.050423in}{0.901322in}}%
\pgfpathclose%
\pgfusepath{stroke,fill}%
\end{pgfscope}%
\begin{pgfscope}%
\pgfpathrectangle{\pgfqpoint{2.816705in}{0.516222in}}{\pgfqpoint{1.962733in}{1.783528in}} %
\pgfusepath{clip}%
\pgfsetbuttcap%
\pgfsetroundjoin%
\definecolor{currentfill}{rgb}{0.298039,0.447059,0.690196}%
\pgfsetfillcolor{currentfill}%
\pgfsetlinewidth{0.240900pt}%
\definecolor{currentstroke}{rgb}{1.000000,1.000000,1.000000}%
\pgfsetstrokecolor{currentstroke}%
\pgfsetdash{}{0pt}%
\pgfpathmoveto{\pgfqpoint{3.601798in}{1.644459in}}%
\pgfpathcurveto{\pgfqpoint{3.610034in}{1.644459in}}{\pgfqpoint{3.617934in}{1.647731in}}{\pgfqpoint{3.623758in}{1.653555in}}%
\pgfpathcurveto{\pgfqpoint{3.629582in}{1.659379in}}{\pgfqpoint{3.632854in}{1.667279in}}{\pgfqpoint{3.632854in}{1.675515in}}%
\pgfpathcurveto{\pgfqpoint{3.632854in}{1.683752in}}{\pgfqpoint{3.629582in}{1.691652in}}{\pgfqpoint{3.623758in}{1.697476in}}%
\pgfpathcurveto{\pgfqpoint{3.617934in}{1.703299in}}{\pgfqpoint{3.610034in}{1.706572in}}{\pgfqpoint{3.601798in}{1.706572in}}%
\pgfpathcurveto{\pgfqpoint{3.593562in}{1.706572in}}{\pgfqpoint{3.585662in}{1.703299in}}{\pgfqpoint{3.579838in}{1.697476in}}%
\pgfpathcurveto{\pgfqpoint{3.574014in}{1.691652in}}{\pgfqpoint{3.570741in}{1.683752in}}{\pgfqpoint{3.570741in}{1.675515in}}%
\pgfpathcurveto{\pgfqpoint{3.570741in}{1.667279in}}{\pgfqpoint{3.574014in}{1.659379in}}{\pgfqpoint{3.579838in}{1.653555in}}%
\pgfpathcurveto{\pgfqpoint{3.585662in}{1.647731in}}{\pgfqpoint{3.593562in}{1.644459in}}{\pgfqpoint{3.601798in}{1.644459in}}%
\pgfpathclose%
\pgfusepath{stroke,fill}%
\end{pgfscope}%
\begin{pgfscope}%
\pgfpathrectangle{\pgfqpoint{2.816705in}{0.516222in}}{\pgfqpoint{1.962733in}{1.783528in}} %
\pgfusepath{clip}%
\pgfsetbuttcap%
\pgfsetroundjoin%
\definecolor{currentfill}{rgb}{0.298039,0.447059,0.690196}%
\pgfsetfillcolor{currentfill}%
\pgfsetlinewidth{0.240900pt}%
\definecolor{currentstroke}{rgb}{1.000000,1.000000,1.000000}%
\pgfsetstrokecolor{currentstroke}%
\pgfsetdash{}{0pt}%
\pgfpathmoveto{\pgfqpoint{3.433564in}{0.871597in}}%
\pgfpathcurveto{\pgfqpoint{3.441800in}{0.871597in}}{\pgfqpoint{3.449700in}{0.874869in}}{\pgfqpoint{3.455524in}{0.880693in}}%
\pgfpathcurveto{\pgfqpoint{3.461348in}{0.886517in}}{\pgfqpoint{3.464620in}{0.894417in}}{\pgfqpoint{3.464620in}{0.902653in}}%
\pgfpathcurveto{\pgfqpoint{3.464620in}{0.910890in}}{\pgfqpoint{3.461348in}{0.918790in}}{\pgfqpoint{3.455524in}{0.924614in}}%
\pgfpathcurveto{\pgfqpoint{3.449700in}{0.930437in}}{\pgfqpoint{3.441800in}{0.933710in}}{\pgfqpoint{3.433564in}{0.933710in}}%
\pgfpathcurveto{\pgfqpoint{3.425327in}{0.933710in}}{\pgfqpoint{3.417427in}{0.930437in}}{\pgfqpoint{3.411603in}{0.924614in}}%
\pgfpathcurveto{\pgfqpoint{3.405779in}{0.918790in}}{\pgfqpoint{3.402507in}{0.910890in}}{\pgfqpoint{3.402507in}{0.902653in}}%
\pgfpathcurveto{\pgfqpoint{3.402507in}{0.894417in}}{\pgfqpoint{3.405779in}{0.886517in}}{\pgfqpoint{3.411603in}{0.880693in}}%
\pgfpathcurveto{\pgfqpoint{3.417427in}{0.874869in}}{\pgfqpoint{3.425327in}{0.871597in}}{\pgfqpoint{3.433564in}{0.871597in}}%
\pgfpathclose%
\pgfusepath{stroke,fill}%
\end{pgfscope}%
\begin{pgfscope}%
\pgfpathrectangle{\pgfqpoint{2.816705in}{0.516222in}}{\pgfqpoint{1.962733in}{1.783528in}} %
\pgfusepath{clip}%
\pgfsetbuttcap%
\pgfsetroundjoin%
\definecolor{currentfill}{rgb}{0.298039,0.447059,0.690196}%
\pgfsetfillcolor{currentfill}%
\pgfsetlinewidth{0.240900pt}%
\definecolor{currentstroke}{rgb}{1.000000,1.000000,1.000000}%
\pgfsetstrokecolor{currentstroke}%
\pgfsetdash{}{0pt}%
\pgfpathmoveto{\pgfqpoint{3.433564in}{1.198577in}}%
\pgfpathcurveto{\pgfqpoint{3.441800in}{1.198577in}}{\pgfqpoint{3.449700in}{1.201849in}}{\pgfqpoint{3.455524in}{1.207673in}}%
\pgfpathcurveto{\pgfqpoint{3.461348in}{1.213497in}}{\pgfqpoint{3.464620in}{1.221397in}}{\pgfqpoint{3.464620in}{1.229633in}}%
\pgfpathcurveto{\pgfqpoint{3.464620in}{1.237870in}}{\pgfqpoint{3.461348in}{1.245770in}}{\pgfqpoint{3.455524in}{1.251594in}}%
\pgfpathcurveto{\pgfqpoint{3.449700in}{1.257418in}}{\pgfqpoint{3.441800in}{1.260690in}}{\pgfqpoint{3.433564in}{1.260690in}}%
\pgfpathcurveto{\pgfqpoint{3.425327in}{1.260690in}}{\pgfqpoint{3.417427in}{1.257418in}}{\pgfqpoint{3.411603in}{1.251594in}}%
\pgfpathcurveto{\pgfqpoint{3.405779in}{1.245770in}}{\pgfqpoint{3.402507in}{1.237870in}}{\pgfqpoint{3.402507in}{1.229633in}}%
\pgfpathcurveto{\pgfqpoint{3.402507in}{1.221397in}}{\pgfqpoint{3.405779in}{1.213497in}}{\pgfqpoint{3.411603in}{1.207673in}}%
\pgfpathcurveto{\pgfqpoint{3.417427in}{1.201849in}}{\pgfqpoint{3.425327in}{1.198577in}}{\pgfqpoint{3.433564in}{1.198577in}}%
\pgfpathclose%
\pgfusepath{stroke,fill}%
\end{pgfscope}%
\begin{pgfscope}%
\pgfpathrectangle{\pgfqpoint{2.816705in}{0.516222in}}{\pgfqpoint{1.962733in}{1.783528in}} %
\pgfusepath{clip}%
\pgfsetbuttcap%
\pgfsetroundjoin%
\definecolor{currentfill}{rgb}{0.298039,0.447059,0.690196}%
\pgfsetfillcolor{currentfill}%
\pgfsetlinewidth{0.240900pt}%
\definecolor{currentstroke}{rgb}{1.000000,1.000000,1.000000}%
\pgfsetstrokecolor{currentstroke}%
\pgfsetdash{}{0pt}%
\pgfpathmoveto{\pgfqpoint{4.050423in}{1.406655in}}%
\pgfpathcurveto{\pgfqpoint{4.058659in}{1.406655in}}{\pgfqpoint{4.066559in}{1.409927in}}{\pgfqpoint{4.072383in}{1.415751in}}%
\pgfpathcurveto{\pgfqpoint{4.078207in}{1.421575in}}{\pgfqpoint{4.081479in}{1.429475in}}{\pgfqpoint{4.081479in}{1.437712in}}%
\pgfpathcurveto{\pgfqpoint{4.081479in}{1.445948in}}{\pgfqpoint{4.078207in}{1.453848in}}{\pgfqpoint{4.072383in}{1.459672in}}%
\pgfpathcurveto{\pgfqpoint{4.066559in}{1.465496in}}{\pgfqpoint{4.058659in}{1.468768in}}{\pgfqpoint{4.050423in}{1.468768in}}%
\pgfpathcurveto{\pgfqpoint{4.042186in}{1.468768in}}{\pgfqpoint{4.034286in}{1.465496in}}{\pgfqpoint{4.028462in}{1.459672in}}%
\pgfpathcurveto{\pgfqpoint{4.022638in}{1.453848in}}{\pgfqpoint{4.019366in}{1.445948in}}{\pgfqpoint{4.019366in}{1.437712in}}%
\pgfpathcurveto{\pgfqpoint{4.019366in}{1.429475in}}{\pgfqpoint{4.022638in}{1.421575in}}{\pgfqpoint{4.028462in}{1.415751in}}%
\pgfpathcurveto{\pgfqpoint{4.034286in}{1.409927in}}{\pgfqpoint{4.042186in}{1.406655in}}{\pgfqpoint{4.050423in}{1.406655in}}%
\pgfpathclose%
\pgfusepath{stroke,fill}%
\end{pgfscope}%
\begin{pgfscope}%
\pgfpathrectangle{\pgfqpoint{2.816705in}{0.516222in}}{\pgfqpoint{1.962733in}{1.783528in}} %
\pgfusepath{clip}%
\pgfsetbuttcap%
\pgfsetroundjoin%
\definecolor{currentfill}{rgb}{0.298039,0.447059,0.690196}%
\pgfsetfillcolor{currentfill}%
\pgfsetlinewidth{0.240900pt}%
\definecolor{currentstroke}{rgb}{1.000000,1.000000,1.000000}%
\pgfsetstrokecolor{currentstroke}%
\pgfsetdash{}{0pt}%
\pgfpathmoveto{\pgfqpoint{3.377486in}{1.139126in}}%
\pgfpathcurveto{\pgfqpoint{3.385722in}{1.139126in}}{\pgfqpoint{3.393622in}{1.142398in}}{\pgfqpoint{3.399446in}{1.148222in}}%
\pgfpathcurveto{\pgfqpoint{3.405270in}{1.154046in}}{\pgfqpoint{3.408542in}{1.161946in}}{\pgfqpoint{3.408542in}{1.170182in}}%
\pgfpathcurveto{\pgfqpoint{3.408542in}{1.178419in}}{\pgfqpoint{3.405270in}{1.186319in}}{\pgfqpoint{3.399446in}{1.192143in}}%
\pgfpathcurveto{\pgfqpoint{3.393622in}{1.197967in}}{\pgfqpoint{3.385722in}{1.201239in}}{\pgfqpoint{3.377486in}{1.201239in}}%
\pgfpathcurveto{\pgfqpoint{3.369249in}{1.201239in}}{\pgfqpoint{3.361349in}{1.197967in}}{\pgfqpoint{3.355525in}{1.192143in}}%
\pgfpathcurveto{\pgfqpoint{3.349701in}{1.186319in}}{\pgfqpoint{3.346429in}{1.178419in}}{\pgfqpoint{3.346429in}{1.170182in}}%
\pgfpathcurveto{\pgfqpoint{3.346429in}{1.161946in}}{\pgfqpoint{3.349701in}{1.154046in}}{\pgfqpoint{3.355525in}{1.148222in}}%
\pgfpathcurveto{\pgfqpoint{3.361349in}{1.142398in}}{\pgfqpoint{3.369249in}{1.139126in}}{\pgfqpoint{3.377486in}{1.139126in}}%
\pgfpathclose%
\pgfusepath{stroke,fill}%
\end{pgfscope}%
\begin{pgfscope}%
\pgfpathrectangle{\pgfqpoint{2.816705in}{0.516222in}}{\pgfqpoint{1.962733in}{1.783528in}} %
\pgfusepath{clip}%
\pgfsetbuttcap%
\pgfsetroundjoin%
\definecolor{currentfill}{rgb}{0.298039,0.447059,0.690196}%
\pgfsetfillcolor{currentfill}%
\pgfsetlinewidth{0.240900pt}%
\definecolor{currentstroke}{rgb}{1.000000,1.000000,1.000000}%
\pgfsetstrokecolor{currentstroke}%
\pgfsetdash{}{0pt}%
\pgfpathmoveto{\pgfqpoint{3.489642in}{1.079675in}}%
\pgfpathcurveto{\pgfqpoint{3.497878in}{1.079675in}}{\pgfqpoint{3.505778in}{1.082947in}}{\pgfqpoint{3.511602in}{1.088771in}}%
\pgfpathcurveto{\pgfqpoint{3.517426in}{1.094595in}}{\pgfqpoint{3.520698in}{1.102495in}}{\pgfqpoint{3.520698in}{1.110731in}}%
\pgfpathcurveto{\pgfqpoint{3.520698in}{1.118968in}}{\pgfqpoint{3.517426in}{1.126868in}}{\pgfqpoint{3.511602in}{1.132692in}}%
\pgfpathcurveto{\pgfqpoint{3.505778in}{1.138516in}}{\pgfqpoint{3.497878in}{1.141788in}}{\pgfqpoint{3.489642in}{1.141788in}}%
\pgfpathcurveto{\pgfqpoint{3.481405in}{1.141788in}}{\pgfqpoint{3.473505in}{1.138516in}}{\pgfqpoint{3.467682in}{1.132692in}}%
\pgfpathcurveto{\pgfqpoint{3.461858in}{1.126868in}}{\pgfqpoint{3.458585in}{1.118968in}}{\pgfqpoint{3.458585in}{1.110731in}}%
\pgfpathcurveto{\pgfqpoint{3.458585in}{1.102495in}}{\pgfqpoint{3.461858in}{1.094595in}}{\pgfqpoint{3.467682in}{1.088771in}}%
\pgfpathcurveto{\pgfqpoint{3.473505in}{1.082947in}}{\pgfqpoint{3.481405in}{1.079675in}}{\pgfqpoint{3.489642in}{1.079675in}}%
\pgfpathclose%
\pgfusepath{stroke,fill}%
\end{pgfscope}%
\begin{pgfscope}%
\pgfpathrectangle{\pgfqpoint{2.816705in}{0.516222in}}{\pgfqpoint{1.962733in}{1.783528in}} %
\pgfusepath{clip}%
\pgfsetbuttcap%
\pgfsetroundjoin%
\definecolor{currentfill}{rgb}{0.298039,0.447059,0.690196}%
\pgfsetfillcolor{currentfill}%
\pgfsetlinewidth{0.240900pt}%
\definecolor{currentstroke}{rgb}{1.000000,1.000000,1.000000}%
\pgfsetstrokecolor{currentstroke}%
\pgfsetdash{}{0pt}%
\pgfpathmoveto{\pgfqpoint{3.265329in}{1.168851in}}%
\pgfpathcurveto{\pgfqpoint{3.273566in}{1.168851in}}{\pgfqpoint{3.281466in}{1.172124in}}{\pgfqpoint{3.287290in}{1.177948in}}%
\pgfpathcurveto{\pgfqpoint{3.293114in}{1.183772in}}{\pgfqpoint{3.296386in}{1.191672in}}{\pgfqpoint{3.296386in}{1.199908in}}%
\pgfpathcurveto{\pgfqpoint{3.296386in}{1.208144in}}{\pgfqpoint{3.293114in}{1.216044in}}{\pgfqpoint{3.287290in}{1.221868in}}%
\pgfpathcurveto{\pgfqpoint{3.281466in}{1.227692in}}{\pgfqpoint{3.273566in}{1.230964in}}{\pgfqpoint{3.265329in}{1.230964in}}%
\pgfpathcurveto{\pgfqpoint{3.257093in}{1.230964in}}{\pgfqpoint{3.249193in}{1.227692in}}{\pgfqpoint{3.243369in}{1.221868in}}%
\pgfpathcurveto{\pgfqpoint{3.237545in}{1.216044in}}{\pgfqpoint{3.234273in}{1.208144in}}{\pgfqpoint{3.234273in}{1.199908in}}%
\pgfpathcurveto{\pgfqpoint{3.234273in}{1.191672in}}{\pgfqpoint{3.237545in}{1.183772in}}{\pgfqpoint{3.243369in}{1.177948in}}%
\pgfpathcurveto{\pgfqpoint{3.249193in}{1.172124in}}{\pgfqpoint{3.257093in}{1.168851in}}{\pgfqpoint{3.265329in}{1.168851in}}%
\pgfpathclose%
\pgfusepath{stroke,fill}%
\end{pgfscope}%
\begin{pgfscope}%
\pgfpathrectangle{\pgfqpoint{2.816705in}{0.516222in}}{\pgfqpoint{1.962733in}{1.783528in}} %
\pgfusepath{clip}%
\pgfsetbuttcap%
\pgfsetroundjoin%
\definecolor{currentfill}{rgb}{0.298039,0.447059,0.690196}%
\pgfsetfillcolor{currentfill}%
\pgfsetlinewidth{0.240900pt}%
\definecolor{currentstroke}{rgb}{1.000000,1.000000,1.000000}%
\pgfsetstrokecolor{currentstroke}%
\pgfsetdash{}{0pt}%
\pgfpathmoveto{\pgfqpoint{4.218657in}{1.763361in}}%
\pgfpathcurveto{\pgfqpoint{4.226893in}{1.763361in}}{\pgfqpoint{4.234793in}{1.766633in}}{\pgfqpoint{4.240617in}{1.772457in}}%
\pgfpathcurveto{\pgfqpoint{4.246441in}{1.778281in}}{\pgfqpoint{4.249713in}{1.786181in}}{\pgfqpoint{4.249713in}{1.794417in}}%
\pgfpathcurveto{\pgfqpoint{4.249713in}{1.802653in}}{\pgfqpoint{4.246441in}{1.810553in}}{\pgfqpoint{4.240617in}{1.816377in}}%
\pgfpathcurveto{\pgfqpoint{4.234793in}{1.822201in}}{\pgfqpoint{4.226893in}{1.825474in}}{\pgfqpoint{4.218657in}{1.825474in}}%
\pgfpathcurveto{\pgfqpoint{4.210420in}{1.825474in}}{\pgfqpoint{4.202520in}{1.822201in}}{\pgfqpoint{4.196696in}{1.816377in}}%
\pgfpathcurveto{\pgfqpoint{4.190873in}{1.810553in}}{\pgfqpoint{4.187600in}{1.802653in}}{\pgfqpoint{4.187600in}{1.794417in}}%
\pgfpathcurveto{\pgfqpoint{4.187600in}{1.786181in}}{\pgfqpoint{4.190873in}{1.778281in}}{\pgfqpoint{4.196696in}{1.772457in}}%
\pgfpathcurveto{\pgfqpoint{4.202520in}{1.766633in}}{\pgfqpoint{4.210420in}{1.763361in}}{\pgfqpoint{4.218657in}{1.763361in}}%
\pgfpathclose%
\pgfusepath{stroke,fill}%
\end{pgfscope}%
\begin{pgfscope}%
\pgfsetrectcap%
\pgfsetmiterjoin%
\pgfsetlinewidth{0.000000pt}%
\definecolor{currentstroke}{rgb}{1.000000,1.000000,1.000000}%
\pgfsetstrokecolor{currentstroke}%
\pgfsetdash{}{0pt}%
\pgfpathmoveto{\pgfqpoint{2.816705in}{0.516222in}}%
\pgfpathlineto{\pgfqpoint{2.816705in}{2.299750in}}%
\pgfusepath{}%
\end{pgfscope}%
\begin{pgfscope}%
\pgfsetrectcap%
\pgfsetmiterjoin%
\pgfsetlinewidth{0.000000pt}%
\definecolor{currentstroke}{rgb}{1.000000,1.000000,1.000000}%
\pgfsetstrokecolor{currentstroke}%
\pgfsetdash{}{0pt}%
\pgfpathmoveto{\pgfqpoint{2.816705in}{0.516222in}}%
\pgfpathlineto{\pgfqpoint{4.779438in}{0.516222in}}%
\pgfusepath{}%
\end{pgfscope}%
\end{pgfpicture}%
\makeatother%
\endgroup%

  \caption{Correlation between the arm length (in centimeters) and the falling
  times (in seconds) of the two realizations.}
  \label{fig_al_times}
\end{figure}

Despite of the described relations among the different variables, there are no
strong correlations among the variables of the model. Linear correlations are
just apparent but not clearly strong.

\paragraph{Linear regression.} A first model $m(X)$ was approximated by using a
linear regression. The basis functions were $\phi(x) = [1, x_0, x_0^2]$. Those
functions were based on effects on the wing length ($x_0$). However, the
coefficient of determination ($R^2$) for the obtained model was $0.237$. A
better model is obtained using the basis functions $\phi(x) = [1, x, x_3^2,
\sin(x_3), \cos(x_0)]$. $R^2$  for the new model is $0.778$ which (graphically)
fits better the data according to~\cref{fig_goodlr}.
\begin{figure}
  \centering
  %% Creator: Matplotlib, PGF backend
%%
%% To include the figure in your LaTeX document, write
%%   \input{<filename>.pgf}
%%
%% Make sure the required packages are loaded in your preamble
%%   \usepackage{pgf}
%%
%% Figures using additional raster images can only be included by \input if
%% they are in the same directory as the main LaTeX file. For loading figures
%% from other directories you can use the `import` package
%%   \usepackage{import}
%% and then include the figures with
%%   \import{<path to file>}{<filename>.pgf}
%%
%% Matplotlib used the following preamble
%%   \usepackage[utf8x]{inputenc}
%%   \usepackage[T1]{fontenc}
%%   \usepackage{cmbright}
%%
\begingroup%
\makeatletter%
\begin{pgfpicture}%
\pgfpathrectangle{\pgfpointorigin}{\pgfqpoint{6.000000in}{3.000000in}}%
\pgfusepath{use as bounding box, clip}%
\begin{pgfscope}%
\pgfsetbuttcap%
\pgfsetmiterjoin%
\definecolor{currentfill}{rgb}{1.000000,1.000000,1.000000}%
\pgfsetfillcolor{currentfill}%
\pgfsetlinewidth{0.000000pt}%
\definecolor{currentstroke}{rgb}{1.000000,1.000000,1.000000}%
\pgfsetstrokecolor{currentstroke}%
\pgfsetdash{}{0pt}%
\pgfpathmoveto{\pgfqpoint{0.000000in}{0.000000in}}%
\pgfpathlineto{\pgfqpoint{6.000000in}{0.000000in}}%
\pgfpathlineto{\pgfqpoint{6.000000in}{3.000000in}}%
\pgfpathlineto{\pgfqpoint{0.000000in}{3.000000in}}%
\pgfpathclose%
\pgfusepath{fill}%
\end{pgfscope}%
\begin{pgfscope}%
\pgfsetbuttcap%
\pgfsetmiterjoin%
\definecolor{currentfill}{rgb}{0.917647,0.917647,0.949020}%
\pgfsetfillcolor{currentfill}%
\pgfsetlinewidth{0.000000pt}%
\definecolor{currentstroke}{rgb}{0.000000,0.000000,0.000000}%
\pgfsetstrokecolor{currentstroke}%
\pgfsetstrokeopacity{0.000000}%
\pgfsetdash{}{0pt}%
\pgfpathmoveto{\pgfqpoint{0.750000in}{1.770000in}}%
\pgfpathlineto{\pgfqpoint{2.863636in}{1.770000in}}%
\pgfpathlineto{\pgfqpoint{2.863636in}{2.700000in}}%
\pgfpathlineto{\pgfqpoint{0.750000in}{2.700000in}}%
\pgfpathclose%
\pgfusepath{fill}%
\end{pgfscope}%
\begin{pgfscope}%
\pgfpathrectangle{\pgfqpoint{0.750000in}{1.770000in}}{\pgfqpoint{2.113636in}{0.930000in}} %
\pgfusepath{clip}%
\pgfsetroundcap%
\pgfsetroundjoin%
\pgfsetlinewidth{0.803000pt}%
\definecolor{currentstroke}{rgb}{1.000000,1.000000,1.000000}%
\pgfsetstrokecolor{currentstroke}%
\pgfsetdash{}{0pt}%
\pgfpathmoveto{\pgfqpoint{0.914852in}{1.770000in}}%
\pgfpathlineto{\pgfqpoint{0.914852in}{2.700000in}}%
\pgfusepath{stroke}%
\end{pgfscope}%
\begin{pgfscope}%
\pgfsetbuttcap%
\pgfsetroundjoin%
\definecolor{currentfill}{rgb}{0.150000,0.150000,0.150000}%
\pgfsetfillcolor{currentfill}%
\pgfsetlinewidth{0.803000pt}%
\definecolor{currentstroke}{rgb}{0.150000,0.150000,0.150000}%
\pgfsetstrokecolor{currentstroke}%
\pgfsetdash{}{0pt}%
\pgfsys@defobject{currentmarker}{\pgfqpoint{0.000000in}{0.000000in}}{\pgfqpoint{0.000000in}{0.000000in}}{%
\pgfpathmoveto{\pgfqpoint{0.000000in}{0.000000in}}%
\pgfpathlineto{\pgfqpoint{0.000000in}{0.000000in}}%
\pgfusepath{stroke,fill}%
}%
\begin{pgfscope}%
\pgfsys@transformshift{0.914852in}{1.770000in}%
\pgfsys@useobject{currentmarker}{}%
\end{pgfscope}%
\end{pgfscope}%
\begin{pgfscope}%
\pgfsetbuttcap%
\pgfsetroundjoin%
\definecolor{currentfill}{rgb}{0.150000,0.150000,0.150000}%
\pgfsetfillcolor{currentfill}%
\pgfsetlinewidth{0.803000pt}%
\definecolor{currentstroke}{rgb}{0.150000,0.150000,0.150000}%
\pgfsetstrokecolor{currentstroke}%
\pgfsetdash{}{0pt}%
\pgfsys@defobject{currentmarker}{\pgfqpoint{0.000000in}{0.000000in}}{\pgfqpoint{0.000000in}{0.000000in}}{%
\pgfpathmoveto{\pgfqpoint{0.000000in}{0.000000in}}%
\pgfpathlineto{\pgfqpoint{0.000000in}{0.000000in}}%
\pgfusepath{stroke,fill}%
}%
\begin{pgfscope}%
\pgfsys@transformshift{0.914852in}{2.700000in}%
\pgfsys@useobject{currentmarker}{}%
\end{pgfscope}%
\end{pgfscope}%
\begin{pgfscope}%
\definecolor{textcolor}{rgb}{0.150000,0.150000,0.150000}%
\pgfsetstrokecolor{textcolor}%
\pgfsetfillcolor{textcolor}%
\pgftext[x=0.914852in,y=1.692222in,,top]{\color{textcolor}\sffamily\fontsize{8.000000}{9.600000}\selectfont 3.5}%
\end{pgfscope}%
\begin{pgfscope}%
\pgfpathrectangle{\pgfqpoint{0.750000in}{1.770000in}}{\pgfqpoint{2.113636in}{0.930000in}} %
\pgfusepath{clip}%
\pgfsetroundcap%
\pgfsetroundjoin%
\pgfsetlinewidth{0.803000pt}%
\definecolor{currentstroke}{rgb}{1.000000,1.000000,1.000000}%
\pgfsetstrokecolor{currentstroke}%
\pgfsetdash{}{0pt}%
\pgfpathmoveto{\pgfqpoint{1.209230in}{1.770000in}}%
\pgfpathlineto{\pgfqpoint{1.209230in}{2.700000in}}%
\pgfusepath{stroke}%
\end{pgfscope}%
\begin{pgfscope}%
\pgfsetbuttcap%
\pgfsetroundjoin%
\definecolor{currentfill}{rgb}{0.150000,0.150000,0.150000}%
\pgfsetfillcolor{currentfill}%
\pgfsetlinewidth{0.803000pt}%
\definecolor{currentstroke}{rgb}{0.150000,0.150000,0.150000}%
\pgfsetstrokecolor{currentstroke}%
\pgfsetdash{}{0pt}%
\pgfsys@defobject{currentmarker}{\pgfqpoint{0.000000in}{0.000000in}}{\pgfqpoint{0.000000in}{0.000000in}}{%
\pgfpathmoveto{\pgfqpoint{0.000000in}{0.000000in}}%
\pgfpathlineto{\pgfqpoint{0.000000in}{0.000000in}}%
\pgfusepath{stroke,fill}%
}%
\begin{pgfscope}%
\pgfsys@transformshift{1.209230in}{1.770000in}%
\pgfsys@useobject{currentmarker}{}%
\end{pgfscope}%
\end{pgfscope}%
\begin{pgfscope}%
\pgfsetbuttcap%
\pgfsetroundjoin%
\definecolor{currentfill}{rgb}{0.150000,0.150000,0.150000}%
\pgfsetfillcolor{currentfill}%
\pgfsetlinewidth{0.803000pt}%
\definecolor{currentstroke}{rgb}{0.150000,0.150000,0.150000}%
\pgfsetstrokecolor{currentstroke}%
\pgfsetdash{}{0pt}%
\pgfsys@defobject{currentmarker}{\pgfqpoint{0.000000in}{0.000000in}}{\pgfqpoint{0.000000in}{0.000000in}}{%
\pgfpathmoveto{\pgfqpoint{0.000000in}{0.000000in}}%
\pgfpathlineto{\pgfqpoint{0.000000in}{0.000000in}}%
\pgfusepath{stroke,fill}%
}%
\begin{pgfscope}%
\pgfsys@transformshift{1.209230in}{2.700000in}%
\pgfsys@useobject{currentmarker}{}%
\end{pgfscope}%
\end{pgfscope}%
\begin{pgfscope}%
\definecolor{textcolor}{rgb}{0.150000,0.150000,0.150000}%
\pgfsetstrokecolor{textcolor}%
\pgfsetfillcolor{textcolor}%
\pgftext[x=1.209230in,y=1.692222in,,top]{\color{textcolor}\sffamily\fontsize{8.000000}{9.600000}\selectfont 4.0}%
\end{pgfscope}%
\begin{pgfscope}%
\pgfpathrectangle{\pgfqpoint{0.750000in}{1.770000in}}{\pgfqpoint{2.113636in}{0.930000in}} %
\pgfusepath{clip}%
\pgfsetroundcap%
\pgfsetroundjoin%
\pgfsetlinewidth{0.803000pt}%
\definecolor{currentstroke}{rgb}{1.000000,1.000000,1.000000}%
\pgfsetstrokecolor{currentstroke}%
\pgfsetdash{}{0pt}%
\pgfpathmoveto{\pgfqpoint{1.503609in}{1.770000in}}%
\pgfpathlineto{\pgfqpoint{1.503609in}{2.700000in}}%
\pgfusepath{stroke}%
\end{pgfscope}%
\begin{pgfscope}%
\pgfsetbuttcap%
\pgfsetroundjoin%
\definecolor{currentfill}{rgb}{0.150000,0.150000,0.150000}%
\pgfsetfillcolor{currentfill}%
\pgfsetlinewidth{0.803000pt}%
\definecolor{currentstroke}{rgb}{0.150000,0.150000,0.150000}%
\pgfsetstrokecolor{currentstroke}%
\pgfsetdash{}{0pt}%
\pgfsys@defobject{currentmarker}{\pgfqpoint{0.000000in}{0.000000in}}{\pgfqpoint{0.000000in}{0.000000in}}{%
\pgfpathmoveto{\pgfqpoint{0.000000in}{0.000000in}}%
\pgfpathlineto{\pgfqpoint{0.000000in}{0.000000in}}%
\pgfusepath{stroke,fill}%
}%
\begin{pgfscope}%
\pgfsys@transformshift{1.503609in}{1.770000in}%
\pgfsys@useobject{currentmarker}{}%
\end{pgfscope}%
\end{pgfscope}%
\begin{pgfscope}%
\pgfsetbuttcap%
\pgfsetroundjoin%
\definecolor{currentfill}{rgb}{0.150000,0.150000,0.150000}%
\pgfsetfillcolor{currentfill}%
\pgfsetlinewidth{0.803000pt}%
\definecolor{currentstroke}{rgb}{0.150000,0.150000,0.150000}%
\pgfsetstrokecolor{currentstroke}%
\pgfsetdash{}{0pt}%
\pgfsys@defobject{currentmarker}{\pgfqpoint{0.000000in}{0.000000in}}{\pgfqpoint{0.000000in}{0.000000in}}{%
\pgfpathmoveto{\pgfqpoint{0.000000in}{0.000000in}}%
\pgfpathlineto{\pgfqpoint{0.000000in}{0.000000in}}%
\pgfusepath{stroke,fill}%
}%
\begin{pgfscope}%
\pgfsys@transformshift{1.503609in}{2.700000in}%
\pgfsys@useobject{currentmarker}{}%
\end{pgfscope}%
\end{pgfscope}%
\begin{pgfscope}%
\definecolor{textcolor}{rgb}{0.150000,0.150000,0.150000}%
\pgfsetstrokecolor{textcolor}%
\pgfsetfillcolor{textcolor}%
\pgftext[x=1.503609in,y=1.692222in,,top]{\color{textcolor}\sffamily\fontsize{8.000000}{9.600000}\selectfont 4.5}%
\end{pgfscope}%
\begin{pgfscope}%
\pgfpathrectangle{\pgfqpoint{0.750000in}{1.770000in}}{\pgfqpoint{2.113636in}{0.930000in}} %
\pgfusepath{clip}%
\pgfsetroundcap%
\pgfsetroundjoin%
\pgfsetlinewidth{0.803000pt}%
\definecolor{currentstroke}{rgb}{1.000000,1.000000,1.000000}%
\pgfsetstrokecolor{currentstroke}%
\pgfsetdash{}{0pt}%
\pgfpathmoveto{\pgfqpoint{1.797987in}{1.770000in}}%
\pgfpathlineto{\pgfqpoint{1.797987in}{2.700000in}}%
\pgfusepath{stroke}%
\end{pgfscope}%
\begin{pgfscope}%
\pgfsetbuttcap%
\pgfsetroundjoin%
\definecolor{currentfill}{rgb}{0.150000,0.150000,0.150000}%
\pgfsetfillcolor{currentfill}%
\pgfsetlinewidth{0.803000pt}%
\definecolor{currentstroke}{rgb}{0.150000,0.150000,0.150000}%
\pgfsetstrokecolor{currentstroke}%
\pgfsetdash{}{0pt}%
\pgfsys@defobject{currentmarker}{\pgfqpoint{0.000000in}{0.000000in}}{\pgfqpoint{0.000000in}{0.000000in}}{%
\pgfpathmoveto{\pgfqpoint{0.000000in}{0.000000in}}%
\pgfpathlineto{\pgfqpoint{0.000000in}{0.000000in}}%
\pgfusepath{stroke,fill}%
}%
\begin{pgfscope}%
\pgfsys@transformshift{1.797987in}{1.770000in}%
\pgfsys@useobject{currentmarker}{}%
\end{pgfscope}%
\end{pgfscope}%
\begin{pgfscope}%
\pgfsetbuttcap%
\pgfsetroundjoin%
\definecolor{currentfill}{rgb}{0.150000,0.150000,0.150000}%
\pgfsetfillcolor{currentfill}%
\pgfsetlinewidth{0.803000pt}%
\definecolor{currentstroke}{rgb}{0.150000,0.150000,0.150000}%
\pgfsetstrokecolor{currentstroke}%
\pgfsetdash{}{0pt}%
\pgfsys@defobject{currentmarker}{\pgfqpoint{0.000000in}{0.000000in}}{\pgfqpoint{0.000000in}{0.000000in}}{%
\pgfpathmoveto{\pgfqpoint{0.000000in}{0.000000in}}%
\pgfpathlineto{\pgfqpoint{0.000000in}{0.000000in}}%
\pgfusepath{stroke,fill}%
}%
\begin{pgfscope}%
\pgfsys@transformshift{1.797987in}{2.700000in}%
\pgfsys@useobject{currentmarker}{}%
\end{pgfscope}%
\end{pgfscope}%
\begin{pgfscope}%
\definecolor{textcolor}{rgb}{0.150000,0.150000,0.150000}%
\pgfsetstrokecolor{textcolor}%
\pgfsetfillcolor{textcolor}%
\pgftext[x=1.797987in,y=1.692222in,,top]{\color{textcolor}\sffamily\fontsize{8.000000}{9.600000}\selectfont 5.0}%
\end{pgfscope}%
\begin{pgfscope}%
\pgfpathrectangle{\pgfqpoint{0.750000in}{1.770000in}}{\pgfqpoint{2.113636in}{0.930000in}} %
\pgfusepath{clip}%
\pgfsetroundcap%
\pgfsetroundjoin%
\pgfsetlinewidth{0.803000pt}%
\definecolor{currentstroke}{rgb}{1.000000,1.000000,1.000000}%
\pgfsetstrokecolor{currentstroke}%
\pgfsetdash{}{0pt}%
\pgfpathmoveto{\pgfqpoint{2.092365in}{1.770000in}}%
\pgfpathlineto{\pgfqpoint{2.092365in}{2.700000in}}%
\pgfusepath{stroke}%
\end{pgfscope}%
\begin{pgfscope}%
\pgfsetbuttcap%
\pgfsetroundjoin%
\definecolor{currentfill}{rgb}{0.150000,0.150000,0.150000}%
\pgfsetfillcolor{currentfill}%
\pgfsetlinewidth{0.803000pt}%
\definecolor{currentstroke}{rgb}{0.150000,0.150000,0.150000}%
\pgfsetstrokecolor{currentstroke}%
\pgfsetdash{}{0pt}%
\pgfsys@defobject{currentmarker}{\pgfqpoint{0.000000in}{0.000000in}}{\pgfqpoint{0.000000in}{0.000000in}}{%
\pgfpathmoveto{\pgfqpoint{0.000000in}{0.000000in}}%
\pgfpathlineto{\pgfqpoint{0.000000in}{0.000000in}}%
\pgfusepath{stroke,fill}%
}%
\begin{pgfscope}%
\pgfsys@transformshift{2.092365in}{1.770000in}%
\pgfsys@useobject{currentmarker}{}%
\end{pgfscope}%
\end{pgfscope}%
\begin{pgfscope}%
\pgfsetbuttcap%
\pgfsetroundjoin%
\definecolor{currentfill}{rgb}{0.150000,0.150000,0.150000}%
\pgfsetfillcolor{currentfill}%
\pgfsetlinewidth{0.803000pt}%
\definecolor{currentstroke}{rgb}{0.150000,0.150000,0.150000}%
\pgfsetstrokecolor{currentstroke}%
\pgfsetdash{}{0pt}%
\pgfsys@defobject{currentmarker}{\pgfqpoint{0.000000in}{0.000000in}}{\pgfqpoint{0.000000in}{0.000000in}}{%
\pgfpathmoveto{\pgfqpoint{0.000000in}{0.000000in}}%
\pgfpathlineto{\pgfqpoint{0.000000in}{0.000000in}}%
\pgfusepath{stroke,fill}%
}%
\begin{pgfscope}%
\pgfsys@transformshift{2.092365in}{2.700000in}%
\pgfsys@useobject{currentmarker}{}%
\end{pgfscope}%
\end{pgfscope}%
\begin{pgfscope}%
\definecolor{textcolor}{rgb}{0.150000,0.150000,0.150000}%
\pgfsetstrokecolor{textcolor}%
\pgfsetfillcolor{textcolor}%
\pgftext[x=2.092365in,y=1.692222in,,top]{\color{textcolor}\sffamily\fontsize{8.000000}{9.600000}\selectfont 5.5}%
\end{pgfscope}%
\begin{pgfscope}%
\pgfpathrectangle{\pgfqpoint{0.750000in}{1.770000in}}{\pgfqpoint{2.113636in}{0.930000in}} %
\pgfusepath{clip}%
\pgfsetroundcap%
\pgfsetroundjoin%
\pgfsetlinewidth{0.803000pt}%
\definecolor{currentstroke}{rgb}{1.000000,1.000000,1.000000}%
\pgfsetstrokecolor{currentstroke}%
\pgfsetdash{}{0pt}%
\pgfpathmoveto{\pgfqpoint{2.386743in}{1.770000in}}%
\pgfpathlineto{\pgfqpoint{2.386743in}{2.700000in}}%
\pgfusepath{stroke}%
\end{pgfscope}%
\begin{pgfscope}%
\pgfsetbuttcap%
\pgfsetroundjoin%
\definecolor{currentfill}{rgb}{0.150000,0.150000,0.150000}%
\pgfsetfillcolor{currentfill}%
\pgfsetlinewidth{0.803000pt}%
\definecolor{currentstroke}{rgb}{0.150000,0.150000,0.150000}%
\pgfsetstrokecolor{currentstroke}%
\pgfsetdash{}{0pt}%
\pgfsys@defobject{currentmarker}{\pgfqpoint{0.000000in}{0.000000in}}{\pgfqpoint{0.000000in}{0.000000in}}{%
\pgfpathmoveto{\pgfqpoint{0.000000in}{0.000000in}}%
\pgfpathlineto{\pgfqpoint{0.000000in}{0.000000in}}%
\pgfusepath{stroke,fill}%
}%
\begin{pgfscope}%
\pgfsys@transformshift{2.386743in}{1.770000in}%
\pgfsys@useobject{currentmarker}{}%
\end{pgfscope}%
\end{pgfscope}%
\begin{pgfscope}%
\pgfsetbuttcap%
\pgfsetroundjoin%
\definecolor{currentfill}{rgb}{0.150000,0.150000,0.150000}%
\pgfsetfillcolor{currentfill}%
\pgfsetlinewidth{0.803000pt}%
\definecolor{currentstroke}{rgb}{0.150000,0.150000,0.150000}%
\pgfsetstrokecolor{currentstroke}%
\pgfsetdash{}{0pt}%
\pgfsys@defobject{currentmarker}{\pgfqpoint{0.000000in}{0.000000in}}{\pgfqpoint{0.000000in}{0.000000in}}{%
\pgfpathmoveto{\pgfqpoint{0.000000in}{0.000000in}}%
\pgfpathlineto{\pgfqpoint{0.000000in}{0.000000in}}%
\pgfusepath{stroke,fill}%
}%
\begin{pgfscope}%
\pgfsys@transformshift{2.386743in}{2.700000in}%
\pgfsys@useobject{currentmarker}{}%
\end{pgfscope}%
\end{pgfscope}%
\begin{pgfscope}%
\definecolor{textcolor}{rgb}{0.150000,0.150000,0.150000}%
\pgfsetstrokecolor{textcolor}%
\pgfsetfillcolor{textcolor}%
\pgftext[x=2.386743in,y=1.692222in,,top]{\color{textcolor}\sffamily\fontsize{8.000000}{9.600000}\selectfont 6.0}%
\end{pgfscope}%
\begin{pgfscope}%
\pgfpathrectangle{\pgfqpoint{0.750000in}{1.770000in}}{\pgfqpoint{2.113636in}{0.930000in}} %
\pgfusepath{clip}%
\pgfsetroundcap%
\pgfsetroundjoin%
\pgfsetlinewidth{0.803000pt}%
\definecolor{currentstroke}{rgb}{1.000000,1.000000,1.000000}%
\pgfsetstrokecolor{currentstroke}%
\pgfsetdash{}{0pt}%
\pgfpathmoveto{\pgfqpoint{2.681122in}{1.770000in}}%
\pgfpathlineto{\pgfqpoint{2.681122in}{2.700000in}}%
\pgfusepath{stroke}%
\end{pgfscope}%
\begin{pgfscope}%
\pgfsetbuttcap%
\pgfsetroundjoin%
\definecolor{currentfill}{rgb}{0.150000,0.150000,0.150000}%
\pgfsetfillcolor{currentfill}%
\pgfsetlinewidth{0.803000pt}%
\definecolor{currentstroke}{rgb}{0.150000,0.150000,0.150000}%
\pgfsetstrokecolor{currentstroke}%
\pgfsetdash{}{0pt}%
\pgfsys@defobject{currentmarker}{\pgfqpoint{0.000000in}{0.000000in}}{\pgfqpoint{0.000000in}{0.000000in}}{%
\pgfpathmoveto{\pgfqpoint{0.000000in}{0.000000in}}%
\pgfpathlineto{\pgfqpoint{0.000000in}{0.000000in}}%
\pgfusepath{stroke,fill}%
}%
\begin{pgfscope}%
\pgfsys@transformshift{2.681122in}{1.770000in}%
\pgfsys@useobject{currentmarker}{}%
\end{pgfscope}%
\end{pgfscope}%
\begin{pgfscope}%
\pgfsetbuttcap%
\pgfsetroundjoin%
\definecolor{currentfill}{rgb}{0.150000,0.150000,0.150000}%
\pgfsetfillcolor{currentfill}%
\pgfsetlinewidth{0.803000pt}%
\definecolor{currentstroke}{rgb}{0.150000,0.150000,0.150000}%
\pgfsetstrokecolor{currentstroke}%
\pgfsetdash{}{0pt}%
\pgfsys@defobject{currentmarker}{\pgfqpoint{0.000000in}{0.000000in}}{\pgfqpoint{0.000000in}{0.000000in}}{%
\pgfpathmoveto{\pgfqpoint{0.000000in}{0.000000in}}%
\pgfpathlineto{\pgfqpoint{0.000000in}{0.000000in}}%
\pgfusepath{stroke,fill}%
}%
\begin{pgfscope}%
\pgfsys@transformshift{2.681122in}{2.700000in}%
\pgfsys@useobject{currentmarker}{}%
\end{pgfscope}%
\end{pgfscope}%
\begin{pgfscope}%
\definecolor{textcolor}{rgb}{0.150000,0.150000,0.150000}%
\pgfsetstrokecolor{textcolor}%
\pgfsetfillcolor{textcolor}%
\pgftext[x=2.681122in,y=1.692222in,,top]{\color{textcolor}\sffamily\fontsize{8.000000}{9.600000}\selectfont 6.5}%
\end{pgfscope}%
\begin{pgfscope}%
\definecolor{textcolor}{rgb}{0.150000,0.150000,0.150000}%
\pgfsetstrokecolor{textcolor}%
\pgfsetfillcolor{textcolor}%
\pgftext[x=1.806818in,y=1.527099in,,top]{\color{textcolor}\sffamily\fontsize{8.800000}{10.560000}\selectfont Wing length}%
\end{pgfscope}%
\begin{pgfscope}%
\pgfpathrectangle{\pgfqpoint{0.750000in}{1.770000in}}{\pgfqpoint{2.113636in}{0.930000in}} %
\pgfusepath{clip}%
\pgfsetroundcap%
\pgfsetroundjoin%
\pgfsetlinewidth{0.803000pt}%
\definecolor{currentstroke}{rgb}{1.000000,1.000000,1.000000}%
\pgfsetstrokecolor{currentstroke}%
\pgfsetdash{}{0pt}%
\pgfpathmoveto{\pgfqpoint{0.750000in}{1.770000in}}%
\pgfpathlineto{\pgfqpoint{2.863636in}{1.770000in}}%
\pgfusepath{stroke}%
\end{pgfscope}%
\begin{pgfscope}%
\pgfsetbuttcap%
\pgfsetroundjoin%
\definecolor{currentfill}{rgb}{0.150000,0.150000,0.150000}%
\pgfsetfillcolor{currentfill}%
\pgfsetlinewidth{0.803000pt}%
\definecolor{currentstroke}{rgb}{0.150000,0.150000,0.150000}%
\pgfsetstrokecolor{currentstroke}%
\pgfsetdash{}{0pt}%
\pgfsys@defobject{currentmarker}{\pgfqpoint{0.000000in}{0.000000in}}{\pgfqpoint{0.000000in}{0.000000in}}{%
\pgfpathmoveto{\pgfqpoint{0.000000in}{0.000000in}}%
\pgfpathlineto{\pgfqpoint{0.000000in}{0.000000in}}%
\pgfusepath{stroke,fill}%
}%
\begin{pgfscope}%
\pgfsys@transformshift{0.750000in}{1.770000in}%
\pgfsys@useobject{currentmarker}{}%
\end{pgfscope}%
\end{pgfscope}%
\begin{pgfscope}%
\pgfsetbuttcap%
\pgfsetroundjoin%
\definecolor{currentfill}{rgb}{0.150000,0.150000,0.150000}%
\pgfsetfillcolor{currentfill}%
\pgfsetlinewidth{0.803000pt}%
\definecolor{currentstroke}{rgb}{0.150000,0.150000,0.150000}%
\pgfsetstrokecolor{currentstroke}%
\pgfsetdash{}{0pt}%
\pgfsys@defobject{currentmarker}{\pgfqpoint{0.000000in}{0.000000in}}{\pgfqpoint{0.000000in}{0.000000in}}{%
\pgfpathmoveto{\pgfqpoint{0.000000in}{0.000000in}}%
\pgfpathlineto{\pgfqpoint{0.000000in}{0.000000in}}%
\pgfusepath{stroke,fill}%
}%
\begin{pgfscope}%
\pgfsys@transformshift{2.863636in}{1.770000in}%
\pgfsys@useobject{currentmarker}{}%
\end{pgfscope}%
\end{pgfscope}%
\begin{pgfscope}%
\definecolor{textcolor}{rgb}{0.150000,0.150000,0.150000}%
\pgfsetstrokecolor{textcolor}%
\pgfsetfillcolor{textcolor}%
\pgftext[x=0.672222in,y=1.770000in,right,]{\color{textcolor}\sffamily\fontsize{8.000000}{9.600000}\selectfont 2.0}%
\end{pgfscope}%
\begin{pgfscope}%
\pgfpathrectangle{\pgfqpoint{0.750000in}{1.770000in}}{\pgfqpoint{2.113636in}{0.930000in}} %
\pgfusepath{clip}%
\pgfsetroundcap%
\pgfsetroundjoin%
\pgfsetlinewidth{0.803000pt}%
\definecolor{currentstroke}{rgb}{1.000000,1.000000,1.000000}%
\pgfsetstrokecolor{currentstroke}%
\pgfsetdash{}{0pt}%
\pgfpathmoveto{\pgfqpoint{0.750000in}{1.925000in}}%
\pgfpathlineto{\pgfqpoint{2.863636in}{1.925000in}}%
\pgfusepath{stroke}%
\end{pgfscope}%
\begin{pgfscope}%
\pgfsetbuttcap%
\pgfsetroundjoin%
\definecolor{currentfill}{rgb}{0.150000,0.150000,0.150000}%
\pgfsetfillcolor{currentfill}%
\pgfsetlinewidth{0.803000pt}%
\definecolor{currentstroke}{rgb}{0.150000,0.150000,0.150000}%
\pgfsetstrokecolor{currentstroke}%
\pgfsetdash{}{0pt}%
\pgfsys@defobject{currentmarker}{\pgfqpoint{0.000000in}{0.000000in}}{\pgfqpoint{0.000000in}{0.000000in}}{%
\pgfpathmoveto{\pgfqpoint{0.000000in}{0.000000in}}%
\pgfpathlineto{\pgfqpoint{0.000000in}{0.000000in}}%
\pgfusepath{stroke,fill}%
}%
\begin{pgfscope}%
\pgfsys@transformshift{0.750000in}{1.925000in}%
\pgfsys@useobject{currentmarker}{}%
\end{pgfscope}%
\end{pgfscope}%
\begin{pgfscope}%
\pgfsetbuttcap%
\pgfsetroundjoin%
\definecolor{currentfill}{rgb}{0.150000,0.150000,0.150000}%
\pgfsetfillcolor{currentfill}%
\pgfsetlinewidth{0.803000pt}%
\definecolor{currentstroke}{rgb}{0.150000,0.150000,0.150000}%
\pgfsetstrokecolor{currentstroke}%
\pgfsetdash{}{0pt}%
\pgfsys@defobject{currentmarker}{\pgfqpoint{0.000000in}{0.000000in}}{\pgfqpoint{0.000000in}{0.000000in}}{%
\pgfpathmoveto{\pgfqpoint{0.000000in}{0.000000in}}%
\pgfpathlineto{\pgfqpoint{0.000000in}{0.000000in}}%
\pgfusepath{stroke,fill}%
}%
\begin{pgfscope}%
\pgfsys@transformshift{2.863636in}{1.925000in}%
\pgfsys@useobject{currentmarker}{}%
\end{pgfscope}%
\end{pgfscope}%
\begin{pgfscope}%
\definecolor{textcolor}{rgb}{0.150000,0.150000,0.150000}%
\pgfsetstrokecolor{textcolor}%
\pgfsetfillcolor{textcolor}%
\pgftext[x=0.672222in,y=1.925000in,right,]{\color{textcolor}\sffamily\fontsize{8.000000}{9.600000}\selectfont 2.5}%
\end{pgfscope}%
\begin{pgfscope}%
\pgfpathrectangle{\pgfqpoint{0.750000in}{1.770000in}}{\pgfqpoint{2.113636in}{0.930000in}} %
\pgfusepath{clip}%
\pgfsetroundcap%
\pgfsetroundjoin%
\pgfsetlinewidth{0.803000pt}%
\definecolor{currentstroke}{rgb}{1.000000,1.000000,1.000000}%
\pgfsetstrokecolor{currentstroke}%
\pgfsetdash{}{0pt}%
\pgfpathmoveto{\pgfqpoint{0.750000in}{2.080000in}}%
\pgfpathlineto{\pgfqpoint{2.863636in}{2.080000in}}%
\pgfusepath{stroke}%
\end{pgfscope}%
\begin{pgfscope}%
\pgfsetbuttcap%
\pgfsetroundjoin%
\definecolor{currentfill}{rgb}{0.150000,0.150000,0.150000}%
\pgfsetfillcolor{currentfill}%
\pgfsetlinewidth{0.803000pt}%
\definecolor{currentstroke}{rgb}{0.150000,0.150000,0.150000}%
\pgfsetstrokecolor{currentstroke}%
\pgfsetdash{}{0pt}%
\pgfsys@defobject{currentmarker}{\pgfqpoint{0.000000in}{0.000000in}}{\pgfqpoint{0.000000in}{0.000000in}}{%
\pgfpathmoveto{\pgfqpoint{0.000000in}{0.000000in}}%
\pgfpathlineto{\pgfqpoint{0.000000in}{0.000000in}}%
\pgfusepath{stroke,fill}%
}%
\begin{pgfscope}%
\pgfsys@transformshift{0.750000in}{2.080000in}%
\pgfsys@useobject{currentmarker}{}%
\end{pgfscope}%
\end{pgfscope}%
\begin{pgfscope}%
\pgfsetbuttcap%
\pgfsetroundjoin%
\definecolor{currentfill}{rgb}{0.150000,0.150000,0.150000}%
\pgfsetfillcolor{currentfill}%
\pgfsetlinewidth{0.803000pt}%
\definecolor{currentstroke}{rgb}{0.150000,0.150000,0.150000}%
\pgfsetstrokecolor{currentstroke}%
\pgfsetdash{}{0pt}%
\pgfsys@defobject{currentmarker}{\pgfqpoint{0.000000in}{0.000000in}}{\pgfqpoint{0.000000in}{0.000000in}}{%
\pgfpathmoveto{\pgfqpoint{0.000000in}{0.000000in}}%
\pgfpathlineto{\pgfqpoint{0.000000in}{0.000000in}}%
\pgfusepath{stroke,fill}%
}%
\begin{pgfscope}%
\pgfsys@transformshift{2.863636in}{2.080000in}%
\pgfsys@useobject{currentmarker}{}%
\end{pgfscope}%
\end{pgfscope}%
\begin{pgfscope}%
\definecolor{textcolor}{rgb}{0.150000,0.150000,0.150000}%
\pgfsetstrokecolor{textcolor}%
\pgfsetfillcolor{textcolor}%
\pgftext[x=0.672222in,y=2.080000in,right,]{\color{textcolor}\sffamily\fontsize{8.000000}{9.600000}\selectfont 3.0}%
\end{pgfscope}%
\begin{pgfscope}%
\pgfpathrectangle{\pgfqpoint{0.750000in}{1.770000in}}{\pgfqpoint{2.113636in}{0.930000in}} %
\pgfusepath{clip}%
\pgfsetroundcap%
\pgfsetroundjoin%
\pgfsetlinewidth{0.803000pt}%
\definecolor{currentstroke}{rgb}{1.000000,1.000000,1.000000}%
\pgfsetstrokecolor{currentstroke}%
\pgfsetdash{}{0pt}%
\pgfpathmoveto{\pgfqpoint{0.750000in}{2.235000in}}%
\pgfpathlineto{\pgfqpoint{2.863636in}{2.235000in}}%
\pgfusepath{stroke}%
\end{pgfscope}%
\begin{pgfscope}%
\pgfsetbuttcap%
\pgfsetroundjoin%
\definecolor{currentfill}{rgb}{0.150000,0.150000,0.150000}%
\pgfsetfillcolor{currentfill}%
\pgfsetlinewidth{0.803000pt}%
\definecolor{currentstroke}{rgb}{0.150000,0.150000,0.150000}%
\pgfsetstrokecolor{currentstroke}%
\pgfsetdash{}{0pt}%
\pgfsys@defobject{currentmarker}{\pgfqpoint{0.000000in}{0.000000in}}{\pgfqpoint{0.000000in}{0.000000in}}{%
\pgfpathmoveto{\pgfqpoint{0.000000in}{0.000000in}}%
\pgfpathlineto{\pgfqpoint{0.000000in}{0.000000in}}%
\pgfusepath{stroke,fill}%
}%
\begin{pgfscope}%
\pgfsys@transformshift{0.750000in}{2.235000in}%
\pgfsys@useobject{currentmarker}{}%
\end{pgfscope}%
\end{pgfscope}%
\begin{pgfscope}%
\pgfsetbuttcap%
\pgfsetroundjoin%
\definecolor{currentfill}{rgb}{0.150000,0.150000,0.150000}%
\pgfsetfillcolor{currentfill}%
\pgfsetlinewidth{0.803000pt}%
\definecolor{currentstroke}{rgb}{0.150000,0.150000,0.150000}%
\pgfsetstrokecolor{currentstroke}%
\pgfsetdash{}{0pt}%
\pgfsys@defobject{currentmarker}{\pgfqpoint{0.000000in}{0.000000in}}{\pgfqpoint{0.000000in}{0.000000in}}{%
\pgfpathmoveto{\pgfqpoint{0.000000in}{0.000000in}}%
\pgfpathlineto{\pgfqpoint{0.000000in}{0.000000in}}%
\pgfusepath{stroke,fill}%
}%
\begin{pgfscope}%
\pgfsys@transformshift{2.863636in}{2.235000in}%
\pgfsys@useobject{currentmarker}{}%
\end{pgfscope}%
\end{pgfscope}%
\begin{pgfscope}%
\definecolor{textcolor}{rgb}{0.150000,0.150000,0.150000}%
\pgfsetstrokecolor{textcolor}%
\pgfsetfillcolor{textcolor}%
\pgftext[x=0.672222in,y=2.235000in,right,]{\color{textcolor}\sffamily\fontsize{8.000000}{9.600000}\selectfont 3.5}%
\end{pgfscope}%
\begin{pgfscope}%
\pgfpathrectangle{\pgfqpoint{0.750000in}{1.770000in}}{\pgfqpoint{2.113636in}{0.930000in}} %
\pgfusepath{clip}%
\pgfsetroundcap%
\pgfsetroundjoin%
\pgfsetlinewidth{0.803000pt}%
\definecolor{currentstroke}{rgb}{1.000000,1.000000,1.000000}%
\pgfsetstrokecolor{currentstroke}%
\pgfsetdash{}{0pt}%
\pgfpathmoveto{\pgfqpoint{0.750000in}{2.390000in}}%
\pgfpathlineto{\pgfqpoint{2.863636in}{2.390000in}}%
\pgfusepath{stroke}%
\end{pgfscope}%
\begin{pgfscope}%
\pgfsetbuttcap%
\pgfsetroundjoin%
\definecolor{currentfill}{rgb}{0.150000,0.150000,0.150000}%
\pgfsetfillcolor{currentfill}%
\pgfsetlinewidth{0.803000pt}%
\definecolor{currentstroke}{rgb}{0.150000,0.150000,0.150000}%
\pgfsetstrokecolor{currentstroke}%
\pgfsetdash{}{0pt}%
\pgfsys@defobject{currentmarker}{\pgfqpoint{0.000000in}{0.000000in}}{\pgfqpoint{0.000000in}{0.000000in}}{%
\pgfpathmoveto{\pgfqpoint{0.000000in}{0.000000in}}%
\pgfpathlineto{\pgfqpoint{0.000000in}{0.000000in}}%
\pgfusepath{stroke,fill}%
}%
\begin{pgfscope}%
\pgfsys@transformshift{0.750000in}{2.390000in}%
\pgfsys@useobject{currentmarker}{}%
\end{pgfscope}%
\end{pgfscope}%
\begin{pgfscope}%
\pgfsetbuttcap%
\pgfsetroundjoin%
\definecolor{currentfill}{rgb}{0.150000,0.150000,0.150000}%
\pgfsetfillcolor{currentfill}%
\pgfsetlinewidth{0.803000pt}%
\definecolor{currentstroke}{rgb}{0.150000,0.150000,0.150000}%
\pgfsetstrokecolor{currentstroke}%
\pgfsetdash{}{0pt}%
\pgfsys@defobject{currentmarker}{\pgfqpoint{0.000000in}{0.000000in}}{\pgfqpoint{0.000000in}{0.000000in}}{%
\pgfpathmoveto{\pgfqpoint{0.000000in}{0.000000in}}%
\pgfpathlineto{\pgfqpoint{0.000000in}{0.000000in}}%
\pgfusepath{stroke,fill}%
}%
\begin{pgfscope}%
\pgfsys@transformshift{2.863636in}{2.390000in}%
\pgfsys@useobject{currentmarker}{}%
\end{pgfscope}%
\end{pgfscope}%
\begin{pgfscope}%
\definecolor{textcolor}{rgb}{0.150000,0.150000,0.150000}%
\pgfsetstrokecolor{textcolor}%
\pgfsetfillcolor{textcolor}%
\pgftext[x=0.672222in,y=2.390000in,right,]{\color{textcolor}\sffamily\fontsize{8.000000}{9.600000}\selectfont 4.0}%
\end{pgfscope}%
\begin{pgfscope}%
\pgfpathrectangle{\pgfqpoint{0.750000in}{1.770000in}}{\pgfqpoint{2.113636in}{0.930000in}} %
\pgfusepath{clip}%
\pgfsetroundcap%
\pgfsetroundjoin%
\pgfsetlinewidth{0.803000pt}%
\definecolor{currentstroke}{rgb}{1.000000,1.000000,1.000000}%
\pgfsetstrokecolor{currentstroke}%
\pgfsetdash{}{0pt}%
\pgfpathmoveto{\pgfqpoint{0.750000in}{2.545000in}}%
\pgfpathlineto{\pgfqpoint{2.863636in}{2.545000in}}%
\pgfusepath{stroke}%
\end{pgfscope}%
\begin{pgfscope}%
\pgfsetbuttcap%
\pgfsetroundjoin%
\definecolor{currentfill}{rgb}{0.150000,0.150000,0.150000}%
\pgfsetfillcolor{currentfill}%
\pgfsetlinewidth{0.803000pt}%
\definecolor{currentstroke}{rgb}{0.150000,0.150000,0.150000}%
\pgfsetstrokecolor{currentstroke}%
\pgfsetdash{}{0pt}%
\pgfsys@defobject{currentmarker}{\pgfqpoint{0.000000in}{0.000000in}}{\pgfqpoint{0.000000in}{0.000000in}}{%
\pgfpathmoveto{\pgfqpoint{0.000000in}{0.000000in}}%
\pgfpathlineto{\pgfqpoint{0.000000in}{0.000000in}}%
\pgfusepath{stroke,fill}%
}%
\begin{pgfscope}%
\pgfsys@transformshift{0.750000in}{2.545000in}%
\pgfsys@useobject{currentmarker}{}%
\end{pgfscope}%
\end{pgfscope}%
\begin{pgfscope}%
\pgfsetbuttcap%
\pgfsetroundjoin%
\definecolor{currentfill}{rgb}{0.150000,0.150000,0.150000}%
\pgfsetfillcolor{currentfill}%
\pgfsetlinewidth{0.803000pt}%
\definecolor{currentstroke}{rgb}{0.150000,0.150000,0.150000}%
\pgfsetstrokecolor{currentstroke}%
\pgfsetdash{}{0pt}%
\pgfsys@defobject{currentmarker}{\pgfqpoint{0.000000in}{0.000000in}}{\pgfqpoint{0.000000in}{0.000000in}}{%
\pgfpathmoveto{\pgfqpoint{0.000000in}{0.000000in}}%
\pgfpathlineto{\pgfqpoint{0.000000in}{0.000000in}}%
\pgfusepath{stroke,fill}%
}%
\begin{pgfscope}%
\pgfsys@transformshift{2.863636in}{2.545000in}%
\pgfsys@useobject{currentmarker}{}%
\end{pgfscope}%
\end{pgfscope}%
\begin{pgfscope}%
\definecolor{textcolor}{rgb}{0.150000,0.150000,0.150000}%
\pgfsetstrokecolor{textcolor}%
\pgfsetfillcolor{textcolor}%
\pgftext[x=0.672222in,y=2.545000in,right,]{\color{textcolor}\sffamily\fontsize{8.000000}{9.600000}\selectfont 4.5}%
\end{pgfscope}%
\begin{pgfscope}%
\pgfpathrectangle{\pgfqpoint{0.750000in}{1.770000in}}{\pgfqpoint{2.113636in}{0.930000in}} %
\pgfusepath{clip}%
\pgfsetroundcap%
\pgfsetroundjoin%
\pgfsetlinewidth{0.803000pt}%
\definecolor{currentstroke}{rgb}{1.000000,1.000000,1.000000}%
\pgfsetstrokecolor{currentstroke}%
\pgfsetdash{}{0pt}%
\pgfpathmoveto{\pgfqpoint{0.750000in}{2.700000in}}%
\pgfpathlineto{\pgfqpoint{2.863636in}{2.700000in}}%
\pgfusepath{stroke}%
\end{pgfscope}%
\begin{pgfscope}%
\pgfsetbuttcap%
\pgfsetroundjoin%
\definecolor{currentfill}{rgb}{0.150000,0.150000,0.150000}%
\pgfsetfillcolor{currentfill}%
\pgfsetlinewidth{0.803000pt}%
\definecolor{currentstroke}{rgb}{0.150000,0.150000,0.150000}%
\pgfsetstrokecolor{currentstroke}%
\pgfsetdash{}{0pt}%
\pgfsys@defobject{currentmarker}{\pgfqpoint{0.000000in}{0.000000in}}{\pgfqpoint{0.000000in}{0.000000in}}{%
\pgfpathmoveto{\pgfqpoint{0.000000in}{0.000000in}}%
\pgfpathlineto{\pgfqpoint{0.000000in}{0.000000in}}%
\pgfusepath{stroke,fill}%
}%
\begin{pgfscope}%
\pgfsys@transformshift{0.750000in}{2.700000in}%
\pgfsys@useobject{currentmarker}{}%
\end{pgfscope}%
\end{pgfscope}%
\begin{pgfscope}%
\pgfsetbuttcap%
\pgfsetroundjoin%
\definecolor{currentfill}{rgb}{0.150000,0.150000,0.150000}%
\pgfsetfillcolor{currentfill}%
\pgfsetlinewidth{0.803000pt}%
\definecolor{currentstroke}{rgb}{0.150000,0.150000,0.150000}%
\pgfsetstrokecolor{currentstroke}%
\pgfsetdash{}{0pt}%
\pgfsys@defobject{currentmarker}{\pgfqpoint{0.000000in}{0.000000in}}{\pgfqpoint{0.000000in}{0.000000in}}{%
\pgfpathmoveto{\pgfqpoint{0.000000in}{0.000000in}}%
\pgfpathlineto{\pgfqpoint{0.000000in}{0.000000in}}%
\pgfusepath{stroke,fill}%
}%
\begin{pgfscope}%
\pgfsys@transformshift{2.863636in}{2.700000in}%
\pgfsys@useobject{currentmarker}{}%
\end{pgfscope}%
\end{pgfscope}%
\begin{pgfscope}%
\definecolor{textcolor}{rgb}{0.150000,0.150000,0.150000}%
\pgfsetstrokecolor{textcolor}%
\pgfsetfillcolor{textcolor}%
\pgftext[x=0.672222in,y=2.700000in,right,]{\color{textcolor}\sffamily\fontsize{8.000000}{9.600000}\selectfont 5.0}%
\end{pgfscope}%
\begin{pgfscope}%
\definecolor{textcolor}{rgb}{0.150000,0.150000,0.150000}%
\pgfsetstrokecolor{textcolor}%
\pgfsetfillcolor{textcolor}%
\pgftext[x=0.444830in,y=2.235000in,,bottom,rotate=90.000000]{\color{textcolor}\sffamily\fontsize{8.800000}{10.560000}\selectfont Flying time}%
\end{pgfscope}%
\begin{pgfscope}%
\pgfpathrectangle{\pgfqpoint{0.750000in}{1.770000in}}{\pgfqpoint{2.113636in}{0.930000in}} %
\pgfusepath{clip}%
\pgfsetbuttcap%
\pgfsetmiterjoin%
\definecolor{currentfill}{rgb}{0.447059,0.623529,0.811765}%
\pgfsetfillcolor{currentfill}%
\pgfsetfillopacity{0.300000}%
\pgfsetlinewidth{0.240900pt}%
\definecolor{currentstroke}{rgb}{0.447059,0.623529,0.811765}%
\pgfsetstrokecolor{currentstroke}%
\pgfsetstrokeopacity{0.300000}%
\pgfsetdash{}{0pt}%
\pgfpathmoveto{\pgfqpoint{0.867751in}{2.255378in}}%
\pgfpathlineto{\pgfqpoint{0.886722in}{2.229695in}}%
\pgfpathlineto{\pgfqpoint{0.905693in}{2.205998in}}%
\pgfpathlineto{\pgfqpoint{0.924664in}{2.184290in}}%
\pgfpathlineto{\pgfqpoint{0.943636in}{2.164560in}}%
\pgfpathlineto{\pgfqpoint{0.962607in}{2.146780in}}%
\pgfpathlineto{\pgfqpoint{0.981578in}{2.130902in}}%
\pgfpathlineto{\pgfqpoint{1.000549in}{2.116863in}}%
\pgfpathlineto{\pgfqpoint{1.019520in}{2.104582in}}%
\pgfpathlineto{\pgfqpoint{1.038491in}{2.093964in}}%
\pgfpathlineto{\pgfqpoint{1.057462in}{2.084911in}}%
\pgfpathlineto{\pgfqpoint{1.076433in}{2.077318in}}%
\pgfpathlineto{\pgfqpoint{1.095404in}{2.071082in}}%
\pgfpathlineto{\pgfqpoint{1.114375in}{2.066107in}}%
\pgfpathlineto{\pgfqpoint{1.133346in}{2.062301in}}%
\pgfpathlineto{\pgfqpoint{1.152317in}{2.059580in}}%
\pgfpathlineto{\pgfqpoint{1.171288in}{2.057869in}}%
\pgfpathlineto{\pgfqpoint{1.190259in}{2.057101in}}%
\pgfpathlineto{\pgfqpoint{1.209230in}{2.057215in}}%
\pgfpathlineto{\pgfqpoint{1.228201in}{2.058158in}}%
\pgfpathlineto{\pgfqpoint{1.247172in}{2.059882in}}%
\pgfpathlineto{\pgfqpoint{1.266143in}{2.062345in}}%
\pgfpathlineto{\pgfqpoint{1.285114in}{2.065507in}}%
\pgfpathlineto{\pgfqpoint{1.304085in}{2.069334in}}%
\pgfpathlineto{\pgfqpoint{1.323056in}{2.073795in}}%
\pgfpathlineto{\pgfqpoint{1.342028in}{2.078861in}}%
\pgfpathlineto{\pgfqpoint{1.360999in}{2.084506in}}%
\pgfpathlineto{\pgfqpoint{1.379970in}{2.090707in}}%
\pgfpathlineto{\pgfqpoint{1.398941in}{2.097443in}}%
\pgfpathlineto{\pgfqpoint{1.417912in}{2.104694in}}%
\pgfpathlineto{\pgfqpoint{1.436883in}{2.112444in}}%
\pgfpathlineto{\pgfqpoint{1.455854in}{2.120678in}}%
\pgfpathlineto{\pgfqpoint{1.474825in}{2.129384in}}%
\pgfpathlineto{\pgfqpoint{1.493796in}{2.138553in}}%
\pgfpathlineto{\pgfqpoint{1.512767in}{2.148177in}}%
\pgfpathlineto{\pgfqpoint{1.531738in}{2.158252in}}%
\pgfpathlineto{\pgfqpoint{1.550709in}{2.168776in}}%
\pgfpathlineto{\pgfqpoint{1.569680in}{2.179750in}}%
\pgfpathlineto{\pgfqpoint{1.588651in}{2.191176in}}%
\pgfpathlineto{\pgfqpoint{1.607622in}{2.203058in}}%
\pgfpathlineto{\pgfqpoint{1.626593in}{2.215399in}}%
\pgfpathlineto{\pgfqpoint{1.645564in}{2.228197in}}%
\pgfpathlineto{\pgfqpoint{1.664535in}{2.241448in}}%
\pgfpathlineto{\pgfqpoint{1.683506in}{2.255139in}}%
\pgfpathlineto{\pgfqpoint{1.702477in}{2.269247in}}%
\pgfpathlineto{\pgfqpoint{1.721448in}{2.283737in}}%
\pgfpathlineto{\pgfqpoint{1.740420in}{2.298561in}}%
\pgfpathlineto{\pgfqpoint{1.759391in}{2.313661in}}%
\pgfpathlineto{\pgfqpoint{1.778362in}{2.328968in}}%
\pgfpathlineto{\pgfqpoint{1.797333in}{2.344408in}}%
\pgfpathlineto{\pgfqpoint{1.816304in}{2.359901in}}%
\pgfpathlineto{\pgfqpoint{1.835275in}{2.375365in}}%
\pgfpathlineto{\pgfqpoint{1.854246in}{2.390721in}}%
\pgfpathlineto{\pgfqpoint{1.873217in}{2.405889in}}%
\pgfpathlineto{\pgfqpoint{1.892188in}{2.420793in}}%
\pgfpathlineto{\pgfqpoint{1.911159in}{2.435363in}}%
\pgfpathlineto{\pgfqpoint{1.930130in}{2.449529in}}%
\pgfpathlineto{\pgfqpoint{1.949101in}{2.463229in}}%
\pgfpathlineto{\pgfqpoint{1.968072in}{2.476403in}}%
\pgfpathlineto{\pgfqpoint{1.987043in}{2.488994in}}%
\pgfpathlineto{\pgfqpoint{2.006014in}{2.500953in}}%
\pgfpathlineto{\pgfqpoint{2.024985in}{2.512231in}}%
\pgfpathlineto{\pgfqpoint{2.043956in}{2.522785in}}%
\pgfpathlineto{\pgfqpoint{2.062927in}{2.532575in}}%
\pgfpathlineto{\pgfqpoint{2.081898in}{2.541567in}}%
\pgfpathlineto{\pgfqpoint{2.100869in}{2.549730in}}%
\pgfpathlineto{\pgfqpoint{2.119840in}{2.557036in}}%
\pgfpathlineto{\pgfqpoint{2.138812in}{2.563463in}}%
\pgfpathlineto{\pgfqpoint{2.157783in}{2.568996in}}%
\pgfpathlineto{\pgfqpoint{2.176754in}{2.573622in}}%
\pgfpathlineto{\pgfqpoint{2.195725in}{2.577336in}}%
\pgfpathlineto{\pgfqpoint{2.214696in}{2.580138in}}%
\pgfpathlineto{\pgfqpoint{2.233667in}{2.582036in}}%
\pgfpathlineto{\pgfqpoint{2.252638in}{2.583045in}}%
\pgfpathlineto{\pgfqpoint{2.271609in}{2.583189in}}%
\pgfpathlineto{\pgfqpoint{2.290580in}{2.582500in}}%
\pgfpathlineto{\pgfqpoint{2.309551in}{2.581019in}}%
\pgfpathlineto{\pgfqpoint{2.328522in}{2.578797in}}%
\pgfpathlineto{\pgfqpoint{2.347493in}{2.575893in}}%
\pgfpathlineto{\pgfqpoint{2.366464in}{2.572375in}}%
\pgfpathlineto{\pgfqpoint{2.385435in}{2.568316in}}%
\pgfpathlineto{\pgfqpoint{2.404406in}{2.563793in}}%
\pgfpathlineto{\pgfqpoint{2.423377in}{2.558882in}}%
\pgfpathlineto{\pgfqpoint{2.442348in}{2.553655in}}%
\pgfpathlineto{\pgfqpoint{2.461319in}{2.548177in}}%
\pgfpathlineto{\pgfqpoint{2.480290in}{2.542500in}}%
\pgfpathlineto{\pgfqpoint{2.499261in}{2.536665in}}%
\pgfpathlineto{\pgfqpoint{2.518232in}{2.530698in}}%
\pgfpathlineto{\pgfqpoint{2.537204in}{2.524614in}}%
\pgfpathlineto{\pgfqpoint{2.556175in}{2.518418in}}%
\pgfpathlineto{\pgfqpoint{2.575146in}{2.512104in}}%
\pgfpathlineto{\pgfqpoint{2.594117in}{2.505665in}}%
\pgfpathlineto{\pgfqpoint{2.613088in}{2.499088in}}%
\pgfpathlineto{\pgfqpoint{2.632059in}{2.492362in}}%
\pgfpathlineto{\pgfqpoint{2.651030in}{2.485473in}}%
\pgfpathlineto{\pgfqpoint{2.670001in}{2.478410in}}%
\pgfpathlineto{\pgfqpoint{2.688972in}{2.471165in}}%
\pgfpathlineto{\pgfqpoint{2.707943in}{2.463730in}}%
\pgfpathlineto{\pgfqpoint{2.726914in}{2.456102in}}%
\pgfpathlineto{\pgfqpoint{2.745885in}{2.448278in}}%
\pgfpathlineto{\pgfqpoint{2.745885in}{2.039630in}}%
\pgfpathlineto{\pgfqpoint{2.726914in}{2.070412in}}%
\pgfpathlineto{\pgfqpoint{2.707943in}{2.100131in}}%
\pgfpathlineto{\pgfqpoint{2.688972in}{2.128738in}}%
\pgfpathlineto{\pgfqpoint{2.670001in}{2.156184in}}%
\pgfpathlineto{\pgfqpoint{2.651030in}{2.182418in}}%
\pgfpathlineto{\pgfqpoint{2.632059in}{2.207391in}}%
\pgfpathlineto{\pgfqpoint{2.613088in}{2.231054in}}%
\pgfpathlineto{\pgfqpoint{2.594117in}{2.253359in}}%
\pgfpathlineto{\pgfqpoint{2.575146in}{2.274261in}}%
\pgfpathlineto{\pgfqpoint{2.556175in}{2.293719in}}%
\pgfpathlineto{\pgfqpoint{2.537204in}{2.311695in}}%
\pgfpathlineto{\pgfqpoint{2.518232in}{2.328162in}}%
\pgfpathlineto{\pgfqpoint{2.499261in}{2.343102in}}%
\pgfpathlineto{\pgfqpoint{2.480290in}{2.356511in}}%
\pgfpathlineto{\pgfqpoint{2.461319in}{2.368401in}}%
\pgfpathlineto{\pgfqpoint{2.442348in}{2.378798in}}%
\pgfpathlineto{\pgfqpoint{2.423377in}{2.387746in}}%
\pgfpathlineto{\pgfqpoint{2.404406in}{2.395304in}}%
\pgfpathlineto{\pgfqpoint{2.385435in}{2.401538in}}%
\pgfpathlineto{\pgfqpoint{2.366464in}{2.406526in}}%
\pgfpathlineto{\pgfqpoint{2.347493in}{2.410347in}}%
\pgfpathlineto{\pgfqpoint{2.328522in}{2.413080in}}%
\pgfpathlineto{\pgfqpoint{2.309551in}{2.414802in}}%
\pgfpathlineto{\pgfqpoint{2.290580in}{2.415583in}}%
\pgfpathlineto{\pgfqpoint{2.271609in}{2.415492in}}%
\pgfpathlineto{\pgfqpoint{2.252638in}{2.414586in}}%
\pgfpathlineto{\pgfqpoint{2.233667in}{2.412919in}}%
\pgfpathlineto{\pgfqpoint{2.214696in}{2.410540in}}%
\pgfpathlineto{\pgfqpoint{2.195725in}{2.407492in}}%
\pgfpathlineto{\pgfqpoint{2.176754in}{2.403812in}}%
\pgfpathlineto{\pgfqpoint{2.157783in}{2.399535in}}%
\pgfpathlineto{\pgfqpoint{2.138812in}{2.394690in}}%
\pgfpathlineto{\pgfqpoint{2.119840in}{2.389307in}}%
\pgfpathlineto{\pgfqpoint{2.100869in}{2.383410in}}%
\pgfpathlineto{\pgfqpoint{2.081898in}{2.377022in}}%
\pgfpathlineto{\pgfqpoint{2.062927in}{2.370163in}}%
\pgfpathlineto{\pgfqpoint{2.043956in}{2.362853in}}%
\pgfpathlineto{\pgfqpoint{2.024985in}{2.355108in}}%
\pgfpathlineto{\pgfqpoint{2.006014in}{2.346945in}}%
\pgfpathlineto{\pgfqpoint{1.987043in}{2.338377in}}%
\pgfpathlineto{\pgfqpoint{1.968072in}{2.329416in}}%
\pgfpathlineto{\pgfqpoint{1.949101in}{2.320072in}}%
\pgfpathlineto{\pgfqpoint{1.930130in}{2.310354in}}%
\pgfpathlineto{\pgfqpoint{1.911159in}{2.300267in}}%
\pgfpathlineto{\pgfqpoint{1.892188in}{2.289816in}}%
\pgfpathlineto{\pgfqpoint{1.873217in}{2.279001in}}%
\pgfpathlineto{\pgfqpoint{1.854246in}{2.267822in}}%
\pgfpathlineto{\pgfqpoint{1.835275in}{2.256276in}}%
\pgfpathlineto{\pgfqpoint{1.816304in}{2.244358in}}%
\pgfpathlineto{\pgfqpoint{1.797333in}{2.232061in}}%
\pgfpathlineto{\pgfqpoint{1.778362in}{2.219382in}}%
\pgfpathlineto{\pgfqpoint{1.759391in}{2.206317in}}%
\pgfpathlineto{\pgfqpoint{1.740420in}{2.192870in}}%
\pgfpathlineto{\pgfqpoint{1.721448in}{2.179052in}}%
\pgfpathlineto{\pgfqpoint{1.702477in}{2.164883in}}%
\pgfpathlineto{\pgfqpoint{1.683506in}{2.150397in}}%
\pgfpathlineto{\pgfqpoint{1.664535in}{2.135638in}}%
\pgfpathlineto{\pgfqpoint{1.645564in}{2.120665in}}%
\pgfpathlineto{\pgfqpoint{1.626593in}{2.105545in}}%
\pgfpathlineto{\pgfqpoint{1.607622in}{2.090354in}}%
\pgfpathlineto{\pgfqpoint{1.588651in}{2.075173in}}%
\pgfpathlineto{\pgfqpoint{1.569680in}{2.060085in}}%
\pgfpathlineto{\pgfqpoint{1.550709in}{2.045173in}}%
\pgfpathlineto{\pgfqpoint{1.531738in}{2.030520in}}%
\pgfpathlineto{\pgfqpoint{1.512767in}{2.016204in}}%
\pgfpathlineto{\pgfqpoint{1.493796in}{2.002302in}}%
\pgfpathlineto{\pgfqpoint{1.474825in}{1.988887in}}%
\pgfpathlineto{\pgfqpoint{1.455854in}{1.976028in}}%
\pgfpathlineto{\pgfqpoint{1.436883in}{1.963790in}}%
\pgfpathlineto{\pgfqpoint{1.417912in}{1.952236in}}%
\pgfpathlineto{\pgfqpoint{1.398941in}{1.941422in}}%
\pgfpathlineto{\pgfqpoint{1.379970in}{1.931403in}}%
\pgfpathlineto{\pgfqpoint{1.360999in}{1.922229in}}%
\pgfpathlineto{\pgfqpoint{1.342028in}{1.913946in}}%
\pgfpathlineto{\pgfqpoint{1.323056in}{1.906596in}}%
\pgfpathlineto{\pgfqpoint{1.304085in}{1.900218in}}%
\pgfpathlineto{\pgfqpoint{1.285114in}{1.894843in}}%
\pgfpathlineto{\pgfqpoint{1.266143in}{1.890500in}}%
\pgfpathlineto{\pgfqpoint{1.247172in}{1.887211in}}%
\pgfpathlineto{\pgfqpoint{1.228201in}{1.884994in}}%
\pgfpathlineto{\pgfqpoint{1.209230in}{1.883856in}}%
\pgfpathlineto{\pgfqpoint{1.190259in}{1.883802in}}%
\pgfpathlineto{\pgfqpoint{1.171288in}{1.884823in}}%
\pgfpathlineto{\pgfqpoint{1.152317in}{1.886906in}}%
\pgfpathlineto{\pgfqpoint{1.133346in}{1.890023in}}%
\pgfpathlineto{\pgfqpoint{1.114375in}{1.894140in}}%
\pgfpathlineto{\pgfqpoint{1.095404in}{1.899209in}}%
\pgfpathlineto{\pgfqpoint{1.076433in}{1.905173in}}%
\pgfpathlineto{\pgfqpoint{1.057462in}{1.911965in}}%
\pgfpathlineto{\pgfqpoint{1.038491in}{1.919509in}}%
\pgfpathlineto{\pgfqpoint{1.019520in}{1.927727in}}%
\pgfpathlineto{\pgfqpoint{1.000549in}{1.936540in}}%
\pgfpathlineto{\pgfqpoint{0.981578in}{1.945874in}}%
\pgfpathlineto{\pgfqpoint{0.962607in}{1.955662in}}%
\pgfpathlineto{\pgfqpoint{0.943636in}{1.965854in}}%
\pgfpathlineto{\pgfqpoint{0.924664in}{1.976411in}}%
\pgfpathlineto{\pgfqpoint{0.905693in}{1.987312in}}%
\pgfpathlineto{\pgfqpoint{0.886722in}{1.998548in}}%
\pgfpathlineto{\pgfqpoint{0.867751in}{2.010124in}}%
\pgfpathclose%
\pgfusepath{stroke,fill}%
\end{pgfscope}%
\begin{pgfscope}%
\pgfpathrectangle{\pgfqpoint{0.750000in}{1.770000in}}{\pgfqpoint{2.113636in}{0.930000in}} %
\pgfusepath{clip}%
\pgfsetroundcap%
\pgfsetroundjoin%
\pgfsetlinewidth{2.007500pt}%
\definecolor{currentstroke}{rgb}{0.125490,0.290196,0.529412}%
\pgfsetstrokecolor{currentstroke}%
\pgfsetdash{}{0pt}%
\pgfpathmoveto{\pgfqpoint{0.867751in}{2.132751in}}%
\pgfpathlineto{\pgfqpoint{0.886722in}{2.114122in}}%
\pgfpathlineto{\pgfqpoint{0.905693in}{2.096655in}}%
\pgfpathlineto{\pgfqpoint{0.924664in}{2.080351in}}%
\pgfpathlineto{\pgfqpoint{0.943636in}{2.065207in}}%
\pgfpathlineto{\pgfqpoint{0.962607in}{2.051221in}}%
\pgfpathlineto{\pgfqpoint{0.981578in}{2.038388in}}%
\pgfpathlineto{\pgfqpoint{1.000549in}{2.026702in}}%
\pgfpathlineto{\pgfqpoint{1.019520in}{2.016154in}}%
\pgfpathlineto{\pgfqpoint{1.038491in}{2.006737in}}%
\pgfpathlineto{\pgfqpoint{1.057462in}{1.998438in}}%
\pgfpathlineto{\pgfqpoint{1.076433in}{1.991246in}}%
\pgfpathlineto{\pgfqpoint{1.095404in}{1.985146in}}%
\pgfpathlineto{\pgfqpoint{1.114375in}{1.980124in}}%
\pgfpathlineto{\pgfqpoint{1.133346in}{1.976162in}}%
\pgfpathlineto{\pgfqpoint{1.152317in}{1.973243in}}%
\pgfpathlineto{\pgfqpoint{1.171288in}{1.971346in}}%
\pgfpathlineto{\pgfqpoint{1.190259in}{1.970451in}}%
\pgfpathlineto{\pgfqpoint{1.209230in}{1.970536in}}%
\pgfpathlineto{\pgfqpoint{1.228201in}{1.971576in}}%
\pgfpathlineto{\pgfqpoint{1.247172in}{1.973547in}}%
\pgfpathlineto{\pgfqpoint{1.266143in}{1.976422in}}%
\pgfpathlineto{\pgfqpoint{1.285114in}{1.980175in}}%
\pgfpathlineto{\pgfqpoint{1.304085in}{1.984776in}}%
\pgfpathlineto{\pgfqpoint{1.323056in}{1.990196in}}%
\pgfpathlineto{\pgfqpoint{1.342028in}{1.996404in}}%
\pgfpathlineto{\pgfqpoint{1.360999in}{2.003368in}}%
\pgfpathlineto{\pgfqpoint{1.379970in}{2.011055in}}%
\pgfpathlineto{\pgfqpoint{1.398941in}{2.019432in}}%
\pgfpathlineto{\pgfqpoint{1.417912in}{2.028465in}}%
\pgfpathlineto{\pgfqpoint{1.436883in}{2.038117in}}%
\pgfpathlineto{\pgfqpoint{1.455854in}{2.048353in}}%
\pgfpathlineto{\pgfqpoint{1.474825in}{2.059135in}}%
\pgfpathlineto{\pgfqpoint{1.493796in}{2.070427in}}%
\pgfpathlineto{\pgfqpoint{1.512767in}{2.082190in}}%
\pgfpathlineto{\pgfqpoint{1.531738in}{2.094386in}}%
\pgfpathlineto{\pgfqpoint{1.550709in}{2.106975in}}%
\pgfpathlineto{\pgfqpoint{1.569680in}{2.119918in}}%
\pgfpathlineto{\pgfqpoint{1.588651in}{2.133175in}}%
\pgfpathlineto{\pgfqpoint{1.607622in}{2.146706in}}%
\pgfpathlineto{\pgfqpoint{1.626593in}{2.160472in}}%
\pgfpathlineto{\pgfqpoint{1.645564in}{2.174431in}}%
\pgfpathlineto{\pgfqpoint{1.664535in}{2.188543in}}%
\pgfpathlineto{\pgfqpoint{1.683506in}{2.202768in}}%
\pgfpathlineto{\pgfqpoint{1.702477in}{2.217065in}}%
\pgfpathlineto{\pgfqpoint{1.721448in}{2.231394in}}%
\pgfpathlineto{\pgfqpoint{1.740420in}{2.245716in}}%
\pgfpathlineto{\pgfqpoint{1.759391in}{2.259989in}}%
\pgfpathlineto{\pgfqpoint{1.778362in}{2.274175in}}%
\pgfpathlineto{\pgfqpoint{1.797333in}{2.288235in}}%
\pgfpathlineto{\pgfqpoint{1.816304in}{2.302129in}}%
\pgfpathlineto{\pgfqpoint{1.835275in}{2.315821in}}%
\pgfpathlineto{\pgfqpoint{1.854246in}{2.329272in}}%
\pgfpathlineto{\pgfqpoint{1.873217in}{2.342445in}}%
\pgfpathlineto{\pgfqpoint{1.892188in}{2.355305in}}%
\pgfpathlineto{\pgfqpoint{1.911159in}{2.367815in}}%
\pgfpathlineto{\pgfqpoint{1.930130in}{2.379942in}}%
\pgfpathlineto{\pgfqpoint{1.949101in}{2.391651in}}%
\pgfpathlineto{\pgfqpoint{1.968072in}{2.402909in}}%
\pgfpathlineto{\pgfqpoint{1.987043in}{2.413686in}}%
\pgfpathlineto{\pgfqpoint{2.006014in}{2.423949in}}%
\pgfpathlineto{\pgfqpoint{2.024985in}{2.433670in}}%
\pgfpathlineto{\pgfqpoint{2.043956in}{2.442819in}}%
\pgfpathlineto{\pgfqpoint{2.062927in}{2.451369in}}%
\pgfpathlineto{\pgfqpoint{2.081898in}{2.459295in}}%
\pgfpathlineto{\pgfqpoint{2.100869in}{2.466570in}}%
\pgfpathlineto{\pgfqpoint{2.119840in}{2.473171in}}%
\pgfpathlineto{\pgfqpoint{2.138812in}{2.479077in}}%
\pgfpathlineto{\pgfqpoint{2.157783in}{2.484265in}}%
\pgfpathlineto{\pgfqpoint{2.176754in}{2.488717in}}%
\pgfpathlineto{\pgfqpoint{2.195725in}{2.492414in}}%
\pgfpathlineto{\pgfqpoint{2.214696in}{2.495339in}}%
\pgfpathlineto{\pgfqpoint{2.233667in}{2.497478in}}%
\pgfpathlineto{\pgfqpoint{2.252638in}{2.498816in}}%
\pgfpathlineto{\pgfqpoint{2.271609in}{2.499340in}}%
\pgfpathlineto{\pgfqpoint{2.290580in}{2.499042in}}%
\pgfpathlineto{\pgfqpoint{2.309551in}{2.497910in}}%
\pgfpathlineto{\pgfqpoint{2.328522in}{2.495938in}}%
\pgfpathlineto{\pgfqpoint{2.347493in}{2.493120in}}%
\pgfpathlineto{\pgfqpoint{2.366464in}{2.489451in}}%
\pgfpathlineto{\pgfqpoint{2.385435in}{2.484927in}}%
\pgfpathlineto{\pgfqpoint{2.404406in}{2.479548in}}%
\pgfpathlineto{\pgfqpoint{2.423377in}{2.473314in}}%
\pgfpathlineto{\pgfqpoint{2.442348in}{2.466227in}}%
\pgfpathlineto{\pgfqpoint{2.461319in}{2.458289in}}%
\pgfpathlineto{\pgfqpoint{2.480290in}{2.449506in}}%
\pgfpathlineto{\pgfqpoint{2.499261in}{2.439883in}}%
\pgfpathlineto{\pgfqpoint{2.518232in}{2.429430in}}%
\pgfpathlineto{\pgfqpoint{2.537204in}{2.418155in}}%
\pgfpathlineto{\pgfqpoint{2.556175in}{2.406068in}}%
\pgfpathlineto{\pgfqpoint{2.575146in}{2.393183in}}%
\pgfpathlineto{\pgfqpoint{2.594117in}{2.379512in}}%
\pgfpathlineto{\pgfqpoint{2.613088in}{2.365071in}}%
\pgfpathlineto{\pgfqpoint{2.632059in}{2.349876in}}%
\pgfpathlineto{\pgfqpoint{2.651030in}{2.333945in}}%
\pgfpathlineto{\pgfqpoint{2.670001in}{2.317297in}}%
\pgfpathlineto{\pgfqpoint{2.688972in}{2.299952in}}%
\pgfpathlineto{\pgfqpoint{2.707943in}{2.281931in}}%
\pgfpathlineto{\pgfqpoint{2.726914in}{2.263257in}}%
\pgfpathlineto{\pgfqpoint{2.745885in}{2.243954in}}%
\pgfusepath{stroke}%
\end{pgfscope}%
\begin{pgfscope}%
\pgfpathrectangle{\pgfqpoint{0.750000in}{1.770000in}}{\pgfqpoint{2.113636in}{0.930000in}} %
\pgfusepath{clip}%
\pgfsetroundcap%
\pgfsetroundjoin%
\pgfsetlinewidth{0.200750pt}%
\definecolor{currentstroke}{rgb}{0.125490,0.290196,0.529412}%
\pgfsetstrokecolor{currentstroke}%
\pgfsetdash{}{0pt}%
\pgfpathmoveto{\pgfqpoint{0.867751in}{2.255378in}}%
\pgfpathlineto{\pgfqpoint{0.886722in}{2.229695in}}%
\pgfpathlineto{\pgfqpoint{0.905693in}{2.205998in}}%
\pgfpathlineto{\pgfqpoint{0.924664in}{2.184290in}}%
\pgfpathlineto{\pgfqpoint{0.943636in}{2.164560in}}%
\pgfpathlineto{\pgfqpoint{0.962607in}{2.146780in}}%
\pgfpathlineto{\pgfqpoint{0.981578in}{2.130902in}}%
\pgfpathlineto{\pgfqpoint{1.000549in}{2.116863in}}%
\pgfpathlineto{\pgfqpoint{1.019520in}{2.104582in}}%
\pgfpathlineto{\pgfqpoint{1.038491in}{2.093964in}}%
\pgfpathlineto{\pgfqpoint{1.057462in}{2.084911in}}%
\pgfpathlineto{\pgfqpoint{1.076433in}{2.077318in}}%
\pgfpathlineto{\pgfqpoint{1.095404in}{2.071082in}}%
\pgfpathlineto{\pgfqpoint{1.114375in}{2.066107in}}%
\pgfpathlineto{\pgfqpoint{1.133346in}{2.062301in}}%
\pgfpathlineto{\pgfqpoint{1.152317in}{2.059580in}}%
\pgfpathlineto{\pgfqpoint{1.171288in}{2.057869in}}%
\pgfpathlineto{\pgfqpoint{1.190259in}{2.057101in}}%
\pgfpathlineto{\pgfqpoint{1.209230in}{2.057215in}}%
\pgfpathlineto{\pgfqpoint{1.228201in}{2.058158in}}%
\pgfpathlineto{\pgfqpoint{1.247172in}{2.059882in}}%
\pgfpathlineto{\pgfqpoint{1.266143in}{2.062345in}}%
\pgfpathlineto{\pgfqpoint{1.285114in}{2.065507in}}%
\pgfpathlineto{\pgfqpoint{1.304085in}{2.069334in}}%
\pgfpathlineto{\pgfqpoint{1.323056in}{2.073795in}}%
\pgfpathlineto{\pgfqpoint{1.342028in}{2.078861in}}%
\pgfpathlineto{\pgfqpoint{1.360999in}{2.084506in}}%
\pgfpathlineto{\pgfqpoint{1.379970in}{2.090707in}}%
\pgfpathlineto{\pgfqpoint{1.398941in}{2.097443in}}%
\pgfpathlineto{\pgfqpoint{1.417912in}{2.104694in}}%
\pgfpathlineto{\pgfqpoint{1.436883in}{2.112444in}}%
\pgfpathlineto{\pgfqpoint{1.455854in}{2.120678in}}%
\pgfpathlineto{\pgfqpoint{1.474825in}{2.129384in}}%
\pgfpathlineto{\pgfqpoint{1.493796in}{2.138553in}}%
\pgfpathlineto{\pgfqpoint{1.512767in}{2.148177in}}%
\pgfpathlineto{\pgfqpoint{1.531738in}{2.158252in}}%
\pgfpathlineto{\pgfqpoint{1.550709in}{2.168776in}}%
\pgfpathlineto{\pgfqpoint{1.569680in}{2.179750in}}%
\pgfpathlineto{\pgfqpoint{1.588651in}{2.191176in}}%
\pgfpathlineto{\pgfqpoint{1.607622in}{2.203058in}}%
\pgfpathlineto{\pgfqpoint{1.626593in}{2.215399in}}%
\pgfpathlineto{\pgfqpoint{1.645564in}{2.228197in}}%
\pgfpathlineto{\pgfqpoint{1.664535in}{2.241448in}}%
\pgfpathlineto{\pgfqpoint{1.683506in}{2.255139in}}%
\pgfpathlineto{\pgfqpoint{1.702477in}{2.269247in}}%
\pgfpathlineto{\pgfqpoint{1.721448in}{2.283737in}}%
\pgfpathlineto{\pgfqpoint{1.740420in}{2.298561in}}%
\pgfpathlineto{\pgfqpoint{1.759391in}{2.313661in}}%
\pgfpathlineto{\pgfqpoint{1.778362in}{2.328968in}}%
\pgfpathlineto{\pgfqpoint{1.797333in}{2.344408in}}%
\pgfpathlineto{\pgfqpoint{1.816304in}{2.359901in}}%
\pgfpathlineto{\pgfqpoint{1.835275in}{2.375365in}}%
\pgfpathlineto{\pgfqpoint{1.854246in}{2.390721in}}%
\pgfpathlineto{\pgfqpoint{1.873217in}{2.405889in}}%
\pgfpathlineto{\pgfqpoint{1.892188in}{2.420793in}}%
\pgfpathlineto{\pgfqpoint{1.911159in}{2.435363in}}%
\pgfpathlineto{\pgfqpoint{1.930130in}{2.449529in}}%
\pgfpathlineto{\pgfqpoint{1.949101in}{2.463229in}}%
\pgfpathlineto{\pgfqpoint{1.968072in}{2.476403in}}%
\pgfpathlineto{\pgfqpoint{1.987043in}{2.488994in}}%
\pgfpathlineto{\pgfqpoint{2.006014in}{2.500953in}}%
\pgfpathlineto{\pgfqpoint{2.024985in}{2.512231in}}%
\pgfpathlineto{\pgfqpoint{2.043956in}{2.522785in}}%
\pgfpathlineto{\pgfqpoint{2.062927in}{2.532575in}}%
\pgfpathlineto{\pgfqpoint{2.081898in}{2.541567in}}%
\pgfpathlineto{\pgfqpoint{2.100869in}{2.549730in}}%
\pgfpathlineto{\pgfqpoint{2.119840in}{2.557036in}}%
\pgfpathlineto{\pgfqpoint{2.138812in}{2.563463in}}%
\pgfpathlineto{\pgfqpoint{2.157783in}{2.568996in}}%
\pgfpathlineto{\pgfqpoint{2.176754in}{2.573622in}}%
\pgfpathlineto{\pgfqpoint{2.195725in}{2.577336in}}%
\pgfpathlineto{\pgfqpoint{2.214696in}{2.580138in}}%
\pgfpathlineto{\pgfqpoint{2.233667in}{2.582036in}}%
\pgfpathlineto{\pgfqpoint{2.252638in}{2.583045in}}%
\pgfpathlineto{\pgfqpoint{2.271609in}{2.583189in}}%
\pgfpathlineto{\pgfqpoint{2.290580in}{2.582500in}}%
\pgfpathlineto{\pgfqpoint{2.309551in}{2.581019in}}%
\pgfpathlineto{\pgfqpoint{2.328522in}{2.578797in}}%
\pgfpathlineto{\pgfqpoint{2.347493in}{2.575893in}}%
\pgfpathlineto{\pgfqpoint{2.366464in}{2.572375in}}%
\pgfpathlineto{\pgfqpoint{2.385435in}{2.568316in}}%
\pgfpathlineto{\pgfqpoint{2.404406in}{2.563793in}}%
\pgfpathlineto{\pgfqpoint{2.423377in}{2.558882in}}%
\pgfpathlineto{\pgfqpoint{2.442348in}{2.553655in}}%
\pgfpathlineto{\pgfqpoint{2.461319in}{2.548177in}}%
\pgfpathlineto{\pgfqpoint{2.480290in}{2.542500in}}%
\pgfpathlineto{\pgfqpoint{2.499261in}{2.536665in}}%
\pgfpathlineto{\pgfqpoint{2.518232in}{2.530698in}}%
\pgfpathlineto{\pgfqpoint{2.537204in}{2.524614in}}%
\pgfpathlineto{\pgfqpoint{2.556175in}{2.518418in}}%
\pgfpathlineto{\pgfqpoint{2.575146in}{2.512104in}}%
\pgfpathlineto{\pgfqpoint{2.594117in}{2.505665in}}%
\pgfpathlineto{\pgfqpoint{2.613088in}{2.499088in}}%
\pgfpathlineto{\pgfqpoint{2.632059in}{2.492362in}}%
\pgfpathlineto{\pgfqpoint{2.651030in}{2.485473in}}%
\pgfpathlineto{\pgfqpoint{2.670001in}{2.478410in}}%
\pgfpathlineto{\pgfqpoint{2.688972in}{2.471165in}}%
\pgfpathlineto{\pgfqpoint{2.707943in}{2.463730in}}%
\pgfpathlineto{\pgfqpoint{2.726914in}{2.456102in}}%
\pgfpathlineto{\pgfqpoint{2.745885in}{2.448278in}}%
\pgfusepath{stroke}%
\end{pgfscope}%
\begin{pgfscope}%
\pgfpathrectangle{\pgfqpoint{0.750000in}{1.770000in}}{\pgfqpoint{2.113636in}{0.930000in}} %
\pgfusepath{clip}%
\pgfsetroundcap%
\pgfsetroundjoin%
\pgfsetlinewidth{0.200750pt}%
\definecolor{currentstroke}{rgb}{0.125490,0.290196,0.529412}%
\pgfsetstrokecolor{currentstroke}%
\pgfsetdash{}{0pt}%
\pgfpathmoveto{\pgfqpoint{0.867751in}{2.010124in}}%
\pgfpathlineto{\pgfqpoint{0.886722in}{1.998548in}}%
\pgfpathlineto{\pgfqpoint{0.905693in}{1.987312in}}%
\pgfpathlineto{\pgfqpoint{0.924664in}{1.976411in}}%
\pgfpathlineto{\pgfqpoint{0.943636in}{1.965854in}}%
\pgfpathlineto{\pgfqpoint{0.962607in}{1.955662in}}%
\pgfpathlineto{\pgfqpoint{0.981578in}{1.945874in}}%
\pgfpathlineto{\pgfqpoint{1.000549in}{1.936540in}}%
\pgfpathlineto{\pgfqpoint{1.019520in}{1.927727in}}%
\pgfpathlineto{\pgfqpoint{1.038491in}{1.919509in}}%
\pgfpathlineto{\pgfqpoint{1.057462in}{1.911965in}}%
\pgfpathlineto{\pgfqpoint{1.076433in}{1.905173in}}%
\pgfpathlineto{\pgfqpoint{1.095404in}{1.899209in}}%
\pgfpathlineto{\pgfqpoint{1.114375in}{1.894140in}}%
\pgfpathlineto{\pgfqpoint{1.133346in}{1.890023in}}%
\pgfpathlineto{\pgfqpoint{1.152317in}{1.886906in}}%
\pgfpathlineto{\pgfqpoint{1.171288in}{1.884823in}}%
\pgfpathlineto{\pgfqpoint{1.190259in}{1.883802in}}%
\pgfpathlineto{\pgfqpoint{1.209230in}{1.883856in}}%
\pgfpathlineto{\pgfqpoint{1.228201in}{1.884994in}}%
\pgfpathlineto{\pgfqpoint{1.247172in}{1.887211in}}%
\pgfpathlineto{\pgfqpoint{1.266143in}{1.890500in}}%
\pgfpathlineto{\pgfqpoint{1.285114in}{1.894843in}}%
\pgfpathlineto{\pgfqpoint{1.304085in}{1.900218in}}%
\pgfpathlineto{\pgfqpoint{1.323056in}{1.906596in}}%
\pgfpathlineto{\pgfqpoint{1.342028in}{1.913946in}}%
\pgfpathlineto{\pgfqpoint{1.360999in}{1.922229in}}%
\pgfpathlineto{\pgfqpoint{1.379970in}{1.931403in}}%
\pgfpathlineto{\pgfqpoint{1.398941in}{1.941422in}}%
\pgfpathlineto{\pgfqpoint{1.417912in}{1.952236in}}%
\pgfpathlineto{\pgfqpoint{1.436883in}{1.963790in}}%
\pgfpathlineto{\pgfqpoint{1.455854in}{1.976028in}}%
\pgfpathlineto{\pgfqpoint{1.474825in}{1.988887in}}%
\pgfpathlineto{\pgfqpoint{1.493796in}{2.002302in}}%
\pgfpathlineto{\pgfqpoint{1.512767in}{2.016204in}}%
\pgfpathlineto{\pgfqpoint{1.531738in}{2.030520in}}%
\pgfpathlineto{\pgfqpoint{1.550709in}{2.045173in}}%
\pgfpathlineto{\pgfqpoint{1.569680in}{2.060085in}}%
\pgfpathlineto{\pgfqpoint{1.588651in}{2.075173in}}%
\pgfpathlineto{\pgfqpoint{1.607622in}{2.090354in}}%
\pgfpathlineto{\pgfqpoint{1.626593in}{2.105545in}}%
\pgfpathlineto{\pgfqpoint{1.645564in}{2.120665in}}%
\pgfpathlineto{\pgfqpoint{1.664535in}{2.135638in}}%
\pgfpathlineto{\pgfqpoint{1.683506in}{2.150397in}}%
\pgfpathlineto{\pgfqpoint{1.702477in}{2.164883in}}%
\pgfpathlineto{\pgfqpoint{1.721448in}{2.179052in}}%
\pgfpathlineto{\pgfqpoint{1.740420in}{2.192870in}}%
\pgfpathlineto{\pgfqpoint{1.759391in}{2.206317in}}%
\pgfpathlineto{\pgfqpoint{1.778362in}{2.219382in}}%
\pgfpathlineto{\pgfqpoint{1.797333in}{2.232061in}}%
\pgfpathlineto{\pgfqpoint{1.816304in}{2.244358in}}%
\pgfpathlineto{\pgfqpoint{1.835275in}{2.256276in}}%
\pgfpathlineto{\pgfqpoint{1.854246in}{2.267822in}}%
\pgfpathlineto{\pgfqpoint{1.873217in}{2.279001in}}%
\pgfpathlineto{\pgfqpoint{1.892188in}{2.289816in}}%
\pgfpathlineto{\pgfqpoint{1.911159in}{2.300267in}}%
\pgfpathlineto{\pgfqpoint{1.930130in}{2.310354in}}%
\pgfpathlineto{\pgfqpoint{1.949101in}{2.320072in}}%
\pgfpathlineto{\pgfqpoint{1.968072in}{2.329416in}}%
\pgfpathlineto{\pgfqpoint{1.987043in}{2.338377in}}%
\pgfpathlineto{\pgfqpoint{2.006014in}{2.346945in}}%
\pgfpathlineto{\pgfqpoint{2.024985in}{2.355108in}}%
\pgfpathlineto{\pgfqpoint{2.043956in}{2.362853in}}%
\pgfpathlineto{\pgfqpoint{2.062927in}{2.370163in}}%
\pgfpathlineto{\pgfqpoint{2.081898in}{2.377022in}}%
\pgfpathlineto{\pgfqpoint{2.100869in}{2.383410in}}%
\pgfpathlineto{\pgfqpoint{2.119840in}{2.389307in}}%
\pgfpathlineto{\pgfqpoint{2.138812in}{2.394690in}}%
\pgfpathlineto{\pgfqpoint{2.157783in}{2.399535in}}%
\pgfpathlineto{\pgfqpoint{2.176754in}{2.403812in}}%
\pgfpathlineto{\pgfqpoint{2.195725in}{2.407492in}}%
\pgfpathlineto{\pgfqpoint{2.214696in}{2.410540in}}%
\pgfpathlineto{\pgfqpoint{2.233667in}{2.412919in}}%
\pgfpathlineto{\pgfqpoint{2.252638in}{2.414586in}}%
\pgfpathlineto{\pgfqpoint{2.271609in}{2.415492in}}%
\pgfpathlineto{\pgfqpoint{2.290580in}{2.415583in}}%
\pgfpathlineto{\pgfqpoint{2.309551in}{2.414802in}}%
\pgfpathlineto{\pgfqpoint{2.328522in}{2.413080in}}%
\pgfpathlineto{\pgfqpoint{2.347493in}{2.410347in}}%
\pgfpathlineto{\pgfqpoint{2.366464in}{2.406526in}}%
\pgfpathlineto{\pgfqpoint{2.385435in}{2.401538in}}%
\pgfpathlineto{\pgfqpoint{2.404406in}{2.395304in}}%
\pgfpathlineto{\pgfqpoint{2.423377in}{2.387746in}}%
\pgfpathlineto{\pgfqpoint{2.442348in}{2.378798in}}%
\pgfpathlineto{\pgfqpoint{2.461319in}{2.368401in}}%
\pgfpathlineto{\pgfqpoint{2.480290in}{2.356511in}}%
\pgfpathlineto{\pgfqpoint{2.499261in}{2.343102in}}%
\pgfpathlineto{\pgfqpoint{2.518232in}{2.328162in}}%
\pgfpathlineto{\pgfqpoint{2.537204in}{2.311695in}}%
\pgfpathlineto{\pgfqpoint{2.556175in}{2.293719in}}%
\pgfpathlineto{\pgfqpoint{2.575146in}{2.274261in}}%
\pgfpathlineto{\pgfqpoint{2.594117in}{2.253359in}}%
\pgfpathlineto{\pgfqpoint{2.613088in}{2.231054in}}%
\pgfpathlineto{\pgfqpoint{2.632059in}{2.207391in}}%
\pgfpathlineto{\pgfqpoint{2.651030in}{2.182418in}}%
\pgfpathlineto{\pgfqpoint{2.670001in}{2.156184in}}%
\pgfpathlineto{\pgfqpoint{2.688972in}{2.128738in}}%
\pgfpathlineto{\pgfqpoint{2.707943in}{2.100131in}}%
\pgfpathlineto{\pgfqpoint{2.726914in}{2.070412in}}%
\pgfpathlineto{\pgfqpoint{2.745885in}{2.039630in}}%
\pgfusepath{stroke}%
\end{pgfscope}%
\begin{pgfscope}%
\pgfpathrectangle{\pgfqpoint{0.750000in}{1.770000in}}{\pgfqpoint{2.113636in}{0.930000in}} %
\pgfusepath{clip}%
\pgfsetbuttcap%
\pgfsetbeveljoin%
\definecolor{currentfill}{rgb}{0.298039,0.447059,0.690196}%
\pgfsetfillcolor{currentfill}%
\pgfsetlinewidth{0.000000pt}%
\definecolor{currentstroke}{rgb}{0.000000,0.000000,0.000000}%
\pgfsetstrokecolor{currentstroke}%
\pgfsetdash{}{0pt}%
\pgfsys@defobject{currentmarker}{\pgfqpoint{-0.036986in}{-0.031462in}}{\pgfqpoint{0.036986in}{0.038889in}}{%
\pgfpathmoveto{\pgfqpoint{0.000000in}{0.038889in}}%
\pgfpathlineto{\pgfqpoint{-0.008731in}{0.012017in}}%
\pgfpathlineto{\pgfqpoint{-0.036986in}{0.012017in}}%
\pgfpathlineto{\pgfqpoint{-0.014127in}{-0.004590in}}%
\pgfpathlineto{\pgfqpoint{-0.022858in}{-0.031462in}}%
\pgfpathlineto{\pgfqpoint{-0.000000in}{-0.014854in}}%
\pgfpathlineto{\pgfqpoint{0.022858in}{-0.031462in}}%
\pgfpathlineto{\pgfqpoint{0.014127in}{-0.004590in}}%
\pgfpathlineto{\pgfqpoint{0.036986in}{0.012017in}}%
\pgfpathlineto{\pgfqpoint{0.008731in}{0.012017in}}%
\pgfpathclose%
\pgfusepath{fill}%
}%
\begin{pgfscope}%
\pgfsys@transformshift{1.503609in}{2.343500in}%
\pgfsys@useobject{currentmarker}{}%
\end{pgfscope}%
\begin{pgfscope}%
\pgfsys@transformshift{1.097366in}{2.219500in}%
\pgfsys@useobject{currentmarker}{}%
\end{pgfscope}%
\begin{pgfscope}%
\pgfsys@transformshift{1.721448in}{2.328000in}%
\pgfsys@useobject{currentmarker}{}%
\end{pgfscope}%
\begin{pgfscope}%
\pgfsys@transformshift{2.180679in}{2.529500in}%
\pgfsys@useobject{currentmarker}{}%
\end{pgfscope}%
\begin{pgfscope}%
\pgfsys@transformshift{1.273993in}{2.204000in}%
\pgfsys@useobject{currentmarker}{}%
\end{pgfscope}%
\begin{pgfscope}%
\pgfsys@transformshift{2.669347in}{2.591500in}%
\pgfsys@useobject{currentmarker}{}%
\end{pgfscope}%
\begin{pgfscope}%
\pgfsys@transformshift{0.956065in}{2.204000in}%
\pgfsys@useobject{currentmarker}{}%
\end{pgfscope}%
\begin{pgfscope}%
\pgfsys@transformshift{2.304318in}{2.653500in}%
\pgfsys@useobject{currentmarker}{}%
\end{pgfscope}%
\begin{pgfscope}%
\pgfsys@transformshift{1.085591in}{1.940500in}%
\pgfsys@useobject{currentmarker}{}%
\end{pgfscope}%
\begin{pgfscope}%
\pgfsys@transformshift{1.821537in}{2.235000in}%
\pgfsys@useobject{currentmarker}{}%
\end{pgfscope}%
\begin{pgfscope}%
\pgfsys@transformshift{1.044378in}{2.188500in}%
\pgfsys@useobject{currentmarker}{}%
\end{pgfscope}%
\begin{pgfscope}%
\pgfsys@transformshift{1.026716in}{2.204000in}%
\pgfsys@useobject{currentmarker}{}%
\end{pgfscope}%
\begin{pgfscope}%
\pgfsys@transformshift{2.745885in}{2.250500in}%
\pgfsys@useobject{currentmarker}{}%
\end{pgfscope}%
\begin{pgfscope}%
\pgfsys@transformshift{1.344644in}{1.878500in}%
\pgfsys@useobject{currentmarker}{}%
\end{pgfscope}%
\begin{pgfscope}%
\pgfsys@transformshift{1.903963in}{2.421000in}%
\pgfsys@useobject{currentmarker}{}%
\end{pgfscope}%
\begin{pgfscope}%
\pgfsys@transformshift{1.792099in}{2.374500in}%
\pgfsys@useobject{currentmarker}{}%
\end{pgfscope}%
\begin{pgfscope}%
\pgfsys@transformshift{0.867751in}{2.126500in}%
\pgfsys@useobject{currentmarker}{}%
\end{pgfscope}%
\begin{pgfscope}%
\pgfsys@transformshift{1.162130in}{2.219500in}%
\pgfsys@useobject{currentmarker}{}%
\end{pgfscope}%
\begin{pgfscope}%
\pgfsys@transformshift{1.268106in}{2.002500in}%
\pgfsys@useobject{currentmarker}{}%
\end{pgfscope}%
\begin{pgfscope}%
\pgfsys@transformshift{0.944290in}{2.142000in}%
\pgfsys@useobject{currentmarker}{}%
\end{pgfscope}%
\begin{pgfscope}%
\pgfsys@transformshift{2.563370in}{2.219500in}%
\pgfsys@useobject{currentmarker}{}%
\end{pgfscope}%
\begin{pgfscope}%
\pgfsys@transformshift{1.815650in}{2.436500in}%
\pgfsys@useobject{currentmarker}{}%
\end{pgfscope}%
\begin{pgfscope}%
\pgfsys@transformshift{2.557483in}{2.266000in}%
\pgfsys@useobject{currentmarker}{}%
\end{pgfscope}%
\begin{pgfscope}%
\pgfsys@transformshift{2.227779in}{2.049000in}%
\pgfsys@useobject{currentmarker}{}%
\end{pgfscope}%
\begin{pgfscope}%
\pgfsys@transformshift{2.245442in}{2.312500in}%
\pgfsys@useobject{currentmarker}{}%
\end{pgfscope}%
\begin{pgfscope}%
\pgfsys@transformshift{2.186566in}{2.514000in}%
\pgfsys@useobject{currentmarker}{}%
\end{pgfscope}%
\begin{pgfscope}%
\pgfsys@transformshift{0.938402in}{2.064500in}%
\pgfsys@useobject{currentmarker}{}%
\end{pgfscope}%
\begin{pgfscope}%
\pgfsys@transformshift{2.545708in}{1.956000in}%
\pgfsys@useobject{currentmarker}{}%
\end{pgfscope}%
\begin{pgfscope}%
\pgfsys@transformshift{1.803874in}{2.018000in}%
\pgfsys@useobject{currentmarker}{}%
\end{pgfscope}%
\begin{pgfscope}%
\pgfsys@transformshift{2.604583in}{2.529500in}%
\pgfsys@useobject{currentmarker}{}%
\end{pgfscope}%
\end{pgfscope}%
\begin{pgfscope}%
\pgfsetrectcap%
\pgfsetmiterjoin%
\pgfsetlinewidth{0.000000pt}%
\definecolor{currentstroke}{rgb}{1.000000,1.000000,1.000000}%
\pgfsetstrokecolor{currentstroke}%
\pgfsetdash{}{0pt}%
\pgfpathmoveto{\pgfqpoint{0.750000in}{1.770000in}}%
\pgfpathlineto{\pgfqpoint{2.863636in}{1.770000in}}%
\pgfusepath{}%
\end{pgfscope}%
\begin{pgfscope}%
\pgfsetrectcap%
\pgfsetmiterjoin%
\pgfsetlinewidth{0.000000pt}%
\definecolor{currentstroke}{rgb}{1.000000,1.000000,1.000000}%
\pgfsetstrokecolor{currentstroke}%
\pgfsetdash{}{0pt}%
\pgfpathmoveto{\pgfqpoint{0.750000in}{1.770000in}}%
\pgfpathlineto{\pgfqpoint{0.750000in}{2.700000in}}%
\pgfusepath{}%
\end{pgfscope}%
\begin{pgfscope}%
\pgfsetrectcap%
\pgfsetmiterjoin%
\pgfsetlinewidth{0.000000pt}%
\definecolor{currentstroke}{rgb}{1.000000,1.000000,1.000000}%
\pgfsetstrokecolor{currentstroke}%
\pgfsetdash{}{0pt}%
\pgfpathmoveto{\pgfqpoint{0.750000in}{2.700000in}}%
\pgfpathlineto{\pgfqpoint{2.863636in}{2.700000in}}%
\pgfusepath{}%
\end{pgfscope}%
\begin{pgfscope}%
\pgfsetrectcap%
\pgfsetmiterjoin%
\pgfsetlinewidth{0.000000pt}%
\definecolor{currentstroke}{rgb}{1.000000,1.000000,1.000000}%
\pgfsetstrokecolor{currentstroke}%
\pgfsetdash{}{0pt}%
\pgfpathmoveto{\pgfqpoint{2.863636in}{1.770000in}}%
\pgfpathlineto{\pgfqpoint{2.863636in}{2.700000in}}%
\pgfusepath{}%
\end{pgfscope}%
\begin{pgfscope}%
\pgfsetbuttcap%
\pgfsetmiterjoin%
\definecolor{currentfill}{rgb}{0.917647,0.917647,0.949020}%
\pgfsetfillcolor{currentfill}%
\pgfsetlinewidth{0.000000pt}%
\definecolor{currentstroke}{rgb}{0.000000,0.000000,0.000000}%
\pgfsetstrokecolor{currentstroke}%
\pgfsetstrokeopacity{0.000000}%
\pgfsetdash{}{0pt}%
\pgfpathmoveto{\pgfqpoint{3.286364in}{1.770000in}}%
\pgfpathlineto{\pgfqpoint{5.400000in}{1.770000in}}%
\pgfpathlineto{\pgfqpoint{5.400000in}{2.700000in}}%
\pgfpathlineto{\pgfqpoint{3.286364in}{2.700000in}}%
\pgfpathclose%
\pgfusepath{fill}%
\end{pgfscope}%
\begin{pgfscope}%
\pgfpathrectangle{\pgfqpoint{3.286364in}{1.770000in}}{\pgfqpoint{2.113636in}{0.930000in}} %
\pgfusepath{clip}%
\pgfsetroundcap%
\pgfsetroundjoin%
\pgfsetlinewidth{0.803000pt}%
\definecolor{currentstroke}{rgb}{1.000000,1.000000,1.000000}%
\pgfsetstrokecolor{currentstroke}%
\pgfsetdash{}{0pt}%
\pgfpathmoveto{\pgfqpoint{3.480818in}{1.770000in}}%
\pgfpathlineto{\pgfqpoint{3.480818in}{2.700000in}}%
\pgfusepath{stroke}%
\end{pgfscope}%
\begin{pgfscope}%
\pgfsetbuttcap%
\pgfsetroundjoin%
\definecolor{currentfill}{rgb}{0.150000,0.150000,0.150000}%
\pgfsetfillcolor{currentfill}%
\pgfsetlinewidth{0.803000pt}%
\definecolor{currentstroke}{rgb}{0.150000,0.150000,0.150000}%
\pgfsetstrokecolor{currentstroke}%
\pgfsetdash{}{0pt}%
\pgfsys@defobject{currentmarker}{\pgfqpoint{0.000000in}{0.000000in}}{\pgfqpoint{0.000000in}{0.000000in}}{%
\pgfpathmoveto{\pgfqpoint{0.000000in}{0.000000in}}%
\pgfpathlineto{\pgfqpoint{0.000000in}{0.000000in}}%
\pgfusepath{stroke,fill}%
}%
\begin{pgfscope}%
\pgfsys@transformshift{3.480818in}{1.770000in}%
\pgfsys@useobject{currentmarker}{}%
\end{pgfscope}%
\end{pgfscope}%
\begin{pgfscope}%
\pgfsetbuttcap%
\pgfsetroundjoin%
\definecolor{currentfill}{rgb}{0.150000,0.150000,0.150000}%
\pgfsetfillcolor{currentfill}%
\pgfsetlinewidth{0.803000pt}%
\definecolor{currentstroke}{rgb}{0.150000,0.150000,0.150000}%
\pgfsetstrokecolor{currentstroke}%
\pgfsetdash{}{0pt}%
\pgfsys@defobject{currentmarker}{\pgfqpoint{0.000000in}{0.000000in}}{\pgfqpoint{0.000000in}{0.000000in}}{%
\pgfpathmoveto{\pgfqpoint{0.000000in}{0.000000in}}%
\pgfpathlineto{\pgfqpoint{0.000000in}{0.000000in}}%
\pgfusepath{stroke,fill}%
}%
\begin{pgfscope}%
\pgfsys@transformshift{3.480818in}{2.700000in}%
\pgfsys@useobject{currentmarker}{}%
\end{pgfscope}%
\end{pgfscope}%
\begin{pgfscope}%
\definecolor{textcolor}{rgb}{0.150000,0.150000,0.150000}%
\pgfsetstrokecolor{textcolor}%
\pgfsetfillcolor{textcolor}%
\pgftext[x=3.480818in,y=1.692222in,,top]{\color{textcolor}\sffamily\fontsize{8.000000}{9.600000}\selectfont 2.5}%
\end{pgfscope}%
\begin{pgfscope}%
\pgfpathrectangle{\pgfqpoint{3.286364in}{1.770000in}}{\pgfqpoint{2.113636in}{0.930000in}} %
\pgfusepath{clip}%
\pgfsetroundcap%
\pgfsetroundjoin%
\pgfsetlinewidth{0.803000pt}%
\definecolor{currentstroke}{rgb}{1.000000,1.000000,1.000000}%
\pgfsetstrokecolor{currentstroke}%
\pgfsetdash{}{0pt}%
\pgfpathmoveto{\pgfqpoint{3.903545in}{1.770000in}}%
\pgfpathlineto{\pgfqpoint{3.903545in}{2.700000in}}%
\pgfusepath{stroke}%
\end{pgfscope}%
\begin{pgfscope}%
\pgfsetbuttcap%
\pgfsetroundjoin%
\definecolor{currentfill}{rgb}{0.150000,0.150000,0.150000}%
\pgfsetfillcolor{currentfill}%
\pgfsetlinewidth{0.803000pt}%
\definecolor{currentstroke}{rgb}{0.150000,0.150000,0.150000}%
\pgfsetstrokecolor{currentstroke}%
\pgfsetdash{}{0pt}%
\pgfsys@defobject{currentmarker}{\pgfqpoint{0.000000in}{0.000000in}}{\pgfqpoint{0.000000in}{0.000000in}}{%
\pgfpathmoveto{\pgfqpoint{0.000000in}{0.000000in}}%
\pgfpathlineto{\pgfqpoint{0.000000in}{0.000000in}}%
\pgfusepath{stroke,fill}%
}%
\begin{pgfscope}%
\pgfsys@transformshift{3.903545in}{1.770000in}%
\pgfsys@useobject{currentmarker}{}%
\end{pgfscope}%
\end{pgfscope}%
\begin{pgfscope}%
\pgfsetbuttcap%
\pgfsetroundjoin%
\definecolor{currentfill}{rgb}{0.150000,0.150000,0.150000}%
\pgfsetfillcolor{currentfill}%
\pgfsetlinewidth{0.803000pt}%
\definecolor{currentstroke}{rgb}{0.150000,0.150000,0.150000}%
\pgfsetstrokecolor{currentstroke}%
\pgfsetdash{}{0pt}%
\pgfsys@defobject{currentmarker}{\pgfqpoint{0.000000in}{0.000000in}}{\pgfqpoint{0.000000in}{0.000000in}}{%
\pgfpathmoveto{\pgfqpoint{0.000000in}{0.000000in}}%
\pgfpathlineto{\pgfqpoint{0.000000in}{0.000000in}}%
\pgfusepath{stroke,fill}%
}%
\begin{pgfscope}%
\pgfsys@transformshift{3.903545in}{2.700000in}%
\pgfsys@useobject{currentmarker}{}%
\end{pgfscope}%
\end{pgfscope}%
\begin{pgfscope}%
\definecolor{textcolor}{rgb}{0.150000,0.150000,0.150000}%
\pgfsetstrokecolor{textcolor}%
\pgfsetfillcolor{textcolor}%
\pgftext[x=3.903545in,y=1.692222in,,top]{\color{textcolor}\sffamily\fontsize{8.000000}{9.600000}\selectfont 3.0}%
\end{pgfscope}%
\begin{pgfscope}%
\pgfpathrectangle{\pgfqpoint{3.286364in}{1.770000in}}{\pgfqpoint{2.113636in}{0.930000in}} %
\pgfusepath{clip}%
\pgfsetroundcap%
\pgfsetroundjoin%
\pgfsetlinewidth{0.803000pt}%
\definecolor{currentstroke}{rgb}{1.000000,1.000000,1.000000}%
\pgfsetstrokecolor{currentstroke}%
\pgfsetdash{}{0pt}%
\pgfpathmoveto{\pgfqpoint{4.326273in}{1.770000in}}%
\pgfpathlineto{\pgfqpoint{4.326273in}{2.700000in}}%
\pgfusepath{stroke}%
\end{pgfscope}%
\begin{pgfscope}%
\pgfsetbuttcap%
\pgfsetroundjoin%
\definecolor{currentfill}{rgb}{0.150000,0.150000,0.150000}%
\pgfsetfillcolor{currentfill}%
\pgfsetlinewidth{0.803000pt}%
\definecolor{currentstroke}{rgb}{0.150000,0.150000,0.150000}%
\pgfsetstrokecolor{currentstroke}%
\pgfsetdash{}{0pt}%
\pgfsys@defobject{currentmarker}{\pgfqpoint{0.000000in}{0.000000in}}{\pgfqpoint{0.000000in}{0.000000in}}{%
\pgfpathmoveto{\pgfqpoint{0.000000in}{0.000000in}}%
\pgfpathlineto{\pgfqpoint{0.000000in}{0.000000in}}%
\pgfusepath{stroke,fill}%
}%
\begin{pgfscope}%
\pgfsys@transformshift{4.326273in}{1.770000in}%
\pgfsys@useobject{currentmarker}{}%
\end{pgfscope}%
\end{pgfscope}%
\begin{pgfscope}%
\pgfsetbuttcap%
\pgfsetroundjoin%
\definecolor{currentfill}{rgb}{0.150000,0.150000,0.150000}%
\pgfsetfillcolor{currentfill}%
\pgfsetlinewidth{0.803000pt}%
\definecolor{currentstroke}{rgb}{0.150000,0.150000,0.150000}%
\pgfsetstrokecolor{currentstroke}%
\pgfsetdash{}{0pt}%
\pgfsys@defobject{currentmarker}{\pgfqpoint{0.000000in}{0.000000in}}{\pgfqpoint{0.000000in}{0.000000in}}{%
\pgfpathmoveto{\pgfqpoint{0.000000in}{0.000000in}}%
\pgfpathlineto{\pgfqpoint{0.000000in}{0.000000in}}%
\pgfusepath{stroke,fill}%
}%
\begin{pgfscope}%
\pgfsys@transformshift{4.326273in}{2.700000in}%
\pgfsys@useobject{currentmarker}{}%
\end{pgfscope}%
\end{pgfscope}%
\begin{pgfscope}%
\definecolor{textcolor}{rgb}{0.150000,0.150000,0.150000}%
\pgfsetstrokecolor{textcolor}%
\pgfsetfillcolor{textcolor}%
\pgftext[x=4.326273in,y=1.692222in,,top]{\color{textcolor}\sffamily\fontsize{8.000000}{9.600000}\selectfont 3.5}%
\end{pgfscope}%
\begin{pgfscope}%
\pgfpathrectangle{\pgfqpoint{3.286364in}{1.770000in}}{\pgfqpoint{2.113636in}{0.930000in}} %
\pgfusepath{clip}%
\pgfsetroundcap%
\pgfsetroundjoin%
\pgfsetlinewidth{0.803000pt}%
\definecolor{currentstroke}{rgb}{1.000000,1.000000,1.000000}%
\pgfsetstrokecolor{currentstroke}%
\pgfsetdash{}{0pt}%
\pgfpathmoveto{\pgfqpoint{4.749000in}{1.770000in}}%
\pgfpathlineto{\pgfqpoint{4.749000in}{2.700000in}}%
\pgfusepath{stroke}%
\end{pgfscope}%
\begin{pgfscope}%
\pgfsetbuttcap%
\pgfsetroundjoin%
\definecolor{currentfill}{rgb}{0.150000,0.150000,0.150000}%
\pgfsetfillcolor{currentfill}%
\pgfsetlinewidth{0.803000pt}%
\definecolor{currentstroke}{rgb}{0.150000,0.150000,0.150000}%
\pgfsetstrokecolor{currentstroke}%
\pgfsetdash{}{0pt}%
\pgfsys@defobject{currentmarker}{\pgfqpoint{0.000000in}{0.000000in}}{\pgfqpoint{0.000000in}{0.000000in}}{%
\pgfpathmoveto{\pgfqpoint{0.000000in}{0.000000in}}%
\pgfpathlineto{\pgfqpoint{0.000000in}{0.000000in}}%
\pgfusepath{stroke,fill}%
}%
\begin{pgfscope}%
\pgfsys@transformshift{4.749000in}{1.770000in}%
\pgfsys@useobject{currentmarker}{}%
\end{pgfscope}%
\end{pgfscope}%
\begin{pgfscope}%
\pgfsetbuttcap%
\pgfsetroundjoin%
\definecolor{currentfill}{rgb}{0.150000,0.150000,0.150000}%
\pgfsetfillcolor{currentfill}%
\pgfsetlinewidth{0.803000pt}%
\definecolor{currentstroke}{rgb}{0.150000,0.150000,0.150000}%
\pgfsetstrokecolor{currentstroke}%
\pgfsetdash{}{0pt}%
\pgfsys@defobject{currentmarker}{\pgfqpoint{0.000000in}{0.000000in}}{\pgfqpoint{0.000000in}{0.000000in}}{%
\pgfpathmoveto{\pgfqpoint{0.000000in}{0.000000in}}%
\pgfpathlineto{\pgfqpoint{0.000000in}{0.000000in}}%
\pgfusepath{stroke,fill}%
}%
\begin{pgfscope}%
\pgfsys@transformshift{4.749000in}{2.700000in}%
\pgfsys@useobject{currentmarker}{}%
\end{pgfscope}%
\end{pgfscope}%
\begin{pgfscope}%
\definecolor{textcolor}{rgb}{0.150000,0.150000,0.150000}%
\pgfsetstrokecolor{textcolor}%
\pgfsetfillcolor{textcolor}%
\pgftext[x=4.749000in,y=1.692222in,,top]{\color{textcolor}\sffamily\fontsize{8.000000}{9.600000}\selectfont 4.0}%
\end{pgfscope}%
\begin{pgfscope}%
\pgfpathrectangle{\pgfqpoint{3.286364in}{1.770000in}}{\pgfqpoint{2.113636in}{0.930000in}} %
\pgfusepath{clip}%
\pgfsetroundcap%
\pgfsetroundjoin%
\pgfsetlinewidth{0.803000pt}%
\definecolor{currentstroke}{rgb}{1.000000,1.000000,1.000000}%
\pgfsetstrokecolor{currentstroke}%
\pgfsetdash{}{0pt}%
\pgfpathmoveto{\pgfqpoint{5.171727in}{1.770000in}}%
\pgfpathlineto{\pgfqpoint{5.171727in}{2.700000in}}%
\pgfusepath{stroke}%
\end{pgfscope}%
\begin{pgfscope}%
\pgfsetbuttcap%
\pgfsetroundjoin%
\definecolor{currentfill}{rgb}{0.150000,0.150000,0.150000}%
\pgfsetfillcolor{currentfill}%
\pgfsetlinewidth{0.803000pt}%
\definecolor{currentstroke}{rgb}{0.150000,0.150000,0.150000}%
\pgfsetstrokecolor{currentstroke}%
\pgfsetdash{}{0pt}%
\pgfsys@defobject{currentmarker}{\pgfqpoint{0.000000in}{0.000000in}}{\pgfqpoint{0.000000in}{0.000000in}}{%
\pgfpathmoveto{\pgfqpoint{0.000000in}{0.000000in}}%
\pgfpathlineto{\pgfqpoint{0.000000in}{0.000000in}}%
\pgfusepath{stroke,fill}%
}%
\begin{pgfscope}%
\pgfsys@transformshift{5.171727in}{1.770000in}%
\pgfsys@useobject{currentmarker}{}%
\end{pgfscope}%
\end{pgfscope}%
\begin{pgfscope}%
\pgfsetbuttcap%
\pgfsetroundjoin%
\definecolor{currentfill}{rgb}{0.150000,0.150000,0.150000}%
\pgfsetfillcolor{currentfill}%
\pgfsetlinewidth{0.803000pt}%
\definecolor{currentstroke}{rgb}{0.150000,0.150000,0.150000}%
\pgfsetstrokecolor{currentstroke}%
\pgfsetdash{}{0pt}%
\pgfsys@defobject{currentmarker}{\pgfqpoint{0.000000in}{0.000000in}}{\pgfqpoint{0.000000in}{0.000000in}}{%
\pgfpathmoveto{\pgfqpoint{0.000000in}{0.000000in}}%
\pgfpathlineto{\pgfqpoint{0.000000in}{0.000000in}}%
\pgfusepath{stroke,fill}%
}%
\begin{pgfscope}%
\pgfsys@transformshift{5.171727in}{2.700000in}%
\pgfsys@useobject{currentmarker}{}%
\end{pgfscope}%
\end{pgfscope}%
\begin{pgfscope}%
\definecolor{textcolor}{rgb}{0.150000,0.150000,0.150000}%
\pgfsetstrokecolor{textcolor}%
\pgfsetfillcolor{textcolor}%
\pgftext[x=5.171727in,y=1.692222in,,top]{\color{textcolor}\sffamily\fontsize{8.000000}{9.600000}\selectfont 4.5}%
\end{pgfscope}%
\begin{pgfscope}%
\definecolor{textcolor}{rgb}{0.150000,0.150000,0.150000}%
\pgfsetstrokecolor{textcolor}%
\pgfsetfillcolor{textcolor}%
\pgftext[x=4.343182in,y=1.527099in,,top]{\color{textcolor}\sffamily\fontsize{8.800000}{10.560000}\selectfont Wing width}%
\end{pgfscope}%
\begin{pgfscope}%
\pgfpathrectangle{\pgfqpoint{3.286364in}{1.770000in}}{\pgfqpoint{2.113636in}{0.930000in}} %
\pgfusepath{clip}%
\pgfsetroundcap%
\pgfsetroundjoin%
\pgfsetlinewidth{0.803000pt}%
\definecolor{currentstroke}{rgb}{1.000000,1.000000,1.000000}%
\pgfsetstrokecolor{currentstroke}%
\pgfsetdash{}{0pt}%
\pgfpathmoveto{\pgfqpoint{3.286364in}{1.770000in}}%
\pgfpathlineto{\pgfqpoint{5.400000in}{1.770000in}}%
\pgfusepath{stroke}%
\end{pgfscope}%
\begin{pgfscope}%
\pgfsetbuttcap%
\pgfsetroundjoin%
\definecolor{currentfill}{rgb}{0.150000,0.150000,0.150000}%
\pgfsetfillcolor{currentfill}%
\pgfsetlinewidth{0.803000pt}%
\definecolor{currentstroke}{rgb}{0.150000,0.150000,0.150000}%
\pgfsetstrokecolor{currentstroke}%
\pgfsetdash{}{0pt}%
\pgfsys@defobject{currentmarker}{\pgfqpoint{0.000000in}{0.000000in}}{\pgfqpoint{0.000000in}{0.000000in}}{%
\pgfpathmoveto{\pgfqpoint{0.000000in}{0.000000in}}%
\pgfpathlineto{\pgfqpoint{0.000000in}{0.000000in}}%
\pgfusepath{stroke,fill}%
}%
\begin{pgfscope}%
\pgfsys@transformshift{3.286364in}{1.770000in}%
\pgfsys@useobject{currentmarker}{}%
\end{pgfscope}%
\end{pgfscope}%
\begin{pgfscope}%
\pgfsetbuttcap%
\pgfsetroundjoin%
\definecolor{currentfill}{rgb}{0.150000,0.150000,0.150000}%
\pgfsetfillcolor{currentfill}%
\pgfsetlinewidth{0.803000pt}%
\definecolor{currentstroke}{rgb}{0.150000,0.150000,0.150000}%
\pgfsetstrokecolor{currentstroke}%
\pgfsetdash{}{0pt}%
\pgfsys@defobject{currentmarker}{\pgfqpoint{0.000000in}{0.000000in}}{\pgfqpoint{0.000000in}{0.000000in}}{%
\pgfpathmoveto{\pgfqpoint{0.000000in}{0.000000in}}%
\pgfpathlineto{\pgfqpoint{0.000000in}{0.000000in}}%
\pgfusepath{stroke,fill}%
}%
\begin{pgfscope}%
\pgfsys@transformshift{5.400000in}{1.770000in}%
\pgfsys@useobject{currentmarker}{}%
\end{pgfscope}%
\end{pgfscope}%
\begin{pgfscope}%
\pgfpathrectangle{\pgfqpoint{3.286364in}{1.770000in}}{\pgfqpoint{2.113636in}{0.930000in}} %
\pgfusepath{clip}%
\pgfsetroundcap%
\pgfsetroundjoin%
\pgfsetlinewidth{0.803000pt}%
\definecolor{currentstroke}{rgb}{1.000000,1.000000,1.000000}%
\pgfsetstrokecolor{currentstroke}%
\pgfsetdash{}{0pt}%
\pgfpathmoveto{\pgfqpoint{3.286364in}{1.925000in}}%
\pgfpathlineto{\pgfqpoint{5.400000in}{1.925000in}}%
\pgfusepath{stroke}%
\end{pgfscope}%
\begin{pgfscope}%
\pgfsetbuttcap%
\pgfsetroundjoin%
\definecolor{currentfill}{rgb}{0.150000,0.150000,0.150000}%
\pgfsetfillcolor{currentfill}%
\pgfsetlinewidth{0.803000pt}%
\definecolor{currentstroke}{rgb}{0.150000,0.150000,0.150000}%
\pgfsetstrokecolor{currentstroke}%
\pgfsetdash{}{0pt}%
\pgfsys@defobject{currentmarker}{\pgfqpoint{0.000000in}{0.000000in}}{\pgfqpoint{0.000000in}{0.000000in}}{%
\pgfpathmoveto{\pgfqpoint{0.000000in}{0.000000in}}%
\pgfpathlineto{\pgfqpoint{0.000000in}{0.000000in}}%
\pgfusepath{stroke,fill}%
}%
\begin{pgfscope}%
\pgfsys@transformshift{3.286364in}{1.925000in}%
\pgfsys@useobject{currentmarker}{}%
\end{pgfscope}%
\end{pgfscope}%
\begin{pgfscope}%
\pgfsetbuttcap%
\pgfsetroundjoin%
\definecolor{currentfill}{rgb}{0.150000,0.150000,0.150000}%
\pgfsetfillcolor{currentfill}%
\pgfsetlinewidth{0.803000pt}%
\definecolor{currentstroke}{rgb}{0.150000,0.150000,0.150000}%
\pgfsetstrokecolor{currentstroke}%
\pgfsetdash{}{0pt}%
\pgfsys@defobject{currentmarker}{\pgfqpoint{0.000000in}{0.000000in}}{\pgfqpoint{0.000000in}{0.000000in}}{%
\pgfpathmoveto{\pgfqpoint{0.000000in}{0.000000in}}%
\pgfpathlineto{\pgfqpoint{0.000000in}{0.000000in}}%
\pgfusepath{stroke,fill}%
}%
\begin{pgfscope}%
\pgfsys@transformshift{5.400000in}{1.925000in}%
\pgfsys@useobject{currentmarker}{}%
\end{pgfscope}%
\end{pgfscope}%
\begin{pgfscope}%
\pgfpathrectangle{\pgfqpoint{3.286364in}{1.770000in}}{\pgfqpoint{2.113636in}{0.930000in}} %
\pgfusepath{clip}%
\pgfsetroundcap%
\pgfsetroundjoin%
\pgfsetlinewidth{0.803000pt}%
\definecolor{currentstroke}{rgb}{1.000000,1.000000,1.000000}%
\pgfsetstrokecolor{currentstroke}%
\pgfsetdash{}{0pt}%
\pgfpathmoveto{\pgfqpoint{3.286364in}{2.080000in}}%
\pgfpathlineto{\pgfqpoint{5.400000in}{2.080000in}}%
\pgfusepath{stroke}%
\end{pgfscope}%
\begin{pgfscope}%
\pgfsetbuttcap%
\pgfsetroundjoin%
\definecolor{currentfill}{rgb}{0.150000,0.150000,0.150000}%
\pgfsetfillcolor{currentfill}%
\pgfsetlinewidth{0.803000pt}%
\definecolor{currentstroke}{rgb}{0.150000,0.150000,0.150000}%
\pgfsetstrokecolor{currentstroke}%
\pgfsetdash{}{0pt}%
\pgfsys@defobject{currentmarker}{\pgfqpoint{0.000000in}{0.000000in}}{\pgfqpoint{0.000000in}{0.000000in}}{%
\pgfpathmoveto{\pgfqpoint{0.000000in}{0.000000in}}%
\pgfpathlineto{\pgfqpoint{0.000000in}{0.000000in}}%
\pgfusepath{stroke,fill}%
}%
\begin{pgfscope}%
\pgfsys@transformshift{3.286364in}{2.080000in}%
\pgfsys@useobject{currentmarker}{}%
\end{pgfscope}%
\end{pgfscope}%
\begin{pgfscope}%
\pgfsetbuttcap%
\pgfsetroundjoin%
\definecolor{currentfill}{rgb}{0.150000,0.150000,0.150000}%
\pgfsetfillcolor{currentfill}%
\pgfsetlinewidth{0.803000pt}%
\definecolor{currentstroke}{rgb}{0.150000,0.150000,0.150000}%
\pgfsetstrokecolor{currentstroke}%
\pgfsetdash{}{0pt}%
\pgfsys@defobject{currentmarker}{\pgfqpoint{0.000000in}{0.000000in}}{\pgfqpoint{0.000000in}{0.000000in}}{%
\pgfpathmoveto{\pgfqpoint{0.000000in}{0.000000in}}%
\pgfpathlineto{\pgfqpoint{0.000000in}{0.000000in}}%
\pgfusepath{stroke,fill}%
}%
\begin{pgfscope}%
\pgfsys@transformshift{5.400000in}{2.080000in}%
\pgfsys@useobject{currentmarker}{}%
\end{pgfscope}%
\end{pgfscope}%
\begin{pgfscope}%
\pgfpathrectangle{\pgfqpoint{3.286364in}{1.770000in}}{\pgfqpoint{2.113636in}{0.930000in}} %
\pgfusepath{clip}%
\pgfsetroundcap%
\pgfsetroundjoin%
\pgfsetlinewidth{0.803000pt}%
\definecolor{currentstroke}{rgb}{1.000000,1.000000,1.000000}%
\pgfsetstrokecolor{currentstroke}%
\pgfsetdash{}{0pt}%
\pgfpathmoveto{\pgfqpoint{3.286364in}{2.235000in}}%
\pgfpathlineto{\pgfqpoint{5.400000in}{2.235000in}}%
\pgfusepath{stroke}%
\end{pgfscope}%
\begin{pgfscope}%
\pgfsetbuttcap%
\pgfsetroundjoin%
\definecolor{currentfill}{rgb}{0.150000,0.150000,0.150000}%
\pgfsetfillcolor{currentfill}%
\pgfsetlinewidth{0.803000pt}%
\definecolor{currentstroke}{rgb}{0.150000,0.150000,0.150000}%
\pgfsetstrokecolor{currentstroke}%
\pgfsetdash{}{0pt}%
\pgfsys@defobject{currentmarker}{\pgfqpoint{0.000000in}{0.000000in}}{\pgfqpoint{0.000000in}{0.000000in}}{%
\pgfpathmoveto{\pgfqpoint{0.000000in}{0.000000in}}%
\pgfpathlineto{\pgfqpoint{0.000000in}{0.000000in}}%
\pgfusepath{stroke,fill}%
}%
\begin{pgfscope}%
\pgfsys@transformshift{3.286364in}{2.235000in}%
\pgfsys@useobject{currentmarker}{}%
\end{pgfscope}%
\end{pgfscope}%
\begin{pgfscope}%
\pgfsetbuttcap%
\pgfsetroundjoin%
\definecolor{currentfill}{rgb}{0.150000,0.150000,0.150000}%
\pgfsetfillcolor{currentfill}%
\pgfsetlinewidth{0.803000pt}%
\definecolor{currentstroke}{rgb}{0.150000,0.150000,0.150000}%
\pgfsetstrokecolor{currentstroke}%
\pgfsetdash{}{0pt}%
\pgfsys@defobject{currentmarker}{\pgfqpoint{0.000000in}{0.000000in}}{\pgfqpoint{0.000000in}{0.000000in}}{%
\pgfpathmoveto{\pgfqpoint{0.000000in}{0.000000in}}%
\pgfpathlineto{\pgfqpoint{0.000000in}{0.000000in}}%
\pgfusepath{stroke,fill}%
}%
\begin{pgfscope}%
\pgfsys@transformshift{5.400000in}{2.235000in}%
\pgfsys@useobject{currentmarker}{}%
\end{pgfscope}%
\end{pgfscope}%
\begin{pgfscope}%
\pgfpathrectangle{\pgfqpoint{3.286364in}{1.770000in}}{\pgfqpoint{2.113636in}{0.930000in}} %
\pgfusepath{clip}%
\pgfsetroundcap%
\pgfsetroundjoin%
\pgfsetlinewidth{0.803000pt}%
\definecolor{currentstroke}{rgb}{1.000000,1.000000,1.000000}%
\pgfsetstrokecolor{currentstroke}%
\pgfsetdash{}{0pt}%
\pgfpathmoveto{\pgfqpoint{3.286364in}{2.390000in}}%
\pgfpathlineto{\pgfqpoint{5.400000in}{2.390000in}}%
\pgfusepath{stroke}%
\end{pgfscope}%
\begin{pgfscope}%
\pgfsetbuttcap%
\pgfsetroundjoin%
\definecolor{currentfill}{rgb}{0.150000,0.150000,0.150000}%
\pgfsetfillcolor{currentfill}%
\pgfsetlinewidth{0.803000pt}%
\definecolor{currentstroke}{rgb}{0.150000,0.150000,0.150000}%
\pgfsetstrokecolor{currentstroke}%
\pgfsetdash{}{0pt}%
\pgfsys@defobject{currentmarker}{\pgfqpoint{0.000000in}{0.000000in}}{\pgfqpoint{0.000000in}{0.000000in}}{%
\pgfpathmoveto{\pgfqpoint{0.000000in}{0.000000in}}%
\pgfpathlineto{\pgfqpoint{0.000000in}{0.000000in}}%
\pgfusepath{stroke,fill}%
}%
\begin{pgfscope}%
\pgfsys@transformshift{3.286364in}{2.390000in}%
\pgfsys@useobject{currentmarker}{}%
\end{pgfscope}%
\end{pgfscope}%
\begin{pgfscope}%
\pgfsetbuttcap%
\pgfsetroundjoin%
\definecolor{currentfill}{rgb}{0.150000,0.150000,0.150000}%
\pgfsetfillcolor{currentfill}%
\pgfsetlinewidth{0.803000pt}%
\definecolor{currentstroke}{rgb}{0.150000,0.150000,0.150000}%
\pgfsetstrokecolor{currentstroke}%
\pgfsetdash{}{0pt}%
\pgfsys@defobject{currentmarker}{\pgfqpoint{0.000000in}{0.000000in}}{\pgfqpoint{0.000000in}{0.000000in}}{%
\pgfpathmoveto{\pgfqpoint{0.000000in}{0.000000in}}%
\pgfpathlineto{\pgfqpoint{0.000000in}{0.000000in}}%
\pgfusepath{stroke,fill}%
}%
\begin{pgfscope}%
\pgfsys@transformshift{5.400000in}{2.390000in}%
\pgfsys@useobject{currentmarker}{}%
\end{pgfscope}%
\end{pgfscope}%
\begin{pgfscope}%
\pgfpathrectangle{\pgfqpoint{3.286364in}{1.770000in}}{\pgfqpoint{2.113636in}{0.930000in}} %
\pgfusepath{clip}%
\pgfsetroundcap%
\pgfsetroundjoin%
\pgfsetlinewidth{0.803000pt}%
\definecolor{currentstroke}{rgb}{1.000000,1.000000,1.000000}%
\pgfsetstrokecolor{currentstroke}%
\pgfsetdash{}{0pt}%
\pgfpathmoveto{\pgfqpoint{3.286364in}{2.545000in}}%
\pgfpathlineto{\pgfqpoint{5.400000in}{2.545000in}}%
\pgfusepath{stroke}%
\end{pgfscope}%
\begin{pgfscope}%
\pgfsetbuttcap%
\pgfsetroundjoin%
\definecolor{currentfill}{rgb}{0.150000,0.150000,0.150000}%
\pgfsetfillcolor{currentfill}%
\pgfsetlinewidth{0.803000pt}%
\definecolor{currentstroke}{rgb}{0.150000,0.150000,0.150000}%
\pgfsetstrokecolor{currentstroke}%
\pgfsetdash{}{0pt}%
\pgfsys@defobject{currentmarker}{\pgfqpoint{0.000000in}{0.000000in}}{\pgfqpoint{0.000000in}{0.000000in}}{%
\pgfpathmoveto{\pgfqpoint{0.000000in}{0.000000in}}%
\pgfpathlineto{\pgfqpoint{0.000000in}{0.000000in}}%
\pgfusepath{stroke,fill}%
}%
\begin{pgfscope}%
\pgfsys@transformshift{3.286364in}{2.545000in}%
\pgfsys@useobject{currentmarker}{}%
\end{pgfscope}%
\end{pgfscope}%
\begin{pgfscope}%
\pgfsetbuttcap%
\pgfsetroundjoin%
\definecolor{currentfill}{rgb}{0.150000,0.150000,0.150000}%
\pgfsetfillcolor{currentfill}%
\pgfsetlinewidth{0.803000pt}%
\definecolor{currentstroke}{rgb}{0.150000,0.150000,0.150000}%
\pgfsetstrokecolor{currentstroke}%
\pgfsetdash{}{0pt}%
\pgfsys@defobject{currentmarker}{\pgfqpoint{0.000000in}{0.000000in}}{\pgfqpoint{0.000000in}{0.000000in}}{%
\pgfpathmoveto{\pgfqpoint{0.000000in}{0.000000in}}%
\pgfpathlineto{\pgfqpoint{0.000000in}{0.000000in}}%
\pgfusepath{stroke,fill}%
}%
\begin{pgfscope}%
\pgfsys@transformshift{5.400000in}{2.545000in}%
\pgfsys@useobject{currentmarker}{}%
\end{pgfscope}%
\end{pgfscope}%
\begin{pgfscope}%
\pgfpathrectangle{\pgfqpoint{3.286364in}{1.770000in}}{\pgfqpoint{2.113636in}{0.930000in}} %
\pgfusepath{clip}%
\pgfsetroundcap%
\pgfsetroundjoin%
\pgfsetlinewidth{0.803000pt}%
\definecolor{currentstroke}{rgb}{1.000000,1.000000,1.000000}%
\pgfsetstrokecolor{currentstroke}%
\pgfsetdash{}{0pt}%
\pgfpathmoveto{\pgfqpoint{3.286364in}{2.700000in}}%
\pgfpathlineto{\pgfqpoint{5.400000in}{2.700000in}}%
\pgfusepath{stroke}%
\end{pgfscope}%
\begin{pgfscope}%
\pgfsetbuttcap%
\pgfsetroundjoin%
\definecolor{currentfill}{rgb}{0.150000,0.150000,0.150000}%
\pgfsetfillcolor{currentfill}%
\pgfsetlinewidth{0.803000pt}%
\definecolor{currentstroke}{rgb}{0.150000,0.150000,0.150000}%
\pgfsetstrokecolor{currentstroke}%
\pgfsetdash{}{0pt}%
\pgfsys@defobject{currentmarker}{\pgfqpoint{0.000000in}{0.000000in}}{\pgfqpoint{0.000000in}{0.000000in}}{%
\pgfpathmoveto{\pgfqpoint{0.000000in}{0.000000in}}%
\pgfpathlineto{\pgfqpoint{0.000000in}{0.000000in}}%
\pgfusepath{stroke,fill}%
}%
\begin{pgfscope}%
\pgfsys@transformshift{3.286364in}{2.700000in}%
\pgfsys@useobject{currentmarker}{}%
\end{pgfscope}%
\end{pgfscope}%
\begin{pgfscope}%
\pgfsetbuttcap%
\pgfsetroundjoin%
\definecolor{currentfill}{rgb}{0.150000,0.150000,0.150000}%
\pgfsetfillcolor{currentfill}%
\pgfsetlinewidth{0.803000pt}%
\definecolor{currentstroke}{rgb}{0.150000,0.150000,0.150000}%
\pgfsetstrokecolor{currentstroke}%
\pgfsetdash{}{0pt}%
\pgfsys@defobject{currentmarker}{\pgfqpoint{0.000000in}{0.000000in}}{\pgfqpoint{0.000000in}{0.000000in}}{%
\pgfpathmoveto{\pgfqpoint{0.000000in}{0.000000in}}%
\pgfpathlineto{\pgfqpoint{0.000000in}{0.000000in}}%
\pgfusepath{stroke,fill}%
}%
\begin{pgfscope}%
\pgfsys@transformshift{5.400000in}{2.700000in}%
\pgfsys@useobject{currentmarker}{}%
\end{pgfscope}%
\end{pgfscope}%
\begin{pgfscope}%
\pgfpathrectangle{\pgfqpoint{3.286364in}{1.770000in}}{\pgfqpoint{2.113636in}{0.930000in}} %
\pgfusepath{clip}%
\pgfsetbuttcap%
\pgfsetmiterjoin%
\definecolor{currentfill}{rgb}{0.447059,0.623529,0.811765}%
\pgfsetfillcolor{currentfill}%
\pgfsetfillopacity{0.300000}%
\pgfsetlinewidth{0.240900pt}%
\definecolor{currentstroke}{rgb}{0.447059,0.623529,0.811765}%
\pgfsetstrokecolor{currentstroke}%
\pgfsetstrokeopacity{0.300000}%
\pgfsetdash{}{0pt}%
\pgfpathmoveto{\pgfqpoint{3.455455in}{2.255378in}}%
\pgfpathlineto{\pgfqpoint{3.473388in}{2.229695in}}%
\pgfpathlineto{\pgfqpoint{3.491322in}{2.205998in}}%
\pgfpathlineto{\pgfqpoint{3.509256in}{2.184290in}}%
\pgfpathlineto{\pgfqpoint{3.527190in}{2.164560in}}%
\pgfpathlineto{\pgfqpoint{3.545124in}{2.146780in}}%
\pgfpathlineto{\pgfqpoint{3.563058in}{2.130902in}}%
\pgfpathlineto{\pgfqpoint{3.580992in}{2.116863in}}%
\pgfpathlineto{\pgfqpoint{3.598926in}{2.104582in}}%
\pgfpathlineto{\pgfqpoint{3.616860in}{2.093964in}}%
\pgfpathlineto{\pgfqpoint{3.634793in}{2.084911in}}%
\pgfpathlineto{\pgfqpoint{3.652727in}{2.077318in}}%
\pgfpathlineto{\pgfqpoint{3.670661in}{2.071082in}}%
\pgfpathlineto{\pgfqpoint{3.688595in}{2.066107in}}%
\pgfpathlineto{\pgfqpoint{3.706529in}{2.062301in}}%
\pgfpathlineto{\pgfqpoint{3.724463in}{2.059580in}}%
\pgfpathlineto{\pgfqpoint{3.742397in}{2.057869in}}%
\pgfpathlineto{\pgfqpoint{3.760331in}{2.057101in}}%
\pgfpathlineto{\pgfqpoint{3.778264in}{2.057215in}}%
\pgfpathlineto{\pgfqpoint{3.796198in}{2.058158in}}%
\pgfpathlineto{\pgfqpoint{3.814132in}{2.059882in}}%
\pgfpathlineto{\pgfqpoint{3.832066in}{2.062345in}}%
\pgfpathlineto{\pgfqpoint{3.850000in}{2.065507in}}%
\pgfpathlineto{\pgfqpoint{3.867934in}{2.069334in}}%
\pgfpathlineto{\pgfqpoint{3.885868in}{2.073795in}}%
\pgfpathlineto{\pgfqpoint{3.903802in}{2.078861in}}%
\pgfpathlineto{\pgfqpoint{3.921736in}{2.084506in}}%
\pgfpathlineto{\pgfqpoint{3.939669in}{2.090707in}}%
\pgfpathlineto{\pgfqpoint{3.957603in}{2.097443in}}%
\pgfpathlineto{\pgfqpoint{3.975537in}{2.104694in}}%
\pgfpathlineto{\pgfqpoint{3.993471in}{2.112444in}}%
\pgfpathlineto{\pgfqpoint{4.011405in}{2.120678in}}%
\pgfpathlineto{\pgfqpoint{4.029339in}{2.129384in}}%
\pgfpathlineto{\pgfqpoint{4.047273in}{2.138553in}}%
\pgfpathlineto{\pgfqpoint{4.065207in}{2.148177in}}%
\pgfpathlineto{\pgfqpoint{4.083140in}{2.158252in}}%
\pgfpathlineto{\pgfqpoint{4.101074in}{2.168776in}}%
\pgfpathlineto{\pgfqpoint{4.119008in}{2.179750in}}%
\pgfpathlineto{\pgfqpoint{4.136942in}{2.191176in}}%
\pgfpathlineto{\pgfqpoint{4.154876in}{2.203058in}}%
\pgfpathlineto{\pgfqpoint{4.172810in}{2.215399in}}%
\pgfpathlineto{\pgfqpoint{4.190744in}{2.228197in}}%
\pgfpathlineto{\pgfqpoint{4.208678in}{2.241448in}}%
\pgfpathlineto{\pgfqpoint{4.226612in}{2.255139in}}%
\pgfpathlineto{\pgfqpoint{4.244545in}{2.269247in}}%
\pgfpathlineto{\pgfqpoint{4.262479in}{2.283737in}}%
\pgfpathlineto{\pgfqpoint{4.280413in}{2.298561in}}%
\pgfpathlineto{\pgfqpoint{4.298347in}{2.313661in}}%
\pgfpathlineto{\pgfqpoint{4.316281in}{2.328968in}}%
\pgfpathlineto{\pgfqpoint{4.334215in}{2.344408in}}%
\pgfpathlineto{\pgfqpoint{4.352149in}{2.359901in}}%
\pgfpathlineto{\pgfqpoint{4.370083in}{2.375365in}}%
\pgfpathlineto{\pgfqpoint{4.388017in}{2.390721in}}%
\pgfpathlineto{\pgfqpoint{4.405950in}{2.405889in}}%
\pgfpathlineto{\pgfqpoint{4.423884in}{2.420793in}}%
\pgfpathlineto{\pgfqpoint{4.441818in}{2.435363in}}%
\pgfpathlineto{\pgfqpoint{4.459752in}{2.449529in}}%
\pgfpathlineto{\pgfqpoint{4.477686in}{2.463229in}}%
\pgfpathlineto{\pgfqpoint{4.495620in}{2.476403in}}%
\pgfpathlineto{\pgfqpoint{4.513554in}{2.488994in}}%
\pgfpathlineto{\pgfqpoint{4.531488in}{2.500953in}}%
\pgfpathlineto{\pgfqpoint{4.549421in}{2.512231in}}%
\pgfpathlineto{\pgfqpoint{4.567355in}{2.522785in}}%
\pgfpathlineto{\pgfqpoint{4.585289in}{2.532575in}}%
\pgfpathlineto{\pgfqpoint{4.603223in}{2.541567in}}%
\pgfpathlineto{\pgfqpoint{4.621157in}{2.549730in}}%
\pgfpathlineto{\pgfqpoint{4.639091in}{2.557036in}}%
\pgfpathlineto{\pgfqpoint{4.657025in}{2.563463in}}%
\pgfpathlineto{\pgfqpoint{4.674959in}{2.568996in}}%
\pgfpathlineto{\pgfqpoint{4.692893in}{2.573622in}}%
\pgfpathlineto{\pgfqpoint{4.710826in}{2.577336in}}%
\pgfpathlineto{\pgfqpoint{4.728760in}{2.580138in}}%
\pgfpathlineto{\pgfqpoint{4.746694in}{2.582036in}}%
\pgfpathlineto{\pgfqpoint{4.764628in}{2.583045in}}%
\pgfpathlineto{\pgfqpoint{4.782562in}{2.583189in}}%
\pgfpathlineto{\pgfqpoint{4.800496in}{2.582500in}}%
\pgfpathlineto{\pgfqpoint{4.818430in}{2.581019in}}%
\pgfpathlineto{\pgfqpoint{4.836364in}{2.578797in}}%
\pgfpathlineto{\pgfqpoint{4.854298in}{2.575893in}}%
\pgfpathlineto{\pgfqpoint{4.872231in}{2.572375in}}%
\pgfpathlineto{\pgfqpoint{4.890165in}{2.568316in}}%
\pgfpathlineto{\pgfqpoint{4.908099in}{2.563793in}}%
\pgfpathlineto{\pgfqpoint{4.926033in}{2.558882in}}%
\pgfpathlineto{\pgfqpoint{4.943967in}{2.553655in}}%
\pgfpathlineto{\pgfqpoint{4.961901in}{2.548177in}}%
\pgfpathlineto{\pgfqpoint{4.979835in}{2.542500in}}%
\pgfpathlineto{\pgfqpoint{4.997769in}{2.536665in}}%
\pgfpathlineto{\pgfqpoint{5.015702in}{2.530698in}}%
\pgfpathlineto{\pgfqpoint{5.033636in}{2.524614in}}%
\pgfpathlineto{\pgfqpoint{5.051570in}{2.518418in}}%
\pgfpathlineto{\pgfqpoint{5.069504in}{2.512104in}}%
\pgfpathlineto{\pgfqpoint{5.087438in}{2.505665in}}%
\pgfpathlineto{\pgfqpoint{5.105372in}{2.499088in}}%
\pgfpathlineto{\pgfqpoint{5.123306in}{2.492362in}}%
\pgfpathlineto{\pgfqpoint{5.141240in}{2.485473in}}%
\pgfpathlineto{\pgfqpoint{5.159174in}{2.478410in}}%
\pgfpathlineto{\pgfqpoint{5.177107in}{2.471165in}}%
\pgfpathlineto{\pgfqpoint{5.195041in}{2.463730in}}%
\pgfpathlineto{\pgfqpoint{5.212975in}{2.456102in}}%
\pgfpathlineto{\pgfqpoint{5.230909in}{2.448278in}}%
\pgfpathlineto{\pgfqpoint{5.230909in}{2.039630in}}%
\pgfpathlineto{\pgfqpoint{5.212975in}{2.070412in}}%
\pgfpathlineto{\pgfqpoint{5.195041in}{2.100131in}}%
\pgfpathlineto{\pgfqpoint{5.177107in}{2.128738in}}%
\pgfpathlineto{\pgfqpoint{5.159174in}{2.156184in}}%
\pgfpathlineto{\pgfqpoint{5.141240in}{2.182418in}}%
\pgfpathlineto{\pgfqpoint{5.123306in}{2.207391in}}%
\pgfpathlineto{\pgfqpoint{5.105372in}{2.231054in}}%
\pgfpathlineto{\pgfqpoint{5.087438in}{2.253359in}}%
\pgfpathlineto{\pgfqpoint{5.069504in}{2.274261in}}%
\pgfpathlineto{\pgfqpoint{5.051570in}{2.293719in}}%
\pgfpathlineto{\pgfqpoint{5.033636in}{2.311695in}}%
\pgfpathlineto{\pgfqpoint{5.015702in}{2.328162in}}%
\pgfpathlineto{\pgfqpoint{4.997769in}{2.343102in}}%
\pgfpathlineto{\pgfqpoint{4.979835in}{2.356511in}}%
\pgfpathlineto{\pgfqpoint{4.961901in}{2.368401in}}%
\pgfpathlineto{\pgfqpoint{4.943967in}{2.378798in}}%
\pgfpathlineto{\pgfqpoint{4.926033in}{2.387746in}}%
\pgfpathlineto{\pgfqpoint{4.908099in}{2.395304in}}%
\pgfpathlineto{\pgfqpoint{4.890165in}{2.401538in}}%
\pgfpathlineto{\pgfqpoint{4.872231in}{2.406526in}}%
\pgfpathlineto{\pgfqpoint{4.854298in}{2.410347in}}%
\pgfpathlineto{\pgfqpoint{4.836364in}{2.413080in}}%
\pgfpathlineto{\pgfqpoint{4.818430in}{2.414802in}}%
\pgfpathlineto{\pgfqpoint{4.800496in}{2.415583in}}%
\pgfpathlineto{\pgfqpoint{4.782562in}{2.415492in}}%
\pgfpathlineto{\pgfqpoint{4.764628in}{2.414586in}}%
\pgfpathlineto{\pgfqpoint{4.746694in}{2.412919in}}%
\pgfpathlineto{\pgfqpoint{4.728760in}{2.410540in}}%
\pgfpathlineto{\pgfqpoint{4.710826in}{2.407492in}}%
\pgfpathlineto{\pgfqpoint{4.692893in}{2.403812in}}%
\pgfpathlineto{\pgfqpoint{4.674959in}{2.399535in}}%
\pgfpathlineto{\pgfqpoint{4.657025in}{2.394690in}}%
\pgfpathlineto{\pgfqpoint{4.639091in}{2.389307in}}%
\pgfpathlineto{\pgfqpoint{4.621157in}{2.383410in}}%
\pgfpathlineto{\pgfqpoint{4.603223in}{2.377022in}}%
\pgfpathlineto{\pgfqpoint{4.585289in}{2.370163in}}%
\pgfpathlineto{\pgfqpoint{4.567355in}{2.362853in}}%
\pgfpathlineto{\pgfqpoint{4.549421in}{2.355108in}}%
\pgfpathlineto{\pgfqpoint{4.531488in}{2.346945in}}%
\pgfpathlineto{\pgfqpoint{4.513554in}{2.338377in}}%
\pgfpathlineto{\pgfqpoint{4.495620in}{2.329416in}}%
\pgfpathlineto{\pgfqpoint{4.477686in}{2.320072in}}%
\pgfpathlineto{\pgfqpoint{4.459752in}{2.310354in}}%
\pgfpathlineto{\pgfqpoint{4.441818in}{2.300267in}}%
\pgfpathlineto{\pgfqpoint{4.423884in}{2.289816in}}%
\pgfpathlineto{\pgfqpoint{4.405950in}{2.279001in}}%
\pgfpathlineto{\pgfqpoint{4.388017in}{2.267822in}}%
\pgfpathlineto{\pgfqpoint{4.370083in}{2.256276in}}%
\pgfpathlineto{\pgfqpoint{4.352149in}{2.244358in}}%
\pgfpathlineto{\pgfqpoint{4.334215in}{2.232061in}}%
\pgfpathlineto{\pgfqpoint{4.316281in}{2.219382in}}%
\pgfpathlineto{\pgfqpoint{4.298347in}{2.206317in}}%
\pgfpathlineto{\pgfqpoint{4.280413in}{2.192870in}}%
\pgfpathlineto{\pgfqpoint{4.262479in}{2.179052in}}%
\pgfpathlineto{\pgfqpoint{4.244545in}{2.164883in}}%
\pgfpathlineto{\pgfqpoint{4.226612in}{2.150397in}}%
\pgfpathlineto{\pgfqpoint{4.208678in}{2.135638in}}%
\pgfpathlineto{\pgfqpoint{4.190744in}{2.120665in}}%
\pgfpathlineto{\pgfqpoint{4.172810in}{2.105545in}}%
\pgfpathlineto{\pgfqpoint{4.154876in}{2.090354in}}%
\pgfpathlineto{\pgfqpoint{4.136942in}{2.075173in}}%
\pgfpathlineto{\pgfqpoint{4.119008in}{2.060085in}}%
\pgfpathlineto{\pgfqpoint{4.101074in}{2.045173in}}%
\pgfpathlineto{\pgfqpoint{4.083140in}{2.030520in}}%
\pgfpathlineto{\pgfqpoint{4.065207in}{2.016204in}}%
\pgfpathlineto{\pgfqpoint{4.047273in}{2.002302in}}%
\pgfpathlineto{\pgfqpoint{4.029339in}{1.988887in}}%
\pgfpathlineto{\pgfqpoint{4.011405in}{1.976028in}}%
\pgfpathlineto{\pgfqpoint{3.993471in}{1.963790in}}%
\pgfpathlineto{\pgfqpoint{3.975537in}{1.952236in}}%
\pgfpathlineto{\pgfqpoint{3.957603in}{1.941422in}}%
\pgfpathlineto{\pgfqpoint{3.939669in}{1.931403in}}%
\pgfpathlineto{\pgfqpoint{3.921736in}{1.922229in}}%
\pgfpathlineto{\pgfqpoint{3.903802in}{1.913946in}}%
\pgfpathlineto{\pgfqpoint{3.885868in}{1.906596in}}%
\pgfpathlineto{\pgfqpoint{3.867934in}{1.900218in}}%
\pgfpathlineto{\pgfqpoint{3.850000in}{1.894843in}}%
\pgfpathlineto{\pgfqpoint{3.832066in}{1.890500in}}%
\pgfpathlineto{\pgfqpoint{3.814132in}{1.887211in}}%
\pgfpathlineto{\pgfqpoint{3.796198in}{1.884994in}}%
\pgfpathlineto{\pgfqpoint{3.778264in}{1.883856in}}%
\pgfpathlineto{\pgfqpoint{3.760331in}{1.883802in}}%
\pgfpathlineto{\pgfqpoint{3.742397in}{1.884823in}}%
\pgfpathlineto{\pgfqpoint{3.724463in}{1.886906in}}%
\pgfpathlineto{\pgfqpoint{3.706529in}{1.890023in}}%
\pgfpathlineto{\pgfqpoint{3.688595in}{1.894140in}}%
\pgfpathlineto{\pgfqpoint{3.670661in}{1.899209in}}%
\pgfpathlineto{\pgfqpoint{3.652727in}{1.905173in}}%
\pgfpathlineto{\pgfqpoint{3.634793in}{1.911965in}}%
\pgfpathlineto{\pgfqpoint{3.616860in}{1.919509in}}%
\pgfpathlineto{\pgfqpoint{3.598926in}{1.927727in}}%
\pgfpathlineto{\pgfqpoint{3.580992in}{1.936540in}}%
\pgfpathlineto{\pgfqpoint{3.563058in}{1.945874in}}%
\pgfpathlineto{\pgfqpoint{3.545124in}{1.955662in}}%
\pgfpathlineto{\pgfqpoint{3.527190in}{1.965854in}}%
\pgfpathlineto{\pgfqpoint{3.509256in}{1.976411in}}%
\pgfpathlineto{\pgfqpoint{3.491322in}{1.987312in}}%
\pgfpathlineto{\pgfqpoint{3.473388in}{1.998548in}}%
\pgfpathlineto{\pgfqpoint{3.455455in}{2.010124in}}%
\pgfpathclose%
\pgfusepath{stroke,fill}%
\end{pgfscope}%
\begin{pgfscope}%
\pgfpathrectangle{\pgfqpoint{3.286364in}{1.770000in}}{\pgfqpoint{2.113636in}{0.930000in}} %
\pgfusepath{clip}%
\pgfsetroundcap%
\pgfsetroundjoin%
\pgfsetlinewidth{2.007500pt}%
\definecolor{currentstroke}{rgb}{0.125490,0.290196,0.529412}%
\pgfsetstrokecolor{currentstroke}%
\pgfsetdash{}{0pt}%
\pgfpathmoveto{\pgfqpoint{3.455455in}{2.132751in}}%
\pgfpathlineto{\pgfqpoint{3.473388in}{2.114122in}}%
\pgfpathlineto{\pgfqpoint{3.491322in}{2.096655in}}%
\pgfpathlineto{\pgfqpoint{3.509256in}{2.080351in}}%
\pgfpathlineto{\pgfqpoint{3.527190in}{2.065207in}}%
\pgfpathlineto{\pgfqpoint{3.545124in}{2.051221in}}%
\pgfpathlineto{\pgfqpoint{3.563058in}{2.038388in}}%
\pgfpathlineto{\pgfqpoint{3.580992in}{2.026702in}}%
\pgfpathlineto{\pgfqpoint{3.598926in}{2.016154in}}%
\pgfpathlineto{\pgfqpoint{3.616860in}{2.006737in}}%
\pgfpathlineto{\pgfqpoint{3.634793in}{1.998438in}}%
\pgfpathlineto{\pgfqpoint{3.652727in}{1.991246in}}%
\pgfpathlineto{\pgfqpoint{3.670661in}{1.985146in}}%
\pgfpathlineto{\pgfqpoint{3.688595in}{1.980124in}}%
\pgfpathlineto{\pgfqpoint{3.706529in}{1.976162in}}%
\pgfpathlineto{\pgfqpoint{3.724463in}{1.973243in}}%
\pgfpathlineto{\pgfqpoint{3.742397in}{1.971346in}}%
\pgfpathlineto{\pgfqpoint{3.760331in}{1.970451in}}%
\pgfpathlineto{\pgfqpoint{3.778264in}{1.970536in}}%
\pgfpathlineto{\pgfqpoint{3.796198in}{1.971576in}}%
\pgfpathlineto{\pgfqpoint{3.814132in}{1.973547in}}%
\pgfpathlineto{\pgfqpoint{3.832066in}{1.976422in}}%
\pgfpathlineto{\pgfqpoint{3.850000in}{1.980175in}}%
\pgfpathlineto{\pgfqpoint{3.867934in}{1.984776in}}%
\pgfpathlineto{\pgfqpoint{3.885868in}{1.990196in}}%
\pgfpathlineto{\pgfqpoint{3.903802in}{1.996404in}}%
\pgfpathlineto{\pgfqpoint{3.921736in}{2.003368in}}%
\pgfpathlineto{\pgfqpoint{3.939669in}{2.011055in}}%
\pgfpathlineto{\pgfqpoint{3.957603in}{2.019432in}}%
\pgfpathlineto{\pgfqpoint{3.975537in}{2.028465in}}%
\pgfpathlineto{\pgfqpoint{3.993471in}{2.038117in}}%
\pgfpathlineto{\pgfqpoint{4.011405in}{2.048353in}}%
\pgfpathlineto{\pgfqpoint{4.029339in}{2.059135in}}%
\pgfpathlineto{\pgfqpoint{4.047273in}{2.070427in}}%
\pgfpathlineto{\pgfqpoint{4.065207in}{2.082190in}}%
\pgfpathlineto{\pgfqpoint{4.083140in}{2.094386in}}%
\pgfpathlineto{\pgfqpoint{4.101074in}{2.106975in}}%
\pgfpathlineto{\pgfqpoint{4.119008in}{2.119918in}}%
\pgfpathlineto{\pgfqpoint{4.136942in}{2.133175in}}%
\pgfpathlineto{\pgfqpoint{4.154876in}{2.146706in}}%
\pgfpathlineto{\pgfqpoint{4.172810in}{2.160472in}}%
\pgfpathlineto{\pgfqpoint{4.190744in}{2.174431in}}%
\pgfpathlineto{\pgfqpoint{4.208678in}{2.188543in}}%
\pgfpathlineto{\pgfqpoint{4.226612in}{2.202768in}}%
\pgfpathlineto{\pgfqpoint{4.244545in}{2.217065in}}%
\pgfpathlineto{\pgfqpoint{4.262479in}{2.231394in}}%
\pgfpathlineto{\pgfqpoint{4.280413in}{2.245716in}}%
\pgfpathlineto{\pgfqpoint{4.298347in}{2.259989in}}%
\pgfpathlineto{\pgfqpoint{4.316281in}{2.274175in}}%
\pgfpathlineto{\pgfqpoint{4.334215in}{2.288235in}}%
\pgfpathlineto{\pgfqpoint{4.352149in}{2.302129in}}%
\pgfpathlineto{\pgfqpoint{4.370083in}{2.315821in}}%
\pgfpathlineto{\pgfqpoint{4.388017in}{2.329272in}}%
\pgfpathlineto{\pgfqpoint{4.405950in}{2.342445in}}%
\pgfpathlineto{\pgfqpoint{4.423884in}{2.355305in}}%
\pgfpathlineto{\pgfqpoint{4.441818in}{2.367815in}}%
\pgfpathlineto{\pgfqpoint{4.459752in}{2.379942in}}%
\pgfpathlineto{\pgfqpoint{4.477686in}{2.391651in}}%
\pgfpathlineto{\pgfqpoint{4.495620in}{2.402909in}}%
\pgfpathlineto{\pgfqpoint{4.513554in}{2.413686in}}%
\pgfpathlineto{\pgfqpoint{4.531488in}{2.423949in}}%
\pgfpathlineto{\pgfqpoint{4.549421in}{2.433670in}}%
\pgfpathlineto{\pgfqpoint{4.567355in}{2.442819in}}%
\pgfpathlineto{\pgfqpoint{4.585289in}{2.451369in}}%
\pgfpathlineto{\pgfqpoint{4.603223in}{2.459295in}}%
\pgfpathlineto{\pgfqpoint{4.621157in}{2.466570in}}%
\pgfpathlineto{\pgfqpoint{4.639091in}{2.473171in}}%
\pgfpathlineto{\pgfqpoint{4.657025in}{2.479077in}}%
\pgfpathlineto{\pgfqpoint{4.674959in}{2.484265in}}%
\pgfpathlineto{\pgfqpoint{4.692893in}{2.488717in}}%
\pgfpathlineto{\pgfqpoint{4.710826in}{2.492414in}}%
\pgfpathlineto{\pgfqpoint{4.728760in}{2.495339in}}%
\pgfpathlineto{\pgfqpoint{4.746694in}{2.497478in}}%
\pgfpathlineto{\pgfqpoint{4.764628in}{2.498816in}}%
\pgfpathlineto{\pgfqpoint{4.782562in}{2.499340in}}%
\pgfpathlineto{\pgfqpoint{4.800496in}{2.499042in}}%
\pgfpathlineto{\pgfqpoint{4.818430in}{2.497910in}}%
\pgfpathlineto{\pgfqpoint{4.836364in}{2.495938in}}%
\pgfpathlineto{\pgfqpoint{4.854298in}{2.493120in}}%
\pgfpathlineto{\pgfqpoint{4.872231in}{2.489451in}}%
\pgfpathlineto{\pgfqpoint{4.890165in}{2.484927in}}%
\pgfpathlineto{\pgfqpoint{4.908099in}{2.479548in}}%
\pgfpathlineto{\pgfqpoint{4.926033in}{2.473314in}}%
\pgfpathlineto{\pgfqpoint{4.943967in}{2.466227in}}%
\pgfpathlineto{\pgfqpoint{4.961901in}{2.458289in}}%
\pgfpathlineto{\pgfqpoint{4.979835in}{2.449506in}}%
\pgfpathlineto{\pgfqpoint{4.997769in}{2.439883in}}%
\pgfpathlineto{\pgfqpoint{5.015702in}{2.429430in}}%
\pgfpathlineto{\pgfqpoint{5.033636in}{2.418155in}}%
\pgfpathlineto{\pgfqpoint{5.051570in}{2.406068in}}%
\pgfpathlineto{\pgfqpoint{5.069504in}{2.393183in}}%
\pgfpathlineto{\pgfqpoint{5.087438in}{2.379512in}}%
\pgfpathlineto{\pgfqpoint{5.105372in}{2.365071in}}%
\pgfpathlineto{\pgfqpoint{5.123306in}{2.349876in}}%
\pgfpathlineto{\pgfqpoint{5.141240in}{2.333945in}}%
\pgfpathlineto{\pgfqpoint{5.159174in}{2.317297in}}%
\pgfpathlineto{\pgfqpoint{5.177107in}{2.299952in}}%
\pgfpathlineto{\pgfqpoint{5.195041in}{2.281931in}}%
\pgfpathlineto{\pgfqpoint{5.212975in}{2.263257in}}%
\pgfpathlineto{\pgfqpoint{5.230909in}{2.243954in}}%
\pgfusepath{stroke}%
\end{pgfscope}%
\begin{pgfscope}%
\pgfpathrectangle{\pgfqpoint{3.286364in}{1.770000in}}{\pgfqpoint{2.113636in}{0.930000in}} %
\pgfusepath{clip}%
\pgfsetroundcap%
\pgfsetroundjoin%
\pgfsetlinewidth{0.200750pt}%
\definecolor{currentstroke}{rgb}{0.125490,0.290196,0.529412}%
\pgfsetstrokecolor{currentstroke}%
\pgfsetdash{}{0pt}%
\pgfpathmoveto{\pgfqpoint{3.455455in}{2.255378in}}%
\pgfpathlineto{\pgfqpoint{3.473388in}{2.229695in}}%
\pgfpathlineto{\pgfqpoint{3.491322in}{2.205998in}}%
\pgfpathlineto{\pgfqpoint{3.509256in}{2.184290in}}%
\pgfpathlineto{\pgfqpoint{3.527190in}{2.164560in}}%
\pgfpathlineto{\pgfqpoint{3.545124in}{2.146780in}}%
\pgfpathlineto{\pgfqpoint{3.563058in}{2.130902in}}%
\pgfpathlineto{\pgfqpoint{3.580992in}{2.116863in}}%
\pgfpathlineto{\pgfqpoint{3.598926in}{2.104582in}}%
\pgfpathlineto{\pgfqpoint{3.616860in}{2.093964in}}%
\pgfpathlineto{\pgfqpoint{3.634793in}{2.084911in}}%
\pgfpathlineto{\pgfqpoint{3.652727in}{2.077318in}}%
\pgfpathlineto{\pgfqpoint{3.670661in}{2.071082in}}%
\pgfpathlineto{\pgfqpoint{3.688595in}{2.066107in}}%
\pgfpathlineto{\pgfqpoint{3.706529in}{2.062301in}}%
\pgfpathlineto{\pgfqpoint{3.724463in}{2.059580in}}%
\pgfpathlineto{\pgfqpoint{3.742397in}{2.057869in}}%
\pgfpathlineto{\pgfqpoint{3.760331in}{2.057101in}}%
\pgfpathlineto{\pgfqpoint{3.778264in}{2.057215in}}%
\pgfpathlineto{\pgfqpoint{3.796198in}{2.058158in}}%
\pgfpathlineto{\pgfqpoint{3.814132in}{2.059882in}}%
\pgfpathlineto{\pgfqpoint{3.832066in}{2.062345in}}%
\pgfpathlineto{\pgfqpoint{3.850000in}{2.065507in}}%
\pgfpathlineto{\pgfqpoint{3.867934in}{2.069334in}}%
\pgfpathlineto{\pgfqpoint{3.885868in}{2.073795in}}%
\pgfpathlineto{\pgfqpoint{3.903802in}{2.078861in}}%
\pgfpathlineto{\pgfqpoint{3.921736in}{2.084506in}}%
\pgfpathlineto{\pgfqpoint{3.939669in}{2.090707in}}%
\pgfpathlineto{\pgfqpoint{3.957603in}{2.097443in}}%
\pgfpathlineto{\pgfqpoint{3.975537in}{2.104694in}}%
\pgfpathlineto{\pgfqpoint{3.993471in}{2.112444in}}%
\pgfpathlineto{\pgfqpoint{4.011405in}{2.120678in}}%
\pgfpathlineto{\pgfqpoint{4.029339in}{2.129384in}}%
\pgfpathlineto{\pgfqpoint{4.047273in}{2.138553in}}%
\pgfpathlineto{\pgfqpoint{4.065207in}{2.148177in}}%
\pgfpathlineto{\pgfqpoint{4.083140in}{2.158252in}}%
\pgfpathlineto{\pgfqpoint{4.101074in}{2.168776in}}%
\pgfpathlineto{\pgfqpoint{4.119008in}{2.179750in}}%
\pgfpathlineto{\pgfqpoint{4.136942in}{2.191176in}}%
\pgfpathlineto{\pgfqpoint{4.154876in}{2.203058in}}%
\pgfpathlineto{\pgfqpoint{4.172810in}{2.215399in}}%
\pgfpathlineto{\pgfqpoint{4.190744in}{2.228197in}}%
\pgfpathlineto{\pgfqpoint{4.208678in}{2.241448in}}%
\pgfpathlineto{\pgfqpoint{4.226612in}{2.255139in}}%
\pgfpathlineto{\pgfqpoint{4.244545in}{2.269247in}}%
\pgfpathlineto{\pgfqpoint{4.262479in}{2.283737in}}%
\pgfpathlineto{\pgfqpoint{4.280413in}{2.298561in}}%
\pgfpathlineto{\pgfqpoint{4.298347in}{2.313661in}}%
\pgfpathlineto{\pgfqpoint{4.316281in}{2.328968in}}%
\pgfpathlineto{\pgfqpoint{4.334215in}{2.344408in}}%
\pgfpathlineto{\pgfqpoint{4.352149in}{2.359901in}}%
\pgfpathlineto{\pgfqpoint{4.370083in}{2.375365in}}%
\pgfpathlineto{\pgfqpoint{4.388017in}{2.390721in}}%
\pgfpathlineto{\pgfqpoint{4.405950in}{2.405889in}}%
\pgfpathlineto{\pgfqpoint{4.423884in}{2.420793in}}%
\pgfpathlineto{\pgfqpoint{4.441818in}{2.435363in}}%
\pgfpathlineto{\pgfqpoint{4.459752in}{2.449529in}}%
\pgfpathlineto{\pgfqpoint{4.477686in}{2.463229in}}%
\pgfpathlineto{\pgfqpoint{4.495620in}{2.476403in}}%
\pgfpathlineto{\pgfqpoint{4.513554in}{2.488994in}}%
\pgfpathlineto{\pgfqpoint{4.531488in}{2.500953in}}%
\pgfpathlineto{\pgfqpoint{4.549421in}{2.512231in}}%
\pgfpathlineto{\pgfqpoint{4.567355in}{2.522785in}}%
\pgfpathlineto{\pgfqpoint{4.585289in}{2.532575in}}%
\pgfpathlineto{\pgfqpoint{4.603223in}{2.541567in}}%
\pgfpathlineto{\pgfqpoint{4.621157in}{2.549730in}}%
\pgfpathlineto{\pgfqpoint{4.639091in}{2.557036in}}%
\pgfpathlineto{\pgfqpoint{4.657025in}{2.563463in}}%
\pgfpathlineto{\pgfqpoint{4.674959in}{2.568996in}}%
\pgfpathlineto{\pgfqpoint{4.692893in}{2.573622in}}%
\pgfpathlineto{\pgfqpoint{4.710826in}{2.577336in}}%
\pgfpathlineto{\pgfqpoint{4.728760in}{2.580138in}}%
\pgfpathlineto{\pgfqpoint{4.746694in}{2.582036in}}%
\pgfpathlineto{\pgfqpoint{4.764628in}{2.583045in}}%
\pgfpathlineto{\pgfqpoint{4.782562in}{2.583189in}}%
\pgfpathlineto{\pgfqpoint{4.800496in}{2.582500in}}%
\pgfpathlineto{\pgfqpoint{4.818430in}{2.581019in}}%
\pgfpathlineto{\pgfqpoint{4.836364in}{2.578797in}}%
\pgfpathlineto{\pgfqpoint{4.854298in}{2.575893in}}%
\pgfpathlineto{\pgfqpoint{4.872231in}{2.572375in}}%
\pgfpathlineto{\pgfqpoint{4.890165in}{2.568316in}}%
\pgfpathlineto{\pgfqpoint{4.908099in}{2.563793in}}%
\pgfpathlineto{\pgfqpoint{4.926033in}{2.558882in}}%
\pgfpathlineto{\pgfqpoint{4.943967in}{2.553655in}}%
\pgfpathlineto{\pgfqpoint{4.961901in}{2.548177in}}%
\pgfpathlineto{\pgfqpoint{4.979835in}{2.542500in}}%
\pgfpathlineto{\pgfqpoint{4.997769in}{2.536665in}}%
\pgfpathlineto{\pgfqpoint{5.015702in}{2.530698in}}%
\pgfpathlineto{\pgfqpoint{5.033636in}{2.524614in}}%
\pgfpathlineto{\pgfqpoint{5.051570in}{2.518418in}}%
\pgfpathlineto{\pgfqpoint{5.069504in}{2.512104in}}%
\pgfpathlineto{\pgfqpoint{5.087438in}{2.505665in}}%
\pgfpathlineto{\pgfqpoint{5.105372in}{2.499088in}}%
\pgfpathlineto{\pgfqpoint{5.123306in}{2.492362in}}%
\pgfpathlineto{\pgfqpoint{5.141240in}{2.485473in}}%
\pgfpathlineto{\pgfqpoint{5.159174in}{2.478410in}}%
\pgfpathlineto{\pgfqpoint{5.177107in}{2.471165in}}%
\pgfpathlineto{\pgfqpoint{5.195041in}{2.463730in}}%
\pgfpathlineto{\pgfqpoint{5.212975in}{2.456102in}}%
\pgfpathlineto{\pgfqpoint{5.230909in}{2.448278in}}%
\pgfusepath{stroke}%
\end{pgfscope}%
\begin{pgfscope}%
\pgfpathrectangle{\pgfqpoint{3.286364in}{1.770000in}}{\pgfqpoint{2.113636in}{0.930000in}} %
\pgfusepath{clip}%
\pgfsetroundcap%
\pgfsetroundjoin%
\pgfsetlinewidth{0.200750pt}%
\definecolor{currentstroke}{rgb}{0.125490,0.290196,0.529412}%
\pgfsetstrokecolor{currentstroke}%
\pgfsetdash{}{0pt}%
\pgfpathmoveto{\pgfqpoint{3.455455in}{2.010124in}}%
\pgfpathlineto{\pgfqpoint{3.473388in}{1.998548in}}%
\pgfpathlineto{\pgfqpoint{3.491322in}{1.987312in}}%
\pgfpathlineto{\pgfqpoint{3.509256in}{1.976411in}}%
\pgfpathlineto{\pgfqpoint{3.527190in}{1.965854in}}%
\pgfpathlineto{\pgfqpoint{3.545124in}{1.955662in}}%
\pgfpathlineto{\pgfqpoint{3.563058in}{1.945874in}}%
\pgfpathlineto{\pgfqpoint{3.580992in}{1.936540in}}%
\pgfpathlineto{\pgfqpoint{3.598926in}{1.927727in}}%
\pgfpathlineto{\pgfqpoint{3.616860in}{1.919509in}}%
\pgfpathlineto{\pgfqpoint{3.634793in}{1.911965in}}%
\pgfpathlineto{\pgfqpoint{3.652727in}{1.905173in}}%
\pgfpathlineto{\pgfqpoint{3.670661in}{1.899209in}}%
\pgfpathlineto{\pgfqpoint{3.688595in}{1.894140in}}%
\pgfpathlineto{\pgfqpoint{3.706529in}{1.890023in}}%
\pgfpathlineto{\pgfqpoint{3.724463in}{1.886906in}}%
\pgfpathlineto{\pgfqpoint{3.742397in}{1.884823in}}%
\pgfpathlineto{\pgfqpoint{3.760331in}{1.883802in}}%
\pgfpathlineto{\pgfqpoint{3.778264in}{1.883856in}}%
\pgfpathlineto{\pgfqpoint{3.796198in}{1.884994in}}%
\pgfpathlineto{\pgfqpoint{3.814132in}{1.887211in}}%
\pgfpathlineto{\pgfqpoint{3.832066in}{1.890500in}}%
\pgfpathlineto{\pgfqpoint{3.850000in}{1.894843in}}%
\pgfpathlineto{\pgfqpoint{3.867934in}{1.900218in}}%
\pgfpathlineto{\pgfqpoint{3.885868in}{1.906596in}}%
\pgfpathlineto{\pgfqpoint{3.903802in}{1.913946in}}%
\pgfpathlineto{\pgfqpoint{3.921736in}{1.922229in}}%
\pgfpathlineto{\pgfqpoint{3.939669in}{1.931403in}}%
\pgfpathlineto{\pgfqpoint{3.957603in}{1.941422in}}%
\pgfpathlineto{\pgfqpoint{3.975537in}{1.952236in}}%
\pgfpathlineto{\pgfqpoint{3.993471in}{1.963790in}}%
\pgfpathlineto{\pgfqpoint{4.011405in}{1.976028in}}%
\pgfpathlineto{\pgfqpoint{4.029339in}{1.988887in}}%
\pgfpathlineto{\pgfqpoint{4.047273in}{2.002302in}}%
\pgfpathlineto{\pgfqpoint{4.065207in}{2.016204in}}%
\pgfpathlineto{\pgfqpoint{4.083140in}{2.030520in}}%
\pgfpathlineto{\pgfqpoint{4.101074in}{2.045173in}}%
\pgfpathlineto{\pgfqpoint{4.119008in}{2.060085in}}%
\pgfpathlineto{\pgfqpoint{4.136942in}{2.075173in}}%
\pgfpathlineto{\pgfqpoint{4.154876in}{2.090354in}}%
\pgfpathlineto{\pgfqpoint{4.172810in}{2.105545in}}%
\pgfpathlineto{\pgfqpoint{4.190744in}{2.120665in}}%
\pgfpathlineto{\pgfqpoint{4.208678in}{2.135638in}}%
\pgfpathlineto{\pgfqpoint{4.226612in}{2.150397in}}%
\pgfpathlineto{\pgfqpoint{4.244545in}{2.164883in}}%
\pgfpathlineto{\pgfqpoint{4.262479in}{2.179052in}}%
\pgfpathlineto{\pgfqpoint{4.280413in}{2.192870in}}%
\pgfpathlineto{\pgfqpoint{4.298347in}{2.206317in}}%
\pgfpathlineto{\pgfqpoint{4.316281in}{2.219382in}}%
\pgfpathlineto{\pgfqpoint{4.334215in}{2.232061in}}%
\pgfpathlineto{\pgfqpoint{4.352149in}{2.244358in}}%
\pgfpathlineto{\pgfqpoint{4.370083in}{2.256276in}}%
\pgfpathlineto{\pgfqpoint{4.388017in}{2.267822in}}%
\pgfpathlineto{\pgfqpoint{4.405950in}{2.279001in}}%
\pgfpathlineto{\pgfqpoint{4.423884in}{2.289816in}}%
\pgfpathlineto{\pgfqpoint{4.441818in}{2.300267in}}%
\pgfpathlineto{\pgfqpoint{4.459752in}{2.310354in}}%
\pgfpathlineto{\pgfqpoint{4.477686in}{2.320072in}}%
\pgfpathlineto{\pgfqpoint{4.495620in}{2.329416in}}%
\pgfpathlineto{\pgfqpoint{4.513554in}{2.338377in}}%
\pgfpathlineto{\pgfqpoint{4.531488in}{2.346945in}}%
\pgfpathlineto{\pgfqpoint{4.549421in}{2.355108in}}%
\pgfpathlineto{\pgfqpoint{4.567355in}{2.362853in}}%
\pgfpathlineto{\pgfqpoint{4.585289in}{2.370163in}}%
\pgfpathlineto{\pgfqpoint{4.603223in}{2.377022in}}%
\pgfpathlineto{\pgfqpoint{4.621157in}{2.383410in}}%
\pgfpathlineto{\pgfqpoint{4.639091in}{2.389307in}}%
\pgfpathlineto{\pgfqpoint{4.657025in}{2.394690in}}%
\pgfpathlineto{\pgfqpoint{4.674959in}{2.399535in}}%
\pgfpathlineto{\pgfqpoint{4.692893in}{2.403812in}}%
\pgfpathlineto{\pgfqpoint{4.710826in}{2.407492in}}%
\pgfpathlineto{\pgfqpoint{4.728760in}{2.410540in}}%
\pgfpathlineto{\pgfqpoint{4.746694in}{2.412919in}}%
\pgfpathlineto{\pgfqpoint{4.764628in}{2.414586in}}%
\pgfpathlineto{\pgfqpoint{4.782562in}{2.415492in}}%
\pgfpathlineto{\pgfqpoint{4.800496in}{2.415583in}}%
\pgfpathlineto{\pgfqpoint{4.818430in}{2.414802in}}%
\pgfpathlineto{\pgfqpoint{4.836364in}{2.413080in}}%
\pgfpathlineto{\pgfqpoint{4.854298in}{2.410347in}}%
\pgfpathlineto{\pgfqpoint{4.872231in}{2.406526in}}%
\pgfpathlineto{\pgfqpoint{4.890165in}{2.401538in}}%
\pgfpathlineto{\pgfqpoint{4.908099in}{2.395304in}}%
\pgfpathlineto{\pgfqpoint{4.926033in}{2.387746in}}%
\pgfpathlineto{\pgfqpoint{4.943967in}{2.378798in}}%
\pgfpathlineto{\pgfqpoint{4.961901in}{2.368401in}}%
\pgfpathlineto{\pgfqpoint{4.979835in}{2.356511in}}%
\pgfpathlineto{\pgfqpoint{4.997769in}{2.343102in}}%
\pgfpathlineto{\pgfqpoint{5.015702in}{2.328162in}}%
\pgfpathlineto{\pgfqpoint{5.033636in}{2.311695in}}%
\pgfpathlineto{\pgfqpoint{5.051570in}{2.293719in}}%
\pgfpathlineto{\pgfqpoint{5.069504in}{2.274261in}}%
\pgfpathlineto{\pgfqpoint{5.087438in}{2.253359in}}%
\pgfpathlineto{\pgfqpoint{5.105372in}{2.231054in}}%
\pgfpathlineto{\pgfqpoint{5.123306in}{2.207391in}}%
\pgfpathlineto{\pgfqpoint{5.141240in}{2.182418in}}%
\pgfpathlineto{\pgfqpoint{5.159174in}{2.156184in}}%
\pgfpathlineto{\pgfqpoint{5.177107in}{2.128738in}}%
\pgfpathlineto{\pgfqpoint{5.195041in}{2.100131in}}%
\pgfpathlineto{\pgfqpoint{5.212975in}{2.070412in}}%
\pgfpathlineto{\pgfqpoint{5.230909in}{2.039630in}}%
\pgfusepath{stroke}%
\end{pgfscope}%
\begin{pgfscope}%
\pgfpathrectangle{\pgfqpoint{3.286364in}{1.770000in}}{\pgfqpoint{2.113636in}{0.930000in}} %
\pgfusepath{clip}%
\pgfsetbuttcap%
\pgfsetbeveljoin%
\definecolor{currentfill}{rgb}{0.298039,0.447059,0.690196}%
\pgfsetfillcolor{currentfill}%
\pgfsetlinewidth{0.000000pt}%
\definecolor{currentstroke}{rgb}{0.000000,0.000000,0.000000}%
\pgfsetstrokecolor{currentstroke}%
\pgfsetdash{}{0pt}%
\pgfsys@defobject{currentmarker}{\pgfqpoint{-0.036986in}{-0.031462in}}{\pgfqpoint{0.036986in}{0.038889in}}{%
\pgfpathmoveto{\pgfqpoint{0.000000in}{0.038889in}}%
\pgfpathlineto{\pgfqpoint{-0.008731in}{0.012017in}}%
\pgfpathlineto{\pgfqpoint{-0.036986in}{0.012017in}}%
\pgfpathlineto{\pgfqpoint{-0.014127in}{-0.004590in}}%
\pgfpathlineto{\pgfqpoint{-0.022858in}{-0.031462in}}%
\pgfpathlineto{\pgfqpoint{-0.000000in}{-0.014854in}}%
\pgfpathlineto{\pgfqpoint{0.022858in}{-0.031462in}}%
\pgfpathlineto{\pgfqpoint{0.014127in}{-0.004590in}}%
\pgfpathlineto{\pgfqpoint{0.036986in}{0.012017in}}%
\pgfpathlineto{\pgfqpoint{0.008731in}{0.012017in}}%
\pgfpathclose%
\pgfusepath{fill}%
}%
\begin{pgfscope}%
\pgfsys@transformshift{3.886636in}{2.343500in}%
\pgfsys@useobject{currentmarker}{}%
\end{pgfscope}%
\begin{pgfscope}%
\pgfsys@transformshift{4.546091in}{2.219500in}%
\pgfsys@useobject{currentmarker}{}%
\end{pgfscope}%
\begin{pgfscope}%
\pgfsys@transformshift{4.317818in}{2.328000in}%
\pgfsys@useobject{currentmarker}{}%
\end{pgfscope}%
\begin{pgfscope}%
\pgfsys@transformshift{3.489273in}{2.529500in}%
\pgfsys@useobject{currentmarker}{}%
\end{pgfscope}%
\begin{pgfscope}%
\pgfsys@transformshift{3.497727in}{2.204000in}%
\pgfsys@useobject{currentmarker}{}%
\end{pgfscope}%
\begin{pgfscope}%
\pgfsys@transformshift{5.070273in}{2.591500in}%
\pgfsys@useobject{currentmarker}{}%
\end{pgfscope}%
\begin{pgfscope}%
\pgfsys@transformshift{5.061818in}{2.204000in}%
\pgfsys@useobject{currentmarker}{}%
\end{pgfscope}%
\begin{pgfscope}%
\pgfsys@transformshift{5.230909in}{2.653500in}%
\pgfsys@useobject{currentmarker}{}%
\end{pgfscope}%
\begin{pgfscope}%
\pgfsys@transformshift{3.523091in}{1.940500in}%
\pgfsys@useobject{currentmarker}{}%
\end{pgfscope}%
\begin{pgfscope}%
\pgfsys@transformshift{4.174091in}{2.235000in}%
\pgfsys@useobject{currentmarker}{}%
\end{pgfscope}%
\begin{pgfscope}%
\pgfsys@transformshift{4.918091in}{2.188500in}%
\pgfsys@useobject{currentmarker}{}%
\end{pgfscope}%
\begin{pgfscope}%
\pgfsys@transformshift{5.044909in}{2.204000in}%
\pgfsys@useobject{currentmarker}{}%
\end{pgfscope}%
\begin{pgfscope}%
\pgfsys@transformshift{5.205545in}{2.250500in}%
\pgfsys@useobject{currentmarker}{}%
\end{pgfscope}%
\begin{pgfscope}%
\pgfsys@transformshift{3.455455in}{1.878500in}%
\pgfsys@useobject{currentmarker}{}%
\end{pgfscope}%
\begin{pgfscope}%
\pgfsys@transformshift{4.038818in}{2.421000in}%
\pgfsys@useobject{currentmarker}{}%
\end{pgfscope}%
\begin{pgfscope}%
\pgfsys@transformshift{5.061818in}{2.374500in}%
\pgfsys@useobject{currentmarker}{}%
\end{pgfscope}%
\begin{pgfscope}%
\pgfsys@transformshift{4.157182in}{2.126500in}%
\pgfsys@useobject{currentmarker}{}%
\end{pgfscope}%
\begin{pgfscope}%
\pgfsys@transformshift{3.709091in}{2.219500in}%
\pgfsys@useobject{currentmarker}{}%
\end{pgfscope}%
\begin{pgfscope}%
\pgfsys@transformshift{4.985727in}{2.002500in}%
\pgfsys@useobject{currentmarker}{}%
\end{pgfscope}%
\begin{pgfscope}%
\pgfsys@transformshift{5.078727in}{2.142000in}%
\pgfsys@useobject{currentmarker}{}%
\end{pgfscope}%
\begin{pgfscope}%
\pgfsys@transformshift{3.742909in}{2.219500in}%
\pgfsys@useobject{currentmarker}{}%
\end{pgfscope}%
\begin{pgfscope}%
\pgfsys@transformshift{5.044909in}{2.436500in}%
\pgfsys@useobject{currentmarker}{}%
\end{pgfscope}%
\begin{pgfscope}%
\pgfsys@transformshift{3.692182in}{2.266000in}%
\pgfsys@useobject{currentmarker}{}%
\end{pgfscope}%
\begin{pgfscope}%
\pgfsys@transformshift{3.675273in}{2.049000in}%
\pgfsys@useobject{currentmarker}{}%
\end{pgfscope}%
\begin{pgfscope}%
\pgfsys@transformshift{5.061818in}{2.312500in}%
\pgfsys@useobject{currentmarker}{}%
\end{pgfscope}%
\begin{pgfscope}%
\pgfsys@transformshift{4.825091in}{2.514000in}%
\pgfsys@useobject{currentmarker}{}%
\end{pgfscope}%
\begin{pgfscope}%
\pgfsys@transformshift{3.497727in}{2.064500in}%
\pgfsys@useobject{currentmarker}{}%
\end{pgfscope}%
\begin{pgfscope}%
\pgfsys@transformshift{4.233273in}{1.956000in}%
\pgfsys@useobject{currentmarker}{}%
\end{pgfscope}%
\begin{pgfscope}%
\pgfsys@transformshift{3.641455in}{2.018000in}%
\pgfsys@useobject{currentmarker}{}%
\end{pgfscope}%
\begin{pgfscope}%
\pgfsys@transformshift{3.675273in}{2.529500in}%
\pgfsys@useobject{currentmarker}{}%
\end{pgfscope}%
\end{pgfscope}%
\begin{pgfscope}%
\pgfsetrectcap%
\pgfsetmiterjoin%
\pgfsetlinewidth{0.000000pt}%
\definecolor{currentstroke}{rgb}{1.000000,1.000000,1.000000}%
\pgfsetstrokecolor{currentstroke}%
\pgfsetdash{}{0pt}%
\pgfpathmoveto{\pgfqpoint{3.286364in}{1.770000in}}%
\pgfpathlineto{\pgfqpoint{5.400000in}{1.770000in}}%
\pgfusepath{}%
\end{pgfscope}%
\begin{pgfscope}%
\pgfsetrectcap%
\pgfsetmiterjoin%
\pgfsetlinewidth{0.000000pt}%
\definecolor{currentstroke}{rgb}{1.000000,1.000000,1.000000}%
\pgfsetstrokecolor{currentstroke}%
\pgfsetdash{}{0pt}%
\pgfpathmoveto{\pgfqpoint{3.286364in}{1.770000in}}%
\pgfpathlineto{\pgfqpoint{3.286364in}{2.700000in}}%
\pgfusepath{}%
\end{pgfscope}%
\begin{pgfscope}%
\pgfsetrectcap%
\pgfsetmiterjoin%
\pgfsetlinewidth{0.000000pt}%
\definecolor{currentstroke}{rgb}{1.000000,1.000000,1.000000}%
\pgfsetstrokecolor{currentstroke}%
\pgfsetdash{}{0pt}%
\pgfpathmoveto{\pgfqpoint{3.286364in}{2.700000in}}%
\pgfpathlineto{\pgfqpoint{5.400000in}{2.700000in}}%
\pgfusepath{}%
\end{pgfscope}%
\begin{pgfscope}%
\pgfsetrectcap%
\pgfsetmiterjoin%
\pgfsetlinewidth{0.000000pt}%
\definecolor{currentstroke}{rgb}{1.000000,1.000000,1.000000}%
\pgfsetstrokecolor{currentstroke}%
\pgfsetdash{}{0pt}%
\pgfpathmoveto{\pgfqpoint{5.400000in}{1.770000in}}%
\pgfpathlineto{\pgfqpoint{5.400000in}{2.700000in}}%
\pgfusepath{}%
\end{pgfscope}%
\begin{pgfscope}%
\pgfsetbuttcap%
\pgfsetmiterjoin%
\definecolor{currentfill}{rgb}{0.917647,0.917647,0.949020}%
\pgfsetfillcolor{currentfill}%
\pgfsetlinewidth{0.000000pt}%
\definecolor{currentstroke}{rgb}{0.000000,0.000000,0.000000}%
\pgfsetstrokecolor{currentstroke}%
\pgfsetstrokeopacity{0.000000}%
\pgfsetdash{}{0pt}%
\pgfpathmoveto{\pgfqpoint{0.750000in}{0.375000in}}%
\pgfpathlineto{\pgfqpoint{2.863636in}{0.375000in}}%
\pgfpathlineto{\pgfqpoint{2.863636in}{1.305000in}}%
\pgfpathlineto{\pgfqpoint{0.750000in}{1.305000in}}%
\pgfpathclose%
\pgfusepath{fill}%
\end{pgfscope}%
\begin{pgfscope}%
\pgfpathrectangle{\pgfqpoint{0.750000in}{0.375000in}}{\pgfqpoint{2.113636in}{0.930000in}} %
\pgfusepath{clip}%
\pgfsetroundcap%
\pgfsetroundjoin%
\pgfsetlinewidth{0.803000pt}%
\definecolor{currentstroke}{rgb}{1.000000,1.000000,1.000000}%
\pgfsetstrokecolor{currentstroke}%
\pgfsetdash{}{0pt}%
\pgfpathmoveto{\pgfqpoint{1.091912in}{0.375000in}}%
\pgfpathlineto{\pgfqpoint{1.091912in}{1.305000in}}%
\pgfusepath{stroke}%
\end{pgfscope}%
\begin{pgfscope}%
\pgfsetbuttcap%
\pgfsetroundjoin%
\definecolor{currentfill}{rgb}{0.150000,0.150000,0.150000}%
\pgfsetfillcolor{currentfill}%
\pgfsetlinewidth{0.803000pt}%
\definecolor{currentstroke}{rgb}{0.150000,0.150000,0.150000}%
\pgfsetstrokecolor{currentstroke}%
\pgfsetdash{}{0pt}%
\pgfsys@defobject{currentmarker}{\pgfqpoint{0.000000in}{0.000000in}}{\pgfqpoint{0.000000in}{0.000000in}}{%
\pgfpathmoveto{\pgfqpoint{0.000000in}{0.000000in}}%
\pgfpathlineto{\pgfqpoint{0.000000in}{0.000000in}}%
\pgfusepath{stroke,fill}%
}%
\begin{pgfscope}%
\pgfsys@transformshift{1.091912in}{0.375000in}%
\pgfsys@useobject{currentmarker}{}%
\end{pgfscope}%
\end{pgfscope}%
\begin{pgfscope}%
\pgfsetbuttcap%
\pgfsetroundjoin%
\definecolor{currentfill}{rgb}{0.150000,0.150000,0.150000}%
\pgfsetfillcolor{currentfill}%
\pgfsetlinewidth{0.803000pt}%
\definecolor{currentstroke}{rgb}{0.150000,0.150000,0.150000}%
\pgfsetstrokecolor{currentstroke}%
\pgfsetdash{}{0pt}%
\pgfsys@defobject{currentmarker}{\pgfqpoint{0.000000in}{0.000000in}}{\pgfqpoint{0.000000in}{0.000000in}}{%
\pgfpathmoveto{\pgfqpoint{0.000000in}{0.000000in}}%
\pgfpathlineto{\pgfqpoint{0.000000in}{0.000000in}}%
\pgfusepath{stroke,fill}%
}%
\begin{pgfscope}%
\pgfsys@transformshift{1.091912in}{1.305000in}%
\pgfsys@useobject{currentmarker}{}%
\end{pgfscope}%
\end{pgfscope}%
\begin{pgfscope}%
\definecolor{textcolor}{rgb}{0.150000,0.150000,0.150000}%
\pgfsetstrokecolor{textcolor}%
\pgfsetfillcolor{textcolor}%
\pgftext[x=1.091912in,y=0.297222in,,top]{\color{textcolor}\sffamily\fontsize{8.000000}{9.600000}\selectfont 6}%
\end{pgfscope}%
\begin{pgfscope}%
\pgfpathrectangle{\pgfqpoint{0.750000in}{0.375000in}}{\pgfqpoint{2.113636in}{0.930000in}} %
\pgfusepath{clip}%
\pgfsetroundcap%
\pgfsetroundjoin%
\pgfsetlinewidth{0.803000pt}%
\definecolor{currentstroke}{rgb}{1.000000,1.000000,1.000000}%
\pgfsetstrokecolor{currentstroke}%
\pgfsetdash{}{0pt}%
\pgfpathmoveto{\pgfqpoint{1.480448in}{0.375000in}}%
\pgfpathlineto{\pgfqpoint{1.480448in}{1.305000in}}%
\pgfusepath{stroke}%
\end{pgfscope}%
\begin{pgfscope}%
\pgfsetbuttcap%
\pgfsetroundjoin%
\definecolor{currentfill}{rgb}{0.150000,0.150000,0.150000}%
\pgfsetfillcolor{currentfill}%
\pgfsetlinewidth{0.803000pt}%
\definecolor{currentstroke}{rgb}{0.150000,0.150000,0.150000}%
\pgfsetstrokecolor{currentstroke}%
\pgfsetdash{}{0pt}%
\pgfsys@defobject{currentmarker}{\pgfqpoint{0.000000in}{0.000000in}}{\pgfqpoint{0.000000in}{0.000000in}}{%
\pgfpathmoveto{\pgfqpoint{0.000000in}{0.000000in}}%
\pgfpathlineto{\pgfqpoint{0.000000in}{0.000000in}}%
\pgfusepath{stroke,fill}%
}%
\begin{pgfscope}%
\pgfsys@transformshift{1.480448in}{0.375000in}%
\pgfsys@useobject{currentmarker}{}%
\end{pgfscope}%
\end{pgfscope}%
\begin{pgfscope}%
\pgfsetbuttcap%
\pgfsetroundjoin%
\definecolor{currentfill}{rgb}{0.150000,0.150000,0.150000}%
\pgfsetfillcolor{currentfill}%
\pgfsetlinewidth{0.803000pt}%
\definecolor{currentstroke}{rgb}{0.150000,0.150000,0.150000}%
\pgfsetstrokecolor{currentstroke}%
\pgfsetdash{}{0pt}%
\pgfsys@defobject{currentmarker}{\pgfqpoint{0.000000in}{0.000000in}}{\pgfqpoint{0.000000in}{0.000000in}}{%
\pgfpathmoveto{\pgfqpoint{0.000000in}{0.000000in}}%
\pgfpathlineto{\pgfqpoint{0.000000in}{0.000000in}}%
\pgfusepath{stroke,fill}%
}%
\begin{pgfscope}%
\pgfsys@transformshift{1.480448in}{1.305000in}%
\pgfsys@useobject{currentmarker}{}%
\end{pgfscope}%
\end{pgfscope}%
\begin{pgfscope}%
\definecolor{textcolor}{rgb}{0.150000,0.150000,0.150000}%
\pgfsetstrokecolor{textcolor}%
\pgfsetfillcolor{textcolor}%
\pgftext[x=1.480448in,y=0.297222in,,top]{\color{textcolor}\sffamily\fontsize{8.000000}{9.600000}\selectfont 7}%
\end{pgfscope}%
\begin{pgfscope}%
\pgfpathrectangle{\pgfqpoint{0.750000in}{0.375000in}}{\pgfqpoint{2.113636in}{0.930000in}} %
\pgfusepath{clip}%
\pgfsetroundcap%
\pgfsetroundjoin%
\pgfsetlinewidth{0.803000pt}%
\definecolor{currentstroke}{rgb}{1.000000,1.000000,1.000000}%
\pgfsetstrokecolor{currentstroke}%
\pgfsetdash{}{0pt}%
\pgfpathmoveto{\pgfqpoint{1.868984in}{0.375000in}}%
\pgfpathlineto{\pgfqpoint{1.868984in}{1.305000in}}%
\pgfusepath{stroke}%
\end{pgfscope}%
\begin{pgfscope}%
\pgfsetbuttcap%
\pgfsetroundjoin%
\definecolor{currentfill}{rgb}{0.150000,0.150000,0.150000}%
\pgfsetfillcolor{currentfill}%
\pgfsetlinewidth{0.803000pt}%
\definecolor{currentstroke}{rgb}{0.150000,0.150000,0.150000}%
\pgfsetstrokecolor{currentstroke}%
\pgfsetdash{}{0pt}%
\pgfsys@defobject{currentmarker}{\pgfqpoint{0.000000in}{0.000000in}}{\pgfqpoint{0.000000in}{0.000000in}}{%
\pgfpathmoveto{\pgfqpoint{0.000000in}{0.000000in}}%
\pgfpathlineto{\pgfqpoint{0.000000in}{0.000000in}}%
\pgfusepath{stroke,fill}%
}%
\begin{pgfscope}%
\pgfsys@transformshift{1.868984in}{0.375000in}%
\pgfsys@useobject{currentmarker}{}%
\end{pgfscope}%
\end{pgfscope}%
\begin{pgfscope}%
\pgfsetbuttcap%
\pgfsetroundjoin%
\definecolor{currentfill}{rgb}{0.150000,0.150000,0.150000}%
\pgfsetfillcolor{currentfill}%
\pgfsetlinewidth{0.803000pt}%
\definecolor{currentstroke}{rgb}{0.150000,0.150000,0.150000}%
\pgfsetstrokecolor{currentstroke}%
\pgfsetdash{}{0pt}%
\pgfsys@defobject{currentmarker}{\pgfqpoint{0.000000in}{0.000000in}}{\pgfqpoint{0.000000in}{0.000000in}}{%
\pgfpathmoveto{\pgfqpoint{0.000000in}{0.000000in}}%
\pgfpathlineto{\pgfqpoint{0.000000in}{0.000000in}}%
\pgfusepath{stroke,fill}%
}%
\begin{pgfscope}%
\pgfsys@transformshift{1.868984in}{1.305000in}%
\pgfsys@useobject{currentmarker}{}%
\end{pgfscope}%
\end{pgfscope}%
\begin{pgfscope}%
\definecolor{textcolor}{rgb}{0.150000,0.150000,0.150000}%
\pgfsetstrokecolor{textcolor}%
\pgfsetfillcolor{textcolor}%
\pgftext[x=1.868984in,y=0.297222in,,top]{\color{textcolor}\sffamily\fontsize{8.000000}{9.600000}\selectfont 8}%
\end{pgfscope}%
\begin{pgfscope}%
\pgfpathrectangle{\pgfqpoint{0.750000in}{0.375000in}}{\pgfqpoint{2.113636in}{0.930000in}} %
\pgfusepath{clip}%
\pgfsetroundcap%
\pgfsetroundjoin%
\pgfsetlinewidth{0.803000pt}%
\definecolor{currentstroke}{rgb}{1.000000,1.000000,1.000000}%
\pgfsetstrokecolor{currentstroke}%
\pgfsetdash{}{0pt}%
\pgfpathmoveto{\pgfqpoint{2.257520in}{0.375000in}}%
\pgfpathlineto{\pgfqpoint{2.257520in}{1.305000in}}%
\pgfusepath{stroke}%
\end{pgfscope}%
\begin{pgfscope}%
\pgfsetbuttcap%
\pgfsetroundjoin%
\definecolor{currentfill}{rgb}{0.150000,0.150000,0.150000}%
\pgfsetfillcolor{currentfill}%
\pgfsetlinewidth{0.803000pt}%
\definecolor{currentstroke}{rgb}{0.150000,0.150000,0.150000}%
\pgfsetstrokecolor{currentstroke}%
\pgfsetdash{}{0pt}%
\pgfsys@defobject{currentmarker}{\pgfqpoint{0.000000in}{0.000000in}}{\pgfqpoint{0.000000in}{0.000000in}}{%
\pgfpathmoveto{\pgfqpoint{0.000000in}{0.000000in}}%
\pgfpathlineto{\pgfqpoint{0.000000in}{0.000000in}}%
\pgfusepath{stroke,fill}%
}%
\begin{pgfscope}%
\pgfsys@transformshift{2.257520in}{0.375000in}%
\pgfsys@useobject{currentmarker}{}%
\end{pgfscope}%
\end{pgfscope}%
\begin{pgfscope}%
\pgfsetbuttcap%
\pgfsetroundjoin%
\definecolor{currentfill}{rgb}{0.150000,0.150000,0.150000}%
\pgfsetfillcolor{currentfill}%
\pgfsetlinewidth{0.803000pt}%
\definecolor{currentstroke}{rgb}{0.150000,0.150000,0.150000}%
\pgfsetstrokecolor{currentstroke}%
\pgfsetdash{}{0pt}%
\pgfsys@defobject{currentmarker}{\pgfqpoint{0.000000in}{0.000000in}}{\pgfqpoint{0.000000in}{0.000000in}}{%
\pgfpathmoveto{\pgfqpoint{0.000000in}{0.000000in}}%
\pgfpathlineto{\pgfqpoint{0.000000in}{0.000000in}}%
\pgfusepath{stroke,fill}%
}%
\begin{pgfscope}%
\pgfsys@transformshift{2.257520in}{1.305000in}%
\pgfsys@useobject{currentmarker}{}%
\end{pgfscope}%
\end{pgfscope}%
\begin{pgfscope}%
\definecolor{textcolor}{rgb}{0.150000,0.150000,0.150000}%
\pgfsetstrokecolor{textcolor}%
\pgfsetfillcolor{textcolor}%
\pgftext[x=2.257520in,y=0.297222in,,top]{\color{textcolor}\sffamily\fontsize{8.000000}{9.600000}\selectfont 9}%
\end{pgfscope}%
\begin{pgfscope}%
\pgfpathrectangle{\pgfqpoint{0.750000in}{0.375000in}}{\pgfqpoint{2.113636in}{0.930000in}} %
\pgfusepath{clip}%
\pgfsetroundcap%
\pgfsetroundjoin%
\pgfsetlinewidth{0.803000pt}%
\definecolor{currentstroke}{rgb}{1.000000,1.000000,1.000000}%
\pgfsetstrokecolor{currentstroke}%
\pgfsetdash{}{0pt}%
\pgfpathmoveto{\pgfqpoint{2.646056in}{0.375000in}}%
\pgfpathlineto{\pgfqpoint{2.646056in}{1.305000in}}%
\pgfusepath{stroke}%
\end{pgfscope}%
\begin{pgfscope}%
\pgfsetbuttcap%
\pgfsetroundjoin%
\definecolor{currentfill}{rgb}{0.150000,0.150000,0.150000}%
\pgfsetfillcolor{currentfill}%
\pgfsetlinewidth{0.803000pt}%
\definecolor{currentstroke}{rgb}{0.150000,0.150000,0.150000}%
\pgfsetstrokecolor{currentstroke}%
\pgfsetdash{}{0pt}%
\pgfsys@defobject{currentmarker}{\pgfqpoint{0.000000in}{0.000000in}}{\pgfqpoint{0.000000in}{0.000000in}}{%
\pgfpathmoveto{\pgfqpoint{0.000000in}{0.000000in}}%
\pgfpathlineto{\pgfqpoint{0.000000in}{0.000000in}}%
\pgfusepath{stroke,fill}%
}%
\begin{pgfscope}%
\pgfsys@transformshift{2.646056in}{0.375000in}%
\pgfsys@useobject{currentmarker}{}%
\end{pgfscope}%
\end{pgfscope}%
\begin{pgfscope}%
\pgfsetbuttcap%
\pgfsetroundjoin%
\definecolor{currentfill}{rgb}{0.150000,0.150000,0.150000}%
\pgfsetfillcolor{currentfill}%
\pgfsetlinewidth{0.803000pt}%
\definecolor{currentstroke}{rgb}{0.150000,0.150000,0.150000}%
\pgfsetstrokecolor{currentstroke}%
\pgfsetdash{}{0pt}%
\pgfsys@defobject{currentmarker}{\pgfqpoint{0.000000in}{0.000000in}}{\pgfqpoint{0.000000in}{0.000000in}}{%
\pgfpathmoveto{\pgfqpoint{0.000000in}{0.000000in}}%
\pgfpathlineto{\pgfqpoint{0.000000in}{0.000000in}}%
\pgfusepath{stroke,fill}%
}%
\begin{pgfscope}%
\pgfsys@transformshift{2.646056in}{1.305000in}%
\pgfsys@useobject{currentmarker}{}%
\end{pgfscope}%
\end{pgfscope}%
\begin{pgfscope}%
\definecolor{textcolor}{rgb}{0.150000,0.150000,0.150000}%
\pgfsetstrokecolor{textcolor}%
\pgfsetfillcolor{textcolor}%
\pgftext[x=2.646056in,y=0.297222in,,top]{\color{textcolor}\sffamily\fontsize{8.000000}{9.600000}\selectfont 10}%
\end{pgfscope}%
\begin{pgfscope}%
\definecolor{textcolor}{rgb}{0.150000,0.150000,0.150000}%
\pgfsetstrokecolor{textcolor}%
\pgfsetfillcolor{textcolor}%
\pgftext[x=1.806818in,y=0.132099in,,top]{\color{textcolor}\sffamily\fontsize{8.800000}{10.560000}\selectfont Tail length}%
\end{pgfscope}%
\begin{pgfscope}%
\pgfpathrectangle{\pgfqpoint{0.750000in}{0.375000in}}{\pgfqpoint{2.113636in}{0.930000in}} %
\pgfusepath{clip}%
\pgfsetroundcap%
\pgfsetroundjoin%
\pgfsetlinewidth{0.803000pt}%
\definecolor{currentstroke}{rgb}{1.000000,1.000000,1.000000}%
\pgfsetstrokecolor{currentstroke}%
\pgfsetdash{}{0pt}%
\pgfpathmoveto{\pgfqpoint{0.750000in}{0.375000in}}%
\pgfpathlineto{\pgfqpoint{2.863636in}{0.375000in}}%
\pgfusepath{stroke}%
\end{pgfscope}%
\begin{pgfscope}%
\pgfsetbuttcap%
\pgfsetroundjoin%
\definecolor{currentfill}{rgb}{0.150000,0.150000,0.150000}%
\pgfsetfillcolor{currentfill}%
\pgfsetlinewidth{0.803000pt}%
\definecolor{currentstroke}{rgb}{0.150000,0.150000,0.150000}%
\pgfsetstrokecolor{currentstroke}%
\pgfsetdash{}{0pt}%
\pgfsys@defobject{currentmarker}{\pgfqpoint{0.000000in}{0.000000in}}{\pgfqpoint{0.000000in}{0.000000in}}{%
\pgfpathmoveto{\pgfqpoint{0.000000in}{0.000000in}}%
\pgfpathlineto{\pgfqpoint{0.000000in}{0.000000in}}%
\pgfusepath{stroke,fill}%
}%
\begin{pgfscope}%
\pgfsys@transformshift{0.750000in}{0.375000in}%
\pgfsys@useobject{currentmarker}{}%
\end{pgfscope}%
\end{pgfscope}%
\begin{pgfscope}%
\pgfsetbuttcap%
\pgfsetroundjoin%
\definecolor{currentfill}{rgb}{0.150000,0.150000,0.150000}%
\pgfsetfillcolor{currentfill}%
\pgfsetlinewidth{0.803000pt}%
\definecolor{currentstroke}{rgb}{0.150000,0.150000,0.150000}%
\pgfsetstrokecolor{currentstroke}%
\pgfsetdash{}{0pt}%
\pgfsys@defobject{currentmarker}{\pgfqpoint{0.000000in}{0.000000in}}{\pgfqpoint{0.000000in}{0.000000in}}{%
\pgfpathmoveto{\pgfqpoint{0.000000in}{0.000000in}}%
\pgfpathlineto{\pgfqpoint{0.000000in}{0.000000in}}%
\pgfusepath{stroke,fill}%
}%
\begin{pgfscope}%
\pgfsys@transformshift{2.863636in}{0.375000in}%
\pgfsys@useobject{currentmarker}{}%
\end{pgfscope}%
\end{pgfscope}%
\begin{pgfscope}%
\definecolor{textcolor}{rgb}{0.150000,0.150000,0.150000}%
\pgfsetstrokecolor{textcolor}%
\pgfsetfillcolor{textcolor}%
\pgftext[x=0.672222in,y=0.375000in,right,]{\color{textcolor}\sffamily\fontsize{8.000000}{9.600000}\selectfont 2.0}%
\end{pgfscope}%
\begin{pgfscope}%
\pgfpathrectangle{\pgfqpoint{0.750000in}{0.375000in}}{\pgfqpoint{2.113636in}{0.930000in}} %
\pgfusepath{clip}%
\pgfsetroundcap%
\pgfsetroundjoin%
\pgfsetlinewidth{0.803000pt}%
\definecolor{currentstroke}{rgb}{1.000000,1.000000,1.000000}%
\pgfsetstrokecolor{currentstroke}%
\pgfsetdash{}{0pt}%
\pgfpathmoveto{\pgfqpoint{0.750000in}{0.530000in}}%
\pgfpathlineto{\pgfqpoint{2.863636in}{0.530000in}}%
\pgfusepath{stroke}%
\end{pgfscope}%
\begin{pgfscope}%
\pgfsetbuttcap%
\pgfsetroundjoin%
\definecolor{currentfill}{rgb}{0.150000,0.150000,0.150000}%
\pgfsetfillcolor{currentfill}%
\pgfsetlinewidth{0.803000pt}%
\definecolor{currentstroke}{rgb}{0.150000,0.150000,0.150000}%
\pgfsetstrokecolor{currentstroke}%
\pgfsetdash{}{0pt}%
\pgfsys@defobject{currentmarker}{\pgfqpoint{0.000000in}{0.000000in}}{\pgfqpoint{0.000000in}{0.000000in}}{%
\pgfpathmoveto{\pgfqpoint{0.000000in}{0.000000in}}%
\pgfpathlineto{\pgfqpoint{0.000000in}{0.000000in}}%
\pgfusepath{stroke,fill}%
}%
\begin{pgfscope}%
\pgfsys@transformshift{0.750000in}{0.530000in}%
\pgfsys@useobject{currentmarker}{}%
\end{pgfscope}%
\end{pgfscope}%
\begin{pgfscope}%
\pgfsetbuttcap%
\pgfsetroundjoin%
\definecolor{currentfill}{rgb}{0.150000,0.150000,0.150000}%
\pgfsetfillcolor{currentfill}%
\pgfsetlinewidth{0.803000pt}%
\definecolor{currentstroke}{rgb}{0.150000,0.150000,0.150000}%
\pgfsetstrokecolor{currentstroke}%
\pgfsetdash{}{0pt}%
\pgfsys@defobject{currentmarker}{\pgfqpoint{0.000000in}{0.000000in}}{\pgfqpoint{0.000000in}{0.000000in}}{%
\pgfpathmoveto{\pgfqpoint{0.000000in}{0.000000in}}%
\pgfpathlineto{\pgfqpoint{0.000000in}{0.000000in}}%
\pgfusepath{stroke,fill}%
}%
\begin{pgfscope}%
\pgfsys@transformshift{2.863636in}{0.530000in}%
\pgfsys@useobject{currentmarker}{}%
\end{pgfscope}%
\end{pgfscope}%
\begin{pgfscope}%
\definecolor{textcolor}{rgb}{0.150000,0.150000,0.150000}%
\pgfsetstrokecolor{textcolor}%
\pgfsetfillcolor{textcolor}%
\pgftext[x=0.672222in,y=0.530000in,right,]{\color{textcolor}\sffamily\fontsize{8.000000}{9.600000}\selectfont 2.5}%
\end{pgfscope}%
\begin{pgfscope}%
\pgfpathrectangle{\pgfqpoint{0.750000in}{0.375000in}}{\pgfqpoint{2.113636in}{0.930000in}} %
\pgfusepath{clip}%
\pgfsetroundcap%
\pgfsetroundjoin%
\pgfsetlinewidth{0.803000pt}%
\definecolor{currentstroke}{rgb}{1.000000,1.000000,1.000000}%
\pgfsetstrokecolor{currentstroke}%
\pgfsetdash{}{0pt}%
\pgfpathmoveto{\pgfqpoint{0.750000in}{0.685000in}}%
\pgfpathlineto{\pgfqpoint{2.863636in}{0.685000in}}%
\pgfusepath{stroke}%
\end{pgfscope}%
\begin{pgfscope}%
\pgfsetbuttcap%
\pgfsetroundjoin%
\definecolor{currentfill}{rgb}{0.150000,0.150000,0.150000}%
\pgfsetfillcolor{currentfill}%
\pgfsetlinewidth{0.803000pt}%
\definecolor{currentstroke}{rgb}{0.150000,0.150000,0.150000}%
\pgfsetstrokecolor{currentstroke}%
\pgfsetdash{}{0pt}%
\pgfsys@defobject{currentmarker}{\pgfqpoint{0.000000in}{0.000000in}}{\pgfqpoint{0.000000in}{0.000000in}}{%
\pgfpathmoveto{\pgfqpoint{0.000000in}{0.000000in}}%
\pgfpathlineto{\pgfqpoint{0.000000in}{0.000000in}}%
\pgfusepath{stroke,fill}%
}%
\begin{pgfscope}%
\pgfsys@transformshift{0.750000in}{0.685000in}%
\pgfsys@useobject{currentmarker}{}%
\end{pgfscope}%
\end{pgfscope}%
\begin{pgfscope}%
\pgfsetbuttcap%
\pgfsetroundjoin%
\definecolor{currentfill}{rgb}{0.150000,0.150000,0.150000}%
\pgfsetfillcolor{currentfill}%
\pgfsetlinewidth{0.803000pt}%
\definecolor{currentstroke}{rgb}{0.150000,0.150000,0.150000}%
\pgfsetstrokecolor{currentstroke}%
\pgfsetdash{}{0pt}%
\pgfsys@defobject{currentmarker}{\pgfqpoint{0.000000in}{0.000000in}}{\pgfqpoint{0.000000in}{0.000000in}}{%
\pgfpathmoveto{\pgfqpoint{0.000000in}{0.000000in}}%
\pgfpathlineto{\pgfqpoint{0.000000in}{0.000000in}}%
\pgfusepath{stroke,fill}%
}%
\begin{pgfscope}%
\pgfsys@transformshift{2.863636in}{0.685000in}%
\pgfsys@useobject{currentmarker}{}%
\end{pgfscope}%
\end{pgfscope}%
\begin{pgfscope}%
\definecolor{textcolor}{rgb}{0.150000,0.150000,0.150000}%
\pgfsetstrokecolor{textcolor}%
\pgfsetfillcolor{textcolor}%
\pgftext[x=0.672222in,y=0.685000in,right,]{\color{textcolor}\sffamily\fontsize{8.000000}{9.600000}\selectfont 3.0}%
\end{pgfscope}%
\begin{pgfscope}%
\pgfpathrectangle{\pgfqpoint{0.750000in}{0.375000in}}{\pgfqpoint{2.113636in}{0.930000in}} %
\pgfusepath{clip}%
\pgfsetroundcap%
\pgfsetroundjoin%
\pgfsetlinewidth{0.803000pt}%
\definecolor{currentstroke}{rgb}{1.000000,1.000000,1.000000}%
\pgfsetstrokecolor{currentstroke}%
\pgfsetdash{}{0pt}%
\pgfpathmoveto{\pgfqpoint{0.750000in}{0.840000in}}%
\pgfpathlineto{\pgfqpoint{2.863636in}{0.840000in}}%
\pgfusepath{stroke}%
\end{pgfscope}%
\begin{pgfscope}%
\pgfsetbuttcap%
\pgfsetroundjoin%
\definecolor{currentfill}{rgb}{0.150000,0.150000,0.150000}%
\pgfsetfillcolor{currentfill}%
\pgfsetlinewidth{0.803000pt}%
\definecolor{currentstroke}{rgb}{0.150000,0.150000,0.150000}%
\pgfsetstrokecolor{currentstroke}%
\pgfsetdash{}{0pt}%
\pgfsys@defobject{currentmarker}{\pgfqpoint{0.000000in}{0.000000in}}{\pgfqpoint{0.000000in}{0.000000in}}{%
\pgfpathmoveto{\pgfqpoint{0.000000in}{0.000000in}}%
\pgfpathlineto{\pgfqpoint{0.000000in}{0.000000in}}%
\pgfusepath{stroke,fill}%
}%
\begin{pgfscope}%
\pgfsys@transformshift{0.750000in}{0.840000in}%
\pgfsys@useobject{currentmarker}{}%
\end{pgfscope}%
\end{pgfscope}%
\begin{pgfscope}%
\pgfsetbuttcap%
\pgfsetroundjoin%
\definecolor{currentfill}{rgb}{0.150000,0.150000,0.150000}%
\pgfsetfillcolor{currentfill}%
\pgfsetlinewidth{0.803000pt}%
\definecolor{currentstroke}{rgb}{0.150000,0.150000,0.150000}%
\pgfsetstrokecolor{currentstroke}%
\pgfsetdash{}{0pt}%
\pgfsys@defobject{currentmarker}{\pgfqpoint{0.000000in}{0.000000in}}{\pgfqpoint{0.000000in}{0.000000in}}{%
\pgfpathmoveto{\pgfqpoint{0.000000in}{0.000000in}}%
\pgfpathlineto{\pgfqpoint{0.000000in}{0.000000in}}%
\pgfusepath{stroke,fill}%
}%
\begin{pgfscope}%
\pgfsys@transformshift{2.863636in}{0.840000in}%
\pgfsys@useobject{currentmarker}{}%
\end{pgfscope}%
\end{pgfscope}%
\begin{pgfscope}%
\definecolor{textcolor}{rgb}{0.150000,0.150000,0.150000}%
\pgfsetstrokecolor{textcolor}%
\pgfsetfillcolor{textcolor}%
\pgftext[x=0.672222in,y=0.840000in,right,]{\color{textcolor}\sffamily\fontsize{8.000000}{9.600000}\selectfont 3.5}%
\end{pgfscope}%
\begin{pgfscope}%
\pgfpathrectangle{\pgfqpoint{0.750000in}{0.375000in}}{\pgfqpoint{2.113636in}{0.930000in}} %
\pgfusepath{clip}%
\pgfsetroundcap%
\pgfsetroundjoin%
\pgfsetlinewidth{0.803000pt}%
\definecolor{currentstroke}{rgb}{1.000000,1.000000,1.000000}%
\pgfsetstrokecolor{currentstroke}%
\pgfsetdash{}{0pt}%
\pgfpathmoveto{\pgfqpoint{0.750000in}{0.995000in}}%
\pgfpathlineto{\pgfqpoint{2.863636in}{0.995000in}}%
\pgfusepath{stroke}%
\end{pgfscope}%
\begin{pgfscope}%
\pgfsetbuttcap%
\pgfsetroundjoin%
\definecolor{currentfill}{rgb}{0.150000,0.150000,0.150000}%
\pgfsetfillcolor{currentfill}%
\pgfsetlinewidth{0.803000pt}%
\definecolor{currentstroke}{rgb}{0.150000,0.150000,0.150000}%
\pgfsetstrokecolor{currentstroke}%
\pgfsetdash{}{0pt}%
\pgfsys@defobject{currentmarker}{\pgfqpoint{0.000000in}{0.000000in}}{\pgfqpoint{0.000000in}{0.000000in}}{%
\pgfpathmoveto{\pgfqpoint{0.000000in}{0.000000in}}%
\pgfpathlineto{\pgfqpoint{0.000000in}{0.000000in}}%
\pgfusepath{stroke,fill}%
}%
\begin{pgfscope}%
\pgfsys@transformshift{0.750000in}{0.995000in}%
\pgfsys@useobject{currentmarker}{}%
\end{pgfscope}%
\end{pgfscope}%
\begin{pgfscope}%
\pgfsetbuttcap%
\pgfsetroundjoin%
\definecolor{currentfill}{rgb}{0.150000,0.150000,0.150000}%
\pgfsetfillcolor{currentfill}%
\pgfsetlinewidth{0.803000pt}%
\definecolor{currentstroke}{rgb}{0.150000,0.150000,0.150000}%
\pgfsetstrokecolor{currentstroke}%
\pgfsetdash{}{0pt}%
\pgfsys@defobject{currentmarker}{\pgfqpoint{0.000000in}{0.000000in}}{\pgfqpoint{0.000000in}{0.000000in}}{%
\pgfpathmoveto{\pgfqpoint{0.000000in}{0.000000in}}%
\pgfpathlineto{\pgfqpoint{0.000000in}{0.000000in}}%
\pgfusepath{stroke,fill}%
}%
\begin{pgfscope}%
\pgfsys@transformshift{2.863636in}{0.995000in}%
\pgfsys@useobject{currentmarker}{}%
\end{pgfscope}%
\end{pgfscope}%
\begin{pgfscope}%
\definecolor{textcolor}{rgb}{0.150000,0.150000,0.150000}%
\pgfsetstrokecolor{textcolor}%
\pgfsetfillcolor{textcolor}%
\pgftext[x=0.672222in,y=0.995000in,right,]{\color{textcolor}\sffamily\fontsize{8.000000}{9.600000}\selectfont 4.0}%
\end{pgfscope}%
\begin{pgfscope}%
\pgfpathrectangle{\pgfqpoint{0.750000in}{0.375000in}}{\pgfqpoint{2.113636in}{0.930000in}} %
\pgfusepath{clip}%
\pgfsetroundcap%
\pgfsetroundjoin%
\pgfsetlinewidth{0.803000pt}%
\definecolor{currentstroke}{rgb}{1.000000,1.000000,1.000000}%
\pgfsetstrokecolor{currentstroke}%
\pgfsetdash{}{0pt}%
\pgfpathmoveto{\pgfqpoint{0.750000in}{1.150000in}}%
\pgfpathlineto{\pgfqpoint{2.863636in}{1.150000in}}%
\pgfusepath{stroke}%
\end{pgfscope}%
\begin{pgfscope}%
\pgfsetbuttcap%
\pgfsetroundjoin%
\definecolor{currentfill}{rgb}{0.150000,0.150000,0.150000}%
\pgfsetfillcolor{currentfill}%
\pgfsetlinewidth{0.803000pt}%
\definecolor{currentstroke}{rgb}{0.150000,0.150000,0.150000}%
\pgfsetstrokecolor{currentstroke}%
\pgfsetdash{}{0pt}%
\pgfsys@defobject{currentmarker}{\pgfqpoint{0.000000in}{0.000000in}}{\pgfqpoint{0.000000in}{0.000000in}}{%
\pgfpathmoveto{\pgfqpoint{0.000000in}{0.000000in}}%
\pgfpathlineto{\pgfqpoint{0.000000in}{0.000000in}}%
\pgfusepath{stroke,fill}%
}%
\begin{pgfscope}%
\pgfsys@transformshift{0.750000in}{1.150000in}%
\pgfsys@useobject{currentmarker}{}%
\end{pgfscope}%
\end{pgfscope}%
\begin{pgfscope}%
\pgfsetbuttcap%
\pgfsetroundjoin%
\definecolor{currentfill}{rgb}{0.150000,0.150000,0.150000}%
\pgfsetfillcolor{currentfill}%
\pgfsetlinewidth{0.803000pt}%
\definecolor{currentstroke}{rgb}{0.150000,0.150000,0.150000}%
\pgfsetstrokecolor{currentstroke}%
\pgfsetdash{}{0pt}%
\pgfsys@defobject{currentmarker}{\pgfqpoint{0.000000in}{0.000000in}}{\pgfqpoint{0.000000in}{0.000000in}}{%
\pgfpathmoveto{\pgfqpoint{0.000000in}{0.000000in}}%
\pgfpathlineto{\pgfqpoint{0.000000in}{0.000000in}}%
\pgfusepath{stroke,fill}%
}%
\begin{pgfscope}%
\pgfsys@transformshift{2.863636in}{1.150000in}%
\pgfsys@useobject{currentmarker}{}%
\end{pgfscope}%
\end{pgfscope}%
\begin{pgfscope}%
\definecolor{textcolor}{rgb}{0.150000,0.150000,0.150000}%
\pgfsetstrokecolor{textcolor}%
\pgfsetfillcolor{textcolor}%
\pgftext[x=0.672222in,y=1.150000in,right,]{\color{textcolor}\sffamily\fontsize{8.000000}{9.600000}\selectfont 4.5}%
\end{pgfscope}%
\begin{pgfscope}%
\pgfpathrectangle{\pgfqpoint{0.750000in}{0.375000in}}{\pgfqpoint{2.113636in}{0.930000in}} %
\pgfusepath{clip}%
\pgfsetroundcap%
\pgfsetroundjoin%
\pgfsetlinewidth{0.803000pt}%
\definecolor{currentstroke}{rgb}{1.000000,1.000000,1.000000}%
\pgfsetstrokecolor{currentstroke}%
\pgfsetdash{}{0pt}%
\pgfpathmoveto{\pgfqpoint{0.750000in}{1.305000in}}%
\pgfpathlineto{\pgfqpoint{2.863636in}{1.305000in}}%
\pgfusepath{stroke}%
\end{pgfscope}%
\begin{pgfscope}%
\pgfsetbuttcap%
\pgfsetroundjoin%
\definecolor{currentfill}{rgb}{0.150000,0.150000,0.150000}%
\pgfsetfillcolor{currentfill}%
\pgfsetlinewidth{0.803000pt}%
\definecolor{currentstroke}{rgb}{0.150000,0.150000,0.150000}%
\pgfsetstrokecolor{currentstroke}%
\pgfsetdash{}{0pt}%
\pgfsys@defobject{currentmarker}{\pgfqpoint{0.000000in}{0.000000in}}{\pgfqpoint{0.000000in}{0.000000in}}{%
\pgfpathmoveto{\pgfqpoint{0.000000in}{0.000000in}}%
\pgfpathlineto{\pgfqpoint{0.000000in}{0.000000in}}%
\pgfusepath{stroke,fill}%
}%
\begin{pgfscope}%
\pgfsys@transformshift{0.750000in}{1.305000in}%
\pgfsys@useobject{currentmarker}{}%
\end{pgfscope}%
\end{pgfscope}%
\begin{pgfscope}%
\pgfsetbuttcap%
\pgfsetroundjoin%
\definecolor{currentfill}{rgb}{0.150000,0.150000,0.150000}%
\pgfsetfillcolor{currentfill}%
\pgfsetlinewidth{0.803000pt}%
\definecolor{currentstroke}{rgb}{0.150000,0.150000,0.150000}%
\pgfsetstrokecolor{currentstroke}%
\pgfsetdash{}{0pt}%
\pgfsys@defobject{currentmarker}{\pgfqpoint{0.000000in}{0.000000in}}{\pgfqpoint{0.000000in}{0.000000in}}{%
\pgfpathmoveto{\pgfqpoint{0.000000in}{0.000000in}}%
\pgfpathlineto{\pgfqpoint{0.000000in}{0.000000in}}%
\pgfusepath{stroke,fill}%
}%
\begin{pgfscope}%
\pgfsys@transformshift{2.863636in}{1.305000in}%
\pgfsys@useobject{currentmarker}{}%
\end{pgfscope}%
\end{pgfscope}%
\begin{pgfscope}%
\definecolor{textcolor}{rgb}{0.150000,0.150000,0.150000}%
\pgfsetstrokecolor{textcolor}%
\pgfsetfillcolor{textcolor}%
\pgftext[x=0.672222in,y=1.305000in,right,]{\color{textcolor}\sffamily\fontsize{8.000000}{9.600000}\selectfont 5.0}%
\end{pgfscope}%
\begin{pgfscope}%
\definecolor{textcolor}{rgb}{0.150000,0.150000,0.150000}%
\pgfsetstrokecolor{textcolor}%
\pgfsetfillcolor{textcolor}%
\pgftext[x=0.444830in,y=0.840000in,,bottom,rotate=90.000000]{\color{textcolor}\sffamily\fontsize{8.800000}{10.560000}\selectfont Flying time}%
\end{pgfscope}%
\begin{pgfscope}%
\pgfpathrectangle{\pgfqpoint{0.750000in}{0.375000in}}{\pgfqpoint{2.113636in}{0.930000in}} %
\pgfusepath{clip}%
\pgfsetbuttcap%
\pgfsetmiterjoin%
\definecolor{currentfill}{rgb}{0.447059,0.623529,0.811765}%
\pgfsetfillcolor{currentfill}%
\pgfsetfillopacity{0.300000}%
\pgfsetlinewidth{0.240900pt}%
\definecolor{currentstroke}{rgb}{0.447059,0.623529,0.811765}%
\pgfsetstrokecolor{currentstroke}%
\pgfsetstrokeopacity{0.300000}%
\pgfsetdash{}{0pt}%
\pgfpathmoveto{\pgfqpoint{0.827707in}{0.860378in}}%
\pgfpathlineto{\pgfqpoint{0.847487in}{0.834695in}}%
\pgfpathlineto{\pgfqpoint{0.867267in}{0.810998in}}%
\pgfpathlineto{\pgfqpoint{0.887047in}{0.789290in}}%
\pgfpathlineto{\pgfqpoint{0.906827in}{0.769560in}}%
\pgfpathlineto{\pgfqpoint{0.926607in}{0.751780in}}%
\pgfpathlineto{\pgfqpoint{0.946387in}{0.735902in}}%
\pgfpathlineto{\pgfqpoint{0.966167in}{0.721863in}}%
\pgfpathlineto{\pgfqpoint{0.985947in}{0.709582in}}%
\pgfpathlineto{\pgfqpoint{1.005727in}{0.698964in}}%
\pgfpathlineto{\pgfqpoint{1.025507in}{0.689911in}}%
\pgfpathlineto{\pgfqpoint{1.045287in}{0.682318in}}%
\pgfpathlineto{\pgfqpoint{1.065067in}{0.676082in}}%
\pgfpathlineto{\pgfqpoint{1.084847in}{0.671107in}}%
\pgfpathlineto{\pgfqpoint{1.104627in}{0.667301in}}%
\pgfpathlineto{\pgfqpoint{1.124408in}{0.664580in}}%
\pgfpathlineto{\pgfqpoint{1.144188in}{0.662869in}}%
\pgfpathlineto{\pgfqpoint{1.163968in}{0.662101in}}%
\pgfpathlineto{\pgfqpoint{1.183748in}{0.662215in}}%
\pgfpathlineto{\pgfqpoint{1.203528in}{0.663158in}}%
\pgfpathlineto{\pgfqpoint{1.223308in}{0.664882in}}%
\pgfpathlineto{\pgfqpoint{1.243088in}{0.667345in}}%
\pgfpathlineto{\pgfqpoint{1.262868in}{0.670507in}}%
\pgfpathlineto{\pgfqpoint{1.282648in}{0.674334in}}%
\pgfpathlineto{\pgfqpoint{1.302428in}{0.678795in}}%
\pgfpathlineto{\pgfqpoint{1.322208in}{0.683861in}}%
\pgfpathlineto{\pgfqpoint{1.341988in}{0.689506in}}%
\pgfpathlineto{\pgfqpoint{1.361768in}{0.695707in}}%
\pgfpathlineto{\pgfqpoint{1.381548in}{0.702443in}}%
\pgfpathlineto{\pgfqpoint{1.401328in}{0.709694in}}%
\pgfpathlineto{\pgfqpoint{1.421108in}{0.717444in}}%
\pgfpathlineto{\pgfqpoint{1.440888in}{0.725678in}}%
\pgfpathlineto{\pgfqpoint{1.460668in}{0.734384in}}%
\pgfpathlineto{\pgfqpoint{1.480448in}{0.743553in}}%
\pgfpathlineto{\pgfqpoint{1.500228in}{0.753177in}}%
\pgfpathlineto{\pgfqpoint{1.520008in}{0.763252in}}%
\pgfpathlineto{\pgfqpoint{1.539788in}{0.773776in}}%
\pgfpathlineto{\pgfqpoint{1.559568in}{0.784750in}}%
\pgfpathlineto{\pgfqpoint{1.579348in}{0.796176in}}%
\pgfpathlineto{\pgfqpoint{1.599128in}{0.808058in}}%
\pgfpathlineto{\pgfqpoint{1.618908in}{0.820399in}}%
\pgfpathlineto{\pgfqpoint{1.638688in}{0.833197in}}%
\pgfpathlineto{\pgfqpoint{1.658468in}{0.846448in}}%
\pgfpathlineto{\pgfqpoint{1.678248in}{0.860139in}}%
\pgfpathlineto{\pgfqpoint{1.698028in}{0.874247in}}%
\pgfpathlineto{\pgfqpoint{1.717808in}{0.888737in}}%
\pgfpathlineto{\pgfqpoint{1.737588in}{0.903561in}}%
\pgfpathlineto{\pgfqpoint{1.757368in}{0.918661in}}%
\pgfpathlineto{\pgfqpoint{1.777148in}{0.933968in}}%
\pgfpathlineto{\pgfqpoint{1.796928in}{0.949408in}}%
\pgfpathlineto{\pgfqpoint{1.816708in}{0.964901in}}%
\pgfpathlineto{\pgfqpoint{1.836488in}{0.980365in}}%
\pgfpathlineto{\pgfqpoint{1.856268in}{0.995721in}}%
\pgfpathlineto{\pgfqpoint{1.876048in}{1.010889in}}%
\pgfpathlineto{\pgfqpoint{1.895828in}{1.025793in}}%
\pgfpathlineto{\pgfqpoint{1.915608in}{1.040363in}}%
\pgfpathlineto{\pgfqpoint{1.935388in}{1.054529in}}%
\pgfpathlineto{\pgfqpoint{1.955168in}{1.068229in}}%
\pgfpathlineto{\pgfqpoint{1.974948in}{1.081403in}}%
\pgfpathlineto{\pgfqpoint{1.994728in}{1.093994in}}%
\pgfpathlineto{\pgfqpoint{2.014508in}{1.105953in}}%
\pgfpathlineto{\pgfqpoint{2.034288in}{1.117231in}}%
\pgfpathlineto{\pgfqpoint{2.054068in}{1.127785in}}%
\pgfpathlineto{\pgfqpoint{2.073848in}{1.137575in}}%
\pgfpathlineto{\pgfqpoint{2.093628in}{1.146567in}}%
\pgfpathlineto{\pgfqpoint{2.113408in}{1.154730in}}%
\pgfpathlineto{\pgfqpoint{2.133189in}{1.162036in}}%
\pgfpathlineto{\pgfqpoint{2.152969in}{1.168463in}}%
\pgfpathlineto{\pgfqpoint{2.172749in}{1.173996in}}%
\pgfpathlineto{\pgfqpoint{2.192529in}{1.178622in}}%
\pgfpathlineto{\pgfqpoint{2.212309in}{1.182336in}}%
\pgfpathlineto{\pgfqpoint{2.232089in}{1.185138in}}%
\pgfpathlineto{\pgfqpoint{2.251869in}{1.187036in}}%
\pgfpathlineto{\pgfqpoint{2.271649in}{1.188045in}}%
\pgfpathlineto{\pgfqpoint{2.291429in}{1.188189in}}%
\pgfpathlineto{\pgfqpoint{2.311209in}{1.187500in}}%
\pgfpathlineto{\pgfqpoint{2.330989in}{1.186019in}}%
\pgfpathlineto{\pgfqpoint{2.350769in}{1.183797in}}%
\pgfpathlineto{\pgfqpoint{2.370549in}{1.180893in}}%
\pgfpathlineto{\pgfqpoint{2.390329in}{1.177375in}}%
\pgfpathlineto{\pgfqpoint{2.410109in}{1.173316in}}%
\pgfpathlineto{\pgfqpoint{2.429889in}{1.168793in}}%
\pgfpathlineto{\pgfqpoint{2.449669in}{1.163882in}}%
\pgfpathlineto{\pgfqpoint{2.469449in}{1.158655in}}%
\pgfpathlineto{\pgfqpoint{2.489229in}{1.153177in}}%
\pgfpathlineto{\pgfqpoint{2.509009in}{1.147500in}}%
\pgfpathlineto{\pgfqpoint{2.528789in}{1.141665in}}%
\pgfpathlineto{\pgfqpoint{2.548569in}{1.135698in}}%
\pgfpathlineto{\pgfqpoint{2.568349in}{1.129614in}}%
\pgfpathlineto{\pgfqpoint{2.588129in}{1.123418in}}%
\pgfpathlineto{\pgfqpoint{2.607909in}{1.117104in}}%
\pgfpathlineto{\pgfqpoint{2.627689in}{1.110665in}}%
\pgfpathlineto{\pgfqpoint{2.647469in}{1.104088in}}%
\pgfpathlineto{\pgfqpoint{2.667249in}{1.097362in}}%
\pgfpathlineto{\pgfqpoint{2.687029in}{1.090473in}}%
\pgfpathlineto{\pgfqpoint{2.706809in}{1.083410in}}%
\pgfpathlineto{\pgfqpoint{2.726589in}{1.076165in}}%
\pgfpathlineto{\pgfqpoint{2.746369in}{1.068730in}}%
\pgfpathlineto{\pgfqpoint{2.766149in}{1.061102in}}%
\pgfpathlineto{\pgfqpoint{2.785929in}{1.053278in}}%
\pgfpathlineto{\pgfqpoint{2.785929in}{0.644630in}}%
\pgfpathlineto{\pgfqpoint{2.766149in}{0.675412in}}%
\pgfpathlineto{\pgfqpoint{2.746369in}{0.705131in}}%
\pgfpathlineto{\pgfqpoint{2.726589in}{0.733738in}}%
\pgfpathlineto{\pgfqpoint{2.706809in}{0.761184in}}%
\pgfpathlineto{\pgfqpoint{2.687029in}{0.787418in}}%
\pgfpathlineto{\pgfqpoint{2.667249in}{0.812391in}}%
\pgfpathlineto{\pgfqpoint{2.647469in}{0.836054in}}%
\pgfpathlineto{\pgfqpoint{2.627689in}{0.858359in}}%
\pgfpathlineto{\pgfqpoint{2.607909in}{0.879261in}}%
\pgfpathlineto{\pgfqpoint{2.588129in}{0.898719in}}%
\pgfpathlineto{\pgfqpoint{2.568349in}{0.916695in}}%
\pgfpathlineto{\pgfqpoint{2.548569in}{0.933162in}}%
\pgfpathlineto{\pgfqpoint{2.528789in}{0.948102in}}%
\pgfpathlineto{\pgfqpoint{2.509009in}{0.961511in}}%
\pgfpathlineto{\pgfqpoint{2.489229in}{0.973401in}}%
\pgfpathlineto{\pgfqpoint{2.469449in}{0.983798in}}%
\pgfpathlineto{\pgfqpoint{2.449669in}{0.992746in}}%
\pgfpathlineto{\pgfqpoint{2.429889in}{1.000304in}}%
\pgfpathlineto{\pgfqpoint{2.410109in}{1.006538in}}%
\pgfpathlineto{\pgfqpoint{2.390329in}{1.011526in}}%
\pgfpathlineto{\pgfqpoint{2.370549in}{1.015347in}}%
\pgfpathlineto{\pgfqpoint{2.350769in}{1.018080in}}%
\pgfpathlineto{\pgfqpoint{2.330989in}{1.019802in}}%
\pgfpathlineto{\pgfqpoint{2.311209in}{1.020583in}}%
\pgfpathlineto{\pgfqpoint{2.291429in}{1.020492in}}%
\pgfpathlineto{\pgfqpoint{2.271649in}{1.019586in}}%
\pgfpathlineto{\pgfqpoint{2.251869in}{1.017919in}}%
\pgfpathlineto{\pgfqpoint{2.232089in}{1.015540in}}%
\pgfpathlineto{\pgfqpoint{2.212309in}{1.012492in}}%
\pgfpathlineto{\pgfqpoint{2.192529in}{1.008812in}}%
\pgfpathlineto{\pgfqpoint{2.172749in}{1.004535in}}%
\pgfpathlineto{\pgfqpoint{2.152969in}{0.999690in}}%
\pgfpathlineto{\pgfqpoint{2.133189in}{0.994307in}}%
\pgfpathlineto{\pgfqpoint{2.113408in}{0.988410in}}%
\pgfpathlineto{\pgfqpoint{2.093628in}{0.982022in}}%
\pgfpathlineto{\pgfqpoint{2.073848in}{0.975163in}}%
\pgfpathlineto{\pgfqpoint{2.054068in}{0.967853in}}%
\pgfpathlineto{\pgfqpoint{2.034288in}{0.960108in}}%
\pgfpathlineto{\pgfqpoint{2.014508in}{0.951945in}}%
\pgfpathlineto{\pgfqpoint{1.994728in}{0.943377in}}%
\pgfpathlineto{\pgfqpoint{1.974948in}{0.934416in}}%
\pgfpathlineto{\pgfqpoint{1.955168in}{0.925072in}}%
\pgfpathlineto{\pgfqpoint{1.935388in}{0.915354in}}%
\pgfpathlineto{\pgfqpoint{1.915608in}{0.905267in}}%
\pgfpathlineto{\pgfqpoint{1.895828in}{0.894816in}}%
\pgfpathlineto{\pgfqpoint{1.876048in}{0.884001in}}%
\pgfpathlineto{\pgfqpoint{1.856268in}{0.872822in}}%
\pgfpathlineto{\pgfqpoint{1.836488in}{0.861276in}}%
\pgfpathlineto{\pgfqpoint{1.816708in}{0.849358in}}%
\pgfpathlineto{\pgfqpoint{1.796928in}{0.837061in}}%
\pgfpathlineto{\pgfqpoint{1.777148in}{0.824382in}}%
\pgfpathlineto{\pgfqpoint{1.757368in}{0.811317in}}%
\pgfpathlineto{\pgfqpoint{1.737588in}{0.797870in}}%
\pgfpathlineto{\pgfqpoint{1.717808in}{0.784052in}}%
\pgfpathlineto{\pgfqpoint{1.698028in}{0.769883in}}%
\pgfpathlineto{\pgfqpoint{1.678248in}{0.755397in}}%
\pgfpathlineto{\pgfqpoint{1.658468in}{0.740638in}}%
\pgfpathlineto{\pgfqpoint{1.638688in}{0.725665in}}%
\pgfpathlineto{\pgfqpoint{1.618908in}{0.710545in}}%
\pgfpathlineto{\pgfqpoint{1.599128in}{0.695354in}}%
\pgfpathlineto{\pgfqpoint{1.579348in}{0.680173in}}%
\pgfpathlineto{\pgfqpoint{1.559568in}{0.665085in}}%
\pgfpathlineto{\pgfqpoint{1.539788in}{0.650173in}}%
\pgfpathlineto{\pgfqpoint{1.520008in}{0.635520in}}%
\pgfpathlineto{\pgfqpoint{1.500228in}{0.621204in}}%
\pgfpathlineto{\pgfqpoint{1.480448in}{0.607302in}}%
\pgfpathlineto{\pgfqpoint{1.460668in}{0.593887in}}%
\pgfpathlineto{\pgfqpoint{1.440888in}{0.581028in}}%
\pgfpathlineto{\pgfqpoint{1.421108in}{0.568790in}}%
\pgfpathlineto{\pgfqpoint{1.401328in}{0.557236in}}%
\pgfpathlineto{\pgfqpoint{1.381548in}{0.546422in}}%
\pgfpathlineto{\pgfqpoint{1.361768in}{0.536403in}}%
\pgfpathlineto{\pgfqpoint{1.341988in}{0.527229in}}%
\pgfpathlineto{\pgfqpoint{1.322208in}{0.518946in}}%
\pgfpathlineto{\pgfqpoint{1.302428in}{0.511596in}}%
\pgfpathlineto{\pgfqpoint{1.282648in}{0.505218in}}%
\pgfpathlineto{\pgfqpoint{1.262868in}{0.499843in}}%
\pgfpathlineto{\pgfqpoint{1.243088in}{0.495500in}}%
\pgfpathlineto{\pgfqpoint{1.223308in}{0.492211in}}%
\pgfpathlineto{\pgfqpoint{1.203528in}{0.489994in}}%
\pgfpathlineto{\pgfqpoint{1.183748in}{0.488856in}}%
\pgfpathlineto{\pgfqpoint{1.163968in}{0.488802in}}%
\pgfpathlineto{\pgfqpoint{1.144188in}{0.489823in}}%
\pgfpathlineto{\pgfqpoint{1.124408in}{0.491906in}}%
\pgfpathlineto{\pgfqpoint{1.104627in}{0.495023in}}%
\pgfpathlineto{\pgfqpoint{1.084847in}{0.499140in}}%
\pgfpathlineto{\pgfqpoint{1.065067in}{0.504209in}}%
\pgfpathlineto{\pgfqpoint{1.045287in}{0.510173in}}%
\pgfpathlineto{\pgfqpoint{1.025507in}{0.516965in}}%
\pgfpathlineto{\pgfqpoint{1.005727in}{0.524509in}}%
\pgfpathlineto{\pgfqpoint{0.985947in}{0.532727in}}%
\pgfpathlineto{\pgfqpoint{0.966167in}{0.541540in}}%
\pgfpathlineto{\pgfqpoint{0.946387in}{0.550874in}}%
\pgfpathlineto{\pgfqpoint{0.926607in}{0.560662in}}%
\pgfpathlineto{\pgfqpoint{0.906827in}{0.570854in}}%
\pgfpathlineto{\pgfqpoint{0.887047in}{0.581411in}}%
\pgfpathlineto{\pgfqpoint{0.867267in}{0.592312in}}%
\pgfpathlineto{\pgfqpoint{0.847487in}{0.603548in}}%
\pgfpathlineto{\pgfqpoint{0.827707in}{0.615124in}}%
\pgfpathclose%
\pgfusepath{stroke,fill}%
\end{pgfscope}%
\begin{pgfscope}%
\pgfpathrectangle{\pgfqpoint{0.750000in}{0.375000in}}{\pgfqpoint{2.113636in}{0.930000in}} %
\pgfusepath{clip}%
\pgfsetroundcap%
\pgfsetroundjoin%
\pgfsetlinewidth{2.007500pt}%
\definecolor{currentstroke}{rgb}{0.125490,0.290196,0.529412}%
\pgfsetstrokecolor{currentstroke}%
\pgfsetdash{}{0pt}%
\pgfpathmoveto{\pgfqpoint{0.827707in}{0.737751in}}%
\pgfpathlineto{\pgfqpoint{0.847487in}{0.719122in}}%
\pgfpathlineto{\pgfqpoint{0.867267in}{0.701655in}}%
\pgfpathlineto{\pgfqpoint{0.887047in}{0.685351in}}%
\pgfpathlineto{\pgfqpoint{0.906827in}{0.670207in}}%
\pgfpathlineto{\pgfqpoint{0.926607in}{0.656221in}}%
\pgfpathlineto{\pgfqpoint{0.946387in}{0.643388in}}%
\pgfpathlineto{\pgfqpoint{0.966167in}{0.631702in}}%
\pgfpathlineto{\pgfqpoint{0.985947in}{0.621154in}}%
\pgfpathlineto{\pgfqpoint{1.005727in}{0.611737in}}%
\pgfpathlineto{\pgfqpoint{1.025507in}{0.603438in}}%
\pgfpathlineto{\pgfqpoint{1.045287in}{0.596246in}}%
\pgfpathlineto{\pgfqpoint{1.065067in}{0.590146in}}%
\pgfpathlineto{\pgfqpoint{1.084847in}{0.585124in}}%
\pgfpathlineto{\pgfqpoint{1.104627in}{0.581162in}}%
\pgfpathlineto{\pgfqpoint{1.124408in}{0.578243in}}%
\pgfpathlineto{\pgfqpoint{1.144188in}{0.576346in}}%
\pgfpathlineto{\pgfqpoint{1.163968in}{0.575451in}}%
\pgfpathlineto{\pgfqpoint{1.183748in}{0.575536in}}%
\pgfpathlineto{\pgfqpoint{1.203528in}{0.576576in}}%
\pgfpathlineto{\pgfqpoint{1.223308in}{0.578547in}}%
\pgfpathlineto{\pgfqpoint{1.243088in}{0.581422in}}%
\pgfpathlineto{\pgfqpoint{1.262868in}{0.585175in}}%
\pgfpathlineto{\pgfqpoint{1.282648in}{0.589776in}}%
\pgfpathlineto{\pgfqpoint{1.302428in}{0.595196in}}%
\pgfpathlineto{\pgfqpoint{1.322208in}{0.601404in}}%
\pgfpathlineto{\pgfqpoint{1.341988in}{0.608368in}}%
\pgfpathlineto{\pgfqpoint{1.361768in}{0.616055in}}%
\pgfpathlineto{\pgfqpoint{1.381548in}{0.624432in}}%
\pgfpathlineto{\pgfqpoint{1.401328in}{0.633465in}}%
\pgfpathlineto{\pgfqpoint{1.421108in}{0.643117in}}%
\pgfpathlineto{\pgfqpoint{1.440888in}{0.653353in}}%
\pgfpathlineto{\pgfqpoint{1.460668in}{0.664135in}}%
\pgfpathlineto{\pgfqpoint{1.480448in}{0.675427in}}%
\pgfpathlineto{\pgfqpoint{1.500228in}{0.687190in}}%
\pgfpathlineto{\pgfqpoint{1.520008in}{0.699386in}}%
\pgfpathlineto{\pgfqpoint{1.539788in}{0.711975in}}%
\pgfpathlineto{\pgfqpoint{1.559568in}{0.724918in}}%
\pgfpathlineto{\pgfqpoint{1.579348in}{0.738175in}}%
\pgfpathlineto{\pgfqpoint{1.599128in}{0.751706in}}%
\pgfpathlineto{\pgfqpoint{1.618908in}{0.765472in}}%
\pgfpathlineto{\pgfqpoint{1.638688in}{0.779431in}}%
\pgfpathlineto{\pgfqpoint{1.658468in}{0.793543in}}%
\pgfpathlineto{\pgfqpoint{1.678248in}{0.807768in}}%
\pgfpathlineto{\pgfqpoint{1.698028in}{0.822065in}}%
\pgfpathlineto{\pgfqpoint{1.717808in}{0.836394in}}%
\pgfpathlineto{\pgfqpoint{1.737588in}{0.850716in}}%
\pgfpathlineto{\pgfqpoint{1.757368in}{0.864989in}}%
\pgfpathlineto{\pgfqpoint{1.777148in}{0.879175in}}%
\pgfpathlineto{\pgfqpoint{1.796928in}{0.893235in}}%
\pgfpathlineto{\pgfqpoint{1.816708in}{0.907129in}}%
\pgfpathlineto{\pgfqpoint{1.836488in}{0.920821in}}%
\pgfpathlineto{\pgfqpoint{1.856268in}{0.934272in}}%
\pgfpathlineto{\pgfqpoint{1.876048in}{0.947445in}}%
\pgfpathlineto{\pgfqpoint{1.895828in}{0.960305in}}%
\pgfpathlineto{\pgfqpoint{1.915608in}{0.972815in}}%
\pgfpathlineto{\pgfqpoint{1.935388in}{0.984942in}}%
\pgfpathlineto{\pgfqpoint{1.955168in}{0.996651in}}%
\pgfpathlineto{\pgfqpoint{1.974948in}{1.007909in}}%
\pgfpathlineto{\pgfqpoint{1.994728in}{1.018686in}}%
\pgfpathlineto{\pgfqpoint{2.014508in}{1.028949in}}%
\pgfpathlineto{\pgfqpoint{2.034288in}{1.038670in}}%
\pgfpathlineto{\pgfqpoint{2.054068in}{1.047819in}}%
\pgfpathlineto{\pgfqpoint{2.073848in}{1.056369in}}%
\pgfpathlineto{\pgfqpoint{2.093628in}{1.064295in}}%
\pgfpathlineto{\pgfqpoint{2.113408in}{1.071570in}}%
\pgfpathlineto{\pgfqpoint{2.133189in}{1.078171in}}%
\pgfpathlineto{\pgfqpoint{2.152969in}{1.084077in}}%
\pgfpathlineto{\pgfqpoint{2.172749in}{1.089265in}}%
\pgfpathlineto{\pgfqpoint{2.192529in}{1.093717in}}%
\pgfpathlineto{\pgfqpoint{2.212309in}{1.097414in}}%
\pgfpathlineto{\pgfqpoint{2.232089in}{1.100339in}}%
\pgfpathlineto{\pgfqpoint{2.251869in}{1.102478in}}%
\pgfpathlineto{\pgfqpoint{2.271649in}{1.103816in}}%
\pgfpathlineto{\pgfqpoint{2.291429in}{1.104340in}}%
\pgfpathlineto{\pgfqpoint{2.311209in}{1.104042in}}%
\pgfpathlineto{\pgfqpoint{2.330989in}{1.102910in}}%
\pgfpathlineto{\pgfqpoint{2.350769in}{1.100938in}}%
\pgfpathlineto{\pgfqpoint{2.370549in}{1.098120in}}%
\pgfpathlineto{\pgfqpoint{2.390329in}{1.094451in}}%
\pgfpathlineto{\pgfqpoint{2.410109in}{1.089927in}}%
\pgfpathlineto{\pgfqpoint{2.429889in}{1.084548in}}%
\pgfpathlineto{\pgfqpoint{2.449669in}{1.078314in}}%
\pgfpathlineto{\pgfqpoint{2.469449in}{1.071227in}}%
\pgfpathlineto{\pgfqpoint{2.489229in}{1.063289in}}%
\pgfpathlineto{\pgfqpoint{2.509009in}{1.054506in}}%
\pgfpathlineto{\pgfqpoint{2.528789in}{1.044883in}}%
\pgfpathlineto{\pgfqpoint{2.548569in}{1.034430in}}%
\pgfpathlineto{\pgfqpoint{2.568349in}{1.023155in}}%
\pgfpathlineto{\pgfqpoint{2.588129in}{1.011068in}}%
\pgfpathlineto{\pgfqpoint{2.607909in}{0.998183in}}%
\pgfpathlineto{\pgfqpoint{2.627689in}{0.984512in}}%
\pgfpathlineto{\pgfqpoint{2.647469in}{0.970071in}}%
\pgfpathlineto{\pgfqpoint{2.667249in}{0.954876in}}%
\pgfpathlineto{\pgfqpoint{2.687029in}{0.938945in}}%
\pgfpathlineto{\pgfqpoint{2.706809in}{0.922297in}}%
\pgfpathlineto{\pgfqpoint{2.726589in}{0.904952in}}%
\pgfpathlineto{\pgfqpoint{2.746369in}{0.886931in}}%
\pgfpathlineto{\pgfqpoint{2.766149in}{0.868257in}}%
\pgfpathlineto{\pgfqpoint{2.785929in}{0.848954in}}%
\pgfusepath{stroke}%
\end{pgfscope}%
\begin{pgfscope}%
\pgfpathrectangle{\pgfqpoint{0.750000in}{0.375000in}}{\pgfqpoint{2.113636in}{0.930000in}} %
\pgfusepath{clip}%
\pgfsetroundcap%
\pgfsetroundjoin%
\pgfsetlinewidth{0.200750pt}%
\definecolor{currentstroke}{rgb}{0.125490,0.290196,0.529412}%
\pgfsetstrokecolor{currentstroke}%
\pgfsetdash{}{0pt}%
\pgfpathmoveto{\pgfqpoint{0.827707in}{0.860378in}}%
\pgfpathlineto{\pgfqpoint{0.847487in}{0.834695in}}%
\pgfpathlineto{\pgfqpoint{0.867267in}{0.810998in}}%
\pgfpathlineto{\pgfqpoint{0.887047in}{0.789290in}}%
\pgfpathlineto{\pgfqpoint{0.906827in}{0.769560in}}%
\pgfpathlineto{\pgfqpoint{0.926607in}{0.751780in}}%
\pgfpathlineto{\pgfqpoint{0.946387in}{0.735902in}}%
\pgfpathlineto{\pgfqpoint{0.966167in}{0.721863in}}%
\pgfpathlineto{\pgfqpoint{0.985947in}{0.709582in}}%
\pgfpathlineto{\pgfqpoint{1.005727in}{0.698964in}}%
\pgfpathlineto{\pgfqpoint{1.025507in}{0.689911in}}%
\pgfpathlineto{\pgfqpoint{1.045287in}{0.682318in}}%
\pgfpathlineto{\pgfqpoint{1.065067in}{0.676082in}}%
\pgfpathlineto{\pgfqpoint{1.084847in}{0.671107in}}%
\pgfpathlineto{\pgfqpoint{1.104627in}{0.667301in}}%
\pgfpathlineto{\pgfqpoint{1.124408in}{0.664580in}}%
\pgfpathlineto{\pgfqpoint{1.144188in}{0.662869in}}%
\pgfpathlineto{\pgfqpoint{1.163968in}{0.662101in}}%
\pgfpathlineto{\pgfqpoint{1.183748in}{0.662215in}}%
\pgfpathlineto{\pgfqpoint{1.203528in}{0.663158in}}%
\pgfpathlineto{\pgfqpoint{1.223308in}{0.664882in}}%
\pgfpathlineto{\pgfqpoint{1.243088in}{0.667345in}}%
\pgfpathlineto{\pgfqpoint{1.262868in}{0.670507in}}%
\pgfpathlineto{\pgfqpoint{1.282648in}{0.674334in}}%
\pgfpathlineto{\pgfqpoint{1.302428in}{0.678795in}}%
\pgfpathlineto{\pgfqpoint{1.322208in}{0.683861in}}%
\pgfpathlineto{\pgfqpoint{1.341988in}{0.689506in}}%
\pgfpathlineto{\pgfqpoint{1.361768in}{0.695707in}}%
\pgfpathlineto{\pgfqpoint{1.381548in}{0.702443in}}%
\pgfpathlineto{\pgfqpoint{1.401328in}{0.709694in}}%
\pgfpathlineto{\pgfqpoint{1.421108in}{0.717444in}}%
\pgfpathlineto{\pgfqpoint{1.440888in}{0.725678in}}%
\pgfpathlineto{\pgfqpoint{1.460668in}{0.734384in}}%
\pgfpathlineto{\pgfqpoint{1.480448in}{0.743553in}}%
\pgfpathlineto{\pgfqpoint{1.500228in}{0.753177in}}%
\pgfpathlineto{\pgfqpoint{1.520008in}{0.763252in}}%
\pgfpathlineto{\pgfqpoint{1.539788in}{0.773776in}}%
\pgfpathlineto{\pgfqpoint{1.559568in}{0.784750in}}%
\pgfpathlineto{\pgfqpoint{1.579348in}{0.796176in}}%
\pgfpathlineto{\pgfqpoint{1.599128in}{0.808058in}}%
\pgfpathlineto{\pgfqpoint{1.618908in}{0.820399in}}%
\pgfpathlineto{\pgfqpoint{1.638688in}{0.833197in}}%
\pgfpathlineto{\pgfqpoint{1.658468in}{0.846448in}}%
\pgfpathlineto{\pgfqpoint{1.678248in}{0.860139in}}%
\pgfpathlineto{\pgfqpoint{1.698028in}{0.874247in}}%
\pgfpathlineto{\pgfqpoint{1.717808in}{0.888737in}}%
\pgfpathlineto{\pgfqpoint{1.737588in}{0.903561in}}%
\pgfpathlineto{\pgfqpoint{1.757368in}{0.918661in}}%
\pgfpathlineto{\pgfqpoint{1.777148in}{0.933968in}}%
\pgfpathlineto{\pgfqpoint{1.796928in}{0.949408in}}%
\pgfpathlineto{\pgfqpoint{1.816708in}{0.964901in}}%
\pgfpathlineto{\pgfqpoint{1.836488in}{0.980365in}}%
\pgfpathlineto{\pgfqpoint{1.856268in}{0.995721in}}%
\pgfpathlineto{\pgfqpoint{1.876048in}{1.010889in}}%
\pgfpathlineto{\pgfqpoint{1.895828in}{1.025793in}}%
\pgfpathlineto{\pgfqpoint{1.915608in}{1.040363in}}%
\pgfpathlineto{\pgfqpoint{1.935388in}{1.054529in}}%
\pgfpathlineto{\pgfqpoint{1.955168in}{1.068229in}}%
\pgfpathlineto{\pgfqpoint{1.974948in}{1.081403in}}%
\pgfpathlineto{\pgfqpoint{1.994728in}{1.093994in}}%
\pgfpathlineto{\pgfqpoint{2.014508in}{1.105953in}}%
\pgfpathlineto{\pgfqpoint{2.034288in}{1.117231in}}%
\pgfpathlineto{\pgfqpoint{2.054068in}{1.127785in}}%
\pgfpathlineto{\pgfqpoint{2.073848in}{1.137575in}}%
\pgfpathlineto{\pgfqpoint{2.093628in}{1.146567in}}%
\pgfpathlineto{\pgfqpoint{2.113408in}{1.154730in}}%
\pgfpathlineto{\pgfqpoint{2.133189in}{1.162036in}}%
\pgfpathlineto{\pgfqpoint{2.152969in}{1.168463in}}%
\pgfpathlineto{\pgfqpoint{2.172749in}{1.173996in}}%
\pgfpathlineto{\pgfqpoint{2.192529in}{1.178622in}}%
\pgfpathlineto{\pgfqpoint{2.212309in}{1.182336in}}%
\pgfpathlineto{\pgfqpoint{2.232089in}{1.185138in}}%
\pgfpathlineto{\pgfqpoint{2.251869in}{1.187036in}}%
\pgfpathlineto{\pgfqpoint{2.271649in}{1.188045in}}%
\pgfpathlineto{\pgfqpoint{2.291429in}{1.188189in}}%
\pgfpathlineto{\pgfqpoint{2.311209in}{1.187500in}}%
\pgfpathlineto{\pgfqpoint{2.330989in}{1.186019in}}%
\pgfpathlineto{\pgfqpoint{2.350769in}{1.183797in}}%
\pgfpathlineto{\pgfqpoint{2.370549in}{1.180893in}}%
\pgfpathlineto{\pgfqpoint{2.390329in}{1.177375in}}%
\pgfpathlineto{\pgfqpoint{2.410109in}{1.173316in}}%
\pgfpathlineto{\pgfqpoint{2.429889in}{1.168793in}}%
\pgfpathlineto{\pgfqpoint{2.449669in}{1.163882in}}%
\pgfpathlineto{\pgfqpoint{2.469449in}{1.158655in}}%
\pgfpathlineto{\pgfqpoint{2.489229in}{1.153177in}}%
\pgfpathlineto{\pgfqpoint{2.509009in}{1.147500in}}%
\pgfpathlineto{\pgfqpoint{2.528789in}{1.141665in}}%
\pgfpathlineto{\pgfqpoint{2.548569in}{1.135698in}}%
\pgfpathlineto{\pgfqpoint{2.568349in}{1.129614in}}%
\pgfpathlineto{\pgfqpoint{2.588129in}{1.123418in}}%
\pgfpathlineto{\pgfqpoint{2.607909in}{1.117104in}}%
\pgfpathlineto{\pgfqpoint{2.627689in}{1.110665in}}%
\pgfpathlineto{\pgfqpoint{2.647469in}{1.104088in}}%
\pgfpathlineto{\pgfqpoint{2.667249in}{1.097362in}}%
\pgfpathlineto{\pgfqpoint{2.687029in}{1.090473in}}%
\pgfpathlineto{\pgfqpoint{2.706809in}{1.083410in}}%
\pgfpathlineto{\pgfqpoint{2.726589in}{1.076165in}}%
\pgfpathlineto{\pgfqpoint{2.746369in}{1.068730in}}%
\pgfpathlineto{\pgfqpoint{2.766149in}{1.061102in}}%
\pgfpathlineto{\pgfqpoint{2.785929in}{1.053278in}}%
\pgfusepath{stroke}%
\end{pgfscope}%
\begin{pgfscope}%
\pgfpathrectangle{\pgfqpoint{0.750000in}{0.375000in}}{\pgfqpoint{2.113636in}{0.930000in}} %
\pgfusepath{clip}%
\pgfsetroundcap%
\pgfsetroundjoin%
\pgfsetlinewidth{0.200750pt}%
\definecolor{currentstroke}{rgb}{0.125490,0.290196,0.529412}%
\pgfsetstrokecolor{currentstroke}%
\pgfsetdash{}{0pt}%
\pgfpathmoveto{\pgfqpoint{0.827707in}{0.615124in}}%
\pgfpathlineto{\pgfqpoint{0.847487in}{0.603548in}}%
\pgfpathlineto{\pgfqpoint{0.867267in}{0.592312in}}%
\pgfpathlineto{\pgfqpoint{0.887047in}{0.581411in}}%
\pgfpathlineto{\pgfqpoint{0.906827in}{0.570854in}}%
\pgfpathlineto{\pgfqpoint{0.926607in}{0.560662in}}%
\pgfpathlineto{\pgfqpoint{0.946387in}{0.550874in}}%
\pgfpathlineto{\pgfqpoint{0.966167in}{0.541540in}}%
\pgfpathlineto{\pgfqpoint{0.985947in}{0.532727in}}%
\pgfpathlineto{\pgfqpoint{1.005727in}{0.524509in}}%
\pgfpathlineto{\pgfqpoint{1.025507in}{0.516965in}}%
\pgfpathlineto{\pgfqpoint{1.045287in}{0.510173in}}%
\pgfpathlineto{\pgfqpoint{1.065067in}{0.504209in}}%
\pgfpathlineto{\pgfqpoint{1.084847in}{0.499140in}}%
\pgfpathlineto{\pgfqpoint{1.104627in}{0.495023in}}%
\pgfpathlineto{\pgfqpoint{1.124408in}{0.491906in}}%
\pgfpathlineto{\pgfqpoint{1.144188in}{0.489823in}}%
\pgfpathlineto{\pgfqpoint{1.163968in}{0.488802in}}%
\pgfpathlineto{\pgfqpoint{1.183748in}{0.488856in}}%
\pgfpathlineto{\pgfqpoint{1.203528in}{0.489994in}}%
\pgfpathlineto{\pgfqpoint{1.223308in}{0.492211in}}%
\pgfpathlineto{\pgfqpoint{1.243088in}{0.495500in}}%
\pgfpathlineto{\pgfqpoint{1.262868in}{0.499843in}}%
\pgfpathlineto{\pgfqpoint{1.282648in}{0.505218in}}%
\pgfpathlineto{\pgfqpoint{1.302428in}{0.511596in}}%
\pgfpathlineto{\pgfqpoint{1.322208in}{0.518946in}}%
\pgfpathlineto{\pgfqpoint{1.341988in}{0.527229in}}%
\pgfpathlineto{\pgfqpoint{1.361768in}{0.536403in}}%
\pgfpathlineto{\pgfqpoint{1.381548in}{0.546422in}}%
\pgfpathlineto{\pgfqpoint{1.401328in}{0.557236in}}%
\pgfpathlineto{\pgfqpoint{1.421108in}{0.568790in}}%
\pgfpathlineto{\pgfqpoint{1.440888in}{0.581028in}}%
\pgfpathlineto{\pgfqpoint{1.460668in}{0.593887in}}%
\pgfpathlineto{\pgfqpoint{1.480448in}{0.607302in}}%
\pgfpathlineto{\pgfqpoint{1.500228in}{0.621204in}}%
\pgfpathlineto{\pgfqpoint{1.520008in}{0.635520in}}%
\pgfpathlineto{\pgfqpoint{1.539788in}{0.650173in}}%
\pgfpathlineto{\pgfqpoint{1.559568in}{0.665085in}}%
\pgfpathlineto{\pgfqpoint{1.579348in}{0.680173in}}%
\pgfpathlineto{\pgfqpoint{1.599128in}{0.695354in}}%
\pgfpathlineto{\pgfqpoint{1.618908in}{0.710545in}}%
\pgfpathlineto{\pgfqpoint{1.638688in}{0.725665in}}%
\pgfpathlineto{\pgfqpoint{1.658468in}{0.740638in}}%
\pgfpathlineto{\pgfqpoint{1.678248in}{0.755397in}}%
\pgfpathlineto{\pgfqpoint{1.698028in}{0.769883in}}%
\pgfpathlineto{\pgfqpoint{1.717808in}{0.784052in}}%
\pgfpathlineto{\pgfqpoint{1.737588in}{0.797870in}}%
\pgfpathlineto{\pgfqpoint{1.757368in}{0.811317in}}%
\pgfpathlineto{\pgfqpoint{1.777148in}{0.824382in}}%
\pgfpathlineto{\pgfqpoint{1.796928in}{0.837061in}}%
\pgfpathlineto{\pgfqpoint{1.816708in}{0.849358in}}%
\pgfpathlineto{\pgfqpoint{1.836488in}{0.861276in}}%
\pgfpathlineto{\pgfqpoint{1.856268in}{0.872822in}}%
\pgfpathlineto{\pgfqpoint{1.876048in}{0.884001in}}%
\pgfpathlineto{\pgfqpoint{1.895828in}{0.894816in}}%
\pgfpathlineto{\pgfqpoint{1.915608in}{0.905267in}}%
\pgfpathlineto{\pgfqpoint{1.935388in}{0.915354in}}%
\pgfpathlineto{\pgfqpoint{1.955168in}{0.925072in}}%
\pgfpathlineto{\pgfqpoint{1.974948in}{0.934416in}}%
\pgfpathlineto{\pgfqpoint{1.994728in}{0.943377in}}%
\pgfpathlineto{\pgfqpoint{2.014508in}{0.951945in}}%
\pgfpathlineto{\pgfqpoint{2.034288in}{0.960108in}}%
\pgfpathlineto{\pgfqpoint{2.054068in}{0.967853in}}%
\pgfpathlineto{\pgfqpoint{2.073848in}{0.975163in}}%
\pgfpathlineto{\pgfqpoint{2.093628in}{0.982022in}}%
\pgfpathlineto{\pgfqpoint{2.113408in}{0.988410in}}%
\pgfpathlineto{\pgfqpoint{2.133189in}{0.994307in}}%
\pgfpathlineto{\pgfqpoint{2.152969in}{0.999690in}}%
\pgfpathlineto{\pgfqpoint{2.172749in}{1.004535in}}%
\pgfpathlineto{\pgfqpoint{2.192529in}{1.008812in}}%
\pgfpathlineto{\pgfqpoint{2.212309in}{1.012492in}}%
\pgfpathlineto{\pgfqpoint{2.232089in}{1.015540in}}%
\pgfpathlineto{\pgfqpoint{2.251869in}{1.017919in}}%
\pgfpathlineto{\pgfqpoint{2.271649in}{1.019586in}}%
\pgfpathlineto{\pgfqpoint{2.291429in}{1.020492in}}%
\pgfpathlineto{\pgfqpoint{2.311209in}{1.020583in}}%
\pgfpathlineto{\pgfqpoint{2.330989in}{1.019802in}}%
\pgfpathlineto{\pgfqpoint{2.350769in}{1.018080in}}%
\pgfpathlineto{\pgfqpoint{2.370549in}{1.015347in}}%
\pgfpathlineto{\pgfqpoint{2.390329in}{1.011526in}}%
\pgfpathlineto{\pgfqpoint{2.410109in}{1.006538in}}%
\pgfpathlineto{\pgfqpoint{2.429889in}{1.000304in}}%
\pgfpathlineto{\pgfqpoint{2.449669in}{0.992746in}}%
\pgfpathlineto{\pgfqpoint{2.469449in}{0.983798in}}%
\pgfpathlineto{\pgfqpoint{2.489229in}{0.973401in}}%
\pgfpathlineto{\pgfqpoint{2.509009in}{0.961511in}}%
\pgfpathlineto{\pgfqpoint{2.528789in}{0.948102in}}%
\pgfpathlineto{\pgfqpoint{2.548569in}{0.933162in}}%
\pgfpathlineto{\pgfqpoint{2.568349in}{0.916695in}}%
\pgfpathlineto{\pgfqpoint{2.588129in}{0.898719in}}%
\pgfpathlineto{\pgfqpoint{2.607909in}{0.879261in}}%
\pgfpathlineto{\pgfqpoint{2.627689in}{0.858359in}}%
\pgfpathlineto{\pgfqpoint{2.647469in}{0.836054in}}%
\pgfpathlineto{\pgfqpoint{2.667249in}{0.812391in}}%
\pgfpathlineto{\pgfqpoint{2.687029in}{0.787418in}}%
\pgfpathlineto{\pgfqpoint{2.706809in}{0.761184in}}%
\pgfpathlineto{\pgfqpoint{2.726589in}{0.733738in}}%
\pgfpathlineto{\pgfqpoint{2.746369in}{0.705131in}}%
\pgfpathlineto{\pgfqpoint{2.766149in}{0.675412in}}%
\pgfpathlineto{\pgfqpoint{2.785929in}{0.644630in}}%
\pgfusepath{stroke}%
\end{pgfscope}%
\begin{pgfscope}%
\pgfpathrectangle{\pgfqpoint{0.750000in}{0.375000in}}{\pgfqpoint{2.113636in}{0.930000in}} %
\pgfusepath{clip}%
\pgfsetbuttcap%
\pgfsetbeveljoin%
\definecolor{currentfill}{rgb}{0.298039,0.447059,0.690196}%
\pgfsetfillcolor{currentfill}%
\pgfsetlinewidth{0.000000pt}%
\definecolor{currentstroke}{rgb}{0.000000,0.000000,0.000000}%
\pgfsetstrokecolor{currentstroke}%
\pgfsetdash{}{0pt}%
\pgfsys@defobject{currentmarker}{\pgfqpoint{-0.036986in}{-0.031462in}}{\pgfqpoint{0.036986in}{0.038889in}}{%
\pgfpathmoveto{\pgfqpoint{0.000000in}{0.038889in}}%
\pgfpathlineto{\pgfqpoint{-0.008731in}{0.012017in}}%
\pgfpathlineto{\pgfqpoint{-0.036986in}{0.012017in}}%
\pgfpathlineto{\pgfqpoint{-0.014127in}{-0.004590in}}%
\pgfpathlineto{\pgfqpoint{-0.022858in}{-0.031462in}}%
\pgfpathlineto{\pgfqpoint{-0.000000in}{-0.014854in}}%
\pgfpathlineto{\pgfqpoint{0.022858in}{-0.031462in}}%
\pgfpathlineto{\pgfqpoint{0.014127in}{-0.004590in}}%
\pgfpathlineto{\pgfqpoint{0.036986in}{0.012017in}}%
\pgfpathlineto{\pgfqpoint{0.008731in}{0.012017in}}%
\pgfpathclose%
\pgfusepath{fill}%
}%
\begin{pgfscope}%
\pgfsys@transformshift{1.332804in}{0.948500in}%
\pgfsys@useobject{currentmarker}{}%
\end{pgfscope}%
\begin{pgfscope}%
\pgfsys@transformshift{2.785929in}{0.824500in}%
\pgfsys@useobject{currentmarker}{}%
\end{pgfscope}%
\begin{pgfscope}%
\pgfsys@transformshift{1.752423in}{0.933000in}%
\pgfsys@useobject{currentmarker}{}%
\end{pgfscope}%
\begin{pgfscope}%
\pgfsys@transformshift{1.989430in}{1.134500in}%
\pgfsys@useobject{currentmarker}{}%
\end{pgfscope}%
\begin{pgfscope}%
\pgfsys@transformshift{2.346883in}{0.809000in}%
\pgfsys@useobject{currentmarker}{}%
\end{pgfscope}%
\begin{pgfscope}%
\pgfsys@transformshift{1.585353in}{1.196500in}%
\pgfsys@useobject{currentmarker}{}%
\end{pgfscope}%
\begin{pgfscope}%
\pgfsys@transformshift{1.748538in}{0.809000in}%
\pgfsys@useobject{currentmarker}{}%
\end{pgfscope}%
\begin{pgfscope}%
\pgfsys@transformshift{2.234208in}{1.258500in}%
\pgfsys@useobject{currentmarker}{}%
\end{pgfscope}%
\begin{pgfscope}%
\pgfsys@transformshift{2.704337in}{0.545500in}%
\pgfsys@useobject{currentmarker}{}%
\end{pgfscope}%
\begin{pgfscope}%
\pgfsys@transformshift{2.774273in}{0.840000in}%
\pgfsys@useobject{currentmarker}{}%
\end{pgfscope}%
\begin{pgfscope}%
\pgfsys@transformshift{1.154078in}{0.793500in}%
\pgfsys@useobject{currentmarker}{}%
\end{pgfscope}%
\begin{pgfscope}%
\pgfsys@transformshift{2.463444in}{0.809000in}%
\pgfsys@useobject{currentmarker}{}%
\end{pgfscope}%
\begin{pgfscope}%
\pgfsys@transformshift{1.495989in}{0.855500in}%
\pgfsys@useobject{currentmarker}{}%
\end{pgfscope}%
\begin{pgfscope}%
\pgfsys@transformshift{1.348346in}{0.483500in}%
\pgfsys@useobject{currentmarker}{}%
\end{pgfscope}%
\begin{pgfscope}%
\pgfsys@transformshift{2.241979in}{1.026000in}%
\pgfsys@useobject{currentmarker}{}%
\end{pgfscope}%
\begin{pgfscope}%
\pgfsys@transformshift{2.350769in}{0.979500in}%
\pgfsys@useobject{currentmarker}{}%
\end{pgfscope}%
\begin{pgfscope}%
\pgfsys@transformshift{2.140959in}{0.731500in}%
\pgfsys@useobject{currentmarker}{}%
\end{pgfscope}%
\begin{pgfscope}%
\pgfsys@transformshift{0.827707in}{0.824500in}%
\pgfsys@useobject{currentmarker}{}%
\end{pgfscope}%
\begin{pgfscope}%
\pgfsys@transformshift{1.705799in}{0.607500in}%
\pgfsys@useobject{currentmarker}{}%
\end{pgfscope}%
\begin{pgfscope}%
\pgfsys@transformshift{2.482871in}{0.747000in}%
\pgfsys@useobject{currentmarker}{}%
\end{pgfscope}%
\begin{pgfscope}%
\pgfsys@transformshift{1.126880in}{0.824500in}%
\pgfsys@useobject{currentmarker}{}%
\end{pgfscope}%
\begin{pgfscope}%
\pgfsys@transformshift{0.940383in}{1.041500in}%
\pgfsys@useobject{currentmarker}{}%
\end{pgfscope}%
\begin{pgfscope}%
\pgfsys@transformshift{2.311915in}{0.871000in}%
\pgfsys@useobject{currentmarker}{}%
\end{pgfscope}%
\begin{pgfscope}%
\pgfsys@transformshift{0.893758in}{0.654000in}%
\pgfsys@useobject{currentmarker}{}%
\end{pgfscope}%
\begin{pgfscope}%
\pgfsys@transformshift{1.923379in}{0.917500in}%
\pgfsys@useobject{currentmarker}{}%
\end{pgfscope}%
\begin{pgfscope}%
\pgfsys@transformshift{1.150192in}{1.119000in}%
\pgfsys@useobject{currentmarker}{}%
\end{pgfscope}%
\begin{pgfscope}%
\pgfsys@transformshift{1.670831in}{0.669500in}%
\pgfsys@useobject{currentmarker}{}%
\end{pgfscope}%
\begin{pgfscope}%
\pgfsys@transformshift{1.313377in}{0.561000in}%
\pgfsys@useobject{currentmarker}{}%
\end{pgfscope}%
\begin{pgfscope}%
\pgfsys@transformshift{2.214781in}{0.623000in}%
\pgfsys@useobject{currentmarker}{}%
\end{pgfscope}%
\begin{pgfscope}%
\pgfsys@transformshift{1.577582in}{1.134500in}%
\pgfsys@useobject{currentmarker}{}%
\end{pgfscope}%
\end{pgfscope}%
\begin{pgfscope}%
\pgfsetrectcap%
\pgfsetmiterjoin%
\pgfsetlinewidth{0.000000pt}%
\definecolor{currentstroke}{rgb}{1.000000,1.000000,1.000000}%
\pgfsetstrokecolor{currentstroke}%
\pgfsetdash{}{0pt}%
\pgfpathmoveto{\pgfqpoint{0.750000in}{0.375000in}}%
\pgfpathlineto{\pgfqpoint{2.863636in}{0.375000in}}%
\pgfusepath{}%
\end{pgfscope}%
\begin{pgfscope}%
\pgfsetrectcap%
\pgfsetmiterjoin%
\pgfsetlinewidth{0.000000pt}%
\definecolor{currentstroke}{rgb}{1.000000,1.000000,1.000000}%
\pgfsetstrokecolor{currentstroke}%
\pgfsetdash{}{0pt}%
\pgfpathmoveto{\pgfqpoint{0.750000in}{0.375000in}}%
\pgfpathlineto{\pgfqpoint{0.750000in}{1.305000in}}%
\pgfusepath{}%
\end{pgfscope}%
\begin{pgfscope}%
\pgfsetrectcap%
\pgfsetmiterjoin%
\pgfsetlinewidth{0.000000pt}%
\definecolor{currentstroke}{rgb}{1.000000,1.000000,1.000000}%
\pgfsetstrokecolor{currentstroke}%
\pgfsetdash{}{0pt}%
\pgfpathmoveto{\pgfqpoint{0.750000in}{1.305000in}}%
\pgfpathlineto{\pgfqpoint{2.863636in}{1.305000in}}%
\pgfusepath{}%
\end{pgfscope}%
\begin{pgfscope}%
\pgfsetrectcap%
\pgfsetmiterjoin%
\pgfsetlinewidth{0.000000pt}%
\definecolor{currentstroke}{rgb}{1.000000,1.000000,1.000000}%
\pgfsetstrokecolor{currentstroke}%
\pgfsetdash{}{0pt}%
\pgfpathmoveto{\pgfqpoint{2.863636in}{0.375000in}}%
\pgfpathlineto{\pgfqpoint{2.863636in}{1.305000in}}%
\pgfusepath{}%
\end{pgfscope}%
\begin{pgfscope}%
\pgfsetbuttcap%
\pgfsetmiterjoin%
\definecolor{currentfill}{rgb}{0.917647,0.917647,0.949020}%
\pgfsetfillcolor{currentfill}%
\pgfsetlinewidth{0.000000pt}%
\definecolor{currentstroke}{rgb}{0.000000,0.000000,0.000000}%
\pgfsetstrokecolor{currentstroke}%
\pgfsetstrokeopacity{0.000000}%
\pgfsetdash{}{0pt}%
\pgfpathmoveto{\pgfqpoint{3.286364in}{0.375000in}}%
\pgfpathlineto{\pgfqpoint{5.400000in}{0.375000in}}%
\pgfpathlineto{\pgfqpoint{5.400000in}{1.305000in}}%
\pgfpathlineto{\pgfqpoint{3.286364in}{1.305000in}}%
\pgfpathclose%
\pgfusepath{fill}%
\end{pgfscope}%
\begin{pgfscope}%
\pgfpathrectangle{\pgfqpoint{3.286364in}{0.375000in}}{\pgfqpoint{2.113636in}{0.930000in}} %
\pgfusepath{clip}%
\pgfsetroundcap%
\pgfsetroundjoin%
\pgfsetlinewidth{0.803000pt}%
\definecolor{currentstroke}{rgb}{1.000000,1.000000,1.000000}%
\pgfsetstrokecolor{currentstroke}%
\pgfsetdash{}{0pt}%
\pgfpathmoveto{\pgfqpoint{3.582273in}{0.375000in}}%
\pgfpathlineto{\pgfqpoint{3.582273in}{1.305000in}}%
\pgfusepath{stroke}%
\end{pgfscope}%
\begin{pgfscope}%
\pgfsetbuttcap%
\pgfsetroundjoin%
\definecolor{currentfill}{rgb}{0.150000,0.150000,0.150000}%
\pgfsetfillcolor{currentfill}%
\pgfsetlinewidth{0.803000pt}%
\definecolor{currentstroke}{rgb}{0.150000,0.150000,0.150000}%
\pgfsetstrokecolor{currentstroke}%
\pgfsetdash{}{0pt}%
\pgfsys@defobject{currentmarker}{\pgfqpoint{0.000000in}{0.000000in}}{\pgfqpoint{0.000000in}{0.000000in}}{%
\pgfpathmoveto{\pgfqpoint{0.000000in}{0.000000in}}%
\pgfpathlineto{\pgfqpoint{0.000000in}{0.000000in}}%
\pgfusepath{stroke,fill}%
}%
\begin{pgfscope}%
\pgfsys@transformshift{3.582273in}{0.375000in}%
\pgfsys@useobject{currentmarker}{}%
\end{pgfscope}%
\end{pgfscope}%
\begin{pgfscope}%
\pgfsetbuttcap%
\pgfsetroundjoin%
\definecolor{currentfill}{rgb}{0.150000,0.150000,0.150000}%
\pgfsetfillcolor{currentfill}%
\pgfsetlinewidth{0.803000pt}%
\definecolor{currentstroke}{rgb}{0.150000,0.150000,0.150000}%
\pgfsetstrokecolor{currentstroke}%
\pgfsetdash{}{0pt}%
\pgfsys@defobject{currentmarker}{\pgfqpoint{0.000000in}{0.000000in}}{\pgfqpoint{0.000000in}{0.000000in}}{%
\pgfpathmoveto{\pgfqpoint{0.000000in}{0.000000in}}%
\pgfpathlineto{\pgfqpoint{0.000000in}{0.000000in}}%
\pgfusepath{stroke,fill}%
}%
\begin{pgfscope}%
\pgfsys@transformshift{3.582273in}{1.305000in}%
\pgfsys@useobject{currentmarker}{}%
\end{pgfscope}%
\end{pgfscope}%
\begin{pgfscope}%
\definecolor{textcolor}{rgb}{0.150000,0.150000,0.150000}%
\pgfsetstrokecolor{textcolor}%
\pgfsetfillcolor{textcolor}%
\pgftext[x=3.582273in,y=0.297222in,,top]{\color{textcolor}\sffamily\fontsize{8.000000}{9.600000}\selectfont 8}%
\end{pgfscope}%
\begin{pgfscope}%
\pgfpathrectangle{\pgfqpoint{3.286364in}{0.375000in}}{\pgfqpoint{2.113636in}{0.930000in}} %
\pgfusepath{clip}%
\pgfsetroundcap%
\pgfsetroundjoin%
\pgfsetlinewidth{0.803000pt}%
\definecolor{currentstroke}{rgb}{1.000000,1.000000,1.000000}%
\pgfsetstrokecolor{currentstroke}%
\pgfsetdash{}{0pt}%
\pgfpathmoveto{\pgfqpoint{4.005000in}{0.375000in}}%
\pgfpathlineto{\pgfqpoint{4.005000in}{1.305000in}}%
\pgfusepath{stroke}%
\end{pgfscope}%
\begin{pgfscope}%
\pgfsetbuttcap%
\pgfsetroundjoin%
\definecolor{currentfill}{rgb}{0.150000,0.150000,0.150000}%
\pgfsetfillcolor{currentfill}%
\pgfsetlinewidth{0.803000pt}%
\definecolor{currentstroke}{rgb}{0.150000,0.150000,0.150000}%
\pgfsetstrokecolor{currentstroke}%
\pgfsetdash{}{0pt}%
\pgfsys@defobject{currentmarker}{\pgfqpoint{0.000000in}{0.000000in}}{\pgfqpoint{0.000000in}{0.000000in}}{%
\pgfpathmoveto{\pgfqpoint{0.000000in}{0.000000in}}%
\pgfpathlineto{\pgfqpoint{0.000000in}{0.000000in}}%
\pgfusepath{stroke,fill}%
}%
\begin{pgfscope}%
\pgfsys@transformshift{4.005000in}{0.375000in}%
\pgfsys@useobject{currentmarker}{}%
\end{pgfscope}%
\end{pgfscope}%
\begin{pgfscope}%
\pgfsetbuttcap%
\pgfsetroundjoin%
\definecolor{currentfill}{rgb}{0.150000,0.150000,0.150000}%
\pgfsetfillcolor{currentfill}%
\pgfsetlinewidth{0.803000pt}%
\definecolor{currentstroke}{rgb}{0.150000,0.150000,0.150000}%
\pgfsetstrokecolor{currentstroke}%
\pgfsetdash{}{0pt}%
\pgfsys@defobject{currentmarker}{\pgfqpoint{0.000000in}{0.000000in}}{\pgfqpoint{0.000000in}{0.000000in}}{%
\pgfpathmoveto{\pgfqpoint{0.000000in}{0.000000in}}%
\pgfpathlineto{\pgfqpoint{0.000000in}{0.000000in}}%
\pgfusepath{stroke,fill}%
}%
\begin{pgfscope}%
\pgfsys@transformshift{4.005000in}{1.305000in}%
\pgfsys@useobject{currentmarker}{}%
\end{pgfscope}%
\end{pgfscope}%
\begin{pgfscope}%
\definecolor{textcolor}{rgb}{0.150000,0.150000,0.150000}%
\pgfsetstrokecolor{textcolor}%
\pgfsetfillcolor{textcolor}%
\pgftext[x=4.005000in,y=0.297222in,,top]{\color{textcolor}\sffamily\fontsize{8.000000}{9.600000}\selectfont 9}%
\end{pgfscope}%
\begin{pgfscope}%
\pgfpathrectangle{\pgfqpoint{3.286364in}{0.375000in}}{\pgfqpoint{2.113636in}{0.930000in}} %
\pgfusepath{clip}%
\pgfsetroundcap%
\pgfsetroundjoin%
\pgfsetlinewidth{0.803000pt}%
\definecolor{currentstroke}{rgb}{1.000000,1.000000,1.000000}%
\pgfsetstrokecolor{currentstroke}%
\pgfsetdash{}{0pt}%
\pgfpathmoveto{\pgfqpoint{4.427727in}{0.375000in}}%
\pgfpathlineto{\pgfqpoint{4.427727in}{1.305000in}}%
\pgfusepath{stroke}%
\end{pgfscope}%
\begin{pgfscope}%
\pgfsetbuttcap%
\pgfsetroundjoin%
\definecolor{currentfill}{rgb}{0.150000,0.150000,0.150000}%
\pgfsetfillcolor{currentfill}%
\pgfsetlinewidth{0.803000pt}%
\definecolor{currentstroke}{rgb}{0.150000,0.150000,0.150000}%
\pgfsetstrokecolor{currentstroke}%
\pgfsetdash{}{0pt}%
\pgfsys@defobject{currentmarker}{\pgfqpoint{0.000000in}{0.000000in}}{\pgfqpoint{0.000000in}{0.000000in}}{%
\pgfpathmoveto{\pgfqpoint{0.000000in}{0.000000in}}%
\pgfpathlineto{\pgfqpoint{0.000000in}{0.000000in}}%
\pgfusepath{stroke,fill}%
}%
\begin{pgfscope}%
\pgfsys@transformshift{4.427727in}{0.375000in}%
\pgfsys@useobject{currentmarker}{}%
\end{pgfscope}%
\end{pgfscope}%
\begin{pgfscope}%
\pgfsetbuttcap%
\pgfsetroundjoin%
\definecolor{currentfill}{rgb}{0.150000,0.150000,0.150000}%
\pgfsetfillcolor{currentfill}%
\pgfsetlinewidth{0.803000pt}%
\definecolor{currentstroke}{rgb}{0.150000,0.150000,0.150000}%
\pgfsetstrokecolor{currentstroke}%
\pgfsetdash{}{0pt}%
\pgfsys@defobject{currentmarker}{\pgfqpoint{0.000000in}{0.000000in}}{\pgfqpoint{0.000000in}{0.000000in}}{%
\pgfpathmoveto{\pgfqpoint{0.000000in}{0.000000in}}%
\pgfpathlineto{\pgfqpoint{0.000000in}{0.000000in}}%
\pgfusepath{stroke,fill}%
}%
\begin{pgfscope}%
\pgfsys@transformshift{4.427727in}{1.305000in}%
\pgfsys@useobject{currentmarker}{}%
\end{pgfscope}%
\end{pgfscope}%
\begin{pgfscope}%
\definecolor{textcolor}{rgb}{0.150000,0.150000,0.150000}%
\pgfsetstrokecolor{textcolor}%
\pgfsetfillcolor{textcolor}%
\pgftext[x=4.427727in,y=0.297222in,,top]{\color{textcolor}\sffamily\fontsize{8.000000}{9.600000}\selectfont 10}%
\end{pgfscope}%
\begin{pgfscope}%
\pgfpathrectangle{\pgfqpoint{3.286364in}{0.375000in}}{\pgfqpoint{2.113636in}{0.930000in}} %
\pgfusepath{clip}%
\pgfsetroundcap%
\pgfsetroundjoin%
\pgfsetlinewidth{0.803000pt}%
\definecolor{currentstroke}{rgb}{1.000000,1.000000,1.000000}%
\pgfsetstrokecolor{currentstroke}%
\pgfsetdash{}{0pt}%
\pgfpathmoveto{\pgfqpoint{4.850455in}{0.375000in}}%
\pgfpathlineto{\pgfqpoint{4.850455in}{1.305000in}}%
\pgfusepath{stroke}%
\end{pgfscope}%
\begin{pgfscope}%
\pgfsetbuttcap%
\pgfsetroundjoin%
\definecolor{currentfill}{rgb}{0.150000,0.150000,0.150000}%
\pgfsetfillcolor{currentfill}%
\pgfsetlinewidth{0.803000pt}%
\definecolor{currentstroke}{rgb}{0.150000,0.150000,0.150000}%
\pgfsetstrokecolor{currentstroke}%
\pgfsetdash{}{0pt}%
\pgfsys@defobject{currentmarker}{\pgfqpoint{0.000000in}{0.000000in}}{\pgfqpoint{0.000000in}{0.000000in}}{%
\pgfpathmoveto{\pgfqpoint{0.000000in}{0.000000in}}%
\pgfpathlineto{\pgfqpoint{0.000000in}{0.000000in}}%
\pgfusepath{stroke,fill}%
}%
\begin{pgfscope}%
\pgfsys@transformshift{4.850455in}{0.375000in}%
\pgfsys@useobject{currentmarker}{}%
\end{pgfscope}%
\end{pgfscope}%
\begin{pgfscope}%
\pgfsetbuttcap%
\pgfsetroundjoin%
\definecolor{currentfill}{rgb}{0.150000,0.150000,0.150000}%
\pgfsetfillcolor{currentfill}%
\pgfsetlinewidth{0.803000pt}%
\definecolor{currentstroke}{rgb}{0.150000,0.150000,0.150000}%
\pgfsetstrokecolor{currentstroke}%
\pgfsetdash{}{0pt}%
\pgfsys@defobject{currentmarker}{\pgfqpoint{0.000000in}{0.000000in}}{\pgfqpoint{0.000000in}{0.000000in}}{%
\pgfpathmoveto{\pgfqpoint{0.000000in}{0.000000in}}%
\pgfpathlineto{\pgfqpoint{0.000000in}{0.000000in}}%
\pgfusepath{stroke,fill}%
}%
\begin{pgfscope}%
\pgfsys@transformshift{4.850455in}{1.305000in}%
\pgfsys@useobject{currentmarker}{}%
\end{pgfscope}%
\end{pgfscope}%
\begin{pgfscope}%
\definecolor{textcolor}{rgb}{0.150000,0.150000,0.150000}%
\pgfsetstrokecolor{textcolor}%
\pgfsetfillcolor{textcolor}%
\pgftext[x=4.850455in,y=0.297222in,,top]{\color{textcolor}\sffamily\fontsize{8.000000}{9.600000}\selectfont 11}%
\end{pgfscope}%
\begin{pgfscope}%
\pgfpathrectangle{\pgfqpoint{3.286364in}{0.375000in}}{\pgfqpoint{2.113636in}{0.930000in}} %
\pgfusepath{clip}%
\pgfsetroundcap%
\pgfsetroundjoin%
\pgfsetlinewidth{0.803000pt}%
\definecolor{currentstroke}{rgb}{1.000000,1.000000,1.000000}%
\pgfsetstrokecolor{currentstroke}%
\pgfsetdash{}{0pt}%
\pgfpathmoveto{\pgfqpoint{5.273182in}{0.375000in}}%
\pgfpathlineto{\pgfqpoint{5.273182in}{1.305000in}}%
\pgfusepath{stroke}%
\end{pgfscope}%
\begin{pgfscope}%
\pgfsetbuttcap%
\pgfsetroundjoin%
\definecolor{currentfill}{rgb}{0.150000,0.150000,0.150000}%
\pgfsetfillcolor{currentfill}%
\pgfsetlinewidth{0.803000pt}%
\definecolor{currentstroke}{rgb}{0.150000,0.150000,0.150000}%
\pgfsetstrokecolor{currentstroke}%
\pgfsetdash{}{0pt}%
\pgfsys@defobject{currentmarker}{\pgfqpoint{0.000000in}{0.000000in}}{\pgfqpoint{0.000000in}{0.000000in}}{%
\pgfpathmoveto{\pgfqpoint{0.000000in}{0.000000in}}%
\pgfpathlineto{\pgfqpoint{0.000000in}{0.000000in}}%
\pgfusepath{stroke,fill}%
}%
\begin{pgfscope}%
\pgfsys@transformshift{5.273182in}{0.375000in}%
\pgfsys@useobject{currentmarker}{}%
\end{pgfscope}%
\end{pgfscope}%
\begin{pgfscope}%
\pgfsetbuttcap%
\pgfsetroundjoin%
\definecolor{currentfill}{rgb}{0.150000,0.150000,0.150000}%
\pgfsetfillcolor{currentfill}%
\pgfsetlinewidth{0.803000pt}%
\definecolor{currentstroke}{rgb}{0.150000,0.150000,0.150000}%
\pgfsetstrokecolor{currentstroke}%
\pgfsetdash{}{0pt}%
\pgfsys@defobject{currentmarker}{\pgfqpoint{0.000000in}{0.000000in}}{\pgfqpoint{0.000000in}{0.000000in}}{%
\pgfpathmoveto{\pgfqpoint{0.000000in}{0.000000in}}%
\pgfpathlineto{\pgfqpoint{0.000000in}{0.000000in}}%
\pgfusepath{stroke,fill}%
}%
\begin{pgfscope}%
\pgfsys@transformshift{5.273182in}{1.305000in}%
\pgfsys@useobject{currentmarker}{}%
\end{pgfscope}%
\end{pgfscope}%
\begin{pgfscope}%
\definecolor{textcolor}{rgb}{0.150000,0.150000,0.150000}%
\pgfsetstrokecolor{textcolor}%
\pgfsetfillcolor{textcolor}%
\pgftext[x=5.273182in,y=0.297222in,,top]{\color{textcolor}\sffamily\fontsize{8.000000}{9.600000}\selectfont 12}%
\end{pgfscope}%
\begin{pgfscope}%
\definecolor{textcolor}{rgb}{0.150000,0.150000,0.150000}%
\pgfsetstrokecolor{textcolor}%
\pgfsetfillcolor{textcolor}%
\pgftext[x=4.343182in,y=0.132099in,,top]{\color{textcolor}\sffamily\fontsize{8.800000}{10.560000}\selectfont Arm length}%
\end{pgfscope}%
\begin{pgfscope}%
\pgfpathrectangle{\pgfqpoint{3.286364in}{0.375000in}}{\pgfqpoint{2.113636in}{0.930000in}} %
\pgfusepath{clip}%
\pgfsetroundcap%
\pgfsetroundjoin%
\pgfsetlinewidth{0.803000pt}%
\definecolor{currentstroke}{rgb}{1.000000,1.000000,1.000000}%
\pgfsetstrokecolor{currentstroke}%
\pgfsetdash{}{0pt}%
\pgfpathmoveto{\pgfqpoint{3.286364in}{0.375000in}}%
\pgfpathlineto{\pgfqpoint{5.400000in}{0.375000in}}%
\pgfusepath{stroke}%
\end{pgfscope}%
\begin{pgfscope}%
\pgfsetbuttcap%
\pgfsetroundjoin%
\definecolor{currentfill}{rgb}{0.150000,0.150000,0.150000}%
\pgfsetfillcolor{currentfill}%
\pgfsetlinewidth{0.803000pt}%
\definecolor{currentstroke}{rgb}{0.150000,0.150000,0.150000}%
\pgfsetstrokecolor{currentstroke}%
\pgfsetdash{}{0pt}%
\pgfsys@defobject{currentmarker}{\pgfqpoint{0.000000in}{0.000000in}}{\pgfqpoint{0.000000in}{0.000000in}}{%
\pgfpathmoveto{\pgfqpoint{0.000000in}{0.000000in}}%
\pgfpathlineto{\pgfqpoint{0.000000in}{0.000000in}}%
\pgfusepath{stroke,fill}%
}%
\begin{pgfscope}%
\pgfsys@transformshift{3.286364in}{0.375000in}%
\pgfsys@useobject{currentmarker}{}%
\end{pgfscope}%
\end{pgfscope}%
\begin{pgfscope}%
\pgfsetbuttcap%
\pgfsetroundjoin%
\definecolor{currentfill}{rgb}{0.150000,0.150000,0.150000}%
\pgfsetfillcolor{currentfill}%
\pgfsetlinewidth{0.803000pt}%
\definecolor{currentstroke}{rgb}{0.150000,0.150000,0.150000}%
\pgfsetstrokecolor{currentstroke}%
\pgfsetdash{}{0pt}%
\pgfsys@defobject{currentmarker}{\pgfqpoint{0.000000in}{0.000000in}}{\pgfqpoint{0.000000in}{0.000000in}}{%
\pgfpathmoveto{\pgfqpoint{0.000000in}{0.000000in}}%
\pgfpathlineto{\pgfqpoint{0.000000in}{0.000000in}}%
\pgfusepath{stroke,fill}%
}%
\begin{pgfscope}%
\pgfsys@transformshift{5.400000in}{0.375000in}%
\pgfsys@useobject{currentmarker}{}%
\end{pgfscope}%
\end{pgfscope}%
\begin{pgfscope}%
\pgfpathrectangle{\pgfqpoint{3.286364in}{0.375000in}}{\pgfqpoint{2.113636in}{0.930000in}} %
\pgfusepath{clip}%
\pgfsetroundcap%
\pgfsetroundjoin%
\pgfsetlinewidth{0.803000pt}%
\definecolor{currentstroke}{rgb}{1.000000,1.000000,1.000000}%
\pgfsetstrokecolor{currentstroke}%
\pgfsetdash{}{0pt}%
\pgfpathmoveto{\pgfqpoint{3.286364in}{0.530000in}}%
\pgfpathlineto{\pgfqpoint{5.400000in}{0.530000in}}%
\pgfusepath{stroke}%
\end{pgfscope}%
\begin{pgfscope}%
\pgfsetbuttcap%
\pgfsetroundjoin%
\definecolor{currentfill}{rgb}{0.150000,0.150000,0.150000}%
\pgfsetfillcolor{currentfill}%
\pgfsetlinewidth{0.803000pt}%
\definecolor{currentstroke}{rgb}{0.150000,0.150000,0.150000}%
\pgfsetstrokecolor{currentstroke}%
\pgfsetdash{}{0pt}%
\pgfsys@defobject{currentmarker}{\pgfqpoint{0.000000in}{0.000000in}}{\pgfqpoint{0.000000in}{0.000000in}}{%
\pgfpathmoveto{\pgfqpoint{0.000000in}{0.000000in}}%
\pgfpathlineto{\pgfqpoint{0.000000in}{0.000000in}}%
\pgfusepath{stroke,fill}%
}%
\begin{pgfscope}%
\pgfsys@transformshift{3.286364in}{0.530000in}%
\pgfsys@useobject{currentmarker}{}%
\end{pgfscope}%
\end{pgfscope}%
\begin{pgfscope}%
\pgfsetbuttcap%
\pgfsetroundjoin%
\definecolor{currentfill}{rgb}{0.150000,0.150000,0.150000}%
\pgfsetfillcolor{currentfill}%
\pgfsetlinewidth{0.803000pt}%
\definecolor{currentstroke}{rgb}{0.150000,0.150000,0.150000}%
\pgfsetstrokecolor{currentstroke}%
\pgfsetdash{}{0pt}%
\pgfsys@defobject{currentmarker}{\pgfqpoint{0.000000in}{0.000000in}}{\pgfqpoint{0.000000in}{0.000000in}}{%
\pgfpathmoveto{\pgfqpoint{0.000000in}{0.000000in}}%
\pgfpathlineto{\pgfqpoint{0.000000in}{0.000000in}}%
\pgfusepath{stroke,fill}%
}%
\begin{pgfscope}%
\pgfsys@transformshift{5.400000in}{0.530000in}%
\pgfsys@useobject{currentmarker}{}%
\end{pgfscope}%
\end{pgfscope}%
\begin{pgfscope}%
\pgfpathrectangle{\pgfqpoint{3.286364in}{0.375000in}}{\pgfqpoint{2.113636in}{0.930000in}} %
\pgfusepath{clip}%
\pgfsetroundcap%
\pgfsetroundjoin%
\pgfsetlinewidth{0.803000pt}%
\definecolor{currentstroke}{rgb}{1.000000,1.000000,1.000000}%
\pgfsetstrokecolor{currentstroke}%
\pgfsetdash{}{0pt}%
\pgfpathmoveto{\pgfqpoint{3.286364in}{0.685000in}}%
\pgfpathlineto{\pgfqpoint{5.400000in}{0.685000in}}%
\pgfusepath{stroke}%
\end{pgfscope}%
\begin{pgfscope}%
\pgfsetbuttcap%
\pgfsetroundjoin%
\definecolor{currentfill}{rgb}{0.150000,0.150000,0.150000}%
\pgfsetfillcolor{currentfill}%
\pgfsetlinewidth{0.803000pt}%
\definecolor{currentstroke}{rgb}{0.150000,0.150000,0.150000}%
\pgfsetstrokecolor{currentstroke}%
\pgfsetdash{}{0pt}%
\pgfsys@defobject{currentmarker}{\pgfqpoint{0.000000in}{0.000000in}}{\pgfqpoint{0.000000in}{0.000000in}}{%
\pgfpathmoveto{\pgfqpoint{0.000000in}{0.000000in}}%
\pgfpathlineto{\pgfqpoint{0.000000in}{0.000000in}}%
\pgfusepath{stroke,fill}%
}%
\begin{pgfscope}%
\pgfsys@transformshift{3.286364in}{0.685000in}%
\pgfsys@useobject{currentmarker}{}%
\end{pgfscope}%
\end{pgfscope}%
\begin{pgfscope}%
\pgfsetbuttcap%
\pgfsetroundjoin%
\definecolor{currentfill}{rgb}{0.150000,0.150000,0.150000}%
\pgfsetfillcolor{currentfill}%
\pgfsetlinewidth{0.803000pt}%
\definecolor{currentstroke}{rgb}{0.150000,0.150000,0.150000}%
\pgfsetstrokecolor{currentstroke}%
\pgfsetdash{}{0pt}%
\pgfsys@defobject{currentmarker}{\pgfqpoint{0.000000in}{0.000000in}}{\pgfqpoint{0.000000in}{0.000000in}}{%
\pgfpathmoveto{\pgfqpoint{0.000000in}{0.000000in}}%
\pgfpathlineto{\pgfqpoint{0.000000in}{0.000000in}}%
\pgfusepath{stroke,fill}%
}%
\begin{pgfscope}%
\pgfsys@transformshift{5.400000in}{0.685000in}%
\pgfsys@useobject{currentmarker}{}%
\end{pgfscope}%
\end{pgfscope}%
\begin{pgfscope}%
\pgfpathrectangle{\pgfqpoint{3.286364in}{0.375000in}}{\pgfqpoint{2.113636in}{0.930000in}} %
\pgfusepath{clip}%
\pgfsetroundcap%
\pgfsetroundjoin%
\pgfsetlinewidth{0.803000pt}%
\definecolor{currentstroke}{rgb}{1.000000,1.000000,1.000000}%
\pgfsetstrokecolor{currentstroke}%
\pgfsetdash{}{0pt}%
\pgfpathmoveto{\pgfqpoint{3.286364in}{0.840000in}}%
\pgfpathlineto{\pgfqpoint{5.400000in}{0.840000in}}%
\pgfusepath{stroke}%
\end{pgfscope}%
\begin{pgfscope}%
\pgfsetbuttcap%
\pgfsetroundjoin%
\definecolor{currentfill}{rgb}{0.150000,0.150000,0.150000}%
\pgfsetfillcolor{currentfill}%
\pgfsetlinewidth{0.803000pt}%
\definecolor{currentstroke}{rgb}{0.150000,0.150000,0.150000}%
\pgfsetstrokecolor{currentstroke}%
\pgfsetdash{}{0pt}%
\pgfsys@defobject{currentmarker}{\pgfqpoint{0.000000in}{0.000000in}}{\pgfqpoint{0.000000in}{0.000000in}}{%
\pgfpathmoveto{\pgfqpoint{0.000000in}{0.000000in}}%
\pgfpathlineto{\pgfqpoint{0.000000in}{0.000000in}}%
\pgfusepath{stroke,fill}%
}%
\begin{pgfscope}%
\pgfsys@transformshift{3.286364in}{0.840000in}%
\pgfsys@useobject{currentmarker}{}%
\end{pgfscope}%
\end{pgfscope}%
\begin{pgfscope}%
\pgfsetbuttcap%
\pgfsetroundjoin%
\definecolor{currentfill}{rgb}{0.150000,0.150000,0.150000}%
\pgfsetfillcolor{currentfill}%
\pgfsetlinewidth{0.803000pt}%
\definecolor{currentstroke}{rgb}{0.150000,0.150000,0.150000}%
\pgfsetstrokecolor{currentstroke}%
\pgfsetdash{}{0pt}%
\pgfsys@defobject{currentmarker}{\pgfqpoint{0.000000in}{0.000000in}}{\pgfqpoint{0.000000in}{0.000000in}}{%
\pgfpathmoveto{\pgfqpoint{0.000000in}{0.000000in}}%
\pgfpathlineto{\pgfqpoint{0.000000in}{0.000000in}}%
\pgfusepath{stroke,fill}%
}%
\begin{pgfscope}%
\pgfsys@transformshift{5.400000in}{0.840000in}%
\pgfsys@useobject{currentmarker}{}%
\end{pgfscope}%
\end{pgfscope}%
\begin{pgfscope}%
\pgfpathrectangle{\pgfqpoint{3.286364in}{0.375000in}}{\pgfqpoint{2.113636in}{0.930000in}} %
\pgfusepath{clip}%
\pgfsetroundcap%
\pgfsetroundjoin%
\pgfsetlinewidth{0.803000pt}%
\definecolor{currentstroke}{rgb}{1.000000,1.000000,1.000000}%
\pgfsetstrokecolor{currentstroke}%
\pgfsetdash{}{0pt}%
\pgfpathmoveto{\pgfqpoint{3.286364in}{0.995000in}}%
\pgfpathlineto{\pgfqpoint{5.400000in}{0.995000in}}%
\pgfusepath{stroke}%
\end{pgfscope}%
\begin{pgfscope}%
\pgfsetbuttcap%
\pgfsetroundjoin%
\definecolor{currentfill}{rgb}{0.150000,0.150000,0.150000}%
\pgfsetfillcolor{currentfill}%
\pgfsetlinewidth{0.803000pt}%
\definecolor{currentstroke}{rgb}{0.150000,0.150000,0.150000}%
\pgfsetstrokecolor{currentstroke}%
\pgfsetdash{}{0pt}%
\pgfsys@defobject{currentmarker}{\pgfqpoint{0.000000in}{0.000000in}}{\pgfqpoint{0.000000in}{0.000000in}}{%
\pgfpathmoveto{\pgfqpoint{0.000000in}{0.000000in}}%
\pgfpathlineto{\pgfqpoint{0.000000in}{0.000000in}}%
\pgfusepath{stroke,fill}%
}%
\begin{pgfscope}%
\pgfsys@transformshift{3.286364in}{0.995000in}%
\pgfsys@useobject{currentmarker}{}%
\end{pgfscope}%
\end{pgfscope}%
\begin{pgfscope}%
\pgfsetbuttcap%
\pgfsetroundjoin%
\definecolor{currentfill}{rgb}{0.150000,0.150000,0.150000}%
\pgfsetfillcolor{currentfill}%
\pgfsetlinewidth{0.803000pt}%
\definecolor{currentstroke}{rgb}{0.150000,0.150000,0.150000}%
\pgfsetstrokecolor{currentstroke}%
\pgfsetdash{}{0pt}%
\pgfsys@defobject{currentmarker}{\pgfqpoint{0.000000in}{0.000000in}}{\pgfqpoint{0.000000in}{0.000000in}}{%
\pgfpathmoveto{\pgfqpoint{0.000000in}{0.000000in}}%
\pgfpathlineto{\pgfqpoint{0.000000in}{0.000000in}}%
\pgfusepath{stroke,fill}%
}%
\begin{pgfscope}%
\pgfsys@transformshift{5.400000in}{0.995000in}%
\pgfsys@useobject{currentmarker}{}%
\end{pgfscope}%
\end{pgfscope}%
\begin{pgfscope}%
\pgfpathrectangle{\pgfqpoint{3.286364in}{0.375000in}}{\pgfqpoint{2.113636in}{0.930000in}} %
\pgfusepath{clip}%
\pgfsetroundcap%
\pgfsetroundjoin%
\pgfsetlinewidth{0.803000pt}%
\definecolor{currentstroke}{rgb}{1.000000,1.000000,1.000000}%
\pgfsetstrokecolor{currentstroke}%
\pgfsetdash{}{0pt}%
\pgfpathmoveto{\pgfqpoint{3.286364in}{1.150000in}}%
\pgfpathlineto{\pgfqpoint{5.400000in}{1.150000in}}%
\pgfusepath{stroke}%
\end{pgfscope}%
\begin{pgfscope}%
\pgfsetbuttcap%
\pgfsetroundjoin%
\definecolor{currentfill}{rgb}{0.150000,0.150000,0.150000}%
\pgfsetfillcolor{currentfill}%
\pgfsetlinewidth{0.803000pt}%
\definecolor{currentstroke}{rgb}{0.150000,0.150000,0.150000}%
\pgfsetstrokecolor{currentstroke}%
\pgfsetdash{}{0pt}%
\pgfsys@defobject{currentmarker}{\pgfqpoint{0.000000in}{0.000000in}}{\pgfqpoint{0.000000in}{0.000000in}}{%
\pgfpathmoveto{\pgfqpoint{0.000000in}{0.000000in}}%
\pgfpathlineto{\pgfqpoint{0.000000in}{0.000000in}}%
\pgfusepath{stroke,fill}%
}%
\begin{pgfscope}%
\pgfsys@transformshift{3.286364in}{1.150000in}%
\pgfsys@useobject{currentmarker}{}%
\end{pgfscope}%
\end{pgfscope}%
\begin{pgfscope}%
\pgfsetbuttcap%
\pgfsetroundjoin%
\definecolor{currentfill}{rgb}{0.150000,0.150000,0.150000}%
\pgfsetfillcolor{currentfill}%
\pgfsetlinewidth{0.803000pt}%
\definecolor{currentstroke}{rgb}{0.150000,0.150000,0.150000}%
\pgfsetstrokecolor{currentstroke}%
\pgfsetdash{}{0pt}%
\pgfsys@defobject{currentmarker}{\pgfqpoint{0.000000in}{0.000000in}}{\pgfqpoint{0.000000in}{0.000000in}}{%
\pgfpathmoveto{\pgfqpoint{0.000000in}{0.000000in}}%
\pgfpathlineto{\pgfqpoint{0.000000in}{0.000000in}}%
\pgfusepath{stroke,fill}%
}%
\begin{pgfscope}%
\pgfsys@transformshift{5.400000in}{1.150000in}%
\pgfsys@useobject{currentmarker}{}%
\end{pgfscope}%
\end{pgfscope}%
\begin{pgfscope}%
\pgfpathrectangle{\pgfqpoint{3.286364in}{0.375000in}}{\pgfqpoint{2.113636in}{0.930000in}} %
\pgfusepath{clip}%
\pgfsetroundcap%
\pgfsetroundjoin%
\pgfsetlinewidth{0.803000pt}%
\definecolor{currentstroke}{rgb}{1.000000,1.000000,1.000000}%
\pgfsetstrokecolor{currentstroke}%
\pgfsetdash{}{0pt}%
\pgfpathmoveto{\pgfqpoint{3.286364in}{1.305000in}}%
\pgfpathlineto{\pgfqpoint{5.400000in}{1.305000in}}%
\pgfusepath{stroke}%
\end{pgfscope}%
\begin{pgfscope}%
\pgfsetbuttcap%
\pgfsetroundjoin%
\definecolor{currentfill}{rgb}{0.150000,0.150000,0.150000}%
\pgfsetfillcolor{currentfill}%
\pgfsetlinewidth{0.803000pt}%
\definecolor{currentstroke}{rgb}{0.150000,0.150000,0.150000}%
\pgfsetstrokecolor{currentstroke}%
\pgfsetdash{}{0pt}%
\pgfsys@defobject{currentmarker}{\pgfqpoint{0.000000in}{0.000000in}}{\pgfqpoint{0.000000in}{0.000000in}}{%
\pgfpathmoveto{\pgfqpoint{0.000000in}{0.000000in}}%
\pgfpathlineto{\pgfqpoint{0.000000in}{0.000000in}}%
\pgfusepath{stroke,fill}%
}%
\begin{pgfscope}%
\pgfsys@transformshift{3.286364in}{1.305000in}%
\pgfsys@useobject{currentmarker}{}%
\end{pgfscope}%
\end{pgfscope}%
\begin{pgfscope}%
\pgfsetbuttcap%
\pgfsetroundjoin%
\definecolor{currentfill}{rgb}{0.150000,0.150000,0.150000}%
\pgfsetfillcolor{currentfill}%
\pgfsetlinewidth{0.803000pt}%
\definecolor{currentstroke}{rgb}{0.150000,0.150000,0.150000}%
\pgfsetstrokecolor{currentstroke}%
\pgfsetdash{}{0pt}%
\pgfsys@defobject{currentmarker}{\pgfqpoint{0.000000in}{0.000000in}}{\pgfqpoint{0.000000in}{0.000000in}}{%
\pgfpathmoveto{\pgfqpoint{0.000000in}{0.000000in}}%
\pgfpathlineto{\pgfqpoint{0.000000in}{0.000000in}}%
\pgfusepath{stroke,fill}%
}%
\begin{pgfscope}%
\pgfsys@transformshift{5.400000in}{1.305000in}%
\pgfsys@useobject{currentmarker}{}%
\end{pgfscope}%
\end{pgfscope}%
\begin{pgfscope}%
\pgfpathrectangle{\pgfqpoint{3.286364in}{0.375000in}}{\pgfqpoint{2.113636in}{0.930000in}} %
\pgfusepath{clip}%
\pgfsetbuttcap%
\pgfsetmiterjoin%
\definecolor{currentfill}{rgb}{0.447059,0.623529,0.811765}%
\pgfsetfillcolor{currentfill}%
\pgfsetfillopacity{0.300000}%
\pgfsetlinewidth{0.240900pt}%
\definecolor{currentstroke}{rgb}{0.447059,0.623529,0.811765}%
\pgfsetstrokecolor{currentstroke}%
\pgfsetstrokeopacity{0.300000}%
\pgfsetdash{}{0pt}%
\pgfpathmoveto{\pgfqpoint{3.370909in}{0.860378in}}%
\pgfpathlineto{\pgfqpoint{3.390551in}{0.834695in}}%
\pgfpathlineto{\pgfqpoint{3.410193in}{0.810998in}}%
\pgfpathlineto{\pgfqpoint{3.429835in}{0.789290in}}%
\pgfpathlineto{\pgfqpoint{3.449477in}{0.769560in}}%
\pgfpathlineto{\pgfqpoint{3.469118in}{0.751780in}}%
\pgfpathlineto{\pgfqpoint{3.488760in}{0.735902in}}%
\pgfpathlineto{\pgfqpoint{3.508402in}{0.721863in}}%
\pgfpathlineto{\pgfqpoint{3.528044in}{0.709582in}}%
\pgfpathlineto{\pgfqpoint{3.547686in}{0.698964in}}%
\pgfpathlineto{\pgfqpoint{3.567328in}{0.689911in}}%
\pgfpathlineto{\pgfqpoint{3.586970in}{0.682318in}}%
\pgfpathlineto{\pgfqpoint{3.606612in}{0.676082in}}%
\pgfpathlineto{\pgfqpoint{3.626253in}{0.671107in}}%
\pgfpathlineto{\pgfqpoint{3.645895in}{0.667301in}}%
\pgfpathlineto{\pgfqpoint{3.665537in}{0.664580in}}%
\pgfpathlineto{\pgfqpoint{3.685179in}{0.662869in}}%
\pgfpathlineto{\pgfqpoint{3.704821in}{0.662101in}}%
\pgfpathlineto{\pgfqpoint{3.724463in}{0.662215in}}%
\pgfpathlineto{\pgfqpoint{3.744105in}{0.663158in}}%
\pgfpathlineto{\pgfqpoint{3.763747in}{0.664882in}}%
\pgfpathlineto{\pgfqpoint{3.783388in}{0.667345in}}%
\pgfpathlineto{\pgfqpoint{3.803030in}{0.670507in}}%
\pgfpathlineto{\pgfqpoint{3.822672in}{0.674334in}}%
\pgfpathlineto{\pgfqpoint{3.842314in}{0.678795in}}%
\pgfpathlineto{\pgfqpoint{3.861956in}{0.683861in}}%
\pgfpathlineto{\pgfqpoint{3.881598in}{0.689506in}}%
\pgfpathlineto{\pgfqpoint{3.901240in}{0.695707in}}%
\pgfpathlineto{\pgfqpoint{3.920882in}{0.702443in}}%
\pgfpathlineto{\pgfqpoint{3.940523in}{0.709694in}}%
\pgfpathlineto{\pgfqpoint{3.960165in}{0.717444in}}%
\pgfpathlineto{\pgfqpoint{3.979807in}{0.725678in}}%
\pgfpathlineto{\pgfqpoint{3.999449in}{0.734384in}}%
\pgfpathlineto{\pgfqpoint{4.019091in}{0.743553in}}%
\pgfpathlineto{\pgfqpoint{4.038733in}{0.753177in}}%
\pgfpathlineto{\pgfqpoint{4.058375in}{0.763252in}}%
\pgfpathlineto{\pgfqpoint{4.078017in}{0.773776in}}%
\pgfpathlineto{\pgfqpoint{4.097658in}{0.784750in}}%
\pgfpathlineto{\pgfqpoint{4.117300in}{0.796176in}}%
\pgfpathlineto{\pgfqpoint{4.136942in}{0.808058in}}%
\pgfpathlineto{\pgfqpoint{4.156584in}{0.820399in}}%
\pgfpathlineto{\pgfqpoint{4.176226in}{0.833197in}}%
\pgfpathlineto{\pgfqpoint{4.195868in}{0.846448in}}%
\pgfpathlineto{\pgfqpoint{4.215510in}{0.860139in}}%
\pgfpathlineto{\pgfqpoint{4.235152in}{0.874247in}}%
\pgfpathlineto{\pgfqpoint{4.254793in}{0.888737in}}%
\pgfpathlineto{\pgfqpoint{4.274435in}{0.903561in}}%
\pgfpathlineto{\pgfqpoint{4.294077in}{0.918661in}}%
\pgfpathlineto{\pgfqpoint{4.313719in}{0.933968in}}%
\pgfpathlineto{\pgfqpoint{4.333361in}{0.949408in}}%
\pgfpathlineto{\pgfqpoint{4.353003in}{0.964901in}}%
\pgfpathlineto{\pgfqpoint{4.372645in}{0.980365in}}%
\pgfpathlineto{\pgfqpoint{4.392287in}{0.995721in}}%
\pgfpathlineto{\pgfqpoint{4.411928in}{1.010889in}}%
\pgfpathlineto{\pgfqpoint{4.431570in}{1.025793in}}%
\pgfpathlineto{\pgfqpoint{4.451212in}{1.040363in}}%
\pgfpathlineto{\pgfqpoint{4.470854in}{1.054529in}}%
\pgfpathlineto{\pgfqpoint{4.490496in}{1.068229in}}%
\pgfpathlineto{\pgfqpoint{4.510138in}{1.081403in}}%
\pgfpathlineto{\pgfqpoint{4.529780in}{1.093994in}}%
\pgfpathlineto{\pgfqpoint{4.549421in}{1.105953in}}%
\pgfpathlineto{\pgfqpoint{4.569063in}{1.117231in}}%
\pgfpathlineto{\pgfqpoint{4.588705in}{1.127785in}}%
\pgfpathlineto{\pgfqpoint{4.608347in}{1.137575in}}%
\pgfpathlineto{\pgfqpoint{4.627989in}{1.146567in}}%
\pgfpathlineto{\pgfqpoint{4.647631in}{1.154730in}}%
\pgfpathlineto{\pgfqpoint{4.667273in}{1.162036in}}%
\pgfpathlineto{\pgfqpoint{4.686915in}{1.168463in}}%
\pgfpathlineto{\pgfqpoint{4.706556in}{1.173996in}}%
\pgfpathlineto{\pgfqpoint{4.726198in}{1.178622in}}%
\pgfpathlineto{\pgfqpoint{4.745840in}{1.182336in}}%
\pgfpathlineto{\pgfqpoint{4.765482in}{1.185138in}}%
\pgfpathlineto{\pgfqpoint{4.785124in}{1.187036in}}%
\pgfpathlineto{\pgfqpoint{4.804766in}{1.188045in}}%
\pgfpathlineto{\pgfqpoint{4.824408in}{1.188189in}}%
\pgfpathlineto{\pgfqpoint{4.844050in}{1.187500in}}%
\pgfpathlineto{\pgfqpoint{4.863691in}{1.186019in}}%
\pgfpathlineto{\pgfqpoint{4.883333in}{1.183797in}}%
\pgfpathlineto{\pgfqpoint{4.902975in}{1.180893in}}%
\pgfpathlineto{\pgfqpoint{4.922617in}{1.177375in}}%
\pgfpathlineto{\pgfqpoint{4.942259in}{1.173316in}}%
\pgfpathlineto{\pgfqpoint{4.961901in}{1.168793in}}%
\pgfpathlineto{\pgfqpoint{4.981543in}{1.163882in}}%
\pgfpathlineto{\pgfqpoint{5.001185in}{1.158655in}}%
\pgfpathlineto{\pgfqpoint{5.020826in}{1.153177in}}%
\pgfpathlineto{\pgfqpoint{5.040468in}{1.147500in}}%
\pgfpathlineto{\pgfqpoint{5.060110in}{1.141665in}}%
\pgfpathlineto{\pgfqpoint{5.079752in}{1.135698in}}%
\pgfpathlineto{\pgfqpoint{5.099394in}{1.129614in}}%
\pgfpathlineto{\pgfqpoint{5.119036in}{1.123418in}}%
\pgfpathlineto{\pgfqpoint{5.138678in}{1.117104in}}%
\pgfpathlineto{\pgfqpoint{5.158320in}{1.110665in}}%
\pgfpathlineto{\pgfqpoint{5.177961in}{1.104088in}}%
\pgfpathlineto{\pgfqpoint{5.197603in}{1.097362in}}%
\pgfpathlineto{\pgfqpoint{5.217245in}{1.090473in}}%
\pgfpathlineto{\pgfqpoint{5.236887in}{1.083410in}}%
\pgfpathlineto{\pgfqpoint{5.256529in}{1.076165in}}%
\pgfpathlineto{\pgfqpoint{5.276171in}{1.068730in}}%
\pgfpathlineto{\pgfqpoint{5.295813in}{1.061102in}}%
\pgfpathlineto{\pgfqpoint{5.315455in}{1.053278in}}%
\pgfpathlineto{\pgfqpoint{5.315455in}{0.644630in}}%
\pgfpathlineto{\pgfqpoint{5.295813in}{0.675412in}}%
\pgfpathlineto{\pgfqpoint{5.276171in}{0.705131in}}%
\pgfpathlineto{\pgfqpoint{5.256529in}{0.733738in}}%
\pgfpathlineto{\pgfqpoint{5.236887in}{0.761184in}}%
\pgfpathlineto{\pgfqpoint{5.217245in}{0.787418in}}%
\pgfpathlineto{\pgfqpoint{5.197603in}{0.812391in}}%
\pgfpathlineto{\pgfqpoint{5.177961in}{0.836054in}}%
\pgfpathlineto{\pgfqpoint{5.158320in}{0.858359in}}%
\pgfpathlineto{\pgfqpoint{5.138678in}{0.879261in}}%
\pgfpathlineto{\pgfqpoint{5.119036in}{0.898719in}}%
\pgfpathlineto{\pgfqpoint{5.099394in}{0.916695in}}%
\pgfpathlineto{\pgfqpoint{5.079752in}{0.933162in}}%
\pgfpathlineto{\pgfqpoint{5.060110in}{0.948102in}}%
\pgfpathlineto{\pgfqpoint{5.040468in}{0.961511in}}%
\pgfpathlineto{\pgfqpoint{5.020826in}{0.973401in}}%
\pgfpathlineto{\pgfqpoint{5.001185in}{0.983798in}}%
\pgfpathlineto{\pgfqpoint{4.981543in}{0.992746in}}%
\pgfpathlineto{\pgfqpoint{4.961901in}{1.000304in}}%
\pgfpathlineto{\pgfqpoint{4.942259in}{1.006538in}}%
\pgfpathlineto{\pgfqpoint{4.922617in}{1.011526in}}%
\pgfpathlineto{\pgfqpoint{4.902975in}{1.015347in}}%
\pgfpathlineto{\pgfqpoint{4.883333in}{1.018080in}}%
\pgfpathlineto{\pgfqpoint{4.863691in}{1.019802in}}%
\pgfpathlineto{\pgfqpoint{4.844050in}{1.020583in}}%
\pgfpathlineto{\pgfqpoint{4.824408in}{1.020492in}}%
\pgfpathlineto{\pgfqpoint{4.804766in}{1.019586in}}%
\pgfpathlineto{\pgfqpoint{4.785124in}{1.017919in}}%
\pgfpathlineto{\pgfqpoint{4.765482in}{1.015540in}}%
\pgfpathlineto{\pgfqpoint{4.745840in}{1.012492in}}%
\pgfpathlineto{\pgfqpoint{4.726198in}{1.008812in}}%
\pgfpathlineto{\pgfqpoint{4.706556in}{1.004535in}}%
\pgfpathlineto{\pgfqpoint{4.686915in}{0.999690in}}%
\pgfpathlineto{\pgfqpoint{4.667273in}{0.994307in}}%
\pgfpathlineto{\pgfqpoint{4.647631in}{0.988410in}}%
\pgfpathlineto{\pgfqpoint{4.627989in}{0.982022in}}%
\pgfpathlineto{\pgfqpoint{4.608347in}{0.975163in}}%
\pgfpathlineto{\pgfqpoint{4.588705in}{0.967853in}}%
\pgfpathlineto{\pgfqpoint{4.569063in}{0.960108in}}%
\pgfpathlineto{\pgfqpoint{4.549421in}{0.951945in}}%
\pgfpathlineto{\pgfqpoint{4.529780in}{0.943377in}}%
\pgfpathlineto{\pgfqpoint{4.510138in}{0.934416in}}%
\pgfpathlineto{\pgfqpoint{4.490496in}{0.925072in}}%
\pgfpathlineto{\pgfqpoint{4.470854in}{0.915354in}}%
\pgfpathlineto{\pgfqpoint{4.451212in}{0.905267in}}%
\pgfpathlineto{\pgfqpoint{4.431570in}{0.894816in}}%
\pgfpathlineto{\pgfqpoint{4.411928in}{0.884001in}}%
\pgfpathlineto{\pgfqpoint{4.392287in}{0.872822in}}%
\pgfpathlineto{\pgfqpoint{4.372645in}{0.861276in}}%
\pgfpathlineto{\pgfqpoint{4.353003in}{0.849358in}}%
\pgfpathlineto{\pgfqpoint{4.333361in}{0.837061in}}%
\pgfpathlineto{\pgfqpoint{4.313719in}{0.824382in}}%
\pgfpathlineto{\pgfqpoint{4.294077in}{0.811317in}}%
\pgfpathlineto{\pgfqpoint{4.274435in}{0.797870in}}%
\pgfpathlineto{\pgfqpoint{4.254793in}{0.784052in}}%
\pgfpathlineto{\pgfqpoint{4.235152in}{0.769883in}}%
\pgfpathlineto{\pgfqpoint{4.215510in}{0.755397in}}%
\pgfpathlineto{\pgfqpoint{4.195868in}{0.740638in}}%
\pgfpathlineto{\pgfqpoint{4.176226in}{0.725665in}}%
\pgfpathlineto{\pgfqpoint{4.156584in}{0.710545in}}%
\pgfpathlineto{\pgfqpoint{4.136942in}{0.695354in}}%
\pgfpathlineto{\pgfqpoint{4.117300in}{0.680173in}}%
\pgfpathlineto{\pgfqpoint{4.097658in}{0.665085in}}%
\pgfpathlineto{\pgfqpoint{4.078017in}{0.650173in}}%
\pgfpathlineto{\pgfqpoint{4.058375in}{0.635520in}}%
\pgfpathlineto{\pgfqpoint{4.038733in}{0.621204in}}%
\pgfpathlineto{\pgfqpoint{4.019091in}{0.607302in}}%
\pgfpathlineto{\pgfqpoint{3.999449in}{0.593887in}}%
\pgfpathlineto{\pgfqpoint{3.979807in}{0.581028in}}%
\pgfpathlineto{\pgfqpoint{3.960165in}{0.568790in}}%
\pgfpathlineto{\pgfqpoint{3.940523in}{0.557236in}}%
\pgfpathlineto{\pgfqpoint{3.920882in}{0.546422in}}%
\pgfpathlineto{\pgfqpoint{3.901240in}{0.536403in}}%
\pgfpathlineto{\pgfqpoint{3.881598in}{0.527229in}}%
\pgfpathlineto{\pgfqpoint{3.861956in}{0.518946in}}%
\pgfpathlineto{\pgfqpoint{3.842314in}{0.511596in}}%
\pgfpathlineto{\pgfqpoint{3.822672in}{0.505218in}}%
\pgfpathlineto{\pgfqpoint{3.803030in}{0.499843in}}%
\pgfpathlineto{\pgfqpoint{3.783388in}{0.495500in}}%
\pgfpathlineto{\pgfqpoint{3.763747in}{0.492211in}}%
\pgfpathlineto{\pgfqpoint{3.744105in}{0.489994in}}%
\pgfpathlineto{\pgfqpoint{3.724463in}{0.488856in}}%
\pgfpathlineto{\pgfqpoint{3.704821in}{0.488802in}}%
\pgfpathlineto{\pgfqpoint{3.685179in}{0.489823in}}%
\pgfpathlineto{\pgfqpoint{3.665537in}{0.491906in}}%
\pgfpathlineto{\pgfqpoint{3.645895in}{0.495023in}}%
\pgfpathlineto{\pgfqpoint{3.626253in}{0.499140in}}%
\pgfpathlineto{\pgfqpoint{3.606612in}{0.504209in}}%
\pgfpathlineto{\pgfqpoint{3.586970in}{0.510173in}}%
\pgfpathlineto{\pgfqpoint{3.567328in}{0.516965in}}%
\pgfpathlineto{\pgfqpoint{3.547686in}{0.524509in}}%
\pgfpathlineto{\pgfqpoint{3.528044in}{0.532727in}}%
\pgfpathlineto{\pgfqpoint{3.508402in}{0.541540in}}%
\pgfpathlineto{\pgfqpoint{3.488760in}{0.550874in}}%
\pgfpathlineto{\pgfqpoint{3.469118in}{0.560662in}}%
\pgfpathlineto{\pgfqpoint{3.449477in}{0.570854in}}%
\pgfpathlineto{\pgfqpoint{3.429835in}{0.581411in}}%
\pgfpathlineto{\pgfqpoint{3.410193in}{0.592312in}}%
\pgfpathlineto{\pgfqpoint{3.390551in}{0.603548in}}%
\pgfpathlineto{\pgfqpoint{3.370909in}{0.615124in}}%
\pgfpathclose%
\pgfusepath{stroke,fill}%
\end{pgfscope}%
\begin{pgfscope}%
\pgfpathrectangle{\pgfqpoint{3.286364in}{0.375000in}}{\pgfqpoint{2.113636in}{0.930000in}} %
\pgfusepath{clip}%
\pgfsetroundcap%
\pgfsetroundjoin%
\pgfsetlinewidth{2.007500pt}%
\definecolor{currentstroke}{rgb}{0.125490,0.290196,0.529412}%
\pgfsetstrokecolor{currentstroke}%
\pgfsetdash{}{0pt}%
\pgfpathmoveto{\pgfqpoint{3.370909in}{0.737751in}}%
\pgfpathlineto{\pgfqpoint{3.390551in}{0.719122in}}%
\pgfpathlineto{\pgfqpoint{3.410193in}{0.701655in}}%
\pgfpathlineto{\pgfqpoint{3.429835in}{0.685351in}}%
\pgfpathlineto{\pgfqpoint{3.449477in}{0.670207in}}%
\pgfpathlineto{\pgfqpoint{3.469118in}{0.656221in}}%
\pgfpathlineto{\pgfqpoint{3.488760in}{0.643388in}}%
\pgfpathlineto{\pgfqpoint{3.508402in}{0.631702in}}%
\pgfpathlineto{\pgfqpoint{3.528044in}{0.621154in}}%
\pgfpathlineto{\pgfqpoint{3.547686in}{0.611737in}}%
\pgfpathlineto{\pgfqpoint{3.567328in}{0.603438in}}%
\pgfpathlineto{\pgfqpoint{3.586970in}{0.596246in}}%
\pgfpathlineto{\pgfqpoint{3.606612in}{0.590146in}}%
\pgfpathlineto{\pgfqpoint{3.626253in}{0.585124in}}%
\pgfpathlineto{\pgfqpoint{3.645895in}{0.581162in}}%
\pgfpathlineto{\pgfqpoint{3.665537in}{0.578243in}}%
\pgfpathlineto{\pgfqpoint{3.685179in}{0.576346in}}%
\pgfpathlineto{\pgfqpoint{3.704821in}{0.575451in}}%
\pgfpathlineto{\pgfqpoint{3.724463in}{0.575536in}}%
\pgfpathlineto{\pgfqpoint{3.744105in}{0.576576in}}%
\pgfpathlineto{\pgfqpoint{3.763747in}{0.578547in}}%
\pgfpathlineto{\pgfqpoint{3.783388in}{0.581422in}}%
\pgfpathlineto{\pgfqpoint{3.803030in}{0.585175in}}%
\pgfpathlineto{\pgfqpoint{3.822672in}{0.589776in}}%
\pgfpathlineto{\pgfqpoint{3.842314in}{0.595196in}}%
\pgfpathlineto{\pgfqpoint{3.861956in}{0.601404in}}%
\pgfpathlineto{\pgfqpoint{3.881598in}{0.608368in}}%
\pgfpathlineto{\pgfqpoint{3.901240in}{0.616055in}}%
\pgfpathlineto{\pgfqpoint{3.920882in}{0.624432in}}%
\pgfpathlineto{\pgfqpoint{3.940523in}{0.633465in}}%
\pgfpathlineto{\pgfqpoint{3.960165in}{0.643117in}}%
\pgfpathlineto{\pgfqpoint{3.979807in}{0.653353in}}%
\pgfpathlineto{\pgfqpoint{3.999449in}{0.664135in}}%
\pgfpathlineto{\pgfqpoint{4.019091in}{0.675427in}}%
\pgfpathlineto{\pgfqpoint{4.038733in}{0.687190in}}%
\pgfpathlineto{\pgfqpoint{4.058375in}{0.699386in}}%
\pgfpathlineto{\pgfqpoint{4.078017in}{0.711975in}}%
\pgfpathlineto{\pgfqpoint{4.097658in}{0.724918in}}%
\pgfpathlineto{\pgfqpoint{4.117300in}{0.738175in}}%
\pgfpathlineto{\pgfqpoint{4.136942in}{0.751706in}}%
\pgfpathlineto{\pgfqpoint{4.156584in}{0.765472in}}%
\pgfpathlineto{\pgfqpoint{4.176226in}{0.779431in}}%
\pgfpathlineto{\pgfqpoint{4.195868in}{0.793543in}}%
\pgfpathlineto{\pgfqpoint{4.215510in}{0.807768in}}%
\pgfpathlineto{\pgfqpoint{4.235152in}{0.822065in}}%
\pgfpathlineto{\pgfqpoint{4.254793in}{0.836394in}}%
\pgfpathlineto{\pgfqpoint{4.274435in}{0.850716in}}%
\pgfpathlineto{\pgfqpoint{4.294077in}{0.864989in}}%
\pgfpathlineto{\pgfqpoint{4.313719in}{0.879175in}}%
\pgfpathlineto{\pgfqpoint{4.333361in}{0.893235in}}%
\pgfpathlineto{\pgfqpoint{4.353003in}{0.907129in}}%
\pgfpathlineto{\pgfqpoint{4.372645in}{0.920821in}}%
\pgfpathlineto{\pgfqpoint{4.392287in}{0.934272in}}%
\pgfpathlineto{\pgfqpoint{4.411928in}{0.947445in}}%
\pgfpathlineto{\pgfqpoint{4.431570in}{0.960305in}}%
\pgfpathlineto{\pgfqpoint{4.451212in}{0.972815in}}%
\pgfpathlineto{\pgfqpoint{4.470854in}{0.984942in}}%
\pgfpathlineto{\pgfqpoint{4.490496in}{0.996651in}}%
\pgfpathlineto{\pgfqpoint{4.510138in}{1.007909in}}%
\pgfpathlineto{\pgfqpoint{4.529780in}{1.018686in}}%
\pgfpathlineto{\pgfqpoint{4.549421in}{1.028949in}}%
\pgfpathlineto{\pgfqpoint{4.569063in}{1.038670in}}%
\pgfpathlineto{\pgfqpoint{4.588705in}{1.047819in}}%
\pgfpathlineto{\pgfqpoint{4.608347in}{1.056369in}}%
\pgfpathlineto{\pgfqpoint{4.627989in}{1.064295in}}%
\pgfpathlineto{\pgfqpoint{4.647631in}{1.071570in}}%
\pgfpathlineto{\pgfqpoint{4.667273in}{1.078171in}}%
\pgfpathlineto{\pgfqpoint{4.686915in}{1.084077in}}%
\pgfpathlineto{\pgfqpoint{4.706556in}{1.089265in}}%
\pgfpathlineto{\pgfqpoint{4.726198in}{1.093717in}}%
\pgfpathlineto{\pgfqpoint{4.745840in}{1.097414in}}%
\pgfpathlineto{\pgfqpoint{4.765482in}{1.100339in}}%
\pgfpathlineto{\pgfqpoint{4.785124in}{1.102478in}}%
\pgfpathlineto{\pgfqpoint{4.804766in}{1.103816in}}%
\pgfpathlineto{\pgfqpoint{4.824408in}{1.104340in}}%
\pgfpathlineto{\pgfqpoint{4.844050in}{1.104042in}}%
\pgfpathlineto{\pgfqpoint{4.863691in}{1.102910in}}%
\pgfpathlineto{\pgfqpoint{4.883333in}{1.100938in}}%
\pgfpathlineto{\pgfqpoint{4.902975in}{1.098120in}}%
\pgfpathlineto{\pgfqpoint{4.922617in}{1.094451in}}%
\pgfpathlineto{\pgfqpoint{4.942259in}{1.089927in}}%
\pgfpathlineto{\pgfqpoint{4.961901in}{1.084548in}}%
\pgfpathlineto{\pgfqpoint{4.981543in}{1.078314in}}%
\pgfpathlineto{\pgfqpoint{5.001185in}{1.071227in}}%
\pgfpathlineto{\pgfqpoint{5.020826in}{1.063289in}}%
\pgfpathlineto{\pgfqpoint{5.040468in}{1.054506in}}%
\pgfpathlineto{\pgfqpoint{5.060110in}{1.044883in}}%
\pgfpathlineto{\pgfqpoint{5.079752in}{1.034430in}}%
\pgfpathlineto{\pgfqpoint{5.099394in}{1.023155in}}%
\pgfpathlineto{\pgfqpoint{5.119036in}{1.011068in}}%
\pgfpathlineto{\pgfqpoint{5.138678in}{0.998183in}}%
\pgfpathlineto{\pgfqpoint{5.158320in}{0.984512in}}%
\pgfpathlineto{\pgfqpoint{5.177961in}{0.970071in}}%
\pgfpathlineto{\pgfqpoint{5.197603in}{0.954876in}}%
\pgfpathlineto{\pgfqpoint{5.217245in}{0.938945in}}%
\pgfpathlineto{\pgfqpoint{5.236887in}{0.922297in}}%
\pgfpathlineto{\pgfqpoint{5.256529in}{0.904952in}}%
\pgfpathlineto{\pgfqpoint{5.276171in}{0.886931in}}%
\pgfpathlineto{\pgfqpoint{5.295813in}{0.868257in}}%
\pgfpathlineto{\pgfqpoint{5.315455in}{0.848954in}}%
\pgfusepath{stroke}%
\end{pgfscope}%
\begin{pgfscope}%
\pgfpathrectangle{\pgfqpoint{3.286364in}{0.375000in}}{\pgfqpoint{2.113636in}{0.930000in}} %
\pgfusepath{clip}%
\pgfsetroundcap%
\pgfsetroundjoin%
\pgfsetlinewidth{0.200750pt}%
\definecolor{currentstroke}{rgb}{0.125490,0.290196,0.529412}%
\pgfsetstrokecolor{currentstroke}%
\pgfsetdash{}{0pt}%
\pgfpathmoveto{\pgfqpoint{3.370909in}{0.860378in}}%
\pgfpathlineto{\pgfqpoint{3.390551in}{0.834695in}}%
\pgfpathlineto{\pgfqpoint{3.410193in}{0.810998in}}%
\pgfpathlineto{\pgfqpoint{3.429835in}{0.789290in}}%
\pgfpathlineto{\pgfqpoint{3.449477in}{0.769560in}}%
\pgfpathlineto{\pgfqpoint{3.469118in}{0.751780in}}%
\pgfpathlineto{\pgfqpoint{3.488760in}{0.735902in}}%
\pgfpathlineto{\pgfqpoint{3.508402in}{0.721863in}}%
\pgfpathlineto{\pgfqpoint{3.528044in}{0.709582in}}%
\pgfpathlineto{\pgfqpoint{3.547686in}{0.698964in}}%
\pgfpathlineto{\pgfqpoint{3.567328in}{0.689911in}}%
\pgfpathlineto{\pgfqpoint{3.586970in}{0.682318in}}%
\pgfpathlineto{\pgfqpoint{3.606612in}{0.676082in}}%
\pgfpathlineto{\pgfqpoint{3.626253in}{0.671107in}}%
\pgfpathlineto{\pgfqpoint{3.645895in}{0.667301in}}%
\pgfpathlineto{\pgfqpoint{3.665537in}{0.664580in}}%
\pgfpathlineto{\pgfqpoint{3.685179in}{0.662869in}}%
\pgfpathlineto{\pgfqpoint{3.704821in}{0.662101in}}%
\pgfpathlineto{\pgfqpoint{3.724463in}{0.662215in}}%
\pgfpathlineto{\pgfqpoint{3.744105in}{0.663158in}}%
\pgfpathlineto{\pgfqpoint{3.763747in}{0.664882in}}%
\pgfpathlineto{\pgfqpoint{3.783388in}{0.667345in}}%
\pgfpathlineto{\pgfqpoint{3.803030in}{0.670507in}}%
\pgfpathlineto{\pgfqpoint{3.822672in}{0.674334in}}%
\pgfpathlineto{\pgfqpoint{3.842314in}{0.678795in}}%
\pgfpathlineto{\pgfqpoint{3.861956in}{0.683861in}}%
\pgfpathlineto{\pgfqpoint{3.881598in}{0.689506in}}%
\pgfpathlineto{\pgfqpoint{3.901240in}{0.695707in}}%
\pgfpathlineto{\pgfqpoint{3.920882in}{0.702443in}}%
\pgfpathlineto{\pgfqpoint{3.940523in}{0.709694in}}%
\pgfpathlineto{\pgfqpoint{3.960165in}{0.717444in}}%
\pgfpathlineto{\pgfqpoint{3.979807in}{0.725678in}}%
\pgfpathlineto{\pgfqpoint{3.999449in}{0.734384in}}%
\pgfpathlineto{\pgfqpoint{4.019091in}{0.743553in}}%
\pgfpathlineto{\pgfqpoint{4.038733in}{0.753177in}}%
\pgfpathlineto{\pgfqpoint{4.058375in}{0.763252in}}%
\pgfpathlineto{\pgfqpoint{4.078017in}{0.773776in}}%
\pgfpathlineto{\pgfqpoint{4.097658in}{0.784750in}}%
\pgfpathlineto{\pgfqpoint{4.117300in}{0.796176in}}%
\pgfpathlineto{\pgfqpoint{4.136942in}{0.808058in}}%
\pgfpathlineto{\pgfqpoint{4.156584in}{0.820399in}}%
\pgfpathlineto{\pgfqpoint{4.176226in}{0.833197in}}%
\pgfpathlineto{\pgfqpoint{4.195868in}{0.846448in}}%
\pgfpathlineto{\pgfqpoint{4.215510in}{0.860139in}}%
\pgfpathlineto{\pgfqpoint{4.235152in}{0.874247in}}%
\pgfpathlineto{\pgfqpoint{4.254793in}{0.888737in}}%
\pgfpathlineto{\pgfqpoint{4.274435in}{0.903561in}}%
\pgfpathlineto{\pgfqpoint{4.294077in}{0.918661in}}%
\pgfpathlineto{\pgfqpoint{4.313719in}{0.933968in}}%
\pgfpathlineto{\pgfqpoint{4.333361in}{0.949408in}}%
\pgfpathlineto{\pgfqpoint{4.353003in}{0.964901in}}%
\pgfpathlineto{\pgfqpoint{4.372645in}{0.980365in}}%
\pgfpathlineto{\pgfqpoint{4.392287in}{0.995721in}}%
\pgfpathlineto{\pgfqpoint{4.411928in}{1.010889in}}%
\pgfpathlineto{\pgfqpoint{4.431570in}{1.025793in}}%
\pgfpathlineto{\pgfqpoint{4.451212in}{1.040363in}}%
\pgfpathlineto{\pgfqpoint{4.470854in}{1.054529in}}%
\pgfpathlineto{\pgfqpoint{4.490496in}{1.068229in}}%
\pgfpathlineto{\pgfqpoint{4.510138in}{1.081403in}}%
\pgfpathlineto{\pgfqpoint{4.529780in}{1.093994in}}%
\pgfpathlineto{\pgfqpoint{4.549421in}{1.105953in}}%
\pgfpathlineto{\pgfqpoint{4.569063in}{1.117231in}}%
\pgfpathlineto{\pgfqpoint{4.588705in}{1.127785in}}%
\pgfpathlineto{\pgfqpoint{4.608347in}{1.137575in}}%
\pgfpathlineto{\pgfqpoint{4.627989in}{1.146567in}}%
\pgfpathlineto{\pgfqpoint{4.647631in}{1.154730in}}%
\pgfpathlineto{\pgfqpoint{4.667273in}{1.162036in}}%
\pgfpathlineto{\pgfqpoint{4.686915in}{1.168463in}}%
\pgfpathlineto{\pgfqpoint{4.706556in}{1.173996in}}%
\pgfpathlineto{\pgfqpoint{4.726198in}{1.178622in}}%
\pgfpathlineto{\pgfqpoint{4.745840in}{1.182336in}}%
\pgfpathlineto{\pgfqpoint{4.765482in}{1.185138in}}%
\pgfpathlineto{\pgfqpoint{4.785124in}{1.187036in}}%
\pgfpathlineto{\pgfqpoint{4.804766in}{1.188045in}}%
\pgfpathlineto{\pgfqpoint{4.824408in}{1.188189in}}%
\pgfpathlineto{\pgfqpoint{4.844050in}{1.187500in}}%
\pgfpathlineto{\pgfqpoint{4.863691in}{1.186019in}}%
\pgfpathlineto{\pgfqpoint{4.883333in}{1.183797in}}%
\pgfpathlineto{\pgfqpoint{4.902975in}{1.180893in}}%
\pgfpathlineto{\pgfqpoint{4.922617in}{1.177375in}}%
\pgfpathlineto{\pgfqpoint{4.942259in}{1.173316in}}%
\pgfpathlineto{\pgfqpoint{4.961901in}{1.168793in}}%
\pgfpathlineto{\pgfqpoint{4.981543in}{1.163882in}}%
\pgfpathlineto{\pgfqpoint{5.001185in}{1.158655in}}%
\pgfpathlineto{\pgfqpoint{5.020826in}{1.153177in}}%
\pgfpathlineto{\pgfqpoint{5.040468in}{1.147500in}}%
\pgfpathlineto{\pgfqpoint{5.060110in}{1.141665in}}%
\pgfpathlineto{\pgfqpoint{5.079752in}{1.135698in}}%
\pgfpathlineto{\pgfqpoint{5.099394in}{1.129614in}}%
\pgfpathlineto{\pgfqpoint{5.119036in}{1.123418in}}%
\pgfpathlineto{\pgfqpoint{5.138678in}{1.117104in}}%
\pgfpathlineto{\pgfqpoint{5.158320in}{1.110665in}}%
\pgfpathlineto{\pgfqpoint{5.177961in}{1.104088in}}%
\pgfpathlineto{\pgfqpoint{5.197603in}{1.097362in}}%
\pgfpathlineto{\pgfqpoint{5.217245in}{1.090473in}}%
\pgfpathlineto{\pgfqpoint{5.236887in}{1.083410in}}%
\pgfpathlineto{\pgfqpoint{5.256529in}{1.076165in}}%
\pgfpathlineto{\pgfqpoint{5.276171in}{1.068730in}}%
\pgfpathlineto{\pgfqpoint{5.295813in}{1.061102in}}%
\pgfpathlineto{\pgfqpoint{5.315455in}{1.053278in}}%
\pgfusepath{stroke}%
\end{pgfscope}%
\begin{pgfscope}%
\pgfpathrectangle{\pgfqpoint{3.286364in}{0.375000in}}{\pgfqpoint{2.113636in}{0.930000in}} %
\pgfusepath{clip}%
\pgfsetroundcap%
\pgfsetroundjoin%
\pgfsetlinewidth{0.200750pt}%
\definecolor{currentstroke}{rgb}{0.125490,0.290196,0.529412}%
\pgfsetstrokecolor{currentstroke}%
\pgfsetdash{}{0pt}%
\pgfpathmoveto{\pgfqpoint{3.370909in}{0.615124in}}%
\pgfpathlineto{\pgfqpoint{3.390551in}{0.603548in}}%
\pgfpathlineto{\pgfqpoint{3.410193in}{0.592312in}}%
\pgfpathlineto{\pgfqpoint{3.429835in}{0.581411in}}%
\pgfpathlineto{\pgfqpoint{3.449477in}{0.570854in}}%
\pgfpathlineto{\pgfqpoint{3.469118in}{0.560662in}}%
\pgfpathlineto{\pgfqpoint{3.488760in}{0.550874in}}%
\pgfpathlineto{\pgfqpoint{3.508402in}{0.541540in}}%
\pgfpathlineto{\pgfqpoint{3.528044in}{0.532727in}}%
\pgfpathlineto{\pgfqpoint{3.547686in}{0.524509in}}%
\pgfpathlineto{\pgfqpoint{3.567328in}{0.516965in}}%
\pgfpathlineto{\pgfqpoint{3.586970in}{0.510173in}}%
\pgfpathlineto{\pgfqpoint{3.606612in}{0.504209in}}%
\pgfpathlineto{\pgfqpoint{3.626253in}{0.499140in}}%
\pgfpathlineto{\pgfqpoint{3.645895in}{0.495023in}}%
\pgfpathlineto{\pgfqpoint{3.665537in}{0.491906in}}%
\pgfpathlineto{\pgfqpoint{3.685179in}{0.489823in}}%
\pgfpathlineto{\pgfqpoint{3.704821in}{0.488802in}}%
\pgfpathlineto{\pgfqpoint{3.724463in}{0.488856in}}%
\pgfpathlineto{\pgfqpoint{3.744105in}{0.489994in}}%
\pgfpathlineto{\pgfqpoint{3.763747in}{0.492211in}}%
\pgfpathlineto{\pgfqpoint{3.783388in}{0.495500in}}%
\pgfpathlineto{\pgfqpoint{3.803030in}{0.499843in}}%
\pgfpathlineto{\pgfqpoint{3.822672in}{0.505218in}}%
\pgfpathlineto{\pgfqpoint{3.842314in}{0.511596in}}%
\pgfpathlineto{\pgfqpoint{3.861956in}{0.518946in}}%
\pgfpathlineto{\pgfqpoint{3.881598in}{0.527229in}}%
\pgfpathlineto{\pgfqpoint{3.901240in}{0.536403in}}%
\pgfpathlineto{\pgfqpoint{3.920882in}{0.546422in}}%
\pgfpathlineto{\pgfqpoint{3.940523in}{0.557236in}}%
\pgfpathlineto{\pgfqpoint{3.960165in}{0.568790in}}%
\pgfpathlineto{\pgfqpoint{3.979807in}{0.581028in}}%
\pgfpathlineto{\pgfqpoint{3.999449in}{0.593887in}}%
\pgfpathlineto{\pgfqpoint{4.019091in}{0.607302in}}%
\pgfpathlineto{\pgfqpoint{4.038733in}{0.621204in}}%
\pgfpathlineto{\pgfqpoint{4.058375in}{0.635520in}}%
\pgfpathlineto{\pgfqpoint{4.078017in}{0.650173in}}%
\pgfpathlineto{\pgfqpoint{4.097658in}{0.665085in}}%
\pgfpathlineto{\pgfqpoint{4.117300in}{0.680173in}}%
\pgfpathlineto{\pgfqpoint{4.136942in}{0.695354in}}%
\pgfpathlineto{\pgfqpoint{4.156584in}{0.710545in}}%
\pgfpathlineto{\pgfqpoint{4.176226in}{0.725665in}}%
\pgfpathlineto{\pgfqpoint{4.195868in}{0.740638in}}%
\pgfpathlineto{\pgfqpoint{4.215510in}{0.755397in}}%
\pgfpathlineto{\pgfqpoint{4.235152in}{0.769883in}}%
\pgfpathlineto{\pgfqpoint{4.254793in}{0.784052in}}%
\pgfpathlineto{\pgfqpoint{4.274435in}{0.797870in}}%
\pgfpathlineto{\pgfqpoint{4.294077in}{0.811317in}}%
\pgfpathlineto{\pgfqpoint{4.313719in}{0.824382in}}%
\pgfpathlineto{\pgfqpoint{4.333361in}{0.837061in}}%
\pgfpathlineto{\pgfqpoint{4.353003in}{0.849358in}}%
\pgfpathlineto{\pgfqpoint{4.372645in}{0.861276in}}%
\pgfpathlineto{\pgfqpoint{4.392287in}{0.872822in}}%
\pgfpathlineto{\pgfqpoint{4.411928in}{0.884001in}}%
\pgfpathlineto{\pgfqpoint{4.431570in}{0.894816in}}%
\pgfpathlineto{\pgfqpoint{4.451212in}{0.905267in}}%
\pgfpathlineto{\pgfqpoint{4.470854in}{0.915354in}}%
\pgfpathlineto{\pgfqpoint{4.490496in}{0.925072in}}%
\pgfpathlineto{\pgfqpoint{4.510138in}{0.934416in}}%
\pgfpathlineto{\pgfqpoint{4.529780in}{0.943377in}}%
\pgfpathlineto{\pgfqpoint{4.549421in}{0.951945in}}%
\pgfpathlineto{\pgfqpoint{4.569063in}{0.960108in}}%
\pgfpathlineto{\pgfqpoint{4.588705in}{0.967853in}}%
\pgfpathlineto{\pgfqpoint{4.608347in}{0.975163in}}%
\pgfpathlineto{\pgfqpoint{4.627989in}{0.982022in}}%
\pgfpathlineto{\pgfqpoint{4.647631in}{0.988410in}}%
\pgfpathlineto{\pgfqpoint{4.667273in}{0.994307in}}%
\pgfpathlineto{\pgfqpoint{4.686915in}{0.999690in}}%
\pgfpathlineto{\pgfqpoint{4.706556in}{1.004535in}}%
\pgfpathlineto{\pgfqpoint{4.726198in}{1.008812in}}%
\pgfpathlineto{\pgfqpoint{4.745840in}{1.012492in}}%
\pgfpathlineto{\pgfqpoint{4.765482in}{1.015540in}}%
\pgfpathlineto{\pgfqpoint{4.785124in}{1.017919in}}%
\pgfpathlineto{\pgfqpoint{4.804766in}{1.019586in}}%
\pgfpathlineto{\pgfqpoint{4.824408in}{1.020492in}}%
\pgfpathlineto{\pgfqpoint{4.844050in}{1.020583in}}%
\pgfpathlineto{\pgfqpoint{4.863691in}{1.019802in}}%
\pgfpathlineto{\pgfqpoint{4.883333in}{1.018080in}}%
\pgfpathlineto{\pgfqpoint{4.902975in}{1.015347in}}%
\pgfpathlineto{\pgfqpoint{4.922617in}{1.011526in}}%
\pgfpathlineto{\pgfqpoint{4.942259in}{1.006538in}}%
\pgfpathlineto{\pgfqpoint{4.961901in}{1.000304in}}%
\pgfpathlineto{\pgfqpoint{4.981543in}{0.992746in}}%
\pgfpathlineto{\pgfqpoint{5.001185in}{0.983798in}}%
\pgfpathlineto{\pgfqpoint{5.020826in}{0.973401in}}%
\pgfpathlineto{\pgfqpoint{5.040468in}{0.961511in}}%
\pgfpathlineto{\pgfqpoint{5.060110in}{0.948102in}}%
\pgfpathlineto{\pgfqpoint{5.079752in}{0.933162in}}%
\pgfpathlineto{\pgfqpoint{5.099394in}{0.916695in}}%
\pgfpathlineto{\pgfqpoint{5.119036in}{0.898719in}}%
\pgfpathlineto{\pgfqpoint{5.138678in}{0.879261in}}%
\pgfpathlineto{\pgfqpoint{5.158320in}{0.858359in}}%
\pgfpathlineto{\pgfqpoint{5.177961in}{0.836054in}}%
\pgfpathlineto{\pgfqpoint{5.197603in}{0.812391in}}%
\pgfpathlineto{\pgfqpoint{5.217245in}{0.787418in}}%
\pgfpathlineto{\pgfqpoint{5.236887in}{0.761184in}}%
\pgfpathlineto{\pgfqpoint{5.256529in}{0.733738in}}%
\pgfpathlineto{\pgfqpoint{5.276171in}{0.705131in}}%
\pgfpathlineto{\pgfqpoint{5.295813in}{0.675412in}}%
\pgfpathlineto{\pgfqpoint{5.315455in}{0.644630in}}%
\pgfusepath{stroke}%
\end{pgfscope}%
\begin{pgfscope}%
\pgfpathrectangle{\pgfqpoint{3.286364in}{0.375000in}}{\pgfqpoint{2.113636in}{0.930000in}} %
\pgfusepath{clip}%
\pgfsetbuttcap%
\pgfsetbeveljoin%
\definecolor{currentfill}{rgb}{0.298039,0.447059,0.690196}%
\pgfsetfillcolor{currentfill}%
\pgfsetlinewidth{0.000000pt}%
\definecolor{currentstroke}{rgb}{0.000000,0.000000,0.000000}%
\pgfsetstrokecolor{currentstroke}%
\pgfsetdash{}{0pt}%
\pgfsys@defobject{currentmarker}{\pgfqpoint{-0.036986in}{-0.031462in}}{\pgfqpoint{0.036986in}{0.038889in}}{%
\pgfpathmoveto{\pgfqpoint{0.000000in}{0.038889in}}%
\pgfpathlineto{\pgfqpoint{-0.008731in}{0.012017in}}%
\pgfpathlineto{\pgfqpoint{-0.036986in}{0.012017in}}%
\pgfpathlineto{\pgfqpoint{-0.014127in}{-0.004590in}}%
\pgfpathlineto{\pgfqpoint{-0.022858in}{-0.031462in}}%
\pgfpathlineto{\pgfqpoint{-0.000000in}{-0.014854in}}%
\pgfpathlineto{\pgfqpoint{0.022858in}{-0.031462in}}%
\pgfpathlineto{\pgfqpoint{0.014127in}{-0.004590in}}%
\pgfpathlineto{\pgfqpoint{0.036986in}{0.012017in}}%
\pgfpathlineto{\pgfqpoint{0.008731in}{0.012017in}}%
\pgfpathclose%
\pgfusepath{fill}%
}%
\begin{pgfscope}%
\pgfsys@transformshift{4.300909in}{0.948500in}%
\pgfsys@useobject{currentmarker}{}%
\end{pgfscope}%
\begin{pgfscope}%
\pgfsys@transformshift{4.977273in}{0.824500in}%
\pgfsys@useobject{currentmarker}{}%
\end{pgfscope}%
\begin{pgfscope}%
\pgfsys@transformshift{4.512273in}{0.933000in}%
\pgfsys@useobject{currentmarker}{}%
\end{pgfscope}%
\begin{pgfscope}%
\pgfsys@transformshift{5.061818in}{1.134500in}%
\pgfsys@useobject{currentmarker}{}%
\end{pgfscope}%
\begin{pgfscope}%
\pgfsys@transformshift{4.935000in}{0.809000in}%
\pgfsys@useobject{currentmarker}{}%
\end{pgfscope}%
\begin{pgfscope}%
\pgfsys@transformshift{5.188636in}{1.196500in}%
\pgfsys@useobject{currentmarker}{}%
\end{pgfscope}%
\begin{pgfscope}%
\pgfsys@transformshift{4.131818in}{0.809000in}%
\pgfsys@useobject{currentmarker}{}%
\end{pgfscope}%
\begin{pgfscope}%
\pgfsys@transformshift{4.850455in}{1.258500in}%
\pgfsys@useobject{currentmarker}{}%
\end{pgfscope}%
\begin{pgfscope}%
\pgfsys@transformshift{4.723636in}{0.545500in}%
\pgfsys@useobject{currentmarker}{}%
\end{pgfscope}%
\begin{pgfscope}%
\pgfsys@transformshift{4.512273in}{0.840000in}%
\pgfsys@useobject{currentmarker}{}%
\end{pgfscope}%
\begin{pgfscope}%
\pgfsys@transformshift{3.370909in}{0.793500in}%
\pgfsys@useobject{currentmarker}{}%
\end{pgfscope}%
\begin{pgfscope}%
\pgfsys@transformshift{5.019545in}{0.809000in}%
\pgfsys@useobject{currentmarker}{}%
\end{pgfscope}%
\begin{pgfscope}%
\pgfsys@transformshift{4.427727in}{0.855500in}%
\pgfsys@useobject{currentmarker}{}%
\end{pgfscope}%
\begin{pgfscope}%
\pgfsys@transformshift{3.455455in}{0.483500in}%
\pgfsys@useobject{currentmarker}{}%
\end{pgfscope}%
\begin{pgfscope}%
\pgfsys@transformshift{5.315455in}{1.026000in}%
\pgfsys@useobject{currentmarker}{}%
\end{pgfscope}%
\begin{pgfscope}%
\pgfsys@transformshift{4.554545in}{0.979500in}%
\pgfsys@useobject{currentmarker}{}%
\end{pgfscope}%
\begin{pgfscope}%
\pgfsys@transformshift{4.258636in}{0.731500in}%
\pgfsys@useobject{currentmarker}{}%
\end{pgfscope}%
\begin{pgfscope}%
\pgfsys@transformshift{3.370909in}{0.824500in}%
\pgfsys@useobject{currentmarker}{}%
\end{pgfscope}%
\begin{pgfscope}%
\pgfsys@transformshift{3.962727in}{0.607500in}%
\pgfsys@useobject{currentmarker}{}%
\end{pgfscope}%
\begin{pgfscope}%
\pgfsys@transformshift{4.089545in}{0.747000in}%
\pgfsys@useobject{currentmarker}{}%
\end{pgfscope}%
\begin{pgfscope}%
\pgfsys@transformshift{4.596818in}{0.824500in}%
\pgfsys@useobject{currentmarker}{}%
\end{pgfscope}%
\begin{pgfscope}%
\pgfsys@transformshift{3.751364in}{1.041500in}%
\pgfsys@useobject{currentmarker}{}%
\end{pgfscope}%
\begin{pgfscope}%
\pgfsys@transformshift{4.808182in}{0.871000in}%
\pgfsys@useobject{currentmarker}{}%
\end{pgfscope}%
\begin{pgfscope}%
\pgfsys@transformshift{3.709091in}{0.654000in}%
\pgfsys@useobject{currentmarker}{}%
\end{pgfscope}%
\begin{pgfscope}%
\pgfsys@transformshift{4.174091in}{0.917500in}%
\pgfsys@useobject{currentmarker}{}%
\end{pgfscope}%
\begin{pgfscope}%
\pgfsys@transformshift{4.470000in}{1.119000in}%
\pgfsys@useobject{currentmarker}{}%
\end{pgfscope}%
\begin{pgfscope}%
\pgfsys@transformshift{4.089545in}{0.669500in}%
\pgfsys@useobject{currentmarker}{}%
\end{pgfscope}%
\begin{pgfscope}%
\pgfsys@transformshift{4.005000in}{0.561000in}%
\pgfsys@useobject{currentmarker}{}%
\end{pgfscope}%
\begin{pgfscope}%
\pgfsys@transformshift{4.131818in}{0.623000in}%
\pgfsys@useobject{currentmarker}{}%
\end{pgfscope}%
\begin{pgfscope}%
\pgfsys@transformshift{4.977273in}{1.134500in}%
\pgfsys@useobject{currentmarker}{}%
\end{pgfscope}%
\end{pgfscope}%
\begin{pgfscope}%
\pgfsetrectcap%
\pgfsetmiterjoin%
\pgfsetlinewidth{0.000000pt}%
\definecolor{currentstroke}{rgb}{1.000000,1.000000,1.000000}%
\pgfsetstrokecolor{currentstroke}%
\pgfsetdash{}{0pt}%
\pgfpathmoveto{\pgfqpoint{3.286364in}{0.375000in}}%
\pgfpathlineto{\pgfqpoint{5.400000in}{0.375000in}}%
\pgfusepath{}%
\end{pgfscope}%
\begin{pgfscope}%
\pgfsetrectcap%
\pgfsetmiterjoin%
\pgfsetlinewidth{0.000000pt}%
\definecolor{currentstroke}{rgb}{1.000000,1.000000,1.000000}%
\pgfsetstrokecolor{currentstroke}%
\pgfsetdash{}{0pt}%
\pgfpathmoveto{\pgfqpoint{3.286364in}{0.375000in}}%
\pgfpathlineto{\pgfqpoint{3.286364in}{1.305000in}}%
\pgfusepath{}%
\end{pgfscope}%
\begin{pgfscope}%
\pgfsetrectcap%
\pgfsetmiterjoin%
\pgfsetlinewidth{0.000000pt}%
\definecolor{currentstroke}{rgb}{1.000000,1.000000,1.000000}%
\pgfsetstrokecolor{currentstroke}%
\pgfsetdash{}{0pt}%
\pgfpathmoveto{\pgfqpoint{3.286364in}{1.305000in}}%
\pgfpathlineto{\pgfqpoint{5.400000in}{1.305000in}}%
\pgfusepath{}%
\end{pgfscope}%
\begin{pgfscope}%
\pgfsetrectcap%
\pgfsetmiterjoin%
\pgfsetlinewidth{0.000000pt}%
\definecolor{currentstroke}{rgb}{1.000000,1.000000,1.000000}%
\pgfsetstrokecolor{currentstroke}%
\pgfsetdash{}{0pt}%
\pgfpathmoveto{\pgfqpoint{5.400000in}{0.375000in}}%
\pgfpathlineto{\pgfqpoint{5.400000in}{1.305000in}}%
\pgfusepath{}%
\end{pgfscope}%
\end{pgfpicture}%
\makeatother%
\endgroup%

  \caption{Linear regression based prediction.}
  \label{fig_goodlr}
\end{figure}

\begin{comment}
  % In case there is some space at the end this is the plot of the first model
  \begin{figure}
    \centering
    %% Creator: Matplotlib, PGF backend
%%
%% To include the figure in your LaTeX document, write
%%   \input{<filename>.pgf}
%%
%% Make sure the required packages are loaded in your preamble
%%   \usepackage{pgf}
%%
%% Figures using additional raster images can only be included by \input if
%% they are in the same directory as the main LaTeX file. For loading figures
%% from other directories you can use the `import` package
%%   \usepackage{import}
%% and then include the figures with
%%   \import{<path to file>}{<filename>.pgf}
%%
%% Matplotlib used the following preamble
%%   \usepackage[utf8x]{inputenc}
%%   \usepackage[T1]{fontenc}
%%   \usepackage{cmbright}
%%
\begingroup%
\makeatletter%
\begin{pgfpicture}%
\pgfpathrectangle{\pgfpointorigin}{\pgfqpoint{6.000000in}{3.000000in}}%
\pgfusepath{use as bounding box, clip}%
\begin{pgfscope}%
\pgfsetbuttcap%
\pgfsetmiterjoin%
\definecolor{currentfill}{rgb}{1.000000,1.000000,1.000000}%
\pgfsetfillcolor{currentfill}%
\pgfsetlinewidth{0.000000pt}%
\definecolor{currentstroke}{rgb}{1.000000,1.000000,1.000000}%
\pgfsetstrokecolor{currentstroke}%
\pgfsetdash{}{0pt}%
\pgfpathmoveto{\pgfqpoint{0.000000in}{0.000000in}}%
\pgfpathlineto{\pgfqpoint{6.000000in}{0.000000in}}%
\pgfpathlineto{\pgfqpoint{6.000000in}{3.000000in}}%
\pgfpathlineto{\pgfqpoint{0.000000in}{3.000000in}}%
\pgfpathclose%
\pgfusepath{fill}%
\end{pgfscope}%
\begin{pgfscope}%
\pgfsetbuttcap%
\pgfsetmiterjoin%
\definecolor{currentfill}{rgb}{0.917647,0.917647,0.949020}%
\pgfsetfillcolor{currentfill}%
\pgfsetlinewidth{0.000000pt}%
\definecolor{currentstroke}{rgb}{0.000000,0.000000,0.000000}%
\pgfsetstrokecolor{currentstroke}%
\pgfsetstrokeopacity{0.000000}%
\pgfsetdash{}{0pt}%
\pgfpathmoveto{\pgfqpoint{0.556847in}{1.886339in}}%
\pgfpathlineto{\pgfqpoint{3.134424in}{1.886339in}}%
\pgfpathlineto{\pgfqpoint{3.134424in}{2.799750in}}%
\pgfpathlineto{\pgfqpoint{0.556847in}{2.799750in}}%
\pgfpathclose%
\pgfusepath{fill}%
\end{pgfscope}%
\begin{pgfscope}%
\pgfpathrectangle{\pgfqpoint{0.556847in}{1.886339in}}{\pgfqpoint{2.577576in}{0.913411in}} %
\pgfusepath{clip}%
\pgfsetroundcap%
\pgfsetroundjoin%
\pgfsetlinewidth{0.803000pt}%
\definecolor{currentstroke}{rgb}{1.000000,1.000000,1.000000}%
\pgfsetstrokecolor{currentstroke}%
\pgfsetdash{}{0pt}%
\pgfpathmoveto{\pgfqpoint{0.757884in}{1.886339in}}%
\pgfpathlineto{\pgfqpoint{0.757884in}{2.799750in}}%
\pgfusepath{stroke}%
\end{pgfscope}%
\begin{pgfscope}%
\pgfsetbuttcap%
\pgfsetroundjoin%
\definecolor{currentfill}{rgb}{0.150000,0.150000,0.150000}%
\pgfsetfillcolor{currentfill}%
\pgfsetlinewidth{0.803000pt}%
\definecolor{currentstroke}{rgb}{0.150000,0.150000,0.150000}%
\pgfsetstrokecolor{currentstroke}%
\pgfsetdash{}{0pt}%
\pgfsys@defobject{currentmarker}{\pgfqpoint{0.000000in}{0.000000in}}{\pgfqpoint{0.000000in}{0.000000in}}{%
\pgfpathmoveto{\pgfqpoint{0.000000in}{0.000000in}}%
\pgfpathlineto{\pgfqpoint{0.000000in}{0.000000in}}%
\pgfusepath{stroke,fill}%
}%
\begin{pgfscope}%
\pgfsys@transformshift{0.757884in}{1.886339in}%
\pgfsys@useobject{currentmarker}{}%
\end{pgfscope}%
\end{pgfscope}%
\begin{pgfscope}%
\pgfsetbuttcap%
\pgfsetroundjoin%
\definecolor{currentfill}{rgb}{0.150000,0.150000,0.150000}%
\pgfsetfillcolor{currentfill}%
\pgfsetlinewidth{0.803000pt}%
\definecolor{currentstroke}{rgb}{0.150000,0.150000,0.150000}%
\pgfsetstrokecolor{currentstroke}%
\pgfsetdash{}{0pt}%
\pgfsys@defobject{currentmarker}{\pgfqpoint{0.000000in}{0.000000in}}{\pgfqpoint{0.000000in}{0.000000in}}{%
\pgfpathmoveto{\pgfqpoint{0.000000in}{0.000000in}}%
\pgfpathlineto{\pgfqpoint{0.000000in}{0.000000in}}%
\pgfusepath{stroke,fill}%
}%
\begin{pgfscope}%
\pgfsys@transformshift{0.757884in}{2.799750in}%
\pgfsys@useobject{currentmarker}{}%
\end{pgfscope}%
\end{pgfscope}%
\begin{pgfscope}%
\definecolor{textcolor}{rgb}{0.150000,0.150000,0.150000}%
\pgfsetstrokecolor{textcolor}%
\pgfsetfillcolor{textcolor}%
\pgftext[x=0.757884in,y=1.808561in,,top]{\color{textcolor}\sffamily\fontsize{8.000000}{9.600000}\selectfont 3.5}%
\end{pgfscope}%
\begin{pgfscope}%
\pgfpathrectangle{\pgfqpoint{0.556847in}{1.886339in}}{\pgfqpoint{2.577576in}{0.913411in}} %
\pgfusepath{clip}%
\pgfsetroundcap%
\pgfsetroundjoin%
\pgfsetlinewidth{0.803000pt}%
\definecolor{currentstroke}{rgb}{1.000000,1.000000,1.000000}%
\pgfsetstrokecolor{currentstroke}%
\pgfsetdash{}{0pt}%
\pgfpathmoveto{\pgfqpoint{1.116878in}{1.886339in}}%
\pgfpathlineto{\pgfqpoint{1.116878in}{2.799750in}}%
\pgfusepath{stroke}%
\end{pgfscope}%
\begin{pgfscope}%
\pgfsetbuttcap%
\pgfsetroundjoin%
\definecolor{currentfill}{rgb}{0.150000,0.150000,0.150000}%
\pgfsetfillcolor{currentfill}%
\pgfsetlinewidth{0.803000pt}%
\definecolor{currentstroke}{rgb}{0.150000,0.150000,0.150000}%
\pgfsetstrokecolor{currentstroke}%
\pgfsetdash{}{0pt}%
\pgfsys@defobject{currentmarker}{\pgfqpoint{0.000000in}{0.000000in}}{\pgfqpoint{0.000000in}{0.000000in}}{%
\pgfpathmoveto{\pgfqpoint{0.000000in}{0.000000in}}%
\pgfpathlineto{\pgfqpoint{0.000000in}{0.000000in}}%
\pgfusepath{stroke,fill}%
}%
\begin{pgfscope}%
\pgfsys@transformshift{1.116878in}{1.886339in}%
\pgfsys@useobject{currentmarker}{}%
\end{pgfscope}%
\end{pgfscope}%
\begin{pgfscope}%
\pgfsetbuttcap%
\pgfsetroundjoin%
\definecolor{currentfill}{rgb}{0.150000,0.150000,0.150000}%
\pgfsetfillcolor{currentfill}%
\pgfsetlinewidth{0.803000pt}%
\definecolor{currentstroke}{rgb}{0.150000,0.150000,0.150000}%
\pgfsetstrokecolor{currentstroke}%
\pgfsetdash{}{0pt}%
\pgfsys@defobject{currentmarker}{\pgfqpoint{0.000000in}{0.000000in}}{\pgfqpoint{0.000000in}{0.000000in}}{%
\pgfpathmoveto{\pgfqpoint{0.000000in}{0.000000in}}%
\pgfpathlineto{\pgfqpoint{0.000000in}{0.000000in}}%
\pgfusepath{stroke,fill}%
}%
\begin{pgfscope}%
\pgfsys@transformshift{1.116878in}{2.799750in}%
\pgfsys@useobject{currentmarker}{}%
\end{pgfscope}%
\end{pgfscope}%
\begin{pgfscope}%
\definecolor{textcolor}{rgb}{0.150000,0.150000,0.150000}%
\pgfsetstrokecolor{textcolor}%
\pgfsetfillcolor{textcolor}%
\pgftext[x=1.116878in,y=1.808561in,,top]{\color{textcolor}\sffamily\fontsize{8.000000}{9.600000}\selectfont 4.0}%
\end{pgfscope}%
\begin{pgfscope}%
\pgfpathrectangle{\pgfqpoint{0.556847in}{1.886339in}}{\pgfqpoint{2.577576in}{0.913411in}} %
\pgfusepath{clip}%
\pgfsetroundcap%
\pgfsetroundjoin%
\pgfsetlinewidth{0.803000pt}%
\definecolor{currentstroke}{rgb}{1.000000,1.000000,1.000000}%
\pgfsetstrokecolor{currentstroke}%
\pgfsetdash{}{0pt}%
\pgfpathmoveto{\pgfqpoint{1.475872in}{1.886339in}}%
\pgfpathlineto{\pgfqpoint{1.475872in}{2.799750in}}%
\pgfusepath{stroke}%
\end{pgfscope}%
\begin{pgfscope}%
\pgfsetbuttcap%
\pgfsetroundjoin%
\definecolor{currentfill}{rgb}{0.150000,0.150000,0.150000}%
\pgfsetfillcolor{currentfill}%
\pgfsetlinewidth{0.803000pt}%
\definecolor{currentstroke}{rgb}{0.150000,0.150000,0.150000}%
\pgfsetstrokecolor{currentstroke}%
\pgfsetdash{}{0pt}%
\pgfsys@defobject{currentmarker}{\pgfqpoint{0.000000in}{0.000000in}}{\pgfqpoint{0.000000in}{0.000000in}}{%
\pgfpathmoveto{\pgfqpoint{0.000000in}{0.000000in}}%
\pgfpathlineto{\pgfqpoint{0.000000in}{0.000000in}}%
\pgfusepath{stroke,fill}%
}%
\begin{pgfscope}%
\pgfsys@transformshift{1.475872in}{1.886339in}%
\pgfsys@useobject{currentmarker}{}%
\end{pgfscope}%
\end{pgfscope}%
\begin{pgfscope}%
\pgfsetbuttcap%
\pgfsetroundjoin%
\definecolor{currentfill}{rgb}{0.150000,0.150000,0.150000}%
\pgfsetfillcolor{currentfill}%
\pgfsetlinewidth{0.803000pt}%
\definecolor{currentstroke}{rgb}{0.150000,0.150000,0.150000}%
\pgfsetstrokecolor{currentstroke}%
\pgfsetdash{}{0pt}%
\pgfsys@defobject{currentmarker}{\pgfqpoint{0.000000in}{0.000000in}}{\pgfqpoint{0.000000in}{0.000000in}}{%
\pgfpathmoveto{\pgfqpoint{0.000000in}{0.000000in}}%
\pgfpathlineto{\pgfqpoint{0.000000in}{0.000000in}}%
\pgfusepath{stroke,fill}%
}%
\begin{pgfscope}%
\pgfsys@transformshift{1.475872in}{2.799750in}%
\pgfsys@useobject{currentmarker}{}%
\end{pgfscope}%
\end{pgfscope}%
\begin{pgfscope}%
\definecolor{textcolor}{rgb}{0.150000,0.150000,0.150000}%
\pgfsetstrokecolor{textcolor}%
\pgfsetfillcolor{textcolor}%
\pgftext[x=1.475872in,y=1.808561in,,top]{\color{textcolor}\sffamily\fontsize{8.000000}{9.600000}\selectfont 4.5}%
\end{pgfscope}%
\begin{pgfscope}%
\pgfpathrectangle{\pgfqpoint{0.556847in}{1.886339in}}{\pgfqpoint{2.577576in}{0.913411in}} %
\pgfusepath{clip}%
\pgfsetroundcap%
\pgfsetroundjoin%
\pgfsetlinewidth{0.803000pt}%
\definecolor{currentstroke}{rgb}{1.000000,1.000000,1.000000}%
\pgfsetstrokecolor{currentstroke}%
\pgfsetdash{}{0pt}%
\pgfpathmoveto{\pgfqpoint{1.834866in}{1.886339in}}%
\pgfpathlineto{\pgfqpoint{1.834866in}{2.799750in}}%
\pgfusepath{stroke}%
\end{pgfscope}%
\begin{pgfscope}%
\pgfsetbuttcap%
\pgfsetroundjoin%
\definecolor{currentfill}{rgb}{0.150000,0.150000,0.150000}%
\pgfsetfillcolor{currentfill}%
\pgfsetlinewidth{0.803000pt}%
\definecolor{currentstroke}{rgb}{0.150000,0.150000,0.150000}%
\pgfsetstrokecolor{currentstroke}%
\pgfsetdash{}{0pt}%
\pgfsys@defobject{currentmarker}{\pgfqpoint{0.000000in}{0.000000in}}{\pgfqpoint{0.000000in}{0.000000in}}{%
\pgfpathmoveto{\pgfqpoint{0.000000in}{0.000000in}}%
\pgfpathlineto{\pgfqpoint{0.000000in}{0.000000in}}%
\pgfusepath{stroke,fill}%
}%
\begin{pgfscope}%
\pgfsys@transformshift{1.834866in}{1.886339in}%
\pgfsys@useobject{currentmarker}{}%
\end{pgfscope}%
\end{pgfscope}%
\begin{pgfscope}%
\pgfsetbuttcap%
\pgfsetroundjoin%
\definecolor{currentfill}{rgb}{0.150000,0.150000,0.150000}%
\pgfsetfillcolor{currentfill}%
\pgfsetlinewidth{0.803000pt}%
\definecolor{currentstroke}{rgb}{0.150000,0.150000,0.150000}%
\pgfsetstrokecolor{currentstroke}%
\pgfsetdash{}{0pt}%
\pgfsys@defobject{currentmarker}{\pgfqpoint{0.000000in}{0.000000in}}{\pgfqpoint{0.000000in}{0.000000in}}{%
\pgfpathmoveto{\pgfqpoint{0.000000in}{0.000000in}}%
\pgfpathlineto{\pgfqpoint{0.000000in}{0.000000in}}%
\pgfusepath{stroke,fill}%
}%
\begin{pgfscope}%
\pgfsys@transformshift{1.834866in}{2.799750in}%
\pgfsys@useobject{currentmarker}{}%
\end{pgfscope}%
\end{pgfscope}%
\begin{pgfscope}%
\definecolor{textcolor}{rgb}{0.150000,0.150000,0.150000}%
\pgfsetstrokecolor{textcolor}%
\pgfsetfillcolor{textcolor}%
\pgftext[x=1.834866in,y=1.808561in,,top]{\color{textcolor}\sffamily\fontsize{8.000000}{9.600000}\selectfont 5.0}%
\end{pgfscope}%
\begin{pgfscope}%
\pgfpathrectangle{\pgfqpoint{0.556847in}{1.886339in}}{\pgfqpoint{2.577576in}{0.913411in}} %
\pgfusepath{clip}%
\pgfsetroundcap%
\pgfsetroundjoin%
\pgfsetlinewidth{0.803000pt}%
\definecolor{currentstroke}{rgb}{1.000000,1.000000,1.000000}%
\pgfsetstrokecolor{currentstroke}%
\pgfsetdash{}{0pt}%
\pgfpathmoveto{\pgfqpoint{2.193860in}{1.886339in}}%
\pgfpathlineto{\pgfqpoint{2.193860in}{2.799750in}}%
\pgfusepath{stroke}%
\end{pgfscope}%
\begin{pgfscope}%
\pgfsetbuttcap%
\pgfsetroundjoin%
\definecolor{currentfill}{rgb}{0.150000,0.150000,0.150000}%
\pgfsetfillcolor{currentfill}%
\pgfsetlinewidth{0.803000pt}%
\definecolor{currentstroke}{rgb}{0.150000,0.150000,0.150000}%
\pgfsetstrokecolor{currentstroke}%
\pgfsetdash{}{0pt}%
\pgfsys@defobject{currentmarker}{\pgfqpoint{0.000000in}{0.000000in}}{\pgfqpoint{0.000000in}{0.000000in}}{%
\pgfpathmoveto{\pgfqpoint{0.000000in}{0.000000in}}%
\pgfpathlineto{\pgfqpoint{0.000000in}{0.000000in}}%
\pgfusepath{stroke,fill}%
}%
\begin{pgfscope}%
\pgfsys@transformshift{2.193860in}{1.886339in}%
\pgfsys@useobject{currentmarker}{}%
\end{pgfscope}%
\end{pgfscope}%
\begin{pgfscope}%
\pgfsetbuttcap%
\pgfsetroundjoin%
\definecolor{currentfill}{rgb}{0.150000,0.150000,0.150000}%
\pgfsetfillcolor{currentfill}%
\pgfsetlinewidth{0.803000pt}%
\definecolor{currentstroke}{rgb}{0.150000,0.150000,0.150000}%
\pgfsetstrokecolor{currentstroke}%
\pgfsetdash{}{0pt}%
\pgfsys@defobject{currentmarker}{\pgfqpoint{0.000000in}{0.000000in}}{\pgfqpoint{0.000000in}{0.000000in}}{%
\pgfpathmoveto{\pgfqpoint{0.000000in}{0.000000in}}%
\pgfpathlineto{\pgfqpoint{0.000000in}{0.000000in}}%
\pgfusepath{stroke,fill}%
}%
\begin{pgfscope}%
\pgfsys@transformshift{2.193860in}{2.799750in}%
\pgfsys@useobject{currentmarker}{}%
\end{pgfscope}%
\end{pgfscope}%
\begin{pgfscope}%
\definecolor{textcolor}{rgb}{0.150000,0.150000,0.150000}%
\pgfsetstrokecolor{textcolor}%
\pgfsetfillcolor{textcolor}%
\pgftext[x=2.193860in,y=1.808561in,,top]{\color{textcolor}\sffamily\fontsize{8.000000}{9.600000}\selectfont 5.5}%
\end{pgfscope}%
\begin{pgfscope}%
\pgfpathrectangle{\pgfqpoint{0.556847in}{1.886339in}}{\pgfqpoint{2.577576in}{0.913411in}} %
\pgfusepath{clip}%
\pgfsetroundcap%
\pgfsetroundjoin%
\pgfsetlinewidth{0.803000pt}%
\definecolor{currentstroke}{rgb}{1.000000,1.000000,1.000000}%
\pgfsetstrokecolor{currentstroke}%
\pgfsetdash{}{0pt}%
\pgfpathmoveto{\pgfqpoint{2.552853in}{1.886339in}}%
\pgfpathlineto{\pgfqpoint{2.552853in}{2.799750in}}%
\pgfusepath{stroke}%
\end{pgfscope}%
\begin{pgfscope}%
\pgfsetbuttcap%
\pgfsetroundjoin%
\definecolor{currentfill}{rgb}{0.150000,0.150000,0.150000}%
\pgfsetfillcolor{currentfill}%
\pgfsetlinewidth{0.803000pt}%
\definecolor{currentstroke}{rgb}{0.150000,0.150000,0.150000}%
\pgfsetstrokecolor{currentstroke}%
\pgfsetdash{}{0pt}%
\pgfsys@defobject{currentmarker}{\pgfqpoint{0.000000in}{0.000000in}}{\pgfqpoint{0.000000in}{0.000000in}}{%
\pgfpathmoveto{\pgfqpoint{0.000000in}{0.000000in}}%
\pgfpathlineto{\pgfqpoint{0.000000in}{0.000000in}}%
\pgfusepath{stroke,fill}%
}%
\begin{pgfscope}%
\pgfsys@transformshift{2.552853in}{1.886339in}%
\pgfsys@useobject{currentmarker}{}%
\end{pgfscope}%
\end{pgfscope}%
\begin{pgfscope}%
\pgfsetbuttcap%
\pgfsetroundjoin%
\definecolor{currentfill}{rgb}{0.150000,0.150000,0.150000}%
\pgfsetfillcolor{currentfill}%
\pgfsetlinewidth{0.803000pt}%
\definecolor{currentstroke}{rgb}{0.150000,0.150000,0.150000}%
\pgfsetstrokecolor{currentstroke}%
\pgfsetdash{}{0pt}%
\pgfsys@defobject{currentmarker}{\pgfqpoint{0.000000in}{0.000000in}}{\pgfqpoint{0.000000in}{0.000000in}}{%
\pgfpathmoveto{\pgfqpoint{0.000000in}{0.000000in}}%
\pgfpathlineto{\pgfqpoint{0.000000in}{0.000000in}}%
\pgfusepath{stroke,fill}%
}%
\begin{pgfscope}%
\pgfsys@transformshift{2.552853in}{2.799750in}%
\pgfsys@useobject{currentmarker}{}%
\end{pgfscope}%
\end{pgfscope}%
\begin{pgfscope}%
\definecolor{textcolor}{rgb}{0.150000,0.150000,0.150000}%
\pgfsetstrokecolor{textcolor}%
\pgfsetfillcolor{textcolor}%
\pgftext[x=2.552853in,y=1.808561in,,top]{\color{textcolor}\sffamily\fontsize{8.000000}{9.600000}\selectfont 6.0}%
\end{pgfscope}%
\begin{pgfscope}%
\pgfpathrectangle{\pgfqpoint{0.556847in}{1.886339in}}{\pgfqpoint{2.577576in}{0.913411in}} %
\pgfusepath{clip}%
\pgfsetroundcap%
\pgfsetroundjoin%
\pgfsetlinewidth{0.803000pt}%
\definecolor{currentstroke}{rgb}{1.000000,1.000000,1.000000}%
\pgfsetstrokecolor{currentstroke}%
\pgfsetdash{}{0pt}%
\pgfpathmoveto{\pgfqpoint{2.911847in}{1.886339in}}%
\pgfpathlineto{\pgfqpoint{2.911847in}{2.799750in}}%
\pgfusepath{stroke}%
\end{pgfscope}%
\begin{pgfscope}%
\pgfsetbuttcap%
\pgfsetroundjoin%
\definecolor{currentfill}{rgb}{0.150000,0.150000,0.150000}%
\pgfsetfillcolor{currentfill}%
\pgfsetlinewidth{0.803000pt}%
\definecolor{currentstroke}{rgb}{0.150000,0.150000,0.150000}%
\pgfsetstrokecolor{currentstroke}%
\pgfsetdash{}{0pt}%
\pgfsys@defobject{currentmarker}{\pgfqpoint{0.000000in}{0.000000in}}{\pgfqpoint{0.000000in}{0.000000in}}{%
\pgfpathmoveto{\pgfqpoint{0.000000in}{0.000000in}}%
\pgfpathlineto{\pgfqpoint{0.000000in}{0.000000in}}%
\pgfusepath{stroke,fill}%
}%
\begin{pgfscope}%
\pgfsys@transformshift{2.911847in}{1.886339in}%
\pgfsys@useobject{currentmarker}{}%
\end{pgfscope}%
\end{pgfscope}%
\begin{pgfscope}%
\pgfsetbuttcap%
\pgfsetroundjoin%
\definecolor{currentfill}{rgb}{0.150000,0.150000,0.150000}%
\pgfsetfillcolor{currentfill}%
\pgfsetlinewidth{0.803000pt}%
\definecolor{currentstroke}{rgb}{0.150000,0.150000,0.150000}%
\pgfsetstrokecolor{currentstroke}%
\pgfsetdash{}{0pt}%
\pgfsys@defobject{currentmarker}{\pgfqpoint{0.000000in}{0.000000in}}{\pgfqpoint{0.000000in}{0.000000in}}{%
\pgfpathmoveto{\pgfqpoint{0.000000in}{0.000000in}}%
\pgfpathlineto{\pgfqpoint{0.000000in}{0.000000in}}%
\pgfusepath{stroke,fill}%
}%
\begin{pgfscope}%
\pgfsys@transformshift{2.911847in}{2.799750in}%
\pgfsys@useobject{currentmarker}{}%
\end{pgfscope}%
\end{pgfscope}%
\begin{pgfscope}%
\definecolor{textcolor}{rgb}{0.150000,0.150000,0.150000}%
\pgfsetstrokecolor{textcolor}%
\pgfsetfillcolor{textcolor}%
\pgftext[x=2.911847in,y=1.808561in,,top]{\color{textcolor}\sffamily\fontsize{8.000000}{9.600000}\selectfont 6.5}%
\end{pgfscope}%
\begin{pgfscope}%
\definecolor{textcolor}{rgb}{0.150000,0.150000,0.150000}%
\pgfsetstrokecolor{textcolor}%
\pgfsetfillcolor{textcolor}%
\pgftext[x=1.845635in,y=1.643438in,,top]{\color{textcolor}\sffamily\fontsize{8.800000}{10.560000}\selectfont Wing length}%
\end{pgfscope}%
\begin{pgfscope}%
\pgfpathrectangle{\pgfqpoint{0.556847in}{1.886339in}}{\pgfqpoint{2.577576in}{0.913411in}} %
\pgfusepath{clip}%
\pgfsetroundcap%
\pgfsetroundjoin%
\pgfsetlinewidth{0.803000pt}%
\definecolor{currentstroke}{rgb}{1.000000,1.000000,1.000000}%
\pgfsetstrokecolor{currentstroke}%
\pgfsetdash{}{0pt}%
\pgfpathmoveto{\pgfqpoint{0.556847in}{1.886339in}}%
\pgfpathlineto{\pgfqpoint{3.134424in}{1.886339in}}%
\pgfusepath{stroke}%
\end{pgfscope}%
\begin{pgfscope}%
\pgfsetbuttcap%
\pgfsetroundjoin%
\definecolor{currentfill}{rgb}{0.150000,0.150000,0.150000}%
\pgfsetfillcolor{currentfill}%
\pgfsetlinewidth{0.803000pt}%
\definecolor{currentstroke}{rgb}{0.150000,0.150000,0.150000}%
\pgfsetstrokecolor{currentstroke}%
\pgfsetdash{}{0pt}%
\pgfsys@defobject{currentmarker}{\pgfqpoint{0.000000in}{0.000000in}}{\pgfqpoint{0.000000in}{0.000000in}}{%
\pgfpathmoveto{\pgfqpoint{0.000000in}{0.000000in}}%
\pgfpathlineto{\pgfqpoint{0.000000in}{0.000000in}}%
\pgfusepath{stroke,fill}%
}%
\begin{pgfscope}%
\pgfsys@transformshift{0.556847in}{1.886339in}%
\pgfsys@useobject{currentmarker}{}%
\end{pgfscope}%
\end{pgfscope}%
\begin{pgfscope}%
\pgfsetbuttcap%
\pgfsetroundjoin%
\definecolor{currentfill}{rgb}{0.150000,0.150000,0.150000}%
\pgfsetfillcolor{currentfill}%
\pgfsetlinewidth{0.803000pt}%
\definecolor{currentstroke}{rgb}{0.150000,0.150000,0.150000}%
\pgfsetstrokecolor{currentstroke}%
\pgfsetdash{}{0pt}%
\pgfsys@defobject{currentmarker}{\pgfqpoint{0.000000in}{0.000000in}}{\pgfqpoint{0.000000in}{0.000000in}}{%
\pgfpathmoveto{\pgfqpoint{0.000000in}{0.000000in}}%
\pgfpathlineto{\pgfqpoint{0.000000in}{0.000000in}}%
\pgfusepath{stroke,fill}%
}%
\begin{pgfscope}%
\pgfsys@transformshift{3.134424in}{1.886339in}%
\pgfsys@useobject{currentmarker}{}%
\end{pgfscope}%
\end{pgfscope}%
\begin{pgfscope}%
\definecolor{textcolor}{rgb}{0.150000,0.150000,0.150000}%
\pgfsetstrokecolor{textcolor}%
\pgfsetfillcolor{textcolor}%
\pgftext[x=0.479069in,y=1.886339in,right,]{\color{textcolor}\sffamily\fontsize{8.000000}{9.600000}\selectfont 2.0}%
\end{pgfscope}%
\begin{pgfscope}%
\pgfpathrectangle{\pgfqpoint{0.556847in}{1.886339in}}{\pgfqpoint{2.577576in}{0.913411in}} %
\pgfusepath{clip}%
\pgfsetroundcap%
\pgfsetroundjoin%
\pgfsetlinewidth{0.803000pt}%
\definecolor{currentstroke}{rgb}{1.000000,1.000000,1.000000}%
\pgfsetstrokecolor{currentstroke}%
\pgfsetdash{}{0pt}%
\pgfpathmoveto{\pgfqpoint{0.556847in}{2.038574in}}%
\pgfpathlineto{\pgfqpoint{3.134424in}{2.038574in}}%
\pgfusepath{stroke}%
\end{pgfscope}%
\begin{pgfscope}%
\pgfsetbuttcap%
\pgfsetroundjoin%
\definecolor{currentfill}{rgb}{0.150000,0.150000,0.150000}%
\pgfsetfillcolor{currentfill}%
\pgfsetlinewidth{0.803000pt}%
\definecolor{currentstroke}{rgb}{0.150000,0.150000,0.150000}%
\pgfsetstrokecolor{currentstroke}%
\pgfsetdash{}{0pt}%
\pgfsys@defobject{currentmarker}{\pgfqpoint{0.000000in}{0.000000in}}{\pgfqpoint{0.000000in}{0.000000in}}{%
\pgfpathmoveto{\pgfqpoint{0.000000in}{0.000000in}}%
\pgfpathlineto{\pgfqpoint{0.000000in}{0.000000in}}%
\pgfusepath{stroke,fill}%
}%
\begin{pgfscope}%
\pgfsys@transformshift{0.556847in}{2.038574in}%
\pgfsys@useobject{currentmarker}{}%
\end{pgfscope}%
\end{pgfscope}%
\begin{pgfscope}%
\pgfsetbuttcap%
\pgfsetroundjoin%
\definecolor{currentfill}{rgb}{0.150000,0.150000,0.150000}%
\pgfsetfillcolor{currentfill}%
\pgfsetlinewidth{0.803000pt}%
\definecolor{currentstroke}{rgb}{0.150000,0.150000,0.150000}%
\pgfsetstrokecolor{currentstroke}%
\pgfsetdash{}{0pt}%
\pgfsys@defobject{currentmarker}{\pgfqpoint{0.000000in}{0.000000in}}{\pgfqpoint{0.000000in}{0.000000in}}{%
\pgfpathmoveto{\pgfqpoint{0.000000in}{0.000000in}}%
\pgfpathlineto{\pgfqpoint{0.000000in}{0.000000in}}%
\pgfusepath{stroke,fill}%
}%
\begin{pgfscope}%
\pgfsys@transformshift{3.134424in}{2.038574in}%
\pgfsys@useobject{currentmarker}{}%
\end{pgfscope}%
\end{pgfscope}%
\begin{pgfscope}%
\definecolor{textcolor}{rgb}{0.150000,0.150000,0.150000}%
\pgfsetstrokecolor{textcolor}%
\pgfsetfillcolor{textcolor}%
\pgftext[x=0.479069in,y=2.038574in,right,]{\color{textcolor}\sffamily\fontsize{8.000000}{9.600000}\selectfont 2.5}%
\end{pgfscope}%
\begin{pgfscope}%
\pgfpathrectangle{\pgfqpoint{0.556847in}{1.886339in}}{\pgfqpoint{2.577576in}{0.913411in}} %
\pgfusepath{clip}%
\pgfsetroundcap%
\pgfsetroundjoin%
\pgfsetlinewidth{0.803000pt}%
\definecolor{currentstroke}{rgb}{1.000000,1.000000,1.000000}%
\pgfsetstrokecolor{currentstroke}%
\pgfsetdash{}{0pt}%
\pgfpathmoveto{\pgfqpoint{0.556847in}{2.190809in}}%
\pgfpathlineto{\pgfqpoint{3.134424in}{2.190809in}}%
\pgfusepath{stroke}%
\end{pgfscope}%
\begin{pgfscope}%
\pgfsetbuttcap%
\pgfsetroundjoin%
\definecolor{currentfill}{rgb}{0.150000,0.150000,0.150000}%
\pgfsetfillcolor{currentfill}%
\pgfsetlinewidth{0.803000pt}%
\definecolor{currentstroke}{rgb}{0.150000,0.150000,0.150000}%
\pgfsetstrokecolor{currentstroke}%
\pgfsetdash{}{0pt}%
\pgfsys@defobject{currentmarker}{\pgfqpoint{0.000000in}{0.000000in}}{\pgfqpoint{0.000000in}{0.000000in}}{%
\pgfpathmoveto{\pgfqpoint{0.000000in}{0.000000in}}%
\pgfpathlineto{\pgfqpoint{0.000000in}{0.000000in}}%
\pgfusepath{stroke,fill}%
}%
\begin{pgfscope}%
\pgfsys@transformshift{0.556847in}{2.190809in}%
\pgfsys@useobject{currentmarker}{}%
\end{pgfscope}%
\end{pgfscope}%
\begin{pgfscope}%
\pgfsetbuttcap%
\pgfsetroundjoin%
\definecolor{currentfill}{rgb}{0.150000,0.150000,0.150000}%
\pgfsetfillcolor{currentfill}%
\pgfsetlinewidth{0.803000pt}%
\definecolor{currentstroke}{rgb}{0.150000,0.150000,0.150000}%
\pgfsetstrokecolor{currentstroke}%
\pgfsetdash{}{0pt}%
\pgfsys@defobject{currentmarker}{\pgfqpoint{0.000000in}{0.000000in}}{\pgfqpoint{0.000000in}{0.000000in}}{%
\pgfpathmoveto{\pgfqpoint{0.000000in}{0.000000in}}%
\pgfpathlineto{\pgfqpoint{0.000000in}{0.000000in}}%
\pgfusepath{stroke,fill}%
}%
\begin{pgfscope}%
\pgfsys@transformshift{3.134424in}{2.190809in}%
\pgfsys@useobject{currentmarker}{}%
\end{pgfscope}%
\end{pgfscope}%
\begin{pgfscope}%
\definecolor{textcolor}{rgb}{0.150000,0.150000,0.150000}%
\pgfsetstrokecolor{textcolor}%
\pgfsetfillcolor{textcolor}%
\pgftext[x=0.479069in,y=2.190809in,right,]{\color{textcolor}\sffamily\fontsize{8.000000}{9.600000}\selectfont 3.0}%
\end{pgfscope}%
\begin{pgfscope}%
\pgfpathrectangle{\pgfqpoint{0.556847in}{1.886339in}}{\pgfqpoint{2.577576in}{0.913411in}} %
\pgfusepath{clip}%
\pgfsetroundcap%
\pgfsetroundjoin%
\pgfsetlinewidth{0.803000pt}%
\definecolor{currentstroke}{rgb}{1.000000,1.000000,1.000000}%
\pgfsetstrokecolor{currentstroke}%
\pgfsetdash{}{0pt}%
\pgfpathmoveto{\pgfqpoint{0.556847in}{2.343044in}}%
\pgfpathlineto{\pgfqpoint{3.134424in}{2.343044in}}%
\pgfusepath{stroke}%
\end{pgfscope}%
\begin{pgfscope}%
\pgfsetbuttcap%
\pgfsetroundjoin%
\definecolor{currentfill}{rgb}{0.150000,0.150000,0.150000}%
\pgfsetfillcolor{currentfill}%
\pgfsetlinewidth{0.803000pt}%
\definecolor{currentstroke}{rgb}{0.150000,0.150000,0.150000}%
\pgfsetstrokecolor{currentstroke}%
\pgfsetdash{}{0pt}%
\pgfsys@defobject{currentmarker}{\pgfqpoint{0.000000in}{0.000000in}}{\pgfqpoint{0.000000in}{0.000000in}}{%
\pgfpathmoveto{\pgfqpoint{0.000000in}{0.000000in}}%
\pgfpathlineto{\pgfqpoint{0.000000in}{0.000000in}}%
\pgfusepath{stroke,fill}%
}%
\begin{pgfscope}%
\pgfsys@transformshift{0.556847in}{2.343044in}%
\pgfsys@useobject{currentmarker}{}%
\end{pgfscope}%
\end{pgfscope}%
\begin{pgfscope}%
\pgfsetbuttcap%
\pgfsetroundjoin%
\definecolor{currentfill}{rgb}{0.150000,0.150000,0.150000}%
\pgfsetfillcolor{currentfill}%
\pgfsetlinewidth{0.803000pt}%
\definecolor{currentstroke}{rgb}{0.150000,0.150000,0.150000}%
\pgfsetstrokecolor{currentstroke}%
\pgfsetdash{}{0pt}%
\pgfsys@defobject{currentmarker}{\pgfqpoint{0.000000in}{0.000000in}}{\pgfqpoint{0.000000in}{0.000000in}}{%
\pgfpathmoveto{\pgfqpoint{0.000000in}{0.000000in}}%
\pgfpathlineto{\pgfqpoint{0.000000in}{0.000000in}}%
\pgfusepath{stroke,fill}%
}%
\begin{pgfscope}%
\pgfsys@transformshift{3.134424in}{2.343044in}%
\pgfsys@useobject{currentmarker}{}%
\end{pgfscope}%
\end{pgfscope}%
\begin{pgfscope}%
\definecolor{textcolor}{rgb}{0.150000,0.150000,0.150000}%
\pgfsetstrokecolor{textcolor}%
\pgfsetfillcolor{textcolor}%
\pgftext[x=0.479069in,y=2.343044in,right,]{\color{textcolor}\sffamily\fontsize{8.000000}{9.600000}\selectfont 3.5}%
\end{pgfscope}%
\begin{pgfscope}%
\pgfpathrectangle{\pgfqpoint{0.556847in}{1.886339in}}{\pgfqpoint{2.577576in}{0.913411in}} %
\pgfusepath{clip}%
\pgfsetroundcap%
\pgfsetroundjoin%
\pgfsetlinewidth{0.803000pt}%
\definecolor{currentstroke}{rgb}{1.000000,1.000000,1.000000}%
\pgfsetstrokecolor{currentstroke}%
\pgfsetdash{}{0pt}%
\pgfpathmoveto{\pgfqpoint{0.556847in}{2.495280in}}%
\pgfpathlineto{\pgfqpoint{3.134424in}{2.495280in}}%
\pgfusepath{stroke}%
\end{pgfscope}%
\begin{pgfscope}%
\pgfsetbuttcap%
\pgfsetroundjoin%
\definecolor{currentfill}{rgb}{0.150000,0.150000,0.150000}%
\pgfsetfillcolor{currentfill}%
\pgfsetlinewidth{0.803000pt}%
\definecolor{currentstroke}{rgb}{0.150000,0.150000,0.150000}%
\pgfsetstrokecolor{currentstroke}%
\pgfsetdash{}{0pt}%
\pgfsys@defobject{currentmarker}{\pgfqpoint{0.000000in}{0.000000in}}{\pgfqpoint{0.000000in}{0.000000in}}{%
\pgfpathmoveto{\pgfqpoint{0.000000in}{0.000000in}}%
\pgfpathlineto{\pgfqpoint{0.000000in}{0.000000in}}%
\pgfusepath{stroke,fill}%
}%
\begin{pgfscope}%
\pgfsys@transformshift{0.556847in}{2.495280in}%
\pgfsys@useobject{currentmarker}{}%
\end{pgfscope}%
\end{pgfscope}%
\begin{pgfscope}%
\pgfsetbuttcap%
\pgfsetroundjoin%
\definecolor{currentfill}{rgb}{0.150000,0.150000,0.150000}%
\pgfsetfillcolor{currentfill}%
\pgfsetlinewidth{0.803000pt}%
\definecolor{currentstroke}{rgb}{0.150000,0.150000,0.150000}%
\pgfsetstrokecolor{currentstroke}%
\pgfsetdash{}{0pt}%
\pgfsys@defobject{currentmarker}{\pgfqpoint{0.000000in}{0.000000in}}{\pgfqpoint{0.000000in}{0.000000in}}{%
\pgfpathmoveto{\pgfqpoint{0.000000in}{0.000000in}}%
\pgfpathlineto{\pgfqpoint{0.000000in}{0.000000in}}%
\pgfusepath{stroke,fill}%
}%
\begin{pgfscope}%
\pgfsys@transformshift{3.134424in}{2.495280in}%
\pgfsys@useobject{currentmarker}{}%
\end{pgfscope}%
\end{pgfscope}%
\begin{pgfscope}%
\definecolor{textcolor}{rgb}{0.150000,0.150000,0.150000}%
\pgfsetstrokecolor{textcolor}%
\pgfsetfillcolor{textcolor}%
\pgftext[x=0.479069in,y=2.495280in,right,]{\color{textcolor}\sffamily\fontsize{8.000000}{9.600000}\selectfont 4.0}%
\end{pgfscope}%
\begin{pgfscope}%
\pgfpathrectangle{\pgfqpoint{0.556847in}{1.886339in}}{\pgfqpoint{2.577576in}{0.913411in}} %
\pgfusepath{clip}%
\pgfsetroundcap%
\pgfsetroundjoin%
\pgfsetlinewidth{0.803000pt}%
\definecolor{currentstroke}{rgb}{1.000000,1.000000,1.000000}%
\pgfsetstrokecolor{currentstroke}%
\pgfsetdash{}{0pt}%
\pgfpathmoveto{\pgfqpoint{0.556847in}{2.647515in}}%
\pgfpathlineto{\pgfqpoint{3.134424in}{2.647515in}}%
\pgfusepath{stroke}%
\end{pgfscope}%
\begin{pgfscope}%
\pgfsetbuttcap%
\pgfsetroundjoin%
\definecolor{currentfill}{rgb}{0.150000,0.150000,0.150000}%
\pgfsetfillcolor{currentfill}%
\pgfsetlinewidth{0.803000pt}%
\definecolor{currentstroke}{rgb}{0.150000,0.150000,0.150000}%
\pgfsetstrokecolor{currentstroke}%
\pgfsetdash{}{0pt}%
\pgfsys@defobject{currentmarker}{\pgfqpoint{0.000000in}{0.000000in}}{\pgfqpoint{0.000000in}{0.000000in}}{%
\pgfpathmoveto{\pgfqpoint{0.000000in}{0.000000in}}%
\pgfpathlineto{\pgfqpoint{0.000000in}{0.000000in}}%
\pgfusepath{stroke,fill}%
}%
\begin{pgfscope}%
\pgfsys@transformshift{0.556847in}{2.647515in}%
\pgfsys@useobject{currentmarker}{}%
\end{pgfscope}%
\end{pgfscope}%
\begin{pgfscope}%
\pgfsetbuttcap%
\pgfsetroundjoin%
\definecolor{currentfill}{rgb}{0.150000,0.150000,0.150000}%
\pgfsetfillcolor{currentfill}%
\pgfsetlinewidth{0.803000pt}%
\definecolor{currentstroke}{rgb}{0.150000,0.150000,0.150000}%
\pgfsetstrokecolor{currentstroke}%
\pgfsetdash{}{0pt}%
\pgfsys@defobject{currentmarker}{\pgfqpoint{0.000000in}{0.000000in}}{\pgfqpoint{0.000000in}{0.000000in}}{%
\pgfpathmoveto{\pgfqpoint{0.000000in}{0.000000in}}%
\pgfpathlineto{\pgfqpoint{0.000000in}{0.000000in}}%
\pgfusepath{stroke,fill}%
}%
\begin{pgfscope}%
\pgfsys@transformshift{3.134424in}{2.647515in}%
\pgfsys@useobject{currentmarker}{}%
\end{pgfscope}%
\end{pgfscope}%
\begin{pgfscope}%
\definecolor{textcolor}{rgb}{0.150000,0.150000,0.150000}%
\pgfsetstrokecolor{textcolor}%
\pgfsetfillcolor{textcolor}%
\pgftext[x=0.479069in,y=2.647515in,right,]{\color{textcolor}\sffamily\fontsize{8.000000}{9.600000}\selectfont 4.5}%
\end{pgfscope}%
\begin{pgfscope}%
\pgfpathrectangle{\pgfqpoint{0.556847in}{1.886339in}}{\pgfqpoint{2.577576in}{0.913411in}} %
\pgfusepath{clip}%
\pgfsetroundcap%
\pgfsetroundjoin%
\pgfsetlinewidth{0.803000pt}%
\definecolor{currentstroke}{rgb}{1.000000,1.000000,1.000000}%
\pgfsetstrokecolor{currentstroke}%
\pgfsetdash{}{0pt}%
\pgfpathmoveto{\pgfqpoint{0.556847in}{2.799750in}}%
\pgfpathlineto{\pgfqpoint{3.134424in}{2.799750in}}%
\pgfusepath{stroke}%
\end{pgfscope}%
\begin{pgfscope}%
\pgfsetbuttcap%
\pgfsetroundjoin%
\definecolor{currentfill}{rgb}{0.150000,0.150000,0.150000}%
\pgfsetfillcolor{currentfill}%
\pgfsetlinewidth{0.803000pt}%
\definecolor{currentstroke}{rgb}{0.150000,0.150000,0.150000}%
\pgfsetstrokecolor{currentstroke}%
\pgfsetdash{}{0pt}%
\pgfsys@defobject{currentmarker}{\pgfqpoint{0.000000in}{0.000000in}}{\pgfqpoint{0.000000in}{0.000000in}}{%
\pgfpathmoveto{\pgfqpoint{0.000000in}{0.000000in}}%
\pgfpathlineto{\pgfqpoint{0.000000in}{0.000000in}}%
\pgfusepath{stroke,fill}%
}%
\begin{pgfscope}%
\pgfsys@transformshift{0.556847in}{2.799750in}%
\pgfsys@useobject{currentmarker}{}%
\end{pgfscope}%
\end{pgfscope}%
\begin{pgfscope}%
\pgfsetbuttcap%
\pgfsetroundjoin%
\definecolor{currentfill}{rgb}{0.150000,0.150000,0.150000}%
\pgfsetfillcolor{currentfill}%
\pgfsetlinewidth{0.803000pt}%
\definecolor{currentstroke}{rgb}{0.150000,0.150000,0.150000}%
\pgfsetstrokecolor{currentstroke}%
\pgfsetdash{}{0pt}%
\pgfsys@defobject{currentmarker}{\pgfqpoint{0.000000in}{0.000000in}}{\pgfqpoint{0.000000in}{0.000000in}}{%
\pgfpathmoveto{\pgfqpoint{0.000000in}{0.000000in}}%
\pgfpathlineto{\pgfqpoint{0.000000in}{0.000000in}}%
\pgfusepath{stroke,fill}%
}%
\begin{pgfscope}%
\pgfsys@transformshift{3.134424in}{2.799750in}%
\pgfsys@useobject{currentmarker}{}%
\end{pgfscope}%
\end{pgfscope}%
\begin{pgfscope}%
\definecolor{textcolor}{rgb}{0.150000,0.150000,0.150000}%
\pgfsetstrokecolor{textcolor}%
\pgfsetfillcolor{textcolor}%
\pgftext[x=0.479069in,y=2.799750in,right,]{\color{textcolor}\sffamily\fontsize{8.000000}{9.600000}\selectfont 5.0}%
\end{pgfscope}%
\begin{pgfscope}%
\definecolor{textcolor}{rgb}{0.150000,0.150000,0.150000}%
\pgfsetstrokecolor{textcolor}%
\pgfsetfillcolor{textcolor}%
\pgftext[x=0.251677in,y=2.343044in,,bottom,rotate=90.000000]{\color{textcolor}\sffamily\fontsize{8.800000}{10.560000}\selectfont Flying time}%
\end{pgfscope}%
\begin{pgfscope}%
\pgfpathrectangle{\pgfqpoint{0.556847in}{1.886339in}}{\pgfqpoint{2.577576in}{0.913411in}} %
\pgfusepath{clip}%
\pgfsetbuttcap%
\pgfsetmiterjoin%
\definecolor{currentfill}{rgb}{0.447059,0.623529,0.811765}%
\pgfsetfillcolor{currentfill}%
\pgfsetfillopacity{0.300000}%
\pgfsetlinewidth{0.240900pt}%
\definecolor{currentstroke}{rgb}{0.447059,0.623529,0.811765}%
\pgfsetstrokecolor{currentstroke}%
\pgfsetstrokeopacity{0.300000}%
\pgfsetdash{}{0pt}%
\pgfpathmoveto{\pgfqpoint{0.700445in}{2.263597in}}%
\pgfpathlineto{\pgfqpoint{0.723580in}{2.266203in}}%
\pgfpathlineto{\pgfqpoint{0.746715in}{2.268883in}}%
\pgfpathlineto{\pgfqpoint{0.769850in}{2.271643in}}%
\pgfpathlineto{\pgfqpoint{0.792985in}{2.274486in}}%
\pgfpathlineto{\pgfqpoint{0.816121in}{2.277418in}}%
\pgfpathlineto{\pgfqpoint{0.839256in}{2.280444in}}%
\pgfpathlineto{\pgfqpoint{0.862391in}{2.283569in}}%
\pgfpathlineto{\pgfqpoint{0.885526in}{2.286798in}}%
\pgfpathlineto{\pgfqpoint{0.908661in}{2.290136in}}%
\pgfpathlineto{\pgfqpoint{0.931796in}{2.293586in}}%
\pgfpathlineto{\pgfqpoint{0.954932in}{2.297151in}}%
\pgfpathlineto{\pgfqpoint{0.978067in}{2.300834in}}%
\pgfpathlineto{\pgfqpoint{1.001202in}{2.304634in}}%
\pgfpathlineto{\pgfqpoint{1.024337in}{2.308551in}}%
\pgfpathlineto{\pgfqpoint{1.047472in}{2.312581in}}%
\pgfpathlineto{\pgfqpoint{1.070607in}{2.316719in}}%
\pgfpathlineto{\pgfqpoint{1.093743in}{2.320959in}}%
\pgfpathlineto{\pgfqpoint{1.116878in}{2.325292in}}%
\pgfpathlineto{\pgfqpoint{1.140013in}{2.329708in}}%
\pgfpathlineto{\pgfqpoint{1.163148in}{2.334195in}}%
\pgfpathlineto{\pgfqpoint{1.186283in}{2.338741in}}%
\pgfpathlineto{\pgfqpoint{1.209418in}{2.343333in}}%
\pgfpathlineto{\pgfqpoint{1.232554in}{2.347958in}}%
\pgfpathlineto{\pgfqpoint{1.255689in}{2.352603in}}%
\pgfpathlineto{\pgfqpoint{1.278824in}{2.357254in}}%
\pgfpathlineto{\pgfqpoint{1.301959in}{2.361901in}}%
\pgfpathlineto{\pgfqpoint{1.325094in}{2.366530in}}%
\pgfpathlineto{\pgfqpoint{1.348229in}{2.371131in}}%
\pgfpathlineto{\pgfqpoint{1.371365in}{2.375695in}}%
\pgfpathlineto{\pgfqpoint{1.394500in}{2.380213in}}%
\pgfpathlineto{\pgfqpoint{1.417635in}{2.384675in}}%
\pgfpathlineto{\pgfqpoint{1.440770in}{2.389075in}}%
\pgfpathlineto{\pgfqpoint{1.463905in}{2.393406in}}%
\pgfpathlineto{\pgfqpoint{1.487040in}{2.397662in}}%
\pgfpathlineto{\pgfqpoint{1.510176in}{2.401838in}}%
\pgfpathlineto{\pgfqpoint{1.533311in}{2.405929in}}%
\pgfpathlineto{\pgfqpoint{1.556446in}{2.409931in}}%
\pgfpathlineto{\pgfqpoint{1.579581in}{2.413840in}}%
\pgfpathlineto{\pgfqpoint{1.602716in}{2.417653in}}%
\pgfpathlineto{\pgfqpoint{1.625851in}{2.421368in}}%
\pgfpathlineto{\pgfqpoint{1.648987in}{2.424982in}}%
\pgfpathlineto{\pgfqpoint{1.672122in}{2.428493in}}%
\pgfpathlineto{\pgfqpoint{1.695257in}{2.431899in}}%
\pgfpathlineto{\pgfqpoint{1.718392in}{2.435200in}}%
\pgfpathlineto{\pgfqpoint{1.741527in}{2.438394in}}%
\pgfpathlineto{\pgfqpoint{1.764662in}{2.441480in}}%
\pgfpathlineto{\pgfqpoint{1.787798in}{2.444458in}}%
\pgfpathlineto{\pgfqpoint{1.810933in}{2.447328in}}%
\pgfpathlineto{\pgfqpoint{1.834068in}{2.450089in}}%
\pgfpathlineto{\pgfqpoint{1.857203in}{2.452742in}}%
\pgfpathlineto{\pgfqpoint{1.880338in}{2.455288in}}%
\pgfpathlineto{\pgfqpoint{1.903473in}{2.457726in}}%
\pgfpathlineto{\pgfqpoint{1.926608in}{2.460059in}}%
\pgfpathlineto{\pgfqpoint{1.949744in}{2.462287in}}%
\pgfpathlineto{\pgfqpoint{1.972879in}{2.464412in}}%
\pgfpathlineto{\pgfqpoint{1.996014in}{2.466436in}}%
\pgfpathlineto{\pgfqpoint{2.019149in}{2.468361in}}%
\pgfpathlineto{\pgfqpoint{2.042284in}{2.470190in}}%
\pgfpathlineto{\pgfqpoint{2.065419in}{2.471926in}}%
\pgfpathlineto{\pgfqpoint{2.088555in}{2.473572in}}%
\pgfpathlineto{\pgfqpoint{2.111690in}{2.475131in}}%
\pgfpathlineto{\pgfqpoint{2.134825in}{2.476609in}}%
\pgfpathlineto{\pgfqpoint{2.157960in}{2.478011in}}%
\pgfpathlineto{\pgfqpoint{2.181095in}{2.479341in}}%
\pgfpathlineto{\pgfqpoint{2.204230in}{2.480605in}}%
\pgfpathlineto{\pgfqpoint{2.227366in}{2.481811in}}%
\pgfpathlineto{\pgfqpoint{2.250501in}{2.482967in}}%
\pgfpathlineto{\pgfqpoint{2.273636in}{2.484079in}}%
\pgfpathlineto{\pgfqpoint{2.296771in}{2.485157in}}%
\pgfpathlineto{\pgfqpoint{2.319906in}{2.486211in}}%
\pgfpathlineto{\pgfqpoint{2.343041in}{2.487251in}}%
\pgfpathlineto{\pgfqpoint{2.366177in}{2.488288in}}%
\pgfpathlineto{\pgfqpoint{2.389312in}{2.489332in}}%
\pgfpathlineto{\pgfqpoint{2.412447in}{2.490397in}}%
\pgfpathlineto{\pgfqpoint{2.435582in}{2.491494in}}%
\pgfpathlineto{\pgfqpoint{2.458717in}{2.492634in}}%
\pgfpathlineto{\pgfqpoint{2.481852in}{2.493830in}}%
\pgfpathlineto{\pgfqpoint{2.504988in}{2.495093in}}%
\pgfpathlineto{\pgfqpoint{2.528123in}{2.496431in}}%
\pgfpathlineto{\pgfqpoint{2.551258in}{2.497855in}}%
\pgfpathlineto{\pgfqpoint{2.574393in}{2.499372in}}%
\pgfpathlineto{\pgfqpoint{2.597528in}{2.500988in}}%
\pgfpathlineto{\pgfqpoint{2.620663in}{2.502707in}}%
\pgfpathlineto{\pgfqpoint{2.643799in}{2.504533in}}%
\pgfpathlineto{\pgfqpoint{2.666934in}{2.506467in}}%
\pgfpathlineto{\pgfqpoint{2.690069in}{2.508509in}}%
\pgfpathlineto{\pgfqpoint{2.713204in}{2.510657in}}%
\pgfpathlineto{\pgfqpoint{2.736339in}{2.512910in}}%
\pgfpathlineto{\pgfqpoint{2.759474in}{2.515264in}}%
\pgfpathlineto{\pgfqpoint{2.782610in}{2.517717in}}%
\pgfpathlineto{\pgfqpoint{2.805745in}{2.520263in}}%
\pgfpathlineto{\pgfqpoint{2.828880in}{2.522900in}}%
\pgfpathlineto{\pgfqpoint{2.852015in}{2.525621in}}%
\pgfpathlineto{\pgfqpoint{2.875150in}{2.528424in}}%
\pgfpathlineto{\pgfqpoint{2.898285in}{2.531304in}}%
\pgfpathlineto{\pgfqpoint{2.921421in}{2.534256in}}%
\pgfpathlineto{\pgfqpoint{2.944556in}{2.537277in}}%
\pgfpathlineto{\pgfqpoint{2.967691in}{2.540364in}}%
\pgfpathlineto{\pgfqpoint{2.990826in}{2.543512in}}%
\pgfpathlineto{\pgfqpoint{2.990826in}{2.366189in}}%
\pgfpathlineto{\pgfqpoint{2.967691in}{2.370047in}}%
\pgfpathlineto{\pgfqpoint{2.944556in}{2.373718in}}%
\pgfpathlineto{\pgfqpoint{2.921421in}{2.377200in}}%
\pgfpathlineto{\pgfqpoint{2.898285in}{2.380489in}}%
\pgfpathlineto{\pgfqpoint{2.875150in}{2.383581in}}%
\pgfpathlineto{\pgfqpoint{2.852015in}{2.386471in}}%
\pgfpathlineto{\pgfqpoint{2.828880in}{2.389156in}}%
\pgfpathlineto{\pgfqpoint{2.805745in}{2.391631in}}%
\pgfpathlineto{\pgfqpoint{2.782610in}{2.393892in}}%
\pgfpathlineto{\pgfqpoint{2.759474in}{2.395935in}}%
\pgfpathlineto{\pgfqpoint{2.736339in}{2.397755in}}%
\pgfpathlineto{\pgfqpoint{2.713204in}{2.399349in}}%
\pgfpathlineto{\pgfqpoint{2.690069in}{2.400714in}}%
\pgfpathlineto{\pgfqpoint{2.666934in}{2.401848in}}%
\pgfpathlineto{\pgfqpoint{2.643799in}{2.402750in}}%
\pgfpathlineto{\pgfqpoint{2.620663in}{2.403419in}}%
\pgfpathlineto{\pgfqpoint{2.597528in}{2.403858in}}%
\pgfpathlineto{\pgfqpoint{2.574393in}{2.404068in}}%
\pgfpathlineto{\pgfqpoint{2.551258in}{2.404056in}}%
\pgfpathlineto{\pgfqpoint{2.528123in}{2.403826in}}%
\pgfpathlineto{\pgfqpoint{2.504988in}{2.403386in}}%
\pgfpathlineto{\pgfqpoint{2.481852in}{2.402745in}}%
\pgfpathlineto{\pgfqpoint{2.458717in}{2.401914in}}%
\pgfpathlineto{\pgfqpoint{2.435582in}{2.400903in}}%
\pgfpathlineto{\pgfqpoint{2.412447in}{2.399723in}}%
\pgfpathlineto{\pgfqpoint{2.389312in}{2.398388in}}%
\pgfpathlineto{\pgfqpoint{2.366177in}{2.396908in}}%
\pgfpathlineto{\pgfqpoint{2.343041in}{2.395295in}}%
\pgfpathlineto{\pgfqpoint{2.319906in}{2.393561in}}%
\pgfpathlineto{\pgfqpoint{2.296771in}{2.391717in}}%
\pgfpathlineto{\pgfqpoint{2.273636in}{2.389773in}}%
\pgfpathlineto{\pgfqpoint{2.250501in}{2.387738in}}%
\pgfpathlineto{\pgfqpoint{2.227366in}{2.385622in}}%
\pgfpathlineto{\pgfqpoint{2.204230in}{2.383433in}}%
\pgfpathlineto{\pgfqpoint{2.181095in}{2.381177in}}%
\pgfpathlineto{\pgfqpoint{2.157960in}{2.378863in}}%
\pgfpathlineto{\pgfqpoint{2.134825in}{2.376495in}}%
\pgfpathlineto{\pgfqpoint{2.111690in}{2.374080in}}%
\pgfpathlineto{\pgfqpoint{2.088555in}{2.371622in}}%
\pgfpathlineto{\pgfqpoint{2.065419in}{2.369125in}}%
\pgfpathlineto{\pgfqpoint{2.042284in}{2.366595in}}%
\pgfpathlineto{\pgfqpoint{2.019149in}{2.364033in}}%
\pgfpathlineto{\pgfqpoint{1.996014in}{2.361442in}}%
\pgfpathlineto{\pgfqpoint{1.972879in}{2.358827in}}%
\pgfpathlineto{\pgfqpoint{1.949744in}{2.356188in}}%
\pgfpathlineto{\pgfqpoint{1.926608in}{2.353527in}}%
\pgfpathlineto{\pgfqpoint{1.903473in}{2.350847in}}%
\pgfpathlineto{\pgfqpoint{1.880338in}{2.348148in}}%
\pgfpathlineto{\pgfqpoint{1.857203in}{2.345432in}}%
\pgfpathlineto{\pgfqpoint{1.834068in}{2.342699in}}%
\pgfpathlineto{\pgfqpoint{1.810933in}{2.339950in}}%
\pgfpathlineto{\pgfqpoint{1.787798in}{2.337185in}}%
\pgfpathlineto{\pgfqpoint{1.764662in}{2.334403in}}%
\pgfpathlineto{\pgfqpoint{1.741527in}{2.331606in}}%
\pgfpathlineto{\pgfqpoint{1.718392in}{2.328792in}}%
\pgfpathlineto{\pgfqpoint{1.695257in}{2.325960in}}%
\pgfpathlineto{\pgfqpoint{1.672122in}{2.323109in}}%
\pgfpathlineto{\pgfqpoint{1.648987in}{2.320239in}}%
\pgfpathlineto{\pgfqpoint{1.625851in}{2.317347in}}%
\pgfpathlineto{\pgfqpoint{1.602716in}{2.314432in}}%
\pgfpathlineto{\pgfqpoint{1.579581in}{2.311491in}}%
\pgfpathlineto{\pgfqpoint{1.556446in}{2.308521in}}%
\pgfpathlineto{\pgfqpoint{1.533311in}{2.305519in}}%
\pgfpathlineto{\pgfqpoint{1.510176in}{2.302482in}}%
\pgfpathlineto{\pgfqpoint{1.487040in}{2.299406in}}%
\pgfpathlineto{\pgfqpoint{1.463905in}{2.296285in}}%
\pgfpathlineto{\pgfqpoint{1.440770in}{2.293115in}}%
\pgfpathlineto{\pgfqpoint{1.417635in}{2.289890in}}%
\pgfpathlineto{\pgfqpoint{1.394500in}{2.286603in}}%
\pgfpathlineto{\pgfqpoint{1.371365in}{2.283246in}}%
\pgfpathlineto{\pgfqpoint{1.348229in}{2.279811in}}%
\pgfpathlineto{\pgfqpoint{1.325094in}{2.276290in}}%
\pgfpathlineto{\pgfqpoint{1.301959in}{2.272671in}}%
\pgfpathlineto{\pgfqpoint{1.278824in}{2.268946in}}%
\pgfpathlineto{\pgfqpoint{1.255689in}{2.265101in}}%
\pgfpathlineto{\pgfqpoint{1.232554in}{2.261125in}}%
\pgfpathlineto{\pgfqpoint{1.209418in}{2.257005in}}%
\pgfpathlineto{\pgfqpoint{1.186283in}{2.252728in}}%
\pgfpathlineto{\pgfqpoint{1.163148in}{2.248280in}}%
\pgfpathlineto{\pgfqpoint{1.140013in}{2.243649in}}%
\pgfpathlineto{\pgfqpoint{1.116878in}{2.238822in}}%
\pgfpathlineto{\pgfqpoint{1.093743in}{2.233788in}}%
\pgfpathlineto{\pgfqpoint{1.070607in}{2.228537in}}%
\pgfpathlineto{\pgfqpoint{1.047472in}{2.223059in}}%
\pgfpathlineto{\pgfqpoint{1.024337in}{2.217349in}}%
\pgfpathlineto{\pgfqpoint{1.001202in}{2.211401in}}%
\pgfpathlineto{\pgfqpoint{0.978067in}{2.205212in}}%
\pgfpathlineto{\pgfqpoint{0.954932in}{2.198782in}}%
\pgfpathlineto{\pgfqpoint{0.931796in}{2.192109in}}%
\pgfpathlineto{\pgfqpoint{0.908661in}{2.185197in}}%
\pgfpathlineto{\pgfqpoint{0.885526in}{2.178048in}}%
\pgfpathlineto{\pgfqpoint{0.862391in}{2.170666in}}%
\pgfpathlineto{\pgfqpoint{0.839256in}{2.163056in}}%
\pgfpathlineto{\pgfqpoint{0.816121in}{2.155222in}}%
\pgfpathlineto{\pgfqpoint{0.792985in}{2.147169in}}%
\pgfpathlineto{\pgfqpoint{0.769850in}{2.138904in}}%
\pgfpathlineto{\pgfqpoint{0.746715in}{2.130430in}}%
\pgfpathlineto{\pgfqpoint{0.723580in}{2.121753in}}%
\pgfpathlineto{\pgfqpoint{0.700445in}{2.112877in}}%
\pgfpathclose%
\pgfusepath{stroke,fill}%
\end{pgfscope}%
\begin{pgfscope}%
\pgfpathrectangle{\pgfqpoint{0.556847in}{1.886339in}}{\pgfqpoint{2.577576in}{0.913411in}} %
\pgfusepath{clip}%
\pgfsetroundcap%
\pgfsetroundjoin%
\pgfsetlinewidth{2.007500pt}%
\definecolor{currentstroke}{rgb}{0.125490,0.290196,0.529412}%
\pgfsetstrokecolor{currentstroke}%
\pgfsetdash{}{0pt}%
\pgfpathmoveto{\pgfqpoint{0.700445in}{2.188237in}}%
\pgfpathlineto{\pgfqpoint{0.723580in}{2.193978in}}%
\pgfpathlineto{\pgfqpoint{0.746715in}{2.199657in}}%
\pgfpathlineto{\pgfqpoint{0.769850in}{2.205273in}}%
\pgfpathlineto{\pgfqpoint{0.792985in}{2.210828in}}%
\pgfpathlineto{\pgfqpoint{0.816121in}{2.216320in}}%
\pgfpathlineto{\pgfqpoint{0.839256in}{2.221750in}}%
\pgfpathlineto{\pgfqpoint{0.862391in}{2.227117in}}%
\pgfpathlineto{\pgfqpoint{0.885526in}{2.232423in}}%
\pgfpathlineto{\pgfqpoint{0.908661in}{2.237666in}}%
\pgfpathlineto{\pgfqpoint{0.931796in}{2.242847in}}%
\pgfpathlineto{\pgfqpoint{0.954932in}{2.247966in}}%
\pgfpathlineto{\pgfqpoint{0.978067in}{2.253023in}}%
\pgfpathlineto{\pgfqpoint{1.001202in}{2.258018in}}%
\pgfpathlineto{\pgfqpoint{1.024337in}{2.262950in}}%
\pgfpathlineto{\pgfqpoint{1.047472in}{2.267820in}}%
\pgfpathlineto{\pgfqpoint{1.070607in}{2.272628in}}%
\pgfpathlineto{\pgfqpoint{1.093743in}{2.277374in}}%
\pgfpathlineto{\pgfqpoint{1.116878in}{2.282057in}}%
\pgfpathlineto{\pgfqpoint{1.140013in}{2.286678in}}%
\pgfpathlineto{\pgfqpoint{1.163148in}{2.291238in}}%
\pgfpathlineto{\pgfqpoint{1.186283in}{2.295734in}}%
\pgfpathlineto{\pgfqpoint{1.209418in}{2.300169in}}%
\pgfpathlineto{\pgfqpoint{1.232554in}{2.304542in}}%
\pgfpathlineto{\pgfqpoint{1.255689in}{2.308852in}}%
\pgfpathlineto{\pgfqpoint{1.278824in}{2.313100in}}%
\pgfpathlineto{\pgfqpoint{1.301959in}{2.317286in}}%
\pgfpathlineto{\pgfqpoint{1.325094in}{2.321410in}}%
\pgfpathlineto{\pgfqpoint{1.348229in}{2.325471in}}%
\pgfpathlineto{\pgfqpoint{1.371365in}{2.329471in}}%
\pgfpathlineto{\pgfqpoint{1.394500in}{2.333408in}}%
\pgfpathlineto{\pgfqpoint{1.417635in}{2.337282in}}%
\pgfpathlineto{\pgfqpoint{1.440770in}{2.341095in}}%
\pgfpathlineto{\pgfqpoint{1.463905in}{2.344846in}}%
\pgfpathlineto{\pgfqpoint{1.487040in}{2.348534in}}%
\pgfpathlineto{\pgfqpoint{1.510176in}{2.352160in}}%
\pgfpathlineto{\pgfqpoint{1.533311in}{2.355724in}}%
\pgfpathlineto{\pgfqpoint{1.556446in}{2.359226in}}%
\pgfpathlineto{\pgfqpoint{1.579581in}{2.362665in}}%
\pgfpathlineto{\pgfqpoint{1.602716in}{2.366042in}}%
\pgfpathlineto{\pgfqpoint{1.625851in}{2.369357in}}%
\pgfpathlineto{\pgfqpoint{1.648987in}{2.372610in}}%
\pgfpathlineto{\pgfqpoint{1.672122in}{2.375801in}}%
\pgfpathlineto{\pgfqpoint{1.695257in}{2.378930in}}%
\pgfpathlineto{\pgfqpoint{1.718392in}{2.381996in}}%
\pgfpathlineto{\pgfqpoint{1.741527in}{2.385000in}}%
\pgfpathlineto{\pgfqpoint{1.764662in}{2.387942in}}%
\pgfpathlineto{\pgfqpoint{1.787798in}{2.390821in}}%
\pgfpathlineto{\pgfqpoint{1.810933in}{2.393639in}}%
\pgfpathlineto{\pgfqpoint{1.834068in}{2.396394in}}%
\pgfpathlineto{\pgfqpoint{1.857203in}{2.399087in}}%
\pgfpathlineto{\pgfqpoint{1.880338in}{2.401718in}}%
\pgfpathlineto{\pgfqpoint{1.903473in}{2.404287in}}%
\pgfpathlineto{\pgfqpoint{1.926608in}{2.406793in}}%
\pgfpathlineto{\pgfqpoint{1.949744in}{2.409237in}}%
\pgfpathlineto{\pgfqpoint{1.972879in}{2.411619in}}%
\pgfpathlineto{\pgfqpoint{1.996014in}{2.413939in}}%
\pgfpathlineto{\pgfqpoint{2.019149in}{2.416197in}}%
\pgfpathlineto{\pgfqpoint{2.042284in}{2.418392in}}%
\pgfpathlineto{\pgfqpoint{2.065419in}{2.420526in}}%
\pgfpathlineto{\pgfqpoint{2.088555in}{2.422597in}}%
\pgfpathlineto{\pgfqpoint{2.111690in}{2.424606in}}%
\pgfpathlineto{\pgfqpoint{2.134825in}{2.426552in}}%
\pgfpathlineto{\pgfqpoint{2.157960in}{2.428437in}}%
\pgfpathlineto{\pgfqpoint{2.181095in}{2.430259in}}%
\pgfpathlineto{\pgfqpoint{2.204230in}{2.432019in}}%
\pgfpathlineto{\pgfqpoint{2.227366in}{2.433717in}}%
\pgfpathlineto{\pgfqpoint{2.250501in}{2.435352in}}%
\pgfpathlineto{\pgfqpoint{2.273636in}{2.436926in}}%
\pgfpathlineto{\pgfqpoint{2.296771in}{2.438437in}}%
\pgfpathlineto{\pgfqpoint{2.319906in}{2.439886in}}%
\pgfpathlineto{\pgfqpoint{2.343041in}{2.441273in}}%
\pgfpathlineto{\pgfqpoint{2.366177in}{2.442598in}}%
\pgfpathlineto{\pgfqpoint{2.389312in}{2.443860in}}%
\pgfpathlineto{\pgfqpoint{2.412447in}{2.445060in}}%
\pgfpathlineto{\pgfqpoint{2.435582in}{2.446198in}}%
\pgfpathlineto{\pgfqpoint{2.458717in}{2.447274in}}%
\pgfpathlineto{\pgfqpoint{2.481852in}{2.448288in}}%
\pgfpathlineto{\pgfqpoint{2.504988in}{2.449239in}}%
\pgfpathlineto{\pgfqpoint{2.528123in}{2.450128in}}%
\pgfpathlineto{\pgfqpoint{2.551258in}{2.450955in}}%
\pgfpathlineto{\pgfqpoint{2.574393in}{2.451720in}}%
\pgfpathlineto{\pgfqpoint{2.597528in}{2.452423in}}%
\pgfpathlineto{\pgfqpoint{2.620663in}{2.453063in}}%
\pgfpathlineto{\pgfqpoint{2.643799in}{2.453641in}}%
\pgfpathlineto{\pgfqpoint{2.666934in}{2.454157in}}%
\pgfpathlineto{\pgfqpoint{2.690069in}{2.454611in}}%
\pgfpathlineto{\pgfqpoint{2.713204in}{2.455003in}}%
\pgfpathlineto{\pgfqpoint{2.736339in}{2.455332in}}%
\pgfpathlineto{\pgfqpoint{2.759474in}{2.455599in}}%
\pgfpathlineto{\pgfqpoint{2.782610in}{2.455804in}}%
\pgfpathlineto{\pgfqpoint{2.805745in}{2.455947in}}%
\pgfpathlineto{\pgfqpoint{2.828880in}{2.456028in}}%
\pgfpathlineto{\pgfqpoint{2.852015in}{2.456046in}}%
\pgfpathlineto{\pgfqpoint{2.875150in}{2.456002in}}%
\pgfpathlineto{\pgfqpoint{2.898285in}{2.455896in}}%
\pgfpathlineto{\pgfqpoint{2.921421in}{2.455728in}}%
\pgfpathlineto{\pgfqpoint{2.944556in}{2.455498in}}%
\pgfpathlineto{\pgfqpoint{2.967691in}{2.455205in}}%
\pgfpathlineto{\pgfqpoint{2.990826in}{2.454850in}}%
\pgfusepath{stroke}%
\end{pgfscope}%
\begin{pgfscope}%
\pgfpathrectangle{\pgfqpoint{0.556847in}{1.886339in}}{\pgfqpoint{2.577576in}{0.913411in}} %
\pgfusepath{clip}%
\pgfsetroundcap%
\pgfsetroundjoin%
\pgfsetlinewidth{0.200750pt}%
\definecolor{currentstroke}{rgb}{0.125490,0.290196,0.529412}%
\pgfsetstrokecolor{currentstroke}%
\pgfsetdash{}{0pt}%
\pgfpathmoveto{\pgfqpoint{0.700445in}{2.263597in}}%
\pgfpathlineto{\pgfqpoint{0.723580in}{2.266203in}}%
\pgfpathlineto{\pgfqpoint{0.746715in}{2.268883in}}%
\pgfpathlineto{\pgfqpoint{0.769850in}{2.271643in}}%
\pgfpathlineto{\pgfqpoint{0.792985in}{2.274486in}}%
\pgfpathlineto{\pgfqpoint{0.816121in}{2.277418in}}%
\pgfpathlineto{\pgfqpoint{0.839256in}{2.280444in}}%
\pgfpathlineto{\pgfqpoint{0.862391in}{2.283569in}}%
\pgfpathlineto{\pgfqpoint{0.885526in}{2.286798in}}%
\pgfpathlineto{\pgfqpoint{0.908661in}{2.290136in}}%
\pgfpathlineto{\pgfqpoint{0.931796in}{2.293586in}}%
\pgfpathlineto{\pgfqpoint{0.954932in}{2.297151in}}%
\pgfpathlineto{\pgfqpoint{0.978067in}{2.300834in}}%
\pgfpathlineto{\pgfqpoint{1.001202in}{2.304634in}}%
\pgfpathlineto{\pgfqpoint{1.024337in}{2.308551in}}%
\pgfpathlineto{\pgfqpoint{1.047472in}{2.312581in}}%
\pgfpathlineto{\pgfqpoint{1.070607in}{2.316719in}}%
\pgfpathlineto{\pgfqpoint{1.093743in}{2.320959in}}%
\pgfpathlineto{\pgfqpoint{1.116878in}{2.325292in}}%
\pgfpathlineto{\pgfqpoint{1.140013in}{2.329708in}}%
\pgfpathlineto{\pgfqpoint{1.163148in}{2.334195in}}%
\pgfpathlineto{\pgfqpoint{1.186283in}{2.338741in}}%
\pgfpathlineto{\pgfqpoint{1.209418in}{2.343333in}}%
\pgfpathlineto{\pgfqpoint{1.232554in}{2.347958in}}%
\pgfpathlineto{\pgfqpoint{1.255689in}{2.352603in}}%
\pgfpathlineto{\pgfqpoint{1.278824in}{2.357254in}}%
\pgfpathlineto{\pgfqpoint{1.301959in}{2.361901in}}%
\pgfpathlineto{\pgfqpoint{1.325094in}{2.366530in}}%
\pgfpathlineto{\pgfqpoint{1.348229in}{2.371131in}}%
\pgfpathlineto{\pgfqpoint{1.371365in}{2.375695in}}%
\pgfpathlineto{\pgfqpoint{1.394500in}{2.380213in}}%
\pgfpathlineto{\pgfqpoint{1.417635in}{2.384675in}}%
\pgfpathlineto{\pgfqpoint{1.440770in}{2.389075in}}%
\pgfpathlineto{\pgfqpoint{1.463905in}{2.393406in}}%
\pgfpathlineto{\pgfqpoint{1.487040in}{2.397662in}}%
\pgfpathlineto{\pgfqpoint{1.510176in}{2.401838in}}%
\pgfpathlineto{\pgfqpoint{1.533311in}{2.405929in}}%
\pgfpathlineto{\pgfqpoint{1.556446in}{2.409931in}}%
\pgfpathlineto{\pgfqpoint{1.579581in}{2.413840in}}%
\pgfpathlineto{\pgfqpoint{1.602716in}{2.417653in}}%
\pgfpathlineto{\pgfqpoint{1.625851in}{2.421368in}}%
\pgfpathlineto{\pgfqpoint{1.648987in}{2.424982in}}%
\pgfpathlineto{\pgfqpoint{1.672122in}{2.428493in}}%
\pgfpathlineto{\pgfqpoint{1.695257in}{2.431899in}}%
\pgfpathlineto{\pgfqpoint{1.718392in}{2.435200in}}%
\pgfpathlineto{\pgfqpoint{1.741527in}{2.438394in}}%
\pgfpathlineto{\pgfqpoint{1.764662in}{2.441480in}}%
\pgfpathlineto{\pgfqpoint{1.787798in}{2.444458in}}%
\pgfpathlineto{\pgfqpoint{1.810933in}{2.447328in}}%
\pgfpathlineto{\pgfqpoint{1.834068in}{2.450089in}}%
\pgfpathlineto{\pgfqpoint{1.857203in}{2.452742in}}%
\pgfpathlineto{\pgfqpoint{1.880338in}{2.455288in}}%
\pgfpathlineto{\pgfqpoint{1.903473in}{2.457726in}}%
\pgfpathlineto{\pgfqpoint{1.926608in}{2.460059in}}%
\pgfpathlineto{\pgfqpoint{1.949744in}{2.462287in}}%
\pgfpathlineto{\pgfqpoint{1.972879in}{2.464412in}}%
\pgfpathlineto{\pgfqpoint{1.996014in}{2.466436in}}%
\pgfpathlineto{\pgfqpoint{2.019149in}{2.468361in}}%
\pgfpathlineto{\pgfqpoint{2.042284in}{2.470190in}}%
\pgfpathlineto{\pgfqpoint{2.065419in}{2.471926in}}%
\pgfpathlineto{\pgfqpoint{2.088555in}{2.473572in}}%
\pgfpathlineto{\pgfqpoint{2.111690in}{2.475131in}}%
\pgfpathlineto{\pgfqpoint{2.134825in}{2.476609in}}%
\pgfpathlineto{\pgfqpoint{2.157960in}{2.478011in}}%
\pgfpathlineto{\pgfqpoint{2.181095in}{2.479341in}}%
\pgfpathlineto{\pgfqpoint{2.204230in}{2.480605in}}%
\pgfpathlineto{\pgfqpoint{2.227366in}{2.481811in}}%
\pgfpathlineto{\pgfqpoint{2.250501in}{2.482967in}}%
\pgfpathlineto{\pgfqpoint{2.273636in}{2.484079in}}%
\pgfpathlineto{\pgfqpoint{2.296771in}{2.485157in}}%
\pgfpathlineto{\pgfqpoint{2.319906in}{2.486211in}}%
\pgfpathlineto{\pgfqpoint{2.343041in}{2.487251in}}%
\pgfpathlineto{\pgfqpoint{2.366177in}{2.488288in}}%
\pgfpathlineto{\pgfqpoint{2.389312in}{2.489332in}}%
\pgfpathlineto{\pgfqpoint{2.412447in}{2.490397in}}%
\pgfpathlineto{\pgfqpoint{2.435582in}{2.491494in}}%
\pgfpathlineto{\pgfqpoint{2.458717in}{2.492634in}}%
\pgfpathlineto{\pgfqpoint{2.481852in}{2.493830in}}%
\pgfpathlineto{\pgfqpoint{2.504988in}{2.495093in}}%
\pgfpathlineto{\pgfqpoint{2.528123in}{2.496431in}}%
\pgfpathlineto{\pgfqpoint{2.551258in}{2.497855in}}%
\pgfpathlineto{\pgfqpoint{2.574393in}{2.499372in}}%
\pgfpathlineto{\pgfqpoint{2.597528in}{2.500988in}}%
\pgfpathlineto{\pgfqpoint{2.620663in}{2.502707in}}%
\pgfpathlineto{\pgfqpoint{2.643799in}{2.504533in}}%
\pgfpathlineto{\pgfqpoint{2.666934in}{2.506467in}}%
\pgfpathlineto{\pgfqpoint{2.690069in}{2.508509in}}%
\pgfpathlineto{\pgfqpoint{2.713204in}{2.510657in}}%
\pgfpathlineto{\pgfqpoint{2.736339in}{2.512910in}}%
\pgfpathlineto{\pgfqpoint{2.759474in}{2.515264in}}%
\pgfpathlineto{\pgfqpoint{2.782610in}{2.517717in}}%
\pgfpathlineto{\pgfqpoint{2.805745in}{2.520263in}}%
\pgfpathlineto{\pgfqpoint{2.828880in}{2.522900in}}%
\pgfpathlineto{\pgfqpoint{2.852015in}{2.525621in}}%
\pgfpathlineto{\pgfqpoint{2.875150in}{2.528424in}}%
\pgfpathlineto{\pgfqpoint{2.898285in}{2.531304in}}%
\pgfpathlineto{\pgfqpoint{2.921421in}{2.534256in}}%
\pgfpathlineto{\pgfqpoint{2.944556in}{2.537277in}}%
\pgfpathlineto{\pgfqpoint{2.967691in}{2.540364in}}%
\pgfpathlineto{\pgfqpoint{2.990826in}{2.543512in}}%
\pgfusepath{stroke}%
\end{pgfscope}%
\begin{pgfscope}%
\pgfpathrectangle{\pgfqpoint{0.556847in}{1.886339in}}{\pgfqpoint{2.577576in}{0.913411in}} %
\pgfusepath{clip}%
\pgfsetroundcap%
\pgfsetroundjoin%
\pgfsetlinewidth{0.200750pt}%
\definecolor{currentstroke}{rgb}{0.125490,0.290196,0.529412}%
\pgfsetstrokecolor{currentstroke}%
\pgfsetdash{}{0pt}%
\pgfpathmoveto{\pgfqpoint{0.700445in}{2.112877in}}%
\pgfpathlineto{\pgfqpoint{0.723580in}{2.121753in}}%
\pgfpathlineto{\pgfqpoint{0.746715in}{2.130430in}}%
\pgfpathlineto{\pgfqpoint{0.769850in}{2.138904in}}%
\pgfpathlineto{\pgfqpoint{0.792985in}{2.147169in}}%
\pgfpathlineto{\pgfqpoint{0.816121in}{2.155222in}}%
\pgfpathlineto{\pgfqpoint{0.839256in}{2.163056in}}%
\pgfpathlineto{\pgfqpoint{0.862391in}{2.170666in}}%
\pgfpathlineto{\pgfqpoint{0.885526in}{2.178048in}}%
\pgfpathlineto{\pgfqpoint{0.908661in}{2.185197in}}%
\pgfpathlineto{\pgfqpoint{0.931796in}{2.192109in}}%
\pgfpathlineto{\pgfqpoint{0.954932in}{2.198782in}}%
\pgfpathlineto{\pgfqpoint{0.978067in}{2.205212in}}%
\pgfpathlineto{\pgfqpoint{1.001202in}{2.211401in}}%
\pgfpathlineto{\pgfqpoint{1.024337in}{2.217349in}}%
\pgfpathlineto{\pgfqpoint{1.047472in}{2.223059in}}%
\pgfpathlineto{\pgfqpoint{1.070607in}{2.228537in}}%
\pgfpathlineto{\pgfqpoint{1.093743in}{2.233788in}}%
\pgfpathlineto{\pgfqpoint{1.116878in}{2.238822in}}%
\pgfpathlineto{\pgfqpoint{1.140013in}{2.243649in}}%
\pgfpathlineto{\pgfqpoint{1.163148in}{2.248280in}}%
\pgfpathlineto{\pgfqpoint{1.186283in}{2.252728in}}%
\pgfpathlineto{\pgfqpoint{1.209418in}{2.257005in}}%
\pgfpathlineto{\pgfqpoint{1.232554in}{2.261125in}}%
\pgfpathlineto{\pgfqpoint{1.255689in}{2.265101in}}%
\pgfpathlineto{\pgfqpoint{1.278824in}{2.268946in}}%
\pgfpathlineto{\pgfqpoint{1.301959in}{2.272671in}}%
\pgfpathlineto{\pgfqpoint{1.325094in}{2.276290in}}%
\pgfpathlineto{\pgfqpoint{1.348229in}{2.279811in}}%
\pgfpathlineto{\pgfqpoint{1.371365in}{2.283246in}}%
\pgfpathlineto{\pgfqpoint{1.394500in}{2.286603in}}%
\pgfpathlineto{\pgfqpoint{1.417635in}{2.289890in}}%
\pgfpathlineto{\pgfqpoint{1.440770in}{2.293115in}}%
\pgfpathlineto{\pgfqpoint{1.463905in}{2.296285in}}%
\pgfpathlineto{\pgfqpoint{1.487040in}{2.299406in}}%
\pgfpathlineto{\pgfqpoint{1.510176in}{2.302482in}}%
\pgfpathlineto{\pgfqpoint{1.533311in}{2.305519in}}%
\pgfpathlineto{\pgfqpoint{1.556446in}{2.308521in}}%
\pgfpathlineto{\pgfqpoint{1.579581in}{2.311491in}}%
\pgfpathlineto{\pgfqpoint{1.602716in}{2.314432in}}%
\pgfpathlineto{\pgfqpoint{1.625851in}{2.317347in}}%
\pgfpathlineto{\pgfqpoint{1.648987in}{2.320239in}}%
\pgfpathlineto{\pgfqpoint{1.672122in}{2.323109in}}%
\pgfpathlineto{\pgfqpoint{1.695257in}{2.325960in}}%
\pgfpathlineto{\pgfqpoint{1.718392in}{2.328792in}}%
\pgfpathlineto{\pgfqpoint{1.741527in}{2.331606in}}%
\pgfpathlineto{\pgfqpoint{1.764662in}{2.334403in}}%
\pgfpathlineto{\pgfqpoint{1.787798in}{2.337185in}}%
\pgfpathlineto{\pgfqpoint{1.810933in}{2.339950in}}%
\pgfpathlineto{\pgfqpoint{1.834068in}{2.342699in}}%
\pgfpathlineto{\pgfqpoint{1.857203in}{2.345432in}}%
\pgfpathlineto{\pgfqpoint{1.880338in}{2.348148in}}%
\pgfpathlineto{\pgfqpoint{1.903473in}{2.350847in}}%
\pgfpathlineto{\pgfqpoint{1.926608in}{2.353527in}}%
\pgfpathlineto{\pgfqpoint{1.949744in}{2.356188in}}%
\pgfpathlineto{\pgfqpoint{1.972879in}{2.358827in}}%
\pgfpathlineto{\pgfqpoint{1.996014in}{2.361442in}}%
\pgfpathlineto{\pgfqpoint{2.019149in}{2.364033in}}%
\pgfpathlineto{\pgfqpoint{2.042284in}{2.366595in}}%
\pgfpathlineto{\pgfqpoint{2.065419in}{2.369125in}}%
\pgfpathlineto{\pgfqpoint{2.088555in}{2.371622in}}%
\pgfpathlineto{\pgfqpoint{2.111690in}{2.374080in}}%
\pgfpathlineto{\pgfqpoint{2.134825in}{2.376495in}}%
\pgfpathlineto{\pgfqpoint{2.157960in}{2.378863in}}%
\pgfpathlineto{\pgfqpoint{2.181095in}{2.381177in}}%
\pgfpathlineto{\pgfqpoint{2.204230in}{2.383433in}}%
\pgfpathlineto{\pgfqpoint{2.227366in}{2.385622in}}%
\pgfpathlineto{\pgfqpoint{2.250501in}{2.387738in}}%
\pgfpathlineto{\pgfqpoint{2.273636in}{2.389773in}}%
\pgfpathlineto{\pgfqpoint{2.296771in}{2.391717in}}%
\pgfpathlineto{\pgfqpoint{2.319906in}{2.393561in}}%
\pgfpathlineto{\pgfqpoint{2.343041in}{2.395295in}}%
\pgfpathlineto{\pgfqpoint{2.366177in}{2.396908in}}%
\pgfpathlineto{\pgfqpoint{2.389312in}{2.398388in}}%
\pgfpathlineto{\pgfqpoint{2.412447in}{2.399723in}}%
\pgfpathlineto{\pgfqpoint{2.435582in}{2.400903in}}%
\pgfpathlineto{\pgfqpoint{2.458717in}{2.401914in}}%
\pgfpathlineto{\pgfqpoint{2.481852in}{2.402745in}}%
\pgfpathlineto{\pgfqpoint{2.504988in}{2.403386in}}%
\pgfpathlineto{\pgfqpoint{2.528123in}{2.403826in}}%
\pgfpathlineto{\pgfqpoint{2.551258in}{2.404056in}}%
\pgfpathlineto{\pgfqpoint{2.574393in}{2.404068in}}%
\pgfpathlineto{\pgfqpoint{2.597528in}{2.403858in}}%
\pgfpathlineto{\pgfqpoint{2.620663in}{2.403419in}}%
\pgfpathlineto{\pgfqpoint{2.643799in}{2.402750in}}%
\pgfpathlineto{\pgfqpoint{2.666934in}{2.401848in}}%
\pgfpathlineto{\pgfqpoint{2.690069in}{2.400714in}}%
\pgfpathlineto{\pgfqpoint{2.713204in}{2.399349in}}%
\pgfpathlineto{\pgfqpoint{2.736339in}{2.397755in}}%
\pgfpathlineto{\pgfqpoint{2.759474in}{2.395935in}}%
\pgfpathlineto{\pgfqpoint{2.782610in}{2.393892in}}%
\pgfpathlineto{\pgfqpoint{2.805745in}{2.391631in}}%
\pgfpathlineto{\pgfqpoint{2.828880in}{2.389156in}}%
\pgfpathlineto{\pgfqpoint{2.852015in}{2.386471in}}%
\pgfpathlineto{\pgfqpoint{2.875150in}{2.383581in}}%
\pgfpathlineto{\pgfqpoint{2.898285in}{2.380489in}}%
\pgfpathlineto{\pgfqpoint{2.921421in}{2.377200in}}%
\pgfpathlineto{\pgfqpoint{2.944556in}{2.373718in}}%
\pgfpathlineto{\pgfqpoint{2.967691in}{2.370047in}}%
\pgfpathlineto{\pgfqpoint{2.990826in}{2.366189in}}%
\pgfusepath{stroke}%
\end{pgfscope}%
\begin{pgfscope}%
\pgfpathrectangle{\pgfqpoint{0.556847in}{1.886339in}}{\pgfqpoint{2.577576in}{0.913411in}} %
\pgfusepath{clip}%
\pgfsetbuttcap%
\pgfsetbeveljoin%
\definecolor{currentfill}{rgb}{0.298039,0.447059,0.690196}%
\pgfsetfillcolor{currentfill}%
\pgfsetlinewidth{0.000000pt}%
\definecolor{currentstroke}{rgb}{0.000000,0.000000,0.000000}%
\pgfsetstrokecolor{currentstroke}%
\pgfsetdash{}{0pt}%
\pgfsys@defobject{currentmarker}{\pgfqpoint{-0.036986in}{-0.031462in}}{\pgfqpoint{0.036986in}{0.038889in}}{%
\pgfpathmoveto{\pgfqpoint{0.000000in}{0.038889in}}%
\pgfpathlineto{\pgfqpoint{-0.008731in}{0.012017in}}%
\pgfpathlineto{\pgfqpoint{-0.036986in}{0.012017in}}%
\pgfpathlineto{\pgfqpoint{-0.014127in}{-0.004590in}}%
\pgfpathlineto{\pgfqpoint{-0.022858in}{-0.031462in}}%
\pgfpathlineto{\pgfqpoint{-0.000000in}{-0.014854in}}%
\pgfpathlineto{\pgfqpoint{0.022858in}{-0.031462in}}%
\pgfpathlineto{\pgfqpoint{0.014127in}{-0.004590in}}%
\pgfpathlineto{\pgfqpoint{0.036986in}{0.012017in}}%
\pgfpathlineto{\pgfqpoint{0.008731in}{0.012017in}}%
\pgfpathclose%
\pgfusepath{fill}%
}%
\begin{pgfscope}%
\pgfsys@transformshift{1.475872in}{2.449609in}%
\pgfsys@useobject{currentmarker}{}%
\end{pgfscope}%
\begin{pgfscope}%
\pgfsys@transformshift{0.980460in}{2.327821in}%
\pgfsys@useobject{currentmarker}{}%
\end{pgfscope}%
\begin{pgfscope}%
\pgfsys@transformshift{1.741527in}{2.434386in}%
\pgfsys@useobject{currentmarker}{}%
\end{pgfscope}%
\begin{pgfscope}%
\pgfsys@transformshift{2.301558in}{2.632291in}%
\pgfsys@useobject{currentmarker}{}%
\end{pgfscope}%
\begin{pgfscope}%
\pgfsys@transformshift{1.195856in}{2.312597in}%
\pgfsys@useobject{currentmarker}{}%
\end{pgfscope}%
\begin{pgfscope}%
\pgfsys@transformshift{2.897488in}{2.693185in}%
\pgfsys@useobject{currentmarker}{}%
\end{pgfscope}%
\begin{pgfscope}%
\pgfsys@transformshift{0.808143in}{2.312597in}%
\pgfsys@useobject{currentmarker}{}%
\end{pgfscope}%
\begin{pgfscope}%
\pgfsys@transformshift{2.452335in}{2.754079in}%
\pgfsys@useobject{currentmarker}{}%
\end{pgfscope}%
\begin{pgfscope}%
\pgfsys@transformshift{0.966100in}{2.053798in}%
\pgfsys@useobject{currentmarker}{}%
\end{pgfscope}%
\begin{pgfscope}%
\pgfsys@transformshift{1.863585in}{2.343044in}%
\pgfsys@useobject{currentmarker}{}%
\end{pgfscope}%
\begin{pgfscope}%
\pgfsys@transformshift{0.915841in}{2.297374in}%
\pgfsys@useobject{currentmarker}{}%
\end{pgfscope}%
\begin{pgfscope}%
\pgfsys@transformshift{0.894302in}{2.312597in}%
\pgfsys@useobject{currentmarker}{}%
\end{pgfscope}%
\begin{pgfscope}%
\pgfsys@transformshift{2.990826in}{2.358268in}%
\pgfsys@useobject{currentmarker}{}%
\end{pgfscope}%
\begin{pgfscope}%
\pgfsys@transformshift{1.282015in}{1.992904in}%
\pgfsys@useobject{currentmarker}{}%
\end{pgfscope}%
\begin{pgfscope}%
\pgfsys@transformshift{1.964103in}{2.525727in}%
\pgfsys@useobject{currentmarker}{}%
\end{pgfscope}%
\begin{pgfscope}%
\pgfsys@transformshift{1.827686in}{2.480056in}%
\pgfsys@useobject{currentmarker}{}%
\end{pgfscope}%
\begin{pgfscope}%
\pgfsys@transformshift{0.700445in}{2.236480in}%
\pgfsys@useobject{currentmarker}{}%
\end{pgfscope}%
\begin{pgfscope}%
\pgfsys@transformshift{1.059439in}{2.327821in}%
\pgfsys@useobject{currentmarker}{}%
\end{pgfscope}%
\begin{pgfscope}%
\pgfsys@transformshift{1.188677in}{2.114692in}%
\pgfsys@useobject{currentmarker}{}%
\end{pgfscope}%
\begin{pgfscope}%
\pgfsys@transformshift{0.793783in}{2.251703in}%
\pgfsys@useobject{currentmarker}{}%
\end{pgfscope}%
\begin{pgfscope}%
\pgfsys@transformshift{2.768250in}{2.327821in}%
\pgfsys@useobject{currentmarker}{}%
\end{pgfscope}%
\begin{pgfscope}%
\pgfsys@transformshift{1.856405in}{2.540950in}%
\pgfsys@useobject{currentmarker}{}%
\end{pgfscope}%
\begin{pgfscope}%
\pgfsys@transformshift{2.761070in}{2.373491in}%
\pgfsys@useobject{currentmarker}{}%
\end{pgfscope}%
\begin{pgfscope}%
\pgfsys@transformshift{2.358997in}{2.160362in}%
\pgfsys@useobject{currentmarker}{}%
\end{pgfscope}%
\begin{pgfscope}%
\pgfsys@transformshift{2.380536in}{2.419162in}%
\pgfsys@useobject{currentmarker}{}%
\end{pgfscope}%
\begin{pgfscope}%
\pgfsys@transformshift{2.308738in}{2.617068in}%
\pgfsys@useobject{currentmarker}{}%
\end{pgfscope}%
\begin{pgfscope}%
\pgfsys@transformshift{0.786603in}{2.175586in}%
\pgfsys@useobject{currentmarker}{}%
\end{pgfscope}%
\begin{pgfscope}%
\pgfsys@transformshift{2.746710in}{2.069021in}%
\pgfsys@useobject{currentmarker}{}%
\end{pgfscope}%
\begin{pgfscope}%
\pgfsys@transformshift{1.842045in}{2.129915in}%
\pgfsys@useobject{currentmarker}{}%
\end{pgfscope}%
\begin{pgfscope}%
\pgfsys@transformshift{2.818509in}{2.632291in}%
\pgfsys@useobject{currentmarker}{}%
\end{pgfscope}%
\end{pgfscope}%
\begin{pgfscope}%
\pgfsetrectcap%
\pgfsetmiterjoin%
\pgfsetlinewidth{0.000000pt}%
\definecolor{currentstroke}{rgb}{1.000000,1.000000,1.000000}%
\pgfsetstrokecolor{currentstroke}%
\pgfsetdash{}{0pt}%
\pgfpathmoveto{\pgfqpoint{3.134424in}{1.886339in}}%
\pgfpathlineto{\pgfqpoint{3.134424in}{2.799750in}}%
\pgfusepath{}%
\end{pgfscope}%
\begin{pgfscope}%
\pgfsetrectcap%
\pgfsetmiterjoin%
\pgfsetlinewidth{0.000000pt}%
\definecolor{currentstroke}{rgb}{1.000000,1.000000,1.000000}%
\pgfsetstrokecolor{currentstroke}%
\pgfsetdash{}{0pt}%
\pgfpathmoveto{\pgfqpoint{0.556847in}{1.886339in}}%
\pgfpathlineto{\pgfqpoint{3.134424in}{1.886339in}}%
\pgfusepath{}%
\end{pgfscope}%
\begin{pgfscope}%
\pgfsetrectcap%
\pgfsetmiterjoin%
\pgfsetlinewidth{0.000000pt}%
\definecolor{currentstroke}{rgb}{1.000000,1.000000,1.000000}%
\pgfsetstrokecolor{currentstroke}%
\pgfsetdash{}{0pt}%
\pgfpathmoveto{\pgfqpoint{0.556847in}{1.886339in}}%
\pgfpathlineto{\pgfqpoint{0.556847in}{2.799750in}}%
\pgfusepath{}%
\end{pgfscope}%
\begin{pgfscope}%
\pgfsetrectcap%
\pgfsetmiterjoin%
\pgfsetlinewidth{0.000000pt}%
\definecolor{currentstroke}{rgb}{1.000000,1.000000,1.000000}%
\pgfsetstrokecolor{currentstroke}%
\pgfsetdash{}{0pt}%
\pgfpathmoveto{\pgfqpoint{0.556847in}{2.799750in}}%
\pgfpathlineto{\pgfqpoint{3.134424in}{2.799750in}}%
\pgfusepath{}%
\end{pgfscope}%
\begin{pgfscope}%
\pgfsetbuttcap%
\pgfsetmiterjoin%
\definecolor{currentfill}{rgb}{0.917647,0.917647,0.949020}%
\pgfsetfillcolor{currentfill}%
\pgfsetlinewidth{0.000000pt}%
\definecolor{currentstroke}{rgb}{0.000000,0.000000,0.000000}%
\pgfsetstrokecolor{currentstroke}%
\pgfsetstrokeopacity{0.000000}%
\pgfsetdash{}{0pt}%
\pgfpathmoveto{\pgfqpoint{3.278424in}{1.886339in}}%
\pgfpathlineto{\pgfqpoint{5.856000in}{1.886339in}}%
\pgfpathlineto{\pgfqpoint{5.856000in}{2.799750in}}%
\pgfpathlineto{\pgfqpoint{3.278424in}{2.799750in}}%
\pgfpathclose%
\pgfusepath{fill}%
\end{pgfscope}%
\begin{pgfscope}%
\pgfpathrectangle{\pgfqpoint{3.278424in}{1.886339in}}{\pgfqpoint{2.577576in}{0.913411in}} %
\pgfusepath{clip}%
\pgfsetroundcap%
\pgfsetroundjoin%
\pgfsetlinewidth{0.803000pt}%
\definecolor{currentstroke}{rgb}{1.000000,1.000000,1.000000}%
\pgfsetstrokecolor{currentstroke}%
\pgfsetdash{}{0pt}%
\pgfpathmoveto{\pgfqpoint{3.515561in}{1.886339in}}%
\pgfpathlineto{\pgfqpoint{3.515561in}{2.799750in}}%
\pgfusepath{stroke}%
\end{pgfscope}%
\begin{pgfscope}%
\pgfsetbuttcap%
\pgfsetroundjoin%
\definecolor{currentfill}{rgb}{0.150000,0.150000,0.150000}%
\pgfsetfillcolor{currentfill}%
\pgfsetlinewidth{0.803000pt}%
\definecolor{currentstroke}{rgb}{0.150000,0.150000,0.150000}%
\pgfsetstrokecolor{currentstroke}%
\pgfsetdash{}{0pt}%
\pgfsys@defobject{currentmarker}{\pgfqpoint{0.000000in}{0.000000in}}{\pgfqpoint{0.000000in}{0.000000in}}{%
\pgfpathmoveto{\pgfqpoint{0.000000in}{0.000000in}}%
\pgfpathlineto{\pgfqpoint{0.000000in}{0.000000in}}%
\pgfusepath{stroke,fill}%
}%
\begin{pgfscope}%
\pgfsys@transformshift{3.515561in}{1.886339in}%
\pgfsys@useobject{currentmarker}{}%
\end{pgfscope}%
\end{pgfscope}%
\begin{pgfscope}%
\pgfsetbuttcap%
\pgfsetroundjoin%
\definecolor{currentfill}{rgb}{0.150000,0.150000,0.150000}%
\pgfsetfillcolor{currentfill}%
\pgfsetlinewidth{0.803000pt}%
\definecolor{currentstroke}{rgb}{0.150000,0.150000,0.150000}%
\pgfsetstrokecolor{currentstroke}%
\pgfsetdash{}{0pt}%
\pgfsys@defobject{currentmarker}{\pgfqpoint{0.000000in}{0.000000in}}{\pgfqpoint{0.000000in}{0.000000in}}{%
\pgfpathmoveto{\pgfqpoint{0.000000in}{0.000000in}}%
\pgfpathlineto{\pgfqpoint{0.000000in}{0.000000in}}%
\pgfusepath{stroke,fill}%
}%
\begin{pgfscope}%
\pgfsys@transformshift{3.515561in}{2.799750in}%
\pgfsys@useobject{currentmarker}{}%
\end{pgfscope}%
\end{pgfscope}%
\begin{pgfscope}%
\definecolor{textcolor}{rgb}{0.150000,0.150000,0.150000}%
\pgfsetstrokecolor{textcolor}%
\pgfsetfillcolor{textcolor}%
\pgftext[x=3.515561in,y=1.808561in,,top]{\color{textcolor}\sffamily\fontsize{8.000000}{9.600000}\selectfont 2.5}%
\end{pgfscope}%
\begin{pgfscope}%
\pgfpathrectangle{\pgfqpoint{3.278424in}{1.886339in}}{\pgfqpoint{2.577576in}{0.913411in}} %
\pgfusepath{clip}%
\pgfsetroundcap%
\pgfsetroundjoin%
\pgfsetlinewidth{0.803000pt}%
\definecolor{currentstroke}{rgb}{1.000000,1.000000,1.000000}%
\pgfsetstrokecolor{currentstroke}%
\pgfsetdash{}{0pt}%
\pgfpathmoveto{\pgfqpoint{4.031076in}{1.886339in}}%
\pgfpathlineto{\pgfqpoint{4.031076in}{2.799750in}}%
\pgfusepath{stroke}%
\end{pgfscope}%
\begin{pgfscope}%
\pgfsetbuttcap%
\pgfsetroundjoin%
\definecolor{currentfill}{rgb}{0.150000,0.150000,0.150000}%
\pgfsetfillcolor{currentfill}%
\pgfsetlinewidth{0.803000pt}%
\definecolor{currentstroke}{rgb}{0.150000,0.150000,0.150000}%
\pgfsetstrokecolor{currentstroke}%
\pgfsetdash{}{0pt}%
\pgfsys@defobject{currentmarker}{\pgfqpoint{0.000000in}{0.000000in}}{\pgfqpoint{0.000000in}{0.000000in}}{%
\pgfpathmoveto{\pgfqpoint{0.000000in}{0.000000in}}%
\pgfpathlineto{\pgfqpoint{0.000000in}{0.000000in}}%
\pgfusepath{stroke,fill}%
}%
\begin{pgfscope}%
\pgfsys@transformshift{4.031076in}{1.886339in}%
\pgfsys@useobject{currentmarker}{}%
\end{pgfscope}%
\end{pgfscope}%
\begin{pgfscope}%
\pgfsetbuttcap%
\pgfsetroundjoin%
\definecolor{currentfill}{rgb}{0.150000,0.150000,0.150000}%
\pgfsetfillcolor{currentfill}%
\pgfsetlinewidth{0.803000pt}%
\definecolor{currentstroke}{rgb}{0.150000,0.150000,0.150000}%
\pgfsetstrokecolor{currentstroke}%
\pgfsetdash{}{0pt}%
\pgfsys@defobject{currentmarker}{\pgfqpoint{0.000000in}{0.000000in}}{\pgfqpoint{0.000000in}{0.000000in}}{%
\pgfpathmoveto{\pgfqpoint{0.000000in}{0.000000in}}%
\pgfpathlineto{\pgfqpoint{0.000000in}{0.000000in}}%
\pgfusepath{stroke,fill}%
}%
\begin{pgfscope}%
\pgfsys@transformshift{4.031076in}{2.799750in}%
\pgfsys@useobject{currentmarker}{}%
\end{pgfscope}%
\end{pgfscope}%
\begin{pgfscope}%
\definecolor{textcolor}{rgb}{0.150000,0.150000,0.150000}%
\pgfsetstrokecolor{textcolor}%
\pgfsetfillcolor{textcolor}%
\pgftext[x=4.031076in,y=1.808561in,,top]{\color{textcolor}\sffamily\fontsize{8.000000}{9.600000}\selectfont 3.0}%
\end{pgfscope}%
\begin{pgfscope}%
\pgfpathrectangle{\pgfqpoint{3.278424in}{1.886339in}}{\pgfqpoint{2.577576in}{0.913411in}} %
\pgfusepath{clip}%
\pgfsetroundcap%
\pgfsetroundjoin%
\pgfsetlinewidth{0.803000pt}%
\definecolor{currentstroke}{rgb}{1.000000,1.000000,1.000000}%
\pgfsetstrokecolor{currentstroke}%
\pgfsetdash{}{0pt}%
\pgfpathmoveto{\pgfqpoint{4.546591in}{1.886339in}}%
\pgfpathlineto{\pgfqpoint{4.546591in}{2.799750in}}%
\pgfusepath{stroke}%
\end{pgfscope}%
\begin{pgfscope}%
\pgfsetbuttcap%
\pgfsetroundjoin%
\definecolor{currentfill}{rgb}{0.150000,0.150000,0.150000}%
\pgfsetfillcolor{currentfill}%
\pgfsetlinewidth{0.803000pt}%
\definecolor{currentstroke}{rgb}{0.150000,0.150000,0.150000}%
\pgfsetstrokecolor{currentstroke}%
\pgfsetdash{}{0pt}%
\pgfsys@defobject{currentmarker}{\pgfqpoint{0.000000in}{0.000000in}}{\pgfqpoint{0.000000in}{0.000000in}}{%
\pgfpathmoveto{\pgfqpoint{0.000000in}{0.000000in}}%
\pgfpathlineto{\pgfqpoint{0.000000in}{0.000000in}}%
\pgfusepath{stroke,fill}%
}%
\begin{pgfscope}%
\pgfsys@transformshift{4.546591in}{1.886339in}%
\pgfsys@useobject{currentmarker}{}%
\end{pgfscope}%
\end{pgfscope}%
\begin{pgfscope}%
\pgfsetbuttcap%
\pgfsetroundjoin%
\definecolor{currentfill}{rgb}{0.150000,0.150000,0.150000}%
\pgfsetfillcolor{currentfill}%
\pgfsetlinewidth{0.803000pt}%
\definecolor{currentstroke}{rgb}{0.150000,0.150000,0.150000}%
\pgfsetstrokecolor{currentstroke}%
\pgfsetdash{}{0pt}%
\pgfsys@defobject{currentmarker}{\pgfqpoint{0.000000in}{0.000000in}}{\pgfqpoint{0.000000in}{0.000000in}}{%
\pgfpathmoveto{\pgfqpoint{0.000000in}{0.000000in}}%
\pgfpathlineto{\pgfqpoint{0.000000in}{0.000000in}}%
\pgfusepath{stroke,fill}%
}%
\begin{pgfscope}%
\pgfsys@transformshift{4.546591in}{2.799750in}%
\pgfsys@useobject{currentmarker}{}%
\end{pgfscope}%
\end{pgfscope}%
\begin{pgfscope}%
\definecolor{textcolor}{rgb}{0.150000,0.150000,0.150000}%
\pgfsetstrokecolor{textcolor}%
\pgfsetfillcolor{textcolor}%
\pgftext[x=4.546591in,y=1.808561in,,top]{\color{textcolor}\sffamily\fontsize{8.000000}{9.600000}\selectfont 3.5}%
\end{pgfscope}%
\begin{pgfscope}%
\pgfpathrectangle{\pgfqpoint{3.278424in}{1.886339in}}{\pgfqpoint{2.577576in}{0.913411in}} %
\pgfusepath{clip}%
\pgfsetroundcap%
\pgfsetroundjoin%
\pgfsetlinewidth{0.803000pt}%
\definecolor{currentstroke}{rgb}{1.000000,1.000000,1.000000}%
\pgfsetstrokecolor{currentstroke}%
\pgfsetdash{}{0pt}%
\pgfpathmoveto{\pgfqpoint{5.062106in}{1.886339in}}%
\pgfpathlineto{\pgfqpoint{5.062106in}{2.799750in}}%
\pgfusepath{stroke}%
\end{pgfscope}%
\begin{pgfscope}%
\pgfsetbuttcap%
\pgfsetroundjoin%
\definecolor{currentfill}{rgb}{0.150000,0.150000,0.150000}%
\pgfsetfillcolor{currentfill}%
\pgfsetlinewidth{0.803000pt}%
\definecolor{currentstroke}{rgb}{0.150000,0.150000,0.150000}%
\pgfsetstrokecolor{currentstroke}%
\pgfsetdash{}{0pt}%
\pgfsys@defobject{currentmarker}{\pgfqpoint{0.000000in}{0.000000in}}{\pgfqpoint{0.000000in}{0.000000in}}{%
\pgfpathmoveto{\pgfqpoint{0.000000in}{0.000000in}}%
\pgfpathlineto{\pgfqpoint{0.000000in}{0.000000in}}%
\pgfusepath{stroke,fill}%
}%
\begin{pgfscope}%
\pgfsys@transformshift{5.062106in}{1.886339in}%
\pgfsys@useobject{currentmarker}{}%
\end{pgfscope}%
\end{pgfscope}%
\begin{pgfscope}%
\pgfsetbuttcap%
\pgfsetroundjoin%
\definecolor{currentfill}{rgb}{0.150000,0.150000,0.150000}%
\pgfsetfillcolor{currentfill}%
\pgfsetlinewidth{0.803000pt}%
\definecolor{currentstroke}{rgb}{0.150000,0.150000,0.150000}%
\pgfsetstrokecolor{currentstroke}%
\pgfsetdash{}{0pt}%
\pgfsys@defobject{currentmarker}{\pgfqpoint{0.000000in}{0.000000in}}{\pgfqpoint{0.000000in}{0.000000in}}{%
\pgfpathmoveto{\pgfqpoint{0.000000in}{0.000000in}}%
\pgfpathlineto{\pgfqpoint{0.000000in}{0.000000in}}%
\pgfusepath{stroke,fill}%
}%
\begin{pgfscope}%
\pgfsys@transformshift{5.062106in}{2.799750in}%
\pgfsys@useobject{currentmarker}{}%
\end{pgfscope}%
\end{pgfscope}%
\begin{pgfscope}%
\definecolor{textcolor}{rgb}{0.150000,0.150000,0.150000}%
\pgfsetstrokecolor{textcolor}%
\pgfsetfillcolor{textcolor}%
\pgftext[x=5.062106in,y=1.808561in,,top]{\color{textcolor}\sffamily\fontsize{8.000000}{9.600000}\selectfont 4.0}%
\end{pgfscope}%
\begin{pgfscope}%
\pgfpathrectangle{\pgfqpoint{3.278424in}{1.886339in}}{\pgfqpoint{2.577576in}{0.913411in}} %
\pgfusepath{clip}%
\pgfsetroundcap%
\pgfsetroundjoin%
\pgfsetlinewidth{0.803000pt}%
\definecolor{currentstroke}{rgb}{1.000000,1.000000,1.000000}%
\pgfsetstrokecolor{currentstroke}%
\pgfsetdash{}{0pt}%
\pgfpathmoveto{\pgfqpoint{5.577622in}{1.886339in}}%
\pgfpathlineto{\pgfqpoint{5.577622in}{2.799750in}}%
\pgfusepath{stroke}%
\end{pgfscope}%
\begin{pgfscope}%
\pgfsetbuttcap%
\pgfsetroundjoin%
\definecolor{currentfill}{rgb}{0.150000,0.150000,0.150000}%
\pgfsetfillcolor{currentfill}%
\pgfsetlinewidth{0.803000pt}%
\definecolor{currentstroke}{rgb}{0.150000,0.150000,0.150000}%
\pgfsetstrokecolor{currentstroke}%
\pgfsetdash{}{0pt}%
\pgfsys@defobject{currentmarker}{\pgfqpoint{0.000000in}{0.000000in}}{\pgfqpoint{0.000000in}{0.000000in}}{%
\pgfpathmoveto{\pgfqpoint{0.000000in}{0.000000in}}%
\pgfpathlineto{\pgfqpoint{0.000000in}{0.000000in}}%
\pgfusepath{stroke,fill}%
}%
\begin{pgfscope}%
\pgfsys@transformshift{5.577622in}{1.886339in}%
\pgfsys@useobject{currentmarker}{}%
\end{pgfscope}%
\end{pgfscope}%
\begin{pgfscope}%
\pgfsetbuttcap%
\pgfsetroundjoin%
\definecolor{currentfill}{rgb}{0.150000,0.150000,0.150000}%
\pgfsetfillcolor{currentfill}%
\pgfsetlinewidth{0.803000pt}%
\definecolor{currentstroke}{rgb}{0.150000,0.150000,0.150000}%
\pgfsetstrokecolor{currentstroke}%
\pgfsetdash{}{0pt}%
\pgfsys@defobject{currentmarker}{\pgfqpoint{0.000000in}{0.000000in}}{\pgfqpoint{0.000000in}{0.000000in}}{%
\pgfpathmoveto{\pgfqpoint{0.000000in}{0.000000in}}%
\pgfpathlineto{\pgfqpoint{0.000000in}{0.000000in}}%
\pgfusepath{stroke,fill}%
}%
\begin{pgfscope}%
\pgfsys@transformshift{5.577622in}{2.799750in}%
\pgfsys@useobject{currentmarker}{}%
\end{pgfscope}%
\end{pgfscope}%
\begin{pgfscope}%
\definecolor{textcolor}{rgb}{0.150000,0.150000,0.150000}%
\pgfsetstrokecolor{textcolor}%
\pgfsetfillcolor{textcolor}%
\pgftext[x=5.577622in,y=1.808561in,,top]{\color{textcolor}\sffamily\fontsize{8.000000}{9.600000}\selectfont 4.5}%
\end{pgfscope}%
\begin{pgfscope}%
\definecolor{textcolor}{rgb}{0.150000,0.150000,0.150000}%
\pgfsetstrokecolor{textcolor}%
\pgfsetfillcolor{textcolor}%
\pgftext[x=4.567212in,y=1.643438in,,top]{\color{textcolor}\sffamily\fontsize{8.800000}{10.560000}\selectfont Wing width}%
\end{pgfscope}%
\begin{pgfscope}%
\pgfpathrectangle{\pgfqpoint{3.278424in}{1.886339in}}{\pgfqpoint{2.577576in}{0.913411in}} %
\pgfusepath{clip}%
\pgfsetroundcap%
\pgfsetroundjoin%
\pgfsetlinewidth{0.803000pt}%
\definecolor{currentstroke}{rgb}{1.000000,1.000000,1.000000}%
\pgfsetstrokecolor{currentstroke}%
\pgfsetdash{}{0pt}%
\pgfpathmoveto{\pgfqpoint{3.278424in}{1.886339in}}%
\pgfpathlineto{\pgfqpoint{5.856000in}{1.886339in}}%
\pgfusepath{stroke}%
\end{pgfscope}%
\begin{pgfscope}%
\pgfsetbuttcap%
\pgfsetroundjoin%
\definecolor{currentfill}{rgb}{0.150000,0.150000,0.150000}%
\pgfsetfillcolor{currentfill}%
\pgfsetlinewidth{0.803000pt}%
\definecolor{currentstroke}{rgb}{0.150000,0.150000,0.150000}%
\pgfsetstrokecolor{currentstroke}%
\pgfsetdash{}{0pt}%
\pgfsys@defobject{currentmarker}{\pgfqpoint{0.000000in}{0.000000in}}{\pgfqpoint{0.000000in}{0.000000in}}{%
\pgfpathmoveto{\pgfqpoint{0.000000in}{0.000000in}}%
\pgfpathlineto{\pgfqpoint{0.000000in}{0.000000in}}%
\pgfusepath{stroke,fill}%
}%
\begin{pgfscope}%
\pgfsys@transformshift{3.278424in}{1.886339in}%
\pgfsys@useobject{currentmarker}{}%
\end{pgfscope}%
\end{pgfscope}%
\begin{pgfscope}%
\pgfsetbuttcap%
\pgfsetroundjoin%
\definecolor{currentfill}{rgb}{0.150000,0.150000,0.150000}%
\pgfsetfillcolor{currentfill}%
\pgfsetlinewidth{0.803000pt}%
\definecolor{currentstroke}{rgb}{0.150000,0.150000,0.150000}%
\pgfsetstrokecolor{currentstroke}%
\pgfsetdash{}{0pt}%
\pgfsys@defobject{currentmarker}{\pgfqpoint{0.000000in}{0.000000in}}{\pgfqpoint{0.000000in}{0.000000in}}{%
\pgfpathmoveto{\pgfqpoint{0.000000in}{0.000000in}}%
\pgfpathlineto{\pgfqpoint{0.000000in}{0.000000in}}%
\pgfusepath{stroke,fill}%
}%
\begin{pgfscope}%
\pgfsys@transformshift{5.856000in}{1.886339in}%
\pgfsys@useobject{currentmarker}{}%
\end{pgfscope}%
\end{pgfscope}%
\begin{pgfscope}%
\pgfpathrectangle{\pgfqpoint{3.278424in}{1.886339in}}{\pgfqpoint{2.577576in}{0.913411in}} %
\pgfusepath{clip}%
\pgfsetroundcap%
\pgfsetroundjoin%
\pgfsetlinewidth{0.803000pt}%
\definecolor{currentstroke}{rgb}{1.000000,1.000000,1.000000}%
\pgfsetstrokecolor{currentstroke}%
\pgfsetdash{}{0pt}%
\pgfpathmoveto{\pgfqpoint{3.278424in}{2.038574in}}%
\pgfpathlineto{\pgfqpoint{5.856000in}{2.038574in}}%
\pgfusepath{stroke}%
\end{pgfscope}%
\begin{pgfscope}%
\pgfsetbuttcap%
\pgfsetroundjoin%
\definecolor{currentfill}{rgb}{0.150000,0.150000,0.150000}%
\pgfsetfillcolor{currentfill}%
\pgfsetlinewidth{0.803000pt}%
\definecolor{currentstroke}{rgb}{0.150000,0.150000,0.150000}%
\pgfsetstrokecolor{currentstroke}%
\pgfsetdash{}{0pt}%
\pgfsys@defobject{currentmarker}{\pgfqpoint{0.000000in}{0.000000in}}{\pgfqpoint{0.000000in}{0.000000in}}{%
\pgfpathmoveto{\pgfqpoint{0.000000in}{0.000000in}}%
\pgfpathlineto{\pgfqpoint{0.000000in}{0.000000in}}%
\pgfusepath{stroke,fill}%
}%
\begin{pgfscope}%
\pgfsys@transformshift{3.278424in}{2.038574in}%
\pgfsys@useobject{currentmarker}{}%
\end{pgfscope}%
\end{pgfscope}%
\begin{pgfscope}%
\pgfsetbuttcap%
\pgfsetroundjoin%
\definecolor{currentfill}{rgb}{0.150000,0.150000,0.150000}%
\pgfsetfillcolor{currentfill}%
\pgfsetlinewidth{0.803000pt}%
\definecolor{currentstroke}{rgb}{0.150000,0.150000,0.150000}%
\pgfsetstrokecolor{currentstroke}%
\pgfsetdash{}{0pt}%
\pgfsys@defobject{currentmarker}{\pgfqpoint{0.000000in}{0.000000in}}{\pgfqpoint{0.000000in}{0.000000in}}{%
\pgfpathmoveto{\pgfqpoint{0.000000in}{0.000000in}}%
\pgfpathlineto{\pgfqpoint{0.000000in}{0.000000in}}%
\pgfusepath{stroke,fill}%
}%
\begin{pgfscope}%
\pgfsys@transformshift{5.856000in}{2.038574in}%
\pgfsys@useobject{currentmarker}{}%
\end{pgfscope}%
\end{pgfscope}%
\begin{pgfscope}%
\pgfpathrectangle{\pgfqpoint{3.278424in}{1.886339in}}{\pgfqpoint{2.577576in}{0.913411in}} %
\pgfusepath{clip}%
\pgfsetroundcap%
\pgfsetroundjoin%
\pgfsetlinewidth{0.803000pt}%
\definecolor{currentstroke}{rgb}{1.000000,1.000000,1.000000}%
\pgfsetstrokecolor{currentstroke}%
\pgfsetdash{}{0pt}%
\pgfpathmoveto{\pgfqpoint{3.278424in}{2.190809in}}%
\pgfpathlineto{\pgfqpoint{5.856000in}{2.190809in}}%
\pgfusepath{stroke}%
\end{pgfscope}%
\begin{pgfscope}%
\pgfsetbuttcap%
\pgfsetroundjoin%
\definecolor{currentfill}{rgb}{0.150000,0.150000,0.150000}%
\pgfsetfillcolor{currentfill}%
\pgfsetlinewidth{0.803000pt}%
\definecolor{currentstroke}{rgb}{0.150000,0.150000,0.150000}%
\pgfsetstrokecolor{currentstroke}%
\pgfsetdash{}{0pt}%
\pgfsys@defobject{currentmarker}{\pgfqpoint{0.000000in}{0.000000in}}{\pgfqpoint{0.000000in}{0.000000in}}{%
\pgfpathmoveto{\pgfqpoint{0.000000in}{0.000000in}}%
\pgfpathlineto{\pgfqpoint{0.000000in}{0.000000in}}%
\pgfusepath{stroke,fill}%
}%
\begin{pgfscope}%
\pgfsys@transformshift{3.278424in}{2.190809in}%
\pgfsys@useobject{currentmarker}{}%
\end{pgfscope}%
\end{pgfscope}%
\begin{pgfscope}%
\pgfsetbuttcap%
\pgfsetroundjoin%
\definecolor{currentfill}{rgb}{0.150000,0.150000,0.150000}%
\pgfsetfillcolor{currentfill}%
\pgfsetlinewidth{0.803000pt}%
\definecolor{currentstroke}{rgb}{0.150000,0.150000,0.150000}%
\pgfsetstrokecolor{currentstroke}%
\pgfsetdash{}{0pt}%
\pgfsys@defobject{currentmarker}{\pgfqpoint{0.000000in}{0.000000in}}{\pgfqpoint{0.000000in}{0.000000in}}{%
\pgfpathmoveto{\pgfqpoint{0.000000in}{0.000000in}}%
\pgfpathlineto{\pgfqpoint{0.000000in}{0.000000in}}%
\pgfusepath{stroke,fill}%
}%
\begin{pgfscope}%
\pgfsys@transformshift{5.856000in}{2.190809in}%
\pgfsys@useobject{currentmarker}{}%
\end{pgfscope}%
\end{pgfscope}%
\begin{pgfscope}%
\pgfpathrectangle{\pgfqpoint{3.278424in}{1.886339in}}{\pgfqpoint{2.577576in}{0.913411in}} %
\pgfusepath{clip}%
\pgfsetroundcap%
\pgfsetroundjoin%
\pgfsetlinewidth{0.803000pt}%
\definecolor{currentstroke}{rgb}{1.000000,1.000000,1.000000}%
\pgfsetstrokecolor{currentstroke}%
\pgfsetdash{}{0pt}%
\pgfpathmoveto{\pgfqpoint{3.278424in}{2.343044in}}%
\pgfpathlineto{\pgfqpoint{5.856000in}{2.343044in}}%
\pgfusepath{stroke}%
\end{pgfscope}%
\begin{pgfscope}%
\pgfsetbuttcap%
\pgfsetroundjoin%
\definecolor{currentfill}{rgb}{0.150000,0.150000,0.150000}%
\pgfsetfillcolor{currentfill}%
\pgfsetlinewidth{0.803000pt}%
\definecolor{currentstroke}{rgb}{0.150000,0.150000,0.150000}%
\pgfsetstrokecolor{currentstroke}%
\pgfsetdash{}{0pt}%
\pgfsys@defobject{currentmarker}{\pgfqpoint{0.000000in}{0.000000in}}{\pgfqpoint{0.000000in}{0.000000in}}{%
\pgfpathmoveto{\pgfqpoint{0.000000in}{0.000000in}}%
\pgfpathlineto{\pgfqpoint{0.000000in}{0.000000in}}%
\pgfusepath{stroke,fill}%
}%
\begin{pgfscope}%
\pgfsys@transformshift{3.278424in}{2.343044in}%
\pgfsys@useobject{currentmarker}{}%
\end{pgfscope}%
\end{pgfscope}%
\begin{pgfscope}%
\pgfsetbuttcap%
\pgfsetroundjoin%
\definecolor{currentfill}{rgb}{0.150000,0.150000,0.150000}%
\pgfsetfillcolor{currentfill}%
\pgfsetlinewidth{0.803000pt}%
\definecolor{currentstroke}{rgb}{0.150000,0.150000,0.150000}%
\pgfsetstrokecolor{currentstroke}%
\pgfsetdash{}{0pt}%
\pgfsys@defobject{currentmarker}{\pgfqpoint{0.000000in}{0.000000in}}{\pgfqpoint{0.000000in}{0.000000in}}{%
\pgfpathmoveto{\pgfqpoint{0.000000in}{0.000000in}}%
\pgfpathlineto{\pgfqpoint{0.000000in}{0.000000in}}%
\pgfusepath{stroke,fill}%
}%
\begin{pgfscope}%
\pgfsys@transformshift{5.856000in}{2.343044in}%
\pgfsys@useobject{currentmarker}{}%
\end{pgfscope}%
\end{pgfscope}%
\begin{pgfscope}%
\pgfpathrectangle{\pgfqpoint{3.278424in}{1.886339in}}{\pgfqpoint{2.577576in}{0.913411in}} %
\pgfusepath{clip}%
\pgfsetroundcap%
\pgfsetroundjoin%
\pgfsetlinewidth{0.803000pt}%
\definecolor{currentstroke}{rgb}{1.000000,1.000000,1.000000}%
\pgfsetstrokecolor{currentstroke}%
\pgfsetdash{}{0pt}%
\pgfpathmoveto{\pgfqpoint{3.278424in}{2.495280in}}%
\pgfpathlineto{\pgfqpoint{5.856000in}{2.495280in}}%
\pgfusepath{stroke}%
\end{pgfscope}%
\begin{pgfscope}%
\pgfsetbuttcap%
\pgfsetroundjoin%
\definecolor{currentfill}{rgb}{0.150000,0.150000,0.150000}%
\pgfsetfillcolor{currentfill}%
\pgfsetlinewidth{0.803000pt}%
\definecolor{currentstroke}{rgb}{0.150000,0.150000,0.150000}%
\pgfsetstrokecolor{currentstroke}%
\pgfsetdash{}{0pt}%
\pgfsys@defobject{currentmarker}{\pgfqpoint{0.000000in}{0.000000in}}{\pgfqpoint{0.000000in}{0.000000in}}{%
\pgfpathmoveto{\pgfqpoint{0.000000in}{0.000000in}}%
\pgfpathlineto{\pgfqpoint{0.000000in}{0.000000in}}%
\pgfusepath{stroke,fill}%
}%
\begin{pgfscope}%
\pgfsys@transformshift{3.278424in}{2.495280in}%
\pgfsys@useobject{currentmarker}{}%
\end{pgfscope}%
\end{pgfscope}%
\begin{pgfscope}%
\pgfsetbuttcap%
\pgfsetroundjoin%
\definecolor{currentfill}{rgb}{0.150000,0.150000,0.150000}%
\pgfsetfillcolor{currentfill}%
\pgfsetlinewidth{0.803000pt}%
\definecolor{currentstroke}{rgb}{0.150000,0.150000,0.150000}%
\pgfsetstrokecolor{currentstroke}%
\pgfsetdash{}{0pt}%
\pgfsys@defobject{currentmarker}{\pgfqpoint{0.000000in}{0.000000in}}{\pgfqpoint{0.000000in}{0.000000in}}{%
\pgfpathmoveto{\pgfqpoint{0.000000in}{0.000000in}}%
\pgfpathlineto{\pgfqpoint{0.000000in}{0.000000in}}%
\pgfusepath{stroke,fill}%
}%
\begin{pgfscope}%
\pgfsys@transformshift{5.856000in}{2.495280in}%
\pgfsys@useobject{currentmarker}{}%
\end{pgfscope}%
\end{pgfscope}%
\begin{pgfscope}%
\pgfpathrectangle{\pgfqpoint{3.278424in}{1.886339in}}{\pgfqpoint{2.577576in}{0.913411in}} %
\pgfusepath{clip}%
\pgfsetroundcap%
\pgfsetroundjoin%
\pgfsetlinewidth{0.803000pt}%
\definecolor{currentstroke}{rgb}{1.000000,1.000000,1.000000}%
\pgfsetstrokecolor{currentstroke}%
\pgfsetdash{}{0pt}%
\pgfpathmoveto{\pgfqpoint{3.278424in}{2.647515in}}%
\pgfpathlineto{\pgfqpoint{5.856000in}{2.647515in}}%
\pgfusepath{stroke}%
\end{pgfscope}%
\begin{pgfscope}%
\pgfsetbuttcap%
\pgfsetroundjoin%
\definecolor{currentfill}{rgb}{0.150000,0.150000,0.150000}%
\pgfsetfillcolor{currentfill}%
\pgfsetlinewidth{0.803000pt}%
\definecolor{currentstroke}{rgb}{0.150000,0.150000,0.150000}%
\pgfsetstrokecolor{currentstroke}%
\pgfsetdash{}{0pt}%
\pgfsys@defobject{currentmarker}{\pgfqpoint{0.000000in}{0.000000in}}{\pgfqpoint{0.000000in}{0.000000in}}{%
\pgfpathmoveto{\pgfqpoint{0.000000in}{0.000000in}}%
\pgfpathlineto{\pgfqpoint{0.000000in}{0.000000in}}%
\pgfusepath{stroke,fill}%
}%
\begin{pgfscope}%
\pgfsys@transformshift{3.278424in}{2.647515in}%
\pgfsys@useobject{currentmarker}{}%
\end{pgfscope}%
\end{pgfscope}%
\begin{pgfscope}%
\pgfsetbuttcap%
\pgfsetroundjoin%
\definecolor{currentfill}{rgb}{0.150000,0.150000,0.150000}%
\pgfsetfillcolor{currentfill}%
\pgfsetlinewidth{0.803000pt}%
\definecolor{currentstroke}{rgb}{0.150000,0.150000,0.150000}%
\pgfsetstrokecolor{currentstroke}%
\pgfsetdash{}{0pt}%
\pgfsys@defobject{currentmarker}{\pgfqpoint{0.000000in}{0.000000in}}{\pgfqpoint{0.000000in}{0.000000in}}{%
\pgfpathmoveto{\pgfqpoint{0.000000in}{0.000000in}}%
\pgfpathlineto{\pgfqpoint{0.000000in}{0.000000in}}%
\pgfusepath{stroke,fill}%
}%
\begin{pgfscope}%
\pgfsys@transformshift{5.856000in}{2.647515in}%
\pgfsys@useobject{currentmarker}{}%
\end{pgfscope}%
\end{pgfscope}%
\begin{pgfscope}%
\pgfpathrectangle{\pgfqpoint{3.278424in}{1.886339in}}{\pgfqpoint{2.577576in}{0.913411in}} %
\pgfusepath{clip}%
\pgfsetroundcap%
\pgfsetroundjoin%
\pgfsetlinewidth{0.803000pt}%
\definecolor{currentstroke}{rgb}{1.000000,1.000000,1.000000}%
\pgfsetstrokecolor{currentstroke}%
\pgfsetdash{}{0pt}%
\pgfpathmoveto{\pgfqpoint{3.278424in}{2.799750in}}%
\pgfpathlineto{\pgfqpoint{5.856000in}{2.799750in}}%
\pgfusepath{stroke}%
\end{pgfscope}%
\begin{pgfscope}%
\pgfsetbuttcap%
\pgfsetroundjoin%
\definecolor{currentfill}{rgb}{0.150000,0.150000,0.150000}%
\pgfsetfillcolor{currentfill}%
\pgfsetlinewidth{0.803000pt}%
\definecolor{currentstroke}{rgb}{0.150000,0.150000,0.150000}%
\pgfsetstrokecolor{currentstroke}%
\pgfsetdash{}{0pt}%
\pgfsys@defobject{currentmarker}{\pgfqpoint{0.000000in}{0.000000in}}{\pgfqpoint{0.000000in}{0.000000in}}{%
\pgfpathmoveto{\pgfqpoint{0.000000in}{0.000000in}}%
\pgfpathlineto{\pgfqpoint{0.000000in}{0.000000in}}%
\pgfusepath{stroke,fill}%
}%
\begin{pgfscope}%
\pgfsys@transformshift{3.278424in}{2.799750in}%
\pgfsys@useobject{currentmarker}{}%
\end{pgfscope}%
\end{pgfscope}%
\begin{pgfscope}%
\pgfsetbuttcap%
\pgfsetroundjoin%
\definecolor{currentfill}{rgb}{0.150000,0.150000,0.150000}%
\pgfsetfillcolor{currentfill}%
\pgfsetlinewidth{0.803000pt}%
\definecolor{currentstroke}{rgb}{0.150000,0.150000,0.150000}%
\pgfsetstrokecolor{currentstroke}%
\pgfsetdash{}{0pt}%
\pgfsys@defobject{currentmarker}{\pgfqpoint{0.000000in}{0.000000in}}{\pgfqpoint{0.000000in}{0.000000in}}{%
\pgfpathmoveto{\pgfqpoint{0.000000in}{0.000000in}}%
\pgfpathlineto{\pgfqpoint{0.000000in}{0.000000in}}%
\pgfusepath{stroke,fill}%
}%
\begin{pgfscope}%
\pgfsys@transformshift{5.856000in}{2.799750in}%
\pgfsys@useobject{currentmarker}{}%
\end{pgfscope}%
\end{pgfscope}%
\begin{pgfscope}%
\pgfpathrectangle{\pgfqpoint{3.278424in}{1.886339in}}{\pgfqpoint{2.577576in}{0.913411in}} %
\pgfusepath{clip}%
\pgfsetbuttcap%
\pgfsetmiterjoin%
\definecolor{currentfill}{rgb}{0.447059,0.623529,0.811765}%
\pgfsetfillcolor{currentfill}%
\pgfsetfillopacity{0.300000}%
\pgfsetlinewidth{0.240900pt}%
\definecolor{currentstroke}{rgb}{0.447059,0.623529,0.811765}%
\pgfsetstrokecolor{currentstroke}%
\pgfsetstrokeopacity{0.300000}%
\pgfsetdash{}{0pt}%
\pgfpathmoveto{\pgfqpoint{3.484630in}{2.263597in}}%
\pgfpathlineto{\pgfqpoint{3.506500in}{2.266203in}}%
\pgfpathlineto{\pgfqpoint{3.528370in}{2.268883in}}%
\pgfpathlineto{\pgfqpoint{3.550241in}{2.271643in}}%
\pgfpathlineto{\pgfqpoint{3.572111in}{2.274486in}}%
\pgfpathlineto{\pgfqpoint{3.593981in}{2.277418in}}%
\pgfpathlineto{\pgfqpoint{3.615852in}{2.280444in}}%
\pgfpathlineto{\pgfqpoint{3.637722in}{2.283569in}}%
\pgfpathlineto{\pgfqpoint{3.659592in}{2.286798in}}%
\pgfpathlineto{\pgfqpoint{3.681463in}{2.290136in}}%
\pgfpathlineto{\pgfqpoint{3.703333in}{2.293586in}}%
\pgfpathlineto{\pgfqpoint{3.725204in}{2.297151in}}%
\pgfpathlineto{\pgfqpoint{3.747074in}{2.300834in}}%
\pgfpathlineto{\pgfqpoint{3.768944in}{2.304634in}}%
\pgfpathlineto{\pgfqpoint{3.790815in}{2.308551in}}%
\pgfpathlineto{\pgfqpoint{3.812685in}{2.312581in}}%
\pgfpathlineto{\pgfqpoint{3.834555in}{2.316719in}}%
\pgfpathlineto{\pgfqpoint{3.856426in}{2.320959in}}%
\pgfpathlineto{\pgfqpoint{3.878296in}{2.325292in}}%
\pgfpathlineto{\pgfqpoint{3.900166in}{2.329708in}}%
\pgfpathlineto{\pgfqpoint{3.922037in}{2.334195in}}%
\pgfpathlineto{\pgfqpoint{3.943907in}{2.338741in}}%
\pgfpathlineto{\pgfqpoint{3.965777in}{2.343333in}}%
\pgfpathlineto{\pgfqpoint{3.987648in}{2.347958in}}%
\pgfpathlineto{\pgfqpoint{4.009518in}{2.352603in}}%
\pgfpathlineto{\pgfqpoint{4.031388in}{2.357254in}}%
\pgfpathlineto{\pgfqpoint{4.053259in}{2.361901in}}%
\pgfpathlineto{\pgfqpoint{4.075129in}{2.366530in}}%
\pgfpathlineto{\pgfqpoint{4.096999in}{2.371131in}}%
\pgfpathlineto{\pgfqpoint{4.118870in}{2.375695in}}%
\pgfpathlineto{\pgfqpoint{4.140740in}{2.380213in}}%
\pgfpathlineto{\pgfqpoint{4.162610in}{2.384675in}}%
\pgfpathlineto{\pgfqpoint{4.184481in}{2.389075in}}%
\pgfpathlineto{\pgfqpoint{4.206351in}{2.393406in}}%
\pgfpathlineto{\pgfqpoint{4.228221in}{2.397662in}}%
\pgfpathlineto{\pgfqpoint{4.250092in}{2.401838in}}%
\pgfpathlineto{\pgfqpoint{4.271962in}{2.405929in}}%
\pgfpathlineto{\pgfqpoint{4.293832in}{2.409931in}}%
\pgfpathlineto{\pgfqpoint{4.315703in}{2.413840in}}%
\pgfpathlineto{\pgfqpoint{4.337573in}{2.417653in}}%
\pgfpathlineto{\pgfqpoint{4.359444in}{2.421368in}}%
\pgfpathlineto{\pgfqpoint{4.381314in}{2.424982in}}%
\pgfpathlineto{\pgfqpoint{4.403184in}{2.428493in}}%
\pgfpathlineto{\pgfqpoint{4.425055in}{2.431899in}}%
\pgfpathlineto{\pgfqpoint{4.446925in}{2.435200in}}%
\pgfpathlineto{\pgfqpoint{4.468795in}{2.438394in}}%
\pgfpathlineto{\pgfqpoint{4.490666in}{2.441480in}}%
\pgfpathlineto{\pgfqpoint{4.512536in}{2.444458in}}%
\pgfpathlineto{\pgfqpoint{4.534406in}{2.447328in}}%
\pgfpathlineto{\pgfqpoint{4.556277in}{2.450089in}}%
\pgfpathlineto{\pgfqpoint{4.578147in}{2.452742in}}%
\pgfpathlineto{\pgfqpoint{4.600017in}{2.455288in}}%
\pgfpathlineto{\pgfqpoint{4.621888in}{2.457726in}}%
\pgfpathlineto{\pgfqpoint{4.643758in}{2.460059in}}%
\pgfpathlineto{\pgfqpoint{4.665628in}{2.462287in}}%
\pgfpathlineto{\pgfqpoint{4.687499in}{2.464412in}}%
\pgfpathlineto{\pgfqpoint{4.709369in}{2.466436in}}%
\pgfpathlineto{\pgfqpoint{4.731239in}{2.468361in}}%
\pgfpathlineto{\pgfqpoint{4.753110in}{2.470190in}}%
\pgfpathlineto{\pgfqpoint{4.774980in}{2.471926in}}%
\pgfpathlineto{\pgfqpoint{4.796850in}{2.473572in}}%
\pgfpathlineto{\pgfqpoint{4.818721in}{2.475131in}}%
\pgfpathlineto{\pgfqpoint{4.840591in}{2.476609in}}%
\pgfpathlineto{\pgfqpoint{4.862461in}{2.478011in}}%
\pgfpathlineto{\pgfqpoint{4.884332in}{2.479341in}}%
\pgfpathlineto{\pgfqpoint{4.906202in}{2.480605in}}%
\pgfpathlineto{\pgfqpoint{4.928072in}{2.481811in}}%
\pgfpathlineto{\pgfqpoint{4.949943in}{2.482967in}}%
\pgfpathlineto{\pgfqpoint{4.971813in}{2.484079in}}%
\pgfpathlineto{\pgfqpoint{4.993684in}{2.485157in}}%
\pgfpathlineto{\pgfqpoint{5.015554in}{2.486211in}}%
\pgfpathlineto{\pgfqpoint{5.037424in}{2.487251in}}%
\pgfpathlineto{\pgfqpoint{5.059295in}{2.488288in}}%
\pgfpathlineto{\pgfqpoint{5.081165in}{2.489332in}}%
\pgfpathlineto{\pgfqpoint{5.103035in}{2.490397in}}%
\pgfpathlineto{\pgfqpoint{5.124906in}{2.491494in}}%
\pgfpathlineto{\pgfqpoint{5.146776in}{2.492634in}}%
\pgfpathlineto{\pgfqpoint{5.168646in}{2.493830in}}%
\pgfpathlineto{\pgfqpoint{5.190517in}{2.495093in}}%
\pgfpathlineto{\pgfqpoint{5.212387in}{2.496431in}}%
\pgfpathlineto{\pgfqpoint{5.234257in}{2.497855in}}%
\pgfpathlineto{\pgfqpoint{5.256128in}{2.499372in}}%
\pgfpathlineto{\pgfqpoint{5.277998in}{2.500988in}}%
\pgfpathlineto{\pgfqpoint{5.299868in}{2.502707in}}%
\pgfpathlineto{\pgfqpoint{5.321739in}{2.504533in}}%
\pgfpathlineto{\pgfqpoint{5.343609in}{2.506467in}}%
\pgfpathlineto{\pgfqpoint{5.365479in}{2.508509in}}%
\pgfpathlineto{\pgfqpoint{5.387350in}{2.510657in}}%
\pgfpathlineto{\pgfqpoint{5.409220in}{2.512910in}}%
\pgfpathlineto{\pgfqpoint{5.431090in}{2.515264in}}%
\pgfpathlineto{\pgfqpoint{5.452961in}{2.517717in}}%
\pgfpathlineto{\pgfqpoint{5.474831in}{2.520263in}}%
\pgfpathlineto{\pgfqpoint{5.496701in}{2.522900in}}%
\pgfpathlineto{\pgfqpoint{5.518572in}{2.525621in}}%
\pgfpathlineto{\pgfqpoint{5.540442in}{2.528424in}}%
\pgfpathlineto{\pgfqpoint{5.562313in}{2.531304in}}%
\pgfpathlineto{\pgfqpoint{5.584183in}{2.534256in}}%
\pgfpathlineto{\pgfqpoint{5.606053in}{2.537277in}}%
\pgfpathlineto{\pgfqpoint{5.627924in}{2.540364in}}%
\pgfpathlineto{\pgfqpoint{5.649794in}{2.543512in}}%
\pgfpathlineto{\pgfqpoint{5.649794in}{2.366189in}}%
\pgfpathlineto{\pgfqpoint{5.627924in}{2.370047in}}%
\pgfpathlineto{\pgfqpoint{5.606053in}{2.373718in}}%
\pgfpathlineto{\pgfqpoint{5.584183in}{2.377200in}}%
\pgfpathlineto{\pgfqpoint{5.562313in}{2.380489in}}%
\pgfpathlineto{\pgfqpoint{5.540442in}{2.383581in}}%
\pgfpathlineto{\pgfqpoint{5.518572in}{2.386471in}}%
\pgfpathlineto{\pgfqpoint{5.496701in}{2.389156in}}%
\pgfpathlineto{\pgfqpoint{5.474831in}{2.391631in}}%
\pgfpathlineto{\pgfqpoint{5.452961in}{2.393892in}}%
\pgfpathlineto{\pgfqpoint{5.431090in}{2.395935in}}%
\pgfpathlineto{\pgfqpoint{5.409220in}{2.397755in}}%
\pgfpathlineto{\pgfqpoint{5.387350in}{2.399349in}}%
\pgfpathlineto{\pgfqpoint{5.365479in}{2.400714in}}%
\pgfpathlineto{\pgfqpoint{5.343609in}{2.401848in}}%
\pgfpathlineto{\pgfqpoint{5.321739in}{2.402750in}}%
\pgfpathlineto{\pgfqpoint{5.299868in}{2.403419in}}%
\pgfpathlineto{\pgfqpoint{5.277998in}{2.403858in}}%
\pgfpathlineto{\pgfqpoint{5.256128in}{2.404068in}}%
\pgfpathlineto{\pgfqpoint{5.234257in}{2.404056in}}%
\pgfpathlineto{\pgfqpoint{5.212387in}{2.403826in}}%
\pgfpathlineto{\pgfqpoint{5.190517in}{2.403386in}}%
\pgfpathlineto{\pgfqpoint{5.168646in}{2.402745in}}%
\pgfpathlineto{\pgfqpoint{5.146776in}{2.401914in}}%
\pgfpathlineto{\pgfqpoint{5.124906in}{2.400903in}}%
\pgfpathlineto{\pgfqpoint{5.103035in}{2.399723in}}%
\pgfpathlineto{\pgfqpoint{5.081165in}{2.398388in}}%
\pgfpathlineto{\pgfqpoint{5.059295in}{2.396908in}}%
\pgfpathlineto{\pgfqpoint{5.037424in}{2.395295in}}%
\pgfpathlineto{\pgfqpoint{5.015554in}{2.393561in}}%
\pgfpathlineto{\pgfqpoint{4.993684in}{2.391717in}}%
\pgfpathlineto{\pgfqpoint{4.971813in}{2.389773in}}%
\pgfpathlineto{\pgfqpoint{4.949943in}{2.387738in}}%
\pgfpathlineto{\pgfqpoint{4.928072in}{2.385622in}}%
\pgfpathlineto{\pgfqpoint{4.906202in}{2.383433in}}%
\pgfpathlineto{\pgfqpoint{4.884332in}{2.381177in}}%
\pgfpathlineto{\pgfqpoint{4.862461in}{2.378863in}}%
\pgfpathlineto{\pgfqpoint{4.840591in}{2.376495in}}%
\pgfpathlineto{\pgfqpoint{4.818721in}{2.374080in}}%
\pgfpathlineto{\pgfqpoint{4.796850in}{2.371622in}}%
\pgfpathlineto{\pgfqpoint{4.774980in}{2.369125in}}%
\pgfpathlineto{\pgfqpoint{4.753110in}{2.366595in}}%
\pgfpathlineto{\pgfqpoint{4.731239in}{2.364033in}}%
\pgfpathlineto{\pgfqpoint{4.709369in}{2.361442in}}%
\pgfpathlineto{\pgfqpoint{4.687499in}{2.358827in}}%
\pgfpathlineto{\pgfqpoint{4.665628in}{2.356188in}}%
\pgfpathlineto{\pgfqpoint{4.643758in}{2.353527in}}%
\pgfpathlineto{\pgfqpoint{4.621888in}{2.350847in}}%
\pgfpathlineto{\pgfqpoint{4.600017in}{2.348148in}}%
\pgfpathlineto{\pgfqpoint{4.578147in}{2.345432in}}%
\pgfpathlineto{\pgfqpoint{4.556277in}{2.342699in}}%
\pgfpathlineto{\pgfqpoint{4.534406in}{2.339950in}}%
\pgfpathlineto{\pgfqpoint{4.512536in}{2.337185in}}%
\pgfpathlineto{\pgfqpoint{4.490666in}{2.334403in}}%
\pgfpathlineto{\pgfqpoint{4.468795in}{2.331606in}}%
\pgfpathlineto{\pgfqpoint{4.446925in}{2.328792in}}%
\pgfpathlineto{\pgfqpoint{4.425055in}{2.325960in}}%
\pgfpathlineto{\pgfqpoint{4.403184in}{2.323109in}}%
\pgfpathlineto{\pgfqpoint{4.381314in}{2.320239in}}%
\pgfpathlineto{\pgfqpoint{4.359444in}{2.317347in}}%
\pgfpathlineto{\pgfqpoint{4.337573in}{2.314432in}}%
\pgfpathlineto{\pgfqpoint{4.315703in}{2.311491in}}%
\pgfpathlineto{\pgfqpoint{4.293832in}{2.308521in}}%
\pgfpathlineto{\pgfqpoint{4.271962in}{2.305519in}}%
\pgfpathlineto{\pgfqpoint{4.250092in}{2.302482in}}%
\pgfpathlineto{\pgfqpoint{4.228221in}{2.299406in}}%
\pgfpathlineto{\pgfqpoint{4.206351in}{2.296285in}}%
\pgfpathlineto{\pgfqpoint{4.184481in}{2.293115in}}%
\pgfpathlineto{\pgfqpoint{4.162610in}{2.289890in}}%
\pgfpathlineto{\pgfqpoint{4.140740in}{2.286603in}}%
\pgfpathlineto{\pgfqpoint{4.118870in}{2.283246in}}%
\pgfpathlineto{\pgfqpoint{4.096999in}{2.279811in}}%
\pgfpathlineto{\pgfqpoint{4.075129in}{2.276290in}}%
\pgfpathlineto{\pgfqpoint{4.053259in}{2.272671in}}%
\pgfpathlineto{\pgfqpoint{4.031388in}{2.268946in}}%
\pgfpathlineto{\pgfqpoint{4.009518in}{2.265101in}}%
\pgfpathlineto{\pgfqpoint{3.987648in}{2.261125in}}%
\pgfpathlineto{\pgfqpoint{3.965777in}{2.257005in}}%
\pgfpathlineto{\pgfqpoint{3.943907in}{2.252728in}}%
\pgfpathlineto{\pgfqpoint{3.922037in}{2.248280in}}%
\pgfpathlineto{\pgfqpoint{3.900166in}{2.243649in}}%
\pgfpathlineto{\pgfqpoint{3.878296in}{2.238822in}}%
\pgfpathlineto{\pgfqpoint{3.856426in}{2.233788in}}%
\pgfpathlineto{\pgfqpoint{3.834555in}{2.228537in}}%
\pgfpathlineto{\pgfqpoint{3.812685in}{2.223059in}}%
\pgfpathlineto{\pgfqpoint{3.790815in}{2.217349in}}%
\pgfpathlineto{\pgfqpoint{3.768944in}{2.211401in}}%
\pgfpathlineto{\pgfqpoint{3.747074in}{2.205212in}}%
\pgfpathlineto{\pgfqpoint{3.725204in}{2.198782in}}%
\pgfpathlineto{\pgfqpoint{3.703333in}{2.192109in}}%
\pgfpathlineto{\pgfqpoint{3.681463in}{2.185197in}}%
\pgfpathlineto{\pgfqpoint{3.659592in}{2.178048in}}%
\pgfpathlineto{\pgfqpoint{3.637722in}{2.170666in}}%
\pgfpathlineto{\pgfqpoint{3.615852in}{2.163056in}}%
\pgfpathlineto{\pgfqpoint{3.593981in}{2.155222in}}%
\pgfpathlineto{\pgfqpoint{3.572111in}{2.147169in}}%
\pgfpathlineto{\pgfqpoint{3.550241in}{2.138904in}}%
\pgfpathlineto{\pgfqpoint{3.528370in}{2.130430in}}%
\pgfpathlineto{\pgfqpoint{3.506500in}{2.121753in}}%
\pgfpathlineto{\pgfqpoint{3.484630in}{2.112877in}}%
\pgfpathclose%
\pgfusepath{stroke,fill}%
\end{pgfscope}%
\begin{pgfscope}%
\pgfpathrectangle{\pgfqpoint{3.278424in}{1.886339in}}{\pgfqpoint{2.577576in}{0.913411in}} %
\pgfusepath{clip}%
\pgfsetroundcap%
\pgfsetroundjoin%
\pgfsetlinewidth{2.007500pt}%
\definecolor{currentstroke}{rgb}{0.125490,0.290196,0.529412}%
\pgfsetstrokecolor{currentstroke}%
\pgfsetdash{}{0pt}%
\pgfpathmoveto{\pgfqpoint{3.484630in}{2.188237in}}%
\pgfpathlineto{\pgfqpoint{3.506500in}{2.193978in}}%
\pgfpathlineto{\pgfqpoint{3.528370in}{2.199657in}}%
\pgfpathlineto{\pgfqpoint{3.550241in}{2.205273in}}%
\pgfpathlineto{\pgfqpoint{3.572111in}{2.210828in}}%
\pgfpathlineto{\pgfqpoint{3.593981in}{2.216320in}}%
\pgfpathlineto{\pgfqpoint{3.615852in}{2.221750in}}%
\pgfpathlineto{\pgfqpoint{3.637722in}{2.227117in}}%
\pgfpathlineto{\pgfqpoint{3.659592in}{2.232423in}}%
\pgfpathlineto{\pgfqpoint{3.681463in}{2.237666in}}%
\pgfpathlineto{\pgfqpoint{3.703333in}{2.242847in}}%
\pgfpathlineto{\pgfqpoint{3.725204in}{2.247966in}}%
\pgfpathlineto{\pgfqpoint{3.747074in}{2.253023in}}%
\pgfpathlineto{\pgfqpoint{3.768944in}{2.258018in}}%
\pgfpathlineto{\pgfqpoint{3.790815in}{2.262950in}}%
\pgfpathlineto{\pgfqpoint{3.812685in}{2.267820in}}%
\pgfpathlineto{\pgfqpoint{3.834555in}{2.272628in}}%
\pgfpathlineto{\pgfqpoint{3.856426in}{2.277374in}}%
\pgfpathlineto{\pgfqpoint{3.878296in}{2.282057in}}%
\pgfpathlineto{\pgfqpoint{3.900166in}{2.286678in}}%
\pgfpathlineto{\pgfqpoint{3.922037in}{2.291238in}}%
\pgfpathlineto{\pgfqpoint{3.943907in}{2.295734in}}%
\pgfpathlineto{\pgfqpoint{3.965777in}{2.300169in}}%
\pgfpathlineto{\pgfqpoint{3.987648in}{2.304542in}}%
\pgfpathlineto{\pgfqpoint{4.009518in}{2.308852in}}%
\pgfpathlineto{\pgfqpoint{4.031388in}{2.313100in}}%
\pgfpathlineto{\pgfqpoint{4.053259in}{2.317286in}}%
\pgfpathlineto{\pgfqpoint{4.075129in}{2.321410in}}%
\pgfpathlineto{\pgfqpoint{4.096999in}{2.325471in}}%
\pgfpathlineto{\pgfqpoint{4.118870in}{2.329471in}}%
\pgfpathlineto{\pgfqpoint{4.140740in}{2.333408in}}%
\pgfpathlineto{\pgfqpoint{4.162610in}{2.337282in}}%
\pgfpathlineto{\pgfqpoint{4.184481in}{2.341095in}}%
\pgfpathlineto{\pgfqpoint{4.206351in}{2.344846in}}%
\pgfpathlineto{\pgfqpoint{4.228221in}{2.348534in}}%
\pgfpathlineto{\pgfqpoint{4.250092in}{2.352160in}}%
\pgfpathlineto{\pgfqpoint{4.271962in}{2.355724in}}%
\pgfpathlineto{\pgfqpoint{4.293832in}{2.359226in}}%
\pgfpathlineto{\pgfqpoint{4.315703in}{2.362665in}}%
\pgfpathlineto{\pgfqpoint{4.337573in}{2.366042in}}%
\pgfpathlineto{\pgfqpoint{4.359444in}{2.369357in}}%
\pgfpathlineto{\pgfqpoint{4.381314in}{2.372610in}}%
\pgfpathlineto{\pgfqpoint{4.403184in}{2.375801in}}%
\pgfpathlineto{\pgfqpoint{4.425055in}{2.378930in}}%
\pgfpathlineto{\pgfqpoint{4.446925in}{2.381996in}}%
\pgfpathlineto{\pgfqpoint{4.468795in}{2.385000in}}%
\pgfpathlineto{\pgfqpoint{4.490666in}{2.387942in}}%
\pgfpathlineto{\pgfqpoint{4.512536in}{2.390821in}}%
\pgfpathlineto{\pgfqpoint{4.534406in}{2.393639in}}%
\pgfpathlineto{\pgfqpoint{4.556277in}{2.396394in}}%
\pgfpathlineto{\pgfqpoint{4.578147in}{2.399087in}}%
\pgfpathlineto{\pgfqpoint{4.600017in}{2.401718in}}%
\pgfpathlineto{\pgfqpoint{4.621888in}{2.404287in}}%
\pgfpathlineto{\pgfqpoint{4.643758in}{2.406793in}}%
\pgfpathlineto{\pgfqpoint{4.665628in}{2.409237in}}%
\pgfpathlineto{\pgfqpoint{4.687499in}{2.411619in}}%
\pgfpathlineto{\pgfqpoint{4.709369in}{2.413939in}}%
\pgfpathlineto{\pgfqpoint{4.731239in}{2.416197in}}%
\pgfpathlineto{\pgfqpoint{4.753110in}{2.418392in}}%
\pgfpathlineto{\pgfqpoint{4.774980in}{2.420526in}}%
\pgfpathlineto{\pgfqpoint{4.796850in}{2.422597in}}%
\pgfpathlineto{\pgfqpoint{4.818721in}{2.424606in}}%
\pgfpathlineto{\pgfqpoint{4.840591in}{2.426552in}}%
\pgfpathlineto{\pgfqpoint{4.862461in}{2.428437in}}%
\pgfpathlineto{\pgfqpoint{4.884332in}{2.430259in}}%
\pgfpathlineto{\pgfqpoint{4.906202in}{2.432019in}}%
\pgfpathlineto{\pgfqpoint{4.928072in}{2.433717in}}%
\pgfpathlineto{\pgfqpoint{4.949943in}{2.435352in}}%
\pgfpathlineto{\pgfqpoint{4.971813in}{2.436926in}}%
\pgfpathlineto{\pgfqpoint{4.993684in}{2.438437in}}%
\pgfpathlineto{\pgfqpoint{5.015554in}{2.439886in}}%
\pgfpathlineto{\pgfqpoint{5.037424in}{2.441273in}}%
\pgfpathlineto{\pgfqpoint{5.059295in}{2.442598in}}%
\pgfpathlineto{\pgfqpoint{5.081165in}{2.443860in}}%
\pgfpathlineto{\pgfqpoint{5.103035in}{2.445060in}}%
\pgfpathlineto{\pgfqpoint{5.124906in}{2.446198in}}%
\pgfpathlineto{\pgfqpoint{5.146776in}{2.447274in}}%
\pgfpathlineto{\pgfqpoint{5.168646in}{2.448288in}}%
\pgfpathlineto{\pgfqpoint{5.190517in}{2.449239in}}%
\pgfpathlineto{\pgfqpoint{5.212387in}{2.450128in}}%
\pgfpathlineto{\pgfqpoint{5.234257in}{2.450955in}}%
\pgfpathlineto{\pgfqpoint{5.256128in}{2.451720in}}%
\pgfpathlineto{\pgfqpoint{5.277998in}{2.452423in}}%
\pgfpathlineto{\pgfqpoint{5.299868in}{2.453063in}}%
\pgfpathlineto{\pgfqpoint{5.321739in}{2.453641in}}%
\pgfpathlineto{\pgfqpoint{5.343609in}{2.454157in}}%
\pgfpathlineto{\pgfqpoint{5.365479in}{2.454611in}}%
\pgfpathlineto{\pgfqpoint{5.387350in}{2.455003in}}%
\pgfpathlineto{\pgfqpoint{5.409220in}{2.455332in}}%
\pgfpathlineto{\pgfqpoint{5.431090in}{2.455599in}}%
\pgfpathlineto{\pgfqpoint{5.452961in}{2.455804in}}%
\pgfpathlineto{\pgfqpoint{5.474831in}{2.455947in}}%
\pgfpathlineto{\pgfqpoint{5.496701in}{2.456028in}}%
\pgfpathlineto{\pgfqpoint{5.518572in}{2.456046in}}%
\pgfpathlineto{\pgfqpoint{5.540442in}{2.456002in}}%
\pgfpathlineto{\pgfqpoint{5.562313in}{2.455896in}}%
\pgfpathlineto{\pgfqpoint{5.584183in}{2.455728in}}%
\pgfpathlineto{\pgfqpoint{5.606053in}{2.455498in}}%
\pgfpathlineto{\pgfqpoint{5.627924in}{2.455205in}}%
\pgfpathlineto{\pgfqpoint{5.649794in}{2.454850in}}%
\pgfusepath{stroke}%
\end{pgfscope}%
\begin{pgfscope}%
\pgfpathrectangle{\pgfqpoint{3.278424in}{1.886339in}}{\pgfqpoint{2.577576in}{0.913411in}} %
\pgfusepath{clip}%
\pgfsetroundcap%
\pgfsetroundjoin%
\pgfsetlinewidth{0.200750pt}%
\definecolor{currentstroke}{rgb}{0.125490,0.290196,0.529412}%
\pgfsetstrokecolor{currentstroke}%
\pgfsetdash{}{0pt}%
\pgfpathmoveto{\pgfqpoint{3.484630in}{2.263597in}}%
\pgfpathlineto{\pgfqpoint{3.506500in}{2.266203in}}%
\pgfpathlineto{\pgfqpoint{3.528370in}{2.268883in}}%
\pgfpathlineto{\pgfqpoint{3.550241in}{2.271643in}}%
\pgfpathlineto{\pgfqpoint{3.572111in}{2.274486in}}%
\pgfpathlineto{\pgfqpoint{3.593981in}{2.277418in}}%
\pgfpathlineto{\pgfqpoint{3.615852in}{2.280444in}}%
\pgfpathlineto{\pgfqpoint{3.637722in}{2.283569in}}%
\pgfpathlineto{\pgfqpoint{3.659592in}{2.286798in}}%
\pgfpathlineto{\pgfqpoint{3.681463in}{2.290136in}}%
\pgfpathlineto{\pgfqpoint{3.703333in}{2.293586in}}%
\pgfpathlineto{\pgfqpoint{3.725204in}{2.297151in}}%
\pgfpathlineto{\pgfqpoint{3.747074in}{2.300834in}}%
\pgfpathlineto{\pgfqpoint{3.768944in}{2.304634in}}%
\pgfpathlineto{\pgfqpoint{3.790815in}{2.308551in}}%
\pgfpathlineto{\pgfqpoint{3.812685in}{2.312581in}}%
\pgfpathlineto{\pgfqpoint{3.834555in}{2.316719in}}%
\pgfpathlineto{\pgfqpoint{3.856426in}{2.320959in}}%
\pgfpathlineto{\pgfqpoint{3.878296in}{2.325292in}}%
\pgfpathlineto{\pgfqpoint{3.900166in}{2.329708in}}%
\pgfpathlineto{\pgfqpoint{3.922037in}{2.334195in}}%
\pgfpathlineto{\pgfqpoint{3.943907in}{2.338741in}}%
\pgfpathlineto{\pgfqpoint{3.965777in}{2.343333in}}%
\pgfpathlineto{\pgfqpoint{3.987648in}{2.347958in}}%
\pgfpathlineto{\pgfqpoint{4.009518in}{2.352603in}}%
\pgfpathlineto{\pgfqpoint{4.031388in}{2.357254in}}%
\pgfpathlineto{\pgfqpoint{4.053259in}{2.361901in}}%
\pgfpathlineto{\pgfqpoint{4.075129in}{2.366530in}}%
\pgfpathlineto{\pgfqpoint{4.096999in}{2.371131in}}%
\pgfpathlineto{\pgfqpoint{4.118870in}{2.375695in}}%
\pgfpathlineto{\pgfqpoint{4.140740in}{2.380213in}}%
\pgfpathlineto{\pgfqpoint{4.162610in}{2.384675in}}%
\pgfpathlineto{\pgfqpoint{4.184481in}{2.389075in}}%
\pgfpathlineto{\pgfqpoint{4.206351in}{2.393406in}}%
\pgfpathlineto{\pgfqpoint{4.228221in}{2.397662in}}%
\pgfpathlineto{\pgfqpoint{4.250092in}{2.401838in}}%
\pgfpathlineto{\pgfqpoint{4.271962in}{2.405929in}}%
\pgfpathlineto{\pgfqpoint{4.293832in}{2.409931in}}%
\pgfpathlineto{\pgfqpoint{4.315703in}{2.413840in}}%
\pgfpathlineto{\pgfqpoint{4.337573in}{2.417653in}}%
\pgfpathlineto{\pgfqpoint{4.359444in}{2.421368in}}%
\pgfpathlineto{\pgfqpoint{4.381314in}{2.424982in}}%
\pgfpathlineto{\pgfqpoint{4.403184in}{2.428493in}}%
\pgfpathlineto{\pgfqpoint{4.425055in}{2.431899in}}%
\pgfpathlineto{\pgfqpoint{4.446925in}{2.435200in}}%
\pgfpathlineto{\pgfqpoint{4.468795in}{2.438394in}}%
\pgfpathlineto{\pgfqpoint{4.490666in}{2.441480in}}%
\pgfpathlineto{\pgfqpoint{4.512536in}{2.444458in}}%
\pgfpathlineto{\pgfqpoint{4.534406in}{2.447328in}}%
\pgfpathlineto{\pgfqpoint{4.556277in}{2.450089in}}%
\pgfpathlineto{\pgfqpoint{4.578147in}{2.452742in}}%
\pgfpathlineto{\pgfqpoint{4.600017in}{2.455288in}}%
\pgfpathlineto{\pgfqpoint{4.621888in}{2.457726in}}%
\pgfpathlineto{\pgfqpoint{4.643758in}{2.460059in}}%
\pgfpathlineto{\pgfqpoint{4.665628in}{2.462287in}}%
\pgfpathlineto{\pgfqpoint{4.687499in}{2.464412in}}%
\pgfpathlineto{\pgfqpoint{4.709369in}{2.466436in}}%
\pgfpathlineto{\pgfqpoint{4.731239in}{2.468361in}}%
\pgfpathlineto{\pgfqpoint{4.753110in}{2.470190in}}%
\pgfpathlineto{\pgfqpoint{4.774980in}{2.471926in}}%
\pgfpathlineto{\pgfqpoint{4.796850in}{2.473572in}}%
\pgfpathlineto{\pgfqpoint{4.818721in}{2.475131in}}%
\pgfpathlineto{\pgfqpoint{4.840591in}{2.476609in}}%
\pgfpathlineto{\pgfqpoint{4.862461in}{2.478011in}}%
\pgfpathlineto{\pgfqpoint{4.884332in}{2.479341in}}%
\pgfpathlineto{\pgfqpoint{4.906202in}{2.480605in}}%
\pgfpathlineto{\pgfqpoint{4.928072in}{2.481811in}}%
\pgfpathlineto{\pgfqpoint{4.949943in}{2.482967in}}%
\pgfpathlineto{\pgfqpoint{4.971813in}{2.484079in}}%
\pgfpathlineto{\pgfqpoint{4.993684in}{2.485157in}}%
\pgfpathlineto{\pgfqpoint{5.015554in}{2.486211in}}%
\pgfpathlineto{\pgfqpoint{5.037424in}{2.487251in}}%
\pgfpathlineto{\pgfqpoint{5.059295in}{2.488288in}}%
\pgfpathlineto{\pgfqpoint{5.081165in}{2.489332in}}%
\pgfpathlineto{\pgfqpoint{5.103035in}{2.490397in}}%
\pgfpathlineto{\pgfqpoint{5.124906in}{2.491494in}}%
\pgfpathlineto{\pgfqpoint{5.146776in}{2.492634in}}%
\pgfpathlineto{\pgfqpoint{5.168646in}{2.493830in}}%
\pgfpathlineto{\pgfqpoint{5.190517in}{2.495093in}}%
\pgfpathlineto{\pgfqpoint{5.212387in}{2.496431in}}%
\pgfpathlineto{\pgfqpoint{5.234257in}{2.497855in}}%
\pgfpathlineto{\pgfqpoint{5.256128in}{2.499372in}}%
\pgfpathlineto{\pgfqpoint{5.277998in}{2.500988in}}%
\pgfpathlineto{\pgfqpoint{5.299868in}{2.502707in}}%
\pgfpathlineto{\pgfqpoint{5.321739in}{2.504533in}}%
\pgfpathlineto{\pgfqpoint{5.343609in}{2.506467in}}%
\pgfpathlineto{\pgfqpoint{5.365479in}{2.508509in}}%
\pgfpathlineto{\pgfqpoint{5.387350in}{2.510657in}}%
\pgfpathlineto{\pgfqpoint{5.409220in}{2.512910in}}%
\pgfpathlineto{\pgfqpoint{5.431090in}{2.515264in}}%
\pgfpathlineto{\pgfqpoint{5.452961in}{2.517717in}}%
\pgfpathlineto{\pgfqpoint{5.474831in}{2.520263in}}%
\pgfpathlineto{\pgfqpoint{5.496701in}{2.522900in}}%
\pgfpathlineto{\pgfqpoint{5.518572in}{2.525621in}}%
\pgfpathlineto{\pgfqpoint{5.540442in}{2.528424in}}%
\pgfpathlineto{\pgfqpoint{5.562313in}{2.531304in}}%
\pgfpathlineto{\pgfqpoint{5.584183in}{2.534256in}}%
\pgfpathlineto{\pgfqpoint{5.606053in}{2.537277in}}%
\pgfpathlineto{\pgfqpoint{5.627924in}{2.540364in}}%
\pgfpathlineto{\pgfqpoint{5.649794in}{2.543512in}}%
\pgfusepath{stroke}%
\end{pgfscope}%
\begin{pgfscope}%
\pgfpathrectangle{\pgfqpoint{3.278424in}{1.886339in}}{\pgfqpoint{2.577576in}{0.913411in}} %
\pgfusepath{clip}%
\pgfsetroundcap%
\pgfsetroundjoin%
\pgfsetlinewidth{0.200750pt}%
\definecolor{currentstroke}{rgb}{0.125490,0.290196,0.529412}%
\pgfsetstrokecolor{currentstroke}%
\pgfsetdash{}{0pt}%
\pgfpathmoveto{\pgfqpoint{3.484630in}{2.112877in}}%
\pgfpathlineto{\pgfqpoint{3.506500in}{2.121753in}}%
\pgfpathlineto{\pgfqpoint{3.528370in}{2.130430in}}%
\pgfpathlineto{\pgfqpoint{3.550241in}{2.138904in}}%
\pgfpathlineto{\pgfqpoint{3.572111in}{2.147169in}}%
\pgfpathlineto{\pgfqpoint{3.593981in}{2.155222in}}%
\pgfpathlineto{\pgfqpoint{3.615852in}{2.163056in}}%
\pgfpathlineto{\pgfqpoint{3.637722in}{2.170666in}}%
\pgfpathlineto{\pgfqpoint{3.659592in}{2.178048in}}%
\pgfpathlineto{\pgfqpoint{3.681463in}{2.185197in}}%
\pgfpathlineto{\pgfqpoint{3.703333in}{2.192109in}}%
\pgfpathlineto{\pgfqpoint{3.725204in}{2.198782in}}%
\pgfpathlineto{\pgfqpoint{3.747074in}{2.205212in}}%
\pgfpathlineto{\pgfqpoint{3.768944in}{2.211401in}}%
\pgfpathlineto{\pgfqpoint{3.790815in}{2.217349in}}%
\pgfpathlineto{\pgfqpoint{3.812685in}{2.223059in}}%
\pgfpathlineto{\pgfqpoint{3.834555in}{2.228537in}}%
\pgfpathlineto{\pgfqpoint{3.856426in}{2.233788in}}%
\pgfpathlineto{\pgfqpoint{3.878296in}{2.238822in}}%
\pgfpathlineto{\pgfqpoint{3.900166in}{2.243649in}}%
\pgfpathlineto{\pgfqpoint{3.922037in}{2.248280in}}%
\pgfpathlineto{\pgfqpoint{3.943907in}{2.252728in}}%
\pgfpathlineto{\pgfqpoint{3.965777in}{2.257005in}}%
\pgfpathlineto{\pgfqpoint{3.987648in}{2.261125in}}%
\pgfpathlineto{\pgfqpoint{4.009518in}{2.265101in}}%
\pgfpathlineto{\pgfqpoint{4.031388in}{2.268946in}}%
\pgfpathlineto{\pgfqpoint{4.053259in}{2.272671in}}%
\pgfpathlineto{\pgfqpoint{4.075129in}{2.276290in}}%
\pgfpathlineto{\pgfqpoint{4.096999in}{2.279811in}}%
\pgfpathlineto{\pgfqpoint{4.118870in}{2.283246in}}%
\pgfpathlineto{\pgfqpoint{4.140740in}{2.286603in}}%
\pgfpathlineto{\pgfqpoint{4.162610in}{2.289890in}}%
\pgfpathlineto{\pgfqpoint{4.184481in}{2.293115in}}%
\pgfpathlineto{\pgfqpoint{4.206351in}{2.296285in}}%
\pgfpathlineto{\pgfqpoint{4.228221in}{2.299406in}}%
\pgfpathlineto{\pgfqpoint{4.250092in}{2.302482in}}%
\pgfpathlineto{\pgfqpoint{4.271962in}{2.305519in}}%
\pgfpathlineto{\pgfqpoint{4.293832in}{2.308521in}}%
\pgfpathlineto{\pgfqpoint{4.315703in}{2.311491in}}%
\pgfpathlineto{\pgfqpoint{4.337573in}{2.314432in}}%
\pgfpathlineto{\pgfqpoint{4.359444in}{2.317347in}}%
\pgfpathlineto{\pgfqpoint{4.381314in}{2.320239in}}%
\pgfpathlineto{\pgfqpoint{4.403184in}{2.323109in}}%
\pgfpathlineto{\pgfqpoint{4.425055in}{2.325960in}}%
\pgfpathlineto{\pgfqpoint{4.446925in}{2.328792in}}%
\pgfpathlineto{\pgfqpoint{4.468795in}{2.331606in}}%
\pgfpathlineto{\pgfqpoint{4.490666in}{2.334403in}}%
\pgfpathlineto{\pgfqpoint{4.512536in}{2.337185in}}%
\pgfpathlineto{\pgfqpoint{4.534406in}{2.339950in}}%
\pgfpathlineto{\pgfqpoint{4.556277in}{2.342699in}}%
\pgfpathlineto{\pgfqpoint{4.578147in}{2.345432in}}%
\pgfpathlineto{\pgfqpoint{4.600017in}{2.348148in}}%
\pgfpathlineto{\pgfqpoint{4.621888in}{2.350847in}}%
\pgfpathlineto{\pgfqpoint{4.643758in}{2.353527in}}%
\pgfpathlineto{\pgfqpoint{4.665628in}{2.356188in}}%
\pgfpathlineto{\pgfqpoint{4.687499in}{2.358827in}}%
\pgfpathlineto{\pgfqpoint{4.709369in}{2.361442in}}%
\pgfpathlineto{\pgfqpoint{4.731239in}{2.364033in}}%
\pgfpathlineto{\pgfqpoint{4.753110in}{2.366595in}}%
\pgfpathlineto{\pgfqpoint{4.774980in}{2.369125in}}%
\pgfpathlineto{\pgfqpoint{4.796850in}{2.371622in}}%
\pgfpathlineto{\pgfqpoint{4.818721in}{2.374080in}}%
\pgfpathlineto{\pgfqpoint{4.840591in}{2.376495in}}%
\pgfpathlineto{\pgfqpoint{4.862461in}{2.378863in}}%
\pgfpathlineto{\pgfqpoint{4.884332in}{2.381177in}}%
\pgfpathlineto{\pgfqpoint{4.906202in}{2.383433in}}%
\pgfpathlineto{\pgfqpoint{4.928072in}{2.385622in}}%
\pgfpathlineto{\pgfqpoint{4.949943in}{2.387738in}}%
\pgfpathlineto{\pgfqpoint{4.971813in}{2.389773in}}%
\pgfpathlineto{\pgfqpoint{4.993684in}{2.391717in}}%
\pgfpathlineto{\pgfqpoint{5.015554in}{2.393561in}}%
\pgfpathlineto{\pgfqpoint{5.037424in}{2.395295in}}%
\pgfpathlineto{\pgfqpoint{5.059295in}{2.396908in}}%
\pgfpathlineto{\pgfqpoint{5.081165in}{2.398388in}}%
\pgfpathlineto{\pgfqpoint{5.103035in}{2.399723in}}%
\pgfpathlineto{\pgfqpoint{5.124906in}{2.400903in}}%
\pgfpathlineto{\pgfqpoint{5.146776in}{2.401914in}}%
\pgfpathlineto{\pgfqpoint{5.168646in}{2.402745in}}%
\pgfpathlineto{\pgfqpoint{5.190517in}{2.403386in}}%
\pgfpathlineto{\pgfqpoint{5.212387in}{2.403826in}}%
\pgfpathlineto{\pgfqpoint{5.234257in}{2.404056in}}%
\pgfpathlineto{\pgfqpoint{5.256128in}{2.404068in}}%
\pgfpathlineto{\pgfqpoint{5.277998in}{2.403858in}}%
\pgfpathlineto{\pgfqpoint{5.299868in}{2.403419in}}%
\pgfpathlineto{\pgfqpoint{5.321739in}{2.402750in}}%
\pgfpathlineto{\pgfqpoint{5.343609in}{2.401848in}}%
\pgfpathlineto{\pgfqpoint{5.365479in}{2.400714in}}%
\pgfpathlineto{\pgfqpoint{5.387350in}{2.399349in}}%
\pgfpathlineto{\pgfqpoint{5.409220in}{2.397755in}}%
\pgfpathlineto{\pgfqpoint{5.431090in}{2.395935in}}%
\pgfpathlineto{\pgfqpoint{5.452961in}{2.393892in}}%
\pgfpathlineto{\pgfqpoint{5.474831in}{2.391631in}}%
\pgfpathlineto{\pgfqpoint{5.496701in}{2.389156in}}%
\pgfpathlineto{\pgfqpoint{5.518572in}{2.386471in}}%
\pgfpathlineto{\pgfqpoint{5.540442in}{2.383581in}}%
\pgfpathlineto{\pgfqpoint{5.562313in}{2.380489in}}%
\pgfpathlineto{\pgfqpoint{5.584183in}{2.377200in}}%
\pgfpathlineto{\pgfqpoint{5.606053in}{2.373718in}}%
\pgfpathlineto{\pgfqpoint{5.627924in}{2.370047in}}%
\pgfpathlineto{\pgfqpoint{5.649794in}{2.366189in}}%
\pgfusepath{stroke}%
\end{pgfscope}%
\begin{pgfscope}%
\pgfpathrectangle{\pgfqpoint{3.278424in}{1.886339in}}{\pgfqpoint{2.577576in}{0.913411in}} %
\pgfusepath{clip}%
\pgfsetbuttcap%
\pgfsetbeveljoin%
\definecolor{currentfill}{rgb}{0.298039,0.447059,0.690196}%
\pgfsetfillcolor{currentfill}%
\pgfsetlinewidth{0.000000pt}%
\definecolor{currentstroke}{rgb}{0.000000,0.000000,0.000000}%
\pgfsetstrokecolor{currentstroke}%
\pgfsetdash{}{0pt}%
\pgfsys@defobject{currentmarker}{\pgfqpoint{-0.036986in}{-0.031462in}}{\pgfqpoint{0.036986in}{0.038889in}}{%
\pgfpathmoveto{\pgfqpoint{0.000000in}{0.038889in}}%
\pgfpathlineto{\pgfqpoint{-0.008731in}{0.012017in}}%
\pgfpathlineto{\pgfqpoint{-0.036986in}{0.012017in}}%
\pgfpathlineto{\pgfqpoint{-0.014127in}{-0.004590in}}%
\pgfpathlineto{\pgfqpoint{-0.022858in}{-0.031462in}}%
\pgfpathlineto{\pgfqpoint{-0.000000in}{-0.014854in}}%
\pgfpathlineto{\pgfqpoint{0.022858in}{-0.031462in}}%
\pgfpathlineto{\pgfqpoint{0.014127in}{-0.004590in}}%
\pgfpathlineto{\pgfqpoint{0.036986in}{0.012017in}}%
\pgfpathlineto{\pgfqpoint{0.008731in}{0.012017in}}%
\pgfpathclose%
\pgfusepath{fill}%
}%
\begin{pgfscope}%
\pgfsys@transformshift{4.010455in}{2.449609in}%
\pgfsys@useobject{currentmarker}{}%
\end{pgfscope}%
\begin{pgfscope}%
\pgfsys@transformshift{4.814659in}{2.327821in}%
\pgfsys@useobject{currentmarker}{}%
\end{pgfscope}%
\begin{pgfscope}%
\pgfsys@transformshift{4.536281in}{2.434386in}%
\pgfsys@useobject{currentmarker}{}%
\end{pgfscope}%
\begin{pgfscope}%
\pgfsys@transformshift{3.525871in}{2.632291in}%
\pgfsys@useobject{currentmarker}{}%
\end{pgfscope}%
\begin{pgfscope}%
\pgfsys@transformshift{3.536181in}{2.312597in}%
\pgfsys@useobject{currentmarker}{}%
\end{pgfscope}%
\begin{pgfscope}%
\pgfsys@transformshift{5.453898in}{2.693185in}%
\pgfsys@useobject{currentmarker}{}%
\end{pgfscope}%
\begin{pgfscope}%
\pgfsys@transformshift{5.443588in}{2.312597in}%
\pgfsys@useobject{currentmarker}{}%
\end{pgfscope}%
\begin{pgfscope}%
\pgfsys@transformshift{5.649794in}{2.754079in}%
\pgfsys@useobject{currentmarker}{}%
\end{pgfscope}%
\begin{pgfscope}%
\pgfsys@transformshift{3.567112in}{2.053798in}%
\pgfsys@useobject{currentmarker}{}%
\end{pgfscope}%
\begin{pgfscope}%
\pgfsys@transformshift{4.361006in}{2.343044in}%
\pgfsys@useobject{currentmarker}{}%
\end{pgfscope}%
\begin{pgfscope}%
\pgfsys@transformshift{5.268313in}{2.297374in}%
\pgfsys@useobject{currentmarker}{}%
\end{pgfscope}%
\begin{pgfscope}%
\pgfsys@transformshift{5.422967in}{2.312597in}%
\pgfsys@useobject{currentmarker}{}%
\end{pgfscope}%
\begin{pgfscope}%
\pgfsys@transformshift{5.618863in}{2.358268in}%
\pgfsys@useobject{currentmarker}{}%
\end{pgfscope}%
\begin{pgfscope}%
\pgfsys@transformshift{3.484630in}{1.992904in}%
\pgfsys@useobject{currentmarker}{}%
\end{pgfscope}%
\begin{pgfscope}%
\pgfsys@transformshift{4.196041in}{2.525727in}%
\pgfsys@useobject{currentmarker}{}%
\end{pgfscope}%
\begin{pgfscope}%
\pgfsys@transformshift{5.443588in}{2.480056in}%
\pgfsys@useobject{currentmarker}{}%
\end{pgfscope}%
\begin{pgfscope}%
\pgfsys@transformshift{4.340385in}{2.236480in}%
\pgfsys@useobject{currentmarker}{}%
\end{pgfscope}%
\begin{pgfscope}%
\pgfsys@transformshift{3.793939in}{2.327821in}%
\pgfsys@useobject{currentmarker}{}%
\end{pgfscope}%
\begin{pgfscope}%
\pgfsys@transformshift{5.350795in}{2.114692in}%
\pgfsys@useobject{currentmarker}{}%
\end{pgfscope}%
\begin{pgfscope}%
\pgfsys@transformshift{5.464208in}{2.251703in}%
\pgfsys@useobject{currentmarker}{}%
\end{pgfscope}%
\begin{pgfscope}%
\pgfsys@transformshift{3.835180in}{2.327821in}%
\pgfsys@useobject{currentmarker}{}%
\end{pgfscope}%
\begin{pgfscope}%
\pgfsys@transformshift{5.422967in}{2.540950in}%
\pgfsys@useobject{currentmarker}{}%
\end{pgfscope}%
\begin{pgfscope}%
\pgfsys@transformshift{3.773318in}{2.373491in}%
\pgfsys@useobject{currentmarker}{}%
\end{pgfscope}%
\begin{pgfscope}%
\pgfsys@transformshift{3.752698in}{2.160362in}%
\pgfsys@useobject{currentmarker}{}%
\end{pgfscope}%
\begin{pgfscope}%
\pgfsys@transformshift{5.443588in}{2.419162in}%
\pgfsys@useobject{currentmarker}{}%
\end{pgfscope}%
\begin{pgfscope}%
\pgfsys@transformshift{5.154899in}{2.617068in}%
\pgfsys@useobject{currentmarker}{}%
\end{pgfscope}%
\begin{pgfscope}%
\pgfsys@transformshift{3.536181in}{2.175586in}%
\pgfsys@useobject{currentmarker}{}%
\end{pgfscope}%
\begin{pgfscope}%
\pgfsys@transformshift{4.433178in}{2.069021in}%
\pgfsys@useobject{currentmarker}{}%
\end{pgfscope}%
\begin{pgfscope}%
\pgfsys@transformshift{3.711456in}{2.129915in}%
\pgfsys@useobject{currentmarker}{}%
\end{pgfscope}%
\begin{pgfscope}%
\pgfsys@transformshift{3.752698in}{2.632291in}%
\pgfsys@useobject{currentmarker}{}%
\end{pgfscope}%
\end{pgfscope}%
\begin{pgfscope}%
\pgfsetrectcap%
\pgfsetmiterjoin%
\pgfsetlinewidth{0.000000pt}%
\definecolor{currentstroke}{rgb}{1.000000,1.000000,1.000000}%
\pgfsetstrokecolor{currentstroke}%
\pgfsetdash{}{0pt}%
\pgfpathmoveto{\pgfqpoint{5.856000in}{1.886339in}}%
\pgfpathlineto{\pgfqpoint{5.856000in}{2.799750in}}%
\pgfusepath{}%
\end{pgfscope}%
\begin{pgfscope}%
\pgfsetrectcap%
\pgfsetmiterjoin%
\pgfsetlinewidth{0.000000pt}%
\definecolor{currentstroke}{rgb}{1.000000,1.000000,1.000000}%
\pgfsetstrokecolor{currentstroke}%
\pgfsetdash{}{0pt}%
\pgfpathmoveto{\pgfqpoint{3.278424in}{1.886339in}}%
\pgfpathlineto{\pgfqpoint{5.856000in}{1.886339in}}%
\pgfusepath{}%
\end{pgfscope}%
\begin{pgfscope}%
\pgfsetrectcap%
\pgfsetmiterjoin%
\pgfsetlinewidth{0.000000pt}%
\definecolor{currentstroke}{rgb}{1.000000,1.000000,1.000000}%
\pgfsetstrokecolor{currentstroke}%
\pgfsetdash{}{0pt}%
\pgfpathmoveto{\pgfqpoint{3.278424in}{1.886339in}}%
\pgfpathlineto{\pgfqpoint{3.278424in}{2.799750in}}%
\pgfusepath{}%
\end{pgfscope}%
\begin{pgfscope}%
\pgfsetrectcap%
\pgfsetmiterjoin%
\pgfsetlinewidth{0.000000pt}%
\definecolor{currentstroke}{rgb}{1.000000,1.000000,1.000000}%
\pgfsetstrokecolor{currentstroke}%
\pgfsetdash{}{0pt}%
\pgfpathmoveto{\pgfqpoint{3.278424in}{2.799750in}}%
\pgfpathlineto{\pgfqpoint{5.856000in}{2.799750in}}%
\pgfusepath{}%
\end{pgfscope}%
\begin{pgfscope}%
\pgfsetbuttcap%
\pgfsetmiterjoin%
\definecolor{currentfill}{rgb}{0.917647,0.917647,0.949020}%
\pgfsetfillcolor{currentfill}%
\pgfsetlinewidth{0.000000pt}%
\definecolor{currentstroke}{rgb}{0.000000,0.000000,0.000000}%
\pgfsetstrokecolor{currentstroke}%
\pgfsetstrokeopacity{0.000000}%
\pgfsetdash{}{0pt}%
\pgfpathmoveto{\pgfqpoint{0.556847in}{0.516222in}}%
\pgfpathlineto{\pgfqpoint{3.134424in}{0.516222in}}%
\pgfpathlineto{\pgfqpoint{3.134424in}{1.429633in}}%
\pgfpathlineto{\pgfqpoint{0.556847in}{1.429633in}}%
\pgfpathclose%
\pgfusepath{fill}%
\end{pgfscope}%
\begin{pgfscope}%
\pgfpathrectangle{\pgfqpoint{0.556847in}{0.516222in}}{\pgfqpoint{2.577576in}{0.913411in}} %
\pgfusepath{clip}%
\pgfsetroundcap%
\pgfsetroundjoin%
\pgfsetlinewidth{0.803000pt}%
\definecolor{currentstroke}{rgb}{1.000000,1.000000,1.000000}%
\pgfsetstrokecolor{currentstroke}%
\pgfsetdash{}{0pt}%
\pgfpathmoveto{\pgfqpoint{0.973808in}{0.516222in}}%
\pgfpathlineto{\pgfqpoint{0.973808in}{1.429633in}}%
\pgfusepath{stroke}%
\end{pgfscope}%
\begin{pgfscope}%
\pgfsetbuttcap%
\pgfsetroundjoin%
\definecolor{currentfill}{rgb}{0.150000,0.150000,0.150000}%
\pgfsetfillcolor{currentfill}%
\pgfsetlinewidth{0.803000pt}%
\definecolor{currentstroke}{rgb}{0.150000,0.150000,0.150000}%
\pgfsetstrokecolor{currentstroke}%
\pgfsetdash{}{0pt}%
\pgfsys@defobject{currentmarker}{\pgfqpoint{0.000000in}{0.000000in}}{\pgfqpoint{0.000000in}{0.000000in}}{%
\pgfpathmoveto{\pgfqpoint{0.000000in}{0.000000in}}%
\pgfpathlineto{\pgfqpoint{0.000000in}{0.000000in}}%
\pgfusepath{stroke,fill}%
}%
\begin{pgfscope}%
\pgfsys@transformshift{0.973808in}{0.516222in}%
\pgfsys@useobject{currentmarker}{}%
\end{pgfscope}%
\end{pgfscope}%
\begin{pgfscope}%
\pgfsetbuttcap%
\pgfsetroundjoin%
\definecolor{currentfill}{rgb}{0.150000,0.150000,0.150000}%
\pgfsetfillcolor{currentfill}%
\pgfsetlinewidth{0.803000pt}%
\definecolor{currentstroke}{rgb}{0.150000,0.150000,0.150000}%
\pgfsetstrokecolor{currentstroke}%
\pgfsetdash{}{0pt}%
\pgfsys@defobject{currentmarker}{\pgfqpoint{0.000000in}{0.000000in}}{\pgfqpoint{0.000000in}{0.000000in}}{%
\pgfpathmoveto{\pgfqpoint{0.000000in}{0.000000in}}%
\pgfpathlineto{\pgfqpoint{0.000000in}{0.000000in}}%
\pgfusepath{stroke,fill}%
}%
\begin{pgfscope}%
\pgfsys@transformshift{0.973808in}{1.429633in}%
\pgfsys@useobject{currentmarker}{}%
\end{pgfscope}%
\end{pgfscope}%
\begin{pgfscope}%
\definecolor{textcolor}{rgb}{0.150000,0.150000,0.150000}%
\pgfsetstrokecolor{textcolor}%
\pgfsetfillcolor{textcolor}%
\pgftext[x=0.973808in,y=0.438444in,,top]{\color{textcolor}\sffamily\fontsize{8.000000}{9.600000}\selectfont 6}%
\end{pgfscope}%
\begin{pgfscope}%
\pgfpathrectangle{\pgfqpoint{0.556847in}{0.516222in}}{\pgfqpoint{2.577576in}{0.913411in}} %
\pgfusepath{clip}%
\pgfsetroundcap%
\pgfsetroundjoin%
\pgfsetlinewidth{0.803000pt}%
\definecolor{currentstroke}{rgb}{1.000000,1.000000,1.000000}%
\pgfsetstrokecolor{currentstroke}%
\pgfsetdash{}{0pt}%
\pgfpathmoveto{\pgfqpoint{1.447627in}{0.516222in}}%
\pgfpathlineto{\pgfqpoint{1.447627in}{1.429633in}}%
\pgfusepath{stroke}%
\end{pgfscope}%
\begin{pgfscope}%
\pgfsetbuttcap%
\pgfsetroundjoin%
\definecolor{currentfill}{rgb}{0.150000,0.150000,0.150000}%
\pgfsetfillcolor{currentfill}%
\pgfsetlinewidth{0.803000pt}%
\definecolor{currentstroke}{rgb}{0.150000,0.150000,0.150000}%
\pgfsetstrokecolor{currentstroke}%
\pgfsetdash{}{0pt}%
\pgfsys@defobject{currentmarker}{\pgfqpoint{0.000000in}{0.000000in}}{\pgfqpoint{0.000000in}{0.000000in}}{%
\pgfpathmoveto{\pgfqpoint{0.000000in}{0.000000in}}%
\pgfpathlineto{\pgfqpoint{0.000000in}{0.000000in}}%
\pgfusepath{stroke,fill}%
}%
\begin{pgfscope}%
\pgfsys@transformshift{1.447627in}{0.516222in}%
\pgfsys@useobject{currentmarker}{}%
\end{pgfscope}%
\end{pgfscope}%
\begin{pgfscope}%
\pgfsetbuttcap%
\pgfsetroundjoin%
\definecolor{currentfill}{rgb}{0.150000,0.150000,0.150000}%
\pgfsetfillcolor{currentfill}%
\pgfsetlinewidth{0.803000pt}%
\definecolor{currentstroke}{rgb}{0.150000,0.150000,0.150000}%
\pgfsetstrokecolor{currentstroke}%
\pgfsetdash{}{0pt}%
\pgfsys@defobject{currentmarker}{\pgfqpoint{0.000000in}{0.000000in}}{\pgfqpoint{0.000000in}{0.000000in}}{%
\pgfpathmoveto{\pgfqpoint{0.000000in}{0.000000in}}%
\pgfpathlineto{\pgfqpoint{0.000000in}{0.000000in}}%
\pgfusepath{stroke,fill}%
}%
\begin{pgfscope}%
\pgfsys@transformshift{1.447627in}{1.429633in}%
\pgfsys@useobject{currentmarker}{}%
\end{pgfscope}%
\end{pgfscope}%
\begin{pgfscope}%
\definecolor{textcolor}{rgb}{0.150000,0.150000,0.150000}%
\pgfsetstrokecolor{textcolor}%
\pgfsetfillcolor{textcolor}%
\pgftext[x=1.447627in,y=0.438444in,,top]{\color{textcolor}\sffamily\fontsize{8.000000}{9.600000}\selectfont 7}%
\end{pgfscope}%
\begin{pgfscope}%
\pgfpathrectangle{\pgfqpoint{0.556847in}{0.516222in}}{\pgfqpoint{2.577576in}{0.913411in}} %
\pgfusepath{clip}%
\pgfsetroundcap%
\pgfsetroundjoin%
\pgfsetlinewidth{0.803000pt}%
\definecolor{currentstroke}{rgb}{1.000000,1.000000,1.000000}%
\pgfsetstrokecolor{currentstroke}%
\pgfsetdash{}{0pt}%
\pgfpathmoveto{\pgfqpoint{1.921446in}{0.516222in}}%
\pgfpathlineto{\pgfqpoint{1.921446in}{1.429633in}}%
\pgfusepath{stroke}%
\end{pgfscope}%
\begin{pgfscope}%
\pgfsetbuttcap%
\pgfsetroundjoin%
\definecolor{currentfill}{rgb}{0.150000,0.150000,0.150000}%
\pgfsetfillcolor{currentfill}%
\pgfsetlinewidth{0.803000pt}%
\definecolor{currentstroke}{rgb}{0.150000,0.150000,0.150000}%
\pgfsetstrokecolor{currentstroke}%
\pgfsetdash{}{0pt}%
\pgfsys@defobject{currentmarker}{\pgfqpoint{0.000000in}{0.000000in}}{\pgfqpoint{0.000000in}{0.000000in}}{%
\pgfpathmoveto{\pgfqpoint{0.000000in}{0.000000in}}%
\pgfpathlineto{\pgfqpoint{0.000000in}{0.000000in}}%
\pgfusepath{stroke,fill}%
}%
\begin{pgfscope}%
\pgfsys@transformshift{1.921446in}{0.516222in}%
\pgfsys@useobject{currentmarker}{}%
\end{pgfscope}%
\end{pgfscope}%
\begin{pgfscope}%
\pgfsetbuttcap%
\pgfsetroundjoin%
\definecolor{currentfill}{rgb}{0.150000,0.150000,0.150000}%
\pgfsetfillcolor{currentfill}%
\pgfsetlinewidth{0.803000pt}%
\definecolor{currentstroke}{rgb}{0.150000,0.150000,0.150000}%
\pgfsetstrokecolor{currentstroke}%
\pgfsetdash{}{0pt}%
\pgfsys@defobject{currentmarker}{\pgfqpoint{0.000000in}{0.000000in}}{\pgfqpoint{0.000000in}{0.000000in}}{%
\pgfpathmoveto{\pgfqpoint{0.000000in}{0.000000in}}%
\pgfpathlineto{\pgfqpoint{0.000000in}{0.000000in}}%
\pgfusepath{stroke,fill}%
}%
\begin{pgfscope}%
\pgfsys@transformshift{1.921446in}{1.429633in}%
\pgfsys@useobject{currentmarker}{}%
\end{pgfscope}%
\end{pgfscope}%
\begin{pgfscope}%
\definecolor{textcolor}{rgb}{0.150000,0.150000,0.150000}%
\pgfsetstrokecolor{textcolor}%
\pgfsetfillcolor{textcolor}%
\pgftext[x=1.921446in,y=0.438444in,,top]{\color{textcolor}\sffamily\fontsize{8.000000}{9.600000}\selectfont 8}%
\end{pgfscope}%
\begin{pgfscope}%
\pgfpathrectangle{\pgfqpoint{0.556847in}{0.516222in}}{\pgfqpoint{2.577576in}{0.913411in}} %
\pgfusepath{clip}%
\pgfsetroundcap%
\pgfsetroundjoin%
\pgfsetlinewidth{0.803000pt}%
\definecolor{currentstroke}{rgb}{1.000000,1.000000,1.000000}%
\pgfsetstrokecolor{currentstroke}%
\pgfsetdash{}{0pt}%
\pgfpathmoveto{\pgfqpoint{2.395266in}{0.516222in}}%
\pgfpathlineto{\pgfqpoint{2.395266in}{1.429633in}}%
\pgfusepath{stroke}%
\end{pgfscope}%
\begin{pgfscope}%
\pgfsetbuttcap%
\pgfsetroundjoin%
\definecolor{currentfill}{rgb}{0.150000,0.150000,0.150000}%
\pgfsetfillcolor{currentfill}%
\pgfsetlinewidth{0.803000pt}%
\definecolor{currentstroke}{rgb}{0.150000,0.150000,0.150000}%
\pgfsetstrokecolor{currentstroke}%
\pgfsetdash{}{0pt}%
\pgfsys@defobject{currentmarker}{\pgfqpoint{0.000000in}{0.000000in}}{\pgfqpoint{0.000000in}{0.000000in}}{%
\pgfpathmoveto{\pgfqpoint{0.000000in}{0.000000in}}%
\pgfpathlineto{\pgfqpoint{0.000000in}{0.000000in}}%
\pgfusepath{stroke,fill}%
}%
\begin{pgfscope}%
\pgfsys@transformshift{2.395266in}{0.516222in}%
\pgfsys@useobject{currentmarker}{}%
\end{pgfscope}%
\end{pgfscope}%
\begin{pgfscope}%
\pgfsetbuttcap%
\pgfsetroundjoin%
\definecolor{currentfill}{rgb}{0.150000,0.150000,0.150000}%
\pgfsetfillcolor{currentfill}%
\pgfsetlinewidth{0.803000pt}%
\definecolor{currentstroke}{rgb}{0.150000,0.150000,0.150000}%
\pgfsetstrokecolor{currentstroke}%
\pgfsetdash{}{0pt}%
\pgfsys@defobject{currentmarker}{\pgfqpoint{0.000000in}{0.000000in}}{\pgfqpoint{0.000000in}{0.000000in}}{%
\pgfpathmoveto{\pgfqpoint{0.000000in}{0.000000in}}%
\pgfpathlineto{\pgfqpoint{0.000000in}{0.000000in}}%
\pgfusepath{stroke,fill}%
}%
\begin{pgfscope}%
\pgfsys@transformshift{2.395266in}{1.429633in}%
\pgfsys@useobject{currentmarker}{}%
\end{pgfscope}%
\end{pgfscope}%
\begin{pgfscope}%
\definecolor{textcolor}{rgb}{0.150000,0.150000,0.150000}%
\pgfsetstrokecolor{textcolor}%
\pgfsetfillcolor{textcolor}%
\pgftext[x=2.395266in,y=0.438444in,,top]{\color{textcolor}\sffamily\fontsize{8.000000}{9.600000}\selectfont 9}%
\end{pgfscope}%
\begin{pgfscope}%
\pgfpathrectangle{\pgfqpoint{0.556847in}{0.516222in}}{\pgfqpoint{2.577576in}{0.913411in}} %
\pgfusepath{clip}%
\pgfsetroundcap%
\pgfsetroundjoin%
\pgfsetlinewidth{0.803000pt}%
\definecolor{currentstroke}{rgb}{1.000000,1.000000,1.000000}%
\pgfsetstrokecolor{currentstroke}%
\pgfsetdash{}{0pt}%
\pgfpathmoveto{\pgfqpoint{2.869085in}{0.516222in}}%
\pgfpathlineto{\pgfqpoint{2.869085in}{1.429633in}}%
\pgfusepath{stroke}%
\end{pgfscope}%
\begin{pgfscope}%
\pgfsetbuttcap%
\pgfsetroundjoin%
\definecolor{currentfill}{rgb}{0.150000,0.150000,0.150000}%
\pgfsetfillcolor{currentfill}%
\pgfsetlinewidth{0.803000pt}%
\definecolor{currentstroke}{rgb}{0.150000,0.150000,0.150000}%
\pgfsetstrokecolor{currentstroke}%
\pgfsetdash{}{0pt}%
\pgfsys@defobject{currentmarker}{\pgfqpoint{0.000000in}{0.000000in}}{\pgfqpoint{0.000000in}{0.000000in}}{%
\pgfpathmoveto{\pgfqpoint{0.000000in}{0.000000in}}%
\pgfpathlineto{\pgfqpoint{0.000000in}{0.000000in}}%
\pgfusepath{stroke,fill}%
}%
\begin{pgfscope}%
\pgfsys@transformshift{2.869085in}{0.516222in}%
\pgfsys@useobject{currentmarker}{}%
\end{pgfscope}%
\end{pgfscope}%
\begin{pgfscope}%
\pgfsetbuttcap%
\pgfsetroundjoin%
\definecolor{currentfill}{rgb}{0.150000,0.150000,0.150000}%
\pgfsetfillcolor{currentfill}%
\pgfsetlinewidth{0.803000pt}%
\definecolor{currentstroke}{rgb}{0.150000,0.150000,0.150000}%
\pgfsetstrokecolor{currentstroke}%
\pgfsetdash{}{0pt}%
\pgfsys@defobject{currentmarker}{\pgfqpoint{0.000000in}{0.000000in}}{\pgfqpoint{0.000000in}{0.000000in}}{%
\pgfpathmoveto{\pgfqpoint{0.000000in}{0.000000in}}%
\pgfpathlineto{\pgfqpoint{0.000000in}{0.000000in}}%
\pgfusepath{stroke,fill}%
}%
\begin{pgfscope}%
\pgfsys@transformshift{2.869085in}{1.429633in}%
\pgfsys@useobject{currentmarker}{}%
\end{pgfscope}%
\end{pgfscope}%
\begin{pgfscope}%
\definecolor{textcolor}{rgb}{0.150000,0.150000,0.150000}%
\pgfsetstrokecolor{textcolor}%
\pgfsetfillcolor{textcolor}%
\pgftext[x=2.869085in,y=0.438444in,,top]{\color{textcolor}\sffamily\fontsize{8.000000}{9.600000}\selectfont 10}%
\end{pgfscope}%
\begin{pgfscope}%
\definecolor{textcolor}{rgb}{0.150000,0.150000,0.150000}%
\pgfsetstrokecolor{textcolor}%
\pgfsetfillcolor{textcolor}%
\pgftext[x=1.845635in,y=0.273321in,,top]{\color{textcolor}\sffamily\fontsize{8.800000}{10.560000}\selectfont Tail length}%
\end{pgfscope}%
\begin{pgfscope}%
\pgfpathrectangle{\pgfqpoint{0.556847in}{0.516222in}}{\pgfqpoint{2.577576in}{0.913411in}} %
\pgfusepath{clip}%
\pgfsetroundcap%
\pgfsetroundjoin%
\pgfsetlinewidth{0.803000pt}%
\definecolor{currentstroke}{rgb}{1.000000,1.000000,1.000000}%
\pgfsetstrokecolor{currentstroke}%
\pgfsetdash{}{0pt}%
\pgfpathmoveto{\pgfqpoint{0.556847in}{0.516222in}}%
\pgfpathlineto{\pgfqpoint{3.134424in}{0.516222in}}%
\pgfusepath{stroke}%
\end{pgfscope}%
\begin{pgfscope}%
\pgfsetbuttcap%
\pgfsetroundjoin%
\definecolor{currentfill}{rgb}{0.150000,0.150000,0.150000}%
\pgfsetfillcolor{currentfill}%
\pgfsetlinewidth{0.803000pt}%
\definecolor{currentstroke}{rgb}{0.150000,0.150000,0.150000}%
\pgfsetstrokecolor{currentstroke}%
\pgfsetdash{}{0pt}%
\pgfsys@defobject{currentmarker}{\pgfqpoint{0.000000in}{0.000000in}}{\pgfqpoint{0.000000in}{0.000000in}}{%
\pgfpathmoveto{\pgfqpoint{0.000000in}{0.000000in}}%
\pgfpathlineto{\pgfqpoint{0.000000in}{0.000000in}}%
\pgfusepath{stroke,fill}%
}%
\begin{pgfscope}%
\pgfsys@transformshift{0.556847in}{0.516222in}%
\pgfsys@useobject{currentmarker}{}%
\end{pgfscope}%
\end{pgfscope}%
\begin{pgfscope}%
\pgfsetbuttcap%
\pgfsetroundjoin%
\definecolor{currentfill}{rgb}{0.150000,0.150000,0.150000}%
\pgfsetfillcolor{currentfill}%
\pgfsetlinewidth{0.803000pt}%
\definecolor{currentstroke}{rgb}{0.150000,0.150000,0.150000}%
\pgfsetstrokecolor{currentstroke}%
\pgfsetdash{}{0pt}%
\pgfsys@defobject{currentmarker}{\pgfqpoint{0.000000in}{0.000000in}}{\pgfqpoint{0.000000in}{0.000000in}}{%
\pgfpathmoveto{\pgfqpoint{0.000000in}{0.000000in}}%
\pgfpathlineto{\pgfqpoint{0.000000in}{0.000000in}}%
\pgfusepath{stroke,fill}%
}%
\begin{pgfscope}%
\pgfsys@transformshift{3.134424in}{0.516222in}%
\pgfsys@useobject{currentmarker}{}%
\end{pgfscope}%
\end{pgfscope}%
\begin{pgfscope}%
\definecolor{textcolor}{rgb}{0.150000,0.150000,0.150000}%
\pgfsetstrokecolor{textcolor}%
\pgfsetfillcolor{textcolor}%
\pgftext[x=0.479069in,y=0.516222in,right,]{\color{textcolor}\sffamily\fontsize{8.000000}{9.600000}\selectfont 2.0}%
\end{pgfscope}%
\begin{pgfscope}%
\pgfpathrectangle{\pgfqpoint{0.556847in}{0.516222in}}{\pgfqpoint{2.577576in}{0.913411in}} %
\pgfusepath{clip}%
\pgfsetroundcap%
\pgfsetroundjoin%
\pgfsetlinewidth{0.803000pt}%
\definecolor{currentstroke}{rgb}{1.000000,1.000000,1.000000}%
\pgfsetstrokecolor{currentstroke}%
\pgfsetdash{}{0pt}%
\pgfpathmoveto{\pgfqpoint{0.556847in}{0.668457in}}%
\pgfpathlineto{\pgfqpoint{3.134424in}{0.668457in}}%
\pgfusepath{stroke}%
\end{pgfscope}%
\begin{pgfscope}%
\pgfsetbuttcap%
\pgfsetroundjoin%
\definecolor{currentfill}{rgb}{0.150000,0.150000,0.150000}%
\pgfsetfillcolor{currentfill}%
\pgfsetlinewidth{0.803000pt}%
\definecolor{currentstroke}{rgb}{0.150000,0.150000,0.150000}%
\pgfsetstrokecolor{currentstroke}%
\pgfsetdash{}{0pt}%
\pgfsys@defobject{currentmarker}{\pgfqpoint{0.000000in}{0.000000in}}{\pgfqpoint{0.000000in}{0.000000in}}{%
\pgfpathmoveto{\pgfqpoint{0.000000in}{0.000000in}}%
\pgfpathlineto{\pgfqpoint{0.000000in}{0.000000in}}%
\pgfusepath{stroke,fill}%
}%
\begin{pgfscope}%
\pgfsys@transformshift{0.556847in}{0.668457in}%
\pgfsys@useobject{currentmarker}{}%
\end{pgfscope}%
\end{pgfscope}%
\begin{pgfscope}%
\pgfsetbuttcap%
\pgfsetroundjoin%
\definecolor{currentfill}{rgb}{0.150000,0.150000,0.150000}%
\pgfsetfillcolor{currentfill}%
\pgfsetlinewidth{0.803000pt}%
\definecolor{currentstroke}{rgb}{0.150000,0.150000,0.150000}%
\pgfsetstrokecolor{currentstroke}%
\pgfsetdash{}{0pt}%
\pgfsys@defobject{currentmarker}{\pgfqpoint{0.000000in}{0.000000in}}{\pgfqpoint{0.000000in}{0.000000in}}{%
\pgfpathmoveto{\pgfqpoint{0.000000in}{0.000000in}}%
\pgfpathlineto{\pgfqpoint{0.000000in}{0.000000in}}%
\pgfusepath{stroke,fill}%
}%
\begin{pgfscope}%
\pgfsys@transformshift{3.134424in}{0.668457in}%
\pgfsys@useobject{currentmarker}{}%
\end{pgfscope}%
\end{pgfscope}%
\begin{pgfscope}%
\definecolor{textcolor}{rgb}{0.150000,0.150000,0.150000}%
\pgfsetstrokecolor{textcolor}%
\pgfsetfillcolor{textcolor}%
\pgftext[x=0.479069in,y=0.668457in,right,]{\color{textcolor}\sffamily\fontsize{8.000000}{9.600000}\selectfont 2.5}%
\end{pgfscope}%
\begin{pgfscope}%
\pgfpathrectangle{\pgfqpoint{0.556847in}{0.516222in}}{\pgfqpoint{2.577576in}{0.913411in}} %
\pgfusepath{clip}%
\pgfsetroundcap%
\pgfsetroundjoin%
\pgfsetlinewidth{0.803000pt}%
\definecolor{currentstroke}{rgb}{1.000000,1.000000,1.000000}%
\pgfsetstrokecolor{currentstroke}%
\pgfsetdash{}{0pt}%
\pgfpathmoveto{\pgfqpoint{0.556847in}{0.820693in}}%
\pgfpathlineto{\pgfqpoint{3.134424in}{0.820693in}}%
\pgfusepath{stroke}%
\end{pgfscope}%
\begin{pgfscope}%
\pgfsetbuttcap%
\pgfsetroundjoin%
\definecolor{currentfill}{rgb}{0.150000,0.150000,0.150000}%
\pgfsetfillcolor{currentfill}%
\pgfsetlinewidth{0.803000pt}%
\definecolor{currentstroke}{rgb}{0.150000,0.150000,0.150000}%
\pgfsetstrokecolor{currentstroke}%
\pgfsetdash{}{0pt}%
\pgfsys@defobject{currentmarker}{\pgfqpoint{0.000000in}{0.000000in}}{\pgfqpoint{0.000000in}{0.000000in}}{%
\pgfpathmoveto{\pgfqpoint{0.000000in}{0.000000in}}%
\pgfpathlineto{\pgfqpoint{0.000000in}{0.000000in}}%
\pgfusepath{stroke,fill}%
}%
\begin{pgfscope}%
\pgfsys@transformshift{0.556847in}{0.820693in}%
\pgfsys@useobject{currentmarker}{}%
\end{pgfscope}%
\end{pgfscope}%
\begin{pgfscope}%
\pgfsetbuttcap%
\pgfsetroundjoin%
\definecolor{currentfill}{rgb}{0.150000,0.150000,0.150000}%
\pgfsetfillcolor{currentfill}%
\pgfsetlinewidth{0.803000pt}%
\definecolor{currentstroke}{rgb}{0.150000,0.150000,0.150000}%
\pgfsetstrokecolor{currentstroke}%
\pgfsetdash{}{0pt}%
\pgfsys@defobject{currentmarker}{\pgfqpoint{0.000000in}{0.000000in}}{\pgfqpoint{0.000000in}{0.000000in}}{%
\pgfpathmoveto{\pgfqpoint{0.000000in}{0.000000in}}%
\pgfpathlineto{\pgfqpoint{0.000000in}{0.000000in}}%
\pgfusepath{stroke,fill}%
}%
\begin{pgfscope}%
\pgfsys@transformshift{3.134424in}{0.820693in}%
\pgfsys@useobject{currentmarker}{}%
\end{pgfscope}%
\end{pgfscope}%
\begin{pgfscope}%
\definecolor{textcolor}{rgb}{0.150000,0.150000,0.150000}%
\pgfsetstrokecolor{textcolor}%
\pgfsetfillcolor{textcolor}%
\pgftext[x=0.479069in,y=0.820693in,right,]{\color{textcolor}\sffamily\fontsize{8.000000}{9.600000}\selectfont 3.0}%
\end{pgfscope}%
\begin{pgfscope}%
\pgfpathrectangle{\pgfqpoint{0.556847in}{0.516222in}}{\pgfqpoint{2.577576in}{0.913411in}} %
\pgfusepath{clip}%
\pgfsetroundcap%
\pgfsetroundjoin%
\pgfsetlinewidth{0.803000pt}%
\definecolor{currentstroke}{rgb}{1.000000,1.000000,1.000000}%
\pgfsetstrokecolor{currentstroke}%
\pgfsetdash{}{0pt}%
\pgfpathmoveto{\pgfqpoint{0.556847in}{0.972928in}}%
\pgfpathlineto{\pgfqpoint{3.134424in}{0.972928in}}%
\pgfusepath{stroke}%
\end{pgfscope}%
\begin{pgfscope}%
\pgfsetbuttcap%
\pgfsetroundjoin%
\definecolor{currentfill}{rgb}{0.150000,0.150000,0.150000}%
\pgfsetfillcolor{currentfill}%
\pgfsetlinewidth{0.803000pt}%
\definecolor{currentstroke}{rgb}{0.150000,0.150000,0.150000}%
\pgfsetstrokecolor{currentstroke}%
\pgfsetdash{}{0pt}%
\pgfsys@defobject{currentmarker}{\pgfqpoint{0.000000in}{0.000000in}}{\pgfqpoint{0.000000in}{0.000000in}}{%
\pgfpathmoveto{\pgfqpoint{0.000000in}{0.000000in}}%
\pgfpathlineto{\pgfqpoint{0.000000in}{0.000000in}}%
\pgfusepath{stroke,fill}%
}%
\begin{pgfscope}%
\pgfsys@transformshift{0.556847in}{0.972928in}%
\pgfsys@useobject{currentmarker}{}%
\end{pgfscope}%
\end{pgfscope}%
\begin{pgfscope}%
\pgfsetbuttcap%
\pgfsetroundjoin%
\definecolor{currentfill}{rgb}{0.150000,0.150000,0.150000}%
\pgfsetfillcolor{currentfill}%
\pgfsetlinewidth{0.803000pt}%
\definecolor{currentstroke}{rgb}{0.150000,0.150000,0.150000}%
\pgfsetstrokecolor{currentstroke}%
\pgfsetdash{}{0pt}%
\pgfsys@defobject{currentmarker}{\pgfqpoint{0.000000in}{0.000000in}}{\pgfqpoint{0.000000in}{0.000000in}}{%
\pgfpathmoveto{\pgfqpoint{0.000000in}{0.000000in}}%
\pgfpathlineto{\pgfqpoint{0.000000in}{0.000000in}}%
\pgfusepath{stroke,fill}%
}%
\begin{pgfscope}%
\pgfsys@transformshift{3.134424in}{0.972928in}%
\pgfsys@useobject{currentmarker}{}%
\end{pgfscope}%
\end{pgfscope}%
\begin{pgfscope}%
\definecolor{textcolor}{rgb}{0.150000,0.150000,0.150000}%
\pgfsetstrokecolor{textcolor}%
\pgfsetfillcolor{textcolor}%
\pgftext[x=0.479069in,y=0.972928in,right,]{\color{textcolor}\sffamily\fontsize{8.000000}{9.600000}\selectfont 3.5}%
\end{pgfscope}%
\begin{pgfscope}%
\pgfpathrectangle{\pgfqpoint{0.556847in}{0.516222in}}{\pgfqpoint{2.577576in}{0.913411in}} %
\pgfusepath{clip}%
\pgfsetroundcap%
\pgfsetroundjoin%
\pgfsetlinewidth{0.803000pt}%
\definecolor{currentstroke}{rgb}{1.000000,1.000000,1.000000}%
\pgfsetstrokecolor{currentstroke}%
\pgfsetdash{}{0pt}%
\pgfpathmoveto{\pgfqpoint{0.556847in}{1.125163in}}%
\pgfpathlineto{\pgfqpoint{3.134424in}{1.125163in}}%
\pgfusepath{stroke}%
\end{pgfscope}%
\begin{pgfscope}%
\pgfsetbuttcap%
\pgfsetroundjoin%
\definecolor{currentfill}{rgb}{0.150000,0.150000,0.150000}%
\pgfsetfillcolor{currentfill}%
\pgfsetlinewidth{0.803000pt}%
\definecolor{currentstroke}{rgb}{0.150000,0.150000,0.150000}%
\pgfsetstrokecolor{currentstroke}%
\pgfsetdash{}{0pt}%
\pgfsys@defobject{currentmarker}{\pgfqpoint{0.000000in}{0.000000in}}{\pgfqpoint{0.000000in}{0.000000in}}{%
\pgfpathmoveto{\pgfqpoint{0.000000in}{0.000000in}}%
\pgfpathlineto{\pgfqpoint{0.000000in}{0.000000in}}%
\pgfusepath{stroke,fill}%
}%
\begin{pgfscope}%
\pgfsys@transformshift{0.556847in}{1.125163in}%
\pgfsys@useobject{currentmarker}{}%
\end{pgfscope}%
\end{pgfscope}%
\begin{pgfscope}%
\pgfsetbuttcap%
\pgfsetroundjoin%
\definecolor{currentfill}{rgb}{0.150000,0.150000,0.150000}%
\pgfsetfillcolor{currentfill}%
\pgfsetlinewidth{0.803000pt}%
\definecolor{currentstroke}{rgb}{0.150000,0.150000,0.150000}%
\pgfsetstrokecolor{currentstroke}%
\pgfsetdash{}{0pt}%
\pgfsys@defobject{currentmarker}{\pgfqpoint{0.000000in}{0.000000in}}{\pgfqpoint{0.000000in}{0.000000in}}{%
\pgfpathmoveto{\pgfqpoint{0.000000in}{0.000000in}}%
\pgfpathlineto{\pgfqpoint{0.000000in}{0.000000in}}%
\pgfusepath{stroke,fill}%
}%
\begin{pgfscope}%
\pgfsys@transformshift{3.134424in}{1.125163in}%
\pgfsys@useobject{currentmarker}{}%
\end{pgfscope}%
\end{pgfscope}%
\begin{pgfscope}%
\definecolor{textcolor}{rgb}{0.150000,0.150000,0.150000}%
\pgfsetstrokecolor{textcolor}%
\pgfsetfillcolor{textcolor}%
\pgftext[x=0.479069in,y=1.125163in,right,]{\color{textcolor}\sffamily\fontsize{8.000000}{9.600000}\selectfont 4.0}%
\end{pgfscope}%
\begin{pgfscope}%
\pgfpathrectangle{\pgfqpoint{0.556847in}{0.516222in}}{\pgfqpoint{2.577576in}{0.913411in}} %
\pgfusepath{clip}%
\pgfsetroundcap%
\pgfsetroundjoin%
\pgfsetlinewidth{0.803000pt}%
\definecolor{currentstroke}{rgb}{1.000000,1.000000,1.000000}%
\pgfsetstrokecolor{currentstroke}%
\pgfsetdash{}{0pt}%
\pgfpathmoveto{\pgfqpoint{0.556847in}{1.277398in}}%
\pgfpathlineto{\pgfqpoint{3.134424in}{1.277398in}}%
\pgfusepath{stroke}%
\end{pgfscope}%
\begin{pgfscope}%
\pgfsetbuttcap%
\pgfsetroundjoin%
\definecolor{currentfill}{rgb}{0.150000,0.150000,0.150000}%
\pgfsetfillcolor{currentfill}%
\pgfsetlinewidth{0.803000pt}%
\definecolor{currentstroke}{rgb}{0.150000,0.150000,0.150000}%
\pgfsetstrokecolor{currentstroke}%
\pgfsetdash{}{0pt}%
\pgfsys@defobject{currentmarker}{\pgfqpoint{0.000000in}{0.000000in}}{\pgfqpoint{0.000000in}{0.000000in}}{%
\pgfpathmoveto{\pgfqpoint{0.000000in}{0.000000in}}%
\pgfpathlineto{\pgfqpoint{0.000000in}{0.000000in}}%
\pgfusepath{stroke,fill}%
}%
\begin{pgfscope}%
\pgfsys@transformshift{0.556847in}{1.277398in}%
\pgfsys@useobject{currentmarker}{}%
\end{pgfscope}%
\end{pgfscope}%
\begin{pgfscope}%
\pgfsetbuttcap%
\pgfsetroundjoin%
\definecolor{currentfill}{rgb}{0.150000,0.150000,0.150000}%
\pgfsetfillcolor{currentfill}%
\pgfsetlinewidth{0.803000pt}%
\definecolor{currentstroke}{rgb}{0.150000,0.150000,0.150000}%
\pgfsetstrokecolor{currentstroke}%
\pgfsetdash{}{0pt}%
\pgfsys@defobject{currentmarker}{\pgfqpoint{0.000000in}{0.000000in}}{\pgfqpoint{0.000000in}{0.000000in}}{%
\pgfpathmoveto{\pgfqpoint{0.000000in}{0.000000in}}%
\pgfpathlineto{\pgfqpoint{0.000000in}{0.000000in}}%
\pgfusepath{stroke,fill}%
}%
\begin{pgfscope}%
\pgfsys@transformshift{3.134424in}{1.277398in}%
\pgfsys@useobject{currentmarker}{}%
\end{pgfscope}%
\end{pgfscope}%
\begin{pgfscope}%
\definecolor{textcolor}{rgb}{0.150000,0.150000,0.150000}%
\pgfsetstrokecolor{textcolor}%
\pgfsetfillcolor{textcolor}%
\pgftext[x=0.479069in,y=1.277398in,right,]{\color{textcolor}\sffamily\fontsize{8.000000}{9.600000}\selectfont 4.5}%
\end{pgfscope}%
\begin{pgfscope}%
\pgfpathrectangle{\pgfqpoint{0.556847in}{0.516222in}}{\pgfqpoint{2.577576in}{0.913411in}} %
\pgfusepath{clip}%
\pgfsetroundcap%
\pgfsetroundjoin%
\pgfsetlinewidth{0.803000pt}%
\definecolor{currentstroke}{rgb}{1.000000,1.000000,1.000000}%
\pgfsetstrokecolor{currentstroke}%
\pgfsetdash{}{0pt}%
\pgfpathmoveto{\pgfqpoint{0.556847in}{1.429633in}}%
\pgfpathlineto{\pgfqpoint{3.134424in}{1.429633in}}%
\pgfusepath{stroke}%
\end{pgfscope}%
\begin{pgfscope}%
\pgfsetbuttcap%
\pgfsetroundjoin%
\definecolor{currentfill}{rgb}{0.150000,0.150000,0.150000}%
\pgfsetfillcolor{currentfill}%
\pgfsetlinewidth{0.803000pt}%
\definecolor{currentstroke}{rgb}{0.150000,0.150000,0.150000}%
\pgfsetstrokecolor{currentstroke}%
\pgfsetdash{}{0pt}%
\pgfsys@defobject{currentmarker}{\pgfqpoint{0.000000in}{0.000000in}}{\pgfqpoint{0.000000in}{0.000000in}}{%
\pgfpathmoveto{\pgfqpoint{0.000000in}{0.000000in}}%
\pgfpathlineto{\pgfqpoint{0.000000in}{0.000000in}}%
\pgfusepath{stroke,fill}%
}%
\begin{pgfscope}%
\pgfsys@transformshift{0.556847in}{1.429633in}%
\pgfsys@useobject{currentmarker}{}%
\end{pgfscope}%
\end{pgfscope}%
\begin{pgfscope}%
\pgfsetbuttcap%
\pgfsetroundjoin%
\definecolor{currentfill}{rgb}{0.150000,0.150000,0.150000}%
\pgfsetfillcolor{currentfill}%
\pgfsetlinewidth{0.803000pt}%
\definecolor{currentstroke}{rgb}{0.150000,0.150000,0.150000}%
\pgfsetstrokecolor{currentstroke}%
\pgfsetdash{}{0pt}%
\pgfsys@defobject{currentmarker}{\pgfqpoint{0.000000in}{0.000000in}}{\pgfqpoint{0.000000in}{0.000000in}}{%
\pgfpathmoveto{\pgfqpoint{0.000000in}{0.000000in}}%
\pgfpathlineto{\pgfqpoint{0.000000in}{0.000000in}}%
\pgfusepath{stroke,fill}%
}%
\begin{pgfscope}%
\pgfsys@transformshift{3.134424in}{1.429633in}%
\pgfsys@useobject{currentmarker}{}%
\end{pgfscope}%
\end{pgfscope}%
\begin{pgfscope}%
\definecolor{textcolor}{rgb}{0.150000,0.150000,0.150000}%
\pgfsetstrokecolor{textcolor}%
\pgfsetfillcolor{textcolor}%
\pgftext[x=0.479069in,y=1.429633in,right,]{\color{textcolor}\sffamily\fontsize{8.000000}{9.600000}\selectfont 5.0}%
\end{pgfscope}%
\begin{pgfscope}%
\definecolor{textcolor}{rgb}{0.150000,0.150000,0.150000}%
\pgfsetstrokecolor{textcolor}%
\pgfsetfillcolor{textcolor}%
\pgftext[x=0.251677in,y=0.972928in,,bottom,rotate=90.000000]{\color{textcolor}\sffamily\fontsize{8.800000}{10.560000}\selectfont Flying time}%
\end{pgfscope}%
\begin{pgfscope}%
\pgfpathrectangle{\pgfqpoint{0.556847in}{0.516222in}}{\pgfqpoint{2.577576in}{0.913411in}} %
\pgfusepath{clip}%
\pgfsetbuttcap%
\pgfsetmiterjoin%
\definecolor{currentfill}{rgb}{0.447059,0.623529,0.811765}%
\pgfsetfillcolor{currentfill}%
\pgfsetfillopacity{0.300000}%
\pgfsetlinewidth{0.240900pt}%
\definecolor{currentstroke}{rgb}{0.447059,0.623529,0.811765}%
\pgfsetstrokecolor{currentstroke}%
\pgfsetstrokeopacity{0.300000}%
\pgfsetdash{}{0pt}%
\pgfpathmoveto{\pgfqpoint{0.651611in}{0.893480in}}%
\pgfpathlineto{\pgfqpoint{0.675733in}{0.896086in}}%
\pgfpathlineto{\pgfqpoint{0.699854in}{0.898767in}}%
\pgfpathlineto{\pgfqpoint{0.723976in}{0.901526in}}%
\pgfpathlineto{\pgfqpoint{0.748098in}{0.904369in}}%
\pgfpathlineto{\pgfqpoint{0.772220in}{0.907301in}}%
\pgfpathlineto{\pgfqpoint{0.796341in}{0.910327in}}%
\pgfpathlineto{\pgfqpoint{0.820463in}{0.913452in}}%
\pgfpathlineto{\pgfqpoint{0.844585in}{0.916681in}}%
\pgfpathlineto{\pgfqpoint{0.868706in}{0.920019in}}%
\pgfpathlineto{\pgfqpoint{0.892828in}{0.923469in}}%
\pgfpathlineto{\pgfqpoint{0.916950in}{0.927034in}}%
\pgfpathlineto{\pgfqpoint{0.941072in}{0.930717in}}%
\pgfpathlineto{\pgfqpoint{0.965193in}{0.934517in}}%
\pgfpathlineto{\pgfqpoint{0.989315in}{0.938434in}}%
\pgfpathlineto{\pgfqpoint{1.013437in}{0.942464in}}%
\pgfpathlineto{\pgfqpoint{1.037558in}{0.946602in}}%
\pgfpathlineto{\pgfqpoint{1.061680in}{0.950842in}}%
\pgfpathlineto{\pgfqpoint{1.085802in}{0.955175in}}%
\pgfpathlineto{\pgfqpoint{1.109923in}{0.959591in}}%
\pgfpathlineto{\pgfqpoint{1.134045in}{0.964078in}}%
\pgfpathlineto{\pgfqpoint{1.158167in}{0.968625in}}%
\pgfpathlineto{\pgfqpoint{1.182289in}{0.973217in}}%
\pgfpathlineto{\pgfqpoint{1.206410in}{0.977842in}}%
\pgfpathlineto{\pgfqpoint{1.230532in}{0.982486in}}%
\pgfpathlineto{\pgfqpoint{1.254654in}{0.987138in}}%
\pgfpathlineto{\pgfqpoint{1.278775in}{0.991784in}}%
\pgfpathlineto{\pgfqpoint{1.302897in}{0.996413in}}%
\pgfpathlineto{\pgfqpoint{1.327019in}{1.001015in}}%
\pgfpathlineto{\pgfqpoint{1.351140in}{1.005579in}}%
\pgfpathlineto{\pgfqpoint{1.375262in}{1.010096in}}%
\pgfpathlineto{\pgfqpoint{1.399384in}{1.014558in}}%
\pgfpathlineto{\pgfqpoint{1.423506in}{1.018958in}}%
\pgfpathlineto{\pgfqpoint{1.447627in}{1.023289in}}%
\pgfpathlineto{\pgfqpoint{1.471749in}{1.027546in}}%
\pgfpathlineto{\pgfqpoint{1.495871in}{1.031721in}}%
\pgfpathlineto{\pgfqpoint{1.519992in}{1.035812in}}%
\pgfpathlineto{\pgfqpoint{1.544114in}{1.039814in}}%
\pgfpathlineto{\pgfqpoint{1.568236in}{1.043723in}}%
\pgfpathlineto{\pgfqpoint{1.592358in}{1.047536in}}%
\pgfpathlineto{\pgfqpoint{1.616479in}{1.051251in}}%
\pgfpathlineto{\pgfqpoint{1.640601in}{1.054865in}}%
\pgfpathlineto{\pgfqpoint{1.664723in}{1.058376in}}%
\pgfpathlineto{\pgfqpoint{1.688844in}{1.061783in}}%
\pgfpathlineto{\pgfqpoint{1.712966in}{1.065083in}}%
\pgfpathlineto{\pgfqpoint{1.737088in}{1.068277in}}%
\pgfpathlineto{\pgfqpoint{1.761209in}{1.071363in}}%
\pgfpathlineto{\pgfqpoint{1.785331in}{1.074342in}}%
\pgfpathlineto{\pgfqpoint{1.809453in}{1.077211in}}%
\pgfpathlineto{\pgfqpoint{1.833575in}{1.079972in}}%
\pgfpathlineto{\pgfqpoint{1.857696in}{1.082626in}}%
\pgfpathlineto{\pgfqpoint{1.881818in}{1.085171in}}%
\pgfpathlineto{\pgfqpoint{1.905940in}{1.087610in}}%
\pgfpathlineto{\pgfqpoint{1.930061in}{1.089942in}}%
\pgfpathlineto{\pgfqpoint{1.954183in}{1.092170in}}%
\pgfpathlineto{\pgfqpoint{1.978305in}{1.094296in}}%
\pgfpathlineto{\pgfqpoint{2.002426in}{1.096320in}}%
\pgfpathlineto{\pgfqpoint{2.026548in}{1.098245in}}%
\pgfpathlineto{\pgfqpoint{2.050670in}{1.100074in}}%
\pgfpathlineto{\pgfqpoint{2.074792in}{1.101809in}}%
\pgfpathlineto{\pgfqpoint{2.098913in}{1.103455in}}%
\pgfpathlineto{\pgfqpoint{2.123035in}{1.105015in}}%
\pgfpathlineto{\pgfqpoint{2.147157in}{1.106493in}}%
\pgfpathlineto{\pgfqpoint{2.171278in}{1.107894in}}%
\pgfpathlineto{\pgfqpoint{2.195400in}{1.109224in}}%
\pgfpathlineto{\pgfqpoint{2.219522in}{1.110489in}}%
\pgfpathlineto{\pgfqpoint{2.243644in}{1.111695in}}%
\pgfpathlineto{\pgfqpoint{2.267765in}{1.112850in}}%
\pgfpathlineto{\pgfqpoint{2.291887in}{1.113962in}}%
\pgfpathlineto{\pgfqpoint{2.316009in}{1.115040in}}%
\pgfpathlineto{\pgfqpoint{2.340130in}{1.116094in}}%
\pgfpathlineto{\pgfqpoint{2.364252in}{1.117134in}}%
\pgfpathlineto{\pgfqpoint{2.388374in}{1.118171in}}%
\pgfpathlineto{\pgfqpoint{2.412495in}{1.119216in}}%
\pgfpathlineto{\pgfqpoint{2.436617in}{1.120281in}}%
\pgfpathlineto{\pgfqpoint{2.460739in}{1.121377in}}%
\pgfpathlineto{\pgfqpoint{2.484861in}{1.122518in}}%
\pgfpathlineto{\pgfqpoint{2.508982in}{1.123714in}}%
\pgfpathlineto{\pgfqpoint{2.533104in}{1.124976in}}%
\pgfpathlineto{\pgfqpoint{2.557226in}{1.126315in}}%
\pgfpathlineto{\pgfqpoint{2.581347in}{1.127738in}}%
\pgfpathlineto{\pgfqpoint{2.605469in}{1.129255in}}%
\pgfpathlineto{\pgfqpoint{2.629591in}{1.130871in}}%
\pgfpathlineto{\pgfqpoint{2.653713in}{1.132591in}}%
\pgfpathlineto{\pgfqpoint{2.677834in}{1.134417in}}%
\pgfpathlineto{\pgfqpoint{2.701956in}{1.136350in}}%
\pgfpathlineto{\pgfqpoint{2.726078in}{1.138392in}}%
\pgfpathlineto{\pgfqpoint{2.750199in}{1.140540in}}%
\pgfpathlineto{\pgfqpoint{2.774321in}{1.142793in}}%
\pgfpathlineto{\pgfqpoint{2.798443in}{1.145148in}}%
\pgfpathlineto{\pgfqpoint{2.822564in}{1.147600in}}%
\pgfpathlineto{\pgfqpoint{2.846686in}{1.150147in}}%
\pgfpathlineto{\pgfqpoint{2.870808in}{1.152783in}}%
\pgfpathlineto{\pgfqpoint{2.894930in}{1.155505in}}%
\pgfpathlineto{\pgfqpoint{2.919051in}{1.158307in}}%
\pgfpathlineto{\pgfqpoint{2.943173in}{1.161187in}}%
\pgfpathlineto{\pgfqpoint{2.967295in}{1.164140in}}%
\pgfpathlineto{\pgfqpoint{2.991416in}{1.167161in}}%
\pgfpathlineto{\pgfqpoint{3.015538in}{1.170247in}}%
\pgfpathlineto{\pgfqpoint{3.039660in}{1.173395in}}%
\pgfpathlineto{\pgfqpoint{3.039660in}{0.996072in}}%
\pgfpathlineto{\pgfqpoint{3.015538in}{0.999930in}}%
\pgfpathlineto{\pgfqpoint{2.991416in}{1.003602in}}%
\pgfpathlineto{\pgfqpoint{2.967295in}{1.007084in}}%
\pgfpathlineto{\pgfqpoint{2.943173in}{1.010372in}}%
\pgfpathlineto{\pgfqpoint{2.919051in}{1.013464in}}%
\pgfpathlineto{\pgfqpoint{2.894930in}{1.016355in}}%
\pgfpathlineto{\pgfqpoint{2.870808in}{1.019039in}}%
\pgfpathlineto{\pgfqpoint{2.846686in}{1.021515in}}%
\pgfpathlineto{\pgfqpoint{2.822564in}{1.023775in}}%
\pgfpathlineto{\pgfqpoint{2.798443in}{1.025818in}}%
\pgfpathlineto{\pgfqpoint{2.774321in}{1.027638in}}%
\pgfpathlineto{\pgfqpoint{2.750199in}{1.029232in}}%
\pgfpathlineto{\pgfqpoint{2.726078in}{1.030597in}}%
\pgfpathlineto{\pgfqpoint{2.701956in}{1.031731in}}%
\pgfpathlineto{\pgfqpoint{2.677834in}{1.032633in}}%
\pgfpathlineto{\pgfqpoint{2.653713in}{1.033302in}}%
\pgfpathlineto{\pgfqpoint{2.629591in}{1.033741in}}%
\pgfpathlineto{\pgfqpoint{2.605469in}{1.033952in}}%
\pgfpathlineto{\pgfqpoint{2.581347in}{1.033939in}}%
\pgfpathlineto{\pgfqpoint{2.557226in}{1.033709in}}%
\pgfpathlineto{\pgfqpoint{2.533104in}{1.033269in}}%
\pgfpathlineto{\pgfqpoint{2.508982in}{1.032628in}}%
\pgfpathlineto{\pgfqpoint{2.484861in}{1.031797in}}%
\pgfpathlineto{\pgfqpoint{2.460739in}{1.030786in}}%
\pgfpathlineto{\pgfqpoint{2.436617in}{1.029607in}}%
\pgfpathlineto{\pgfqpoint{2.412495in}{1.028271in}}%
\pgfpathlineto{\pgfqpoint{2.388374in}{1.026791in}}%
\pgfpathlineto{\pgfqpoint{2.364252in}{1.025178in}}%
\pgfpathlineto{\pgfqpoint{2.340130in}{1.023445in}}%
\pgfpathlineto{\pgfqpoint{2.316009in}{1.021600in}}%
\pgfpathlineto{\pgfqpoint{2.291887in}{1.019656in}}%
\pgfpathlineto{\pgfqpoint{2.267765in}{1.017622in}}%
\pgfpathlineto{\pgfqpoint{2.243644in}{1.015506in}}%
\pgfpathlineto{\pgfqpoint{2.219522in}{1.013316in}}%
\pgfpathlineto{\pgfqpoint{2.195400in}{1.011061in}}%
\pgfpathlineto{\pgfqpoint{2.171278in}{1.008746in}}%
\pgfpathlineto{\pgfqpoint{2.147157in}{1.006378in}}%
\pgfpathlineto{\pgfqpoint{2.123035in}{1.003963in}}%
\pgfpathlineto{\pgfqpoint{2.098913in}{1.001505in}}%
\pgfpathlineto{\pgfqpoint{2.074792in}{0.999009in}}%
\pgfpathlineto{\pgfqpoint{2.050670in}{0.996478in}}%
\pgfpathlineto{\pgfqpoint{2.026548in}{0.993916in}}%
\pgfpathlineto{\pgfqpoint{2.002426in}{0.991326in}}%
\pgfpathlineto{\pgfqpoint{1.978305in}{0.988710in}}%
\pgfpathlineto{\pgfqpoint{1.954183in}{0.986071in}}%
\pgfpathlineto{\pgfqpoint{1.930061in}{0.983411in}}%
\pgfpathlineto{\pgfqpoint{1.905940in}{0.980730in}}%
\pgfpathlineto{\pgfqpoint{1.881818in}{0.978032in}}%
\pgfpathlineto{\pgfqpoint{1.857696in}{0.975315in}}%
\pgfpathlineto{\pgfqpoint{1.833575in}{0.972582in}}%
\pgfpathlineto{\pgfqpoint{1.809453in}{0.969833in}}%
\pgfpathlineto{\pgfqpoint{1.785331in}{0.967068in}}%
\pgfpathlineto{\pgfqpoint{1.761209in}{0.964287in}}%
\pgfpathlineto{\pgfqpoint{1.737088in}{0.961489in}}%
\pgfpathlineto{\pgfqpoint{1.712966in}{0.958675in}}%
\pgfpathlineto{\pgfqpoint{1.688844in}{0.955843in}}%
\pgfpathlineto{\pgfqpoint{1.664723in}{0.952993in}}%
\pgfpathlineto{\pgfqpoint{1.640601in}{0.950122in}}%
\pgfpathlineto{\pgfqpoint{1.616479in}{0.947231in}}%
\pgfpathlineto{\pgfqpoint{1.592358in}{0.944315in}}%
\pgfpathlineto{\pgfqpoint{1.568236in}{0.941374in}}%
\pgfpathlineto{\pgfqpoint{1.544114in}{0.938404in}}%
\pgfpathlineto{\pgfqpoint{1.519992in}{0.935402in}}%
\pgfpathlineto{\pgfqpoint{1.495871in}{0.932365in}}%
\pgfpathlineto{\pgfqpoint{1.471749in}{0.929289in}}%
\pgfpathlineto{\pgfqpoint{1.447627in}{0.926169in}}%
\pgfpathlineto{\pgfqpoint{1.423506in}{0.922999in}}%
\pgfpathlineto{\pgfqpoint{1.399384in}{0.919773in}}%
\pgfpathlineto{\pgfqpoint{1.375262in}{0.916486in}}%
\pgfpathlineto{\pgfqpoint{1.351140in}{0.913129in}}%
\pgfpathlineto{\pgfqpoint{1.327019in}{0.909694in}}%
\pgfpathlineto{\pgfqpoint{1.302897in}{0.906173in}}%
\pgfpathlineto{\pgfqpoint{1.278775in}{0.902555in}}%
\pgfpathlineto{\pgfqpoint{1.254654in}{0.898829in}}%
\pgfpathlineto{\pgfqpoint{1.230532in}{0.894984in}}%
\pgfpathlineto{\pgfqpoint{1.206410in}{0.891008in}}%
\pgfpathlineto{\pgfqpoint{1.182289in}{0.886888in}}%
\pgfpathlineto{\pgfqpoint{1.158167in}{0.882611in}}%
\pgfpathlineto{\pgfqpoint{1.134045in}{0.878163in}}%
\pgfpathlineto{\pgfqpoint{1.109923in}{0.873532in}}%
\pgfpathlineto{\pgfqpoint{1.085802in}{0.868705in}}%
\pgfpathlineto{\pgfqpoint{1.061680in}{0.863671in}}%
\pgfpathlineto{\pgfqpoint{1.037558in}{0.858420in}}%
\pgfpathlineto{\pgfqpoint{1.013437in}{0.852943in}}%
\pgfpathlineto{\pgfqpoint{0.989315in}{0.847232in}}%
\pgfpathlineto{\pgfqpoint{0.965193in}{0.841284in}}%
\pgfpathlineto{\pgfqpoint{0.941072in}{0.835096in}}%
\pgfpathlineto{\pgfqpoint{0.916950in}{0.828665in}}%
\pgfpathlineto{\pgfqpoint{0.892828in}{0.821992in}}%
\pgfpathlineto{\pgfqpoint{0.868706in}{0.815080in}}%
\pgfpathlineto{\pgfqpoint{0.844585in}{0.807931in}}%
\pgfpathlineto{\pgfqpoint{0.820463in}{0.800549in}}%
\pgfpathlineto{\pgfqpoint{0.796341in}{0.792939in}}%
\pgfpathlineto{\pgfqpoint{0.772220in}{0.785105in}}%
\pgfpathlineto{\pgfqpoint{0.748098in}{0.777053in}}%
\pgfpathlineto{\pgfqpoint{0.723976in}{0.768787in}}%
\pgfpathlineto{\pgfqpoint{0.699854in}{0.760313in}}%
\pgfpathlineto{\pgfqpoint{0.675733in}{0.751636in}}%
\pgfpathlineto{\pgfqpoint{0.651611in}{0.742761in}}%
\pgfpathclose%
\pgfusepath{stroke,fill}%
\end{pgfscope}%
\begin{pgfscope}%
\pgfpathrectangle{\pgfqpoint{0.556847in}{0.516222in}}{\pgfqpoint{2.577576in}{0.913411in}} %
\pgfusepath{clip}%
\pgfsetroundcap%
\pgfsetroundjoin%
\pgfsetlinewidth{2.007500pt}%
\definecolor{currentstroke}{rgb}{0.125490,0.290196,0.529412}%
\pgfsetstrokecolor{currentstroke}%
\pgfsetdash{}{0pt}%
\pgfpathmoveto{\pgfqpoint{0.651611in}{0.818120in}}%
\pgfpathlineto{\pgfqpoint{0.675733in}{0.823861in}}%
\pgfpathlineto{\pgfqpoint{0.699854in}{0.829540in}}%
\pgfpathlineto{\pgfqpoint{0.723976in}{0.835157in}}%
\pgfpathlineto{\pgfqpoint{0.748098in}{0.840711in}}%
\pgfpathlineto{\pgfqpoint{0.772220in}{0.846203in}}%
\pgfpathlineto{\pgfqpoint{0.796341in}{0.851633in}}%
\pgfpathlineto{\pgfqpoint{0.820463in}{0.857001in}}%
\pgfpathlineto{\pgfqpoint{0.844585in}{0.862306in}}%
\pgfpathlineto{\pgfqpoint{0.868706in}{0.867550in}}%
\pgfpathlineto{\pgfqpoint{0.892828in}{0.872731in}}%
\pgfpathlineto{\pgfqpoint{0.916950in}{0.877850in}}%
\pgfpathlineto{\pgfqpoint{0.941072in}{0.882906in}}%
\pgfpathlineto{\pgfqpoint{0.965193in}{0.887901in}}%
\pgfpathlineto{\pgfqpoint{0.989315in}{0.892833in}}%
\pgfpathlineto{\pgfqpoint{1.013437in}{0.897703in}}%
\pgfpathlineto{\pgfqpoint{1.037558in}{0.902511in}}%
\pgfpathlineto{\pgfqpoint{1.061680in}{0.907257in}}%
\pgfpathlineto{\pgfqpoint{1.085802in}{0.911940in}}%
\pgfpathlineto{\pgfqpoint{1.109923in}{0.916562in}}%
\pgfpathlineto{\pgfqpoint{1.134045in}{0.921121in}}%
\pgfpathlineto{\pgfqpoint{1.158167in}{0.925618in}}%
\pgfpathlineto{\pgfqpoint{1.182289in}{0.930053in}}%
\pgfpathlineto{\pgfqpoint{1.206410in}{0.934425in}}%
\pgfpathlineto{\pgfqpoint{1.230532in}{0.938735in}}%
\pgfpathlineto{\pgfqpoint{1.254654in}{0.942983in}}%
\pgfpathlineto{\pgfqpoint{1.278775in}{0.947169in}}%
\pgfpathlineto{\pgfqpoint{1.302897in}{0.951293in}}%
\pgfpathlineto{\pgfqpoint{1.327019in}{0.955355in}}%
\pgfpathlineto{\pgfqpoint{1.351140in}{0.959354in}}%
\pgfpathlineto{\pgfqpoint{1.375262in}{0.963291in}}%
\pgfpathlineto{\pgfqpoint{1.399384in}{0.967166in}}%
\pgfpathlineto{\pgfqpoint{1.423506in}{0.970979in}}%
\pgfpathlineto{\pgfqpoint{1.447627in}{0.974729in}}%
\pgfpathlineto{\pgfqpoint{1.471749in}{0.978417in}}%
\pgfpathlineto{\pgfqpoint{1.495871in}{0.982043in}}%
\pgfpathlineto{\pgfqpoint{1.519992in}{0.985607in}}%
\pgfpathlineto{\pgfqpoint{1.544114in}{0.989109in}}%
\pgfpathlineto{\pgfqpoint{1.568236in}{0.992548in}}%
\pgfpathlineto{\pgfqpoint{1.592358in}{0.995926in}}%
\pgfpathlineto{\pgfqpoint{1.616479in}{0.999241in}}%
\pgfpathlineto{\pgfqpoint{1.640601in}{1.002494in}}%
\pgfpathlineto{\pgfqpoint{1.664723in}{1.005684in}}%
\pgfpathlineto{\pgfqpoint{1.688844in}{1.008813in}}%
\pgfpathlineto{\pgfqpoint{1.712966in}{1.011879in}}%
\pgfpathlineto{\pgfqpoint{1.737088in}{1.014883in}}%
\pgfpathlineto{\pgfqpoint{1.761209in}{1.017825in}}%
\pgfpathlineto{\pgfqpoint{1.785331in}{1.020705in}}%
\pgfpathlineto{\pgfqpoint{1.809453in}{1.023522in}}%
\pgfpathlineto{\pgfqpoint{1.833575in}{1.026277in}}%
\pgfpathlineto{\pgfqpoint{1.857696in}{1.028971in}}%
\pgfpathlineto{\pgfqpoint{1.881818in}{1.031601in}}%
\pgfpathlineto{\pgfqpoint{1.905940in}{1.034170in}}%
\pgfpathlineto{\pgfqpoint{1.930061in}{1.036677in}}%
\pgfpathlineto{\pgfqpoint{1.954183in}{1.039121in}}%
\pgfpathlineto{\pgfqpoint{1.978305in}{1.041503in}}%
\pgfpathlineto{\pgfqpoint{2.002426in}{1.043823in}}%
\pgfpathlineto{\pgfqpoint{2.026548in}{1.046080in}}%
\pgfpathlineto{\pgfqpoint{2.050670in}{1.048276in}}%
\pgfpathlineto{\pgfqpoint{2.074792in}{1.050409in}}%
\pgfpathlineto{\pgfqpoint{2.098913in}{1.052480in}}%
\pgfpathlineto{\pgfqpoint{2.123035in}{1.054489in}}%
\pgfpathlineto{\pgfqpoint{2.147157in}{1.056436in}}%
\pgfpathlineto{\pgfqpoint{2.171278in}{1.058320in}}%
\pgfpathlineto{\pgfqpoint{2.195400in}{1.060142in}}%
\pgfpathlineto{\pgfqpoint{2.219522in}{1.061902in}}%
\pgfpathlineto{\pgfqpoint{2.243644in}{1.063600in}}%
\pgfpathlineto{\pgfqpoint{2.267765in}{1.065236in}}%
\pgfpathlineto{\pgfqpoint{2.291887in}{1.066809in}}%
\pgfpathlineto{\pgfqpoint{2.316009in}{1.068320in}}%
\pgfpathlineto{\pgfqpoint{2.340130in}{1.069769in}}%
\pgfpathlineto{\pgfqpoint{2.364252in}{1.071156in}}%
\pgfpathlineto{\pgfqpoint{2.388374in}{1.072481in}}%
\pgfpathlineto{\pgfqpoint{2.412495in}{1.073743in}}%
\pgfpathlineto{\pgfqpoint{2.436617in}{1.074944in}}%
\pgfpathlineto{\pgfqpoint{2.460739in}{1.076082in}}%
\pgfpathlineto{\pgfqpoint{2.484861in}{1.077157in}}%
\pgfpathlineto{\pgfqpoint{2.508982in}{1.078171in}}%
\pgfpathlineto{\pgfqpoint{2.533104in}{1.079122in}}%
\pgfpathlineto{\pgfqpoint{2.557226in}{1.080012in}}%
\pgfpathlineto{\pgfqpoint{2.581347in}{1.080839in}}%
\pgfpathlineto{\pgfqpoint{2.605469in}{1.081604in}}%
\pgfpathlineto{\pgfqpoint{2.629591in}{1.082306in}}%
\pgfpathlineto{\pgfqpoint{2.653713in}{1.082947in}}%
\pgfpathlineto{\pgfqpoint{2.677834in}{1.083525in}}%
\pgfpathlineto{\pgfqpoint{2.701956in}{1.084041in}}%
\pgfpathlineto{\pgfqpoint{2.726078in}{1.084495in}}%
\pgfpathlineto{\pgfqpoint{2.750199in}{1.084886in}}%
\pgfpathlineto{\pgfqpoint{2.774321in}{1.085216in}}%
\pgfpathlineto{\pgfqpoint{2.798443in}{1.085483in}}%
\pgfpathlineto{\pgfqpoint{2.822564in}{1.085688in}}%
\pgfpathlineto{\pgfqpoint{2.846686in}{1.085831in}}%
\pgfpathlineto{\pgfqpoint{2.870808in}{1.085911in}}%
\pgfpathlineto{\pgfqpoint{2.894930in}{1.085930in}}%
\pgfpathlineto{\pgfqpoint{2.919051in}{1.085886in}}%
\pgfpathlineto{\pgfqpoint{2.943173in}{1.085780in}}%
\pgfpathlineto{\pgfqpoint{2.967295in}{1.085612in}}%
\pgfpathlineto{\pgfqpoint{2.991416in}{1.085381in}}%
\pgfpathlineto{\pgfqpoint{3.015538in}{1.085089in}}%
\pgfpathlineto{\pgfqpoint{3.039660in}{1.084734in}}%
\pgfusepath{stroke}%
\end{pgfscope}%
\begin{pgfscope}%
\pgfpathrectangle{\pgfqpoint{0.556847in}{0.516222in}}{\pgfqpoint{2.577576in}{0.913411in}} %
\pgfusepath{clip}%
\pgfsetroundcap%
\pgfsetroundjoin%
\pgfsetlinewidth{0.200750pt}%
\definecolor{currentstroke}{rgb}{0.125490,0.290196,0.529412}%
\pgfsetstrokecolor{currentstroke}%
\pgfsetdash{}{0pt}%
\pgfpathmoveto{\pgfqpoint{0.651611in}{0.893480in}}%
\pgfpathlineto{\pgfqpoint{0.675733in}{0.896086in}}%
\pgfpathlineto{\pgfqpoint{0.699854in}{0.898767in}}%
\pgfpathlineto{\pgfqpoint{0.723976in}{0.901526in}}%
\pgfpathlineto{\pgfqpoint{0.748098in}{0.904369in}}%
\pgfpathlineto{\pgfqpoint{0.772220in}{0.907301in}}%
\pgfpathlineto{\pgfqpoint{0.796341in}{0.910327in}}%
\pgfpathlineto{\pgfqpoint{0.820463in}{0.913452in}}%
\pgfpathlineto{\pgfqpoint{0.844585in}{0.916681in}}%
\pgfpathlineto{\pgfqpoint{0.868706in}{0.920019in}}%
\pgfpathlineto{\pgfqpoint{0.892828in}{0.923469in}}%
\pgfpathlineto{\pgfqpoint{0.916950in}{0.927034in}}%
\pgfpathlineto{\pgfqpoint{0.941072in}{0.930717in}}%
\pgfpathlineto{\pgfqpoint{0.965193in}{0.934517in}}%
\pgfpathlineto{\pgfqpoint{0.989315in}{0.938434in}}%
\pgfpathlineto{\pgfqpoint{1.013437in}{0.942464in}}%
\pgfpathlineto{\pgfqpoint{1.037558in}{0.946602in}}%
\pgfpathlineto{\pgfqpoint{1.061680in}{0.950842in}}%
\pgfpathlineto{\pgfqpoint{1.085802in}{0.955175in}}%
\pgfpathlineto{\pgfqpoint{1.109923in}{0.959591in}}%
\pgfpathlineto{\pgfqpoint{1.134045in}{0.964078in}}%
\pgfpathlineto{\pgfqpoint{1.158167in}{0.968625in}}%
\pgfpathlineto{\pgfqpoint{1.182289in}{0.973217in}}%
\pgfpathlineto{\pgfqpoint{1.206410in}{0.977842in}}%
\pgfpathlineto{\pgfqpoint{1.230532in}{0.982486in}}%
\pgfpathlineto{\pgfqpoint{1.254654in}{0.987138in}}%
\pgfpathlineto{\pgfqpoint{1.278775in}{0.991784in}}%
\pgfpathlineto{\pgfqpoint{1.302897in}{0.996413in}}%
\pgfpathlineto{\pgfqpoint{1.327019in}{1.001015in}}%
\pgfpathlineto{\pgfqpoint{1.351140in}{1.005579in}}%
\pgfpathlineto{\pgfqpoint{1.375262in}{1.010096in}}%
\pgfpathlineto{\pgfqpoint{1.399384in}{1.014558in}}%
\pgfpathlineto{\pgfqpoint{1.423506in}{1.018958in}}%
\pgfpathlineto{\pgfqpoint{1.447627in}{1.023289in}}%
\pgfpathlineto{\pgfqpoint{1.471749in}{1.027546in}}%
\pgfpathlineto{\pgfqpoint{1.495871in}{1.031721in}}%
\pgfpathlineto{\pgfqpoint{1.519992in}{1.035812in}}%
\pgfpathlineto{\pgfqpoint{1.544114in}{1.039814in}}%
\pgfpathlineto{\pgfqpoint{1.568236in}{1.043723in}}%
\pgfpathlineto{\pgfqpoint{1.592358in}{1.047536in}}%
\pgfpathlineto{\pgfqpoint{1.616479in}{1.051251in}}%
\pgfpathlineto{\pgfqpoint{1.640601in}{1.054865in}}%
\pgfpathlineto{\pgfqpoint{1.664723in}{1.058376in}}%
\pgfpathlineto{\pgfqpoint{1.688844in}{1.061783in}}%
\pgfpathlineto{\pgfqpoint{1.712966in}{1.065083in}}%
\pgfpathlineto{\pgfqpoint{1.737088in}{1.068277in}}%
\pgfpathlineto{\pgfqpoint{1.761209in}{1.071363in}}%
\pgfpathlineto{\pgfqpoint{1.785331in}{1.074342in}}%
\pgfpathlineto{\pgfqpoint{1.809453in}{1.077211in}}%
\pgfpathlineto{\pgfqpoint{1.833575in}{1.079972in}}%
\pgfpathlineto{\pgfqpoint{1.857696in}{1.082626in}}%
\pgfpathlineto{\pgfqpoint{1.881818in}{1.085171in}}%
\pgfpathlineto{\pgfqpoint{1.905940in}{1.087610in}}%
\pgfpathlineto{\pgfqpoint{1.930061in}{1.089942in}}%
\pgfpathlineto{\pgfqpoint{1.954183in}{1.092170in}}%
\pgfpathlineto{\pgfqpoint{1.978305in}{1.094296in}}%
\pgfpathlineto{\pgfqpoint{2.002426in}{1.096320in}}%
\pgfpathlineto{\pgfqpoint{2.026548in}{1.098245in}}%
\pgfpathlineto{\pgfqpoint{2.050670in}{1.100074in}}%
\pgfpathlineto{\pgfqpoint{2.074792in}{1.101809in}}%
\pgfpathlineto{\pgfqpoint{2.098913in}{1.103455in}}%
\pgfpathlineto{\pgfqpoint{2.123035in}{1.105015in}}%
\pgfpathlineto{\pgfqpoint{2.147157in}{1.106493in}}%
\pgfpathlineto{\pgfqpoint{2.171278in}{1.107894in}}%
\pgfpathlineto{\pgfqpoint{2.195400in}{1.109224in}}%
\pgfpathlineto{\pgfqpoint{2.219522in}{1.110489in}}%
\pgfpathlineto{\pgfqpoint{2.243644in}{1.111695in}}%
\pgfpathlineto{\pgfqpoint{2.267765in}{1.112850in}}%
\pgfpathlineto{\pgfqpoint{2.291887in}{1.113962in}}%
\pgfpathlineto{\pgfqpoint{2.316009in}{1.115040in}}%
\pgfpathlineto{\pgfqpoint{2.340130in}{1.116094in}}%
\pgfpathlineto{\pgfqpoint{2.364252in}{1.117134in}}%
\pgfpathlineto{\pgfqpoint{2.388374in}{1.118171in}}%
\pgfpathlineto{\pgfqpoint{2.412495in}{1.119216in}}%
\pgfpathlineto{\pgfqpoint{2.436617in}{1.120281in}}%
\pgfpathlineto{\pgfqpoint{2.460739in}{1.121377in}}%
\pgfpathlineto{\pgfqpoint{2.484861in}{1.122518in}}%
\pgfpathlineto{\pgfqpoint{2.508982in}{1.123714in}}%
\pgfpathlineto{\pgfqpoint{2.533104in}{1.124976in}}%
\pgfpathlineto{\pgfqpoint{2.557226in}{1.126315in}}%
\pgfpathlineto{\pgfqpoint{2.581347in}{1.127738in}}%
\pgfpathlineto{\pgfqpoint{2.605469in}{1.129255in}}%
\pgfpathlineto{\pgfqpoint{2.629591in}{1.130871in}}%
\pgfpathlineto{\pgfqpoint{2.653713in}{1.132591in}}%
\pgfpathlineto{\pgfqpoint{2.677834in}{1.134417in}}%
\pgfpathlineto{\pgfqpoint{2.701956in}{1.136350in}}%
\pgfpathlineto{\pgfqpoint{2.726078in}{1.138392in}}%
\pgfpathlineto{\pgfqpoint{2.750199in}{1.140540in}}%
\pgfpathlineto{\pgfqpoint{2.774321in}{1.142793in}}%
\pgfpathlineto{\pgfqpoint{2.798443in}{1.145148in}}%
\pgfpathlineto{\pgfqpoint{2.822564in}{1.147600in}}%
\pgfpathlineto{\pgfqpoint{2.846686in}{1.150147in}}%
\pgfpathlineto{\pgfqpoint{2.870808in}{1.152783in}}%
\pgfpathlineto{\pgfqpoint{2.894930in}{1.155505in}}%
\pgfpathlineto{\pgfqpoint{2.919051in}{1.158307in}}%
\pgfpathlineto{\pgfqpoint{2.943173in}{1.161187in}}%
\pgfpathlineto{\pgfqpoint{2.967295in}{1.164140in}}%
\pgfpathlineto{\pgfqpoint{2.991416in}{1.167161in}}%
\pgfpathlineto{\pgfqpoint{3.015538in}{1.170247in}}%
\pgfpathlineto{\pgfqpoint{3.039660in}{1.173395in}}%
\pgfusepath{stroke}%
\end{pgfscope}%
\begin{pgfscope}%
\pgfpathrectangle{\pgfqpoint{0.556847in}{0.516222in}}{\pgfqpoint{2.577576in}{0.913411in}} %
\pgfusepath{clip}%
\pgfsetroundcap%
\pgfsetroundjoin%
\pgfsetlinewidth{0.200750pt}%
\definecolor{currentstroke}{rgb}{0.125490,0.290196,0.529412}%
\pgfsetstrokecolor{currentstroke}%
\pgfsetdash{}{0pt}%
\pgfpathmoveto{\pgfqpoint{0.651611in}{0.742761in}}%
\pgfpathlineto{\pgfqpoint{0.675733in}{0.751636in}}%
\pgfpathlineto{\pgfqpoint{0.699854in}{0.760313in}}%
\pgfpathlineto{\pgfqpoint{0.723976in}{0.768787in}}%
\pgfpathlineto{\pgfqpoint{0.748098in}{0.777053in}}%
\pgfpathlineto{\pgfqpoint{0.772220in}{0.785105in}}%
\pgfpathlineto{\pgfqpoint{0.796341in}{0.792939in}}%
\pgfpathlineto{\pgfqpoint{0.820463in}{0.800549in}}%
\pgfpathlineto{\pgfqpoint{0.844585in}{0.807931in}}%
\pgfpathlineto{\pgfqpoint{0.868706in}{0.815080in}}%
\pgfpathlineto{\pgfqpoint{0.892828in}{0.821992in}}%
\pgfpathlineto{\pgfqpoint{0.916950in}{0.828665in}}%
\pgfpathlineto{\pgfqpoint{0.941072in}{0.835096in}}%
\pgfpathlineto{\pgfqpoint{0.965193in}{0.841284in}}%
\pgfpathlineto{\pgfqpoint{0.989315in}{0.847232in}}%
\pgfpathlineto{\pgfqpoint{1.013437in}{0.852943in}}%
\pgfpathlineto{\pgfqpoint{1.037558in}{0.858420in}}%
\pgfpathlineto{\pgfqpoint{1.061680in}{0.863671in}}%
\pgfpathlineto{\pgfqpoint{1.085802in}{0.868705in}}%
\pgfpathlineto{\pgfqpoint{1.109923in}{0.873532in}}%
\pgfpathlineto{\pgfqpoint{1.134045in}{0.878163in}}%
\pgfpathlineto{\pgfqpoint{1.158167in}{0.882611in}}%
\pgfpathlineto{\pgfqpoint{1.182289in}{0.886888in}}%
\pgfpathlineto{\pgfqpoint{1.206410in}{0.891008in}}%
\pgfpathlineto{\pgfqpoint{1.230532in}{0.894984in}}%
\pgfpathlineto{\pgfqpoint{1.254654in}{0.898829in}}%
\pgfpathlineto{\pgfqpoint{1.278775in}{0.902555in}}%
\pgfpathlineto{\pgfqpoint{1.302897in}{0.906173in}}%
\pgfpathlineto{\pgfqpoint{1.327019in}{0.909694in}}%
\pgfpathlineto{\pgfqpoint{1.351140in}{0.913129in}}%
\pgfpathlineto{\pgfqpoint{1.375262in}{0.916486in}}%
\pgfpathlineto{\pgfqpoint{1.399384in}{0.919773in}}%
\pgfpathlineto{\pgfqpoint{1.423506in}{0.922999in}}%
\pgfpathlineto{\pgfqpoint{1.447627in}{0.926169in}}%
\pgfpathlineto{\pgfqpoint{1.471749in}{0.929289in}}%
\pgfpathlineto{\pgfqpoint{1.495871in}{0.932365in}}%
\pgfpathlineto{\pgfqpoint{1.519992in}{0.935402in}}%
\pgfpathlineto{\pgfqpoint{1.544114in}{0.938404in}}%
\pgfpathlineto{\pgfqpoint{1.568236in}{0.941374in}}%
\pgfpathlineto{\pgfqpoint{1.592358in}{0.944315in}}%
\pgfpathlineto{\pgfqpoint{1.616479in}{0.947231in}}%
\pgfpathlineto{\pgfqpoint{1.640601in}{0.950122in}}%
\pgfpathlineto{\pgfqpoint{1.664723in}{0.952993in}}%
\pgfpathlineto{\pgfqpoint{1.688844in}{0.955843in}}%
\pgfpathlineto{\pgfqpoint{1.712966in}{0.958675in}}%
\pgfpathlineto{\pgfqpoint{1.737088in}{0.961489in}}%
\pgfpathlineto{\pgfqpoint{1.761209in}{0.964287in}}%
\pgfpathlineto{\pgfqpoint{1.785331in}{0.967068in}}%
\pgfpathlineto{\pgfqpoint{1.809453in}{0.969833in}}%
\pgfpathlineto{\pgfqpoint{1.833575in}{0.972582in}}%
\pgfpathlineto{\pgfqpoint{1.857696in}{0.975315in}}%
\pgfpathlineto{\pgfqpoint{1.881818in}{0.978032in}}%
\pgfpathlineto{\pgfqpoint{1.905940in}{0.980730in}}%
\pgfpathlineto{\pgfqpoint{1.930061in}{0.983411in}}%
\pgfpathlineto{\pgfqpoint{1.954183in}{0.986071in}}%
\pgfpathlineto{\pgfqpoint{1.978305in}{0.988710in}}%
\pgfpathlineto{\pgfqpoint{2.002426in}{0.991326in}}%
\pgfpathlineto{\pgfqpoint{2.026548in}{0.993916in}}%
\pgfpathlineto{\pgfqpoint{2.050670in}{0.996478in}}%
\pgfpathlineto{\pgfqpoint{2.074792in}{0.999009in}}%
\pgfpathlineto{\pgfqpoint{2.098913in}{1.001505in}}%
\pgfpathlineto{\pgfqpoint{2.123035in}{1.003963in}}%
\pgfpathlineto{\pgfqpoint{2.147157in}{1.006378in}}%
\pgfpathlineto{\pgfqpoint{2.171278in}{1.008746in}}%
\pgfpathlineto{\pgfqpoint{2.195400in}{1.011061in}}%
\pgfpathlineto{\pgfqpoint{2.219522in}{1.013316in}}%
\pgfpathlineto{\pgfqpoint{2.243644in}{1.015506in}}%
\pgfpathlineto{\pgfqpoint{2.267765in}{1.017622in}}%
\pgfpathlineto{\pgfqpoint{2.291887in}{1.019656in}}%
\pgfpathlineto{\pgfqpoint{2.316009in}{1.021600in}}%
\pgfpathlineto{\pgfqpoint{2.340130in}{1.023445in}}%
\pgfpathlineto{\pgfqpoint{2.364252in}{1.025178in}}%
\pgfpathlineto{\pgfqpoint{2.388374in}{1.026791in}}%
\pgfpathlineto{\pgfqpoint{2.412495in}{1.028271in}}%
\pgfpathlineto{\pgfqpoint{2.436617in}{1.029607in}}%
\pgfpathlineto{\pgfqpoint{2.460739in}{1.030786in}}%
\pgfpathlineto{\pgfqpoint{2.484861in}{1.031797in}}%
\pgfpathlineto{\pgfqpoint{2.508982in}{1.032628in}}%
\pgfpathlineto{\pgfqpoint{2.533104in}{1.033269in}}%
\pgfpathlineto{\pgfqpoint{2.557226in}{1.033709in}}%
\pgfpathlineto{\pgfqpoint{2.581347in}{1.033939in}}%
\pgfpathlineto{\pgfqpoint{2.605469in}{1.033952in}}%
\pgfpathlineto{\pgfqpoint{2.629591in}{1.033741in}}%
\pgfpathlineto{\pgfqpoint{2.653713in}{1.033302in}}%
\pgfpathlineto{\pgfqpoint{2.677834in}{1.032633in}}%
\pgfpathlineto{\pgfqpoint{2.701956in}{1.031731in}}%
\pgfpathlineto{\pgfqpoint{2.726078in}{1.030597in}}%
\pgfpathlineto{\pgfqpoint{2.750199in}{1.029232in}}%
\pgfpathlineto{\pgfqpoint{2.774321in}{1.027638in}}%
\pgfpathlineto{\pgfqpoint{2.798443in}{1.025818in}}%
\pgfpathlineto{\pgfqpoint{2.822564in}{1.023775in}}%
\pgfpathlineto{\pgfqpoint{2.846686in}{1.021515in}}%
\pgfpathlineto{\pgfqpoint{2.870808in}{1.019039in}}%
\pgfpathlineto{\pgfqpoint{2.894930in}{1.016355in}}%
\pgfpathlineto{\pgfqpoint{2.919051in}{1.013464in}}%
\pgfpathlineto{\pgfqpoint{2.943173in}{1.010372in}}%
\pgfpathlineto{\pgfqpoint{2.967295in}{1.007084in}}%
\pgfpathlineto{\pgfqpoint{2.991416in}{1.003602in}}%
\pgfpathlineto{\pgfqpoint{3.015538in}{0.999930in}}%
\pgfpathlineto{\pgfqpoint{3.039660in}{0.996072in}}%
\pgfusepath{stroke}%
\end{pgfscope}%
\begin{pgfscope}%
\pgfpathrectangle{\pgfqpoint{0.556847in}{0.516222in}}{\pgfqpoint{2.577576in}{0.913411in}} %
\pgfusepath{clip}%
\pgfsetbuttcap%
\pgfsetbeveljoin%
\definecolor{currentfill}{rgb}{0.298039,0.447059,0.690196}%
\pgfsetfillcolor{currentfill}%
\pgfsetlinewidth{0.000000pt}%
\definecolor{currentstroke}{rgb}{0.000000,0.000000,0.000000}%
\pgfsetstrokecolor{currentstroke}%
\pgfsetdash{}{0pt}%
\pgfsys@defobject{currentmarker}{\pgfqpoint{-0.036986in}{-0.031462in}}{\pgfqpoint{0.036986in}{0.038889in}}{%
\pgfpathmoveto{\pgfqpoint{0.000000in}{0.038889in}}%
\pgfpathlineto{\pgfqpoint{-0.008731in}{0.012017in}}%
\pgfpathlineto{\pgfqpoint{-0.036986in}{0.012017in}}%
\pgfpathlineto{\pgfqpoint{-0.014127in}{-0.004590in}}%
\pgfpathlineto{\pgfqpoint{-0.022858in}{-0.031462in}}%
\pgfpathlineto{\pgfqpoint{-0.000000in}{-0.014854in}}%
\pgfpathlineto{\pgfqpoint{0.022858in}{-0.031462in}}%
\pgfpathlineto{\pgfqpoint{0.014127in}{-0.004590in}}%
\pgfpathlineto{\pgfqpoint{0.036986in}{0.012017in}}%
\pgfpathlineto{\pgfqpoint{0.008731in}{0.012017in}}%
\pgfpathclose%
\pgfusepath{fill}%
}%
\begin{pgfscope}%
\pgfsys@transformshift{1.267576in}{1.079492in}%
\pgfsys@useobject{currentmarker}{}%
\end{pgfscope}%
\begin{pgfscope}%
\pgfsys@transformshift{3.039660in}{0.957704in}%
\pgfsys@useobject{currentmarker}{}%
\end{pgfscope}%
\begin{pgfscope}%
\pgfsys@transformshift{1.779301in}{1.064269in}%
\pgfsys@useobject{currentmarker}{}%
\end{pgfscope}%
\begin{pgfscope}%
\pgfsys@transformshift{2.068330in}{1.262175in}%
\pgfsys@useobject{currentmarker}{}%
\end{pgfscope}%
\begin{pgfscope}%
\pgfsys@transformshift{2.504244in}{0.942481in}%
\pgfsys@useobject{currentmarker}{}%
\end{pgfscope}%
\begin{pgfscope}%
\pgfsys@transformshift{1.575558in}{1.323069in}%
\pgfsys@useobject{currentmarker}{}%
\end{pgfscope}%
\begin{pgfscope}%
\pgfsys@transformshift{1.774563in}{0.942481in}%
\pgfsys@useobject{currentmarker}{}%
\end{pgfscope}%
\begin{pgfscope}%
\pgfsys@transformshift{2.366837in}{1.383963in}%
\pgfsys@useobject{currentmarker}{}%
\end{pgfscope}%
\begin{pgfscope}%
\pgfsys@transformshift{2.940158in}{0.683681in}%
\pgfsys@useobject{currentmarker}{}%
\end{pgfscope}%
\begin{pgfscope}%
\pgfsys@transformshift{3.025445in}{0.972928in}%
\pgfsys@useobject{currentmarker}{}%
\end{pgfscope}%
\begin{pgfscope}%
\pgfsys@transformshift{1.049619in}{0.927257in}%
\pgfsys@useobject{currentmarker}{}%
\end{pgfscope}%
\begin{pgfscope}%
\pgfsys@transformshift{2.646390in}{0.942481in}%
\pgfsys@useobject{currentmarker}{}%
\end{pgfscope}%
\begin{pgfscope}%
\pgfsys@transformshift{1.466580in}{0.988151in}%
\pgfsys@useobject{currentmarker}{}%
\end{pgfscope}%
\begin{pgfscope}%
\pgfsys@transformshift{1.286529in}{0.622787in}%
\pgfsys@useobject{currentmarker}{}%
\end{pgfscope}%
\begin{pgfscope}%
\pgfsys@transformshift{2.376313in}{1.155610in}%
\pgfsys@useobject{currentmarker}{}%
\end{pgfscope}%
\begin{pgfscope}%
\pgfsys@transformshift{2.508982in}{1.109939in}%
\pgfsys@useobject{currentmarker}{}%
\end{pgfscope}%
\begin{pgfscope}%
\pgfsys@transformshift{2.253120in}{0.866363in}%
\pgfsys@useobject{currentmarker}{}%
\end{pgfscope}%
\begin{pgfscope}%
\pgfsys@transformshift{0.651611in}{0.957704in}%
\pgfsys@useobject{currentmarker}{}%
\end{pgfscope}%
\begin{pgfscope}%
\pgfsys@transformshift{1.722442in}{0.744575in}%
\pgfsys@useobject{currentmarker}{}%
\end{pgfscope}%
\begin{pgfscope}%
\pgfsys@transformshift{2.670081in}{0.881587in}%
\pgfsys@useobject{currentmarker}{}%
\end{pgfscope}%
\begin{pgfscope}%
\pgfsys@transformshift{1.016452in}{0.957704in}%
\pgfsys@useobject{currentmarker}{}%
\end{pgfscope}%
\begin{pgfscope}%
\pgfsys@transformshift{0.789019in}{1.170834in}%
\pgfsys@useobject{currentmarker}{}%
\end{pgfscope}%
\begin{pgfscope}%
\pgfsys@transformshift{2.461600in}{1.003375in}%
\pgfsys@useobject{currentmarker}{}%
\end{pgfscope}%
\begin{pgfscope}%
\pgfsys@transformshift{0.732160in}{0.790246in}%
\pgfsys@useobject{currentmarker}{}%
\end{pgfscope}%
\begin{pgfscope}%
\pgfsys@transformshift{1.987781in}{1.049045in}%
\pgfsys@useobject{currentmarker}{}%
\end{pgfscope}%
\begin{pgfscope}%
\pgfsys@transformshift{1.044881in}{1.246951in}%
\pgfsys@useobject{currentmarker}{}%
\end{pgfscope}%
\begin{pgfscope}%
\pgfsys@transformshift{1.679799in}{0.805469in}%
\pgfsys@useobject{currentmarker}{}%
\end{pgfscope}%
\begin{pgfscope}%
\pgfsys@transformshift{1.243885in}{0.698904in}%
\pgfsys@useobject{currentmarker}{}%
\end{pgfscope}%
\begin{pgfscope}%
\pgfsys@transformshift{2.343146in}{0.759799in}%
\pgfsys@useobject{currentmarker}{}%
\end{pgfscope}%
\begin{pgfscope}%
\pgfsys@transformshift{1.566082in}{1.262175in}%
\pgfsys@useobject{currentmarker}{}%
\end{pgfscope}%
\end{pgfscope}%
\begin{pgfscope}%
\pgfsetrectcap%
\pgfsetmiterjoin%
\pgfsetlinewidth{0.000000pt}%
\definecolor{currentstroke}{rgb}{1.000000,1.000000,1.000000}%
\pgfsetstrokecolor{currentstroke}%
\pgfsetdash{}{0pt}%
\pgfpathmoveto{\pgfqpoint{3.134424in}{0.516222in}}%
\pgfpathlineto{\pgfqpoint{3.134424in}{1.429633in}}%
\pgfusepath{}%
\end{pgfscope}%
\begin{pgfscope}%
\pgfsetrectcap%
\pgfsetmiterjoin%
\pgfsetlinewidth{0.000000pt}%
\definecolor{currentstroke}{rgb}{1.000000,1.000000,1.000000}%
\pgfsetstrokecolor{currentstroke}%
\pgfsetdash{}{0pt}%
\pgfpathmoveto{\pgfqpoint{0.556847in}{0.516222in}}%
\pgfpathlineto{\pgfqpoint{3.134424in}{0.516222in}}%
\pgfusepath{}%
\end{pgfscope}%
\begin{pgfscope}%
\pgfsetrectcap%
\pgfsetmiterjoin%
\pgfsetlinewidth{0.000000pt}%
\definecolor{currentstroke}{rgb}{1.000000,1.000000,1.000000}%
\pgfsetstrokecolor{currentstroke}%
\pgfsetdash{}{0pt}%
\pgfpathmoveto{\pgfqpoint{0.556847in}{0.516222in}}%
\pgfpathlineto{\pgfqpoint{0.556847in}{1.429633in}}%
\pgfusepath{}%
\end{pgfscope}%
\begin{pgfscope}%
\pgfsetrectcap%
\pgfsetmiterjoin%
\pgfsetlinewidth{0.000000pt}%
\definecolor{currentstroke}{rgb}{1.000000,1.000000,1.000000}%
\pgfsetstrokecolor{currentstroke}%
\pgfsetdash{}{0pt}%
\pgfpathmoveto{\pgfqpoint{0.556847in}{1.429633in}}%
\pgfpathlineto{\pgfqpoint{3.134424in}{1.429633in}}%
\pgfusepath{}%
\end{pgfscope}%
\begin{pgfscope}%
\pgfsetbuttcap%
\pgfsetmiterjoin%
\definecolor{currentfill}{rgb}{0.917647,0.917647,0.949020}%
\pgfsetfillcolor{currentfill}%
\pgfsetlinewidth{0.000000pt}%
\definecolor{currentstroke}{rgb}{0.000000,0.000000,0.000000}%
\pgfsetstrokecolor{currentstroke}%
\pgfsetstrokeopacity{0.000000}%
\pgfsetdash{}{0pt}%
\pgfpathmoveto{\pgfqpoint{3.278424in}{0.516222in}}%
\pgfpathlineto{\pgfqpoint{5.856000in}{0.516222in}}%
\pgfpathlineto{\pgfqpoint{5.856000in}{1.429633in}}%
\pgfpathlineto{\pgfqpoint{3.278424in}{1.429633in}}%
\pgfpathclose%
\pgfusepath{fill}%
\end{pgfscope}%
\begin{pgfscope}%
\pgfpathrectangle{\pgfqpoint{3.278424in}{0.516222in}}{\pgfqpoint{2.577576in}{0.913411in}} %
\pgfusepath{clip}%
\pgfsetroundcap%
\pgfsetroundjoin%
\pgfsetlinewidth{0.803000pt}%
\definecolor{currentstroke}{rgb}{1.000000,1.000000,1.000000}%
\pgfsetstrokecolor{currentstroke}%
\pgfsetdash{}{0pt}%
\pgfpathmoveto{\pgfqpoint{3.639284in}{0.516222in}}%
\pgfpathlineto{\pgfqpoint{3.639284in}{1.429633in}}%
\pgfusepath{stroke}%
\end{pgfscope}%
\begin{pgfscope}%
\pgfsetbuttcap%
\pgfsetroundjoin%
\definecolor{currentfill}{rgb}{0.150000,0.150000,0.150000}%
\pgfsetfillcolor{currentfill}%
\pgfsetlinewidth{0.803000pt}%
\definecolor{currentstroke}{rgb}{0.150000,0.150000,0.150000}%
\pgfsetstrokecolor{currentstroke}%
\pgfsetdash{}{0pt}%
\pgfsys@defobject{currentmarker}{\pgfqpoint{0.000000in}{0.000000in}}{\pgfqpoint{0.000000in}{0.000000in}}{%
\pgfpathmoveto{\pgfqpoint{0.000000in}{0.000000in}}%
\pgfpathlineto{\pgfqpoint{0.000000in}{0.000000in}}%
\pgfusepath{stroke,fill}%
}%
\begin{pgfscope}%
\pgfsys@transformshift{3.639284in}{0.516222in}%
\pgfsys@useobject{currentmarker}{}%
\end{pgfscope}%
\end{pgfscope}%
\begin{pgfscope}%
\pgfsetbuttcap%
\pgfsetroundjoin%
\definecolor{currentfill}{rgb}{0.150000,0.150000,0.150000}%
\pgfsetfillcolor{currentfill}%
\pgfsetlinewidth{0.803000pt}%
\definecolor{currentstroke}{rgb}{0.150000,0.150000,0.150000}%
\pgfsetstrokecolor{currentstroke}%
\pgfsetdash{}{0pt}%
\pgfsys@defobject{currentmarker}{\pgfqpoint{0.000000in}{0.000000in}}{\pgfqpoint{0.000000in}{0.000000in}}{%
\pgfpathmoveto{\pgfqpoint{0.000000in}{0.000000in}}%
\pgfpathlineto{\pgfqpoint{0.000000in}{0.000000in}}%
\pgfusepath{stroke,fill}%
}%
\begin{pgfscope}%
\pgfsys@transformshift{3.639284in}{1.429633in}%
\pgfsys@useobject{currentmarker}{}%
\end{pgfscope}%
\end{pgfscope}%
\begin{pgfscope}%
\definecolor{textcolor}{rgb}{0.150000,0.150000,0.150000}%
\pgfsetstrokecolor{textcolor}%
\pgfsetfillcolor{textcolor}%
\pgftext[x=3.639284in,y=0.438444in,,top]{\color{textcolor}\sffamily\fontsize{8.000000}{9.600000}\selectfont 8}%
\end{pgfscope}%
\begin{pgfscope}%
\pgfpathrectangle{\pgfqpoint{3.278424in}{0.516222in}}{\pgfqpoint{2.577576in}{0.913411in}} %
\pgfusepath{clip}%
\pgfsetroundcap%
\pgfsetroundjoin%
\pgfsetlinewidth{0.803000pt}%
\definecolor{currentstroke}{rgb}{1.000000,1.000000,1.000000}%
\pgfsetstrokecolor{currentstroke}%
\pgfsetdash{}{0pt}%
\pgfpathmoveto{\pgfqpoint{4.154800in}{0.516222in}}%
\pgfpathlineto{\pgfqpoint{4.154800in}{1.429633in}}%
\pgfusepath{stroke}%
\end{pgfscope}%
\begin{pgfscope}%
\pgfsetbuttcap%
\pgfsetroundjoin%
\definecolor{currentfill}{rgb}{0.150000,0.150000,0.150000}%
\pgfsetfillcolor{currentfill}%
\pgfsetlinewidth{0.803000pt}%
\definecolor{currentstroke}{rgb}{0.150000,0.150000,0.150000}%
\pgfsetstrokecolor{currentstroke}%
\pgfsetdash{}{0pt}%
\pgfsys@defobject{currentmarker}{\pgfqpoint{0.000000in}{0.000000in}}{\pgfqpoint{0.000000in}{0.000000in}}{%
\pgfpathmoveto{\pgfqpoint{0.000000in}{0.000000in}}%
\pgfpathlineto{\pgfqpoint{0.000000in}{0.000000in}}%
\pgfusepath{stroke,fill}%
}%
\begin{pgfscope}%
\pgfsys@transformshift{4.154800in}{0.516222in}%
\pgfsys@useobject{currentmarker}{}%
\end{pgfscope}%
\end{pgfscope}%
\begin{pgfscope}%
\pgfsetbuttcap%
\pgfsetroundjoin%
\definecolor{currentfill}{rgb}{0.150000,0.150000,0.150000}%
\pgfsetfillcolor{currentfill}%
\pgfsetlinewidth{0.803000pt}%
\definecolor{currentstroke}{rgb}{0.150000,0.150000,0.150000}%
\pgfsetstrokecolor{currentstroke}%
\pgfsetdash{}{0pt}%
\pgfsys@defobject{currentmarker}{\pgfqpoint{0.000000in}{0.000000in}}{\pgfqpoint{0.000000in}{0.000000in}}{%
\pgfpathmoveto{\pgfqpoint{0.000000in}{0.000000in}}%
\pgfpathlineto{\pgfqpoint{0.000000in}{0.000000in}}%
\pgfusepath{stroke,fill}%
}%
\begin{pgfscope}%
\pgfsys@transformshift{4.154800in}{1.429633in}%
\pgfsys@useobject{currentmarker}{}%
\end{pgfscope}%
\end{pgfscope}%
\begin{pgfscope}%
\definecolor{textcolor}{rgb}{0.150000,0.150000,0.150000}%
\pgfsetstrokecolor{textcolor}%
\pgfsetfillcolor{textcolor}%
\pgftext[x=4.154800in,y=0.438444in,,top]{\color{textcolor}\sffamily\fontsize{8.000000}{9.600000}\selectfont 9}%
\end{pgfscope}%
\begin{pgfscope}%
\pgfpathrectangle{\pgfqpoint{3.278424in}{0.516222in}}{\pgfqpoint{2.577576in}{0.913411in}} %
\pgfusepath{clip}%
\pgfsetroundcap%
\pgfsetroundjoin%
\pgfsetlinewidth{0.803000pt}%
\definecolor{currentstroke}{rgb}{1.000000,1.000000,1.000000}%
\pgfsetstrokecolor{currentstroke}%
\pgfsetdash{}{0pt}%
\pgfpathmoveto{\pgfqpoint{4.670315in}{0.516222in}}%
\pgfpathlineto{\pgfqpoint{4.670315in}{1.429633in}}%
\pgfusepath{stroke}%
\end{pgfscope}%
\begin{pgfscope}%
\pgfsetbuttcap%
\pgfsetroundjoin%
\definecolor{currentfill}{rgb}{0.150000,0.150000,0.150000}%
\pgfsetfillcolor{currentfill}%
\pgfsetlinewidth{0.803000pt}%
\definecolor{currentstroke}{rgb}{0.150000,0.150000,0.150000}%
\pgfsetstrokecolor{currentstroke}%
\pgfsetdash{}{0pt}%
\pgfsys@defobject{currentmarker}{\pgfqpoint{0.000000in}{0.000000in}}{\pgfqpoint{0.000000in}{0.000000in}}{%
\pgfpathmoveto{\pgfqpoint{0.000000in}{0.000000in}}%
\pgfpathlineto{\pgfqpoint{0.000000in}{0.000000in}}%
\pgfusepath{stroke,fill}%
}%
\begin{pgfscope}%
\pgfsys@transformshift{4.670315in}{0.516222in}%
\pgfsys@useobject{currentmarker}{}%
\end{pgfscope}%
\end{pgfscope}%
\begin{pgfscope}%
\pgfsetbuttcap%
\pgfsetroundjoin%
\definecolor{currentfill}{rgb}{0.150000,0.150000,0.150000}%
\pgfsetfillcolor{currentfill}%
\pgfsetlinewidth{0.803000pt}%
\definecolor{currentstroke}{rgb}{0.150000,0.150000,0.150000}%
\pgfsetstrokecolor{currentstroke}%
\pgfsetdash{}{0pt}%
\pgfsys@defobject{currentmarker}{\pgfqpoint{0.000000in}{0.000000in}}{\pgfqpoint{0.000000in}{0.000000in}}{%
\pgfpathmoveto{\pgfqpoint{0.000000in}{0.000000in}}%
\pgfpathlineto{\pgfqpoint{0.000000in}{0.000000in}}%
\pgfusepath{stroke,fill}%
}%
\begin{pgfscope}%
\pgfsys@transformshift{4.670315in}{1.429633in}%
\pgfsys@useobject{currentmarker}{}%
\end{pgfscope}%
\end{pgfscope}%
\begin{pgfscope}%
\definecolor{textcolor}{rgb}{0.150000,0.150000,0.150000}%
\pgfsetstrokecolor{textcolor}%
\pgfsetfillcolor{textcolor}%
\pgftext[x=4.670315in,y=0.438444in,,top]{\color{textcolor}\sffamily\fontsize{8.000000}{9.600000}\selectfont 10}%
\end{pgfscope}%
\begin{pgfscope}%
\pgfpathrectangle{\pgfqpoint{3.278424in}{0.516222in}}{\pgfqpoint{2.577576in}{0.913411in}} %
\pgfusepath{clip}%
\pgfsetroundcap%
\pgfsetroundjoin%
\pgfsetlinewidth{0.803000pt}%
\definecolor{currentstroke}{rgb}{1.000000,1.000000,1.000000}%
\pgfsetstrokecolor{currentstroke}%
\pgfsetdash{}{0pt}%
\pgfpathmoveto{\pgfqpoint{5.185830in}{0.516222in}}%
\pgfpathlineto{\pgfqpoint{5.185830in}{1.429633in}}%
\pgfusepath{stroke}%
\end{pgfscope}%
\begin{pgfscope}%
\pgfsetbuttcap%
\pgfsetroundjoin%
\definecolor{currentfill}{rgb}{0.150000,0.150000,0.150000}%
\pgfsetfillcolor{currentfill}%
\pgfsetlinewidth{0.803000pt}%
\definecolor{currentstroke}{rgb}{0.150000,0.150000,0.150000}%
\pgfsetstrokecolor{currentstroke}%
\pgfsetdash{}{0pt}%
\pgfsys@defobject{currentmarker}{\pgfqpoint{0.000000in}{0.000000in}}{\pgfqpoint{0.000000in}{0.000000in}}{%
\pgfpathmoveto{\pgfqpoint{0.000000in}{0.000000in}}%
\pgfpathlineto{\pgfqpoint{0.000000in}{0.000000in}}%
\pgfusepath{stroke,fill}%
}%
\begin{pgfscope}%
\pgfsys@transformshift{5.185830in}{0.516222in}%
\pgfsys@useobject{currentmarker}{}%
\end{pgfscope}%
\end{pgfscope}%
\begin{pgfscope}%
\pgfsetbuttcap%
\pgfsetroundjoin%
\definecolor{currentfill}{rgb}{0.150000,0.150000,0.150000}%
\pgfsetfillcolor{currentfill}%
\pgfsetlinewidth{0.803000pt}%
\definecolor{currentstroke}{rgb}{0.150000,0.150000,0.150000}%
\pgfsetstrokecolor{currentstroke}%
\pgfsetdash{}{0pt}%
\pgfsys@defobject{currentmarker}{\pgfqpoint{0.000000in}{0.000000in}}{\pgfqpoint{0.000000in}{0.000000in}}{%
\pgfpathmoveto{\pgfqpoint{0.000000in}{0.000000in}}%
\pgfpathlineto{\pgfqpoint{0.000000in}{0.000000in}}%
\pgfusepath{stroke,fill}%
}%
\begin{pgfscope}%
\pgfsys@transformshift{5.185830in}{1.429633in}%
\pgfsys@useobject{currentmarker}{}%
\end{pgfscope}%
\end{pgfscope}%
\begin{pgfscope}%
\definecolor{textcolor}{rgb}{0.150000,0.150000,0.150000}%
\pgfsetstrokecolor{textcolor}%
\pgfsetfillcolor{textcolor}%
\pgftext[x=5.185830in,y=0.438444in,,top]{\color{textcolor}\sffamily\fontsize{8.000000}{9.600000}\selectfont 11}%
\end{pgfscope}%
\begin{pgfscope}%
\pgfpathrectangle{\pgfqpoint{3.278424in}{0.516222in}}{\pgfqpoint{2.577576in}{0.913411in}} %
\pgfusepath{clip}%
\pgfsetroundcap%
\pgfsetroundjoin%
\pgfsetlinewidth{0.803000pt}%
\definecolor{currentstroke}{rgb}{1.000000,1.000000,1.000000}%
\pgfsetstrokecolor{currentstroke}%
\pgfsetdash{}{0pt}%
\pgfpathmoveto{\pgfqpoint{5.701345in}{0.516222in}}%
\pgfpathlineto{\pgfqpoint{5.701345in}{1.429633in}}%
\pgfusepath{stroke}%
\end{pgfscope}%
\begin{pgfscope}%
\pgfsetbuttcap%
\pgfsetroundjoin%
\definecolor{currentfill}{rgb}{0.150000,0.150000,0.150000}%
\pgfsetfillcolor{currentfill}%
\pgfsetlinewidth{0.803000pt}%
\definecolor{currentstroke}{rgb}{0.150000,0.150000,0.150000}%
\pgfsetstrokecolor{currentstroke}%
\pgfsetdash{}{0pt}%
\pgfsys@defobject{currentmarker}{\pgfqpoint{0.000000in}{0.000000in}}{\pgfqpoint{0.000000in}{0.000000in}}{%
\pgfpathmoveto{\pgfqpoint{0.000000in}{0.000000in}}%
\pgfpathlineto{\pgfqpoint{0.000000in}{0.000000in}}%
\pgfusepath{stroke,fill}%
}%
\begin{pgfscope}%
\pgfsys@transformshift{5.701345in}{0.516222in}%
\pgfsys@useobject{currentmarker}{}%
\end{pgfscope}%
\end{pgfscope}%
\begin{pgfscope}%
\pgfsetbuttcap%
\pgfsetroundjoin%
\definecolor{currentfill}{rgb}{0.150000,0.150000,0.150000}%
\pgfsetfillcolor{currentfill}%
\pgfsetlinewidth{0.803000pt}%
\definecolor{currentstroke}{rgb}{0.150000,0.150000,0.150000}%
\pgfsetstrokecolor{currentstroke}%
\pgfsetdash{}{0pt}%
\pgfsys@defobject{currentmarker}{\pgfqpoint{0.000000in}{0.000000in}}{\pgfqpoint{0.000000in}{0.000000in}}{%
\pgfpathmoveto{\pgfqpoint{0.000000in}{0.000000in}}%
\pgfpathlineto{\pgfqpoint{0.000000in}{0.000000in}}%
\pgfusepath{stroke,fill}%
}%
\begin{pgfscope}%
\pgfsys@transformshift{5.701345in}{1.429633in}%
\pgfsys@useobject{currentmarker}{}%
\end{pgfscope}%
\end{pgfscope}%
\begin{pgfscope}%
\definecolor{textcolor}{rgb}{0.150000,0.150000,0.150000}%
\pgfsetstrokecolor{textcolor}%
\pgfsetfillcolor{textcolor}%
\pgftext[x=5.701345in,y=0.438444in,,top]{\color{textcolor}\sffamily\fontsize{8.000000}{9.600000}\selectfont 12}%
\end{pgfscope}%
\begin{pgfscope}%
\definecolor{textcolor}{rgb}{0.150000,0.150000,0.150000}%
\pgfsetstrokecolor{textcolor}%
\pgfsetfillcolor{textcolor}%
\pgftext[x=4.567212in,y=0.273321in,,top]{\color{textcolor}\sffamily\fontsize{8.800000}{10.560000}\selectfont Arm length}%
\end{pgfscope}%
\begin{pgfscope}%
\pgfpathrectangle{\pgfqpoint{3.278424in}{0.516222in}}{\pgfqpoint{2.577576in}{0.913411in}} %
\pgfusepath{clip}%
\pgfsetroundcap%
\pgfsetroundjoin%
\pgfsetlinewidth{0.803000pt}%
\definecolor{currentstroke}{rgb}{1.000000,1.000000,1.000000}%
\pgfsetstrokecolor{currentstroke}%
\pgfsetdash{}{0pt}%
\pgfpathmoveto{\pgfqpoint{3.278424in}{0.516222in}}%
\pgfpathlineto{\pgfqpoint{5.856000in}{0.516222in}}%
\pgfusepath{stroke}%
\end{pgfscope}%
\begin{pgfscope}%
\pgfsetbuttcap%
\pgfsetroundjoin%
\definecolor{currentfill}{rgb}{0.150000,0.150000,0.150000}%
\pgfsetfillcolor{currentfill}%
\pgfsetlinewidth{0.803000pt}%
\definecolor{currentstroke}{rgb}{0.150000,0.150000,0.150000}%
\pgfsetstrokecolor{currentstroke}%
\pgfsetdash{}{0pt}%
\pgfsys@defobject{currentmarker}{\pgfqpoint{0.000000in}{0.000000in}}{\pgfqpoint{0.000000in}{0.000000in}}{%
\pgfpathmoveto{\pgfqpoint{0.000000in}{0.000000in}}%
\pgfpathlineto{\pgfqpoint{0.000000in}{0.000000in}}%
\pgfusepath{stroke,fill}%
}%
\begin{pgfscope}%
\pgfsys@transformshift{3.278424in}{0.516222in}%
\pgfsys@useobject{currentmarker}{}%
\end{pgfscope}%
\end{pgfscope}%
\begin{pgfscope}%
\pgfsetbuttcap%
\pgfsetroundjoin%
\definecolor{currentfill}{rgb}{0.150000,0.150000,0.150000}%
\pgfsetfillcolor{currentfill}%
\pgfsetlinewidth{0.803000pt}%
\definecolor{currentstroke}{rgb}{0.150000,0.150000,0.150000}%
\pgfsetstrokecolor{currentstroke}%
\pgfsetdash{}{0pt}%
\pgfsys@defobject{currentmarker}{\pgfqpoint{0.000000in}{0.000000in}}{\pgfqpoint{0.000000in}{0.000000in}}{%
\pgfpathmoveto{\pgfqpoint{0.000000in}{0.000000in}}%
\pgfpathlineto{\pgfqpoint{0.000000in}{0.000000in}}%
\pgfusepath{stroke,fill}%
}%
\begin{pgfscope}%
\pgfsys@transformshift{5.856000in}{0.516222in}%
\pgfsys@useobject{currentmarker}{}%
\end{pgfscope}%
\end{pgfscope}%
\begin{pgfscope}%
\pgfpathrectangle{\pgfqpoint{3.278424in}{0.516222in}}{\pgfqpoint{2.577576in}{0.913411in}} %
\pgfusepath{clip}%
\pgfsetroundcap%
\pgfsetroundjoin%
\pgfsetlinewidth{0.803000pt}%
\definecolor{currentstroke}{rgb}{1.000000,1.000000,1.000000}%
\pgfsetstrokecolor{currentstroke}%
\pgfsetdash{}{0pt}%
\pgfpathmoveto{\pgfqpoint{3.278424in}{0.668457in}}%
\pgfpathlineto{\pgfqpoint{5.856000in}{0.668457in}}%
\pgfusepath{stroke}%
\end{pgfscope}%
\begin{pgfscope}%
\pgfsetbuttcap%
\pgfsetroundjoin%
\definecolor{currentfill}{rgb}{0.150000,0.150000,0.150000}%
\pgfsetfillcolor{currentfill}%
\pgfsetlinewidth{0.803000pt}%
\definecolor{currentstroke}{rgb}{0.150000,0.150000,0.150000}%
\pgfsetstrokecolor{currentstroke}%
\pgfsetdash{}{0pt}%
\pgfsys@defobject{currentmarker}{\pgfqpoint{0.000000in}{0.000000in}}{\pgfqpoint{0.000000in}{0.000000in}}{%
\pgfpathmoveto{\pgfqpoint{0.000000in}{0.000000in}}%
\pgfpathlineto{\pgfqpoint{0.000000in}{0.000000in}}%
\pgfusepath{stroke,fill}%
}%
\begin{pgfscope}%
\pgfsys@transformshift{3.278424in}{0.668457in}%
\pgfsys@useobject{currentmarker}{}%
\end{pgfscope}%
\end{pgfscope}%
\begin{pgfscope}%
\pgfsetbuttcap%
\pgfsetroundjoin%
\definecolor{currentfill}{rgb}{0.150000,0.150000,0.150000}%
\pgfsetfillcolor{currentfill}%
\pgfsetlinewidth{0.803000pt}%
\definecolor{currentstroke}{rgb}{0.150000,0.150000,0.150000}%
\pgfsetstrokecolor{currentstroke}%
\pgfsetdash{}{0pt}%
\pgfsys@defobject{currentmarker}{\pgfqpoint{0.000000in}{0.000000in}}{\pgfqpoint{0.000000in}{0.000000in}}{%
\pgfpathmoveto{\pgfqpoint{0.000000in}{0.000000in}}%
\pgfpathlineto{\pgfqpoint{0.000000in}{0.000000in}}%
\pgfusepath{stroke,fill}%
}%
\begin{pgfscope}%
\pgfsys@transformshift{5.856000in}{0.668457in}%
\pgfsys@useobject{currentmarker}{}%
\end{pgfscope}%
\end{pgfscope}%
\begin{pgfscope}%
\pgfpathrectangle{\pgfqpoint{3.278424in}{0.516222in}}{\pgfqpoint{2.577576in}{0.913411in}} %
\pgfusepath{clip}%
\pgfsetroundcap%
\pgfsetroundjoin%
\pgfsetlinewidth{0.803000pt}%
\definecolor{currentstroke}{rgb}{1.000000,1.000000,1.000000}%
\pgfsetstrokecolor{currentstroke}%
\pgfsetdash{}{0pt}%
\pgfpathmoveto{\pgfqpoint{3.278424in}{0.820693in}}%
\pgfpathlineto{\pgfqpoint{5.856000in}{0.820693in}}%
\pgfusepath{stroke}%
\end{pgfscope}%
\begin{pgfscope}%
\pgfsetbuttcap%
\pgfsetroundjoin%
\definecolor{currentfill}{rgb}{0.150000,0.150000,0.150000}%
\pgfsetfillcolor{currentfill}%
\pgfsetlinewidth{0.803000pt}%
\definecolor{currentstroke}{rgb}{0.150000,0.150000,0.150000}%
\pgfsetstrokecolor{currentstroke}%
\pgfsetdash{}{0pt}%
\pgfsys@defobject{currentmarker}{\pgfqpoint{0.000000in}{0.000000in}}{\pgfqpoint{0.000000in}{0.000000in}}{%
\pgfpathmoveto{\pgfqpoint{0.000000in}{0.000000in}}%
\pgfpathlineto{\pgfqpoint{0.000000in}{0.000000in}}%
\pgfusepath{stroke,fill}%
}%
\begin{pgfscope}%
\pgfsys@transformshift{3.278424in}{0.820693in}%
\pgfsys@useobject{currentmarker}{}%
\end{pgfscope}%
\end{pgfscope}%
\begin{pgfscope}%
\pgfsetbuttcap%
\pgfsetroundjoin%
\definecolor{currentfill}{rgb}{0.150000,0.150000,0.150000}%
\pgfsetfillcolor{currentfill}%
\pgfsetlinewidth{0.803000pt}%
\definecolor{currentstroke}{rgb}{0.150000,0.150000,0.150000}%
\pgfsetstrokecolor{currentstroke}%
\pgfsetdash{}{0pt}%
\pgfsys@defobject{currentmarker}{\pgfqpoint{0.000000in}{0.000000in}}{\pgfqpoint{0.000000in}{0.000000in}}{%
\pgfpathmoveto{\pgfqpoint{0.000000in}{0.000000in}}%
\pgfpathlineto{\pgfqpoint{0.000000in}{0.000000in}}%
\pgfusepath{stroke,fill}%
}%
\begin{pgfscope}%
\pgfsys@transformshift{5.856000in}{0.820693in}%
\pgfsys@useobject{currentmarker}{}%
\end{pgfscope}%
\end{pgfscope}%
\begin{pgfscope}%
\pgfpathrectangle{\pgfqpoint{3.278424in}{0.516222in}}{\pgfqpoint{2.577576in}{0.913411in}} %
\pgfusepath{clip}%
\pgfsetroundcap%
\pgfsetroundjoin%
\pgfsetlinewidth{0.803000pt}%
\definecolor{currentstroke}{rgb}{1.000000,1.000000,1.000000}%
\pgfsetstrokecolor{currentstroke}%
\pgfsetdash{}{0pt}%
\pgfpathmoveto{\pgfqpoint{3.278424in}{0.972928in}}%
\pgfpathlineto{\pgfqpoint{5.856000in}{0.972928in}}%
\pgfusepath{stroke}%
\end{pgfscope}%
\begin{pgfscope}%
\pgfsetbuttcap%
\pgfsetroundjoin%
\definecolor{currentfill}{rgb}{0.150000,0.150000,0.150000}%
\pgfsetfillcolor{currentfill}%
\pgfsetlinewidth{0.803000pt}%
\definecolor{currentstroke}{rgb}{0.150000,0.150000,0.150000}%
\pgfsetstrokecolor{currentstroke}%
\pgfsetdash{}{0pt}%
\pgfsys@defobject{currentmarker}{\pgfqpoint{0.000000in}{0.000000in}}{\pgfqpoint{0.000000in}{0.000000in}}{%
\pgfpathmoveto{\pgfqpoint{0.000000in}{0.000000in}}%
\pgfpathlineto{\pgfqpoint{0.000000in}{0.000000in}}%
\pgfusepath{stroke,fill}%
}%
\begin{pgfscope}%
\pgfsys@transformshift{3.278424in}{0.972928in}%
\pgfsys@useobject{currentmarker}{}%
\end{pgfscope}%
\end{pgfscope}%
\begin{pgfscope}%
\pgfsetbuttcap%
\pgfsetroundjoin%
\definecolor{currentfill}{rgb}{0.150000,0.150000,0.150000}%
\pgfsetfillcolor{currentfill}%
\pgfsetlinewidth{0.803000pt}%
\definecolor{currentstroke}{rgb}{0.150000,0.150000,0.150000}%
\pgfsetstrokecolor{currentstroke}%
\pgfsetdash{}{0pt}%
\pgfsys@defobject{currentmarker}{\pgfqpoint{0.000000in}{0.000000in}}{\pgfqpoint{0.000000in}{0.000000in}}{%
\pgfpathmoveto{\pgfqpoint{0.000000in}{0.000000in}}%
\pgfpathlineto{\pgfqpoint{0.000000in}{0.000000in}}%
\pgfusepath{stroke,fill}%
}%
\begin{pgfscope}%
\pgfsys@transformshift{5.856000in}{0.972928in}%
\pgfsys@useobject{currentmarker}{}%
\end{pgfscope}%
\end{pgfscope}%
\begin{pgfscope}%
\pgfpathrectangle{\pgfqpoint{3.278424in}{0.516222in}}{\pgfqpoint{2.577576in}{0.913411in}} %
\pgfusepath{clip}%
\pgfsetroundcap%
\pgfsetroundjoin%
\pgfsetlinewidth{0.803000pt}%
\definecolor{currentstroke}{rgb}{1.000000,1.000000,1.000000}%
\pgfsetstrokecolor{currentstroke}%
\pgfsetdash{}{0pt}%
\pgfpathmoveto{\pgfqpoint{3.278424in}{1.125163in}}%
\pgfpathlineto{\pgfqpoint{5.856000in}{1.125163in}}%
\pgfusepath{stroke}%
\end{pgfscope}%
\begin{pgfscope}%
\pgfsetbuttcap%
\pgfsetroundjoin%
\definecolor{currentfill}{rgb}{0.150000,0.150000,0.150000}%
\pgfsetfillcolor{currentfill}%
\pgfsetlinewidth{0.803000pt}%
\definecolor{currentstroke}{rgb}{0.150000,0.150000,0.150000}%
\pgfsetstrokecolor{currentstroke}%
\pgfsetdash{}{0pt}%
\pgfsys@defobject{currentmarker}{\pgfqpoint{0.000000in}{0.000000in}}{\pgfqpoint{0.000000in}{0.000000in}}{%
\pgfpathmoveto{\pgfqpoint{0.000000in}{0.000000in}}%
\pgfpathlineto{\pgfqpoint{0.000000in}{0.000000in}}%
\pgfusepath{stroke,fill}%
}%
\begin{pgfscope}%
\pgfsys@transformshift{3.278424in}{1.125163in}%
\pgfsys@useobject{currentmarker}{}%
\end{pgfscope}%
\end{pgfscope}%
\begin{pgfscope}%
\pgfsetbuttcap%
\pgfsetroundjoin%
\definecolor{currentfill}{rgb}{0.150000,0.150000,0.150000}%
\pgfsetfillcolor{currentfill}%
\pgfsetlinewidth{0.803000pt}%
\definecolor{currentstroke}{rgb}{0.150000,0.150000,0.150000}%
\pgfsetstrokecolor{currentstroke}%
\pgfsetdash{}{0pt}%
\pgfsys@defobject{currentmarker}{\pgfqpoint{0.000000in}{0.000000in}}{\pgfqpoint{0.000000in}{0.000000in}}{%
\pgfpathmoveto{\pgfqpoint{0.000000in}{0.000000in}}%
\pgfpathlineto{\pgfqpoint{0.000000in}{0.000000in}}%
\pgfusepath{stroke,fill}%
}%
\begin{pgfscope}%
\pgfsys@transformshift{5.856000in}{1.125163in}%
\pgfsys@useobject{currentmarker}{}%
\end{pgfscope}%
\end{pgfscope}%
\begin{pgfscope}%
\pgfpathrectangle{\pgfqpoint{3.278424in}{0.516222in}}{\pgfqpoint{2.577576in}{0.913411in}} %
\pgfusepath{clip}%
\pgfsetroundcap%
\pgfsetroundjoin%
\pgfsetlinewidth{0.803000pt}%
\definecolor{currentstroke}{rgb}{1.000000,1.000000,1.000000}%
\pgfsetstrokecolor{currentstroke}%
\pgfsetdash{}{0pt}%
\pgfpathmoveto{\pgfqpoint{3.278424in}{1.277398in}}%
\pgfpathlineto{\pgfqpoint{5.856000in}{1.277398in}}%
\pgfusepath{stroke}%
\end{pgfscope}%
\begin{pgfscope}%
\pgfsetbuttcap%
\pgfsetroundjoin%
\definecolor{currentfill}{rgb}{0.150000,0.150000,0.150000}%
\pgfsetfillcolor{currentfill}%
\pgfsetlinewidth{0.803000pt}%
\definecolor{currentstroke}{rgb}{0.150000,0.150000,0.150000}%
\pgfsetstrokecolor{currentstroke}%
\pgfsetdash{}{0pt}%
\pgfsys@defobject{currentmarker}{\pgfqpoint{0.000000in}{0.000000in}}{\pgfqpoint{0.000000in}{0.000000in}}{%
\pgfpathmoveto{\pgfqpoint{0.000000in}{0.000000in}}%
\pgfpathlineto{\pgfqpoint{0.000000in}{0.000000in}}%
\pgfusepath{stroke,fill}%
}%
\begin{pgfscope}%
\pgfsys@transformshift{3.278424in}{1.277398in}%
\pgfsys@useobject{currentmarker}{}%
\end{pgfscope}%
\end{pgfscope}%
\begin{pgfscope}%
\pgfsetbuttcap%
\pgfsetroundjoin%
\definecolor{currentfill}{rgb}{0.150000,0.150000,0.150000}%
\pgfsetfillcolor{currentfill}%
\pgfsetlinewidth{0.803000pt}%
\definecolor{currentstroke}{rgb}{0.150000,0.150000,0.150000}%
\pgfsetstrokecolor{currentstroke}%
\pgfsetdash{}{0pt}%
\pgfsys@defobject{currentmarker}{\pgfqpoint{0.000000in}{0.000000in}}{\pgfqpoint{0.000000in}{0.000000in}}{%
\pgfpathmoveto{\pgfqpoint{0.000000in}{0.000000in}}%
\pgfpathlineto{\pgfqpoint{0.000000in}{0.000000in}}%
\pgfusepath{stroke,fill}%
}%
\begin{pgfscope}%
\pgfsys@transformshift{5.856000in}{1.277398in}%
\pgfsys@useobject{currentmarker}{}%
\end{pgfscope}%
\end{pgfscope}%
\begin{pgfscope}%
\pgfpathrectangle{\pgfqpoint{3.278424in}{0.516222in}}{\pgfqpoint{2.577576in}{0.913411in}} %
\pgfusepath{clip}%
\pgfsetroundcap%
\pgfsetroundjoin%
\pgfsetlinewidth{0.803000pt}%
\definecolor{currentstroke}{rgb}{1.000000,1.000000,1.000000}%
\pgfsetstrokecolor{currentstroke}%
\pgfsetdash{}{0pt}%
\pgfpathmoveto{\pgfqpoint{3.278424in}{1.429633in}}%
\pgfpathlineto{\pgfqpoint{5.856000in}{1.429633in}}%
\pgfusepath{stroke}%
\end{pgfscope}%
\begin{pgfscope}%
\pgfsetbuttcap%
\pgfsetroundjoin%
\definecolor{currentfill}{rgb}{0.150000,0.150000,0.150000}%
\pgfsetfillcolor{currentfill}%
\pgfsetlinewidth{0.803000pt}%
\definecolor{currentstroke}{rgb}{0.150000,0.150000,0.150000}%
\pgfsetstrokecolor{currentstroke}%
\pgfsetdash{}{0pt}%
\pgfsys@defobject{currentmarker}{\pgfqpoint{0.000000in}{0.000000in}}{\pgfqpoint{0.000000in}{0.000000in}}{%
\pgfpathmoveto{\pgfqpoint{0.000000in}{0.000000in}}%
\pgfpathlineto{\pgfqpoint{0.000000in}{0.000000in}}%
\pgfusepath{stroke,fill}%
}%
\begin{pgfscope}%
\pgfsys@transformshift{3.278424in}{1.429633in}%
\pgfsys@useobject{currentmarker}{}%
\end{pgfscope}%
\end{pgfscope}%
\begin{pgfscope}%
\pgfsetbuttcap%
\pgfsetroundjoin%
\definecolor{currentfill}{rgb}{0.150000,0.150000,0.150000}%
\pgfsetfillcolor{currentfill}%
\pgfsetlinewidth{0.803000pt}%
\definecolor{currentstroke}{rgb}{0.150000,0.150000,0.150000}%
\pgfsetstrokecolor{currentstroke}%
\pgfsetdash{}{0pt}%
\pgfsys@defobject{currentmarker}{\pgfqpoint{0.000000in}{0.000000in}}{\pgfqpoint{0.000000in}{0.000000in}}{%
\pgfpathmoveto{\pgfqpoint{0.000000in}{0.000000in}}%
\pgfpathlineto{\pgfqpoint{0.000000in}{0.000000in}}%
\pgfusepath{stroke,fill}%
}%
\begin{pgfscope}%
\pgfsys@transformshift{5.856000in}{1.429633in}%
\pgfsys@useobject{currentmarker}{}%
\end{pgfscope}%
\end{pgfscope}%
\begin{pgfscope}%
\pgfpathrectangle{\pgfqpoint{3.278424in}{0.516222in}}{\pgfqpoint{2.577576in}{0.913411in}} %
\pgfusepath{clip}%
\pgfsetbuttcap%
\pgfsetmiterjoin%
\definecolor{currentfill}{rgb}{0.447059,0.623529,0.811765}%
\pgfsetfillcolor{currentfill}%
\pgfsetfillopacity{0.300000}%
\pgfsetlinewidth{0.240900pt}%
\definecolor{currentstroke}{rgb}{0.447059,0.623529,0.811765}%
\pgfsetstrokecolor{currentstroke}%
\pgfsetstrokeopacity{0.300000}%
\pgfsetdash{}{0pt}%
\pgfpathmoveto{\pgfqpoint{3.381527in}{0.893480in}}%
\pgfpathlineto{\pgfqpoint{3.405480in}{0.896086in}}%
\pgfpathlineto{\pgfqpoint{3.429433in}{0.898767in}}%
\pgfpathlineto{\pgfqpoint{3.453386in}{0.901526in}}%
\pgfpathlineto{\pgfqpoint{3.477340in}{0.904369in}}%
\pgfpathlineto{\pgfqpoint{3.501293in}{0.907301in}}%
\pgfpathlineto{\pgfqpoint{3.525246in}{0.910327in}}%
\pgfpathlineto{\pgfqpoint{3.549199in}{0.913452in}}%
\pgfpathlineto{\pgfqpoint{3.573153in}{0.916681in}}%
\pgfpathlineto{\pgfqpoint{3.597106in}{0.920019in}}%
\pgfpathlineto{\pgfqpoint{3.621059in}{0.923469in}}%
\pgfpathlineto{\pgfqpoint{3.645012in}{0.927034in}}%
\pgfpathlineto{\pgfqpoint{3.668965in}{0.930717in}}%
\pgfpathlineto{\pgfqpoint{3.692919in}{0.934517in}}%
\pgfpathlineto{\pgfqpoint{3.716872in}{0.938434in}}%
\pgfpathlineto{\pgfqpoint{3.740825in}{0.942464in}}%
\pgfpathlineto{\pgfqpoint{3.764778in}{0.946602in}}%
\pgfpathlineto{\pgfqpoint{3.788732in}{0.950842in}}%
\pgfpathlineto{\pgfqpoint{3.812685in}{0.955175in}}%
\pgfpathlineto{\pgfqpoint{3.836638in}{0.959591in}}%
\pgfpathlineto{\pgfqpoint{3.860591in}{0.964078in}}%
\pgfpathlineto{\pgfqpoint{3.884545in}{0.968625in}}%
\pgfpathlineto{\pgfqpoint{3.908498in}{0.973217in}}%
\pgfpathlineto{\pgfqpoint{3.932451in}{0.977842in}}%
\pgfpathlineto{\pgfqpoint{3.956404in}{0.982486in}}%
\pgfpathlineto{\pgfqpoint{3.980358in}{0.987138in}}%
\pgfpathlineto{\pgfqpoint{4.004311in}{0.991784in}}%
\pgfpathlineto{\pgfqpoint{4.028264in}{0.996413in}}%
\pgfpathlineto{\pgfqpoint{4.052217in}{1.001015in}}%
\pgfpathlineto{\pgfqpoint{4.076170in}{1.005579in}}%
\pgfpathlineto{\pgfqpoint{4.100124in}{1.010096in}}%
\pgfpathlineto{\pgfqpoint{4.124077in}{1.014558in}}%
\pgfpathlineto{\pgfqpoint{4.148030in}{1.018958in}}%
\pgfpathlineto{\pgfqpoint{4.171983in}{1.023289in}}%
\pgfpathlineto{\pgfqpoint{4.195937in}{1.027546in}}%
\pgfpathlineto{\pgfqpoint{4.219890in}{1.031721in}}%
\pgfpathlineto{\pgfqpoint{4.243843in}{1.035812in}}%
\pgfpathlineto{\pgfqpoint{4.267796in}{1.039814in}}%
\pgfpathlineto{\pgfqpoint{4.291750in}{1.043723in}}%
\pgfpathlineto{\pgfqpoint{4.315703in}{1.047536in}}%
\pgfpathlineto{\pgfqpoint{4.339656in}{1.051251in}}%
\pgfpathlineto{\pgfqpoint{4.363609in}{1.054865in}}%
\pgfpathlineto{\pgfqpoint{4.387563in}{1.058376in}}%
\pgfpathlineto{\pgfqpoint{4.411516in}{1.061783in}}%
\pgfpathlineto{\pgfqpoint{4.435469in}{1.065083in}}%
\pgfpathlineto{\pgfqpoint{4.459422in}{1.068277in}}%
\pgfpathlineto{\pgfqpoint{4.483375in}{1.071363in}}%
\pgfpathlineto{\pgfqpoint{4.507329in}{1.074342in}}%
\pgfpathlineto{\pgfqpoint{4.531282in}{1.077211in}}%
\pgfpathlineto{\pgfqpoint{4.555235in}{1.079972in}}%
\pgfpathlineto{\pgfqpoint{4.579188in}{1.082626in}}%
\pgfpathlineto{\pgfqpoint{4.603142in}{1.085171in}}%
\pgfpathlineto{\pgfqpoint{4.627095in}{1.087610in}}%
\pgfpathlineto{\pgfqpoint{4.651048in}{1.089942in}}%
\pgfpathlineto{\pgfqpoint{4.675001in}{1.092170in}}%
\pgfpathlineto{\pgfqpoint{4.698955in}{1.094296in}}%
\pgfpathlineto{\pgfqpoint{4.722908in}{1.096320in}}%
\pgfpathlineto{\pgfqpoint{4.746861in}{1.098245in}}%
\pgfpathlineto{\pgfqpoint{4.770814in}{1.100074in}}%
\pgfpathlineto{\pgfqpoint{4.794768in}{1.101809in}}%
\pgfpathlineto{\pgfqpoint{4.818721in}{1.103455in}}%
\pgfpathlineto{\pgfqpoint{4.842674in}{1.105015in}}%
\pgfpathlineto{\pgfqpoint{4.866627in}{1.106493in}}%
\pgfpathlineto{\pgfqpoint{4.890580in}{1.107894in}}%
\pgfpathlineto{\pgfqpoint{4.914534in}{1.109224in}}%
\pgfpathlineto{\pgfqpoint{4.938487in}{1.110489in}}%
\pgfpathlineto{\pgfqpoint{4.962440in}{1.111695in}}%
\pgfpathlineto{\pgfqpoint{4.986393in}{1.112850in}}%
\pgfpathlineto{\pgfqpoint{5.010347in}{1.113962in}}%
\pgfpathlineto{\pgfqpoint{5.034300in}{1.115040in}}%
\pgfpathlineto{\pgfqpoint{5.058253in}{1.116094in}}%
\pgfpathlineto{\pgfqpoint{5.082206in}{1.117134in}}%
\pgfpathlineto{\pgfqpoint{5.106160in}{1.118171in}}%
\pgfpathlineto{\pgfqpoint{5.130113in}{1.119216in}}%
\pgfpathlineto{\pgfqpoint{5.154066in}{1.120281in}}%
\pgfpathlineto{\pgfqpoint{5.178019in}{1.121377in}}%
\pgfpathlineto{\pgfqpoint{5.201973in}{1.122518in}}%
\pgfpathlineto{\pgfqpoint{5.225926in}{1.123714in}}%
\pgfpathlineto{\pgfqpoint{5.249879in}{1.124976in}}%
\pgfpathlineto{\pgfqpoint{5.273832in}{1.126315in}}%
\pgfpathlineto{\pgfqpoint{5.297785in}{1.127738in}}%
\pgfpathlineto{\pgfqpoint{5.321739in}{1.129255in}}%
\pgfpathlineto{\pgfqpoint{5.345692in}{1.130871in}}%
\pgfpathlineto{\pgfqpoint{5.369645in}{1.132591in}}%
\pgfpathlineto{\pgfqpoint{5.393598in}{1.134417in}}%
\pgfpathlineto{\pgfqpoint{5.417552in}{1.136350in}}%
\pgfpathlineto{\pgfqpoint{5.441505in}{1.138392in}}%
\pgfpathlineto{\pgfqpoint{5.465458in}{1.140540in}}%
\pgfpathlineto{\pgfqpoint{5.489411in}{1.142793in}}%
\pgfpathlineto{\pgfqpoint{5.513365in}{1.145148in}}%
\pgfpathlineto{\pgfqpoint{5.537318in}{1.147600in}}%
\pgfpathlineto{\pgfqpoint{5.561271in}{1.150147in}}%
\pgfpathlineto{\pgfqpoint{5.585224in}{1.152783in}}%
\pgfpathlineto{\pgfqpoint{5.609178in}{1.155505in}}%
\pgfpathlineto{\pgfqpoint{5.633131in}{1.158307in}}%
\pgfpathlineto{\pgfqpoint{5.657084in}{1.161187in}}%
\pgfpathlineto{\pgfqpoint{5.681037in}{1.164140in}}%
\pgfpathlineto{\pgfqpoint{5.704990in}{1.167161in}}%
\pgfpathlineto{\pgfqpoint{5.728944in}{1.170247in}}%
\pgfpathlineto{\pgfqpoint{5.752897in}{1.173395in}}%
\pgfpathlineto{\pgfqpoint{5.752897in}{0.996072in}}%
\pgfpathlineto{\pgfqpoint{5.728944in}{0.999930in}}%
\pgfpathlineto{\pgfqpoint{5.704990in}{1.003602in}}%
\pgfpathlineto{\pgfqpoint{5.681037in}{1.007084in}}%
\pgfpathlineto{\pgfqpoint{5.657084in}{1.010372in}}%
\pgfpathlineto{\pgfqpoint{5.633131in}{1.013464in}}%
\pgfpathlineto{\pgfqpoint{5.609178in}{1.016355in}}%
\pgfpathlineto{\pgfqpoint{5.585224in}{1.019039in}}%
\pgfpathlineto{\pgfqpoint{5.561271in}{1.021515in}}%
\pgfpathlineto{\pgfqpoint{5.537318in}{1.023775in}}%
\pgfpathlineto{\pgfqpoint{5.513365in}{1.025818in}}%
\pgfpathlineto{\pgfqpoint{5.489411in}{1.027638in}}%
\pgfpathlineto{\pgfqpoint{5.465458in}{1.029232in}}%
\pgfpathlineto{\pgfqpoint{5.441505in}{1.030597in}}%
\pgfpathlineto{\pgfqpoint{5.417552in}{1.031731in}}%
\pgfpathlineto{\pgfqpoint{5.393598in}{1.032633in}}%
\pgfpathlineto{\pgfqpoint{5.369645in}{1.033302in}}%
\pgfpathlineto{\pgfqpoint{5.345692in}{1.033741in}}%
\pgfpathlineto{\pgfqpoint{5.321739in}{1.033952in}}%
\pgfpathlineto{\pgfqpoint{5.297785in}{1.033939in}}%
\pgfpathlineto{\pgfqpoint{5.273832in}{1.033709in}}%
\pgfpathlineto{\pgfqpoint{5.249879in}{1.033269in}}%
\pgfpathlineto{\pgfqpoint{5.225926in}{1.032628in}}%
\pgfpathlineto{\pgfqpoint{5.201973in}{1.031797in}}%
\pgfpathlineto{\pgfqpoint{5.178019in}{1.030786in}}%
\pgfpathlineto{\pgfqpoint{5.154066in}{1.029607in}}%
\pgfpathlineto{\pgfqpoint{5.130113in}{1.028271in}}%
\pgfpathlineto{\pgfqpoint{5.106160in}{1.026791in}}%
\pgfpathlineto{\pgfqpoint{5.082206in}{1.025178in}}%
\pgfpathlineto{\pgfqpoint{5.058253in}{1.023445in}}%
\pgfpathlineto{\pgfqpoint{5.034300in}{1.021600in}}%
\pgfpathlineto{\pgfqpoint{5.010347in}{1.019656in}}%
\pgfpathlineto{\pgfqpoint{4.986393in}{1.017622in}}%
\pgfpathlineto{\pgfqpoint{4.962440in}{1.015506in}}%
\pgfpathlineto{\pgfqpoint{4.938487in}{1.013316in}}%
\pgfpathlineto{\pgfqpoint{4.914534in}{1.011061in}}%
\pgfpathlineto{\pgfqpoint{4.890580in}{1.008746in}}%
\pgfpathlineto{\pgfqpoint{4.866627in}{1.006378in}}%
\pgfpathlineto{\pgfqpoint{4.842674in}{1.003963in}}%
\pgfpathlineto{\pgfqpoint{4.818721in}{1.001505in}}%
\pgfpathlineto{\pgfqpoint{4.794768in}{0.999009in}}%
\pgfpathlineto{\pgfqpoint{4.770814in}{0.996478in}}%
\pgfpathlineto{\pgfqpoint{4.746861in}{0.993916in}}%
\pgfpathlineto{\pgfqpoint{4.722908in}{0.991326in}}%
\pgfpathlineto{\pgfqpoint{4.698955in}{0.988710in}}%
\pgfpathlineto{\pgfqpoint{4.675001in}{0.986071in}}%
\pgfpathlineto{\pgfqpoint{4.651048in}{0.983411in}}%
\pgfpathlineto{\pgfqpoint{4.627095in}{0.980730in}}%
\pgfpathlineto{\pgfqpoint{4.603142in}{0.978032in}}%
\pgfpathlineto{\pgfqpoint{4.579188in}{0.975315in}}%
\pgfpathlineto{\pgfqpoint{4.555235in}{0.972582in}}%
\pgfpathlineto{\pgfqpoint{4.531282in}{0.969833in}}%
\pgfpathlineto{\pgfqpoint{4.507329in}{0.967068in}}%
\pgfpathlineto{\pgfqpoint{4.483375in}{0.964287in}}%
\pgfpathlineto{\pgfqpoint{4.459422in}{0.961489in}}%
\pgfpathlineto{\pgfqpoint{4.435469in}{0.958675in}}%
\pgfpathlineto{\pgfqpoint{4.411516in}{0.955843in}}%
\pgfpathlineto{\pgfqpoint{4.387563in}{0.952993in}}%
\pgfpathlineto{\pgfqpoint{4.363609in}{0.950122in}}%
\pgfpathlineto{\pgfqpoint{4.339656in}{0.947231in}}%
\pgfpathlineto{\pgfqpoint{4.315703in}{0.944315in}}%
\pgfpathlineto{\pgfqpoint{4.291750in}{0.941374in}}%
\pgfpathlineto{\pgfqpoint{4.267796in}{0.938404in}}%
\pgfpathlineto{\pgfqpoint{4.243843in}{0.935402in}}%
\pgfpathlineto{\pgfqpoint{4.219890in}{0.932365in}}%
\pgfpathlineto{\pgfqpoint{4.195937in}{0.929289in}}%
\pgfpathlineto{\pgfqpoint{4.171983in}{0.926169in}}%
\pgfpathlineto{\pgfqpoint{4.148030in}{0.922999in}}%
\pgfpathlineto{\pgfqpoint{4.124077in}{0.919773in}}%
\pgfpathlineto{\pgfqpoint{4.100124in}{0.916486in}}%
\pgfpathlineto{\pgfqpoint{4.076170in}{0.913129in}}%
\pgfpathlineto{\pgfqpoint{4.052217in}{0.909694in}}%
\pgfpathlineto{\pgfqpoint{4.028264in}{0.906173in}}%
\pgfpathlineto{\pgfqpoint{4.004311in}{0.902555in}}%
\pgfpathlineto{\pgfqpoint{3.980358in}{0.898829in}}%
\pgfpathlineto{\pgfqpoint{3.956404in}{0.894984in}}%
\pgfpathlineto{\pgfqpoint{3.932451in}{0.891008in}}%
\pgfpathlineto{\pgfqpoint{3.908498in}{0.886888in}}%
\pgfpathlineto{\pgfqpoint{3.884545in}{0.882611in}}%
\pgfpathlineto{\pgfqpoint{3.860591in}{0.878163in}}%
\pgfpathlineto{\pgfqpoint{3.836638in}{0.873532in}}%
\pgfpathlineto{\pgfqpoint{3.812685in}{0.868705in}}%
\pgfpathlineto{\pgfqpoint{3.788732in}{0.863671in}}%
\pgfpathlineto{\pgfqpoint{3.764778in}{0.858420in}}%
\pgfpathlineto{\pgfqpoint{3.740825in}{0.852943in}}%
\pgfpathlineto{\pgfqpoint{3.716872in}{0.847232in}}%
\pgfpathlineto{\pgfqpoint{3.692919in}{0.841284in}}%
\pgfpathlineto{\pgfqpoint{3.668965in}{0.835096in}}%
\pgfpathlineto{\pgfqpoint{3.645012in}{0.828665in}}%
\pgfpathlineto{\pgfqpoint{3.621059in}{0.821992in}}%
\pgfpathlineto{\pgfqpoint{3.597106in}{0.815080in}}%
\pgfpathlineto{\pgfqpoint{3.573153in}{0.807931in}}%
\pgfpathlineto{\pgfqpoint{3.549199in}{0.800549in}}%
\pgfpathlineto{\pgfqpoint{3.525246in}{0.792939in}}%
\pgfpathlineto{\pgfqpoint{3.501293in}{0.785105in}}%
\pgfpathlineto{\pgfqpoint{3.477340in}{0.777053in}}%
\pgfpathlineto{\pgfqpoint{3.453386in}{0.768787in}}%
\pgfpathlineto{\pgfqpoint{3.429433in}{0.760313in}}%
\pgfpathlineto{\pgfqpoint{3.405480in}{0.751636in}}%
\pgfpathlineto{\pgfqpoint{3.381527in}{0.742761in}}%
\pgfpathclose%
\pgfusepath{stroke,fill}%
\end{pgfscope}%
\begin{pgfscope}%
\pgfpathrectangle{\pgfqpoint{3.278424in}{0.516222in}}{\pgfqpoint{2.577576in}{0.913411in}} %
\pgfusepath{clip}%
\pgfsetroundcap%
\pgfsetroundjoin%
\pgfsetlinewidth{2.007500pt}%
\definecolor{currentstroke}{rgb}{0.125490,0.290196,0.529412}%
\pgfsetstrokecolor{currentstroke}%
\pgfsetdash{}{0pt}%
\pgfpathmoveto{\pgfqpoint{3.381527in}{0.818120in}}%
\pgfpathlineto{\pgfqpoint{3.405480in}{0.823861in}}%
\pgfpathlineto{\pgfqpoint{3.429433in}{0.829540in}}%
\pgfpathlineto{\pgfqpoint{3.453386in}{0.835157in}}%
\pgfpathlineto{\pgfqpoint{3.477340in}{0.840711in}}%
\pgfpathlineto{\pgfqpoint{3.501293in}{0.846203in}}%
\pgfpathlineto{\pgfqpoint{3.525246in}{0.851633in}}%
\pgfpathlineto{\pgfqpoint{3.549199in}{0.857001in}}%
\pgfpathlineto{\pgfqpoint{3.573153in}{0.862306in}}%
\pgfpathlineto{\pgfqpoint{3.597106in}{0.867550in}}%
\pgfpathlineto{\pgfqpoint{3.621059in}{0.872731in}}%
\pgfpathlineto{\pgfqpoint{3.645012in}{0.877850in}}%
\pgfpathlineto{\pgfqpoint{3.668965in}{0.882906in}}%
\pgfpathlineto{\pgfqpoint{3.692919in}{0.887901in}}%
\pgfpathlineto{\pgfqpoint{3.716872in}{0.892833in}}%
\pgfpathlineto{\pgfqpoint{3.740825in}{0.897703in}}%
\pgfpathlineto{\pgfqpoint{3.764778in}{0.902511in}}%
\pgfpathlineto{\pgfqpoint{3.788732in}{0.907257in}}%
\pgfpathlineto{\pgfqpoint{3.812685in}{0.911940in}}%
\pgfpathlineto{\pgfqpoint{3.836638in}{0.916562in}}%
\pgfpathlineto{\pgfqpoint{3.860591in}{0.921121in}}%
\pgfpathlineto{\pgfqpoint{3.884545in}{0.925618in}}%
\pgfpathlineto{\pgfqpoint{3.908498in}{0.930053in}}%
\pgfpathlineto{\pgfqpoint{3.932451in}{0.934425in}}%
\pgfpathlineto{\pgfqpoint{3.956404in}{0.938735in}}%
\pgfpathlineto{\pgfqpoint{3.980358in}{0.942983in}}%
\pgfpathlineto{\pgfqpoint{4.004311in}{0.947169in}}%
\pgfpathlineto{\pgfqpoint{4.028264in}{0.951293in}}%
\pgfpathlineto{\pgfqpoint{4.052217in}{0.955355in}}%
\pgfpathlineto{\pgfqpoint{4.076170in}{0.959354in}}%
\pgfpathlineto{\pgfqpoint{4.100124in}{0.963291in}}%
\pgfpathlineto{\pgfqpoint{4.124077in}{0.967166in}}%
\pgfpathlineto{\pgfqpoint{4.148030in}{0.970979in}}%
\pgfpathlineto{\pgfqpoint{4.171983in}{0.974729in}}%
\pgfpathlineto{\pgfqpoint{4.195937in}{0.978417in}}%
\pgfpathlineto{\pgfqpoint{4.219890in}{0.982043in}}%
\pgfpathlineto{\pgfqpoint{4.243843in}{0.985607in}}%
\pgfpathlineto{\pgfqpoint{4.267796in}{0.989109in}}%
\pgfpathlineto{\pgfqpoint{4.291750in}{0.992548in}}%
\pgfpathlineto{\pgfqpoint{4.315703in}{0.995926in}}%
\pgfpathlineto{\pgfqpoint{4.339656in}{0.999241in}}%
\pgfpathlineto{\pgfqpoint{4.363609in}{1.002494in}}%
\pgfpathlineto{\pgfqpoint{4.387563in}{1.005684in}}%
\pgfpathlineto{\pgfqpoint{4.411516in}{1.008813in}}%
\pgfpathlineto{\pgfqpoint{4.435469in}{1.011879in}}%
\pgfpathlineto{\pgfqpoint{4.459422in}{1.014883in}}%
\pgfpathlineto{\pgfqpoint{4.483375in}{1.017825in}}%
\pgfpathlineto{\pgfqpoint{4.507329in}{1.020705in}}%
\pgfpathlineto{\pgfqpoint{4.531282in}{1.023522in}}%
\pgfpathlineto{\pgfqpoint{4.555235in}{1.026277in}}%
\pgfpathlineto{\pgfqpoint{4.579188in}{1.028971in}}%
\pgfpathlineto{\pgfqpoint{4.603142in}{1.031601in}}%
\pgfpathlineto{\pgfqpoint{4.627095in}{1.034170in}}%
\pgfpathlineto{\pgfqpoint{4.651048in}{1.036677in}}%
\pgfpathlineto{\pgfqpoint{4.675001in}{1.039121in}}%
\pgfpathlineto{\pgfqpoint{4.698955in}{1.041503in}}%
\pgfpathlineto{\pgfqpoint{4.722908in}{1.043823in}}%
\pgfpathlineto{\pgfqpoint{4.746861in}{1.046080in}}%
\pgfpathlineto{\pgfqpoint{4.770814in}{1.048276in}}%
\pgfpathlineto{\pgfqpoint{4.794768in}{1.050409in}}%
\pgfpathlineto{\pgfqpoint{4.818721in}{1.052480in}}%
\pgfpathlineto{\pgfqpoint{4.842674in}{1.054489in}}%
\pgfpathlineto{\pgfqpoint{4.866627in}{1.056436in}}%
\pgfpathlineto{\pgfqpoint{4.890580in}{1.058320in}}%
\pgfpathlineto{\pgfqpoint{4.914534in}{1.060142in}}%
\pgfpathlineto{\pgfqpoint{4.938487in}{1.061902in}}%
\pgfpathlineto{\pgfqpoint{4.962440in}{1.063600in}}%
\pgfpathlineto{\pgfqpoint{4.986393in}{1.065236in}}%
\pgfpathlineto{\pgfqpoint{5.010347in}{1.066809in}}%
\pgfpathlineto{\pgfqpoint{5.034300in}{1.068320in}}%
\pgfpathlineto{\pgfqpoint{5.058253in}{1.069769in}}%
\pgfpathlineto{\pgfqpoint{5.082206in}{1.071156in}}%
\pgfpathlineto{\pgfqpoint{5.106160in}{1.072481in}}%
\pgfpathlineto{\pgfqpoint{5.130113in}{1.073743in}}%
\pgfpathlineto{\pgfqpoint{5.154066in}{1.074944in}}%
\pgfpathlineto{\pgfqpoint{5.178019in}{1.076082in}}%
\pgfpathlineto{\pgfqpoint{5.201973in}{1.077157in}}%
\pgfpathlineto{\pgfqpoint{5.225926in}{1.078171in}}%
\pgfpathlineto{\pgfqpoint{5.249879in}{1.079122in}}%
\pgfpathlineto{\pgfqpoint{5.273832in}{1.080012in}}%
\pgfpathlineto{\pgfqpoint{5.297785in}{1.080839in}}%
\pgfpathlineto{\pgfqpoint{5.321739in}{1.081604in}}%
\pgfpathlineto{\pgfqpoint{5.345692in}{1.082306in}}%
\pgfpathlineto{\pgfqpoint{5.369645in}{1.082947in}}%
\pgfpathlineto{\pgfqpoint{5.393598in}{1.083525in}}%
\pgfpathlineto{\pgfqpoint{5.417552in}{1.084041in}}%
\pgfpathlineto{\pgfqpoint{5.441505in}{1.084495in}}%
\pgfpathlineto{\pgfqpoint{5.465458in}{1.084886in}}%
\pgfpathlineto{\pgfqpoint{5.489411in}{1.085216in}}%
\pgfpathlineto{\pgfqpoint{5.513365in}{1.085483in}}%
\pgfpathlineto{\pgfqpoint{5.537318in}{1.085688in}}%
\pgfpathlineto{\pgfqpoint{5.561271in}{1.085831in}}%
\pgfpathlineto{\pgfqpoint{5.585224in}{1.085911in}}%
\pgfpathlineto{\pgfqpoint{5.609178in}{1.085930in}}%
\pgfpathlineto{\pgfqpoint{5.633131in}{1.085886in}}%
\pgfpathlineto{\pgfqpoint{5.657084in}{1.085780in}}%
\pgfpathlineto{\pgfqpoint{5.681037in}{1.085612in}}%
\pgfpathlineto{\pgfqpoint{5.704990in}{1.085381in}}%
\pgfpathlineto{\pgfqpoint{5.728944in}{1.085089in}}%
\pgfpathlineto{\pgfqpoint{5.752897in}{1.084734in}}%
\pgfusepath{stroke}%
\end{pgfscope}%
\begin{pgfscope}%
\pgfpathrectangle{\pgfqpoint{3.278424in}{0.516222in}}{\pgfqpoint{2.577576in}{0.913411in}} %
\pgfusepath{clip}%
\pgfsetroundcap%
\pgfsetroundjoin%
\pgfsetlinewidth{0.200750pt}%
\definecolor{currentstroke}{rgb}{0.125490,0.290196,0.529412}%
\pgfsetstrokecolor{currentstroke}%
\pgfsetdash{}{0pt}%
\pgfpathmoveto{\pgfqpoint{3.381527in}{0.893480in}}%
\pgfpathlineto{\pgfqpoint{3.405480in}{0.896086in}}%
\pgfpathlineto{\pgfqpoint{3.429433in}{0.898767in}}%
\pgfpathlineto{\pgfqpoint{3.453386in}{0.901526in}}%
\pgfpathlineto{\pgfqpoint{3.477340in}{0.904369in}}%
\pgfpathlineto{\pgfqpoint{3.501293in}{0.907301in}}%
\pgfpathlineto{\pgfqpoint{3.525246in}{0.910327in}}%
\pgfpathlineto{\pgfqpoint{3.549199in}{0.913452in}}%
\pgfpathlineto{\pgfqpoint{3.573153in}{0.916681in}}%
\pgfpathlineto{\pgfqpoint{3.597106in}{0.920019in}}%
\pgfpathlineto{\pgfqpoint{3.621059in}{0.923469in}}%
\pgfpathlineto{\pgfqpoint{3.645012in}{0.927034in}}%
\pgfpathlineto{\pgfqpoint{3.668965in}{0.930717in}}%
\pgfpathlineto{\pgfqpoint{3.692919in}{0.934517in}}%
\pgfpathlineto{\pgfqpoint{3.716872in}{0.938434in}}%
\pgfpathlineto{\pgfqpoint{3.740825in}{0.942464in}}%
\pgfpathlineto{\pgfqpoint{3.764778in}{0.946602in}}%
\pgfpathlineto{\pgfqpoint{3.788732in}{0.950842in}}%
\pgfpathlineto{\pgfqpoint{3.812685in}{0.955175in}}%
\pgfpathlineto{\pgfqpoint{3.836638in}{0.959591in}}%
\pgfpathlineto{\pgfqpoint{3.860591in}{0.964078in}}%
\pgfpathlineto{\pgfqpoint{3.884545in}{0.968625in}}%
\pgfpathlineto{\pgfqpoint{3.908498in}{0.973217in}}%
\pgfpathlineto{\pgfqpoint{3.932451in}{0.977842in}}%
\pgfpathlineto{\pgfqpoint{3.956404in}{0.982486in}}%
\pgfpathlineto{\pgfqpoint{3.980358in}{0.987138in}}%
\pgfpathlineto{\pgfqpoint{4.004311in}{0.991784in}}%
\pgfpathlineto{\pgfqpoint{4.028264in}{0.996413in}}%
\pgfpathlineto{\pgfqpoint{4.052217in}{1.001015in}}%
\pgfpathlineto{\pgfqpoint{4.076170in}{1.005579in}}%
\pgfpathlineto{\pgfqpoint{4.100124in}{1.010096in}}%
\pgfpathlineto{\pgfqpoint{4.124077in}{1.014558in}}%
\pgfpathlineto{\pgfqpoint{4.148030in}{1.018958in}}%
\pgfpathlineto{\pgfqpoint{4.171983in}{1.023289in}}%
\pgfpathlineto{\pgfqpoint{4.195937in}{1.027546in}}%
\pgfpathlineto{\pgfqpoint{4.219890in}{1.031721in}}%
\pgfpathlineto{\pgfqpoint{4.243843in}{1.035812in}}%
\pgfpathlineto{\pgfqpoint{4.267796in}{1.039814in}}%
\pgfpathlineto{\pgfqpoint{4.291750in}{1.043723in}}%
\pgfpathlineto{\pgfqpoint{4.315703in}{1.047536in}}%
\pgfpathlineto{\pgfqpoint{4.339656in}{1.051251in}}%
\pgfpathlineto{\pgfqpoint{4.363609in}{1.054865in}}%
\pgfpathlineto{\pgfqpoint{4.387563in}{1.058376in}}%
\pgfpathlineto{\pgfqpoint{4.411516in}{1.061783in}}%
\pgfpathlineto{\pgfqpoint{4.435469in}{1.065083in}}%
\pgfpathlineto{\pgfqpoint{4.459422in}{1.068277in}}%
\pgfpathlineto{\pgfqpoint{4.483375in}{1.071363in}}%
\pgfpathlineto{\pgfqpoint{4.507329in}{1.074342in}}%
\pgfpathlineto{\pgfqpoint{4.531282in}{1.077211in}}%
\pgfpathlineto{\pgfqpoint{4.555235in}{1.079972in}}%
\pgfpathlineto{\pgfqpoint{4.579188in}{1.082626in}}%
\pgfpathlineto{\pgfqpoint{4.603142in}{1.085171in}}%
\pgfpathlineto{\pgfqpoint{4.627095in}{1.087610in}}%
\pgfpathlineto{\pgfqpoint{4.651048in}{1.089942in}}%
\pgfpathlineto{\pgfqpoint{4.675001in}{1.092170in}}%
\pgfpathlineto{\pgfqpoint{4.698955in}{1.094296in}}%
\pgfpathlineto{\pgfqpoint{4.722908in}{1.096320in}}%
\pgfpathlineto{\pgfqpoint{4.746861in}{1.098245in}}%
\pgfpathlineto{\pgfqpoint{4.770814in}{1.100074in}}%
\pgfpathlineto{\pgfqpoint{4.794768in}{1.101809in}}%
\pgfpathlineto{\pgfqpoint{4.818721in}{1.103455in}}%
\pgfpathlineto{\pgfqpoint{4.842674in}{1.105015in}}%
\pgfpathlineto{\pgfqpoint{4.866627in}{1.106493in}}%
\pgfpathlineto{\pgfqpoint{4.890580in}{1.107894in}}%
\pgfpathlineto{\pgfqpoint{4.914534in}{1.109224in}}%
\pgfpathlineto{\pgfqpoint{4.938487in}{1.110489in}}%
\pgfpathlineto{\pgfqpoint{4.962440in}{1.111695in}}%
\pgfpathlineto{\pgfqpoint{4.986393in}{1.112850in}}%
\pgfpathlineto{\pgfqpoint{5.010347in}{1.113962in}}%
\pgfpathlineto{\pgfqpoint{5.034300in}{1.115040in}}%
\pgfpathlineto{\pgfqpoint{5.058253in}{1.116094in}}%
\pgfpathlineto{\pgfqpoint{5.082206in}{1.117134in}}%
\pgfpathlineto{\pgfqpoint{5.106160in}{1.118171in}}%
\pgfpathlineto{\pgfqpoint{5.130113in}{1.119216in}}%
\pgfpathlineto{\pgfqpoint{5.154066in}{1.120281in}}%
\pgfpathlineto{\pgfqpoint{5.178019in}{1.121377in}}%
\pgfpathlineto{\pgfqpoint{5.201973in}{1.122518in}}%
\pgfpathlineto{\pgfqpoint{5.225926in}{1.123714in}}%
\pgfpathlineto{\pgfqpoint{5.249879in}{1.124976in}}%
\pgfpathlineto{\pgfqpoint{5.273832in}{1.126315in}}%
\pgfpathlineto{\pgfqpoint{5.297785in}{1.127738in}}%
\pgfpathlineto{\pgfqpoint{5.321739in}{1.129255in}}%
\pgfpathlineto{\pgfqpoint{5.345692in}{1.130871in}}%
\pgfpathlineto{\pgfqpoint{5.369645in}{1.132591in}}%
\pgfpathlineto{\pgfqpoint{5.393598in}{1.134417in}}%
\pgfpathlineto{\pgfqpoint{5.417552in}{1.136350in}}%
\pgfpathlineto{\pgfqpoint{5.441505in}{1.138392in}}%
\pgfpathlineto{\pgfqpoint{5.465458in}{1.140540in}}%
\pgfpathlineto{\pgfqpoint{5.489411in}{1.142793in}}%
\pgfpathlineto{\pgfqpoint{5.513365in}{1.145148in}}%
\pgfpathlineto{\pgfqpoint{5.537318in}{1.147600in}}%
\pgfpathlineto{\pgfqpoint{5.561271in}{1.150147in}}%
\pgfpathlineto{\pgfqpoint{5.585224in}{1.152783in}}%
\pgfpathlineto{\pgfqpoint{5.609178in}{1.155505in}}%
\pgfpathlineto{\pgfqpoint{5.633131in}{1.158307in}}%
\pgfpathlineto{\pgfqpoint{5.657084in}{1.161187in}}%
\pgfpathlineto{\pgfqpoint{5.681037in}{1.164140in}}%
\pgfpathlineto{\pgfqpoint{5.704990in}{1.167161in}}%
\pgfpathlineto{\pgfqpoint{5.728944in}{1.170247in}}%
\pgfpathlineto{\pgfqpoint{5.752897in}{1.173395in}}%
\pgfusepath{stroke}%
\end{pgfscope}%
\begin{pgfscope}%
\pgfpathrectangle{\pgfqpoint{3.278424in}{0.516222in}}{\pgfqpoint{2.577576in}{0.913411in}} %
\pgfusepath{clip}%
\pgfsetroundcap%
\pgfsetroundjoin%
\pgfsetlinewidth{0.200750pt}%
\definecolor{currentstroke}{rgb}{0.125490,0.290196,0.529412}%
\pgfsetstrokecolor{currentstroke}%
\pgfsetdash{}{0pt}%
\pgfpathmoveto{\pgfqpoint{3.381527in}{0.742761in}}%
\pgfpathlineto{\pgfqpoint{3.405480in}{0.751636in}}%
\pgfpathlineto{\pgfqpoint{3.429433in}{0.760313in}}%
\pgfpathlineto{\pgfqpoint{3.453386in}{0.768787in}}%
\pgfpathlineto{\pgfqpoint{3.477340in}{0.777053in}}%
\pgfpathlineto{\pgfqpoint{3.501293in}{0.785105in}}%
\pgfpathlineto{\pgfqpoint{3.525246in}{0.792939in}}%
\pgfpathlineto{\pgfqpoint{3.549199in}{0.800549in}}%
\pgfpathlineto{\pgfqpoint{3.573153in}{0.807931in}}%
\pgfpathlineto{\pgfqpoint{3.597106in}{0.815080in}}%
\pgfpathlineto{\pgfqpoint{3.621059in}{0.821992in}}%
\pgfpathlineto{\pgfqpoint{3.645012in}{0.828665in}}%
\pgfpathlineto{\pgfqpoint{3.668965in}{0.835096in}}%
\pgfpathlineto{\pgfqpoint{3.692919in}{0.841284in}}%
\pgfpathlineto{\pgfqpoint{3.716872in}{0.847232in}}%
\pgfpathlineto{\pgfqpoint{3.740825in}{0.852943in}}%
\pgfpathlineto{\pgfqpoint{3.764778in}{0.858420in}}%
\pgfpathlineto{\pgfqpoint{3.788732in}{0.863671in}}%
\pgfpathlineto{\pgfqpoint{3.812685in}{0.868705in}}%
\pgfpathlineto{\pgfqpoint{3.836638in}{0.873532in}}%
\pgfpathlineto{\pgfqpoint{3.860591in}{0.878163in}}%
\pgfpathlineto{\pgfqpoint{3.884545in}{0.882611in}}%
\pgfpathlineto{\pgfqpoint{3.908498in}{0.886888in}}%
\pgfpathlineto{\pgfqpoint{3.932451in}{0.891008in}}%
\pgfpathlineto{\pgfqpoint{3.956404in}{0.894984in}}%
\pgfpathlineto{\pgfqpoint{3.980358in}{0.898829in}}%
\pgfpathlineto{\pgfqpoint{4.004311in}{0.902555in}}%
\pgfpathlineto{\pgfqpoint{4.028264in}{0.906173in}}%
\pgfpathlineto{\pgfqpoint{4.052217in}{0.909694in}}%
\pgfpathlineto{\pgfqpoint{4.076170in}{0.913129in}}%
\pgfpathlineto{\pgfqpoint{4.100124in}{0.916486in}}%
\pgfpathlineto{\pgfqpoint{4.124077in}{0.919773in}}%
\pgfpathlineto{\pgfqpoint{4.148030in}{0.922999in}}%
\pgfpathlineto{\pgfqpoint{4.171983in}{0.926169in}}%
\pgfpathlineto{\pgfqpoint{4.195937in}{0.929289in}}%
\pgfpathlineto{\pgfqpoint{4.219890in}{0.932365in}}%
\pgfpathlineto{\pgfqpoint{4.243843in}{0.935402in}}%
\pgfpathlineto{\pgfqpoint{4.267796in}{0.938404in}}%
\pgfpathlineto{\pgfqpoint{4.291750in}{0.941374in}}%
\pgfpathlineto{\pgfqpoint{4.315703in}{0.944315in}}%
\pgfpathlineto{\pgfqpoint{4.339656in}{0.947231in}}%
\pgfpathlineto{\pgfqpoint{4.363609in}{0.950122in}}%
\pgfpathlineto{\pgfqpoint{4.387563in}{0.952993in}}%
\pgfpathlineto{\pgfqpoint{4.411516in}{0.955843in}}%
\pgfpathlineto{\pgfqpoint{4.435469in}{0.958675in}}%
\pgfpathlineto{\pgfqpoint{4.459422in}{0.961489in}}%
\pgfpathlineto{\pgfqpoint{4.483375in}{0.964287in}}%
\pgfpathlineto{\pgfqpoint{4.507329in}{0.967068in}}%
\pgfpathlineto{\pgfqpoint{4.531282in}{0.969833in}}%
\pgfpathlineto{\pgfqpoint{4.555235in}{0.972582in}}%
\pgfpathlineto{\pgfqpoint{4.579188in}{0.975315in}}%
\pgfpathlineto{\pgfqpoint{4.603142in}{0.978032in}}%
\pgfpathlineto{\pgfqpoint{4.627095in}{0.980730in}}%
\pgfpathlineto{\pgfqpoint{4.651048in}{0.983411in}}%
\pgfpathlineto{\pgfqpoint{4.675001in}{0.986071in}}%
\pgfpathlineto{\pgfqpoint{4.698955in}{0.988710in}}%
\pgfpathlineto{\pgfqpoint{4.722908in}{0.991326in}}%
\pgfpathlineto{\pgfqpoint{4.746861in}{0.993916in}}%
\pgfpathlineto{\pgfqpoint{4.770814in}{0.996478in}}%
\pgfpathlineto{\pgfqpoint{4.794768in}{0.999009in}}%
\pgfpathlineto{\pgfqpoint{4.818721in}{1.001505in}}%
\pgfpathlineto{\pgfqpoint{4.842674in}{1.003963in}}%
\pgfpathlineto{\pgfqpoint{4.866627in}{1.006378in}}%
\pgfpathlineto{\pgfqpoint{4.890580in}{1.008746in}}%
\pgfpathlineto{\pgfqpoint{4.914534in}{1.011061in}}%
\pgfpathlineto{\pgfqpoint{4.938487in}{1.013316in}}%
\pgfpathlineto{\pgfqpoint{4.962440in}{1.015506in}}%
\pgfpathlineto{\pgfqpoint{4.986393in}{1.017622in}}%
\pgfpathlineto{\pgfqpoint{5.010347in}{1.019656in}}%
\pgfpathlineto{\pgfqpoint{5.034300in}{1.021600in}}%
\pgfpathlineto{\pgfqpoint{5.058253in}{1.023445in}}%
\pgfpathlineto{\pgfqpoint{5.082206in}{1.025178in}}%
\pgfpathlineto{\pgfqpoint{5.106160in}{1.026791in}}%
\pgfpathlineto{\pgfqpoint{5.130113in}{1.028271in}}%
\pgfpathlineto{\pgfqpoint{5.154066in}{1.029607in}}%
\pgfpathlineto{\pgfqpoint{5.178019in}{1.030786in}}%
\pgfpathlineto{\pgfqpoint{5.201973in}{1.031797in}}%
\pgfpathlineto{\pgfqpoint{5.225926in}{1.032628in}}%
\pgfpathlineto{\pgfqpoint{5.249879in}{1.033269in}}%
\pgfpathlineto{\pgfqpoint{5.273832in}{1.033709in}}%
\pgfpathlineto{\pgfqpoint{5.297785in}{1.033939in}}%
\pgfpathlineto{\pgfqpoint{5.321739in}{1.033952in}}%
\pgfpathlineto{\pgfqpoint{5.345692in}{1.033741in}}%
\pgfpathlineto{\pgfqpoint{5.369645in}{1.033302in}}%
\pgfpathlineto{\pgfqpoint{5.393598in}{1.032633in}}%
\pgfpathlineto{\pgfqpoint{5.417552in}{1.031731in}}%
\pgfpathlineto{\pgfqpoint{5.441505in}{1.030597in}}%
\pgfpathlineto{\pgfqpoint{5.465458in}{1.029232in}}%
\pgfpathlineto{\pgfqpoint{5.489411in}{1.027638in}}%
\pgfpathlineto{\pgfqpoint{5.513365in}{1.025818in}}%
\pgfpathlineto{\pgfqpoint{5.537318in}{1.023775in}}%
\pgfpathlineto{\pgfqpoint{5.561271in}{1.021515in}}%
\pgfpathlineto{\pgfqpoint{5.585224in}{1.019039in}}%
\pgfpathlineto{\pgfqpoint{5.609178in}{1.016355in}}%
\pgfpathlineto{\pgfqpoint{5.633131in}{1.013464in}}%
\pgfpathlineto{\pgfqpoint{5.657084in}{1.010372in}}%
\pgfpathlineto{\pgfqpoint{5.681037in}{1.007084in}}%
\pgfpathlineto{\pgfqpoint{5.704990in}{1.003602in}}%
\pgfpathlineto{\pgfqpoint{5.728944in}{0.999930in}}%
\pgfpathlineto{\pgfqpoint{5.752897in}{0.996072in}}%
\pgfusepath{stroke}%
\end{pgfscope}%
\begin{pgfscope}%
\pgfpathrectangle{\pgfqpoint{3.278424in}{0.516222in}}{\pgfqpoint{2.577576in}{0.913411in}} %
\pgfusepath{clip}%
\pgfsetbuttcap%
\pgfsetbeveljoin%
\definecolor{currentfill}{rgb}{0.298039,0.447059,0.690196}%
\pgfsetfillcolor{currentfill}%
\pgfsetlinewidth{0.000000pt}%
\definecolor{currentstroke}{rgb}{0.000000,0.000000,0.000000}%
\pgfsetstrokecolor{currentstroke}%
\pgfsetdash{}{0pt}%
\pgfsys@defobject{currentmarker}{\pgfqpoint{-0.036986in}{-0.031462in}}{\pgfqpoint{0.036986in}{0.038889in}}{%
\pgfpathmoveto{\pgfqpoint{0.000000in}{0.038889in}}%
\pgfpathlineto{\pgfqpoint{-0.008731in}{0.012017in}}%
\pgfpathlineto{\pgfqpoint{-0.036986in}{0.012017in}}%
\pgfpathlineto{\pgfqpoint{-0.014127in}{-0.004590in}}%
\pgfpathlineto{\pgfqpoint{-0.022858in}{-0.031462in}}%
\pgfpathlineto{\pgfqpoint{-0.000000in}{-0.014854in}}%
\pgfpathlineto{\pgfqpoint{0.022858in}{-0.031462in}}%
\pgfpathlineto{\pgfqpoint{0.014127in}{-0.004590in}}%
\pgfpathlineto{\pgfqpoint{0.036986in}{0.012017in}}%
\pgfpathlineto{\pgfqpoint{0.008731in}{0.012017in}}%
\pgfpathclose%
\pgfusepath{fill}%
}%
\begin{pgfscope}%
\pgfsys@transformshift{4.515660in}{1.079492in}%
\pgfsys@useobject{currentmarker}{}%
\end{pgfscope}%
\begin{pgfscope}%
\pgfsys@transformshift{5.340485in}{0.957704in}%
\pgfsys@useobject{currentmarker}{}%
\end{pgfscope}%
\begin{pgfscope}%
\pgfsys@transformshift{4.773418in}{1.064269in}%
\pgfsys@useobject{currentmarker}{}%
\end{pgfscope}%
\begin{pgfscope}%
\pgfsys@transformshift{5.443588in}{1.262175in}%
\pgfsys@useobject{currentmarker}{}%
\end{pgfscope}%
\begin{pgfscope}%
\pgfsys@transformshift{5.288933in}{0.942481in}%
\pgfsys@useobject{currentmarker}{}%
\end{pgfscope}%
\begin{pgfscope}%
\pgfsys@transformshift{5.598242in}{1.323069in}%
\pgfsys@useobject{currentmarker}{}%
\end{pgfscope}%
\begin{pgfscope}%
\pgfsys@transformshift{4.309454in}{0.942481in}%
\pgfsys@useobject{currentmarker}{}%
\end{pgfscope}%
\begin{pgfscope}%
\pgfsys@transformshift{5.185830in}{1.383963in}%
\pgfsys@useobject{currentmarker}{}%
\end{pgfscope}%
\begin{pgfscope}%
\pgfsys@transformshift{5.031176in}{0.683681in}%
\pgfsys@useobject{currentmarker}{}%
\end{pgfscope}%
\begin{pgfscope}%
\pgfsys@transformshift{4.773418in}{0.972928in}%
\pgfsys@useobject{currentmarker}{}%
\end{pgfscope}%
\begin{pgfscope}%
\pgfsys@transformshift{3.381527in}{0.927257in}%
\pgfsys@useobject{currentmarker}{}%
\end{pgfscope}%
\begin{pgfscope}%
\pgfsys@transformshift{5.392036in}{0.942481in}%
\pgfsys@useobject{currentmarker}{}%
\end{pgfscope}%
\begin{pgfscope}%
\pgfsys@transformshift{4.670315in}{0.988151in}%
\pgfsys@useobject{currentmarker}{}%
\end{pgfscope}%
\begin{pgfscope}%
\pgfsys@transformshift{3.484630in}{0.622787in}%
\pgfsys@useobject{currentmarker}{}%
\end{pgfscope}%
\begin{pgfscope}%
\pgfsys@transformshift{5.752897in}{1.155610in}%
\pgfsys@useobject{currentmarker}{}%
\end{pgfscope}%
\begin{pgfscope}%
\pgfsys@transformshift{4.824969in}{1.109939in}%
\pgfsys@useobject{currentmarker}{}%
\end{pgfscope}%
\begin{pgfscope}%
\pgfsys@transformshift{4.464109in}{0.866363in}%
\pgfsys@useobject{currentmarker}{}%
\end{pgfscope}%
\begin{pgfscope}%
\pgfsys@transformshift{3.381527in}{0.957704in}%
\pgfsys@useobject{currentmarker}{}%
\end{pgfscope}%
\begin{pgfscope}%
\pgfsys@transformshift{4.103248in}{0.744575in}%
\pgfsys@useobject{currentmarker}{}%
\end{pgfscope}%
\begin{pgfscope}%
\pgfsys@transformshift{4.257903in}{0.881587in}%
\pgfsys@useobject{currentmarker}{}%
\end{pgfscope}%
\begin{pgfscope}%
\pgfsys@transformshift{4.876521in}{0.957704in}%
\pgfsys@useobject{currentmarker}{}%
\end{pgfscope}%
\begin{pgfscope}%
\pgfsys@transformshift{3.845490in}{1.170834in}%
\pgfsys@useobject{currentmarker}{}%
\end{pgfscope}%
\begin{pgfscope}%
\pgfsys@transformshift{5.134279in}{1.003375in}%
\pgfsys@useobject{currentmarker}{}%
\end{pgfscope}%
\begin{pgfscope}%
\pgfsys@transformshift{3.793939in}{0.790246in}%
\pgfsys@useobject{currentmarker}{}%
\end{pgfscope}%
\begin{pgfscope}%
\pgfsys@transformshift{4.361006in}{1.049045in}%
\pgfsys@useobject{currentmarker}{}%
\end{pgfscope}%
\begin{pgfscope}%
\pgfsys@transformshift{4.721866in}{1.246951in}%
\pgfsys@useobject{currentmarker}{}%
\end{pgfscope}%
\begin{pgfscope}%
\pgfsys@transformshift{4.257903in}{0.805469in}%
\pgfsys@useobject{currentmarker}{}%
\end{pgfscope}%
\begin{pgfscope}%
\pgfsys@transformshift{4.154800in}{0.698904in}%
\pgfsys@useobject{currentmarker}{}%
\end{pgfscope}%
\begin{pgfscope}%
\pgfsys@transformshift{4.309454in}{0.759799in}%
\pgfsys@useobject{currentmarker}{}%
\end{pgfscope}%
\begin{pgfscope}%
\pgfsys@transformshift{5.340485in}{1.262175in}%
\pgfsys@useobject{currentmarker}{}%
\end{pgfscope}%
\end{pgfscope}%
\begin{pgfscope}%
\pgfsetrectcap%
\pgfsetmiterjoin%
\pgfsetlinewidth{0.000000pt}%
\definecolor{currentstroke}{rgb}{1.000000,1.000000,1.000000}%
\pgfsetstrokecolor{currentstroke}%
\pgfsetdash{}{0pt}%
\pgfpathmoveto{\pgfqpoint{5.856000in}{0.516222in}}%
\pgfpathlineto{\pgfqpoint{5.856000in}{1.429633in}}%
\pgfusepath{}%
\end{pgfscope}%
\begin{pgfscope}%
\pgfsetrectcap%
\pgfsetmiterjoin%
\pgfsetlinewidth{0.000000pt}%
\definecolor{currentstroke}{rgb}{1.000000,1.000000,1.000000}%
\pgfsetstrokecolor{currentstroke}%
\pgfsetdash{}{0pt}%
\pgfpathmoveto{\pgfqpoint{3.278424in}{0.516222in}}%
\pgfpathlineto{\pgfqpoint{5.856000in}{0.516222in}}%
\pgfusepath{}%
\end{pgfscope}%
\begin{pgfscope}%
\pgfsetrectcap%
\pgfsetmiterjoin%
\pgfsetlinewidth{0.000000pt}%
\definecolor{currentstroke}{rgb}{1.000000,1.000000,1.000000}%
\pgfsetstrokecolor{currentstroke}%
\pgfsetdash{}{0pt}%
\pgfpathmoveto{\pgfqpoint{3.278424in}{0.516222in}}%
\pgfpathlineto{\pgfqpoint{3.278424in}{1.429633in}}%
\pgfusepath{}%
\end{pgfscope}%
\begin{pgfscope}%
\pgfsetrectcap%
\pgfsetmiterjoin%
\pgfsetlinewidth{0.000000pt}%
\definecolor{currentstroke}{rgb}{1.000000,1.000000,1.000000}%
\pgfsetstrokecolor{currentstroke}%
\pgfsetdash{}{0pt}%
\pgfpathmoveto{\pgfqpoint{3.278424in}{1.429633in}}%
\pgfpathlineto{\pgfqpoint{5.856000in}{1.429633in}}%
\pgfusepath{}%
\end{pgfscope}%
\end{pgfpicture}%
\makeatother%
\endgroup%

    \caption{Comparison between the times measured in the two realizations.}
    \label{fig_badlr}
  \end{figure}
\end{comment}
Several functions were tried as basis. The ones with more effect are the
constant function ($1$), the linear effect on all the variables ($x$), the
cosine effect on the wing length ($\cos(x_0)$). However the other functions also
contribute to improve $R^2$.

\paragraph{P-values.} The contribution of the basis functions to the linear
regression are confirmed by the following vector of p-values: $\langle 0.0,
0.0, 0.0, 0.0, 0.02, .01, 0.01, 0.0 \rangle$ in summary there is a
high probability that all the basis functions contribute to the regression as
each value is close to $0.0$.

\paragraph{Influence of physical quantities.}


\paragraph{Region of interest for optimization.}

\subsection{Gaussian process regression}

\paragraph{Kernel names.} In the first point we are presented a several kernel
implementations and asked to identify them. The first two implementations are
actually the same and both represent the \emph{Matern32} kernel. The third is
the implementation of the constant kernel. The fourth implementation is for the
\emph{Brownian} kernel. The last implementation corresponds to the
\emph{Gaussian} kernel.

\paragraph{Sampling from a Gaussian process and plots.}
The sampling function is completed by evaluating the provided kernel on the
input data. After that $n$ random samples are taken from a multivariate normal
with the provided mean ($\mu$) and the kernel as the covariance parameter.
\Cref{fig_gpsamples} presents samplings of the same kernel with different
parameters. The plot on the left uses $\theta=0.8$ which leads to a wider
``waves''.
\begin{figure}
  \centering
  %% Creator: Matplotlib, PGF backend
%%
%% To include the figure in your LaTeX document, write
%%   \input{<filename>.pgf}
%%
%% Make sure the required packages are loaded in your preamble
%%   \usepackage{pgf}
%%
%% Figures using additional raster images can only be included by \input if
%% they are in the same directory as the main LaTeX file. For loading figures
%% from other directories you can use the `import` package
%%   \usepackage{import}
%% and then include the figures with
%%   \import{<path to file>}{<filename>.pgf}
%%
%% Matplotlib used the following preamble
%%   \usepackage[utf8x]{inputenc}
%%   \usepackage[T1]{fontenc}
%%   \usepackage{cmbright}
%%
\begingroup%
\makeatletter%
\begin{pgfpicture}%
\pgfpathrectangle{\pgfpointorigin}{\pgfqpoint{6.000000in}{2.000000in}}%
\pgfusepath{use as bounding box, clip}%
\begin{pgfscope}%
\pgfsetbuttcap%
\pgfsetmiterjoin%
\definecolor{currentfill}{rgb}{1.000000,1.000000,1.000000}%
\pgfsetfillcolor{currentfill}%
\pgfsetlinewidth{0.000000pt}%
\definecolor{currentstroke}{rgb}{1.000000,1.000000,1.000000}%
\pgfsetstrokecolor{currentstroke}%
\pgfsetdash{}{0pt}%
\pgfpathmoveto{\pgfqpoint{0.000000in}{0.000000in}}%
\pgfpathlineto{\pgfqpoint{6.000000in}{0.000000in}}%
\pgfpathlineto{\pgfqpoint{6.000000in}{2.000000in}}%
\pgfpathlineto{\pgfqpoint{0.000000in}{2.000000in}}%
\pgfpathclose%
\pgfusepath{fill}%
\end{pgfscope}%
\begin{pgfscope}%
\pgfsetbuttcap%
\pgfsetmiterjoin%
\definecolor{currentfill}{rgb}{0.917647,0.917647,0.949020}%
\pgfsetfillcolor{currentfill}%
\pgfsetlinewidth{0.000000pt}%
\definecolor{currentstroke}{rgb}{0.000000,0.000000,0.000000}%
\pgfsetstrokecolor{currentstroke}%
\pgfsetstrokeopacity{0.000000}%
\pgfsetdash{}{0pt}%
\pgfpathmoveto{\pgfqpoint{0.750000in}{0.250000in}}%
\pgfpathlineto{\pgfqpoint{2.863636in}{0.250000in}}%
\pgfpathlineto{\pgfqpoint{2.863636in}{1.800000in}}%
\pgfpathlineto{\pgfqpoint{0.750000in}{1.800000in}}%
\pgfpathclose%
\pgfusepath{fill}%
\end{pgfscope}%
\begin{pgfscope}%
\pgfpathrectangle{\pgfqpoint{0.750000in}{0.250000in}}{\pgfqpoint{2.113636in}{1.550000in}} %
\pgfusepath{clip}%
\pgfsetroundcap%
\pgfsetroundjoin%
\pgfsetlinewidth{0.803000pt}%
\definecolor{currentstroke}{rgb}{1.000000,1.000000,1.000000}%
\pgfsetstrokecolor{currentstroke}%
\pgfsetdash{}{0pt}%
\pgfpathmoveto{\pgfqpoint{0.750000in}{0.250000in}}%
\pgfpathlineto{\pgfqpoint{0.750000in}{1.800000in}}%
\pgfusepath{stroke}%
\end{pgfscope}%
\begin{pgfscope}%
\pgfsetbuttcap%
\pgfsetroundjoin%
\definecolor{currentfill}{rgb}{0.150000,0.150000,0.150000}%
\pgfsetfillcolor{currentfill}%
\pgfsetlinewidth{0.803000pt}%
\definecolor{currentstroke}{rgb}{0.150000,0.150000,0.150000}%
\pgfsetstrokecolor{currentstroke}%
\pgfsetdash{}{0pt}%
\pgfsys@defobject{currentmarker}{\pgfqpoint{0.000000in}{0.000000in}}{\pgfqpoint{0.000000in}{0.000000in}}{%
\pgfpathmoveto{\pgfqpoint{0.000000in}{0.000000in}}%
\pgfpathlineto{\pgfqpoint{0.000000in}{0.000000in}}%
\pgfusepath{stroke,fill}%
}%
\begin{pgfscope}%
\pgfsys@transformshift{0.750000in}{0.250000in}%
\pgfsys@useobject{currentmarker}{}%
\end{pgfscope}%
\end{pgfscope}%
\begin{pgfscope}%
\pgfsetbuttcap%
\pgfsetroundjoin%
\definecolor{currentfill}{rgb}{0.150000,0.150000,0.150000}%
\pgfsetfillcolor{currentfill}%
\pgfsetlinewidth{0.803000pt}%
\definecolor{currentstroke}{rgb}{0.150000,0.150000,0.150000}%
\pgfsetstrokecolor{currentstroke}%
\pgfsetdash{}{0pt}%
\pgfsys@defobject{currentmarker}{\pgfqpoint{0.000000in}{0.000000in}}{\pgfqpoint{0.000000in}{0.000000in}}{%
\pgfpathmoveto{\pgfqpoint{0.000000in}{0.000000in}}%
\pgfpathlineto{\pgfqpoint{0.000000in}{0.000000in}}%
\pgfusepath{stroke,fill}%
}%
\begin{pgfscope}%
\pgfsys@transformshift{0.750000in}{1.800000in}%
\pgfsys@useobject{currentmarker}{}%
\end{pgfscope}%
\end{pgfscope}%
\begin{pgfscope}%
\definecolor{textcolor}{rgb}{0.150000,0.150000,0.150000}%
\pgfsetstrokecolor{textcolor}%
\pgfsetfillcolor{textcolor}%
\pgftext[x=0.750000in,y=0.172222in,,top]{\color{textcolor}\sffamily\fontsize{8.000000}{9.600000}\selectfont 3.0}%
\end{pgfscope}%
\begin{pgfscope}%
\pgfpathrectangle{\pgfqpoint{0.750000in}{0.250000in}}{\pgfqpoint{2.113636in}{1.550000in}} %
\pgfusepath{clip}%
\pgfsetroundcap%
\pgfsetroundjoin%
\pgfsetlinewidth{0.803000pt}%
\definecolor{currentstroke}{rgb}{1.000000,1.000000,1.000000}%
\pgfsetstrokecolor{currentstroke}%
\pgfsetdash{}{0pt}%
\pgfpathmoveto{\pgfqpoint{1.014205in}{0.250000in}}%
\pgfpathlineto{\pgfqpoint{1.014205in}{1.800000in}}%
\pgfusepath{stroke}%
\end{pgfscope}%
\begin{pgfscope}%
\pgfsetbuttcap%
\pgfsetroundjoin%
\definecolor{currentfill}{rgb}{0.150000,0.150000,0.150000}%
\pgfsetfillcolor{currentfill}%
\pgfsetlinewidth{0.803000pt}%
\definecolor{currentstroke}{rgb}{0.150000,0.150000,0.150000}%
\pgfsetstrokecolor{currentstroke}%
\pgfsetdash{}{0pt}%
\pgfsys@defobject{currentmarker}{\pgfqpoint{0.000000in}{0.000000in}}{\pgfqpoint{0.000000in}{0.000000in}}{%
\pgfpathmoveto{\pgfqpoint{0.000000in}{0.000000in}}%
\pgfpathlineto{\pgfqpoint{0.000000in}{0.000000in}}%
\pgfusepath{stroke,fill}%
}%
\begin{pgfscope}%
\pgfsys@transformshift{1.014205in}{0.250000in}%
\pgfsys@useobject{currentmarker}{}%
\end{pgfscope}%
\end{pgfscope}%
\begin{pgfscope}%
\pgfsetbuttcap%
\pgfsetroundjoin%
\definecolor{currentfill}{rgb}{0.150000,0.150000,0.150000}%
\pgfsetfillcolor{currentfill}%
\pgfsetlinewidth{0.803000pt}%
\definecolor{currentstroke}{rgb}{0.150000,0.150000,0.150000}%
\pgfsetstrokecolor{currentstroke}%
\pgfsetdash{}{0pt}%
\pgfsys@defobject{currentmarker}{\pgfqpoint{0.000000in}{0.000000in}}{\pgfqpoint{0.000000in}{0.000000in}}{%
\pgfpathmoveto{\pgfqpoint{0.000000in}{0.000000in}}%
\pgfpathlineto{\pgfqpoint{0.000000in}{0.000000in}}%
\pgfusepath{stroke,fill}%
}%
\begin{pgfscope}%
\pgfsys@transformshift{1.014205in}{1.800000in}%
\pgfsys@useobject{currentmarker}{}%
\end{pgfscope}%
\end{pgfscope}%
\begin{pgfscope}%
\definecolor{textcolor}{rgb}{0.150000,0.150000,0.150000}%
\pgfsetstrokecolor{textcolor}%
\pgfsetfillcolor{textcolor}%
\pgftext[x=1.014205in,y=0.172222in,,top]{\color{textcolor}\sffamily\fontsize{8.000000}{9.600000}\selectfont 3.5}%
\end{pgfscope}%
\begin{pgfscope}%
\pgfpathrectangle{\pgfqpoint{0.750000in}{0.250000in}}{\pgfqpoint{2.113636in}{1.550000in}} %
\pgfusepath{clip}%
\pgfsetroundcap%
\pgfsetroundjoin%
\pgfsetlinewidth{0.803000pt}%
\definecolor{currentstroke}{rgb}{1.000000,1.000000,1.000000}%
\pgfsetstrokecolor{currentstroke}%
\pgfsetdash{}{0pt}%
\pgfpathmoveto{\pgfqpoint{1.278409in}{0.250000in}}%
\pgfpathlineto{\pgfqpoint{1.278409in}{1.800000in}}%
\pgfusepath{stroke}%
\end{pgfscope}%
\begin{pgfscope}%
\pgfsetbuttcap%
\pgfsetroundjoin%
\definecolor{currentfill}{rgb}{0.150000,0.150000,0.150000}%
\pgfsetfillcolor{currentfill}%
\pgfsetlinewidth{0.803000pt}%
\definecolor{currentstroke}{rgb}{0.150000,0.150000,0.150000}%
\pgfsetstrokecolor{currentstroke}%
\pgfsetdash{}{0pt}%
\pgfsys@defobject{currentmarker}{\pgfqpoint{0.000000in}{0.000000in}}{\pgfqpoint{0.000000in}{0.000000in}}{%
\pgfpathmoveto{\pgfqpoint{0.000000in}{0.000000in}}%
\pgfpathlineto{\pgfqpoint{0.000000in}{0.000000in}}%
\pgfusepath{stroke,fill}%
}%
\begin{pgfscope}%
\pgfsys@transformshift{1.278409in}{0.250000in}%
\pgfsys@useobject{currentmarker}{}%
\end{pgfscope}%
\end{pgfscope}%
\begin{pgfscope}%
\pgfsetbuttcap%
\pgfsetroundjoin%
\definecolor{currentfill}{rgb}{0.150000,0.150000,0.150000}%
\pgfsetfillcolor{currentfill}%
\pgfsetlinewidth{0.803000pt}%
\definecolor{currentstroke}{rgb}{0.150000,0.150000,0.150000}%
\pgfsetstrokecolor{currentstroke}%
\pgfsetdash{}{0pt}%
\pgfsys@defobject{currentmarker}{\pgfqpoint{0.000000in}{0.000000in}}{\pgfqpoint{0.000000in}{0.000000in}}{%
\pgfpathmoveto{\pgfqpoint{0.000000in}{0.000000in}}%
\pgfpathlineto{\pgfqpoint{0.000000in}{0.000000in}}%
\pgfusepath{stroke,fill}%
}%
\begin{pgfscope}%
\pgfsys@transformshift{1.278409in}{1.800000in}%
\pgfsys@useobject{currentmarker}{}%
\end{pgfscope}%
\end{pgfscope}%
\begin{pgfscope}%
\definecolor{textcolor}{rgb}{0.150000,0.150000,0.150000}%
\pgfsetstrokecolor{textcolor}%
\pgfsetfillcolor{textcolor}%
\pgftext[x=1.278409in,y=0.172222in,,top]{\color{textcolor}\sffamily\fontsize{8.000000}{9.600000}\selectfont 4.0}%
\end{pgfscope}%
\begin{pgfscope}%
\pgfpathrectangle{\pgfqpoint{0.750000in}{0.250000in}}{\pgfqpoint{2.113636in}{1.550000in}} %
\pgfusepath{clip}%
\pgfsetroundcap%
\pgfsetroundjoin%
\pgfsetlinewidth{0.803000pt}%
\definecolor{currentstroke}{rgb}{1.000000,1.000000,1.000000}%
\pgfsetstrokecolor{currentstroke}%
\pgfsetdash{}{0pt}%
\pgfpathmoveto{\pgfqpoint{1.542614in}{0.250000in}}%
\pgfpathlineto{\pgfqpoint{1.542614in}{1.800000in}}%
\pgfusepath{stroke}%
\end{pgfscope}%
\begin{pgfscope}%
\pgfsetbuttcap%
\pgfsetroundjoin%
\definecolor{currentfill}{rgb}{0.150000,0.150000,0.150000}%
\pgfsetfillcolor{currentfill}%
\pgfsetlinewidth{0.803000pt}%
\definecolor{currentstroke}{rgb}{0.150000,0.150000,0.150000}%
\pgfsetstrokecolor{currentstroke}%
\pgfsetdash{}{0pt}%
\pgfsys@defobject{currentmarker}{\pgfqpoint{0.000000in}{0.000000in}}{\pgfqpoint{0.000000in}{0.000000in}}{%
\pgfpathmoveto{\pgfqpoint{0.000000in}{0.000000in}}%
\pgfpathlineto{\pgfqpoint{0.000000in}{0.000000in}}%
\pgfusepath{stroke,fill}%
}%
\begin{pgfscope}%
\pgfsys@transformshift{1.542614in}{0.250000in}%
\pgfsys@useobject{currentmarker}{}%
\end{pgfscope}%
\end{pgfscope}%
\begin{pgfscope}%
\pgfsetbuttcap%
\pgfsetroundjoin%
\definecolor{currentfill}{rgb}{0.150000,0.150000,0.150000}%
\pgfsetfillcolor{currentfill}%
\pgfsetlinewidth{0.803000pt}%
\definecolor{currentstroke}{rgb}{0.150000,0.150000,0.150000}%
\pgfsetstrokecolor{currentstroke}%
\pgfsetdash{}{0pt}%
\pgfsys@defobject{currentmarker}{\pgfqpoint{0.000000in}{0.000000in}}{\pgfqpoint{0.000000in}{0.000000in}}{%
\pgfpathmoveto{\pgfqpoint{0.000000in}{0.000000in}}%
\pgfpathlineto{\pgfqpoint{0.000000in}{0.000000in}}%
\pgfusepath{stroke,fill}%
}%
\begin{pgfscope}%
\pgfsys@transformshift{1.542614in}{1.800000in}%
\pgfsys@useobject{currentmarker}{}%
\end{pgfscope}%
\end{pgfscope}%
\begin{pgfscope}%
\definecolor{textcolor}{rgb}{0.150000,0.150000,0.150000}%
\pgfsetstrokecolor{textcolor}%
\pgfsetfillcolor{textcolor}%
\pgftext[x=1.542614in,y=0.172222in,,top]{\color{textcolor}\sffamily\fontsize{8.000000}{9.600000}\selectfont 4.5}%
\end{pgfscope}%
\begin{pgfscope}%
\pgfpathrectangle{\pgfqpoint{0.750000in}{0.250000in}}{\pgfqpoint{2.113636in}{1.550000in}} %
\pgfusepath{clip}%
\pgfsetroundcap%
\pgfsetroundjoin%
\pgfsetlinewidth{0.803000pt}%
\definecolor{currentstroke}{rgb}{1.000000,1.000000,1.000000}%
\pgfsetstrokecolor{currentstroke}%
\pgfsetdash{}{0pt}%
\pgfpathmoveto{\pgfqpoint{1.806818in}{0.250000in}}%
\pgfpathlineto{\pgfqpoint{1.806818in}{1.800000in}}%
\pgfusepath{stroke}%
\end{pgfscope}%
\begin{pgfscope}%
\pgfsetbuttcap%
\pgfsetroundjoin%
\definecolor{currentfill}{rgb}{0.150000,0.150000,0.150000}%
\pgfsetfillcolor{currentfill}%
\pgfsetlinewidth{0.803000pt}%
\definecolor{currentstroke}{rgb}{0.150000,0.150000,0.150000}%
\pgfsetstrokecolor{currentstroke}%
\pgfsetdash{}{0pt}%
\pgfsys@defobject{currentmarker}{\pgfqpoint{0.000000in}{0.000000in}}{\pgfqpoint{0.000000in}{0.000000in}}{%
\pgfpathmoveto{\pgfqpoint{0.000000in}{0.000000in}}%
\pgfpathlineto{\pgfqpoint{0.000000in}{0.000000in}}%
\pgfusepath{stroke,fill}%
}%
\begin{pgfscope}%
\pgfsys@transformshift{1.806818in}{0.250000in}%
\pgfsys@useobject{currentmarker}{}%
\end{pgfscope}%
\end{pgfscope}%
\begin{pgfscope}%
\pgfsetbuttcap%
\pgfsetroundjoin%
\definecolor{currentfill}{rgb}{0.150000,0.150000,0.150000}%
\pgfsetfillcolor{currentfill}%
\pgfsetlinewidth{0.803000pt}%
\definecolor{currentstroke}{rgb}{0.150000,0.150000,0.150000}%
\pgfsetstrokecolor{currentstroke}%
\pgfsetdash{}{0pt}%
\pgfsys@defobject{currentmarker}{\pgfqpoint{0.000000in}{0.000000in}}{\pgfqpoint{0.000000in}{0.000000in}}{%
\pgfpathmoveto{\pgfqpoint{0.000000in}{0.000000in}}%
\pgfpathlineto{\pgfqpoint{0.000000in}{0.000000in}}%
\pgfusepath{stroke,fill}%
}%
\begin{pgfscope}%
\pgfsys@transformshift{1.806818in}{1.800000in}%
\pgfsys@useobject{currentmarker}{}%
\end{pgfscope}%
\end{pgfscope}%
\begin{pgfscope}%
\definecolor{textcolor}{rgb}{0.150000,0.150000,0.150000}%
\pgfsetstrokecolor{textcolor}%
\pgfsetfillcolor{textcolor}%
\pgftext[x=1.806818in,y=0.172222in,,top]{\color{textcolor}\sffamily\fontsize{8.000000}{9.600000}\selectfont 5.0}%
\end{pgfscope}%
\begin{pgfscope}%
\pgfpathrectangle{\pgfqpoint{0.750000in}{0.250000in}}{\pgfqpoint{2.113636in}{1.550000in}} %
\pgfusepath{clip}%
\pgfsetroundcap%
\pgfsetroundjoin%
\pgfsetlinewidth{0.803000pt}%
\definecolor{currentstroke}{rgb}{1.000000,1.000000,1.000000}%
\pgfsetstrokecolor{currentstroke}%
\pgfsetdash{}{0pt}%
\pgfpathmoveto{\pgfqpoint{2.071023in}{0.250000in}}%
\pgfpathlineto{\pgfqpoint{2.071023in}{1.800000in}}%
\pgfusepath{stroke}%
\end{pgfscope}%
\begin{pgfscope}%
\pgfsetbuttcap%
\pgfsetroundjoin%
\definecolor{currentfill}{rgb}{0.150000,0.150000,0.150000}%
\pgfsetfillcolor{currentfill}%
\pgfsetlinewidth{0.803000pt}%
\definecolor{currentstroke}{rgb}{0.150000,0.150000,0.150000}%
\pgfsetstrokecolor{currentstroke}%
\pgfsetdash{}{0pt}%
\pgfsys@defobject{currentmarker}{\pgfqpoint{0.000000in}{0.000000in}}{\pgfqpoint{0.000000in}{0.000000in}}{%
\pgfpathmoveto{\pgfqpoint{0.000000in}{0.000000in}}%
\pgfpathlineto{\pgfqpoint{0.000000in}{0.000000in}}%
\pgfusepath{stroke,fill}%
}%
\begin{pgfscope}%
\pgfsys@transformshift{2.071023in}{0.250000in}%
\pgfsys@useobject{currentmarker}{}%
\end{pgfscope}%
\end{pgfscope}%
\begin{pgfscope}%
\pgfsetbuttcap%
\pgfsetroundjoin%
\definecolor{currentfill}{rgb}{0.150000,0.150000,0.150000}%
\pgfsetfillcolor{currentfill}%
\pgfsetlinewidth{0.803000pt}%
\definecolor{currentstroke}{rgb}{0.150000,0.150000,0.150000}%
\pgfsetstrokecolor{currentstroke}%
\pgfsetdash{}{0pt}%
\pgfsys@defobject{currentmarker}{\pgfqpoint{0.000000in}{0.000000in}}{\pgfqpoint{0.000000in}{0.000000in}}{%
\pgfpathmoveto{\pgfqpoint{0.000000in}{0.000000in}}%
\pgfpathlineto{\pgfqpoint{0.000000in}{0.000000in}}%
\pgfusepath{stroke,fill}%
}%
\begin{pgfscope}%
\pgfsys@transformshift{2.071023in}{1.800000in}%
\pgfsys@useobject{currentmarker}{}%
\end{pgfscope}%
\end{pgfscope}%
\begin{pgfscope}%
\definecolor{textcolor}{rgb}{0.150000,0.150000,0.150000}%
\pgfsetstrokecolor{textcolor}%
\pgfsetfillcolor{textcolor}%
\pgftext[x=2.071023in,y=0.172222in,,top]{\color{textcolor}\sffamily\fontsize{8.000000}{9.600000}\selectfont 5.5}%
\end{pgfscope}%
\begin{pgfscope}%
\pgfpathrectangle{\pgfqpoint{0.750000in}{0.250000in}}{\pgfqpoint{2.113636in}{1.550000in}} %
\pgfusepath{clip}%
\pgfsetroundcap%
\pgfsetroundjoin%
\pgfsetlinewidth{0.803000pt}%
\definecolor{currentstroke}{rgb}{1.000000,1.000000,1.000000}%
\pgfsetstrokecolor{currentstroke}%
\pgfsetdash{}{0pt}%
\pgfpathmoveto{\pgfqpoint{2.335227in}{0.250000in}}%
\pgfpathlineto{\pgfqpoint{2.335227in}{1.800000in}}%
\pgfusepath{stroke}%
\end{pgfscope}%
\begin{pgfscope}%
\pgfsetbuttcap%
\pgfsetroundjoin%
\definecolor{currentfill}{rgb}{0.150000,0.150000,0.150000}%
\pgfsetfillcolor{currentfill}%
\pgfsetlinewidth{0.803000pt}%
\definecolor{currentstroke}{rgb}{0.150000,0.150000,0.150000}%
\pgfsetstrokecolor{currentstroke}%
\pgfsetdash{}{0pt}%
\pgfsys@defobject{currentmarker}{\pgfqpoint{0.000000in}{0.000000in}}{\pgfqpoint{0.000000in}{0.000000in}}{%
\pgfpathmoveto{\pgfqpoint{0.000000in}{0.000000in}}%
\pgfpathlineto{\pgfqpoint{0.000000in}{0.000000in}}%
\pgfusepath{stroke,fill}%
}%
\begin{pgfscope}%
\pgfsys@transformshift{2.335227in}{0.250000in}%
\pgfsys@useobject{currentmarker}{}%
\end{pgfscope}%
\end{pgfscope}%
\begin{pgfscope}%
\pgfsetbuttcap%
\pgfsetroundjoin%
\definecolor{currentfill}{rgb}{0.150000,0.150000,0.150000}%
\pgfsetfillcolor{currentfill}%
\pgfsetlinewidth{0.803000pt}%
\definecolor{currentstroke}{rgb}{0.150000,0.150000,0.150000}%
\pgfsetstrokecolor{currentstroke}%
\pgfsetdash{}{0pt}%
\pgfsys@defobject{currentmarker}{\pgfqpoint{0.000000in}{0.000000in}}{\pgfqpoint{0.000000in}{0.000000in}}{%
\pgfpathmoveto{\pgfqpoint{0.000000in}{0.000000in}}%
\pgfpathlineto{\pgfqpoint{0.000000in}{0.000000in}}%
\pgfusepath{stroke,fill}%
}%
\begin{pgfscope}%
\pgfsys@transformshift{2.335227in}{1.800000in}%
\pgfsys@useobject{currentmarker}{}%
\end{pgfscope}%
\end{pgfscope}%
\begin{pgfscope}%
\definecolor{textcolor}{rgb}{0.150000,0.150000,0.150000}%
\pgfsetstrokecolor{textcolor}%
\pgfsetfillcolor{textcolor}%
\pgftext[x=2.335227in,y=0.172222in,,top]{\color{textcolor}\sffamily\fontsize{8.000000}{9.600000}\selectfont 6.0}%
\end{pgfscope}%
\begin{pgfscope}%
\pgfpathrectangle{\pgfqpoint{0.750000in}{0.250000in}}{\pgfqpoint{2.113636in}{1.550000in}} %
\pgfusepath{clip}%
\pgfsetroundcap%
\pgfsetroundjoin%
\pgfsetlinewidth{0.803000pt}%
\definecolor{currentstroke}{rgb}{1.000000,1.000000,1.000000}%
\pgfsetstrokecolor{currentstroke}%
\pgfsetdash{}{0pt}%
\pgfpathmoveto{\pgfqpoint{2.599432in}{0.250000in}}%
\pgfpathlineto{\pgfqpoint{2.599432in}{1.800000in}}%
\pgfusepath{stroke}%
\end{pgfscope}%
\begin{pgfscope}%
\pgfsetbuttcap%
\pgfsetroundjoin%
\definecolor{currentfill}{rgb}{0.150000,0.150000,0.150000}%
\pgfsetfillcolor{currentfill}%
\pgfsetlinewidth{0.803000pt}%
\definecolor{currentstroke}{rgb}{0.150000,0.150000,0.150000}%
\pgfsetstrokecolor{currentstroke}%
\pgfsetdash{}{0pt}%
\pgfsys@defobject{currentmarker}{\pgfqpoint{0.000000in}{0.000000in}}{\pgfqpoint{0.000000in}{0.000000in}}{%
\pgfpathmoveto{\pgfqpoint{0.000000in}{0.000000in}}%
\pgfpathlineto{\pgfqpoint{0.000000in}{0.000000in}}%
\pgfusepath{stroke,fill}%
}%
\begin{pgfscope}%
\pgfsys@transformshift{2.599432in}{0.250000in}%
\pgfsys@useobject{currentmarker}{}%
\end{pgfscope}%
\end{pgfscope}%
\begin{pgfscope}%
\pgfsetbuttcap%
\pgfsetroundjoin%
\definecolor{currentfill}{rgb}{0.150000,0.150000,0.150000}%
\pgfsetfillcolor{currentfill}%
\pgfsetlinewidth{0.803000pt}%
\definecolor{currentstroke}{rgb}{0.150000,0.150000,0.150000}%
\pgfsetstrokecolor{currentstroke}%
\pgfsetdash{}{0pt}%
\pgfsys@defobject{currentmarker}{\pgfqpoint{0.000000in}{0.000000in}}{\pgfqpoint{0.000000in}{0.000000in}}{%
\pgfpathmoveto{\pgfqpoint{0.000000in}{0.000000in}}%
\pgfpathlineto{\pgfqpoint{0.000000in}{0.000000in}}%
\pgfusepath{stroke,fill}%
}%
\begin{pgfscope}%
\pgfsys@transformshift{2.599432in}{1.800000in}%
\pgfsys@useobject{currentmarker}{}%
\end{pgfscope}%
\end{pgfscope}%
\begin{pgfscope}%
\definecolor{textcolor}{rgb}{0.150000,0.150000,0.150000}%
\pgfsetstrokecolor{textcolor}%
\pgfsetfillcolor{textcolor}%
\pgftext[x=2.599432in,y=0.172222in,,top]{\color{textcolor}\sffamily\fontsize{8.000000}{9.600000}\selectfont 6.5}%
\end{pgfscope}%
\begin{pgfscope}%
\pgfpathrectangle{\pgfqpoint{0.750000in}{0.250000in}}{\pgfqpoint{2.113636in}{1.550000in}} %
\pgfusepath{clip}%
\pgfsetroundcap%
\pgfsetroundjoin%
\pgfsetlinewidth{0.803000pt}%
\definecolor{currentstroke}{rgb}{1.000000,1.000000,1.000000}%
\pgfsetstrokecolor{currentstroke}%
\pgfsetdash{}{0pt}%
\pgfpathmoveto{\pgfqpoint{2.863636in}{0.250000in}}%
\pgfpathlineto{\pgfqpoint{2.863636in}{1.800000in}}%
\pgfusepath{stroke}%
\end{pgfscope}%
\begin{pgfscope}%
\pgfsetbuttcap%
\pgfsetroundjoin%
\definecolor{currentfill}{rgb}{0.150000,0.150000,0.150000}%
\pgfsetfillcolor{currentfill}%
\pgfsetlinewidth{0.803000pt}%
\definecolor{currentstroke}{rgb}{0.150000,0.150000,0.150000}%
\pgfsetstrokecolor{currentstroke}%
\pgfsetdash{}{0pt}%
\pgfsys@defobject{currentmarker}{\pgfqpoint{0.000000in}{0.000000in}}{\pgfqpoint{0.000000in}{0.000000in}}{%
\pgfpathmoveto{\pgfqpoint{0.000000in}{0.000000in}}%
\pgfpathlineto{\pgfqpoint{0.000000in}{0.000000in}}%
\pgfusepath{stroke,fill}%
}%
\begin{pgfscope}%
\pgfsys@transformshift{2.863636in}{0.250000in}%
\pgfsys@useobject{currentmarker}{}%
\end{pgfscope}%
\end{pgfscope}%
\begin{pgfscope}%
\pgfsetbuttcap%
\pgfsetroundjoin%
\definecolor{currentfill}{rgb}{0.150000,0.150000,0.150000}%
\pgfsetfillcolor{currentfill}%
\pgfsetlinewidth{0.803000pt}%
\definecolor{currentstroke}{rgb}{0.150000,0.150000,0.150000}%
\pgfsetstrokecolor{currentstroke}%
\pgfsetdash{}{0pt}%
\pgfsys@defobject{currentmarker}{\pgfqpoint{0.000000in}{0.000000in}}{\pgfqpoint{0.000000in}{0.000000in}}{%
\pgfpathmoveto{\pgfqpoint{0.000000in}{0.000000in}}%
\pgfpathlineto{\pgfqpoint{0.000000in}{0.000000in}}%
\pgfusepath{stroke,fill}%
}%
\begin{pgfscope}%
\pgfsys@transformshift{2.863636in}{1.800000in}%
\pgfsys@useobject{currentmarker}{}%
\end{pgfscope}%
\end{pgfscope}%
\begin{pgfscope}%
\definecolor{textcolor}{rgb}{0.150000,0.150000,0.150000}%
\pgfsetstrokecolor{textcolor}%
\pgfsetfillcolor{textcolor}%
\pgftext[x=2.863636in,y=0.172222in,,top]{\color{textcolor}\sffamily\fontsize{8.000000}{9.600000}\selectfont 7.0}%
\end{pgfscope}%
\begin{pgfscope}%
\pgfpathrectangle{\pgfqpoint{0.750000in}{0.250000in}}{\pgfqpoint{2.113636in}{1.550000in}} %
\pgfusepath{clip}%
\pgfsetroundcap%
\pgfsetroundjoin%
\pgfsetlinewidth{0.803000pt}%
\definecolor{currentstroke}{rgb}{1.000000,1.000000,1.000000}%
\pgfsetstrokecolor{currentstroke}%
\pgfsetdash{}{0pt}%
\pgfpathmoveto{\pgfqpoint{0.750000in}{0.250000in}}%
\pgfpathlineto{\pgfqpoint{2.863636in}{0.250000in}}%
\pgfusepath{stroke}%
\end{pgfscope}%
\begin{pgfscope}%
\pgfsetbuttcap%
\pgfsetroundjoin%
\definecolor{currentfill}{rgb}{0.150000,0.150000,0.150000}%
\pgfsetfillcolor{currentfill}%
\pgfsetlinewidth{0.803000pt}%
\definecolor{currentstroke}{rgb}{0.150000,0.150000,0.150000}%
\pgfsetstrokecolor{currentstroke}%
\pgfsetdash{}{0pt}%
\pgfsys@defobject{currentmarker}{\pgfqpoint{0.000000in}{0.000000in}}{\pgfqpoint{0.000000in}{0.000000in}}{%
\pgfpathmoveto{\pgfqpoint{0.000000in}{0.000000in}}%
\pgfpathlineto{\pgfqpoint{0.000000in}{0.000000in}}%
\pgfusepath{stroke,fill}%
}%
\begin{pgfscope}%
\pgfsys@transformshift{0.750000in}{0.250000in}%
\pgfsys@useobject{currentmarker}{}%
\end{pgfscope}%
\end{pgfscope}%
\begin{pgfscope}%
\pgfsetbuttcap%
\pgfsetroundjoin%
\definecolor{currentfill}{rgb}{0.150000,0.150000,0.150000}%
\pgfsetfillcolor{currentfill}%
\pgfsetlinewidth{0.803000pt}%
\definecolor{currentstroke}{rgb}{0.150000,0.150000,0.150000}%
\pgfsetstrokecolor{currentstroke}%
\pgfsetdash{}{0pt}%
\pgfsys@defobject{currentmarker}{\pgfqpoint{0.000000in}{0.000000in}}{\pgfqpoint{0.000000in}{0.000000in}}{%
\pgfpathmoveto{\pgfqpoint{0.000000in}{0.000000in}}%
\pgfpathlineto{\pgfqpoint{0.000000in}{0.000000in}}%
\pgfusepath{stroke,fill}%
}%
\begin{pgfscope}%
\pgfsys@transformshift{2.863636in}{0.250000in}%
\pgfsys@useobject{currentmarker}{}%
\end{pgfscope}%
\end{pgfscope}%
\begin{pgfscope}%
\definecolor{textcolor}{rgb}{0.150000,0.150000,0.150000}%
\pgfsetstrokecolor{textcolor}%
\pgfsetfillcolor{textcolor}%
\pgftext[x=0.672222in,y=0.250000in,right,]{\color{textcolor}\sffamily\fontsize{8.000000}{9.600000}\selectfont −2.5}%
\end{pgfscope}%
\begin{pgfscope}%
\pgfpathrectangle{\pgfqpoint{0.750000in}{0.250000in}}{\pgfqpoint{2.113636in}{1.550000in}} %
\pgfusepath{clip}%
\pgfsetroundcap%
\pgfsetroundjoin%
\pgfsetlinewidth{0.803000pt}%
\definecolor{currentstroke}{rgb}{1.000000,1.000000,1.000000}%
\pgfsetstrokecolor{currentstroke}%
\pgfsetdash{}{0pt}%
\pgfpathmoveto{\pgfqpoint{0.750000in}{0.422222in}}%
\pgfpathlineto{\pgfqpoint{2.863636in}{0.422222in}}%
\pgfusepath{stroke}%
\end{pgfscope}%
\begin{pgfscope}%
\pgfsetbuttcap%
\pgfsetroundjoin%
\definecolor{currentfill}{rgb}{0.150000,0.150000,0.150000}%
\pgfsetfillcolor{currentfill}%
\pgfsetlinewidth{0.803000pt}%
\definecolor{currentstroke}{rgb}{0.150000,0.150000,0.150000}%
\pgfsetstrokecolor{currentstroke}%
\pgfsetdash{}{0pt}%
\pgfsys@defobject{currentmarker}{\pgfqpoint{0.000000in}{0.000000in}}{\pgfqpoint{0.000000in}{0.000000in}}{%
\pgfpathmoveto{\pgfqpoint{0.000000in}{0.000000in}}%
\pgfpathlineto{\pgfqpoint{0.000000in}{0.000000in}}%
\pgfusepath{stroke,fill}%
}%
\begin{pgfscope}%
\pgfsys@transformshift{0.750000in}{0.422222in}%
\pgfsys@useobject{currentmarker}{}%
\end{pgfscope}%
\end{pgfscope}%
\begin{pgfscope}%
\pgfsetbuttcap%
\pgfsetroundjoin%
\definecolor{currentfill}{rgb}{0.150000,0.150000,0.150000}%
\pgfsetfillcolor{currentfill}%
\pgfsetlinewidth{0.803000pt}%
\definecolor{currentstroke}{rgb}{0.150000,0.150000,0.150000}%
\pgfsetstrokecolor{currentstroke}%
\pgfsetdash{}{0pt}%
\pgfsys@defobject{currentmarker}{\pgfqpoint{0.000000in}{0.000000in}}{\pgfqpoint{0.000000in}{0.000000in}}{%
\pgfpathmoveto{\pgfqpoint{0.000000in}{0.000000in}}%
\pgfpathlineto{\pgfqpoint{0.000000in}{0.000000in}}%
\pgfusepath{stroke,fill}%
}%
\begin{pgfscope}%
\pgfsys@transformshift{2.863636in}{0.422222in}%
\pgfsys@useobject{currentmarker}{}%
\end{pgfscope}%
\end{pgfscope}%
\begin{pgfscope}%
\definecolor{textcolor}{rgb}{0.150000,0.150000,0.150000}%
\pgfsetstrokecolor{textcolor}%
\pgfsetfillcolor{textcolor}%
\pgftext[x=0.672222in,y=0.422222in,right,]{\color{textcolor}\sffamily\fontsize{8.000000}{9.600000}\selectfont −2.0}%
\end{pgfscope}%
\begin{pgfscope}%
\pgfpathrectangle{\pgfqpoint{0.750000in}{0.250000in}}{\pgfqpoint{2.113636in}{1.550000in}} %
\pgfusepath{clip}%
\pgfsetroundcap%
\pgfsetroundjoin%
\pgfsetlinewidth{0.803000pt}%
\definecolor{currentstroke}{rgb}{1.000000,1.000000,1.000000}%
\pgfsetstrokecolor{currentstroke}%
\pgfsetdash{}{0pt}%
\pgfpathmoveto{\pgfqpoint{0.750000in}{0.594444in}}%
\pgfpathlineto{\pgfqpoint{2.863636in}{0.594444in}}%
\pgfusepath{stroke}%
\end{pgfscope}%
\begin{pgfscope}%
\pgfsetbuttcap%
\pgfsetroundjoin%
\definecolor{currentfill}{rgb}{0.150000,0.150000,0.150000}%
\pgfsetfillcolor{currentfill}%
\pgfsetlinewidth{0.803000pt}%
\definecolor{currentstroke}{rgb}{0.150000,0.150000,0.150000}%
\pgfsetstrokecolor{currentstroke}%
\pgfsetdash{}{0pt}%
\pgfsys@defobject{currentmarker}{\pgfqpoint{0.000000in}{0.000000in}}{\pgfqpoint{0.000000in}{0.000000in}}{%
\pgfpathmoveto{\pgfqpoint{0.000000in}{0.000000in}}%
\pgfpathlineto{\pgfqpoint{0.000000in}{0.000000in}}%
\pgfusepath{stroke,fill}%
}%
\begin{pgfscope}%
\pgfsys@transformshift{0.750000in}{0.594444in}%
\pgfsys@useobject{currentmarker}{}%
\end{pgfscope}%
\end{pgfscope}%
\begin{pgfscope}%
\pgfsetbuttcap%
\pgfsetroundjoin%
\definecolor{currentfill}{rgb}{0.150000,0.150000,0.150000}%
\pgfsetfillcolor{currentfill}%
\pgfsetlinewidth{0.803000pt}%
\definecolor{currentstroke}{rgb}{0.150000,0.150000,0.150000}%
\pgfsetstrokecolor{currentstroke}%
\pgfsetdash{}{0pt}%
\pgfsys@defobject{currentmarker}{\pgfqpoint{0.000000in}{0.000000in}}{\pgfqpoint{0.000000in}{0.000000in}}{%
\pgfpathmoveto{\pgfqpoint{0.000000in}{0.000000in}}%
\pgfpathlineto{\pgfqpoint{0.000000in}{0.000000in}}%
\pgfusepath{stroke,fill}%
}%
\begin{pgfscope}%
\pgfsys@transformshift{2.863636in}{0.594444in}%
\pgfsys@useobject{currentmarker}{}%
\end{pgfscope}%
\end{pgfscope}%
\begin{pgfscope}%
\definecolor{textcolor}{rgb}{0.150000,0.150000,0.150000}%
\pgfsetstrokecolor{textcolor}%
\pgfsetfillcolor{textcolor}%
\pgftext[x=0.672222in,y=0.594444in,right,]{\color{textcolor}\sffamily\fontsize{8.000000}{9.600000}\selectfont −1.5}%
\end{pgfscope}%
\begin{pgfscope}%
\pgfpathrectangle{\pgfqpoint{0.750000in}{0.250000in}}{\pgfqpoint{2.113636in}{1.550000in}} %
\pgfusepath{clip}%
\pgfsetroundcap%
\pgfsetroundjoin%
\pgfsetlinewidth{0.803000pt}%
\definecolor{currentstroke}{rgb}{1.000000,1.000000,1.000000}%
\pgfsetstrokecolor{currentstroke}%
\pgfsetdash{}{0pt}%
\pgfpathmoveto{\pgfqpoint{0.750000in}{0.766667in}}%
\pgfpathlineto{\pgfqpoint{2.863636in}{0.766667in}}%
\pgfusepath{stroke}%
\end{pgfscope}%
\begin{pgfscope}%
\pgfsetbuttcap%
\pgfsetroundjoin%
\definecolor{currentfill}{rgb}{0.150000,0.150000,0.150000}%
\pgfsetfillcolor{currentfill}%
\pgfsetlinewidth{0.803000pt}%
\definecolor{currentstroke}{rgb}{0.150000,0.150000,0.150000}%
\pgfsetstrokecolor{currentstroke}%
\pgfsetdash{}{0pt}%
\pgfsys@defobject{currentmarker}{\pgfqpoint{0.000000in}{0.000000in}}{\pgfqpoint{0.000000in}{0.000000in}}{%
\pgfpathmoveto{\pgfqpoint{0.000000in}{0.000000in}}%
\pgfpathlineto{\pgfqpoint{0.000000in}{0.000000in}}%
\pgfusepath{stroke,fill}%
}%
\begin{pgfscope}%
\pgfsys@transformshift{0.750000in}{0.766667in}%
\pgfsys@useobject{currentmarker}{}%
\end{pgfscope}%
\end{pgfscope}%
\begin{pgfscope}%
\pgfsetbuttcap%
\pgfsetroundjoin%
\definecolor{currentfill}{rgb}{0.150000,0.150000,0.150000}%
\pgfsetfillcolor{currentfill}%
\pgfsetlinewidth{0.803000pt}%
\definecolor{currentstroke}{rgb}{0.150000,0.150000,0.150000}%
\pgfsetstrokecolor{currentstroke}%
\pgfsetdash{}{0pt}%
\pgfsys@defobject{currentmarker}{\pgfqpoint{0.000000in}{0.000000in}}{\pgfqpoint{0.000000in}{0.000000in}}{%
\pgfpathmoveto{\pgfqpoint{0.000000in}{0.000000in}}%
\pgfpathlineto{\pgfqpoint{0.000000in}{0.000000in}}%
\pgfusepath{stroke,fill}%
}%
\begin{pgfscope}%
\pgfsys@transformshift{2.863636in}{0.766667in}%
\pgfsys@useobject{currentmarker}{}%
\end{pgfscope}%
\end{pgfscope}%
\begin{pgfscope}%
\definecolor{textcolor}{rgb}{0.150000,0.150000,0.150000}%
\pgfsetstrokecolor{textcolor}%
\pgfsetfillcolor{textcolor}%
\pgftext[x=0.672222in,y=0.766667in,right,]{\color{textcolor}\sffamily\fontsize{8.000000}{9.600000}\selectfont −1.0}%
\end{pgfscope}%
\begin{pgfscope}%
\pgfpathrectangle{\pgfqpoint{0.750000in}{0.250000in}}{\pgfqpoint{2.113636in}{1.550000in}} %
\pgfusepath{clip}%
\pgfsetroundcap%
\pgfsetroundjoin%
\pgfsetlinewidth{0.803000pt}%
\definecolor{currentstroke}{rgb}{1.000000,1.000000,1.000000}%
\pgfsetstrokecolor{currentstroke}%
\pgfsetdash{}{0pt}%
\pgfpathmoveto{\pgfqpoint{0.750000in}{0.938889in}}%
\pgfpathlineto{\pgfqpoint{2.863636in}{0.938889in}}%
\pgfusepath{stroke}%
\end{pgfscope}%
\begin{pgfscope}%
\pgfsetbuttcap%
\pgfsetroundjoin%
\definecolor{currentfill}{rgb}{0.150000,0.150000,0.150000}%
\pgfsetfillcolor{currentfill}%
\pgfsetlinewidth{0.803000pt}%
\definecolor{currentstroke}{rgb}{0.150000,0.150000,0.150000}%
\pgfsetstrokecolor{currentstroke}%
\pgfsetdash{}{0pt}%
\pgfsys@defobject{currentmarker}{\pgfqpoint{0.000000in}{0.000000in}}{\pgfqpoint{0.000000in}{0.000000in}}{%
\pgfpathmoveto{\pgfqpoint{0.000000in}{0.000000in}}%
\pgfpathlineto{\pgfqpoint{0.000000in}{0.000000in}}%
\pgfusepath{stroke,fill}%
}%
\begin{pgfscope}%
\pgfsys@transformshift{0.750000in}{0.938889in}%
\pgfsys@useobject{currentmarker}{}%
\end{pgfscope}%
\end{pgfscope}%
\begin{pgfscope}%
\pgfsetbuttcap%
\pgfsetroundjoin%
\definecolor{currentfill}{rgb}{0.150000,0.150000,0.150000}%
\pgfsetfillcolor{currentfill}%
\pgfsetlinewidth{0.803000pt}%
\definecolor{currentstroke}{rgb}{0.150000,0.150000,0.150000}%
\pgfsetstrokecolor{currentstroke}%
\pgfsetdash{}{0pt}%
\pgfsys@defobject{currentmarker}{\pgfqpoint{0.000000in}{0.000000in}}{\pgfqpoint{0.000000in}{0.000000in}}{%
\pgfpathmoveto{\pgfqpoint{0.000000in}{0.000000in}}%
\pgfpathlineto{\pgfqpoint{0.000000in}{0.000000in}}%
\pgfusepath{stroke,fill}%
}%
\begin{pgfscope}%
\pgfsys@transformshift{2.863636in}{0.938889in}%
\pgfsys@useobject{currentmarker}{}%
\end{pgfscope}%
\end{pgfscope}%
\begin{pgfscope}%
\definecolor{textcolor}{rgb}{0.150000,0.150000,0.150000}%
\pgfsetstrokecolor{textcolor}%
\pgfsetfillcolor{textcolor}%
\pgftext[x=0.672222in,y=0.938889in,right,]{\color{textcolor}\sffamily\fontsize{8.000000}{9.600000}\selectfont −0.5}%
\end{pgfscope}%
\begin{pgfscope}%
\pgfpathrectangle{\pgfqpoint{0.750000in}{0.250000in}}{\pgfqpoint{2.113636in}{1.550000in}} %
\pgfusepath{clip}%
\pgfsetroundcap%
\pgfsetroundjoin%
\pgfsetlinewidth{0.803000pt}%
\definecolor{currentstroke}{rgb}{1.000000,1.000000,1.000000}%
\pgfsetstrokecolor{currentstroke}%
\pgfsetdash{}{0pt}%
\pgfpathmoveto{\pgfqpoint{0.750000in}{1.111111in}}%
\pgfpathlineto{\pgfqpoint{2.863636in}{1.111111in}}%
\pgfusepath{stroke}%
\end{pgfscope}%
\begin{pgfscope}%
\pgfsetbuttcap%
\pgfsetroundjoin%
\definecolor{currentfill}{rgb}{0.150000,0.150000,0.150000}%
\pgfsetfillcolor{currentfill}%
\pgfsetlinewidth{0.803000pt}%
\definecolor{currentstroke}{rgb}{0.150000,0.150000,0.150000}%
\pgfsetstrokecolor{currentstroke}%
\pgfsetdash{}{0pt}%
\pgfsys@defobject{currentmarker}{\pgfqpoint{0.000000in}{0.000000in}}{\pgfqpoint{0.000000in}{0.000000in}}{%
\pgfpathmoveto{\pgfqpoint{0.000000in}{0.000000in}}%
\pgfpathlineto{\pgfqpoint{0.000000in}{0.000000in}}%
\pgfusepath{stroke,fill}%
}%
\begin{pgfscope}%
\pgfsys@transformshift{0.750000in}{1.111111in}%
\pgfsys@useobject{currentmarker}{}%
\end{pgfscope}%
\end{pgfscope}%
\begin{pgfscope}%
\pgfsetbuttcap%
\pgfsetroundjoin%
\definecolor{currentfill}{rgb}{0.150000,0.150000,0.150000}%
\pgfsetfillcolor{currentfill}%
\pgfsetlinewidth{0.803000pt}%
\definecolor{currentstroke}{rgb}{0.150000,0.150000,0.150000}%
\pgfsetstrokecolor{currentstroke}%
\pgfsetdash{}{0pt}%
\pgfsys@defobject{currentmarker}{\pgfqpoint{0.000000in}{0.000000in}}{\pgfqpoint{0.000000in}{0.000000in}}{%
\pgfpathmoveto{\pgfqpoint{0.000000in}{0.000000in}}%
\pgfpathlineto{\pgfqpoint{0.000000in}{0.000000in}}%
\pgfusepath{stroke,fill}%
}%
\begin{pgfscope}%
\pgfsys@transformshift{2.863636in}{1.111111in}%
\pgfsys@useobject{currentmarker}{}%
\end{pgfscope}%
\end{pgfscope}%
\begin{pgfscope}%
\definecolor{textcolor}{rgb}{0.150000,0.150000,0.150000}%
\pgfsetstrokecolor{textcolor}%
\pgfsetfillcolor{textcolor}%
\pgftext[x=0.672222in,y=1.111111in,right,]{\color{textcolor}\sffamily\fontsize{8.000000}{9.600000}\selectfont 0.0}%
\end{pgfscope}%
\begin{pgfscope}%
\pgfpathrectangle{\pgfqpoint{0.750000in}{0.250000in}}{\pgfqpoint{2.113636in}{1.550000in}} %
\pgfusepath{clip}%
\pgfsetroundcap%
\pgfsetroundjoin%
\pgfsetlinewidth{0.803000pt}%
\definecolor{currentstroke}{rgb}{1.000000,1.000000,1.000000}%
\pgfsetstrokecolor{currentstroke}%
\pgfsetdash{}{0pt}%
\pgfpathmoveto{\pgfqpoint{0.750000in}{1.283333in}}%
\pgfpathlineto{\pgfqpoint{2.863636in}{1.283333in}}%
\pgfusepath{stroke}%
\end{pgfscope}%
\begin{pgfscope}%
\pgfsetbuttcap%
\pgfsetroundjoin%
\definecolor{currentfill}{rgb}{0.150000,0.150000,0.150000}%
\pgfsetfillcolor{currentfill}%
\pgfsetlinewidth{0.803000pt}%
\definecolor{currentstroke}{rgb}{0.150000,0.150000,0.150000}%
\pgfsetstrokecolor{currentstroke}%
\pgfsetdash{}{0pt}%
\pgfsys@defobject{currentmarker}{\pgfqpoint{0.000000in}{0.000000in}}{\pgfqpoint{0.000000in}{0.000000in}}{%
\pgfpathmoveto{\pgfqpoint{0.000000in}{0.000000in}}%
\pgfpathlineto{\pgfqpoint{0.000000in}{0.000000in}}%
\pgfusepath{stroke,fill}%
}%
\begin{pgfscope}%
\pgfsys@transformshift{0.750000in}{1.283333in}%
\pgfsys@useobject{currentmarker}{}%
\end{pgfscope}%
\end{pgfscope}%
\begin{pgfscope}%
\pgfsetbuttcap%
\pgfsetroundjoin%
\definecolor{currentfill}{rgb}{0.150000,0.150000,0.150000}%
\pgfsetfillcolor{currentfill}%
\pgfsetlinewidth{0.803000pt}%
\definecolor{currentstroke}{rgb}{0.150000,0.150000,0.150000}%
\pgfsetstrokecolor{currentstroke}%
\pgfsetdash{}{0pt}%
\pgfsys@defobject{currentmarker}{\pgfqpoint{0.000000in}{0.000000in}}{\pgfqpoint{0.000000in}{0.000000in}}{%
\pgfpathmoveto{\pgfqpoint{0.000000in}{0.000000in}}%
\pgfpathlineto{\pgfqpoint{0.000000in}{0.000000in}}%
\pgfusepath{stroke,fill}%
}%
\begin{pgfscope}%
\pgfsys@transformshift{2.863636in}{1.283333in}%
\pgfsys@useobject{currentmarker}{}%
\end{pgfscope}%
\end{pgfscope}%
\begin{pgfscope}%
\definecolor{textcolor}{rgb}{0.150000,0.150000,0.150000}%
\pgfsetstrokecolor{textcolor}%
\pgfsetfillcolor{textcolor}%
\pgftext[x=0.672222in,y=1.283333in,right,]{\color{textcolor}\sffamily\fontsize{8.000000}{9.600000}\selectfont 0.5}%
\end{pgfscope}%
\begin{pgfscope}%
\pgfpathrectangle{\pgfqpoint{0.750000in}{0.250000in}}{\pgfqpoint{2.113636in}{1.550000in}} %
\pgfusepath{clip}%
\pgfsetroundcap%
\pgfsetroundjoin%
\pgfsetlinewidth{0.803000pt}%
\definecolor{currentstroke}{rgb}{1.000000,1.000000,1.000000}%
\pgfsetstrokecolor{currentstroke}%
\pgfsetdash{}{0pt}%
\pgfpathmoveto{\pgfqpoint{0.750000in}{1.455556in}}%
\pgfpathlineto{\pgfqpoint{2.863636in}{1.455556in}}%
\pgfusepath{stroke}%
\end{pgfscope}%
\begin{pgfscope}%
\pgfsetbuttcap%
\pgfsetroundjoin%
\definecolor{currentfill}{rgb}{0.150000,0.150000,0.150000}%
\pgfsetfillcolor{currentfill}%
\pgfsetlinewidth{0.803000pt}%
\definecolor{currentstroke}{rgb}{0.150000,0.150000,0.150000}%
\pgfsetstrokecolor{currentstroke}%
\pgfsetdash{}{0pt}%
\pgfsys@defobject{currentmarker}{\pgfqpoint{0.000000in}{0.000000in}}{\pgfqpoint{0.000000in}{0.000000in}}{%
\pgfpathmoveto{\pgfqpoint{0.000000in}{0.000000in}}%
\pgfpathlineto{\pgfqpoint{0.000000in}{0.000000in}}%
\pgfusepath{stroke,fill}%
}%
\begin{pgfscope}%
\pgfsys@transformshift{0.750000in}{1.455556in}%
\pgfsys@useobject{currentmarker}{}%
\end{pgfscope}%
\end{pgfscope}%
\begin{pgfscope}%
\pgfsetbuttcap%
\pgfsetroundjoin%
\definecolor{currentfill}{rgb}{0.150000,0.150000,0.150000}%
\pgfsetfillcolor{currentfill}%
\pgfsetlinewidth{0.803000pt}%
\definecolor{currentstroke}{rgb}{0.150000,0.150000,0.150000}%
\pgfsetstrokecolor{currentstroke}%
\pgfsetdash{}{0pt}%
\pgfsys@defobject{currentmarker}{\pgfqpoint{0.000000in}{0.000000in}}{\pgfqpoint{0.000000in}{0.000000in}}{%
\pgfpathmoveto{\pgfqpoint{0.000000in}{0.000000in}}%
\pgfpathlineto{\pgfqpoint{0.000000in}{0.000000in}}%
\pgfusepath{stroke,fill}%
}%
\begin{pgfscope}%
\pgfsys@transformshift{2.863636in}{1.455556in}%
\pgfsys@useobject{currentmarker}{}%
\end{pgfscope}%
\end{pgfscope}%
\begin{pgfscope}%
\definecolor{textcolor}{rgb}{0.150000,0.150000,0.150000}%
\pgfsetstrokecolor{textcolor}%
\pgfsetfillcolor{textcolor}%
\pgftext[x=0.672222in,y=1.455556in,right,]{\color{textcolor}\sffamily\fontsize{8.000000}{9.600000}\selectfont 1.0}%
\end{pgfscope}%
\begin{pgfscope}%
\pgfpathrectangle{\pgfqpoint{0.750000in}{0.250000in}}{\pgfqpoint{2.113636in}{1.550000in}} %
\pgfusepath{clip}%
\pgfsetroundcap%
\pgfsetroundjoin%
\pgfsetlinewidth{0.803000pt}%
\definecolor{currentstroke}{rgb}{1.000000,1.000000,1.000000}%
\pgfsetstrokecolor{currentstroke}%
\pgfsetdash{}{0pt}%
\pgfpathmoveto{\pgfqpoint{0.750000in}{1.627778in}}%
\pgfpathlineto{\pgfqpoint{2.863636in}{1.627778in}}%
\pgfusepath{stroke}%
\end{pgfscope}%
\begin{pgfscope}%
\pgfsetbuttcap%
\pgfsetroundjoin%
\definecolor{currentfill}{rgb}{0.150000,0.150000,0.150000}%
\pgfsetfillcolor{currentfill}%
\pgfsetlinewidth{0.803000pt}%
\definecolor{currentstroke}{rgb}{0.150000,0.150000,0.150000}%
\pgfsetstrokecolor{currentstroke}%
\pgfsetdash{}{0pt}%
\pgfsys@defobject{currentmarker}{\pgfqpoint{0.000000in}{0.000000in}}{\pgfqpoint{0.000000in}{0.000000in}}{%
\pgfpathmoveto{\pgfqpoint{0.000000in}{0.000000in}}%
\pgfpathlineto{\pgfqpoint{0.000000in}{0.000000in}}%
\pgfusepath{stroke,fill}%
}%
\begin{pgfscope}%
\pgfsys@transformshift{0.750000in}{1.627778in}%
\pgfsys@useobject{currentmarker}{}%
\end{pgfscope}%
\end{pgfscope}%
\begin{pgfscope}%
\pgfsetbuttcap%
\pgfsetroundjoin%
\definecolor{currentfill}{rgb}{0.150000,0.150000,0.150000}%
\pgfsetfillcolor{currentfill}%
\pgfsetlinewidth{0.803000pt}%
\definecolor{currentstroke}{rgb}{0.150000,0.150000,0.150000}%
\pgfsetstrokecolor{currentstroke}%
\pgfsetdash{}{0pt}%
\pgfsys@defobject{currentmarker}{\pgfqpoint{0.000000in}{0.000000in}}{\pgfqpoint{0.000000in}{0.000000in}}{%
\pgfpathmoveto{\pgfqpoint{0.000000in}{0.000000in}}%
\pgfpathlineto{\pgfqpoint{0.000000in}{0.000000in}}%
\pgfusepath{stroke,fill}%
}%
\begin{pgfscope}%
\pgfsys@transformshift{2.863636in}{1.627778in}%
\pgfsys@useobject{currentmarker}{}%
\end{pgfscope}%
\end{pgfscope}%
\begin{pgfscope}%
\definecolor{textcolor}{rgb}{0.150000,0.150000,0.150000}%
\pgfsetstrokecolor{textcolor}%
\pgfsetfillcolor{textcolor}%
\pgftext[x=0.672222in,y=1.627778in,right,]{\color{textcolor}\sffamily\fontsize{8.000000}{9.600000}\selectfont 1.5}%
\end{pgfscope}%
\begin{pgfscope}%
\pgfpathrectangle{\pgfqpoint{0.750000in}{0.250000in}}{\pgfqpoint{2.113636in}{1.550000in}} %
\pgfusepath{clip}%
\pgfsetroundcap%
\pgfsetroundjoin%
\pgfsetlinewidth{0.803000pt}%
\definecolor{currentstroke}{rgb}{1.000000,1.000000,1.000000}%
\pgfsetstrokecolor{currentstroke}%
\pgfsetdash{}{0pt}%
\pgfpathmoveto{\pgfqpoint{0.750000in}{1.800000in}}%
\pgfpathlineto{\pgfqpoint{2.863636in}{1.800000in}}%
\pgfusepath{stroke}%
\end{pgfscope}%
\begin{pgfscope}%
\pgfsetbuttcap%
\pgfsetroundjoin%
\definecolor{currentfill}{rgb}{0.150000,0.150000,0.150000}%
\pgfsetfillcolor{currentfill}%
\pgfsetlinewidth{0.803000pt}%
\definecolor{currentstroke}{rgb}{0.150000,0.150000,0.150000}%
\pgfsetstrokecolor{currentstroke}%
\pgfsetdash{}{0pt}%
\pgfsys@defobject{currentmarker}{\pgfqpoint{0.000000in}{0.000000in}}{\pgfqpoint{0.000000in}{0.000000in}}{%
\pgfpathmoveto{\pgfqpoint{0.000000in}{0.000000in}}%
\pgfpathlineto{\pgfqpoint{0.000000in}{0.000000in}}%
\pgfusepath{stroke,fill}%
}%
\begin{pgfscope}%
\pgfsys@transformshift{0.750000in}{1.800000in}%
\pgfsys@useobject{currentmarker}{}%
\end{pgfscope}%
\end{pgfscope}%
\begin{pgfscope}%
\pgfsetbuttcap%
\pgfsetroundjoin%
\definecolor{currentfill}{rgb}{0.150000,0.150000,0.150000}%
\pgfsetfillcolor{currentfill}%
\pgfsetlinewidth{0.803000pt}%
\definecolor{currentstroke}{rgb}{0.150000,0.150000,0.150000}%
\pgfsetstrokecolor{currentstroke}%
\pgfsetdash{}{0pt}%
\pgfsys@defobject{currentmarker}{\pgfqpoint{0.000000in}{0.000000in}}{\pgfqpoint{0.000000in}{0.000000in}}{%
\pgfpathmoveto{\pgfqpoint{0.000000in}{0.000000in}}%
\pgfpathlineto{\pgfqpoint{0.000000in}{0.000000in}}%
\pgfusepath{stroke,fill}%
}%
\begin{pgfscope}%
\pgfsys@transformshift{2.863636in}{1.800000in}%
\pgfsys@useobject{currentmarker}{}%
\end{pgfscope}%
\end{pgfscope}%
\begin{pgfscope}%
\definecolor{textcolor}{rgb}{0.150000,0.150000,0.150000}%
\pgfsetstrokecolor{textcolor}%
\pgfsetfillcolor{textcolor}%
\pgftext[x=0.672222in,y=1.800000in,right,]{\color{textcolor}\sffamily\fontsize{8.000000}{9.600000}\selectfont 2.0}%
\end{pgfscope}%
\begin{pgfscope}%
\pgfpathrectangle{\pgfqpoint{0.750000in}{0.250000in}}{\pgfqpoint{2.113636in}{1.550000in}} %
\pgfusepath{clip}%
\pgfsetroundcap%
\pgfsetroundjoin%
\pgfsetlinewidth{1.405250pt}%
\definecolor{currentstroke}{rgb}{0.298039,0.447059,0.690196}%
\pgfsetstrokecolor{currentstroke}%
\pgfsetdash{}{0pt}%
\pgfpathmoveto{\pgfqpoint{1.542614in}{0.313441in}}%
\pgfpathlineto{\pgfqpoint{1.178011in}{1.142702in}}%
\pgfpathlineto{\pgfqpoint{1.738125in}{0.686906in}}%
\pgfpathlineto{\pgfqpoint{2.150284in}{0.379151in}}%
\pgfpathlineto{\pgfqpoint{1.336534in}{1.368072in}}%
\pgfpathlineto{\pgfqpoint{2.588864in}{0.831575in}}%
\pgfpathlineto{\pgfqpoint{1.051193in}{1.273719in}}%
\pgfpathlineto{\pgfqpoint{2.261250in}{0.514425in}}%
\pgfpathlineto{\pgfqpoint{1.167443in}{1.166567in}}%
\pgfpathlineto{\pgfqpoint{1.827955in}{1.396050in}}%
\pgfpathlineto{\pgfqpoint{1.130455in}{1.187845in}}%
\pgfpathlineto{\pgfqpoint{1.114602in}{1.244150in}}%
\pgfpathlineto{\pgfqpoint{2.657557in}{1.782295in}}%
\pgfpathlineto{\pgfqpoint{1.399943in}{1.278620in}}%
\pgfpathlineto{\pgfqpoint{1.901932in}{1.311317in}}%
\pgfpathlineto{\pgfqpoint{1.801534in}{1.253962in}}%
\pgfpathlineto{\pgfqpoint{0.971932in}{0.898370in}}%
\pgfpathlineto{\pgfqpoint{1.236136in}{1.080417in}}%
\pgfpathlineto{\pgfqpoint{1.331250in}{1.341894in}}%
\pgfpathlineto{\pgfqpoint{1.040625in}{1.284184in}}%
\pgfpathlineto{\pgfqpoint{2.493750in}{0.939404in}}%
\pgfpathlineto{\pgfqpoint{1.822670in}{1.396868in}}%
\pgfpathlineto{\pgfqpoint{2.488466in}{0.974920in}}%
\pgfpathlineto{\pgfqpoint{2.192557in}{0.325357in}}%
\pgfpathlineto{\pgfqpoint{2.208409in}{0.333976in}}%
\pgfpathlineto{\pgfqpoint{2.155568in}{0.323992in}}%
\pgfpathlineto{\pgfqpoint{1.035341in}{1.265075in}}%
\pgfpathlineto{\pgfqpoint{2.477898in}{0.993848in}}%
\pgfpathlineto{\pgfqpoint{1.812102in}{1.342233in}}%
\pgfpathlineto{\pgfqpoint{2.530739in}{0.911756in}}%
\pgfusepath{stroke}%
\end{pgfscope}%
\begin{pgfscope}%
\pgfpathrectangle{\pgfqpoint{0.750000in}{0.250000in}}{\pgfqpoint{2.113636in}{1.550000in}} %
\pgfusepath{clip}%
\pgfsetroundcap%
\pgfsetroundjoin%
\pgfsetlinewidth{1.405250pt}%
\definecolor{currentstroke}{rgb}{0.333333,0.658824,0.407843}%
\pgfsetstrokecolor{currentstroke}%
\pgfsetdash{}{0pt}%
\pgfpathmoveto{\pgfqpoint{1.542614in}{1.436673in}}%
\pgfpathlineto{\pgfqpoint{1.178011in}{0.685768in}}%
\pgfpathlineto{\pgfqpoint{1.738125in}{0.623993in}}%
\pgfpathlineto{\pgfqpoint{2.150284in}{0.856747in}}%
\pgfpathlineto{\pgfqpoint{1.336534in}{1.036838in}}%
\pgfpathlineto{\pgfqpoint{2.588864in}{1.315941in}}%
\pgfpathlineto{\pgfqpoint{1.051193in}{0.787462in}}%
\pgfpathlineto{\pgfqpoint{2.261250in}{1.112799in}}%
\pgfpathlineto{\pgfqpoint{1.167443in}{0.662902in}}%
\pgfpathlineto{\pgfqpoint{1.827955in}{0.848502in}}%
\pgfpathlineto{\pgfqpoint{1.130455in}{0.522816in}}%
\pgfpathlineto{\pgfqpoint{1.114602in}{0.517957in}}%
\pgfpathlineto{\pgfqpoint{2.657557in}{1.061092in}}%
\pgfpathlineto{\pgfqpoint{1.399943in}{1.471336in}}%
\pgfpathlineto{\pgfqpoint{1.901932in}{1.559961in}}%
\pgfpathlineto{\pgfqpoint{1.801534in}{0.678537in}}%
\pgfpathlineto{\pgfqpoint{0.971932in}{0.957435in}}%
\pgfpathlineto{\pgfqpoint{1.236136in}{0.736818in}}%
\pgfpathlineto{\pgfqpoint{1.331250in}{0.994985in}}%
\pgfpathlineto{\pgfqpoint{1.040625in}{0.812408in}}%
\pgfpathlineto{\pgfqpoint{2.493750in}{0.838810in}}%
\pgfpathlineto{\pgfqpoint{1.822670in}{0.829665in}}%
\pgfpathlineto{\pgfqpoint{2.488466in}{0.843989in}}%
\pgfpathlineto{\pgfqpoint{2.192557in}{0.656508in}}%
\pgfpathlineto{\pgfqpoint{2.208409in}{0.694039in}}%
\pgfpathlineto{\pgfqpoint{2.155568in}{0.831412in}}%
\pgfpathlineto{\pgfqpoint{1.035341in}{0.804387in}}%
\pgfpathlineto{\pgfqpoint{2.477898in}{0.837149in}}%
\pgfpathlineto{\pgfqpoint{1.812102in}{0.748739in}}%
\pgfpathlineto{\pgfqpoint{2.530739in}{0.960691in}}%
\pgfusepath{stroke}%
\end{pgfscope}%
\begin{pgfscope}%
\pgfpathrectangle{\pgfqpoint{0.750000in}{0.250000in}}{\pgfqpoint{2.113636in}{1.550000in}} %
\pgfusepath{clip}%
\pgfsetroundcap%
\pgfsetroundjoin%
\pgfsetlinewidth{1.405250pt}%
\definecolor{currentstroke}{rgb}{0.768627,0.305882,0.321569}%
\pgfsetstrokecolor{currentstroke}%
\pgfsetdash{}{0pt}%
\pgfpathmoveto{\pgfqpoint{1.542614in}{0.912338in}}%
\pgfpathlineto{\pgfqpoint{1.178011in}{0.905269in}}%
\pgfpathlineto{\pgfqpoint{1.738125in}{1.361595in}}%
\pgfpathlineto{\pgfqpoint{2.150284in}{1.265942in}}%
\pgfpathlineto{\pgfqpoint{1.336534in}{0.799990in}}%
\pgfpathlineto{\pgfqpoint{2.588864in}{1.379753in}}%
\pgfpathlineto{\pgfqpoint{1.051193in}{0.365712in}}%
\pgfpathlineto{\pgfqpoint{2.261250in}{0.645368in}}%
\pgfpathlineto{\pgfqpoint{1.167443in}{0.898252in}}%
\pgfpathlineto{\pgfqpoint{1.827955in}{1.392317in}}%
\pgfpathlineto{\pgfqpoint{1.130455in}{0.956059in}}%
\pgfpathlineto{\pgfqpoint{1.114602in}{0.925832in}}%
\pgfpathlineto{\pgfqpoint{2.657557in}{1.566472in}}%
\pgfpathlineto{\pgfqpoint{1.399943in}{0.816788in}}%
\pgfpathlineto{\pgfqpoint{1.901932in}{1.380377in}}%
\pgfpathlineto{\pgfqpoint{1.801534in}{1.334476in}}%
\pgfpathlineto{\pgfqpoint{0.971932in}{0.786854in}}%
\pgfpathlineto{\pgfqpoint{1.236136in}{0.900968in}}%
\pgfpathlineto{\pgfqpoint{1.331250in}{0.795286in}}%
\pgfpathlineto{\pgfqpoint{1.040625in}{0.418126in}}%
\pgfpathlineto{\pgfqpoint{2.493750in}{1.305566in}}%
\pgfpathlineto{\pgfqpoint{1.822670in}{1.374504in}}%
\pgfpathlineto{\pgfqpoint{2.488466in}{1.302135in}}%
\pgfpathlineto{\pgfqpoint{2.192557in}{0.738905in}}%
\pgfpathlineto{\pgfqpoint{2.208409in}{0.613408in}}%
\pgfpathlineto{\pgfqpoint{2.155568in}{1.189813in}}%
\pgfpathlineto{\pgfqpoint{1.035341in}{0.475918in}}%
\pgfpathlineto{\pgfqpoint{2.477898in}{1.324250in}}%
\pgfpathlineto{\pgfqpoint{1.812102in}{1.361044in}}%
\pgfpathlineto{\pgfqpoint{2.530739in}{1.373109in}}%
\pgfusepath{stroke}%
\end{pgfscope}%
\begin{pgfscope}%
\pgfpathrectangle{\pgfqpoint{0.750000in}{0.250000in}}{\pgfqpoint{2.113636in}{1.550000in}} %
\pgfusepath{clip}%
\pgfsetroundcap%
\pgfsetroundjoin%
\pgfsetlinewidth{1.405250pt}%
\definecolor{currentstroke}{rgb}{0.505882,0.447059,0.698039}%
\pgfsetstrokecolor{currentstroke}%
\pgfsetdash{}{0pt}%
\pgfpathmoveto{\pgfqpoint{1.542614in}{1.080164in}}%
\pgfpathlineto{\pgfqpoint{1.178011in}{1.481647in}}%
\pgfpathlineto{\pgfqpoint{1.738125in}{1.544172in}}%
\pgfpathlineto{\pgfqpoint{2.150284in}{0.925829in}}%
\pgfpathlineto{\pgfqpoint{1.336534in}{1.369537in}}%
\pgfpathlineto{\pgfqpoint{2.588864in}{1.049667in}}%
\pgfpathlineto{\pgfqpoint{1.051193in}{1.463924in}}%
\pgfpathlineto{\pgfqpoint{2.261250in}{1.017712in}}%
\pgfpathlineto{\pgfqpoint{1.167443in}{1.478093in}}%
\pgfpathlineto{\pgfqpoint{1.827955in}{1.143839in}}%
\pgfpathlineto{\pgfqpoint{1.130455in}{1.374535in}}%
\pgfpathlineto{\pgfqpoint{1.114602in}{1.331354in}}%
\pgfpathlineto{\pgfqpoint{2.657557in}{0.743418in}}%
\pgfpathlineto{\pgfqpoint{1.399943in}{1.171675in}}%
\pgfpathlineto{\pgfqpoint{1.901932in}{1.304362in}}%
\pgfpathlineto{\pgfqpoint{1.801534in}{1.172572in}}%
\pgfpathlineto{\pgfqpoint{0.971932in}{1.259475in}}%
\pgfpathlineto{\pgfqpoint{1.236136in}{1.530321in}}%
\pgfpathlineto{\pgfqpoint{1.331250in}{1.421614in}}%
\pgfpathlineto{\pgfqpoint{1.040625in}{1.514641in}}%
\pgfpathlineto{\pgfqpoint{2.493750in}{1.147209in}}%
\pgfpathlineto{\pgfqpoint{1.822670in}{1.158535in}}%
\pgfpathlineto{\pgfqpoint{2.488466in}{1.193971in}}%
\pgfpathlineto{\pgfqpoint{2.192557in}{0.731433in}}%
\pgfpathlineto{\pgfqpoint{2.208409in}{0.813516in}}%
\pgfpathlineto{\pgfqpoint{2.155568in}{0.922453in}}%
\pgfpathlineto{\pgfqpoint{1.035341in}{1.516659in}}%
\pgfpathlineto{\pgfqpoint{2.477898in}{1.283448in}}%
\pgfpathlineto{\pgfqpoint{1.812102in}{1.158493in}}%
\pgfpathlineto{\pgfqpoint{2.530739in}{0.918734in}}%
\pgfusepath{stroke}%
\end{pgfscope}%
\begin{pgfscope}%
\pgfpathrectangle{\pgfqpoint{0.750000in}{0.250000in}}{\pgfqpoint{2.113636in}{1.550000in}} %
\pgfusepath{clip}%
\pgfsetroundcap%
\pgfsetroundjoin%
\pgfsetlinewidth{1.405250pt}%
\definecolor{currentstroke}{rgb}{0.800000,0.725490,0.454902}%
\pgfsetstrokecolor{currentstroke}%
\pgfsetdash{}{0pt}%
\pgfpathmoveto{\pgfqpoint{1.542614in}{1.068218in}}%
\pgfpathlineto{\pgfqpoint{1.178011in}{0.847302in}}%
\pgfpathlineto{\pgfqpoint{1.738125in}{1.003985in}}%
\pgfpathlineto{\pgfqpoint{2.150284in}{0.979652in}}%
\pgfpathlineto{\pgfqpoint{1.336534in}{1.208900in}}%
\pgfpathlineto{\pgfqpoint{2.588864in}{1.063590in}}%
\pgfpathlineto{\pgfqpoint{1.051193in}{1.006899in}}%
\pgfpathlineto{\pgfqpoint{2.261250in}{0.998263in}}%
\pgfpathlineto{\pgfqpoint{1.167443in}{0.832972in}}%
\pgfpathlineto{\pgfqpoint{1.827955in}{1.217082in}}%
\pgfpathlineto{\pgfqpoint{1.130455in}{0.812359in}}%
\pgfpathlineto{\pgfqpoint{1.114602in}{0.872761in}}%
\pgfpathlineto{\pgfqpoint{2.657557in}{1.197582in}}%
\pgfpathlineto{\pgfqpoint{1.399943in}{1.044965in}}%
\pgfpathlineto{\pgfqpoint{1.901932in}{0.910807in}}%
\pgfpathlineto{\pgfqpoint{1.801534in}{1.108103in}}%
\pgfpathlineto{\pgfqpoint{0.971932in}{0.652583in}}%
\pgfpathlineto{\pgfqpoint{1.236136in}{1.211817in}}%
\pgfpathlineto{\pgfqpoint{1.331250in}{1.247358in}}%
\pgfpathlineto{\pgfqpoint{1.040625in}{1.004568in}}%
\pgfpathlineto{\pgfqpoint{2.493750in}{1.322929in}}%
\pgfpathlineto{\pgfqpoint{1.822670in}{1.198405in}}%
\pgfpathlineto{\pgfqpoint{2.488466in}{1.317557in}}%
\pgfpathlineto{\pgfqpoint{2.192557in}{1.225106in}}%
\pgfpathlineto{\pgfqpoint{2.208409in}{1.388339in}}%
\pgfpathlineto{\pgfqpoint{2.155568in}{1.003921in}}%
\pgfpathlineto{\pgfqpoint{1.035341in}{0.975577in}}%
\pgfpathlineto{\pgfqpoint{2.477898in}{1.293961in}}%
\pgfpathlineto{\pgfqpoint{1.812102in}{1.159175in}}%
\pgfpathlineto{\pgfqpoint{2.530739in}{1.302600in}}%
\pgfusepath{stroke}%
\end{pgfscope}%
\begin{pgfscope}%
\pgfsetrectcap%
\pgfsetmiterjoin%
\pgfsetlinewidth{0.000000pt}%
\definecolor{currentstroke}{rgb}{1.000000,1.000000,1.000000}%
\pgfsetstrokecolor{currentstroke}%
\pgfsetdash{}{0pt}%
\pgfpathmoveto{\pgfqpoint{0.750000in}{0.250000in}}%
\pgfpathlineto{\pgfqpoint{2.863636in}{0.250000in}}%
\pgfusepath{}%
\end{pgfscope}%
\begin{pgfscope}%
\pgfsetrectcap%
\pgfsetmiterjoin%
\pgfsetlinewidth{0.000000pt}%
\definecolor{currentstroke}{rgb}{1.000000,1.000000,1.000000}%
\pgfsetstrokecolor{currentstroke}%
\pgfsetdash{}{0pt}%
\pgfpathmoveto{\pgfqpoint{0.750000in}{0.250000in}}%
\pgfpathlineto{\pgfqpoint{0.750000in}{1.800000in}}%
\pgfusepath{}%
\end{pgfscope}%
\begin{pgfscope}%
\pgfsetrectcap%
\pgfsetmiterjoin%
\pgfsetlinewidth{0.000000pt}%
\definecolor{currentstroke}{rgb}{1.000000,1.000000,1.000000}%
\pgfsetstrokecolor{currentstroke}%
\pgfsetdash{}{0pt}%
\pgfpathmoveto{\pgfqpoint{0.750000in}{1.800000in}}%
\pgfpathlineto{\pgfqpoint{2.863636in}{1.800000in}}%
\pgfusepath{}%
\end{pgfscope}%
\begin{pgfscope}%
\pgfsetrectcap%
\pgfsetmiterjoin%
\pgfsetlinewidth{0.000000pt}%
\definecolor{currentstroke}{rgb}{1.000000,1.000000,1.000000}%
\pgfsetstrokecolor{currentstroke}%
\pgfsetdash{}{0pt}%
\pgfpathmoveto{\pgfqpoint{2.863636in}{0.250000in}}%
\pgfpathlineto{\pgfqpoint{2.863636in}{1.800000in}}%
\pgfusepath{}%
\end{pgfscope}%
\begin{pgfscope}%
\pgfsetbuttcap%
\pgfsetmiterjoin%
\definecolor{currentfill}{rgb}{0.917647,0.917647,0.949020}%
\pgfsetfillcolor{currentfill}%
\pgfsetlinewidth{0.000000pt}%
\definecolor{currentstroke}{rgb}{0.000000,0.000000,0.000000}%
\pgfsetstrokecolor{currentstroke}%
\pgfsetstrokeopacity{0.000000}%
\pgfsetdash{}{0pt}%
\pgfpathmoveto{\pgfqpoint{3.286364in}{0.250000in}}%
\pgfpathlineto{\pgfqpoint{5.400000in}{0.250000in}}%
\pgfpathlineto{\pgfqpoint{5.400000in}{1.800000in}}%
\pgfpathlineto{\pgfqpoint{3.286364in}{1.800000in}}%
\pgfpathclose%
\pgfusepath{fill}%
\end{pgfscope}%
\begin{pgfscope}%
\pgfpathrectangle{\pgfqpoint{3.286364in}{0.250000in}}{\pgfqpoint{2.113636in}{1.550000in}} %
\pgfusepath{clip}%
\pgfsetroundcap%
\pgfsetroundjoin%
\pgfsetlinewidth{0.803000pt}%
\definecolor{currentstroke}{rgb}{1.000000,1.000000,1.000000}%
\pgfsetstrokecolor{currentstroke}%
\pgfsetdash{}{0pt}%
\pgfpathmoveto{\pgfqpoint{3.286364in}{0.250000in}}%
\pgfpathlineto{\pgfqpoint{3.286364in}{1.800000in}}%
\pgfusepath{stroke}%
\end{pgfscope}%
\begin{pgfscope}%
\pgfsetbuttcap%
\pgfsetroundjoin%
\definecolor{currentfill}{rgb}{0.150000,0.150000,0.150000}%
\pgfsetfillcolor{currentfill}%
\pgfsetlinewidth{0.803000pt}%
\definecolor{currentstroke}{rgb}{0.150000,0.150000,0.150000}%
\pgfsetstrokecolor{currentstroke}%
\pgfsetdash{}{0pt}%
\pgfsys@defobject{currentmarker}{\pgfqpoint{0.000000in}{0.000000in}}{\pgfqpoint{0.000000in}{0.000000in}}{%
\pgfpathmoveto{\pgfqpoint{0.000000in}{0.000000in}}%
\pgfpathlineto{\pgfqpoint{0.000000in}{0.000000in}}%
\pgfusepath{stroke,fill}%
}%
\begin{pgfscope}%
\pgfsys@transformshift{3.286364in}{0.250000in}%
\pgfsys@useobject{currentmarker}{}%
\end{pgfscope}%
\end{pgfscope}%
\begin{pgfscope}%
\pgfsetbuttcap%
\pgfsetroundjoin%
\definecolor{currentfill}{rgb}{0.150000,0.150000,0.150000}%
\pgfsetfillcolor{currentfill}%
\pgfsetlinewidth{0.803000pt}%
\definecolor{currentstroke}{rgb}{0.150000,0.150000,0.150000}%
\pgfsetstrokecolor{currentstroke}%
\pgfsetdash{}{0pt}%
\pgfsys@defobject{currentmarker}{\pgfqpoint{0.000000in}{0.000000in}}{\pgfqpoint{0.000000in}{0.000000in}}{%
\pgfpathmoveto{\pgfqpoint{0.000000in}{0.000000in}}%
\pgfpathlineto{\pgfqpoint{0.000000in}{0.000000in}}%
\pgfusepath{stroke,fill}%
}%
\begin{pgfscope}%
\pgfsys@transformshift{3.286364in}{1.800000in}%
\pgfsys@useobject{currentmarker}{}%
\end{pgfscope}%
\end{pgfscope}%
\begin{pgfscope}%
\definecolor{textcolor}{rgb}{0.150000,0.150000,0.150000}%
\pgfsetstrokecolor{textcolor}%
\pgfsetfillcolor{textcolor}%
\pgftext[x=3.286364in,y=0.172222in,,top]{\color{textcolor}\sffamily\fontsize{8.000000}{9.600000}\selectfont 3.0}%
\end{pgfscope}%
\begin{pgfscope}%
\pgfpathrectangle{\pgfqpoint{3.286364in}{0.250000in}}{\pgfqpoint{2.113636in}{1.550000in}} %
\pgfusepath{clip}%
\pgfsetroundcap%
\pgfsetroundjoin%
\pgfsetlinewidth{0.803000pt}%
\definecolor{currentstroke}{rgb}{1.000000,1.000000,1.000000}%
\pgfsetstrokecolor{currentstroke}%
\pgfsetdash{}{0pt}%
\pgfpathmoveto{\pgfqpoint{3.550568in}{0.250000in}}%
\pgfpathlineto{\pgfqpoint{3.550568in}{1.800000in}}%
\pgfusepath{stroke}%
\end{pgfscope}%
\begin{pgfscope}%
\pgfsetbuttcap%
\pgfsetroundjoin%
\definecolor{currentfill}{rgb}{0.150000,0.150000,0.150000}%
\pgfsetfillcolor{currentfill}%
\pgfsetlinewidth{0.803000pt}%
\definecolor{currentstroke}{rgb}{0.150000,0.150000,0.150000}%
\pgfsetstrokecolor{currentstroke}%
\pgfsetdash{}{0pt}%
\pgfsys@defobject{currentmarker}{\pgfqpoint{0.000000in}{0.000000in}}{\pgfqpoint{0.000000in}{0.000000in}}{%
\pgfpathmoveto{\pgfqpoint{0.000000in}{0.000000in}}%
\pgfpathlineto{\pgfqpoint{0.000000in}{0.000000in}}%
\pgfusepath{stroke,fill}%
}%
\begin{pgfscope}%
\pgfsys@transformshift{3.550568in}{0.250000in}%
\pgfsys@useobject{currentmarker}{}%
\end{pgfscope}%
\end{pgfscope}%
\begin{pgfscope}%
\pgfsetbuttcap%
\pgfsetroundjoin%
\definecolor{currentfill}{rgb}{0.150000,0.150000,0.150000}%
\pgfsetfillcolor{currentfill}%
\pgfsetlinewidth{0.803000pt}%
\definecolor{currentstroke}{rgb}{0.150000,0.150000,0.150000}%
\pgfsetstrokecolor{currentstroke}%
\pgfsetdash{}{0pt}%
\pgfsys@defobject{currentmarker}{\pgfqpoint{0.000000in}{0.000000in}}{\pgfqpoint{0.000000in}{0.000000in}}{%
\pgfpathmoveto{\pgfqpoint{0.000000in}{0.000000in}}%
\pgfpathlineto{\pgfqpoint{0.000000in}{0.000000in}}%
\pgfusepath{stroke,fill}%
}%
\begin{pgfscope}%
\pgfsys@transformshift{3.550568in}{1.800000in}%
\pgfsys@useobject{currentmarker}{}%
\end{pgfscope}%
\end{pgfscope}%
\begin{pgfscope}%
\definecolor{textcolor}{rgb}{0.150000,0.150000,0.150000}%
\pgfsetstrokecolor{textcolor}%
\pgfsetfillcolor{textcolor}%
\pgftext[x=3.550568in,y=0.172222in,,top]{\color{textcolor}\sffamily\fontsize{8.000000}{9.600000}\selectfont 3.5}%
\end{pgfscope}%
\begin{pgfscope}%
\pgfpathrectangle{\pgfqpoint{3.286364in}{0.250000in}}{\pgfqpoint{2.113636in}{1.550000in}} %
\pgfusepath{clip}%
\pgfsetroundcap%
\pgfsetroundjoin%
\pgfsetlinewidth{0.803000pt}%
\definecolor{currentstroke}{rgb}{1.000000,1.000000,1.000000}%
\pgfsetstrokecolor{currentstroke}%
\pgfsetdash{}{0pt}%
\pgfpathmoveto{\pgfqpoint{3.814773in}{0.250000in}}%
\pgfpathlineto{\pgfqpoint{3.814773in}{1.800000in}}%
\pgfusepath{stroke}%
\end{pgfscope}%
\begin{pgfscope}%
\pgfsetbuttcap%
\pgfsetroundjoin%
\definecolor{currentfill}{rgb}{0.150000,0.150000,0.150000}%
\pgfsetfillcolor{currentfill}%
\pgfsetlinewidth{0.803000pt}%
\definecolor{currentstroke}{rgb}{0.150000,0.150000,0.150000}%
\pgfsetstrokecolor{currentstroke}%
\pgfsetdash{}{0pt}%
\pgfsys@defobject{currentmarker}{\pgfqpoint{0.000000in}{0.000000in}}{\pgfqpoint{0.000000in}{0.000000in}}{%
\pgfpathmoveto{\pgfqpoint{0.000000in}{0.000000in}}%
\pgfpathlineto{\pgfqpoint{0.000000in}{0.000000in}}%
\pgfusepath{stroke,fill}%
}%
\begin{pgfscope}%
\pgfsys@transformshift{3.814773in}{0.250000in}%
\pgfsys@useobject{currentmarker}{}%
\end{pgfscope}%
\end{pgfscope}%
\begin{pgfscope}%
\pgfsetbuttcap%
\pgfsetroundjoin%
\definecolor{currentfill}{rgb}{0.150000,0.150000,0.150000}%
\pgfsetfillcolor{currentfill}%
\pgfsetlinewidth{0.803000pt}%
\definecolor{currentstroke}{rgb}{0.150000,0.150000,0.150000}%
\pgfsetstrokecolor{currentstroke}%
\pgfsetdash{}{0pt}%
\pgfsys@defobject{currentmarker}{\pgfqpoint{0.000000in}{0.000000in}}{\pgfqpoint{0.000000in}{0.000000in}}{%
\pgfpathmoveto{\pgfqpoint{0.000000in}{0.000000in}}%
\pgfpathlineto{\pgfqpoint{0.000000in}{0.000000in}}%
\pgfusepath{stroke,fill}%
}%
\begin{pgfscope}%
\pgfsys@transformshift{3.814773in}{1.800000in}%
\pgfsys@useobject{currentmarker}{}%
\end{pgfscope}%
\end{pgfscope}%
\begin{pgfscope}%
\definecolor{textcolor}{rgb}{0.150000,0.150000,0.150000}%
\pgfsetstrokecolor{textcolor}%
\pgfsetfillcolor{textcolor}%
\pgftext[x=3.814773in,y=0.172222in,,top]{\color{textcolor}\sffamily\fontsize{8.000000}{9.600000}\selectfont 4.0}%
\end{pgfscope}%
\begin{pgfscope}%
\pgfpathrectangle{\pgfqpoint{3.286364in}{0.250000in}}{\pgfqpoint{2.113636in}{1.550000in}} %
\pgfusepath{clip}%
\pgfsetroundcap%
\pgfsetroundjoin%
\pgfsetlinewidth{0.803000pt}%
\definecolor{currentstroke}{rgb}{1.000000,1.000000,1.000000}%
\pgfsetstrokecolor{currentstroke}%
\pgfsetdash{}{0pt}%
\pgfpathmoveto{\pgfqpoint{4.078977in}{0.250000in}}%
\pgfpathlineto{\pgfqpoint{4.078977in}{1.800000in}}%
\pgfusepath{stroke}%
\end{pgfscope}%
\begin{pgfscope}%
\pgfsetbuttcap%
\pgfsetroundjoin%
\definecolor{currentfill}{rgb}{0.150000,0.150000,0.150000}%
\pgfsetfillcolor{currentfill}%
\pgfsetlinewidth{0.803000pt}%
\definecolor{currentstroke}{rgb}{0.150000,0.150000,0.150000}%
\pgfsetstrokecolor{currentstroke}%
\pgfsetdash{}{0pt}%
\pgfsys@defobject{currentmarker}{\pgfqpoint{0.000000in}{0.000000in}}{\pgfqpoint{0.000000in}{0.000000in}}{%
\pgfpathmoveto{\pgfqpoint{0.000000in}{0.000000in}}%
\pgfpathlineto{\pgfqpoint{0.000000in}{0.000000in}}%
\pgfusepath{stroke,fill}%
}%
\begin{pgfscope}%
\pgfsys@transformshift{4.078977in}{0.250000in}%
\pgfsys@useobject{currentmarker}{}%
\end{pgfscope}%
\end{pgfscope}%
\begin{pgfscope}%
\pgfsetbuttcap%
\pgfsetroundjoin%
\definecolor{currentfill}{rgb}{0.150000,0.150000,0.150000}%
\pgfsetfillcolor{currentfill}%
\pgfsetlinewidth{0.803000pt}%
\definecolor{currentstroke}{rgb}{0.150000,0.150000,0.150000}%
\pgfsetstrokecolor{currentstroke}%
\pgfsetdash{}{0pt}%
\pgfsys@defobject{currentmarker}{\pgfqpoint{0.000000in}{0.000000in}}{\pgfqpoint{0.000000in}{0.000000in}}{%
\pgfpathmoveto{\pgfqpoint{0.000000in}{0.000000in}}%
\pgfpathlineto{\pgfqpoint{0.000000in}{0.000000in}}%
\pgfusepath{stroke,fill}%
}%
\begin{pgfscope}%
\pgfsys@transformshift{4.078977in}{1.800000in}%
\pgfsys@useobject{currentmarker}{}%
\end{pgfscope}%
\end{pgfscope}%
\begin{pgfscope}%
\definecolor{textcolor}{rgb}{0.150000,0.150000,0.150000}%
\pgfsetstrokecolor{textcolor}%
\pgfsetfillcolor{textcolor}%
\pgftext[x=4.078977in,y=0.172222in,,top]{\color{textcolor}\sffamily\fontsize{8.000000}{9.600000}\selectfont 4.5}%
\end{pgfscope}%
\begin{pgfscope}%
\pgfpathrectangle{\pgfqpoint{3.286364in}{0.250000in}}{\pgfqpoint{2.113636in}{1.550000in}} %
\pgfusepath{clip}%
\pgfsetroundcap%
\pgfsetroundjoin%
\pgfsetlinewidth{0.803000pt}%
\definecolor{currentstroke}{rgb}{1.000000,1.000000,1.000000}%
\pgfsetstrokecolor{currentstroke}%
\pgfsetdash{}{0pt}%
\pgfpathmoveto{\pgfqpoint{4.343182in}{0.250000in}}%
\pgfpathlineto{\pgfqpoint{4.343182in}{1.800000in}}%
\pgfusepath{stroke}%
\end{pgfscope}%
\begin{pgfscope}%
\pgfsetbuttcap%
\pgfsetroundjoin%
\definecolor{currentfill}{rgb}{0.150000,0.150000,0.150000}%
\pgfsetfillcolor{currentfill}%
\pgfsetlinewidth{0.803000pt}%
\definecolor{currentstroke}{rgb}{0.150000,0.150000,0.150000}%
\pgfsetstrokecolor{currentstroke}%
\pgfsetdash{}{0pt}%
\pgfsys@defobject{currentmarker}{\pgfqpoint{0.000000in}{0.000000in}}{\pgfqpoint{0.000000in}{0.000000in}}{%
\pgfpathmoveto{\pgfqpoint{0.000000in}{0.000000in}}%
\pgfpathlineto{\pgfqpoint{0.000000in}{0.000000in}}%
\pgfusepath{stroke,fill}%
}%
\begin{pgfscope}%
\pgfsys@transformshift{4.343182in}{0.250000in}%
\pgfsys@useobject{currentmarker}{}%
\end{pgfscope}%
\end{pgfscope}%
\begin{pgfscope}%
\pgfsetbuttcap%
\pgfsetroundjoin%
\definecolor{currentfill}{rgb}{0.150000,0.150000,0.150000}%
\pgfsetfillcolor{currentfill}%
\pgfsetlinewidth{0.803000pt}%
\definecolor{currentstroke}{rgb}{0.150000,0.150000,0.150000}%
\pgfsetstrokecolor{currentstroke}%
\pgfsetdash{}{0pt}%
\pgfsys@defobject{currentmarker}{\pgfqpoint{0.000000in}{0.000000in}}{\pgfqpoint{0.000000in}{0.000000in}}{%
\pgfpathmoveto{\pgfqpoint{0.000000in}{0.000000in}}%
\pgfpathlineto{\pgfqpoint{0.000000in}{0.000000in}}%
\pgfusepath{stroke,fill}%
}%
\begin{pgfscope}%
\pgfsys@transformshift{4.343182in}{1.800000in}%
\pgfsys@useobject{currentmarker}{}%
\end{pgfscope}%
\end{pgfscope}%
\begin{pgfscope}%
\definecolor{textcolor}{rgb}{0.150000,0.150000,0.150000}%
\pgfsetstrokecolor{textcolor}%
\pgfsetfillcolor{textcolor}%
\pgftext[x=4.343182in,y=0.172222in,,top]{\color{textcolor}\sffamily\fontsize{8.000000}{9.600000}\selectfont 5.0}%
\end{pgfscope}%
\begin{pgfscope}%
\pgfpathrectangle{\pgfqpoint{3.286364in}{0.250000in}}{\pgfqpoint{2.113636in}{1.550000in}} %
\pgfusepath{clip}%
\pgfsetroundcap%
\pgfsetroundjoin%
\pgfsetlinewidth{0.803000pt}%
\definecolor{currentstroke}{rgb}{1.000000,1.000000,1.000000}%
\pgfsetstrokecolor{currentstroke}%
\pgfsetdash{}{0pt}%
\pgfpathmoveto{\pgfqpoint{4.607386in}{0.250000in}}%
\pgfpathlineto{\pgfqpoint{4.607386in}{1.800000in}}%
\pgfusepath{stroke}%
\end{pgfscope}%
\begin{pgfscope}%
\pgfsetbuttcap%
\pgfsetroundjoin%
\definecolor{currentfill}{rgb}{0.150000,0.150000,0.150000}%
\pgfsetfillcolor{currentfill}%
\pgfsetlinewidth{0.803000pt}%
\definecolor{currentstroke}{rgb}{0.150000,0.150000,0.150000}%
\pgfsetstrokecolor{currentstroke}%
\pgfsetdash{}{0pt}%
\pgfsys@defobject{currentmarker}{\pgfqpoint{0.000000in}{0.000000in}}{\pgfqpoint{0.000000in}{0.000000in}}{%
\pgfpathmoveto{\pgfqpoint{0.000000in}{0.000000in}}%
\pgfpathlineto{\pgfqpoint{0.000000in}{0.000000in}}%
\pgfusepath{stroke,fill}%
}%
\begin{pgfscope}%
\pgfsys@transformshift{4.607386in}{0.250000in}%
\pgfsys@useobject{currentmarker}{}%
\end{pgfscope}%
\end{pgfscope}%
\begin{pgfscope}%
\pgfsetbuttcap%
\pgfsetroundjoin%
\definecolor{currentfill}{rgb}{0.150000,0.150000,0.150000}%
\pgfsetfillcolor{currentfill}%
\pgfsetlinewidth{0.803000pt}%
\definecolor{currentstroke}{rgb}{0.150000,0.150000,0.150000}%
\pgfsetstrokecolor{currentstroke}%
\pgfsetdash{}{0pt}%
\pgfsys@defobject{currentmarker}{\pgfqpoint{0.000000in}{0.000000in}}{\pgfqpoint{0.000000in}{0.000000in}}{%
\pgfpathmoveto{\pgfqpoint{0.000000in}{0.000000in}}%
\pgfpathlineto{\pgfqpoint{0.000000in}{0.000000in}}%
\pgfusepath{stroke,fill}%
}%
\begin{pgfscope}%
\pgfsys@transformshift{4.607386in}{1.800000in}%
\pgfsys@useobject{currentmarker}{}%
\end{pgfscope}%
\end{pgfscope}%
\begin{pgfscope}%
\definecolor{textcolor}{rgb}{0.150000,0.150000,0.150000}%
\pgfsetstrokecolor{textcolor}%
\pgfsetfillcolor{textcolor}%
\pgftext[x=4.607386in,y=0.172222in,,top]{\color{textcolor}\sffamily\fontsize{8.000000}{9.600000}\selectfont 5.5}%
\end{pgfscope}%
\begin{pgfscope}%
\pgfpathrectangle{\pgfqpoint{3.286364in}{0.250000in}}{\pgfqpoint{2.113636in}{1.550000in}} %
\pgfusepath{clip}%
\pgfsetroundcap%
\pgfsetroundjoin%
\pgfsetlinewidth{0.803000pt}%
\definecolor{currentstroke}{rgb}{1.000000,1.000000,1.000000}%
\pgfsetstrokecolor{currentstroke}%
\pgfsetdash{}{0pt}%
\pgfpathmoveto{\pgfqpoint{4.871591in}{0.250000in}}%
\pgfpathlineto{\pgfqpoint{4.871591in}{1.800000in}}%
\pgfusepath{stroke}%
\end{pgfscope}%
\begin{pgfscope}%
\pgfsetbuttcap%
\pgfsetroundjoin%
\definecolor{currentfill}{rgb}{0.150000,0.150000,0.150000}%
\pgfsetfillcolor{currentfill}%
\pgfsetlinewidth{0.803000pt}%
\definecolor{currentstroke}{rgb}{0.150000,0.150000,0.150000}%
\pgfsetstrokecolor{currentstroke}%
\pgfsetdash{}{0pt}%
\pgfsys@defobject{currentmarker}{\pgfqpoint{0.000000in}{0.000000in}}{\pgfqpoint{0.000000in}{0.000000in}}{%
\pgfpathmoveto{\pgfqpoint{0.000000in}{0.000000in}}%
\pgfpathlineto{\pgfqpoint{0.000000in}{0.000000in}}%
\pgfusepath{stroke,fill}%
}%
\begin{pgfscope}%
\pgfsys@transformshift{4.871591in}{0.250000in}%
\pgfsys@useobject{currentmarker}{}%
\end{pgfscope}%
\end{pgfscope}%
\begin{pgfscope}%
\pgfsetbuttcap%
\pgfsetroundjoin%
\definecolor{currentfill}{rgb}{0.150000,0.150000,0.150000}%
\pgfsetfillcolor{currentfill}%
\pgfsetlinewidth{0.803000pt}%
\definecolor{currentstroke}{rgb}{0.150000,0.150000,0.150000}%
\pgfsetstrokecolor{currentstroke}%
\pgfsetdash{}{0pt}%
\pgfsys@defobject{currentmarker}{\pgfqpoint{0.000000in}{0.000000in}}{\pgfqpoint{0.000000in}{0.000000in}}{%
\pgfpathmoveto{\pgfqpoint{0.000000in}{0.000000in}}%
\pgfpathlineto{\pgfqpoint{0.000000in}{0.000000in}}%
\pgfusepath{stroke,fill}%
}%
\begin{pgfscope}%
\pgfsys@transformshift{4.871591in}{1.800000in}%
\pgfsys@useobject{currentmarker}{}%
\end{pgfscope}%
\end{pgfscope}%
\begin{pgfscope}%
\definecolor{textcolor}{rgb}{0.150000,0.150000,0.150000}%
\pgfsetstrokecolor{textcolor}%
\pgfsetfillcolor{textcolor}%
\pgftext[x=4.871591in,y=0.172222in,,top]{\color{textcolor}\sffamily\fontsize{8.000000}{9.600000}\selectfont 6.0}%
\end{pgfscope}%
\begin{pgfscope}%
\pgfpathrectangle{\pgfqpoint{3.286364in}{0.250000in}}{\pgfqpoint{2.113636in}{1.550000in}} %
\pgfusepath{clip}%
\pgfsetroundcap%
\pgfsetroundjoin%
\pgfsetlinewidth{0.803000pt}%
\definecolor{currentstroke}{rgb}{1.000000,1.000000,1.000000}%
\pgfsetstrokecolor{currentstroke}%
\pgfsetdash{}{0pt}%
\pgfpathmoveto{\pgfqpoint{5.135795in}{0.250000in}}%
\pgfpathlineto{\pgfqpoint{5.135795in}{1.800000in}}%
\pgfusepath{stroke}%
\end{pgfscope}%
\begin{pgfscope}%
\pgfsetbuttcap%
\pgfsetroundjoin%
\definecolor{currentfill}{rgb}{0.150000,0.150000,0.150000}%
\pgfsetfillcolor{currentfill}%
\pgfsetlinewidth{0.803000pt}%
\definecolor{currentstroke}{rgb}{0.150000,0.150000,0.150000}%
\pgfsetstrokecolor{currentstroke}%
\pgfsetdash{}{0pt}%
\pgfsys@defobject{currentmarker}{\pgfqpoint{0.000000in}{0.000000in}}{\pgfqpoint{0.000000in}{0.000000in}}{%
\pgfpathmoveto{\pgfqpoint{0.000000in}{0.000000in}}%
\pgfpathlineto{\pgfqpoint{0.000000in}{0.000000in}}%
\pgfusepath{stroke,fill}%
}%
\begin{pgfscope}%
\pgfsys@transformshift{5.135795in}{0.250000in}%
\pgfsys@useobject{currentmarker}{}%
\end{pgfscope}%
\end{pgfscope}%
\begin{pgfscope}%
\pgfsetbuttcap%
\pgfsetroundjoin%
\definecolor{currentfill}{rgb}{0.150000,0.150000,0.150000}%
\pgfsetfillcolor{currentfill}%
\pgfsetlinewidth{0.803000pt}%
\definecolor{currentstroke}{rgb}{0.150000,0.150000,0.150000}%
\pgfsetstrokecolor{currentstroke}%
\pgfsetdash{}{0pt}%
\pgfsys@defobject{currentmarker}{\pgfqpoint{0.000000in}{0.000000in}}{\pgfqpoint{0.000000in}{0.000000in}}{%
\pgfpathmoveto{\pgfqpoint{0.000000in}{0.000000in}}%
\pgfpathlineto{\pgfqpoint{0.000000in}{0.000000in}}%
\pgfusepath{stroke,fill}%
}%
\begin{pgfscope}%
\pgfsys@transformshift{5.135795in}{1.800000in}%
\pgfsys@useobject{currentmarker}{}%
\end{pgfscope}%
\end{pgfscope}%
\begin{pgfscope}%
\definecolor{textcolor}{rgb}{0.150000,0.150000,0.150000}%
\pgfsetstrokecolor{textcolor}%
\pgfsetfillcolor{textcolor}%
\pgftext[x=5.135795in,y=0.172222in,,top]{\color{textcolor}\sffamily\fontsize{8.000000}{9.600000}\selectfont 6.5}%
\end{pgfscope}%
\begin{pgfscope}%
\pgfpathrectangle{\pgfqpoint{3.286364in}{0.250000in}}{\pgfqpoint{2.113636in}{1.550000in}} %
\pgfusepath{clip}%
\pgfsetroundcap%
\pgfsetroundjoin%
\pgfsetlinewidth{0.803000pt}%
\definecolor{currentstroke}{rgb}{1.000000,1.000000,1.000000}%
\pgfsetstrokecolor{currentstroke}%
\pgfsetdash{}{0pt}%
\pgfpathmoveto{\pgfqpoint{5.400000in}{0.250000in}}%
\pgfpathlineto{\pgfqpoint{5.400000in}{1.800000in}}%
\pgfusepath{stroke}%
\end{pgfscope}%
\begin{pgfscope}%
\pgfsetbuttcap%
\pgfsetroundjoin%
\definecolor{currentfill}{rgb}{0.150000,0.150000,0.150000}%
\pgfsetfillcolor{currentfill}%
\pgfsetlinewidth{0.803000pt}%
\definecolor{currentstroke}{rgb}{0.150000,0.150000,0.150000}%
\pgfsetstrokecolor{currentstroke}%
\pgfsetdash{}{0pt}%
\pgfsys@defobject{currentmarker}{\pgfqpoint{0.000000in}{0.000000in}}{\pgfqpoint{0.000000in}{0.000000in}}{%
\pgfpathmoveto{\pgfqpoint{0.000000in}{0.000000in}}%
\pgfpathlineto{\pgfqpoint{0.000000in}{0.000000in}}%
\pgfusepath{stroke,fill}%
}%
\begin{pgfscope}%
\pgfsys@transformshift{5.400000in}{0.250000in}%
\pgfsys@useobject{currentmarker}{}%
\end{pgfscope}%
\end{pgfscope}%
\begin{pgfscope}%
\pgfsetbuttcap%
\pgfsetroundjoin%
\definecolor{currentfill}{rgb}{0.150000,0.150000,0.150000}%
\pgfsetfillcolor{currentfill}%
\pgfsetlinewidth{0.803000pt}%
\definecolor{currentstroke}{rgb}{0.150000,0.150000,0.150000}%
\pgfsetstrokecolor{currentstroke}%
\pgfsetdash{}{0pt}%
\pgfsys@defobject{currentmarker}{\pgfqpoint{0.000000in}{0.000000in}}{\pgfqpoint{0.000000in}{0.000000in}}{%
\pgfpathmoveto{\pgfqpoint{0.000000in}{0.000000in}}%
\pgfpathlineto{\pgfqpoint{0.000000in}{0.000000in}}%
\pgfusepath{stroke,fill}%
}%
\begin{pgfscope}%
\pgfsys@transformshift{5.400000in}{1.800000in}%
\pgfsys@useobject{currentmarker}{}%
\end{pgfscope}%
\end{pgfscope}%
\begin{pgfscope}%
\definecolor{textcolor}{rgb}{0.150000,0.150000,0.150000}%
\pgfsetstrokecolor{textcolor}%
\pgfsetfillcolor{textcolor}%
\pgftext[x=5.400000in,y=0.172222in,,top]{\color{textcolor}\sffamily\fontsize{8.000000}{9.600000}\selectfont 7.0}%
\end{pgfscope}%
\begin{pgfscope}%
\pgfpathrectangle{\pgfqpoint{3.286364in}{0.250000in}}{\pgfqpoint{2.113636in}{1.550000in}} %
\pgfusepath{clip}%
\pgfsetroundcap%
\pgfsetroundjoin%
\pgfsetlinewidth{0.803000pt}%
\definecolor{currentstroke}{rgb}{1.000000,1.000000,1.000000}%
\pgfsetstrokecolor{currentstroke}%
\pgfsetdash{}{0pt}%
\pgfpathmoveto{\pgfqpoint{3.286364in}{0.250000in}}%
\pgfpathlineto{\pgfqpoint{5.400000in}{0.250000in}}%
\pgfusepath{stroke}%
\end{pgfscope}%
\begin{pgfscope}%
\pgfsetbuttcap%
\pgfsetroundjoin%
\definecolor{currentfill}{rgb}{0.150000,0.150000,0.150000}%
\pgfsetfillcolor{currentfill}%
\pgfsetlinewidth{0.803000pt}%
\definecolor{currentstroke}{rgb}{0.150000,0.150000,0.150000}%
\pgfsetstrokecolor{currentstroke}%
\pgfsetdash{}{0pt}%
\pgfsys@defobject{currentmarker}{\pgfqpoint{0.000000in}{0.000000in}}{\pgfqpoint{0.000000in}{0.000000in}}{%
\pgfpathmoveto{\pgfqpoint{0.000000in}{0.000000in}}%
\pgfpathlineto{\pgfqpoint{0.000000in}{0.000000in}}%
\pgfusepath{stroke,fill}%
}%
\begin{pgfscope}%
\pgfsys@transformshift{3.286364in}{0.250000in}%
\pgfsys@useobject{currentmarker}{}%
\end{pgfscope}%
\end{pgfscope}%
\begin{pgfscope}%
\pgfsetbuttcap%
\pgfsetroundjoin%
\definecolor{currentfill}{rgb}{0.150000,0.150000,0.150000}%
\pgfsetfillcolor{currentfill}%
\pgfsetlinewidth{0.803000pt}%
\definecolor{currentstroke}{rgb}{0.150000,0.150000,0.150000}%
\pgfsetstrokecolor{currentstroke}%
\pgfsetdash{}{0pt}%
\pgfsys@defobject{currentmarker}{\pgfqpoint{0.000000in}{0.000000in}}{\pgfqpoint{0.000000in}{0.000000in}}{%
\pgfpathmoveto{\pgfqpoint{0.000000in}{0.000000in}}%
\pgfpathlineto{\pgfqpoint{0.000000in}{0.000000in}}%
\pgfusepath{stroke,fill}%
}%
\begin{pgfscope}%
\pgfsys@transformshift{5.400000in}{0.250000in}%
\pgfsys@useobject{currentmarker}{}%
\end{pgfscope}%
\end{pgfscope}%
\begin{pgfscope}%
\definecolor{textcolor}{rgb}{0.150000,0.150000,0.150000}%
\pgfsetstrokecolor{textcolor}%
\pgfsetfillcolor{textcolor}%
\pgftext[x=3.208586in,y=0.250000in,right,]{\color{textcolor}\sffamily\fontsize{8.000000}{9.600000}\selectfont −1.0}%
\end{pgfscope}%
\begin{pgfscope}%
\pgfpathrectangle{\pgfqpoint{3.286364in}{0.250000in}}{\pgfqpoint{2.113636in}{1.550000in}} %
\pgfusepath{clip}%
\pgfsetroundcap%
\pgfsetroundjoin%
\pgfsetlinewidth{0.803000pt}%
\definecolor{currentstroke}{rgb}{1.000000,1.000000,1.000000}%
\pgfsetstrokecolor{currentstroke}%
\pgfsetdash{}{0pt}%
\pgfpathmoveto{\pgfqpoint{3.286364in}{0.560000in}}%
\pgfpathlineto{\pgfqpoint{5.400000in}{0.560000in}}%
\pgfusepath{stroke}%
\end{pgfscope}%
\begin{pgfscope}%
\pgfsetbuttcap%
\pgfsetroundjoin%
\definecolor{currentfill}{rgb}{0.150000,0.150000,0.150000}%
\pgfsetfillcolor{currentfill}%
\pgfsetlinewidth{0.803000pt}%
\definecolor{currentstroke}{rgb}{0.150000,0.150000,0.150000}%
\pgfsetstrokecolor{currentstroke}%
\pgfsetdash{}{0pt}%
\pgfsys@defobject{currentmarker}{\pgfqpoint{0.000000in}{0.000000in}}{\pgfqpoint{0.000000in}{0.000000in}}{%
\pgfpathmoveto{\pgfqpoint{0.000000in}{0.000000in}}%
\pgfpathlineto{\pgfqpoint{0.000000in}{0.000000in}}%
\pgfusepath{stroke,fill}%
}%
\begin{pgfscope}%
\pgfsys@transformshift{3.286364in}{0.560000in}%
\pgfsys@useobject{currentmarker}{}%
\end{pgfscope}%
\end{pgfscope}%
\begin{pgfscope}%
\pgfsetbuttcap%
\pgfsetroundjoin%
\definecolor{currentfill}{rgb}{0.150000,0.150000,0.150000}%
\pgfsetfillcolor{currentfill}%
\pgfsetlinewidth{0.803000pt}%
\definecolor{currentstroke}{rgb}{0.150000,0.150000,0.150000}%
\pgfsetstrokecolor{currentstroke}%
\pgfsetdash{}{0pt}%
\pgfsys@defobject{currentmarker}{\pgfqpoint{0.000000in}{0.000000in}}{\pgfqpoint{0.000000in}{0.000000in}}{%
\pgfpathmoveto{\pgfqpoint{0.000000in}{0.000000in}}%
\pgfpathlineto{\pgfqpoint{0.000000in}{0.000000in}}%
\pgfusepath{stroke,fill}%
}%
\begin{pgfscope}%
\pgfsys@transformshift{5.400000in}{0.560000in}%
\pgfsys@useobject{currentmarker}{}%
\end{pgfscope}%
\end{pgfscope}%
\begin{pgfscope}%
\definecolor{textcolor}{rgb}{0.150000,0.150000,0.150000}%
\pgfsetstrokecolor{textcolor}%
\pgfsetfillcolor{textcolor}%
\pgftext[x=3.208586in,y=0.560000in,right,]{\color{textcolor}\sffamily\fontsize{8.000000}{9.600000}\selectfont −0.5}%
\end{pgfscope}%
\begin{pgfscope}%
\pgfpathrectangle{\pgfqpoint{3.286364in}{0.250000in}}{\pgfqpoint{2.113636in}{1.550000in}} %
\pgfusepath{clip}%
\pgfsetroundcap%
\pgfsetroundjoin%
\pgfsetlinewidth{0.803000pt}%
\definecolor{currentstroke}{rgb}{1.000000,1.000000,1.000000}%
\pgfsetstrokecolor{currentstroke}%
\pgfsetdash{}{0pt}%
\pgfpathmoveto{\pgfqpoint{3.286364in}{0.870000in}}%
\pgfpathlineto{\pgfqpoint{5.400000in}{0.870000in}}%
\pgfusepath{stroke}%
\end{pgfscope}%
\begin{pgfscope}%
\pgfsetbuttcap%
\pgfsetroundjoin%
\definecolor{currentfill}{rgb}{0.150000,0.150000,0.150000}%
\pgfsetfillcolor{currentfill}%
\pgfsetlinewidth{0.803000pt}%
\definecolor{currentstroke}{rgb}{0.150000,0.150000,0.150000}%
\pgfsetstrokecolor{currentstroke}%
\pgfsetdash{}{0pt}%
\pgfsys@defobject{currentmarker}{\pgfqpoint{0.000000in}{0.000000in}}{\pgfqpoint{0.000000in}{0.000000in}}{%
\pgfpathmoveto{\pgfqpoint{0.000000in}{0.000000in}}%
\pgfpathlineto{\pgfqpoint{0.000000in}{0.000000in}}%
\pgfusepath{stroke,fill}%
}%
\begin{pgfscope}%
\pgfsys@transformshift{3.286364in}{0.870000in}%
\pgfsys@useobject{currentmarker}{}%
\end{pgfscope}%
\end{pgfscope}%
\begin{pgfscope}%
\pgfsetbuttcap%
\pgfsetroundjoin%
\definecolor{currentfill}{rgb}{0.150000,0.150000,0.150000}%
\pgfsetfillcolor{currentfill}%
\pgfsetlinewidth{0.803000pt}%
\definecolor{currentstroke}{rgb}{0.150000,0.150000,0.150000}%
\pgfsetstrokecolor{currentstroke}%
\pgfsetdash{}{0pt}%
\pgfsys@defobject{currentmarker}{\pgfqpoint{0.000000in}{0.000000in}}{\pgfqpoint{0.000000in}{0.000000in}}{%
\pgfpathmoveto{\pgfqpoint{0.000000in}{0.000000in}}%
\pgfpathlineto{\pgfqpoint{0.000000in}{0.000000in}}%
\pgfusepath{stroke,fill}%
}%
\begin{pgfscope}%
\pgfsys@transformshift{5.400000in}{0.870000in}%
\pgfsys@useobject{currentmarker}{}%
\end{pgfscope}%
\end{pgfscope}%
\begin{pgfscope}%
\definecolor{textcolor}{rgb}{0.150000,0.150000,0.150000}%
\pgfsetstrokecolor{textcolor}%
\pgfsetfillcolor{textcolor}%
\pgftext[x=3.208586in,y=0.870000in,right,]{\color{textcolor}\sffamily\fontsize{8.000000}{9.600000}\selectfont 0.0}%
\end{pgfscope}%
\begin{pgfscope}%
\pgfpathrectangle{\pgfqpoint{3.286364in}{0.250000in}}{\pgfqpoint{2.113636in}{1.550000in}} %
\pgfusepath{clip}%
\pgfsetroundcap%
\pgfsetroundjoin%
\pgfsetlinewidth{0.803000pt}%
\definecolor{currentstroke}{rgb}{1.000000,1.000000,1.000000}%
\pgfsetstrokecolor{currentstroke}%
\pgfsetdash{}{0pt}%
\pgfpathmoveto{\pgfqpoint{3.286364in}{1.180000in}}%
\pgfpathlineto{\pgfqpoint{5.400000in}{1.180000in}}%
\pgfusepath{stroke}%
\end{pgfscope}%
\begin{pgfscope}%
\pgfsetbuttcap%
\pgfsetroundjoin%
\definecolor{currentfill}{rgb}{0.150000,0.150000,0.150000}%
\pgfsetfillcolor{currentfill}%
\pgfsetlinewidth{0.803000pt}%
\definecolor{currentstroke}{rgb}{0.150000,0.150000,0.150000}%
\pgfsetstrokecolor{currentstroke}%
\pgfsetdash{}{0pt}%
\pgfsys@defobject{currentmarker}{\pgfqpoint{0.000000in}{0.000000in}}{\pgfqpoint{0.000000in}{0.000000in}}{%
\pgfpathmoveto{\pgfqpoint{0.000000in}{0.000000in}}%
\pgfpathlineto{\pgfqpoint{0.000000in}{0.000000in}}%
\pgfusepath{stroke,fill}%
}%
\begin{pgfscope}%
\pgfsys@transformshift{3.286364in}{1.180000in}%
\pgfsys@useobject{currentmarker}{}%
\end{pgfscope}%
\end{pgfscope}%
\begin{pgfscope}%
\pgfsetbuttcap%
\pgfsetroundjoin%
\definecolor{currentfill}{rgb}{0.150000,0.150000,0.150000}%
\pgfsetfillcolor{currentfill}%
\pgfsetlinewidth{0.803000pt}%
\definecolor{currentstroke}{rgb}{0.150000,0.150000,0.150000}%
\pgfsetstrokecolor{currentstroke}%
\pgfsetdash{}{0pt}%
\pgfsys@defobject{currentmarker}{\pgfqpoint{0.000000in}{0.000000in}}{\pgfqpoint{0.000000in}{0.000000in}}{%
\pgfpathmoveto{\pgfqpoint{0.000000in}{0.000000in}}%
\pgfpathlineto{\pgfqpoint{0.000000in}{0.000000in}}%
\pgfusepath{stroke,fill}%
}%
\begin{pgfscope}%
\pgfsys@transformshift{5.400000in}{1.180000in}%
\pgfsys@useobject{currentmarker}{}%
\end{pgfscope}%
\end{pgfscope}%
\begin{pgfscope}%
\definecolor{textcolor}{rgb}{0.150000,0.150000,0.150000}%
\pgfsetstrokecolor{textcolor}%
\pgfsetfillcolor{textcolor}%
\pgftext[x=3.208586in,y=1.180000in,right,]{\color{textcolor}\sffamily\fontsize{8.000000}{9.600000}\selectfont 0.5}%
\end{pgfscope}%
\begin{pgfscope}%
\pgfpathrectangle{\pgfqpoint{3.286364in}{0.250000in}}{\pgfqpoint{2.113636in}{1.550000in}} %
\pgfusepath{clip}%
\pgfsetroundcap%
\pgfsetroundjoin%
\pgfsetlinewidth{0.803000pt}%
\definecolor{currentstroke}{rgb}{1.000000,1.000000,1.000000}%
\pgfsetstrokecolor{currentstroke}%
\pgfsetdash{}{0pt}%
\pgfpathmoveto{\pgfqpoint{3.286364in}{1.490000in}}%
\pgfpathlineto{\pgfqpoint{5.400000in}{1.490000in}}%
\pgfusepath{stroke}%
\end{pgfscope}%
\begin{pgfscope}%
\pgfsetbuttcap%
\pgfsetroundjoin%
\definecolor{currentfill}{rgb}{0.150000,0.150000,0.150000}%
\pgfsetfillcolor{currentfill}%
\pgfsetlinewidth{0.803000pt}%
\definecolor{currentstroke}{rgb}{0.150000,0.150000,0.150000}%
\pgfsetstrokecolor{currentstroke}%
\pgfsetdash{}{0pt}%
\pgfsys@defobject{currentmarker}{\pgfqpoint{0.000000in}{0.000000in}}{\pgfqpoint{0.000000in}{0.000000in}}{%
\pgfpathmoveto{\pgfqpoint{0.000000in}{0.000000in}}%
\pgfpathlineto{\pgfqpoint{0.000000in}{0.000000in}}%
\pgfusepath{stroke,fill}%
}%
\begin{pgfscope}%
\pgfsys@transformshift{3.286364in}{1.490000in}%
\pgfsys@useobject{currentmarker}{}%
\end{pgfscope}%
\end{pgfscope}%
\begin{pgfscope}%
\pgfsetbuttcap%
\pgfsetroundjoin%
\definecolor{currentfill}{rgb}{0.150000,0.150000,0.150000}%
\pgfsetfillcolor{currentfill}%
\pgfsetlinewidth{0.803000pt}%
\definecolor{currentstroke}{rgb}{0.150000,0.150000,0.150000}%
\pgfsetstrokecolor{currentstroke}%
\pgfsetdash{}{0pt}%
\pgfsys@defobject{currentmarker}{\pgfqpoint{0.000000in}{0.000000in}}{\pgfqpoint{0.000000in}{0.000000in}}{%
\pgfpathmoveto{\pgfqpoint{0.000000in}{0.000000in}}%
\pgfpathlineto{\pgfqpoint{0.000000in}{0.000000in}}%
\pgfusepath{stroke,fill}%
}%
\begin{pgfscope}%
\pgfsys@transformshift{5.400000in}{1.490000in}%
\pgfsys@useobject{currentmarker}{}%
\end{pgfscope}%
\end{pgfscope}%
\begin{pgfscope}%
\definecolor{textcolor}{rgb}{0.150000,0.150000,0.150000}%
\pgfsetstrokecolor{textcolor}%
\pgfsetfillcolor{textcolor}%
\pgftext[x=3.208586in,y=1.490000in,right,]{\color{textcolor}\sffamily\fontsize{8.000000}{9.600000}\selectfont 1.0}%
\end{pgfscope}%
\begin{pgfscope}%
\pgfpathrectangle{\pgfqpoint{3.286364in}{0.250000in}}{\pgfqpoint{2.113636in}{1.550000in}} %
\pgfusepath{clip}%
\pgfsetroundcap%
\pgfsetroundjoin%
\pgfsetlinewidth{0.803000pt}%
\definecolor{currentstroke}{rgb}{1.000000,1.000000,1.000000}%
\pgfsetstrokecolor{currentstroke}%
\pgfsetdash{}{0pt}%
\pgfpathmoveto{\pgfqpoint{3.286364in}{1.800000in}}%
\pgfpathlineto{\pgfqpoint{5.400000in}{1.800000in}}%
\pgfusepath{stroke}%
\end{pgfscope}%
\begin{pgfscope}%
\pgfsetbuttcap%
\pgfsetroundjoin%
\definecolor{currentfill}{rgb}{0.150000,0.150000,0.150000}%
\pgfsetfillcolor{currentfill}%
\pgfsetlinewidth{0.803000pt}%
\definecolor{currentstroke}{rgb}{0.150000,0.150000,0.150000}%
\pgfsetstrokecolor{currentstroke}%
\pgfsetdash{}{0pt}%
\pgfsys@defobject{currentmarker}{\pgfqpoint{0.000000in}{0.000000in}}{\pgfqpoint{0.000000in}{0.000000in}}{%
\pgfpathmoveto{\pgfqpoint{0.000000in}{0.000000in}}%
\pgfpathlineto{\pgfqpoint{0.000000in}{0.000000in}}%
\pgfusepath{stroke,fill}%
}%
\begin{pgfscope}%
\pgfsys@transformshift{3.286364in}{1.800000in}%
\pgfsys@useobject{currentmarker}{}%
\end{pgfscope}%
\end{pgfscope}%
\begin{pgfscope}%
\pgfsetbuttcap%
\pgfsetroundjoin%
\definecolor{currentfill}{rgb}{0.150000,0.150000,0.150000}%
\pgfsetfillcolor{currentfill}%
\pgfsetlinewidth{0.803000pt}%
\definecolor{currentstroke}{rgb}{0.150000,0.150000,0.150000}%
\pgfsetstrokecolor{currentstroke}%
\pgfsetdash{}{0pt}%
\pgfsys@defobject{currentmarker}{\pgfqpoint{0.000000in}{0.000000in}}{\pgfqpoint{0.000000in}{0.000000in}}{%
\pgfpathmoveto{\pgfqpoint{0.000000in}{0.000000in}}%
\pgfpathlineto{\pgfqpoint{0.000000in}{0.000000in}}%
\pgfusepath{stroke,fill}%
}%
\begin{pgfscope}%
\pgfsys@transformshift{5.400000in}{1.800000in}%
\pgfsys@useobject{currentmarker}{}%
\end{pgfscope}%
\end{pgfscope}%
\begin{pgfscope}%
\definecolor{textcolor}{rgb}{0.150000,0.150000,0.150000}%
\pgfsetstrokecolor{textcolor}%
\pgfsetfillcolor{textcolor}%
\pgftext[x=3.208586in,y=1.800000in,right,]{\color{textcolor}\sffamily\fontsize{8.000000}{9.600000}\selectfont 1.5}%
\end{pgfscope}%
\begin{pgfscope}%
\pgfpathrectangle{\pgfqpoint{3.286364in}{0.250000in}}{\pgfqpoint{2.113636in}{1.550000in}} %
\pgfusepath{clip}%
\pgfsetroundcap%
\pgfsetroundjoin%
\pgfsetlinewidth{1.405250pt}%
\definecolor{currentstroke}{rgb}{0.298039,0.447059,0.690196}%
\pgfsetstrokecolor{currentstroke}%
\pgfsetdash{}{0pt}%
\pgfpathmoveto{\pgfqpoint{4.078977in}{0.453995in}}%
\pgfpathlineto{\pgfqpoint{3.714375in}{1.018391in}}%
\pgfpathlineto{\pgfqpoint{4.274489in}{0.864943in}}%
\pgfpathlineto{\pgfqpoint{4.686648in}{0.863856in}}%
\pgfpathlineto{\pgfqpoint{3.872898in}{0.586043in}}%
\pgfpathlineto{\pgfqpoint{5.125227in}{1.473810in}}%
\pgfpathlineto{\pgfqpoint{3.587557in}{1.234844in}}%
\pgfpathlineto{\pgfqpoint{4.797614in}{0.736401in}}%
\pgfpathlineto{\pgfqpoint{3.703807in}{1.061767in}}%
\pgfpathlineto{\pgfqpoint{4.364318in}{1.117102in}}%
\pgfpathlineto{\pgfqpoint{3.666818in}{1.198786in}}%
\pgfpathlineto{\pgfqpoint{3.650966in}{1.249184in}}%
\pgfpathlineto{\pgfqpoint{5.193920in}{1.484343in}}%
\pgfpathlineto{\pgfqpoint{3.936307in}{0.534430in}}%
\pgfpathlineto{\pgfqpoint{4.438295in}{1.059963in}}%
\pgfpathlineto{\pgfqpoint{4.337898in}{1.068892in}}%
\pgfpathlineto{\pgfqpoint{3.508295in}{1.295594in}}%
\pgfpathlineto{\pgfqpoint{3.772500in}{0.780876in}}%
\pgfpathlineto{\pgfqpoint{3.867614in}{0.588639in}}%
\pgfpathlineto{\pgfqpoint{3.576989in}{1.226118in}}%
\pgfpathlineto{\pgfqpoint{5.030114in}{1.335218in}}%
\pgfpathlineto{\pgfqpoint{4.359034in}{1.108210in}}%
\pgfpathlineto{\pgfqpoint{5.024830in}{1.319226in}}%
\pgfpathlineto{\pgfqpoint{4.728920in}{0.775355in}}%
\pgfpathlineto{\pgfqpoint{4.744773in}{0.766103in}}%
\pgfpathlineto{\pgfqpoint{4.691932in}{0.854720in}}%
\pgfpathlineto{\pgfqpoint{3.571705in}{1.227757in}}%
\pgfpathlineto{\pgfqpoint{5.014261in}{1.294145in}}%
\pgfpathlineto{\pgfqpoint{4.348466in}{1.089586in}}%
\pgfpathlineto{\pgfqpoint{5.067102in}{1.437041in}}%
\pgfusepath{stroke}%
\end{pgfscope}%
\begin{pgfscope}%
\pgfpathrectangle{\pgfqpoint{3.286364in}{0.250000in}}{\pgfqpoint{2.113636in}{1.550000in}} %
\pgfusepath{clip}%
\pgfsetroundcap%
\pgfsetroundjoin%
\pgfsetlinewidth{1.405250pt}%
\definecolor{currentstroke}{rgb}{0.333333,0.658824,0.407843}%
\pgfsetstrokecolor{currentstroke}%
\pgfsetdash{}{0pt}%
\pgfpathmoveto{\pgfqpoint{4.078977in}{1.117710in}}%
\pgfpathlineto{\pgfqpoint{3.714375in}{1.366704in}}%
\pgfpathlineto{\pgfqpoint{4.274489in}{1.254745in}}%
\pgfpathlineto{\pgfqpoint{4.686648in}{0.598739in}}%
\pgfpathlineto{\pgfqpoint{3.872898in}{1.447311in}}%
\pgfpathlineto{\pgfqpoint{5.125227in}{1.619403in}}%
\pgfpathlineto{\pgfqpoint{3.587557in}{1.364725in}}%
\pgfpathlineto{\pgfqpoint{4.797614in}{0.687959in}}%
\pgfpathlineto{\pgfqpoint{3.703807in}{1.370873in}}%
\pgfpathlineto{\pgfqpoint{4.364318in}{1.368399in}}%
\pgfpathlineto{\pgfqpoint{3.666818in}{1.382406in}}%
\pgfpathlineto{\pgfqpoint{3.650966in}{1.403854in}}%
\pgfpathlineto{\pgfqpoint{5.193920in}{1.708750in}}%
\pgfpathlineto{\pgfqpoint{3.936307in}{1.466456in}}%
\pgfpathlineto{\pgfqpoint{4.438295in}{1.323888in}}%
\pgfpathlineto{\pgfqpoint{4.337898in}{1.365769in}}%
\pgfpathlineto{\pgfqpoint{3.508295in}{1.302704in}}%
\pgfpathlineto{\pgfqpoint{3.772500in}{1.400612in}}%
\pgfpathlineto{\pgfqpoint{3.867614in}{1.436096in}}%
\pgfpathlineto{\pgfqpoint{3.576989in}{1.358480in}}%
\pgfpathlineto{\pgfqpoint{5.030114in}{1.520206in}}%
\pgfpathlineto{\pgfqpoint{4.359034in}{1.369560in}}%
\pgfpathlineto{\pgfqpoint{5.024830in}{1.514206in}}%
\pgfpathlineto{\pgfqpoint{4.728920in}{0.518860in}}%
\pgfpathlineto{\pgfqpoint{4.744773in}{0.528030in}}%
\pgfpathlineto{\pgfqpoint{4.691932in}{0.585319in}}%
\pgfpathlineto{\pgfqpoint{3.571705in}{1.360003in}}%
\pgfpathlineto{\pgfqpoint{5.014261in}{1.492975in}}%
\pgfpathlineto{\pgfqpoint{4.348466in}{1.369791in}}%
\pgfpathlineto{\pgfqpoint{5.067102in}{1.556137in}}%
\pgfusepath{stroke}%
\end{pgfscope}%
\begin{pgfscope}%
\pgfpathrectangle{\pgfqpoint{3.286364in}{0.250000in}}{\pgfqpoint{2.113636in}{1.550000in}} %
\pgfusepath{clip}%
\pgfsetroundcap%
\pgfsetroundjoin%
\pgfsetlinewidth{1.405250pt}%
\definecolor{currentstroke}{rgb}{0.768627,0.305882,0.321569}%
\pgfsetstrokecolor{currentstroke}%
\pgfsetdash{}{0pt}%
\pgfpathmoveto{\pgfqpoint{4.078977in}{0.720147in}}%
\pgfpathlineto{\pgfqpoint{3.714375in}{0.953227in}}%
\pgfpathlineto{\pgfqpoint{4.274489in}{1.200534in}}%
\pgfpathlineto{\pgfqpoint{4.686648in}{0.756953in}}%
\pgfpathlineto{\pgfqpoint{3.872898in}{0.669449in}}%
\pgfpathlineto{\pgfqpoint{5.125227in}{0.758222in}}%
\pgfpathlineto{\pgfqpoint{3.587557in}{1.023172in}}%
\pgfpathlineto{\pgfqpoint{4.797614in}{0.675384in}}%
\pgfpathlineto{\pgfqpoint{3.703807in}{0.960994in}}%
\pgfpathlineto{\pgfqpoint{4.364318in}{1.317566in}}%
\pgfpathlineto{\pgfqpoint{3.666818in}{0.990756in}}%
\pgfpathlineto{\pgfqpoint{3.650966in}{0.992146in}}%
\pgfpathlineto{\pgfqpoint{5.193920in}{0.694662in}}%
\pgfpathlineto{\pgfqpoint{3.936307in}{0.554895in}}%
\pgfpathlineto{\pgfqpoint{4.438295in}{1.076739in}}%
\pgfpathlineto{\pgfqpoint{4.337898in}{1.287038in}}%
\pgfpathlineto{\pgfqpoint{3.508295in}{1.001038in}}%
\pgfpathlineto{\pgfqpoint{3.772500in}{0.959294in}}%
\pgfpathlineto{\pgfqpoint{3.867614in}{0.679918in}}%
\pgfpathlineto{\pgfqpoint{3.576989in}{1.022415in}}%
\pgfpathlineto{\pgfqpoint{5.030114in}{0.764188in}}%
\pgfpathlineto{\pgfqpoint{4.359034in}{1.318264in}}%
\pgfpathlineto{\pgfqpoint{5.024830in}{0.766333in}}%
\pgfpathlineto{\pgfqpoint{4.728920in}{0.780293in}}%
\pgfpathlineto{\pgfqpoint{4.744773in}{0.770450in}}%
\pgfpathlineto{\pgfqpoint{4.691932in}{0.744593in}}%
\pgfpathlineto{\pgfqpoint{3.571705in}{1.023751in}}%
\pgfpathlineto{\pgfqpoint{5.014261in}{0.764957in}}%
\pgfpathlineto{\pgfqpoint{4.348466in}{1.305019in}}%
\pgfpathlineto{\pgfqpoint{5.067102in}{0.750201in}}%
\pgfusepath{stroke}%
\end{pgfscope}%
\begin{pgfscope}%
\pgfsetrectcap%
\pgfsetmiterjoin%
\pgfsetlinewidth{0.000000pt}%
\definecolor{currentstroke}{rgb}{1.000000,1.000000,1.000000}%
\pgfsetstrokecolor{currentstroke}%
\pgfsetdash{}{0pt}%
\pgfpathmoveto{\pgfqpoint{3.286364in}{0.250000in}}%
\pgfpathlineto{\pgfqpoint{5.400000in}{0.250000in}}%
\pgfusepath{}%
\end{pgfscope}%
\begin{pgfscope}%
\pgfsetrectcap%
\pgfsetmiterjoin%
\pgfsetlinewidth{0.000000pt}%
\definecolor{currentstroke}{rgb}{1.000000,1.000000,1.000000}%
\pgfsetstrokecolor{currentstroke}%
\pgfsetdash{}{0pt}%
\pgfpathmoveto{\pgfqpoint{3.286364in}{0.250000in}}%
\pgfpathlineto{\pgfqpoint{3.286364in}{1.800000in}}%
\pgfusepath{}%
\end{pgfscope}%
\begin{pgfscope}%
\pgfsetrectcap%
\pgfsetmiterjoin%
\pgfsetlinewidth{0.000000pt}%
\definecolor{currentstroke}{rgb}{1.000000,1.000000,1.000000}%
\pgfsetstrokecolor{currentstroke}%
\pgfsetdash{}{0pt}%
\pgfpathmoveto{\pgfqpoint{3.286364in}{1.800000in}}%
\pgfpathlineto{\pgfqpoint{5.400000in}{1.800000in}}%
\pgfusepath{}%
\end{pgfscope}%
\begin{pgfscope}%
\pgfsetrectcap%
\pgfsetmiterjoin%
\pgfsetlinewidth{0.000000pt}%
\definecolor{currentstroke}{rgb}{1.000000,1.000000,1.000000}%
\pgfsetstrokecolor{currentstroke}%
\pgfsetdash{}{0pt}%
\pgfpathmoveto{\pgfqpoint{5.400000in}{0.250000in}}%
\pgfpathlineto{\pgfqpoint{5.400000in}{1.800000in}}%
\pgfusepath{}%
\end{pgfscope}%
\end{pgfpicture}%
\makeatother%
\endgroup%

  \caption{Sample from a Gaussian process with kernel evaluated using the wing
  length of the helicopter. The kernel for the left sampling of five functions
  is \emph{Matern32} with $\sigma^2=1.0$ and $\theta=0.2$. The second plot is
  the same kernel, this time with only three functions and $\theta=0.8$}
  \label{fig_gpsamples}
\end{figure}

\paragraph{Gaussian process regression for prediction of a toy function.} In
this point we are asked to approximate the function $f(x) = \sin(3\pi x)$.

\begin{figure}
  \centering
  %% Creator: Matplotlib, PGF backend
%%
%% To include the figure in your LaTeX document, write
%%   \input{<filename>.pgf}
%%
%% Make sure the required packages are loaded in your preamble
%%   \usepackage{pgf}
%%
%% Figures using additional raster images can only be included by \input if
%% they are in the same directory as the main LaTeX file. For loading figures
%% from other directories you can use the `import` package
%%   \usepackage{import}
%% and then include the figures with
%%   \import{<path to file>}{<filename>.pgf}
%%
%% Matplotlib used the following preamble
%%   \usepackage[utf8x]{inputenc}
%%   \usepackage[T1]{fontenc}
%%   \usepackage{cmbright}
%%
\begingroup%
\makeatletter%
\begin{pgfpicture}%
\pgfpathrectangle{\pgfpointorigin}{\pgfqpoint{6.000000in}{2.000000in}}%
\pgfusepath{use as bounding box, clip}%
\begin{pgfscope}%
\pgfsetbuttcap%
\pgfsetmiterjoin%
\definecolor{currentfill}{rgb}{1.000000,1.000000,1.000000}%
\pgfsetfillcolor{currentfill}%
\pgfsetlinewidth{0.000000pt}%
\definecolor{currentstroke}{rgb}{1.000000,1.000000,1.000000}%
\pgfsetstrokecolor{currentstroke}%
\pgfsetdash{}{0pt}%
\pgfpathmoveto{\pgfqpoint{0.000000in}{0.000000in}}%
\pgfpathlineto{\pgfqpoint{6.000000in}{0.000000in}}%
\pgfpathlineto{\pgfqpoint{6.000000in}{2.000000in}}%
\pgfpathlineto{\pgfqpoint{0.000000in}{2.000000in}}%
\pgfpathclose%
\pgfusepath{fill}%
\end{pgfscope}%
\begin{pgfscope}%
\pgfsetbuttcap%
\pgfsetmiterjoin%
\definecolor{currentfill}{rgb}{0.917647,0.917647,0.949020}%
\pgfsetfillcolor{currentfill}%
\pgfsetlinewidth{0.000000pt}%
\definecolor{currentstroke}{rgb}{0.000000,0.000000,0.000000}%
\pgfsetstrokecolor{currentstroke}%
\pgfsetstrokeopacity{0.000000}%
\pgfsetdash{}{0pt}%
\pgfpathmoveto{\pgfqpoint{0.750000in}{0.250000in}}%
\pgfpathlineto{\pgfqpoint{2.863636in}{0.250000in}}%
\pgfpathlineto{\pgfqpoint{2.863636in}{1.800000in}}%
\pgfpathlineto{\pgfqpoint{0.750000in}{1.800000in}}%
\pgfpathclose%
\pgfusepath{fill}%
\end{pgfscope}%
\begin{pgfscope}%
\pgfpathrectangle{\pgfqpoint{0.750000in}{0.250000in}}{\pgfqpoint{2.113636in}{1.550000in}} %
\pgfusepath{clip}%
\pgfsetroundcap%
\pgfsetroundjoin%
\pgfsetlinewidth{0.803000pt}%
\definecolor{currentstroke}{rgb}{1.000000,1.000000,1.000000}%
\pgfsetstrokecolor{currentstroke}%
\pgfsetdash{}{0pt}%
\pgfpathmoveto{\pgfqpoint{0.750000in}{0.250000in}}%
\pgfpathlineto{\pgfqpoint{0.750000in}{1.800000in}}%
\pgfusepath{stroke}%
\end{pgfscope}%
\begin{pgfscope}%
\pgfsetbuttcap%
\pgfsetroundjoin%
\definecolor{currentfill}{rgb}{0.150000,0.150000,0.150000}%
\pgfsetfillcolor{currentfill}%
\pgfsetlinewidth{0.803000pt}%
\definecolor{currentstroke}{rgb}{0.150000,0.150000,0.150000}%
\pgfsetstrokecolor{currentstroke}%
\pgfsetdash{}{0pt}%
\pgfsys@defobject{currentmarker}{\pgfqpoint{0.000000in}{0.000000in}}{\pgfqpoint{0.000000in}{0.000000in}}{%
\pgfpathmoveto{\pgfqpoint{0.000000in}{0.000000in}}%
\pgfpathlineto{\pgfqpoint{0.000000in}{0.000000in}}%
\pgfusepath{stroke,fill}%
}%
\begin{pgfscope}%
\pgfsys@transformshift{0.750000in}{0.250000in}%
\pgfsys@useobject{currentmarker}{}%
\end{pgfscope}%
\end{pgfscope}%
\begin{pgfscope}%
\pgfsetbuttcap%
\pgfsetroundjoin%
\definecolor{currentfill}{rgb}{0.150000,0.150000,0.150000}%
\pgfsetfillcolor{currentfill}%
\pgfsetlinewidth{0.803000pt}%
\definecolor{currentstroke}{rgb}{0.150000,0.150000,0.150000}%
\pgfsetstrokecolor{currentstroke}%
\pgfsetdash{}{0pt}%
\pgfsys@defobject{currentmarker}{\pgfqpoint{0.000000in}{0.000000in}}{\pgfqpoint{0.000000in}{0.000000in}}{%
\pgfpathmoveto{\pgfqpoint{0.000000in}{0.000000in}}%
\pgfpathlineto{\pgfqpoint{0.000000in}{0.000000in}}%
\pgfusepath{stroke,fill}%
}%
\begin{pgfscope}%
\pgfsys@transformshift{0.750000in}{1.800000in}%
\pgfsys@useobject{currentmarker}{}%
\end{pgfscope}%
\end{pgfscope}%
\begin{pgfscope}%
\definecolor{textcolor}{rgb}{0.150000,0.150000,0.150000}%
\pgfsetstrokecolor{textcolor}%
\pgfsetfillcolor{textcolor}%
\pgftext[x=0.750000in,y=0.172222in,,top]{\color{textcolor}\sffamily\fontsize{8.000000}{9.600000}\selectfont 0}%
\end{pgfscope}%
\begin{pgfscope}%
\pgfpathrectangle{\pgfqpoint{0.750000in}{0.250000in}}{\pgfqpoint{2.113636in}{1.550000in}} %
\pgfusepath{clip}%
\pgfsetroundcap%
\pgfsetroundjoin%
\pgfsetlinewidth{0.803000pt}%
\definecolor{currentstroke}{rgb}{1.000000,1.000000,1.000000}%
\pgfsetstrokecolor{currentstroke}%
\pgfsetdash{}{0pt}%
\pgfpathmoveto{\pgfqpoint{1.051948in}{0.250000in}}%
\pgfpathlineto{\pgfqpoint{1.051948in}{1.800000in}}%
\pgfusepath{stroke}%
\end{pgfscope}%
\begin{pgfscope}%
\pgfsetbuttcap%
\pgfsetroundjoin%
\definecolor{currentfill}{rgb}{0.150000,0.150000,0.150000}%
\pgfsetfillcolor{currentfill}%
\pgfsetlinewidth{0.803000pt}%
\definecolor{currentstroke}{rgb}{0.150000,0.150000,0.150000}%
\pgfsetstrokecolor{currentstroke}%
\pgfsetdash{}{0pt}%
\pgfsys@defobject{currentmarker}{\pgfqpoint{0.000000in}{0.000000in}}{\pgfqpoint{0.000000in}{0.000000in}}{%
\pgfpathmoveto{\pgfqpoint{0.000000in}{0.000000in}}%
\pgfpathlineto{\pgfqpoint{0.000000in}{0.000000in}}%
\pgfusepath{stroke,fill}%
}%
\begin{pgfscope}%
\pgfsys@transformshift{1.051948in}{0.250000in}%
\pgfsys@useobject{currentmarker}{}%
\end{pgfscope}%
\end{pgfscope}%
\begin{pgfscope}%
\pgfsetbuttcap%
\pgfsetroundjoin%
\definecolor{currentfill}{rgb}{0.150000,0.150000,0.150000}%
\pgfsetfillcolor{currentfill}%
\pgfsetlinewidth{0.803000pt}%
\definecolor{currentstroke}{rgb}{0.150000,0.150000,0.150000}%
\pgfsetstrokecolor{currentstroke}%
\pgfsetdash{}{0pt}%
\pgfsys@defobject{currentmarker}{\pgfqpoint{0.000000in}{0.000000in}}{\pgfqpoint{0.000000in}{0.000000in}}{%
\pgfpathmoveto{\pgfqpoint{0.000000in}{0.000000in}}%
\pgfpathlineto{\pgfqpoint{0.000000in}{0.000000in}}%
\pgfusepath{stroke,fill}%
}%
\begin{pgfscope}%
\pgfsys@transformshift{1.051948in}{1.800000in}%
\pgfsys@useobject{currentmarker}{}%
\end{pgfscope}%
\end{pgfscope}%
\begin{pgfscope}%
\definecolor{textcolor}{rgb}{0.150000,0.150000,0.150000}%
\pgfsetstrokecolor{textcolor}%
\pgfsetfillcolor{textcolor}%
\pgftext[x=1.051948in,y=0.172222in,,top]{\color{textcolor}\sffamily\fontsize{8.000000}{9.600000}\selectfont 1}%
\end{pgfscope}%
\begin{pgfscope}%
\pgfpathrectangle{\pgfqpoint{0.750000in}{0.250000in}}{\pgfqpoint{2.113636in}{1.550000in}} %
\pgfusepath{clip}%
\pgfsetroundcap%
\pgfsetroundjoin%
\pgfsetlinewidth{0.803000pt}%
\definecolor{currentstroke}{rgb}{1.000000,1.000000,1.000000}%
\pgfsetstrokecolor{currentstroke}%
\pgfsetdash{}{0pt}%
\pgfpathmoveto{\pgfqpoint{1.353896in}{0.250000in}}%
\pgfpathlineto{\pgfqpoint{1.353896in}{1.800000in}}%
\pgfusepath{stroke}%
\end{pgfscope}%
\begin{pgfscope}%
\pgfsetbuttcap%
\pgfsetroundjoin%
\definecolor{currentfill}{rgb}{0.150000,0.150000,0.150000}%
\pgfsetfillcolor{currentfill}%
\pgfsetlinewidth{0.803000pt}%
\definecolor{currentstroke}{rgb}{0.150000,0.150000,0.150000}%
\pgfsetstrokecolor{currentstroke}%
\pgfsetdash{}{0pt}%
\pgfsys@defobject{currentmarker}{\pgfqpoint{0.000000in}{0.000000in}}{\pgfqpoint{0.000000in}{0.000000in}}{%
\pgfpathmoveto{\pgfqpoint{0.000000in}{0.000000in}}%
\pgfpathlineto{\pgfqpoint{0.000000in}{0.000000in}}%
\pgfusepath{stroke,fill}%
}%
\begin{pgfscope}%
\pgfsys@transformshift{1.353896in}{0.250000in}%
\pgfsys@useobject{currentmarker}{}%
\end{pgfscope}%
\end{pgfscope}%
\begin{pgfscope}%
\pgfsetbuttcap%
\pgfsetroundjoin%
\definecolor{currentfill}{rgb}{0.150000,0.150000,0.150000}%
\pgfsetfillcolor{currentfill}%
\pgfsetlinewidth{0.803000pt}%
\definecolor{currentstroke}{rgb}{0.150000,0.150000,0.150000}%
\pgfsetstrokecolor{currentstroke}%
\pgfsetdash{}{0pt}%
\pgfsys@defobject{currentmarker}{\pgfqpoint{0.000000in}{0.000000in}}{\pgfqpoint{0.000000in}{0.000000in}}{%
\pgfpathmoveto{\pgfqpoint{0.000000in}{0.000000in}}%
\pgfpathlineto{\pgfqpoint{0.000000in}{0.000000in}}%
\pgfusepath{stroke,fill}%
}%
\begin{pgfscope}%
\pgfsys@transformshift{1.353896in}{1.800000in}%
\pgfsys@useobject{currentmarker}{}%
\end{pgfscope}%
\end{pgfscope}%
\begin{pgfscope}%
\definecolor{textcolor}{rgb}{0.150000,0.150000,0.150000}%
\pgfsetstrokecolor{textcolor}%
\pgfsetfillcolor{textcolor}%
\pgftext[x=1.353896in,y=0.172222in,,top]{\color{textcolor}\sffamily\fontsize{8.000000}{9.600000}\selectfont 2}%
\end{pgfscope}%
\begin{pgfscope}%
\pgfpathrectangle{\pgfqpoint{0.750000in}{0.250000in}}{\pgfqpoint{2.113636in}{1.550000in}} %
\pgfusepath{clip}%
\pgfsetroundcap%
\pgfsetroundjoin%
\pgfsetlinewidth{0.803000pt}%
\definecolor{currentstroke}{rgb}{1.000000,1.000000,1.000000}%
\pgfsetstrokecolor{currentstroke}%
\pgfsetdash{}{0pt}%
\pgfpathmoveto{\pgfqpoint{1.655844in}{0.250000in}}%
\pgfpathlineto{\pgfqpoint{1.655844in}{1.800000in}}%
\pgfusepath{stroke}%
\end{pgfscope}%
\begin{pgfscope}%
\pgfsetbuttcap%
\pgfsetroundjoin%
\definecolor{currentfill}{rgb}{0.150000,0.150000,0.150000}%
\pgfsetfillcolor{currentfill}%
\pgfsetlinewidth{0.803000pt}%
\definecolor{currentstroke}{rgb}{0.150000,0.150000,0.150000}%
\pgfsetstrokecolor{currentstroke}%
\pgfsetdash{}{0pt}%
\pgfsys@defobject{currentmarker}{\pgfqpoint{0.000000in}{0.000000in}}{\pgfqpoint{0.000000in}{0.000000in}}{%
\pgfpathmoveto{\pgfqpoint{0.000000in}{0.000000in}}%
\pgfpathlineto{\pgfqpoint{0.000000in}{0.000000in}}%
\pgfusepath{stroke,fill}%
}%
\begin{pgfscope}%
\pgfsys@transformshift{1.655844in}{0.250000in}%
\pgfsys@useobject{currentmarker}{}%
\end{pgfscope}%
\end{pgfscope}%
\begin{pgfscope}%
\pgfsetbuttcap%
\pgfsetroundjoin%
\definecolor{currentfill}{rgb}{0.150000,0.150000,0.150000}%
\pgfsetfillcolor{currentfill}%
\pgfsetlinewidth{0.803000pt}%
\definecolor{currentstroke}{rgb}{0.150000,0.150000,0.150000}%
\pgfsetstrokecolor{currentstroke}%
\pgfsetdash{}{0pt}%
\pgfsys@defobject{currentmarker}{\pgfqpoint{0.000000in}{0.000000in}}{\pgfqpoint{0.000000in}{0.000000in}}{%
\pgfpathmoveto{\pgfqpoint{0.000000in}{0.000000in}}%
\pgfpathlineto{\pgfqpoint{0.000000in}{0.000000in}}%
\pgfusepath{stroke,fill}%
}%
\begin{pgfscope}%
\pgfsys@transformshift{1.655844in}{1.800000in}%
\pgfsys@useobject{currentmarker}{}%
\end{pgfscope}%
\end{pgfscope}%
\begin{pgfscope}%
\definecolor{textcolor}{rgb}{0.150000,0.150000,0.150000}%
\pgfsetstrokecolor{textcolor}%
\pgfsetfillcolor{textcolor}%
\pgftext[x=1.655844in,y=0.172222in,,top]{\color{textcolor}\sffamily\fontsize{8.000000}{9.600000}\selectfont 3}%
\end{pgfscope}%
\begin{pgfscope}%
\pgfpathrectangle{\pgfqpoint{0.750000in}{0.250000in}}{\pgfqpoint{2.113636in}{1.550000in}} %
\pgfusepath{clip}%
\pgfsetroundcap%
\pgfsetroundjoin%
\pgfsetlinewidth{0.803000pt}%
\definecolor{currentstroke}{rgb}{1.000000,1.000000,1.000000}%
\pgfsetstrokecolor{currentstroke}%
\pgfsetdash{}{0pt}%
\pgfpathmoveto{\pgfqpoint{1.957792in}{0.250000in}}%
\pgfpathlineto{\pgfqpoint{1.957792in}{1.800000in}}%
\pgfusepath{stroke}%
\end{pgfscope}%
\begin{pgfscope}%
\pgfsetbuttcap%
\pgfsetroundjoin%
\definecolor{currentfill}{rgb}{0.150000,0.150000,0.150000}%
\pgfsetfillcolor{currentfill}%
\pgfsetlinewidth{0.803000pt}%
\definecolor{currentstroke}{rgb}{0.150000,0.150000,0.150000}%
\pgfsetstrokecolor{currentstroke}%
\pgfsetdash{}{0pt}%
\pgfsys@defobject{currentmarker}{\pgfqpoint{0.000000in}{0.000000in}}{\pgfqpoint{0.000000in}{0.000000in}}{%
\pgfpathmoveto{\pgfqpoint{0.000000in}{0.000000in}}%
\pgfpathlineto{\pgfqpoint{0.000000in}{0.000000in}}%
\pgfusepath{stroke,fill}%
}%
\begin{pgfscope}%
\pgfsys@transformshift{1.957792in}{0.250000in}%
\pgfsys@useobject{currentmarker}{}%
\end{pgfscope}%
\end{pgfscope}%
\begin{pgfscope}%
\pgfsetbuttcap%
\pgfsetroundjoin%
\definecolor{currentfill}{rgb}{0.150000,0.150000,0.150000}%
\pgfsetfillcolor{currentfill}%
\pgfsetlinewidth{0.803000pt}%
\definecolor{currentstroke}{rgb}{0.150000,0.150000,0.150000}%
\pgfsetstrokecolor{currentstroke}%
\pgfsetdash{}{0pt}%
\pgfsys@defobject{currentmarker}{\pgfqpoint{0.000000in}{0.000000in}}{\pgfqpoint{0.000000in}{0.000000in}}{%
\pgfpathmoveto{\pgfqpoint{0.000000in}{0.000000in}}%
\pgfpathlineto{\pgfqpoint{0.000000in}{0.000000in}}%
\pgfusepath{stroke,fill}%
}%
\begin{pgfscope}%
\pgfsys@transformshift{1.957792in}{1.800000in}%
\pgfsys@useobject{currentmarker}{}%
\end{pgfscope}%
\end{pgfscope}%
\begin{pgfscope}%
\definecolor{textcolor}{rgb}{0.150000,0.150000,0.150000}%
\pgfsetstrokecolor{textcolor}%
\pgfsetfillcolor{textcolor}%
\pgftext[x=1.957792in,y=0.172222in,,top]{\color{textcolor}\sffamily\fontsize{8.000000}{9.600000}\selectfont 4}%
\end{pgfscope}%
\begin{pgfscope}%
\pgfpathrectangle{\pgfqpoint{0.750000in}{0.250000in}}{\pgfqpoint{2.113636in}{1.550000in}} %
\pgfusepath{clip}%
\pgfsetroundcap%
\pgfsetroundjoin%
\pgfsetlinewidth{0.803000pt}%
\definecolor{currentstroke}{rgb}{1.000000,1.000000,1.000000}%
\pgfsetstrokecolor{currentstroke}%
\pgfsetdash{}{0pt}%
\pgfpathmoveto{\pgfqpoint{2.259740in}{0.250000in}}%
\pgfpathlineto{\pgfqpoint{2.259740in}{1.800000in}}%
\pgfusepath{stroke}%
\end{pgfscope}%
\begin{pgfscope}%
\pgfsetbuttcap%
\pgfsetroundjoin%
\definecolor{currentfill}{rgb}{0.150000,0.150000,0.150000}%
\pgfsetfillcolor{currentfill}%
\pgfsetlinewidth{0.803000pt}%
\definecolor{currentstroke}{rgb}{0.150000,0.150000,0.150000}%
\pgfsetstrokecolor{currentstroke}%
\pgfsetdash{}{0pt}%
\pgfsys@defobject{currentmarker}{\pgfqpoint{0.000000in}{0.000000in}}{\pgfqpoint{0.000000in}{0.000000in}}{%
\pgfpathmoveto{\pgfqpoint{0.000000in}{0.000000in}}%
\pgfpathlineto{\pgfqpoint{0.000000in}{0.000000in}}%
\pgfusepath{stroke,fill}%
}%
\begin{pgfscope}%
\pgfsys@transformshift{2.259740in}{0.250000in}%
\pgfsys@useobject{currentmarker}{}%
\end{pgfscope}%
\end{pgfscope}%
\begin{pgfscope}%
\pgfsetbuttcap%
\pgfsetroundjoin%
\definecolor{currentfill}{rgb}{0.150000,0.150000,0.150000}%
\pgfsetfillcolor{currentfill}%
\pgfsetlinewidth{0.803000pt}%
\definecolor{currentstroke}{rgb}{0.150000,0.150000,0.150000}%
\pgfsetstrokecolor{currentstroke}%
\pgfsetdash{}{0pt}%
\pgfsys@defobject{currentmarker}{\pgfqpoint{0.000000in}{0.000000in}}{\pgfqpoint{0.000000in}{0.000000in}}{%
\pgfpathmoveto{\pgfqpoint{0.000000in}{0.000000in}}%
\pgfpathlineto{\pgfqpoint{0.000000in}{0.000000in}}%
\pgfusepath{stroke,fill}%
}%
\begin{pgfscope}%
\pgfsys@transformshift{2.259740in}{1.800000in}%
\pgfsys@useobject{currentmarker}{}%
\end{pgfscope}%
\end{pgfscope}%
\begin{pgfscope}%
\definecolor{textcolor}{rgb}{0.150000,0.150000,0.150000}%
\pgfsetstrokecolor{textcolor}%
\pgfsetfillcolor{textcolor}%
\pgftext[x=2.259740in,y=0.172222in,,top]{\color{textcolor}\sffamily\fontsize{8.000000}{9.600000}\selectfont 5}%
\end{pgfscope}%
\begin{pgfscope}%
\pgfpathrectangle{\pgfqpoint{0.750000in}{0.250000in}}{\pgfqpoint{2.113636in}{1.550000in}} %
\pgfusepath{clip}%
\pgfsetroundcap%
\pgfsetroundjoin%
\pgfsetlinewidth{0.803000pt}%
\definecolor{currentstroke}{rgb}{1.000000,1.000000,1.000000}%
\pgfsetstrokecolor{currentstroke}%
\pgfsetdash{}{0pt}%
\pgfpathmoveto{\pgfqpoint{2.561688in}{0.250000in}}%
\pgfpathlineto{\pgfqpoint{2.561688in}{1.800000in}}%
\pgfusepath{stroke}%
\end{pgfscope}%
\begin{pgfscope}%
\pgfsetbuttcap%
\pgfsetroundjoin%
\definecolor{currentfill}{rgb}{0.150000,0.150000,0.150000}%
\pgfsetfillcolor{currentfill}%
\pgfsetlinewidth{0.803000pt}%
\definecolor{currentstroke}{rgb}{0.150000,0.150000,0.150000}%
\pgfsetstrokecolor{currentstroke}%
\pgfsetdash{}{0pt}%
\pgfsys@defobject{currentmarker}{\pgfqpoint{0.000000in}{0.000000in}}{\pgfqpoint{0.000000in}{0.000000in}}{%
\pgfpathmoveto{\pgfqpoint{0.000000in}{0.000000in}}%
\pgfpathlineto{\pgfqpoint{0.000000in}{0.000000in}}%
\pgfusepath{stroke,fill}%
}%
\begin{pgfscope}%
\pgfsys@transformshift{2.561688in}{0.250000in}%
\pgfsys@useobject{currentmarker}{}%
\end{pgfscope}%
\end{pgfscope}%
\begin{pgfscope}%
\pgfsetbuttcap%
\pgfsetroundjoin%
\definecolor{currentfill}{rgb}{0.150000,0.150000,0.150000}%
\pgfsetfillcolor{currentfill}%
\pgfsetlinewidth{0.803000pt}%
\definecolor{currentstroke}{rgb}{0.150000,0.150000,0.150000}%
\pgfsetstrokecolor{currentstroke}%
\pgfsetdash{}{0pt}%
\pgfsys@defobject{currentmarker}{\pgfqpoint{0.000000in}{0.000000in}}{\pgfqpoint{0.000000in}{0.000000in}}{%
\pgfpathmoveto{\pgfqpoint{0.000000in}{0.000000in}}%
\pgfpathlineto{\pgfqpoint{0.000000in}{0.000000in}}%
\pgfusepath{stroke,fill}%
}%
\begin{pgfscope}%
\pgfsys@transformshift{2.561688in}{1.800000in}%
\pgfsys@useobject{currentmarker}{}%
\end{pgfscope}%
\end{pgfscope}%
\begin{pgfscope}%
\definecolor{textcolor}{rgb}{0.150000,0.150000,0.150000}%
\pgfsetstrokecolor{textcolor}%
\pgfsetfillcolor{textcolor}%
\pgftext[x=2.561688in,y=0.172222in,,top]{\color{textcolor}\sffamily\fontsize{8.000000}{9.600000}\selectfont 6}%
\end{pgfscope}%
\begin{pgfscope}%
\pgfpathrectangle{\pgfqpoint{0.750000in}{0.250000in}}{\pgfqpoint{2.113636in}{1.550000in}} %
\pgfusepath{clip}%
\pgfsetroundcap%
\pgfsetroundjoin%
\pgfsetlinewidth{0.803000pt}%
\definecolor{currentstroke}{rgb}{1.000000,1.000000,1.000000}%
\pgfsetstrokecolor{currentstroke}%
\pgfsetdash{}{0pt}%
\pgfpathmoveto{\pgfqpoint{2.863636in}{0.250000in}}%
\pgfpathlineto{\pgfqpoint{2.863636in}{1.800000in}}%
\pgfusepath{stroke}%
\end{pgfscope}%
\begin{pgfscope}%
\pgfsetbuttcap%
\pgfsetroundjoin%
\definecolor{currentfill}{rgb}{0.150000,0.150000,0.150000}%
\pgfsetfillcolor{currentfill}%
\pgfsetlinewidth{0.803000pt}%
\definecolor{currentstroke}{rgb}{0.150000,0.150000,0.150000}%
\pgfsetstrokecolor{currentstroke}%
\pgfsetdash{}{0pt}%
\pgfsys@defobject{currentmarker}{\pgfqpoint{0.000000in}{0.000000in}}{\pgfqpoint{0.000000in}{0.000000in}}{%
\pgfpathmoveto{\pgfqpoint{0.000000in}{0.000000in}}%
\pgfpathlineto{\pgfqpoint{0.000000in}{0.000000in}}%
\pgfusepath{stroke,fill}%
}%
\begin{pgfscope}%
\pgfsys@transformshift{2.863636in}{0.250000in}%
\pgfsys@useobject{currentmarker}{}%
\end{pgfscope}%
\end{pgfscope}%
\begin{pgfscope}%
\pgfsetbuttcap%
\pgfsetroundjoin%
\definecolor{currentfill}{rgb}{0.150000,0.150000,0.150000}%
\pgfsetfillcolor{currentfill}%
\pgfsetlinewidth{0.803000pt}%
\definecolor{currentstroke}{rgb}{0.150000,0.150000,0.150000}%
\pgfsetstrokecolor{currentstroke}%
\pgfsetdash{}{0pt}%
\pgfsys@defobject{currentmarker}{\pgfqpoint{0.000000in}{0.000000in}}{\pgfqpoint{0.000000in}{0.000000in}}{%
\pgfpathmoveto{\pgfqpoint{0.000000in}{0.000000in}}%
\pgfpathlineto{\pgfqpoint{0.000000in}{0.000000in}}%
\pgfusepath{stroke,fill}%
}%
\begin{pgfscope}%
\pgfsys@transformshift{2.863636in}{1.800000in}%
\pgfsys@useobject{currentmarker}{}%
\end{pgfscope}%
\end{pgfscope}%
\begin{pgfscope}%
\definecolor{textcolor}{rgb}{0.150000,0.150000,0.150000}%
\pgfsetstrokecolor{textcolor}%
\pgfsetfillcolor{textcolor}%
\pgftext[x=2.863636in,y=0.172222in,,top]{\color{textcolor}\sffamily\fontsize{8.000000}{9.600000}\selectfont 7}%
\end{pgfscope}%
\begin{pgfscope}%
\pgfpathrectangle{\pgfqpoint{0.750000in}{0.250000in}}{\pgfqpoint{2.113636in}{1.550000in}} %
\pgfusepath{clip}%
\pgfsetroundcap%
\pgfsetroundjoin%
\pgfsetlinewidth{0.803000pt}%
\definecolor{currentstroke}{rgb}{1.000000,1.000000,1.000000}%
\pgfsetstrokecolor{currentstroke}%
\pgfsetdash{}{0pt}%
\pgfpathmoveto{\pgfqpoint{0.750000in}{0.250000in}}%
\pgfpathlineto{\pgfqpoint{2.863636in}{0.250000in}}%
\pgfusepath{stroke}%
\end{pgfscope}%
\begin{pgfscope}%
\pgfsetbuttcap%
\pgfsetroundjoin%
\definecolor{currentfill}{rgb}{0.150000,0.150000,0.150000}%
\pgfsetfillcolor{currentfill}%
\pgfsetlinewidth{0.803000pt}%
\definecolor{currentstroke}{rgb}{0.150000,0.150000,0.150000}%
\pgfsetstrokecolor{currentstroke}%
\pgfsetdash{}{0pt}%
\pgfsys@defobject{currentmarker}{\pgfqpoint{0.000000in}{0.000000in}}{\pgfqpoint{0.000000in}{0.000000in}}{%
\pgfpathmoveto{\pgfqpoint{0.000000in}{0.000000in}}%
\pgfpathlineto{\pgfqpoint{0.000000in}{0.000000in}}%
\pgfusepath{stroke,fill}%
}%
\begin{pgfscope}%
\pgfsys@transformshift{0.750000in}{0.250000in}%
\pgfsys@useobject{currentmarker}{}%
\end{pgfscope}%
\end{pgfscope}%
\begin{pgfscope}%
\pgfsetbuttcap%
\pgfsetroundjoin%
\definecolor{currentfill}{rgb}{0.150000,0.150000,0.150000}%
\pgfsetfillcolor{currentfill}%
\pgfsetlinewidth{0.803000pt}%
\definecolor{currentstroke}{rgb}{0.150000,0.150000,0.150000}%
\pgfsetstrokecolor{currentstroke}%
\pgfsetdash{}{0pt}%
\pgfsys@defobject{currentmarker}{\pgfqpoint{0.000000in}{0.000000in}}{\pgfqpoint{0.000000in}{0.000000in}}{%
\pgfpathmoveto{\pgfqpoint{0.000000in}{0.000000in}}%
\pgfpathlineto{\pgfqpoint{0.000000in}{0.000000in}}%
\pgfusepath{stroke,fill}%
}%
\begin{pgfscope}%
\pgfsys@transformshift{2.863636in}{0.250000in}%
\pgfsys@useobject{currentmarker}{}%
\end{pgfscope}%
\end{pgfscope}%
\begin{pgfscope}%
\definecolor{textcolor}{rgb}{0.150000,0.150000,0.150000}%
\pgfsetstrokecolor{textcolor}%
\pgfsetfillcolor{textcolor}%
\pgftext[x=0.672222in,y=0.250000in,right,]{\color{textcolor}\sffamily\fontsize{8.000000}{9.600000}\selectfont −2}%
\end{pgfscope}%
\begin{pgfscope}%
\pgfpathrectangle{\pgfqpoint{0.750000in}{0.250000in}}{\pgfqpoint{2.113636in}{1.550000in}} %
\pgfusepath{clip}%
\pgfsetroundcap%
\pgfsetroundjoin%
\pgfsetlinewidth{0.803000pt}%
\definecolor{currentstroke}{rgb}{1.000000,1.000000,1.000000}%
\pgfsetstrokecolor{currentstroke}%
\pgfsetdash{}{0pt}%
\pgfpathmoveto{\pgfqpoint{0.750000in}{0.443750in}}%
\pgfpathlineto{\pgfqpoint{2.863636in}{0.443750in}}%
\pgfusepath{stroke}%
\end{pgfscope}%
\begin{pgfscope}%
\pgfsetbuttcap%
\pgfsetroundjoin%
\definecolor{currentfill}{rgb}{0.150000,0.150000,0.150000}%
\pgfsetfillcolor{currentfill}%
\pgfsetlinewidth{0.803000pt}%
\definecolor{currentstroke}{rgb}{0.150000,0.150000,0.150000}%
\pgfsetstrokecolor{currentstroke}%
\pgfsetdash{}{0pt}%
\pgfsys@defobject{currentmarker}{\pgfqpoint{0.000000in}{0.000000in}}{\pgfqpoint{0.000000in}{0.000000in}}{%
\pgfpathmoveto{\pgfqpoint{0.000000in}{0.000000in}}%
\pgfpathlineto{\pgfqpoint{0.000000in}{0.000000in}}%
\pgfusepath{stroke,fill}%
}%
\begin{pgfscope}%
\pgfsys@transformshift{0.750000in}{0.443750in}%
\pgfsys@useobject{currentmarker}{}%
\end{pgfscope}%
\end{pgfscope}%
\begin{pgfscope}%
\pgfsetbuttcap%
\pgfsetroundjoin%
\definecolor{currentfill}{rgb}{0.150000,0.150000,0.150000}%
\pgfsetfillcolor{currentfill}%
\pgfsetlinewidth{0.803000pt}%
\definecolor{currentstroke}{rgb}{0.150000,0.150000,0.150000}%
\pgfsetstrokecolor{currentstroke}%
\pgfsetdash{}{0pt}%
\pgfsys@defobject{currentmarker}{\pgfqpoint{0.000000in}{0.000000in}}{\pgfqpoint{0.000000in}{0.000000in}}{%
\pgfpathmoveto{\pgfqpoint{0.000000in}{0.000000in}}%
\pgfpathlineto{\pgfqpoint{0.000000in}{0.000000in}}%
\pgfusepath{stroke,fill}%
}%
\begin{pgfscope}%
\pgfsys@transformshift{2.863636in}{0.443750in}%
\pgfsys@useobject{currentmarker}{}%
\end{pgfscope}%
\end{pgfscope}%
\begin{pgfscope}%
\definecolor{textcolor}{rgb}{0.150000,0.150000,0.150000}%
\pgfsetstrokecolor{textcolor}%
\pgfsetfillcolor{textcolor}%
\pgftext[x=0.672222in,y=0.443750in,right,]{\color{textcolor}\sffamily\fontsize{8.000000}{9.600000}\selectfont 0}%
\end{pgfscope}%
\begin{pgfscope}%
\pgfpathrectangle{\pgfqpoint{0.750000in}{0.250000in}}{\pgfqpoint{2.113636in}{1.550000in}} %
\pgfusepath{clip}%
\pgfsetroundcap%
\pgfsetroundjoin%
\pgfsetlinewidth{0.803000pt}%
\definecolor{currentstroke}{rgb}{1.000000,1.000000,1.000000}%
\pgfsetstrokecolor{currentstroke}%
\pgfsetdash{}{0pt}%
\pgfpathmoveto{\pgfqpoint{0.750000in}{0.637500in}}%
\pgfpathlineto{\pgfqpoint{2.863636in}{0.637500in}}%
\pgfusepath{stroke}%
\end{pgfscope}%
\begin{pgfscope}%
\pgfsetbuttcap%
\pgfsetroundjoin%
\definecolor{currentfill}{rgb}{0.150000,0.150000,0.150000}%
\pgfsetfillcolor{currentfill}%
\pgfsetlinewidth{0.803000pt}%
\definecolor{currentstroke}{rgb}{0.150000,0.150000,0.150000}%
\pgfsetstrokecolor{currentstroke}%
\pgfsetdash{}{0pt}%
\pgfsys@defobject{currentmarker}{\pgfqpoint{0.000000in}{0.000000in}}{\pgfqpoint{0.000000in}{0.000000in}}{%
\pgfpathmoveto{\pgfqpoint{0.000000in}{0.000000in}}%
\pgfpathlineto{\pgfqpoint{0.000000in}{0.000000in}}%
\pgfusepath{stroke,fill}%
}%
\begin{pgfscope}%
\pgfsys@transformshift{0.750000in}{0.637500in}%
\pgfsys@useobject{currentmarker}{}%
\end{pgfscope}%
\end{pgfscope}%
\begin{pgfscope}%
\pgfsetbuttcap%
\pgfsetroundjoin%
\definecolor{currentfill}{rgb}{0.150000,0.150000,0.150000}%
\pgfsetfillcolor{currentfill}%
\pgfsetlinewidth{0.803000pt}%
\definecolor{currentstroke}{rgb}{0.150000,0.150000,0.150000}%
\pgfsetstrokecolor{currentstroke}%
\pgfsetdash{}{0pt}%
\pgfsys@defobject{currentmarker}{\pgfqpoint{0.000000in}{0.000000in}}{\pgfqpoint{0.000000in}{0.000000in}}{%
\pgfpathmoveto{\pgfqpoint{0.000000in}{0.000000in}}%
\pgfpathlineto{\pgfqpoint{0.000000in}{0.000000in}}%
\pgfusepath{stroke,fill}%
}%
\begin{pgfscope}%
\pgfsys@transformshift{2.863636in}{0.637500in}%
\pgfsys@useobject{currentmarker}{}%
\end{pgfscope}%
\end{pgfscope}%
\begin{pgfscope}%
\definecolor{textcolor}{rgb}{0.150000,0.150000,0.150000}%
\pgfsetstrokecolor{textcolor}%
\pgfsetfillcolor{textcolor}%
\pgftext[x=0.672222in,y=0.637500in,right,]{\color{textcolor}\sffamily\fontsize{8.000000}{9.600000}\selectfont 2}%
\end{pgfscope}%
\begin{pgfscope}%
\pgfpathrectangle{\pgfqpoint{0.750000in}{0.250000in}}{\pgfqpoint{2.113636in}{1.550000in}} %
\pgfusepath{clip}%
\pgfsetroundcap%
\pgfsetroundjoin%
\pgfsetlinewidth{0.803000pt}%
\definecolor{currentstroke}{rgb}{1.000000,1.000000,1.000000}%
\pgfsetstrokecolor{currentstroke}%
\pgfsetdash{}{0pt}%
\pgfpathmoveto{\pgfqpoint{0.750000in}{0.831250in}}%
\pgfpathlineto{\pgfqpoint{2.863636in}{0.831250in}}%
\pgfusepath{stroke}%
\end{pgfscope}%
\begin{pgfscope}%
\pgfsetbuttcap%
\pgfsetroundjoin%
\definecolor{currentfill}{rgb}{0.150000,0.150000,0.150000}%
\pgfsetfillcolor{currentfill}%
\pgfsetlinewidth{0.803000pt}%
\definecolor{currentstroke}{rgb}{0.150000,0.150000,0.150000}%
\pgfsetstrokecolor{currentstroke}%
\pgfsetdash{}{0pt}%
\pgfsys@defobject{currentmarker}{\pgfqpoint{0.000000in}{0.000000in}}{\pgfqpoint{0.000000in}{0.000000in}}{%
\pgfpathmoveto{\pgfqpoint{0.000000in}{0.000000in}}%
\pgfpathlineto{\pgfqpoint{0.000000in}{0.000000in}}%
\pgfusepath{stroke,fill}%
}%
\begin{pgfscope}%
\pgfsys@transformshift{0.750000in}{0.831250in}%
\pgfsys@useobject{currentmarker}{}%
\end{pgfscope}%
\end{pgfscope}%
\begin{pgfscope}%
\pgfsetbuttcap%
\pgfsetroundjoin%
\definecolor{currentfill}{rgb}{0.150000,0.150000,0.150000}%
\pgfsetfillcolor{currentfill}%
\pgfsetlinewidth{0.803000pt}%
\definecolor{currentstroke}{rgb}{0.150000,0.150000,0.150000}%
\pgfsetstrokecolor{currentstroke}%
\pgfsetdash{}{0pt}%
\pgfsys@defobject{currentmarker}{\pgfqpoint{0.000000in}{0.000000in}}{\pgfqpoint{0.000000in}{0.000000in}}{%
\pgfpathmoveto{\pgfqpoint{0.000000in}{0.000000in}}%
\pgfpathlineto{\pgfqpoint{0.000000in}{0.000000in}}%
\pgfusepath{stroke,fill}%
}%
\begin{pgfscope}%
\pgfsys@transformshift{2.863636in}{0.831250in}%
\pgfsys@useobject{currentmarker}{}%
\end{pgfscope}%
\end{pgfscope}%
\begin{pgfscope}%
\definecolor{textcolor}{rgb}{0.150000,0.150000,0.150000}%
\pgfsetstrokecolor{textcolor}%
\pgfsetfillcolor{textcolor}%
\pgftext[x=0.672222in,y=0.831250in,right,]{\color{textcolor}\sffamily\fontsize{8.000000}{9.600000}\selectfont 4}%
\end{pgfscope}%
\begin{pgfscope}%
\pgfpathrectangle{\pgfqpoint{0.750000in}{0.250000in}}{\pgfqpoint{2.113636in}{1.550000in}} %
\pgfusepath{clip}%
\pgfsetroundcap%
\pgfsetroundjoin%
\pgfsetlinewidth{0.803000pt}%
\definecolor{currentstroke}{rgb}{1.000000,1.000000,1.000000}%
\pgfsetstrokecolor{currentstroke}%
\pgfsetdash{}{0pt}%
\pgfpathmoveto{\pgfqpoint{0.750000in}{1.025000in}}%
\pgfpathlineto{\pgfqpoint{2.863636in}{1.025000in}}%
\pgfusepath{stroke}%
\end{pgfscope}%
\begin{pgfscope}%
\pgfsetbuttcap%
\pgfsetroundjoin%
\definecolor{currentfill}{rgb}{0.150000,0.150000,0.150000}%
\pgfsetfillcolor{currentfill}%
\pgfsetlinewidth{0.803000pt}%
\definecolor{currentstroke}{rgb}{0.150000,0.150000,0.150000}%
\pgfsetstrokecolor{currentstroke}%
\pgfsetdash{}{0pt}%
\pgfsys@defobject{currentmarker}{\pgfqpoint{0.000000in}{0.000000in}}{\pgfqpoint{0.000000in}{0.000000in}}{%
\pgfpathmoveto{\pgfqpoint{0.000000in}{0.000000in}}%
\pgfpathlineto{\pgfqpoint{0.000000in}{0.000000in}}%
\pgfusepath{stroke,fill}%
}%
\begin{pgfscope}%
\pgfsys@transformshift{0.750000in}{1.025000in}%
\pgfsys@useobject{currentmarker}{}%
\end{pgfscope}%
\end{pgfscope}%
\begin{pgfscope}%
\pgfsetbuttcap%
\pgfsetroundjoin%
\definecolor{currentfill}{rgb}{0.150000,0.150000,0.150000}%
\pgfsetfillcolor{currentfill}%
\pgfsetlinewidth{0.803000pt}%
\definecolor{currentstroke}{rgb}{0.150000,0.150000,0.150000}%
\pgfsetstrokecolor{currentstroke}%
\pgfsetdash{}{0pt}%
\pgfsys@defobject{currentmarker}{\pgfqpoint{0.000000in}{0.000000in}}{\pgfqpoint{0.000000in}{0.000000in}}{%
\pgfpathmoveto{\pgfqpoint{0.000000in}{0.000000in}}%
\pgfpathlineto{\pgfqpoint{0.000000in}{0.000000in}}%
\pgfusepath{stroke,fill}%
}%
\begin{pgfscope}%
\pgfsys@transformshift{2.863636in}{1.025000in}%
\pgfsys@useobject{currentmarker}{}%
\end{pgfscope}%
\end{pgfscope}%
\begin{pgfscope}%
\definecolor{textcolor}{rgb}{0.150000,0.150000,0.150000}%
\pgfsetstrokecolor{textcolor}%
\pgfsetfillcolor{textcolor}%
\pgftext[x=0.672222in,y=1.025000in,right,]{\color{textcolor}\sffamily\fontsize{8.000000}{9.600000}\selectfont 6}%
\end{pgfscope}%
\begin{pgfscope}%
\pgfpathrectangle{\pgfqpoint{0.750000in}{0.250000in}}{\pgfqpoint{2.113636in}{1.550000in}} %
\pgfusepath{clip}%
\pgfsetroundcap%
\pgfsetroundjoin%
\pgfsetlinewidth{0.803000pt}%
\definecolor{currentstroke}{rgb}{1.000000,1.000000,1.000000}%
\pgfsetstrokecolor{currentstroke}%
\pgfsetdash{}{0pt}%
\pgfpathmoveto{\pgfqpoint{0.750000in}{1.218750in}}%
\pgfpathlineto{\pgfqpoint{2.863636in}{1.218750in}}%
\pgfusepath{stroke}%
\end{pgfscope}%
\begin{pgfscope}%
\pgfsetbuttcap%
\pgfsetroundjoin%
\definecolor{currentfill}{rgb}{0.150000,0.150000,0.150000}%
\pgfsetfillcolor{currentfill}%
\pgfsetlinewidth{0.803000pt}%
\definecolor{currentstroke}{rgb}{0.150000,0.150000,0.150000}%
\pgfsetstrokecolor{currentstroke}%
\pgfsetdash{}{0pt}%
\pgfsys@defobject{currentmarker}{\pgfqpoint{0.000000in}{0.000000in}}{\pgfqpoint{0.000000in}{0.000000in}}{%
\pgfpathmoveto{\pgfqpoint{0.000000in}{0.000000in}}%
\pgfpathlineto{\pgfqpoint{0.000000in}{0.000000in}}%
\pgfusepath{stroke,fill}%
}%
\begin{pgfscope}%
\pgfsys@transformshift{0.750000in}{1.218750in}%
\pgfsys@useobject{currentmarker}{}%
\end{pgfscope}%
\end{pgfscope}%
\begin{pgfscope}%
\pgfsetbuttcap%
\pgfsetroundjoin%
\definecolor{currentfill}{rgb}{0.150000,0.150000,0.150000}%
\pgfsetfillcolor{currentfill}%
\pgfsetlinewidth{0.803000pt}%
\definecolor{currentstroke}{rgb}{0.150000,0.150000,0.150000}%
\pgfsetstrokecolor{currentstroke}%
\pgfsetdash{}{0pt}%
\pgfsys@defobject{currentmarker}{\pgfqpoint{0.000000in}{0.000000in}}{\pgfqpoint{0.000000in}{0.000000in}}{%
\pgfpathmoveto{\pgfqpoint{0.000000in}{0.000000in}}%
\pgfpathlineto{\pgfqpoint{0.000000in}{0.000000in}}%
\pgfusepath{stroke,fill}%
}%
\begin{pgfscope}%
\pgfsys@transformshift{2.863636in}{1.218750in}%
\pgfsys@useobject{currentmarker}{}%
\end{pgfscope}%
\end{pgfscope}%
\begin{pgfscope}%
\definecolor{textcolor}{rgb}{0.150000,0.150000,0.150000}%
\pgfsetstrokecolor{textcolor}%
\pgfsetfillcolor{textcolor}%
\pgftext[x=0.672222in,y=1.218750in,right,]{\color{textcolor}\sffamily\fontsize{8.000000}{9.600000}\selectfont 8}%
\end{pgfscope}%
\begin{pgfscope}%
\pgfpathrectangle{\pgfqpoint{0.750000in}{0.250000in}}{\pgfqpoint{2.113636in}{1.550000in}} %
\pgfusepath{clip}%
\pgfsetroundcap%
\pgfsetroundjoin%
\pgfsetlinewidth{0.803000pt}%
\definecolor{currentstroke}{rgb}{1.000000,1.000000,1.000000}%
\pgfsetstrokecolor{currentstroke}%
\pgfsetdash{}{0pt}%
\pgfpathmoveto{\pgfqpoint{0.750000in}{1.412500in}}%
\pgfpathlineto{\pgfqpoint{2.863636in}{1.412500in}}%
\pgfusepath{stroke}%
\end{pgfscope}%
\begin{pgfscope}%
\pgfsetbuttcap%
\pgfsetroundjoin%
\definecolor{currentfill}{rgb}{0.150000,0.150000,0.150000}%
\pgfsetfillcolor{currentfill}%
\pgfsetlinewidth{0.803000pt}%
\definecolor{currentstroke}{rgb}{0.150000,0.150000,0.150000}%
\pgfsetstrokecolor{currentstroke}%
\pgfsetdash{}{0pt}%
\pgfsys@defobject{currentmarker}{\pgfqpoint{0.000000in}{0.000000in}}{\pgfqpoint{0.000000in}{0.000000in}}{%
\pgfpathmoveto{\pgfqpoint{0.000000in}{0.000000in}}%
\pgfpathlineto{\pgfqpoint{0.000000in}{0.000000in}}%
\pgfusepath{stroke,fill}%
}%
\begin{pgfscope}%
\pgfsys@transformshift{0.750000in}{1.412500in}%
\pgfsys@useobject{currentmarker}{}%
\end{pgfscope}%
\end{pgfscope}%
\begin{pgfscope}%
\pgfsetbuttcap%
\pgfsetroundjoin%
\definecolor{currentfill}{rgb}{0.150000,0.150000,0.150000}%
\pgfsetfillcolor{currentfill}%
\pgfsetlinewidth{0.803000pt}%
\definecolor{currentstroke}{rgb}{0.150000,0.150000,0.150000}%
\pgfsetstrokecolor{currentstroke}%
\pgfsetdash{}{0pt}%
\pgfsys@defobject{currentmarker}{\pgfqpoint{0.000000in}{0.000000in}}{\pgfqpoint{0.000000in}{0.000000in}}{%
\pgfpathmoveto{\pgfqpoint{0.000000in}{0.000000in}}%
\pgfpathlineto{\pgfqpoint{0.000000in}{0.000000in}}%
\pgfusepath{stroke,fill}%
}%
\begin{pgfscope}%
\pgfsys@transformshift{2.863636in}{1.412500in}%
\pgfsys@useobject{currentmarker}{}%
\end{pgfscope}%
\end{pgfscope}%
\begin{pgfscope}%
\definecolor{textcolor}{rgb}{0.150000,0.150000,0.150000}%
\pgfsetstrokecolor{textcolor}%
\pgfsetfillcolor{textcolor}%
\pgftext[x=0.672222in,y=1.412500in,right,]{\color{textcolor}\sffamily\fontsize{8.000000}{9.600000}\selectfont 10}%
\end{pgfscope}%
\begin{pgfscope}%
\pgfpathrectangle{\pgfqpoint{0.750000in}{0.250000in}}{\pgfqpoint{2.113636in}{1.550000in}} %
\pgfusepath{clip}%
\pgfsetroundcap%
\pgfsetroundjoin%
\pgfsetlinewidth{0.803000pt}%
\definecolor{currentstroke}{rgb}{1.000000,1.000000,1.000000}%
\pgfsetstrokecolor{currentstroke}%
\pgfsetdash{}{0pt}%
\pgfpathmoveto{\pgfqpoint{0.750000in}{1.606250in}}%
\pgfpathlineto{\pgfqpoint{2.863636in}{1.606250in}}%
\pgfusepath{stroke}%
\end{pgfscope}%
\begin{pgfscope}%
\pgfsetbuttcap%
\pgfsetroundjoin%
\definecolor{currentfill}{rgb}{0.150000,0.150000,0.150000}%
\pgfsetfillcolor{currentfill}%
\pgfsetlinewidth{0.803000pt}%
\definecolor{currentstroke}{rgb}{0.150000,0.150000,0.150000}%
\pgfsetstrokecolor{currentstroke}%
\pgfsetdash{}{0pt}%
\pgfsys@defobject{currentmarker}{\pgfqpoint{0.000000in}{0.000000in}}{\pgfqpoint{0.000000in}{0.000000in}}{%
\pgfpathmoveto{\pgfqpoint{0.000000in}{0.000000in}}%
\pgfpathlineto{\pgfqpoint{0.000000in}{0.000000in}}%
\pgfusepath{stroke,fill}%
}%
\begin{pgfscope}%
\pgfsys@transformshift{0.750000in}{1.606250in}%
\pgfsys@useobject{currentmarker}{}%
\end{pgfscope}%
\end{pgfscope}%
\begin{pgfscope}%
\pgfsetbuttcap%
\pgfsetroundjoin%
\definecolor{currentfill}{rgb}{0.150000,0.150000,0.150000}%
\pgfsetfillcolor{currentfill}%
\pgfsetlinewidth{0.803000pt}%
\definecolor{currentstroke}{rgb}{0.150000,0.150000,0.150000}%
\pgfsetstrokecolor{currentstroke}%
\pgfsetdash{}{0pt}%
\pgfsys@defobject{currentmarker}{\pgfqpoint{0.000000in}{0.000000in}}{\pgfqpoint{0.000000in}{0.000000in}}{%
\pgfpathmoveto{\pgfqpoint{0.000000in}{0.000000in}}%
\pgfpathlineto{\pgfqpoint{0.000000in}{0.000000in}}%
\pgfusepath{stroke,fill}%
}%
\begin{pgfscope}%
\pgfsys@transformshift{2.863636in}{1.606250in}%
\pgfsys@useobject{currentmarker}{}%
\end{pgfscope}%
\end{pgfscope}%
\begin{pgfscope}%
\definecolor{textcolor}{rgb}{0.150000,0.150000,0.150000}%
\pgfsetstrokecolor{textcolor}%
\pgfsetfillcolor{textcolor}%
\pgftext[x=0.672222in,y=1.606250in,right,]{\color{textcolor}\sffamily\fontsize{8.000000}{9.600000}\selectfont 12}%
\end{pgfscope}%
\begin{pgfscope}%
\pgfpathrectangle{\pgfqpoint{0.750000in}{0.250000in}}{\pgfqpoint{2.113636in}{1.550000in}} %
\pgfusepath{clip}%
\pgfsetroundcap%
\pgfsetroundjoin%
\pgfsetlinewidth{0.803000pt}%
\definecolor{currentstroke}{rgb}{1.000000,1.000000,1.000000}%
\pgfsetstrokecolor{currentstroke}%
\pgfsetdash{}{0pt}%
\pgfpathmoveto{\pgfqpoint{0.750000in}{1.800000in}}%
\pgfpathlineto{\pgfqpoint{2.863636in}{1.800000in}}%
\pgfusepath{stroke}%
\end{pgfscope}%
\begin{pgfscope}%
\pgfsetbuttcap%
\pgfsetroundjoin%
\definecolor{currentfill}{rgb}{0.150000,0.150000,0.150000}%
\pgfsetfillcolor{currentfill}%
\pgfsetlinewidth{0.803000pt}%
\definecolor{currentstroke}{rgb}{0.150000,0.150000,0.150000}%
\pgfsetstrokecolor{currentstroke}%
\pgfsetdash{}{0pt}%
\pgfsys@defobject{currentmarker}{\pgfqpoint{0.000000in}{0.000000in}}{\pgfqpoint{0.000000in}{0.000000in}}{%
\pgfpathmoveto{\pgfqpoint{0.000000in}{0.000000in}}%
\pgfpathlineto{\pgfqpoint{0.000000in}{0.000000in}}%
\pgfusepath{stroke,fill}%
}%
\begin{pgfscope}%
\pgfsys@transformshift{0.750000in}{1.800000in}%
\pgfsys@useobject{currentmarker}{}%
\end{pgfscope}%
\end{pgfscope}%
\begin{pgfscope}%
\pgfsetbuttcap%
\pgfsetroundjoin%
\definecolor{currentfill}{rgb}{0.150000,0.150000,0.150000}%
\pgfsetfillcolor{currentfill}%
\pgfsetlinewidth{0.803000pt}%
\definecolor{currentstroke}{rgb}{0.150000,0.150000,0.150000}%
\pgfsetstrokecolor{currentstroke}%
\pgfsetdash{}{0pt}%
\pgfsys@defobject{currentmarker}{\pgfqpoint{0.000000in}{0.000000in}}{\pgfqpoint{0.000000in}{0.000000in}}{%
\pgfpathmoveto{\pgfqpoint{0.000000in}{0.000000in}}%
\pgfpathlineto{\pgfqpoint{0.000000in}{0.000000in}}%
\pgfusepath{stroke,fill}%
}%
\begin{pgfscope}%
\pgfsys@transformshift{2.863636in}{1.800000in}%
\pgfsys@useobject{currentmarker}{}%
\end{pgfscope}%
\end{pgfscope}%
\begin{pgfscope}%
\definecolor{textcolor}{rgb}{0.150000,0.150000,0.150000}%
\pgfsetstrokecolor{textcolor}%
\pgfsetfillcolor{textcolor}%
\pgftext[x=0.672222in,y=1.800000in,right,]{\color{textcolor}\sffamily\fontsize{8.000000}{9.600000}\selectfont 14}%
\end{pgfscope}%
\begin{pgfscope}%
\pgfpathrectangle{\pgfqpoint{0.750000in}{0.250000in}}{\pgfqpoint{2.113636in}{1.550000in}} %
\pgfusepath{clip}%
\pgfsetroundcap%
\pgfsetroundjoin%
\pgfsetlinewidth{1.405250pt}%
\definecolor{currentstroke}{rgb}{0.298039,0.447059,0.690196}%
\pgfsetstrokecolor{currentstroke}%
\pgfsetdash{}{0pt}%
\pgfpathmoveto{\pgfqpoint{0.750000in}{0.443750in}}%
\pgfpathlineto{\pgfqpoint{0.762733in}{0.489416in}}%
\pgfpathlineto{\pgfqpoint{0.775466in}{0.529237in}}%
\pgfpathlineto{\pgfqpoint{0.788199in}{0.558279in}}%
\pgfpathlineto{\pgfqpoint{0.800931in}{0.573289in}}%
\pgfpathlineto{\pgfqpoint{0.813664in}{0.573199in}}%
\pgfpathlineto{\pgfqpoint{0.826397in}{0.559299in}}%
\pgfpathlineto{\pgfqpoint{0.839130in}{0.535028in}}%
\pgfpathlineto{\pgfqpoint{0.864596in}{0.476431in}}%
\pgfpathlineto{\pgfqpoint{0.877329in}{0.453786in}}%
\pgfpathlineto{\pgfqpoint{0.890061in}{0.442313in}}%
\pgfpathlineto{\pgfqpoint{0.902794in}{0.445074in}}%
\pgfpathlineto{\pgfqpoint{0.915527in}{0.462912in}}%
\pgfpathlineto{\pgfqpoint{0.928260in}{0.494320in}}%
\pgfpathlineto{\pgfqpoint{0.953726in}{0.581806in}}%
\pgfpathlineto{\pgfqpoint{0.966459in}{0.626793in}}%
\pgfpathlineto{\pgfqpoint{0.979191in}{0.664898in}}%
\pgfpathlineto{\pgfqpoint{0.991924in}{0.691454in}}%
\pgfpathlineto{\pgfqpoint{1.004657in}{0.703596in}}%
\pgfpathlineto{\pgfqpoint{1.017390in}{0.700704in}}%
\pgfpathlineto{\pgfqpoint{1.030123in}{0.684503in}}%
\pgfpathlineto{\pgfqpoint{1.042856in}{0.658791in}}%
\pgfpathlineto{\pgfqpoint{1.068321in}{0.600624in}}%
\pgfpathlineto{\pgfqpoint{1.081054in}{0.579783in}}%
\pgfpathlineto{\pgfqpoint{1.093787in}{0.570850in}}%
\pgfpathlineto{\pgfqpoint{1.106520in}{0.576492in}}%
\pgfpathlineto{\pgfqpoint{1.119253in}{0.597103in}}%
\pgfpathlineto{\pgfqpoint{1.131986in}{0.630744in}}%
\pgfpathlineto{\pgfqpoint{1.182917in}{0.800133in}}%
\pgfpathlineto{\pgfqpoint{1.195650in}{0.824099in}}%
\pgfpathlineto{\pgfqpoint{1.208383in}{0.833350in}}%
\pgfpathlineto{\pgfqpoint{1.221116in}{0.827719in}}%
\pgfpathlineto{\pgfqpoint{1.233849in}{0.809357in}}%
\pgfpathlineto{\pgfqpoint{1.272047in}{0.725089in}}%
\pgfpathlineto{\pgfqpoint{1.284780in}{0.706217in}}%
\pgfpathlineto{\pgfqpoint{1.297513in}{0.699923in}}%
\pgfpathlineto{\pgfqpoint{1.310246in}{0.708461in}}%
\pgfpathlineto{\pgfqpoint{1.322979in}{0.731773in}}%
\pgfpathlineto{\pgfqpoint{1.335711in}{0.767500in}}%
\pgfpathlineto{\pgfqpoint{1.373910in}{0.900750in}}%
\pgfpathlineto{\pgfqpoint{1.386643in}{0.934918in}}%
\pgfpathlineto{\pgfqpoint{1.399376in}{0.956202in}}%
\pgfpathlineto{\pgfqpoint{1.412109in}{0.962557in}}%
\pgfpathlineto{\pgfqpoint{1.424841in}{0.954265in}}%
\pgfpathlineto{\pgfqpoint{1.437574in}{0.933894in}}%
\pgfpathlineto{\pgfqpoint{1.475773in}{0.849862in}}%
\pgfpathlineto{\pgfqpoint{1.488506in}{0.833115in}}%
\pgfpathlineto{\pgfqpoint{1.501239in}{0.829541in}}%
\pgfpathlineto{\pgfqpoint{1.513971in}{0.840973in}}%
\pgfpathlineto{\pgfqpoint{1.526704in}{0.866900in}}%
\pgfpathlineto{\pgfqpoint{1.539437in}{0.904556in}}%
\pgfpathlineto{\pgfqpoint{1.577636in}{1.037258in}}%
\pgfpathlineto{\pgfqpoint{1.590369in}{1.069229in}}%
\pgfpathlineto{\pgfqpoint{1.603101in}{1.087756in}}%
\pgfpathlineto{\pgfqpoint{1.615834in}{1.091224in}}%
\pgfpathlineto{\pgfqpoint{1.628567in}{1.080367in}}%
\pgfpathlineto{\pgfqpoint{1.641300in}{1.058151in}}%
\pgfpathlineto{\pgfqpoint{1.679498in}{0.974976in}}%
\pgfpathlineto{\pgfqpoint{1.692231in}{0.960497in}}%
\pgfpathlineto{\pgfqpoint{1.704964in}{0.959712in}}%
\pgfpathlineto{\pgfqpoint{1.717697in}{0.974018in}}%
\pgfpathlineto{\pgfqpoint{1.730430in}{1.002459in}}%
\pgfpathlineto{\pgfqpoint{1.743163in}{1.041873in}}%
\pgfpathlineto{\pgfqpoint{1.781361in}{1.173409in}}%
\pgfpathlineto{\pgfqpoint{1.794094in}{1.203045in}}%
\pgfpathlineto{\pgfqpoint{1.806827in}{1.218756in}}%
\pgfpathlineto{\pgfqpoint{1.819560in}{1.219365in}}%
\pgfpathlineto{\pgfqpoint{1.832293in}{1.206051in}}%
\pgfpathlineto{\pgfqpoint{1.845026in}{1.182164in}}%
\pgfpathlineto{\pgfqpoint{1.870491in}{1.123527in}}%
\pgfpathlineto{\pgfqpoint{1.883224in}{1.100465in}}%
\pgfpathlineto{\pgfqpoint{1.895957in}{1.088383in}}%
\pgfpathlineto{\pgfqpoint{1.908690in}{1.090438in}}%
\pgfpathlineto{\pgfqpoint{1.921423in}{1.107584in}}%
\pgfpathlineto{\pgfqpoint{1.934156in}{1.138421in}}%
\pgfpathlineto{\pgfqpoint{1.959621in}{1.225452in}}%
\pgfpathlineto{\pgfqpoint{1.972354in}{1.270626in}}%
\pgfpathlineto{\pgfqpoint{1.985087in}{1.309170in}}%
\pgfpathlineto{\pgfqpoint{1.997820in}{1.336349in}}%
\pgfpathlineto{\pgfqpoint{2.010553in}{1.349200in}}%
\pgfpathlineto{\pgfqpoint{2.023286in}{1.346993in}}%
\pgfpathlineto{\pgfqpoint{2.036018in}{1.331346in}}%
\pgfpathlineto{\pgfqpoint{2.048751in}{1.305972in}}%
\pgfpathlineto{\pgfqpoint{2.074217in}{1.247659in}}%
\pgfpathlineto{\pgfqpoint{2.086950in}{1.226357in}}%
\pgfpathlineto{\pgfqpoint{2.099683in}{1.216789in}}%
\pgfpathlineto{\pgfqpoint{2.112416in}{1.221719in}}%
\pgfpathlineto{\pgfqpoint{2.125148in}{1.241653in}}%
\pgfpathlineto{\pgfqpoint{2.137881in}{1.274757in}}%
\pgfpathlineto{\pgfqpoint{2.163347in}{1.363479in}}%
\pgfpathlineto{\pgfqpoint{2.176080in}{1.407815in}}%
\pgfpathlineto{\pgfqpoint{2.188813in}{1.444513in}}%
\pgfpathlineto{\pgfqpoint{2.201546in}{1.469125in}}%
\pgfpathlineto{\pgfqpoint{2.214278in}{1.479090in}}%
\pgfpathlineto{\pgfqpoint{2.227011in}{1.474127in}}%
\pgfpathlineto{\pgfqpoint{2.239744in}{1.456283in}}%
\pgfpathlineto{\pgfqpoint{2.265210in}{1.399551in}}%
\pgfpathlineto{\pgfqpoint{2.277943in}{1.372053in}}%
\pgfpathlineto{\pgfqpoint{2.290676in}{1.352682in}}%
\pgfpathlineto{\pgfqpoint{2.303408in}{1.345729in}}%
\pgfpathlineto{\pgfqpoint{2.316141in}{1.353553in}}%
\pgfpathlineto{\pgfqpoint{2.328874in}{1.376207in}}%
\pgfpathlineto{\pgfqpoint{2.341607in}{1.411435in}}%
\pgfpathlineto{\pgfqpoint{2.379806in}{1.544723in}}%
\pgfpathlineto{\pgfqpoint{2.392538in}{1.579411in}}%
\pgfpathlineto{\pgfqpoint{2.405271in}{1.601363in}}%
\pgfpathlineto{\pgfqpoint{2.418004in}{1.608431in}}%
\pgfpathlineto{\pgfqpoint{2.430737in}{1.600786in}}%
\pgfpathlineto{\pgfqpoint{2.443470in}{1.580895in}}%
\pgfpathlineto{\pgfqpoint{2.481668in}{1.496747in}}%
\pgfpathlineto{\pgfqpoint{2.494401in}{1.479463in}}%
\pgfpathlineto{\pgfqpoint{2.507134in}{1.475212in}}%
\pgfpathlineto{\pgfqpoint{2.519867in}{1.485932in}}%
\pgfpathlineto{\pgfqpoint{2.532600in}{1.511224in}}%
\pgfpathlineto{\pgfqpoint{2.545333in}{1.548420in}}%
\pgfpathlineto{\pgfqpoint{2.583531in}{1.681315in}}%
\pgfpathlineto{\pgfqpoint{2.596264in}{1.713840in}}%
\pgfpathlineto{\pgfqpoint{2.608997in}{1.733052in}}%
\pgfpathlineto{\pgfqpoint{2.621730in}{1.737230in}}%
\pgfpathlineto{\pgfqpoint{2.634463in}{1.726996in}}%
\pgfpathlineto{\pgfqpoint{2.647196in}{1.705218in}}%
\pgfpathlineto{\pgfqpoint{2.647196in}{1.705218in}}%
\pgfusepath{stroke}%
\end{pgfscope}%
\begin{pgfscope}%
\pgfsetrectcap%
\pgfsetmiterjoin%
\pgfsetlinewidth{0.000000pt}%
\definecolor{currentstroke}{rgb}{1.000000,1.000000,1.000000}%
\pgfsetstrokecolor{currentstroke}%
\pgfsetdash{}{0pt}%
\pgfpathmoveto{\pgfqpoint{2.863636in}{0.250000in}}%
\pgfpathlineto{\pgfqpoint{2.863636in}{1.800000in}}%
\pgfusepath{}%
\end{pgfscope}%
\begin{pgfscope}%
\pgfsetrectcap%
\pgfsetmiterjoin%
\pgfsetlinewidth{0.000000pt}%
\definecolor{currentstroke}{rgb}{1.000000,1.000000,1.000000}%
\pgfsetstrokecolor{currentstroke}%
\pgfsetdash{}{0pt}%
\pgfpathmoveto{\pgfqpoint{0.750000in}{1.800000in}}%
\pgfpathlineto{\pgfqpoint{2.863636in}{1.800000in}}%
\pgfusepath{}%
\end{pgfscope}%
\begin{pgfscope}%
\pgfsetrectcap%
\pgfsetmiterjoin%
\pgfsetlinewidth{0.000000pt}%
\definecolor{currentstroke}{rgb}{1.000000,1.000000,1.000000}%
\pgfsetstrokecolor{currentstroke}%
\pgfsetdash{}{0pt}%
\pgfpathmoveto{\pgfqpoint{0.750000in}{0.250000in}}%
\pgfpathlineto{\pgfqpoint{2.863636in}{0.250000in}}%
\pgfusepath{}%
\end{pgfscope}%
\begin{pgfscope}%
\pgfsetrectcap%
\pgfsetmiterjoin%
\pgfsetlinewidth{0.000000pt}%
\definecolor{currentstroke}{rgb}{1.000000,1.000000,1.000000}%
\pgfsetstrokecolor{currentstroke}%
\pgfsetdash{}{0pt}%
\pgfpathmoveto{\pgfqpoint{0.750000in}{0.250000in}}%
\pgfpathlineto{\pgfqpoint{0.750000in}{1.800000in}}%
\pgfusepath{}%
\end{pgfscope}%
\begin{pgfscope}%
\pgfsetbuttcap%
\pgfsetmiterjoin%
\definecolor{currentfill}{rgb}{0.917647,0.917647,0.949020}%
\pgfsetfillcolor{currentfill}%
\pgfsetlinewidth{0.000000pt}%
\definecolor{currentstroke}{rgb}{0.000000,0.000000,0.000000}%
\pgfsetstrokecolor{currentstroke}%
\pgfsetstrokeopacity{0.000000}%
\pgfsetdash{}{0pt}%
\pgfpathmoveto{\pgfqpoint{3.286364in}{0.250000in}}%
\pgfpathlineto{\pgfqpoint{5.400000in}{0.250000in}}%
\pgfpathlineto{\pgfqpoint{5.400000in}{1.800000in}}%
\pgfpathlineto{\pgfqpoint{3.286364in}{1.800000in}}%
\pgfpathclose%
\pgfusepath{fill}%
\end{pgfscope}%
\begin{pgfscope}%
\pgfpathrectangle{\pgfqpoint{3.286364in}{0.250000in}}{\pgfqpoint{2.113636in}{1.550000in}} %
\pgfusepath{clip}%
\pgfsetroundcap%
\pgfsetroundjoin%
\pgfsetlinewidth{0.803000pt}%
\definecolor{currentstroke}{rgb}{1.000000,1.000000,1.000000}%
\pgfsetstrokecolor{currentstroke}%
\pgfsetdash{}{0pt}%
\pgfpathmoveto{\pgfqpoint{3.286364in}{0.250000in}}%
\pgfpathlineto{\pgfqpoint{3.286364in}{1.800000in}}%
\pgfusepath{stroke}%
\end{pgfscope}%
\begin{pgfscope}%
\pgfsetbuttcap%
\pgfsetroundjoin%
\definecolor{currentfill}{rgb}{0.150000,0.150000,0.150000}%
\pgfsetfillcolor{currentfill}%
\pgfsetlinewidth{0.803000pt}%
\definecolor{currentstroke}{rgb}{0.150000,0.150000,0.150000}%
\pgfsetstrokecolor{currentstroke}%
\pgfsetdash{}{0pt}%
\pgfsys@defobject{currentmarker}{\pgfqpoint{0.000000in}{0.000000in}}{\pgfqpoint{0.000000in}{0.000000in}}{%
\pgfpathmoveto{\pgfqpoint{0.000000in}{0.000000in}}%
\pgfpathlineto{\pgfqpoint{0.000000in}{0.000000in}}%
\pgfusepath{stroke,fill}%
}%
\begin{pgfscope}%
\pgfsys@transformshift{3.286364in}{0.250000in}%
\pgfsys@useobject{currentmarker}{}%
\end{pgfscope}%
\end{pgfscope}%
\begin{pgfscope}%
\pgfsetbuttcap%
\pgfsetroundjoin%
\definecolor{currentfill}{rgb}{0.150000,0.150000,0.150000}%
\pgfsetfillcolor{currentfill}%
\pgfsetlinewidth{0.803000pt}%
\definecolor{currentstroke}{rgb}{0.150000,0.150000,0.150000}%
\pgfsetstrokecolor{currentstroke}%
\pgfsetdash{}{0pt}%
\pgfsys@defobject{currentmarker}{\pgfqpoint{0.000000in}{0.000000in}}{\pgfqpoint{0.000000in}{0.000000in}}{%
\pgfpathmoveto{\pgfqpoint{0.000000in}{0.000000in}}%
\pgfpathlineto{\pgfqpoint{0.000000in}{0.000000in}}%
\pgfusepath{stroke,fill}%
}%
\begin{pgfscope}%
\pgfsys@transformshift{3.286364in}{1.800000in}%
\pgfsys@useobject{currentmarker}{}%
\end{pgfscope}%
\end{pgfscope}%
\begin{pgfscope}%
\definecolor{textcolor}{rgb}{0.150000,0.150000,0.150000}%
\pgfsetstrokecolor{textcolor}%
\pgfsetfillcolor{textcolor}%
\pgftext[x=3.286364in,y=0.172222in,,top]{\color{textcolor}\sffamily\fontsize{8.000000}{9.600000}\selectfont 0}%
\end{pgfscope}%
\begin{pgfscope}%
\pgfpathrectangle{\pgfqpoint{3.286364in}{0.250000in}}{\pgfqpoint{2.113636in}{1.550000in}} %
\pgfusepath{clip}%
\pgfsetroundcap%
\pgfsetroundjoin%
\pgfsetlinewidth{0.803000pt}%
\definecolor{currentstroke}{rgb}{1.000000,1.000000,1.000000}%
\pgfsetstrokecolor{currentstroke}%
\pgfsetdash{}{0pt}%
\pgfpathmoveto{\pgfqpoint{3.588312in}{0.250000in}}%
\pgfpathlineto{\pgfqpoint{3.588312in}{1.800000in}}%
\pgfusepath{stroke}%
\end{pgfscope}%
\begin{pgfscope}%
\pgfsetbuttcap%
\pgfsetroundjoin%
\definecolor{currentfill}{rgb}{0.150000,0.150000,0.150000}%
\pgfsetfillcolor{currentfill}%
\pgfsetlinewidth{0.803000pt}%
\definecolor{currentstroke}{rgb}{0.150000,0.150000,0.150000}%
\pgfsetstrokecolor{currentstroke}%
\pgfsetdash{}{0pt}%
\pgfsys@defobject{currentmarker}{\pgfqpoint{0.000000in}{0.000000in}}{\pgfqpoint{0.000000in}{0.000000in}}{%
\pgfpathmoveto{\pgfqpoint{0.000000in}{0.000000in}}%
\pgfpathlineto{\pgfqpoint{0.000000in}{0.000000in}}%
\pgfusepath{stroke,fill}%
}%
\begin{pgfscope}%
\pgfsys@transformshift{3.588312in}{0.250000in}%
\pgfsys@useobject{currentmarker}{}%
\end{pgfscope}%
\end{pgfscope}%
\begin{pgfscope}%
\pgfsetbuttcap%
\pgfsetroundjoin%
\definecolor{currentfill}{rgb}{0.150000,0.150000,0.150000}%
\pgfsetfillcolor{currentfill}%
\pgfsetlinewidth{0.803000pt}%
\definecolor{currentstroke}{rgb}{0.150000,0.150000,0.150000}%
\pgfsetstrokecolor{currentstroke}%
\pgfsetdash{}{0pt}%
\pgfsys@defobject{currentmarker}{\pgfqpoint{0.000000in}{0.000000in}}{\pgfqpoint{0.000000in}{0.000000in}}{%
\pgfpathmoveto{\pgfqpoint{0.000000in}{0.000000in}}%
\pgfpathlineto{\pgfqpoint{0.000000in}{0.000000in}}%
\pgfusepath{stroke,fill}%
}%
\begin{pgfscope}%
\pgfsys@transformshift{3.588312in}{1.800000in}%
\pgfsys@useobject{currentmarker}{}%
\end{pgfscope}%
\end{pgfscope}%
\begin{pgfscope}%
\definecolor{textcolor}{rgb}{0.150000,0.150000,0.150000}%
\pgfsetstrokecolor{textcolor}%
\pgfsetfillcolor{textcolor}%
\pgftext[x=3.588312in,y=0.172222in,,top]{\color{textcolor}\sffamily\fontsize{8.000000}{9.600000}\selectfont 1}%
\end{pgfscope}%
\begin{pgfscope}%
\pgfpathrectangle{\pgfqpoint{3.286364in}{0.250000in}}{\pgfqpoint{2.113636in}{1.550000in}} %
\pgfusepath{clip}%
\pgfsetroundcap%
\pgfsetroundjoin%
\pgfsetlinewidth{0.803000pt}%
\definecolor{currentstroke}{rgb}{1.000000,1.000000,1.000000}%
\pgfsetstrokecolor{currentstroke}%
\pgfsetdash{}{0pt}%
\pgfpathmoveto{\pgfqpoint{3.890260in}{0.250000in}}%
\pgfpathlineto{\pgfqpoint{3.890260in}{1.800000in}}%
\pgfusepath{stroke}%
\end{pgfscope}%
\begin{pgfscope}%
\pgfsetbuttcap%
\pgfsetroundjoin%
\definecolor{currentfill}{rgb}{0.150000,0.150000,0.150000}%
\pgfsetfillcolor{currentfill}%
\pgfsetlinewidth{0.803000pt}%
\definecolor{currentstroke}{rgb}{0.150000,0.150000,0.150000}%
\pgfsetstrokecolor{currentstroke}%
\pgfsetdash{}{0pt}%
\pgfsys@defobject{currentmarker}{\pgfqpoint{0.000000in}{0.000000in}}{\pgfqpoint{0.000000in}{0.000000in}}{%
\pgfpathmoveto{\pgfqpoint{0.000000in}{0.000000in}}%
\pgfpathlineto{\pgfqpoint{0.000000in}{0.000000in}}%
\pgfusepath{stroke,fill}%
}%
\begin{pgfscope}%
\pgfsys@transformshift{3.890260in}{0.250000in}%
\pgfsys@useobject{currentmarker}{}%
\end{pgfscope}%
\end{pgfscope}%
\begin{pgfscope}%
\pgfsetbuttcap%
\pgfsetroundjoin%
\definecolor{currentfill}{rgb}{0.150000,0.150000,0.150000}%
\pgfsetfillcolor{currentfill}%
\pgfsetlinewidth{0.803000pt}%
\definecolor{currentstroke}{rgb}{0.150000,0.150000,0.150000}%
\pgfsetstrokecolor{currentstroke}%
\pgfsetdash{}{0pt}%
\pgfsys@defobject{currentmarker}{\pgfqpoint{0.000000in}{0.000000in}}{\pgfqpoint{0.000000in}{0.000000in}}{%
\pgfpathmoveto{\pgfqpoint{0.000000in}{0.000000in}}%
\pgfpathlineto{\pgfqpoint{0.000000in}{0.000000in}}%
\pgfusepath{stroke,fill}%
}%
\begin{pgfscope}%
\pgfsys@transformshift{3.890260in}{1.800000in}%
\pgfsys@useobject{currentmarker}{}%
\end{pgfscope}%
\end{pgfscope}%
\begin{pgfscope}%
\definecolor{textcolor}{rgb}{0.150000,0.150000,0.150000}%
\pgfsetstrokecolor{textcolor}%
\pgfsetfillcolor{textcolor}%
\pgftext[x=3.890260in,y=0.172222in,,top]{\color{textcolor}\sffamily\fontsize{8.000000}{9.600000}\selectfont 2}%
\end{pgfscope}%
\begin{pgfscope}%
\pgfpathrectangle{\pgfqpoint{3.286364in}{0.250000in}}{\pgfqpoint{2.113636in}{1.550000in}} %
\pgfusepath{clip}%
\pgfsetroundcap%
\pgfsetroundjoin%
\pgfsetlinewidth{0.803000pt}%
\definecolor{currentstroke}{rgb}{1.000000,1.000000,1.000000}%
\pgfsetstrokecolor{currentstroke}%
\pgfsetdash{}{0pt}%
\pgfpathmoveto{\pgfqpoint{4.192208in}{0.250000in}}%
\pgfpathlineto{\pgfqpoint{4.192208in}{1.800000in}}%
\pgfusepath{stroke}%
\end{pgfscope}%
\begin{pgfscope}%
\pgfsetbuttcap%
\pgfsetroundjoin%
\definecolor{currentfill}{rgb}{0.150000,0.150000,0.150000}%
\pgfsetfillcolor{currentfill}%
\pgfsetlinewidth{0.803000pt}%
\definecolor{currentstroke}{rgb}{0.150000,0.150000,0.150000}%
\pgfsetstrokecolor{currentstroke}%
\pgfsetdash{}{0pt}%
\pgfsys@defobject{currentmarker}{\pgfqpoint{0.000000in}{0.000000in}}{\pgfqpoint{0.000000in}{0.000000in}}{%
\pgfpathmoveto{\pgfqpoint{0.000000in}{0.000000in}}%
\pgfpathlineto{\pgfqpoint{0.000000in}{0.000000in}}%
\pgfusepath{stroke,fill}%
}%
\begin{pgfscope}%
\pgfsys@transformshift{4.192208in}{0.250000in}%
\pgfsys@useobject{currentmarker}{}%
\end{pgfscope}%
\end{pgfscope}%
\begin{pgfscope}%
\pgfsetbuttcap%
\pgfsetroundjoin%
\definecolor{currentfill}{rgb}{0.150000,0.150000,0.150000}%
\pgfsetfillcolor{currentfill}%
\pgfsetlinewidth{0.803000pt}%
\definecolor{currentstroke}{rgb}{0.150000,0.150000,0.150000}%
\pgfsetstrokecolor{currentstroke}%
\pgfsetdash{}{0pt}%
\pgfsys@defobject{currentmarker}{\pgfqpoint{0.000000in}{0.000000in}}{\pgfqpoint{0.000000in}{0.000000in}}{%
\pgfpathmoveto{\pgfqpoint{0.000000in}{0.000000in}}%
\pgfpathlineto{\pgfqpoint{0.000000in}{0.000000in}}%
\pgfusepath{stroke,fill}%
}%
\begin{pgfscope}%
\pgfsys@transformshift{4.192208in}{1.800000in}%
\pgfsys@useobject{currentmarker}{}%
\end{pgfscope}%
\end{pgfscope}%
\begin{pgfscope}%
\definecolor{textcolor}{rgb}{0.150000,0.150000,0.150000}%
\pgfsetstrokecolor{textcolor}%
\pgfsetfillcolor{textcolor}%
\pgftext[x=4.192208in,y=0.172222in,,top]{\color{textcolor}\sffamily\fontsize{8.000000}{9.600000}\selectfont 3}%
\end{pgfscope}%
\begin{pgfscope}%
\pgfpathrectangle{\pgfqpoint{3.286364in}{0.250000in}}{\pgfqpoint{2.113636in}{1.550000in}} %
\pgfusepath{clip}%
\pgfsetroundcap%
\pgfsetroundjoin%
\pgfsetlinewidth{0.803000pt}%
\definecolor{currentstroke}{rgb}{1.000000,1.000000,1.000000}%
\pgfsetstrokecolor{currentstroke}%
\pgfsetdash{}{0pt}%
\pgfpathmoveto{\pgfqpoint{4.494156in}{0.250000in}}%
\pgfpathlineto{\pgfqpoint{4.494156in}{1.800000in}}%
\pgfusepath{stroke}%
\end{pgfscope}%
\begin{pgfscope}%
\pgfsetbuttcap%
\pgfsetroundjoin%
\definecolor{currentfill}{rgb}{0.150000,0.150000,0.150000}%
\pgfsetfillcolor{currentfill}%
\pgfsetlinewidth{0.803000pt}%
\definecolor{currentstroke}{rgb}{0.150000,0.150000,0.150000}%
\pgfsetstrokecolor{currentstroke}%
\pgfsetdash{}{0pt}%
\pgfsys@defobject{currentmarker}{\pgfqpoint{0.000000in}{0.000000in}}{\pgfqpoint{0.000000in}{0.000000in}}{%
\pgfpathmoveto{\pgfqpoint{0.000000in}{0.000000in}}%
\pgfpathlineto{\pgfqpoint{0.000000in}{0.000000in}}%
\pgfusepath{stroke,fill}%
}%
\begin{pgfscope}%
\pgfsys@transformshift{4.494156in}{0.250000in}%
\pgfsys@useobject{currentmarker}{}%
\end{pgfscope}%
\end{pgfscope}%
\begin{pgfscope}%
\pgfsetbuttcap%
\pgfsetroundjoin%
\definecolor{currentfill}{rgb}{0.150000,0.150000,0.150000}%
\pgfsetfillcolor{currentfill}%
\pgfsetlinewidth{0.803000pt}%
\definecolor{currentstroke}{rgb}{0.150000,0.150000,0.150000}%
\pgfsetstrokecolor{currentstroke}%
\pgfsetdash{}{0pt}%
\pgfsys@defobject{currentmarker}{\pgfqpoint{0.000000in}{0.000000in}}{\pgfqpoint{0.000000in}{0.000000in}}{%
\pgfpathmoveto{\pgfqpoint{0.000000in}{0.000000in}}%
\pgfpathlineto{\pgfqpoint{0.000000in}{0.000000in}}%
\pgfusepath{stroke,fill}%
}%
\begin{pgfscope}%
\pgfsys@transformshift{4.494156in}{1.800000in}%
\pgfsys@useobject{currentmarker}{}%
\end{pgfscope}%
\end{pgfscope}%
\begin{pgfscope}%
\definecolor{textcolor}{rgb}{0.150000,0.150000,0.150000}%
\pgfsetstrokecolor{textcolor}%
\pgfsetfillcolor{textcolor}%
\pgftext[x=4.494156in,y=0.172222in,,top]{\color{textcolor}\sffamily\fontsize{8.000000}{9.600000}\selectfont 4}%
\end{pgfscope}%
\begin{pgfscope}%
\pgfpathrectangle{\pgfqpoint{3.286364in}{0.250000in}}{\pgfqpoint{2.113636in}{1.550000in}} %
\pgfusepath{clip}%
\pgfsetroundcap%
\pgfsetroundjoin%
\pgfsetlinewidth{0.803000pt}%
\definecolor{currentstroke}{rgb}{1.000000,1.000000,1.000000}%
\pgfsetstrokecolor{currentstroke}%
\pgfsetdash{}{0pt}%
\pgfpathmoveto{\pgfqpoint{4.796104in}{0.250000in}}%
\pgfpathlineto{\pgfqpoint{4.796104in}{1.800000in}}%
\pgfusepath{stroke}%
\end{pgfscope}%
\begin{pgfscope}%
\pgfsetbuttcap%
\pgfsetroundjoin%
\definecolor{currentfill}{rgb}{0.150000,0.150000,0.150000}%
\pgfsetfillcolor{currentfill}%
\pgfsetlinewidth{0.803000pt}%
\definecolor{currentstroke}{rgb}{0.150000,0.150000,0.150000}%
\pgfsetstrokecolor{currentstroke}%
\pgfsetdash{}{0pt}%
\pgfsys@defobject{currentmarker}{\pgfqpoint{0.000000in}{0.000000in}}{\pgfqpoint{0.000000in}{0.000000in}}{%
\pgfpathmoveto{\pgfqpoint{0.000000in}{0.000000in}}%
\pgfpathlineto{\pgfqpoint{0.000000in}{0.000000in}}%
\pgfusepath{stroke,fill}%
}%
\begin{pgfscope}%
\pgfsys@transformshift{4.796104in}{0.250000in}%
\pgfsys@useobject{currentmarker}{}%
\end{pgfscope}%
\end{pgfscope}%
\begin{pgfscope}%
\pgfsetbuttcap%
\pgfsetroundjoin%
\definecolor{currentfill}{rgb}{0.150000,0.150000,0.150000}%
\pgfsetfillcolor{currentfill}%
\pgfsetlinewidth{0.803000pt}%
\definecolor{currentstroke}{rgb}{0.150000,0.150000,0.150000}%
\pgfsetstrokecolor{currentstroke}%
\pgfsetdash{}{0pt}%
\pgfsys@defobject{currentmarker}{\pgfqpoint{0.000000in}{0.000000in}}{\pgfqpoint{0.000000in}{0.000000in}}{%
\pgfpathmoveto{\pgfqpoint{0.000000in}{0.000000in}}%
\pgfpathlineto{\pgfqpoint{0.000000in}{0.000000in}}%
\pgfusepath{stroke,fill}%
}%
\begin{pgfscope}%
\pgfsys@transformshift{4.796104in}{1.800000in}%
\pgfsys@useobject{currentmarker}{}%
\end{pgfscope}%
\end{pgfscope}%
\begin{pgfscope}%
\definecolor{textcolor}{rgb}{0.150000,0.150000,0.150000}%
\pgfsetstrokecolor{textcolor}%
\pgfsetfillcolor{textcolor}%
\pgftext[x=4.796104in,y=0.172222in,,top]{\color{textcolor}\sffamily\fontsize{8.000000}{9.600000}\selectfont 5}%
\end{pgfscope}%
\begin{pgfscope}%
\pgfpathrectangle{\pgfqpoint{3.286364in}{0.250000in}}{\pgfqpoint{2.113636in}{1.550000in}} %
\pgfusepath{clip}%
\pgfsetroundcap%
\pgfsetroundjoin%
\pgfsetlinewidth{0.803000pt}%
\definecolor{currentstroke}{rgb}{1.000000,1.000000,1.000000}%
\pgfsetstrokecolor{currentstroke}%
\pgfsetdash{}{0pt}%
\pgfpathmoveto{\pgfqpoint{5.098052in}{0.250000in}}%
\pgfpathlineto{\pgfqpoint{5.098052in}{1.800000in}}%
\pgfusepath{stroke}%
\end{pgfscope}%
\begin{pgfscope}%
\pgfsetbuttcap%
\pgfsetroundjoin%
\definecolor{currentfill}{rgb}{0.150000,0.150000,0.150000}%
\pgfsetfillcolor{currentfill}%
\pgfsetlinewidth{0.803000pt}%
\definecolor{currentstroke}{rgb}{0.150000,0.150000,0.150000}%
\pgfsetstrokecolor{currentstroke}%
\pgfsetdash{}{0pt}%
\pgfsys@defobject{currentmarker}{\pgfqpoint{0.000000in}{0.000000in}}{\pgfqpoint{0.000000in}{0.000000in}}{%
\pgfpathmoveto{\pgfqpoint{0.000000in}{0.000000in}}%
\pgfpathlineto{\pgfqpoint{0.000000in}{0.000000in}}%
\pgfusepath{stroke,fill}%
}%
\begin{pgfscope}%
\pgfsys@transformshift{5.098052in}{0.250000in}%
\pgfsys@useobject{currentmarker}{}%
\end{pgfscope}%
\end{pgfscope}%
\begin{pgfscope}%
\pgfsetbuttcap%
\pgfsetroundjoin%
\definecolor{currentfill}{rgb}{0.150000,0.150000,0.150000}%
\pgfsetfillcolor{currentfill}%
\pgfsetlinewidth{0.803000pt}%
\definecolor{currentstroke}{rgb}{0.150000,0.150000,0.150000}%
\pgfsetstrokecolor{currentstroke}%
\pgfsetdash{}{0pt}%
\pgfsys@defobject{currentmarker}{\pgfqpoint{0.000000in}{0.000000in}}{\pgfqpoint{0.000000in}{0.000000in}}{%
\pgfpathmoveto{\pgfqpoint{0.000000in}{0.000000in}}%
\pgfpathlineto{\pgfqpoint{0.000000in}{0.000000in}}%
\pgfusepath{stroke,fill}%
}%
\begin{pgfscope}%
\pgfsys@transformshift{5.098052in}{1.800000in}%
\pgfsys@useobject{currentmarker}{}%
\end{pgfscope}%
\end{pgfscope}%
\begin{pgfscope}%
\definecolor{textcolor}{rgb}{0.150000,0.150000,0.150000}%
\pgfsetstrokecolor{textcolor}%
\pgfsetfillcolor{textcolor}%
\pgftext[x=5.098052in,y=0.172222in,,top]{\color{textcolor}\sffamily\fontsize{8.000000}{9.600000}\selectfont 6}%
\end{pgfscope}%
\begin{pgfscope}%
\pgfpathrectangle{\pgfqpoint{3.286364in}{0.250000in}}{\pgfqpoint{2.113636in}{1.550000in}} %
\pgfusepath{clip}%
\pgfsetroundcap%
\pgfsetroundjoin%
\pgfsetlinewidth{0.803000pt}%
\definecolor{currentstroke}{rgb}{1.000000,1.000000,1.000000}%
\pgfsetstrokecolor{currentstroke}%
\pgfsetdash{}{0pt}%
\pgfpathmoveto{\pgfqpoint{5.400000in}{0.250000in}}%
\pgfpathlineto{\pgfqpoint{5.400000in}{1.800000in}}%
\pgfusepath{stroke}%
\end{pgfscope}%
\begin{pgfscope}%
\pgfsetbuttcap%
\pgfsetroundjoin%
\definecolor{currentfill}{rgb}{0.150000,0.150000,0.150000}%
\pgfsetfillcolor{currentfill}%
\pgfsetlinewidth{0.803000pt}%
\definecolor{currentstroke}{rgb}{0.150000,0.150000,0.150000}%
\pgfsetstrokecolor{currentstroke}%
\pgfsetdash{}{0pt}%
\pgfsys@defobject{currentmarker}{\pgfqpoint{0.000000in}{0.000000in}}{\pgfqpoint{0.000000in}{0.000000in}}{%
\pgfpathmoveto{\pgfqpoint{0.000000in}{0.000000in}}%
\pgfpathlineto{\pgfqpoint{0.000000in}{0.000000in}}%
\pgfusepath{stroke,fill}%
}%
\begin{pgfscope}%
\pgfsys@transformshift{5.400000in}{0.250000in}%
\pgfsys@useobject{currentmarker}{}%
\end{pgfscope}%
\end{pgfscope}%
\begin{pgfscope}%
\pgfsetbuttcap%
\pgfsetroundjoin%
\definecolor{currentfill}{rgb}{0.150000,0.150000,0.150000}%
\pgfsetfillcolor{currentfill}%
\pgfsetlinewidth{0.803000pt}%
\definecolor{currentstroke}{rgb}{0.150000,0.150000,0.150000}%
\pgfsetstrokecolor{currentstroke}%
\pgfsetdash{}{0pt}%
\pgfsys@defobject{currentmarker}{\pgfqpoint{0.000000in}{0.000000in}}{\pgfqpoint{0.000000in}{0.000000in}}{%
\pgfpathmoveto{\pgfqpoint{0.000000in}{0.000000in}}%
\pgfpathlineto{\pgfqpoint{0.000000in}{0.000000in}}%
\pgfusepath{stroke,fill}%
}%
\begin{pgfscope}%
\pgfsys@transformshift{5.400000in}{1.800000in}%
\pgfsys@useobject{currentmarker}{}%
\end{pgfscope}%
\end{pgfscope}%
\begin{pgfscope}%
\definecolor{textcolor}{rgb}{0.150000,0.150000,0.150000}%
\pgfsetstrokecolor{textcolor}%
\pgfsetfillcolor{textcolor}%
\pgftext[x=5.400000in,y=0.172222in,,top]{\color{textcolor}\sffamily\fontsize{8.000000}{9.600000}\selectfont 7}%
\end{pgfscope}%
\begin{pgfscope}%
\pgfpathrectangle{\pgfqpoint{3.286364in}{0.250000in}}{\pgfqpoint{2.113636in}{1.550000in}} %
\pgfusepath{clip}%
\pgfsetroundcap%
\pgfsetroundjoin%
\pgfsetlinewidth{0.803000pt}%
\definecolor{currentstroke}{rgb}{1.000000,1.000000,1.000000}%
\pgfsetstrokecolor{currentstroke}%
\pgfsetdash{}{0pt}%
\pgfpathmoveto{\pgfqpoint{3.286364in}{0.250000in}}%
\pgfpathlineto{\pgfqpoint{5.400000in}{0.250000in}}%
\pgfusepath{stroke}%
\end{pgfscope}%
\begin{pgfscope}%
\pgfsetbuttcap%
\pgfsetroundjoin%
\definecolor{currentfill}{rgb}{0.150000,0.150000,0.150000}%
\pgfsetfillcolor{currentfill}%
\pgfsetlinewidth{0.803000pt}%
\definecolor{currentstroke}{rgb}{0.150000,0.150000,0.150000}%
\pgfsetstrokecolor{currentstroke}%
\pgfsetdash{}{0pt}%
\pgfsys@defobject{currentmarker}{\pgfqpoint{0.000000in}{0.000000in}}{\pgfqpoint{0.000000in}{0.000000in}}{%
\pgfpathmoveto{\pgfqpoint{0.000000in}{0.000000in}}%
\pgfpathlineto{\pgfqpoint{0.000000in}{0.000000in}}%
\pgfusepath{stroke,fill}%
}%
\begin{pgfscope}%
\pgfsys@transformshift{3.286364in}{0.250000in}%
\pgfsys@useobject{currentmarker}{}%
\end{pgfscope}%
\end{pgfscope}%
\begin{pgfscope}%
\pgfsetbuttcap%
\pgfsetroundjoin%
\definecolor{currentfill}{rgb}{0.150000,0.150000,0.150000}%
\pgfsetfillcolor{currentfill}%
\pgfsetlinewidth{0.803000pt}%
\definecolor{currentstroke}{rgb}{0.150000,0.150000,0.150000}%
\pgfsetstrokecolor{currentstroke}%
\pgfsetdash{}{0pt}%
\pgfsys@defobject{currentmarker}{\pgfqpoint{0.000000in}{0.000000in}}{\pgfqpoint{0.000000in}{0.000000in}}{%
\pgfpathmoveto{\pgfqpoint{0.000000in}{0.000000in}}%
\pgfpathlineto{\pgfqpoint{0.000000in}{0.000000in}}%
\pgfusepath{stroke,fill}%
}%
\begin{pgfscope}%
\pgfsys@transformshift{5.400000in}{0.250000in}%
\pgfsys@useobject{currentmarker}{}%
\end{pgfscope}%
\end{pgfscope}%
\begin{pgfscope}%
\definecolor{textcolor}{rgb}{0.150000,0.150000,0.150000}%
\pgfsetstrokecolor{textcolor}%
\pgfsetfillcolor{textcolor}%
\pgftext[x=3.208586in,y=0.250000in,right,]{\color{textcolor}\sffamily\fontsize{8.000000}{9.600000}\selectfont 0}%
\end{pgfscope}%
\begin{pgfscope}%
\pgfpathrectangle{\pgfqpoint{3.286364in}{0.250000in}}{\pgfqpoint{2.113636in}{1.550000in}} %
\pgfusepath{clip}%
\pgfsetroundcap%
\pgfsetroundjoin%
\pgfsetlinewidth{0.803000pt}%
\definecolor{currentstroke}{rgb}{1.000000,1.000000,1.000000}%
\pgfsetstrokecolor{currentstroke}%
\pgfsetdash{}{0pt}%
\pgfpathmoveto{\pgfqpoint{3.286364in}{0.471429in}}%
\pgfpathlineto{\pgfqpoint{5.400000in}{0.471429in}}%
\pgfusepath{stroke}%
\end{pgfscope}%
\begin{pgfscope}%
\pgfsetbuttcap%
\pgfsetroundjoin%
\definecolor{currentfill}{rgb}{0.150000,0.150000,0.150000}%
\pgfsetfillcolor{currentfill}%
\pgfsetlinewidth{0.803000pt}%
\definecolor{currentstroke}{rgb}{0.150000,0.150000,0.150000}%
\pgfsetstrokecolor{currentstroke}%
\pgfsetdash{}{0pt}%
\pgfsys@defobject{currentmarker}{\pgfqpoint{0.000000in}{0.000000in}}{\pgfqpoint{0.000000in}{0.000000in}}{%
\pgfpathmoveto{\pgfqpoint{0.000000in}{0.000000in}}%
\pgfpathlineto{\pgfqpoint{0.000000in}{0.000000in}}%
\pgfusepath{stroke,fill}%
}%
\begin{pgfscope}%
\pgfsys@transformshift{3.286364in}{0.471429in}%
\pgfsys@useobject{currentmarker}{}%
\end{pgfscope}%
\end{pgfscope}%
\begin{pgfscope}%
\pgfsetbuttcap%
\pgfsetroundjoin%
\definecolor{currentfill}{rgb}{0.150000,0.150000,0.150000}%
\pgfsetfillcolor{currentfill}%
\pgfsetlinewidth{0.803000pt}%
\definecolor{currentstroke}{rgb}{0.150000,0.150000,0.150000}%
\pgfsetstrokecolor{currentstroke}%
\pgfsetdash{}{0pt}%
\pgfsys@defobject{currentmarker}{\pgfqpoint{0.000000in}{0.000000in}}{\pgfqpoint{0.000000in}{0.000000in}}{%
\pgfpathmoveto{\pgfqpoint{0.000000in}{0.000000in}}%
\pgfpathlineto{\pgfqpoint{0.000000in}{0.000000in}}%
\pgfusepath{stroke,fill}%
}%
\begin{pgfscope}%
\pgfsys@transformshift{5.400000in}{0.471429in}%
\pgfsys@useobject{currentmarker}{}%
\end{pgfscope}%
\end{pgfscope}%
\begin{pgfscope}%
\definecolor{textcolor}{rgb}{0.150000,0.150000,0.150000}%
\pgfsetstrokecolor{textcolor}%
\pgfsetfillcolor{textcolor}%
\pgftext[x=3.208586in,y=0.471429in,right,]{\color{textcolor}\sffamily\fontsize{8.000000}{9.600000}\selectfont 2}%
\end{pgfscope}%
\begin{pgfscope}%
\pgfpathrectangle{\pgfqpoint{3.286364in}{0.250000in}}{\pgfqpoint{2.113636in}{1.550000in}} %
\pgfusepath{clip}%
\pgfsetroundcap%
\pgfsetroundjoin%
\pgfsetlinewidth{0.803000pt}%
\definecolor{currentstroke}{rgb}{1.000000,1.000000,1.000000}%
\pgfsetstrokecolor{currentstroke}%
\pgfsetdash{}{0pt}%
\pgfpathmoveto{\pgfqpoint{3.286364in}{0.692857in}}%
\pgfpathlineto{\pgfqpoint{5.400000in}{0.692857in}}%
\pgfusepath{stroke}%
\end{pgfscope}%
\begin{pgfscope}%
\pgfsetbuttcap%
\pgfsetroundjoin%
\definecolor{currentfill}{rgb}{0.150000,0.150000,0.150000}%
\pgfsetfillcolor{currentfill}%
\pgfsetlinewidth{0.803000pt}%
\definecolor{currentstroke}{rgb}{0.150000,0.150000,0.150000}%
\pgfsetstrokecolor{currentstroke}%
\pgfsetdash{}{0pt}%
\pgfsys@defobject{currentmarker}{\pgfqpoint{0.000000in}{0.000000in}}{\pgfqpoint{0.000000in}{0.000000in}}{%
\pgfpathmoveto{\pgfqpoint{0.000000in}{0.000000in}}%
\pgfpathlineto{\pgfqpoint{0.000000in}{0.000000in}}%
\pgfusepath{stroke,fill}%
}%
\begin{pgfscope}%
\pgfsys@transformshift{3.286364in}{0.692857in}%
\pgfsys@useobject{currentmarker}{}%
\end{pgfscope}%
\end{pgfscope}%
\begin{pgfscope}%
\pgfsetbuttcap%
\pgfsetroundjoin%
\definecolor{currentfill}{rgb}{0.150000,0.150000,0.150000}%
\pgfsetfillcolor{currentfill}%
\pgfsetlinewidth{0.803000pt}%
\definecolor{currentstroke}{rgb}{0.150000,0.150000,0.150000}%
\pgfsetstrokecolor{currentstroke}%
\pgfsetdash{}{0pt}%
\pgfsys@defobject{currentmarker}{\pgfqpoint{0.000000in}{0.000000in}}{\pgfqpoint{0.000000in}{0.000000in}}{%
\pgfpathmoveto{\pgfqpoint{0.000000in}{0.000000in}}%
\pgfpathlineto{\pgfqpoint{0.000000in}{0.000000in}}%
\pgfusepath{stroke,fill}%
}%
\begin{pgfscope}%
\pgfsys@transformshift{5.400000in}{0.692857in}%
\pgfsys@useobject{currentmarker}{}%
\end{pgfscope}%
\end{pgfscope}%
\begin{pgfscope}%
\definecolor{textcolor}{rgb}{0.150000,0.150000,0.150000}%
\pgfsetstrokecolor{textcolor}%
\pgfsetfillcolor{textcolor}%
\pgftext[x=3.208586in,y=0.692857in,right,]{\color{textcolor}\sffamily\fontsize{8.000000}{9.600000}\selectfont 4}%
\end{pgfscope}%
\begin{pgfscope}%
\pgfpathrectangle{\pgfqpoint{3.286364in}{0.250000in}}{\pgfqpoint{2.113636in}{1.550000in}} %
\pgfusepath{clip}%
\pgfsetroundcap%
\pgfsetroundjoin%
\pgfsetlinewidth{0.803000pt}%
\definecolor{currentstroke}{rgb}{1.000000,1.000000,1.000000}%
\pgfsetstrokecolor{currentstroke}%
\pgfsetdash{}{0pt}%
\pgfpathmoveto{\pgfqpoint{3.286364in}{0.914286in}}%
\pgfpathlineto{\pgfqpoint{5.400000in}{0.914286in}}%
\pgfusepath{stroke}%
\end{pgfscope}%
\begin{pgfscope}%
\pgfsetbuttcap%
\pgfsetroundjoin%
\definecolor{currentfill}{rgb}{0.150000,0.150000,0.150000}%
\pgfsetfillcolor{currentfill}%
\pgfsetlinewidth{0.803000pt}%
\definecolor{currentstroke}{rgb}{0.150000,0.150000,0.150000}%
\pgfsetstrokecolor{currentstroke}%
\pgfsetdash{}{0pt}%
\pgfsys@defobject{currentmarker}{\pgfqpoint{0.000000in}{0.000000in}}{\pgfqpoint{0.000000in}{0.000000in}}{%
\pgfpathmoveto{\pgfqpoint{0.000000in}{0.000000in}}%
\pgfpathlineto{\pgfqpoint{0.000000in}{0.000000in}}%
\pgfusepath{stroke,fill}%
}%
\begin{pgfscope}%
\pgfsys@transformshift{3.286364in}{0.914286in}%
\pgfsys@useobject{currentmarker}{}%
\end{pgfscope}%
\end{pgfscope}%
\begin{pgfscope}%
\pgfsetbuttcap%
\pgfsetroundjoin%
\definecolor{currentfill}{rgb}{0.150000,0.150000,0.150000}%
\pgfsetfillcolor{currentfill}%
\pgfsetlinewidth{0.803000pt}%
\definecolor{currentstroke}{rgb}{0.150000,0.150000,0.150000}%
\pgfsetstrokecolor{currentstroke}%
\pgfsetdash{}{0pt}%
\pgfsys@defobject{currentmarker}{\pgfqpoint{0.000000in}{0.000000in}}{\pgfqpoint{0.000000in}{0.000000in}}{%
\pgfpathmoveto{\pgfqpoint{0.000000in}{0.000000in}}%
\pgfpathlineto{\pgfqpoint{0.000000in}{0.000000in}}%
\pgfusepath{stroke,fill}%
}%
\begin{pgfscope}%
\pgfsys@transformshift{5.400000in}{0.914286in}%
\pgfsys@useobject{currentmarker}{}%
\end{pgfscope}%
\end{pgfscope}%
\begin{pgfscope}%
\definecolor{textcolor}{rgb}{0.150000,0.150000,0.150000}%
\pgfsetstrokecolor{textcolor}%
\pgfsetfillcolor{textcolor}%
\pgftext[x=3.208586in,y=0.914286in,right,]{\color{textcolor}\sffamily\fontsize{8.000000}{9.600000}\selectfont 6}%
\end{pgfscope}%
\begin{pgfscope}%
\pgfpathrectangle{\pgfqpoint{3.286364in}{0.250000in}}{\pgfqpoint{2.113636in}{1.550000in}} %
\pgfusepath{clip}%
\pgfsetroundcap%
\pgfsetroundjoin%
\pgfsetlinewidth{0.803000pt}%
\definecolor{currentstroke}{rgb}{1.000000,1.000000,1.000000}%
\pgfsetstrokecolor{currentstroke}%
\pgfsetdash{}{0pt}%
\pgfpathmoveto{\pgfqpoint{3.286364in}{1.135714in}}%
\pgfpathlineto{\pgfqpoint{5.400000in}{1.135714in}}%
\pgfusepath{stroke}%
\end{pgfscope}%
\begin{pgfscope}%
\pgfsetbuttcap%
\pgfsetroundjoin%
\definecolor{currentfill}{rgb}{0.150000,0.150000,0.150000}%
\pgfsetfillcolor{currentfill}%
\pgfsetlinewidth{0.803000pt}%
\definecolor{currentstroke}{rgb}{0.150000,0.150000,0.150000}%
\pgfsetstrokecolor{currentstroke}%
\pgfsetdash{}{0pt}%
\pgfsys@defobject{currentmarker}{\pgfqpoint{0.000000in}{0.000000in}}{\pgfqpoint{0.000000in}{0.000000in}}{%
\pgfpathmoveto{\pgfqpoint{0.000000in}{0.000000in}}%
\pgfpathlineto{\pgfqpoint{0.000000in}{0.000000in}}%
\pgfusepath{stroke,fill}%
}%
\begin{pgfscope}%
\pgfsys@transformshift{3.286364in}{1.135714in}%
\pgfsys@useobject{currentmarker}{}%
\end{pgfscope}%
\end{pgfscope}%
\begin{pgfscope}%
\pgfsetbuttcap%
\pgfsetroundjoin%
\definecolor{currentfill}{rgb}{0.150000,0.150000,0.150000}%
\pgfsetfillcolor{currentfill}%
\pgfsetlinewidth{0.803000pt}%
\definecolor{currentstroke}{rgb}{0.150000,0.150000,0.150000}%
\pgfsetstrokecolor{currentstroke}%
\pgfsetdash{}{0pt}%
\pgfsys@defobject{currentmarker}{\pgfqpoint{0.000000in}{0.000000in}}{\pgfqpoint{0.000000in}{0.000000in}}{%
\pgfpathmoveto{\pgfqpoint{0.000000in}{0.000000in}}%
\pgfpathlineto{\pgfqpoint{0.000000in}{0.000000in}}%
\pgfusepath{stroke,fill}%
}%
\begin{pgfscope}%
\pgfsys@transformshift{5.400000in}{1.135714in}%
\pgfsys@useobject{currentmarker}{}%
\end{pgfscope}%
\end{pgfscope}%
\begin{pgfscope}%
\definecolor{textcolor}{rgb}{0.150000,0.150000,0.150000}%
\pgfsetstrokecolor{textcolor}%
\pgfsetfillcolor{textcolor}%
\pgftext[x=3.208586in,y=1.135714in,right,]{\color{textcolor}\sffamily\fontsize{8.000000}{9.600000}\selectfont 8}%
\end{pgfscope}%
\begin{pgfscope}%
\pgfpathrectangle{\pgfqpoint{3.286364in}{0.250000in}}{\pgfqpoint{2.113636in}{1.550000in}} %
\pgfusepath{clip}%
\pgfsetroundcap%
\pgfsetroundjoin%
\pgfsetlinewidth{0.803000pt}%
\definecolor{currentstroke}{rgb}{1.000000,1.000000,1.000000}%
\pgfsetstrokecolor{currentstroke}%
\pgfsetdash{}{0pt}%
\pgfpathmoveto{\pgfqpoint{3.286364in}{1.357143in}}%
\pgfpathlineto{\pgfqpoint{5.400000in}{1.357143in}}%
\pgfusepath{stroke}%
\end{pgfscope}%
\begin{pgfscope}%
\pgfsetbuttcap%
\pgfsetroundjoin%
\definecolor{currentfill}{rgb}{0.150000,0.150000,0.150000}%
\pgfsetfillcolor{currentfill}%
\pgfsetlinewidth{0.803000pt}%
\definecolor{currentstroke}{rgb}{0.150000,0.150000,0.150000}%
\pgfsetstrokecolor{currentstroke}%
\pgfsetdash{}{0pt}%
\pgfsys@defobject{currentmarker}{\pgfqpoint{0.000000in}{0.000000in}}{\pgfqpoint{0.000000in}{0.000000in}}{%
\pgfpathmoveto{\pgfqpoint{0.000000in}{0.000000in}}%
\pgfpathlineto{\pgfqpoint{0.000000in}{0.000000in}}%
\pgfusepath{stroke,fill}%
}%
\begin{pgfscope}%
\pgfsys@transformshift{3.286364in}{1.357143in}%
\pgfsys@useobject{currentmarker}{}%
\end{pgfscope}%
\end{pgfscope}%
\begin{pgfscope}%
\pgfsetbuttcap%
\pgfsetroundjoin%
\definecolor{currentfill}{rgb}{0.150000,0.150000,0.150000}%
\pgfsetfillcolor{currentfill}%
\pgfsetlinewidth{0.803000pt}%
\definecolor{currentstroke}{rgb}{0.150000,0.150000,0.150000}%
\pgfsetstrokecolor{currentstroke}%
\pgfsetdash{}{0pt}%
\pgfsys@defobject{currentmarker}{\pgfqpoint{0.000000in}{0.000000in}}{\pgfqpoint{0.000000in}{0.000000in}}{%
\pgfpathmoveto{\pgfqpoint{0.000000in}{0.000000in}}%
\pgfpathlineto{\pgfqpoint{0.000000in}{0.000000in}}%
\pgfusepath{stroke,fill}%
}%
\begin{pgfscope}%
\pgfsys@transformshift{5.400000in}{1.357143in}%
\pgfsys@useobject{currentmarker}{}%
\end{pgfscope}%
\end{pgfscope}%
\begin{pgfscope}%
\definecolor{textcolor}{rgb}{0.150000,0.150000,0.150000}%
\pgfsetstrokecolor{textcolor}%
\pgfsetfillcolor{textcolor}%
\pgftext[x=3.208586in,y=1.357143in,right,]{\color{textcolor}\sffamily\fontsize{8.000000}{9.600000}\selectfont 10}%
\end{pgfscope}%
\begin{pgfscope}%
\pgfpathrectangle{\pgfqpoint{3.286364in}{0.250000in}}{\pgfqpoint{2.113636in}{1.550000in}} %
\pgfusepath{clip}%
\pgfsetroundcap%
\pgfsetroundjoin%
\pgfsetlinewidth{0.803000pt}%
\definecolor{currentstroke}{rgb}{1.000000,1.000000,1.000000}%
\pgfsetstrokecolor{currentstroke}%
\pgfsetdash{}{0pt}%
\pgfpathmoveto{\pgfqpoint{3.286364in}{1.578571in}}%
\pgfpathlineto{\pgfqpoint{5.400000in}{1.578571in}}%
\pgfusepath{stroke}%
\end{pgfscope}%
\begin{pgfscope}%
\pgfsetbuttcap%
\pgfsetroundjoin%
\definecolor{currentfill}{rgb}{0.150000,0.150000,0.150000}%
\pgfsetfillcolor{currentfill}%
\pgfsetlinewidth{0.803000pt}%
\definecolor{currentstroke}{rgb}{0.150000,0.150000,0.150000}%
\pgfsetstrokecolor{currentstroke}%
\pgfsetdash{}{0pt}%
\pgfsys@defobject{currentmarker}{\pgfqpoint{0.000000in}{0.000000in}}{\pgfqpoint{0.000000in}{0.000000in}}{%
\pgfpathmoveto{\pgfqpoint{0.000000in}{0.000000in}}%
\pgfpathlineto{\pgfqpoint{0.000000in}{0.000000in}}%
\pgfusepath{stroke,fill}%
}%
\begin{pgfscope}%
\pgfsys@transformshift{3.286364in}{1.578571in}%
\pgfsys@useobject{currentmarker}{}%
\end{pgfscope}%
\end{pgfscope}%
\begin{pgfscope}%
\pgfsetbuttcap%
\pgfsetroundjoin%
\definecolor{currentfill}{rgb}{0.150000,0.150000,0.150000}%
\pgfsetfillcolor{currentfill}%
\pgfsetlinewidth{0.803000pt}%
\definecolor{currentstroke}{rgb}{0.150000,0.150000,0.150000}%
\pgfsetstrokecolor{currentstroke}%
\pgfsetdash{}{0pt}%
\pgfsys@defobject{currentmarker}{\pgfqpoint{0.000000in}{0.000000in}}{\pgfqpoint{0.000000in}{0.000000in}}{%
\pgfpathmoveto{\pgfqpoint{0.000000in}{0.000000in}}%
\pgfpathlineto{\pgfqpoint{0.000000in}{0.000000in}}%
\pgfusepath{stroke,fill}%
}%
\begin{pgfscope}%
\pgfsys@transformshift{5.400000in}{1.578571in}%
\pgfsys@useobject{currentmarker}{}%
\end{pgfscope}%
\end{pgfscope}%
\begin{pgfscope}%
\definecolor{textcolor}{rgb}{0.150000,0.150000,0.150000}%
\pgfsetstrokecolor{textcolor}%
\pgfsetfillcolor{textcolor}%
\pgftext[x=3.208586in,y=1.578571in,right,]{\color{textcolor}\sffamily\fontsize{8.000000}{9.600000}\selectfont 12}%
\end{pgfscope}%
\begin{pgfscope}%
\pgfpathrectangle{\pgfqpoint{3.286364in}{0.250000in}}{\pgfqpoint{2.113636in}{1.550000in}} %
\pgfusepath{clip}%
\pgfsetroundcap%
\pgfsetroundjoin%
\pgfsetlinewidth{0.803000pt}%
\definecolor{currentstroke}{rgb}{1.000000,1.000000,1.000000}%
\pgfsetstrokecolor{currentstroke}%
\pgfsetdash{}{0pt}%
\pgfpathmoveto{\pgfqpoint{3.286364in}{1.800000in}}%
\pgfpathlineto{\pgfqpoint{5.400000in}{1.800000in}}%
\pgfusepath{stroke}%
\end{pgfscope}%
\begin{pgfscope}%
\pgfsetbuttcap%
\pgfsetroundjoin%
\definecolor{currentfill}{rgb}{0.150000,0.150000,0.150000}%
\pgfsetfillcolor{currentfill}%
\pgfsetlinewidth{0.803000pt}%
\definecolor{currentstroke}{rgb}{0.150000,0.150000,0.150000}%
\pgfsetstrokecolor{currentstroke}%
\pgfsetdash{}{0pt}%
\pgfsys@defobject{currentmarker}{\pgfqpoint{0.000000in}{0.000000in}}{\pgfqpoint{0.000000in}{0.000000in}}{%
\pgfpathmoveto{\pgfqpoint{0.000000in}{0.000000in}}%
\pgfpathlineto{\pgfqpoint{0.000000in}{0.000000in}}%
\pgfusepath{stroke,fill}%
}%
\begin{pgfscope}%
\pgfsys@transformshift{3.286364in}{1.800000in}%
\pgfsys@useobject{currentmarker}{}%
\end{pgfscope}%
\end{pgfscope}%
\begin{pgfscope}%
\pgfsetbuttcap%
\pgfsetroundjoin%
\definecolor{currentfill}{rgb}{0.150000,0.150000,0.150000}%
\pgfsetfillcolor{currentfill}%
\pgfsetlinewidth{0.803000pt}%
\definecolor{currentstroke}{rgb}{0.150000,0.150000,0.150000}%
\pgfsetstrokecolor{currentstroke}%
\pgfsetdash{}{0pt}%
\pgfsys@defobject{currentmarker}{\pgfqpoint{0.000000in}{0.000000in}}{\pgfqpoint{0.000000in}{0.000000in}}{%
\pgfpathmoveto{\pgfqpoint{0.000000in}{0.000000in}}%
\pgfpathlineto{\pgfqpoint{0.000000in}{0.000000in}}%
\pgfusepath{stroke,fill}%
}%
\begin{pgfscope}%
\pgfsys@transformshift{5.400000in}{1.800000in}%
\pgfsys@useobject{currentmarker}{}%
\end{pgfscope}%
\end{pgfscope}%
\begin{pgfscope}%
\definecolor{textcolor}{rgb}{0.150000,0.150000,0.150000}%
\pgfsetstrokecolor{textcolor}%
\pgfsetfillcolor{textcolor}%
\pgftext[x=3.208586in,y=1.800000in,right,]{\color{textcolor}\sffamily\fontsize{8.000000}{9.600000}\selectfont 14}%
\end{pgfscope}%
\begin{pgfscope}%
\pgfpathrectangle{\pgfqpoint{3.286364in}{0.250000in}}{\pgfqpoint{2.113636in}{1.550000in}} %
\pgfusepath{clip}%
\pgfsetroundcap%
\pgfsetroundjoin%
\pgfsetlinewidth{1.405250pt}%
\definecolor{currentstroke}{rgb}{0.298039,0.447059,0.690196}%
\pgfsetstrokecolor{currentstroke}%
\pgfsetdash{}{0pt}%
\pgfpathmoveto{\pgfqpoint{3.286364in}{0.250000in}}%
\pgfpathlineto{\pgfqpoint{3.351784in}{0.396625in}}%
\pgfpathlineto{\pgfqpoint{3.417205in}{0.256387in}}%
\pgfpathlineto{\pgfqpoint{3.482625in}{0.376590in}}%
\pgfpathlineto{\pgfqpoint{3.548046in}{0.547202in}}%
\pgfpathlineto{\pgfqpoint{3.613466in}{0.411607in}}%
\pgfpathlineto{\pgfqpoint{3.678887in}{0.503607in}}%
\pgfpathlineto{\pgfqpoint{3.744307in}{0.695182in}}%
\pgfpathlineto{\pgfqpoint{3.809728in}{0.568759in}}%
\pgfpathlineto{\pgfqpoint{3.875148in}{0.631470in}}%
\pgfpathlineto{\pgfqpoint{3.940569in}{0.840465in}}%
\pgfpathlineto{\pgfqpoint{4.005990in}{0.727515in}}%
\pgfpathlineto{\pgfqpoint{4.071410in}{0.760573in}}%
\pgfpathlineto{\pgfqpoint{4.136831in}{0.983016in}}%
\pgfpathlineto{\pgfqpoint{4.202251in}{0.887509in}}%
\pgfpathlineto{\pgfqpoint{4.267672in}{0.891283in}}%
\pgfpathlineto{\pgfqpoint{4.333092in}{1.122870in}}%
\pgfpathlineto{\pgfqpoint{4.398513in}{1.048346in}}%
\pgfpathlineto{\pgfqpoint{4.463933in}{1.023925in}}%
\pgfpathlineto{\pgfqpoint{4.529354in}{1.260127in}}%
\pgfpathlineto{\pgfqpoint{4.594774in}{1.209608in}}%
\pgfpathlineto{\pgfqpoint{4.660195in}{1.158778in}}%
\pgfpathlineto{\pgfqpoint{4.725615in}{1.394952in}}%
\pgfpathlineto{\pgfqpoint{4.791036in}{1.370867in}}%
\pgfpathlineto{\pgfqpoint{4.856457in}{1.296066in}}%
\pgfpathlineto{\pgfqpoint{4.921877in}{1.527569in}}%
\pgfpathlineto{\pgfqpoint{4.987298in}{1.531696in}}%
\pgfpathlineto{\pgfqpoint{5.052718in}{1.435953in}}%
\pgfpathlineto{\pgfqpoint{5.118139in}{1.658257in}}%
\pgfpathlineto{\pgfqpoint{5.183559in}{1.691678in}}%
\pgfusepath{stroke}%
\end{pgfscope}%
\begin{pgfscope}%
\pgfsetrectcap%
\pgfsetmiterjoin%
\pgfsetlinewidth{0.000000pt}%
\definecolor{currentstroke}{rgb}{1.000000,1.000000,1.000000}%
\pgfsetstrokecolor{currentstroke}%
\pgfsetdash{}{0pt}%
\pgfpathmoveto{\pgfqpoint{5.400000in}{0.250000in}}%
\pgfpathlineto{\pgfqpoint{5.400000in}{1.800000in}}%
\pgfusepath{}%
\end{pgfscope}%
\begin{pgfscope}%
\pgfsetrectcap%
\pgfsetmiterjoin%
\pgfsetlinewidth{0.000000pt}%
\definecolor{currentstroke}{rgb}{1.000000,1.000000,1.000000}%
\pgfsetstrokecolor{currentstroke}%
\pgfsetdash{}{0pt}%
\pgfpathmoveto{\pgfqpoint{3.286364in}{1.800000in}}%
\pgfpathlineto{\pgfqpoint{5.400000in}{1.800000in}}%
\pgfusepath{}%
\end{pgfscope}%
\begin{pgfscope}%
\pgfsetrectcap%
\pgfsetmiterjoin%
\pgfsetlinewidth{0.000000pt}%
\definecolor{currentstroke}{rgb}{1.000000,1.000000,1.000000}%
\pgfsetstrokecolor{currentstroke}%
\pgfsetdash{}{0pt}%
\pgfpathmoveto{\pgfqpoint{3.286364in}{0.250000in}}%
\pgfpathlineto{\pgfqpoint{5.400000in}{0.250000in}}%
\pgfusepath{}%
\end{pgfscope}%
\begin{pgfscope}%
\pgfsetrectcap%
\pgfsetmiterjoin%
\pgfsetlinewidth{0.000000pt}%
\definecolor{currentstroke}{rgb}{1.000000,1.000000,1.000000}%
\pgfsetstrokecolor{currentstroke}%
\pgfsetdash{}{0pt}%
\pgfpathmoveto{\pgfqpoint{3.286364in}{0.250000in}}%
\pgfpathlineto{\pgfqpoint{3.286364in}{1.800000in}}%
\pgfusepath{}%
\end{pgfscope}%
\end{pgfpicture}%
\makeatother%
\endgroup%

  \caption{Plots for $f(x) = \sin(3\pi x)$ using $150$ (left) and $30$ (right)
  samples.}
  \label{fig_gpsamples}
\end{figure}

The prediction was performed using the \emph{Matern32} kernel as it provided the
best results. The model and its variance are presented
in~\cref{fig_gppredftest}. $20$ points were predicted using the two trained
models. The quality of the predictions is measured by the sum of variances
between real and estimated values. The respective quantities were: $0.000132$
and $0.095207$ which let us conclude that the model trained with $150$ samples
is better than the one trained with just $30$. This is also appreciated in the
plots by looking at the variance. The light blue area is quite noticeable for
the second model.

\begin{figure}
  \begin{subfigure}[h]{.5\linewidth}
    %% Creator: Matplotlib, PGF backend
%%
%% To include the figure in your LaTeX document, write
%%   \input{<filename>.pgf}
%%
%% Make sure the required packages are loaded in your preamble
%%   \usepackage{pgf}
%%
%% Figures using additional raster images can only be included by \input if
%% they are in the same directory as the main LaTeX file. For loading figures
%% from other directories you can use the `import` package
%%   \usepackage{import}
%% and then include the figures with
%%   \import{<path to file>}{<filename>.pgf}
%%
%% Matplotlib used the following preamble
%%   \usepackage[utf8x]{inputenc}
%%   \usepackage[T1]{fontenc}
%%   \usepackage{cmbright}
%%
\begingroup%
\makeatletter%
\begin{pgfpicture}%
\pgfpathrectangle{\pgfpointorigin}{\pgfqpoint{3.000000in}{2.000000in}}%
\pgfusepath{use as bounding box, clip}%
\begin{pgfscope}%
\pgfsetbuttcap%
\pgfsetmiterjoin%
\definecolor{currentfill}{rgb}{1.000000,1.000000,1.000000}%
\pgfsetfillcolor{currentfill}%
\pgfsetlinewidth{0.000000pt}%
\definecolor{currentstroke}{rgb}{1.000000,1.000000,1.000000}%
\pgfsetstrokecolor{currentstroke}%
\pgfsetdash{}{0pt}%
\pgfpathmoveto{\pgfqpoint{0.000000in}{0.000000in}}%
\pgfpathlineto{\pgfqpoint{3.000000in}{0.000000in}}%
\pgfpathlineto{\pgfqpoint{3.000000in}{2.000000in}}%
\pgfpathlineto{\pgfqpoint{0.000000in}{2.000000in}}%
\pgfpathclose%
\pgfusepath{fill}%
\end{pgfscope}%
\begin{pgfscope}%
\pgfsetbuttcap%
\pgfsetmiterjoin%
\definecolor{currentfill}{rgb}{0.917647,0.917647,0.949020}%
\pgfsetfillcolor{currentfill}%
\pgfsetlinewidth{0.000000pt}%
\definecolor{currentstroke}{rgb}{0.000000,0.000000,0.000000}%
\pgfsetstrokecolor{currentstroke}%
\pgfsetstrokeopacity{0.000000}%
\pgfsetdash{}{0pt}%
\pgfpathmoveto{\pgfqpoint{0.375000in}{0.250000in}}%
\pgfpathlineto{\pgfqpoint{2.700000in}{0.250000in}}%
\pgfpathlineto{\pgfqpoint{2.700000in}{1.800000in}}%
\pgfpathlineto{\pgfqpoint{0.375000in}{1.800000in}}%
\pgfpathclose%
\pgfusepath{fill}%
\end{pgfscope}%
\begin{pgfscope}%
\pgfpathrectangle{\pgfqpoint{0.375000in}{0.250000in}}{\pgfqpoint{2.325000in}{1.550000in}} %
\pgfusepath{clip}%
\pgfsetroundcap%
\pgfsetroundjoin%
\pgfsetlinewidth{0.803000pt}%
\definecolor{currentstroke}{rgb}{1.000000,1.000000,1.000000}%
\pgfsetstrokecolor{currentstroke}%
\pgfsetdash{}{0pt}%
\pgfpathmoveto{\pgfqpoint{0.375000in}{0.250000in}}%
\pgfpathlineto{\pgfqpoint{0.375000in}{1.800000in}}%
\pgfusepath{stroke}%
\end{pgfscope}%
\begin{pgfscope}%
\pgfsetbuttcap%
\pgfsetroundjoin%
\definecolor{currentfill}{rgb}{0.150000,0.150000,0.150000}%
\pgfsetfillcolor{currentfill}%
\pgfsetlinewidth{0.803000pt}%
\definecolor{currentstroke}{rgb}{0.150000,0.150000,0.150000}%
\pgfsetstrokecolor{currentstroke}%
\pgfsetdash{}{0pt}%
\pgfsys@defobject{currentmarker}{\pgfqpoint{0.000000in}{0.000000in}}{\pgfqpoint{0.000000in}{0.000000in}}{%
\pgfpathmoveto{\pgfqpoint{0.000000in}{0.000000in}}%
\pgfpathlineto{\pgfqpoint{0.000000in}{0.000000in}}%
\pgfusepath{stroke,fill}%
}%
\begin{pgfscope}%
\pgfsys@transformshift{0.375000in}{0.250000in}%
\pgfsys@useobject{currentmarker}{}%
\end{pgfscope}%
\end{pgfscope}%
\begin{pgfscope}%
\pgfsetbuttcap%
\pgfsetroundjoin%
\definecolor{currentfill}{rgb}{0.150000,0.150000,0.150000}%
\pgfsetfillcolor{currentfill}%
\pgfsetlinewidth{0.803000pt}%
\definecolor{currentstroke}{rgb}{0.150000,0.150000,0.150000}%
\pgfsetstrokecolor{currentstroke}%
\pgfsetdash{}{0pt}%
\pgfsys@defobject{currentmarker}{\pgfqpoint{0.000000in}{0.000000in}}{\pgfqpoint{0.000000in}{0.000000in}}{%
\pgfpathmoveto{\pgfqpoint{0.000000in}{0.000000in}}%
\pgfpathlineto{\pgfqpoint{0.000000in}{0.000000in}}%
\pgfusepath{stroke,fill}%
}%
\begin{pgfscope}%
\pgfsys@transformshift{0.375000in}{1.800000in}%
\pgfsys@useobject{currentmarker}{}%
\end{pgfscope}%
\end{pgfscope}%
\begin{pgfscope}%
\definecolor{textcolor}{rgb}{0.150000,0.150000,0.150000}%
\pgfsetstrokecolor{textcolor}%
\pgfsetfillcolor{textcolor}%
\pgftext[x=0.375000in,y=0.172222in,,top]{\color{textcolor}\sffamily\fontsize{8.000000}{9.600000}\selectfont 0}%
\end{pgfscope}%
\begin{pgfscope}%
\pgfpathrectangle{\pgfqpoint{0.375000in}{0.250000in}}{\pgfqpoint{2.325000in}{1.550000in}} %
\pgfusepath{clip}%
\pgfsetroundcap%
\pgfsetroundjoin%
\pgfsetlinewidth{0.803000pt}%
\definecolor{currentstroke}{rgb}{1.000000,1.000000,1.000000}%
\pgfsetstrokecolor{currentstroke}%
\pgfsetdash{}{0pt}%
\pgfpathmoveto{\pgfqpoint{0.707143in}{0.250000in}}%
\pgfpathlineto{\pgfqpoint{0.707143in}{1.800000in}}%
\pgfusepath{stroke}%
\end{pgfscope}%
\begin{pgfscope}%
\pgfsetbuttcap%
\pgfsetroundjoin%
\definecolor{currentfill}{rgb}{0.150000,0.150000,0.150000}%
\pgfsetfillcolor{currentfill}%
\pgfsetlinewidth{0.803000pt}%
\definecolor{currentstroke}{rgb}{0.150000,0.150000,0.150000}%
\pgfsetstrokecolor{currentstroke}%
\pgfsetdash{}{0pt}%
\pgfsys@defobject{currentmarker}{\pgfqpoint{0.000000in}{0.000000in}}{\pgfqpoint{0.000000in}{0.000000in}}{%
\pgfpathmoveto{\pgfqpoint{0.000000in}{0.000000in}}%
\pgfpathlineto{\pgfqpoint{0.000000in}{0.000000in}}%
\pgfusepath{stroke,fill}%
}%
\begin{pgfscope}%
\pgfsys@transformshift{0.707143in}{0.250000in}%
\pgfsys@useobject{currentmarker}{}%
\end{pgfscope}%
\end{pgfscope}%
\begin{pgfscope}%
\pgfsetbuttcap%
\pgfsetroundjoin%
\definecolor{currentfill}{rgb}{0.150000,0.150000,0.150000}%
\pgfsetfillcolor{currentfill}%
\pgfsetlinewidth{0.803000pt}%
\definecolor{currentstroke}{rgb}{0.150000,0.150000,0.150000}%
\pgfsetstrokecolor{currentstroke}%
\pgfsetdash{}{0pt}%
\pgfsys@defobject{currentmarker}{\pgfqpoint{0.000000in}{0.000000in}}{\pgfqpoint{0.000000in}{0.000000in}}{%
\pgfpathmoveto{\pgfqpoint{0.000000in}{0.000000in}}%
\pgfpathlineto{\pgfqpoint{0.000000in}{0.000000in}}%
\pgfusepath{stroke,fill}%
}%
\begin{pgfscope}%
\pgfsys@transformshift{0.707143in}{1.800000in}%
\pgfsys@useobject{currentmarker}{}%
\end{pgfscope}%
\end{pgfscope}%
\begin{pgfscope}%
\definecolor{textcolor}{rgb}{0.150000,0.150000,0.150000}%
\pgfsetstrokecolor{textcolor}%
\pgfsetfillcolor{textcolor}%
\pgftext[x=0.707143in,y=0.172222in,,top]{\color{textcolor}\sffamily\fontsize{8.000000}{9.600000}\selectfont 1}%
\end{pgfscope}%
\begin{pgfscope}%
\pgfpathrectangle{\pgfqpoint{0.375000in}{0.250000in}}{\pgfqpoint{2.325000in}{1.550000in}} %
\pgfusepath{clip}%
\pgfsetroundcap%
\pgfsetroundjoin%
\pgfsetlinewidth{0.803000pt}%
\definecolor{currentstroke}{rgb}{1.000000,1.000000,1.000000}%
\pgfsetstrokecolor{currentstroke}%
\pgfsetdash{}{0pt}%
\pgfpathmoveto{\pgfqpoint{1.039286in}{0.250000in}}%
\pgfpathlineto{\pgfqpoint{1.039286in}{1.800000in}}%
\pgfusepath{stroke}%
\end{pgfscope}%
\begin{pgfscope}%
\pgfsetbuttcap%
\pgfsetroundjoin%
\definecolor{currentfill}{rgb}{0.150000,0.150000,0.150000}%
\pgfsetfillcolor{currentfill}%
\pgfsetlinewidth{0.803000pt}%
\definecolor{currentstroke}{rgb}{0.150000,0.150000,0.150000}%
\pgfsetstrokecolor{currentstroke}%
\pgfsetdash{}{0pt}%
\pgfsys@defobject{currentmarker}{\pgfqpoint{0.000000in}{0.000000in}}{\pgfqpoint{0.000000in}{0.000000in}}{%
\pgfpathmoveto{\pgfqpoint{0.000000in}{0.000000in}}%
\pgfpathlineto{\pgfqpoint{0.000000in}{0.000000in}}%
\pgfusepath{stroke,fill}%
}%
\begin{pgfscope}%
\pgfsys@transformshift{1.039286in}{0.250000in}%
\pgfsys@useobject{currentmarker}{}%
\end{pgfscope}%
\end{pgfscope}%
\begin{pgfscope}%
\pgfsetbuttcap%
\pgfsetroundjoin%
\definecolor{currentfill}{rgb}{0.150000,0.150000,0.150000}%
\pgfsetfillcolor{currentfill}%
\pgfsetlinewidth{0.803000pt}%
\definecolor{currentstroke}{rgb}{0.150000,0.150000,0.150000}%
\pgfsetstrokecolor{currentstroke}%
\pgfsetdash{}{0pt}%
\pgfsys@defobject{currentmarker}{\pgfqpoint{0.000000in}{0.000000in}}{\pgfqpoint{0.000000in}{0.000000in}}{%
\pgfpathmoveto{\pgfqpoint{0.000000in}{0.000000in}}%
\pgfpathlineto{\pgfqpoint{0.000000in}{0.000000in}}%
\pgfusepath{stroke,fill}%
}%
\begin{pgfscope}%
\pgfsys@transformshift{1.039286in}{1.800000in}%
\pgfsys@useobject{currentmarker}{}%
\end{pgfscope}%
\end{pgfscope}%
\begin{pgfscope}%
\definecolor{textcolor}{rgb}{0.150000,0.150000,0.150000}%
\pgfsetstrokecolor{textcolor}%
\pgfsetfillcolor{textcolor}%
\pgftext[x=1.039286in,y=0.172222in,,top]{\color{textcolor}\sffamily\fontsize{8.000000}{9.600000}\selectfont 2}%
\end{pgfscope}%
\begin{pgfscope}%
\pgfpathrectangle{\pgfqpoint{0.375000in}{0.250000in}}{\pgfqpoint{2.325000in}{1.550000in}} %
\pgfusepath{clip}%
\pgfsetroundcap%
\pgfsetroundjoin%
\pgfsetlinewidth{0.803000pt}%
\definecolor{currentstroke}{rgb}{1.000000,1.000000,1.000000}%
\pgfsetstrokecolor{currentstroke}%
\pgfsetdash{}{0pt}%
\pgfpathmoveto{\pgfqpoint{1.371429in}{0.250000in}}%
\pgfpathlineto{\pgfqpoint{1.371429in}{1.800000in}}%
\pgfusepath{stroke}%
\end{pgfscope}%
\begin{pgfscope}%
\pgfsetbuttcap%
\pgfsetroundjoin%
\definecolor{currentfill}{rgb}{0.150000,0.150000,0.150000}%
\pgfsetfillcolor{currentfill}%
\pgfsetlinewidth{0.803000pt}%
\definecolor{currentstroke}{rgb}{0.150000,0.150000,0.150000}%
\pgfsetstrokecolor{currentstroke}%
\pgfsetdash{}{0pt}%
\pgfsys@defobject{currentmarker}{\pgfqpoint{0.000000in}{0.000000in}}{\pgfqpoint{0.000000in}{0.000000in}}{%
\pgfpathmoveto{\pgfqpoint{0.000000in}{0.000000in}}%
\pgfpathlineto{\pgfqpoint{0.000000in}{0.000000in}}%
\pgfusepath{stroke,fill}%
}%
\begin{pgfscope}%
\pgfsys@transformshift{1.371429in}{0.250000in}%
\pgfsys@useobject{currentmarker}{}%
\end{pgfscope}%
\end{pgfscope}%
\begin{pgfscope}%
\pgfsetbuttcap%
\pgfsetroundjoin%
\definecolor{currentfill}{rgb}{0.150000,0.150000,0.150000}%
\pgfsetfillcolor{currentfill}%
\pgfsetlinewidth{0.803000pt}%
\definecolor{currentstroke}{rgb}{0.150000,0.150000,0.150000}%
\pgfsetstrokecolor{currentstroke}%
\pgfsetdash{}{0pt}%
\pgfsys@defobject{currentmarker}{\pgfqpoint{0.000000in}{0.000000in}}{\pgfqpoint{0.000000in}{0.000000in}}{%
\pgfpathmoveto{\pgfqpoint{0.000000in}{0.000000in}}%
\pgfpathlineto{\pgfqpoint{0.000000in}{0.000000in}}%
\pgfusepath{stroke,fill}%
}%
\begin{pgfscope}%
\pgfsys@transformshift{1.371429in}{1.800000in}%
\pgfsys@useobject{currentmarker}{}%
\end{pgfscope}%
\end{pgfscope}%
\begin{pgfscope}%
\definecolor{textcolor}{rgb}{0.150000,0.150000,0.150000}%
\pgfsetstrokecolor{textcolor}%
\pgfsetfillcolor{textcolor}%
\pgftext[x=1.371429in,y=0.172222in,,top]{\color{textcolor}\sffamily\fontsize{8.000000}{9.600000}\selectfont 3}%
\end{pgfscope}%
\begin{pgfscope}%
\pgfpathrectangle{\pgfqpoint{0.375000in}{0.250000in}}{\pgfqpoint{2.325000in}{1.550000in}} %
\pgfusepath{clip}%
\pgfsetroundcap%
\pgfsetroundjoin%
\pgfsetlinewidth{0.803000pt}%
\definecolor{currentstroke}{rgb}{1.000000,1.000000,1.000000}%
\pgfsetstrokecolor{currentstroke}%
\pgfsetdash{}{0pt}%
\pgfpathmoveto{\pgfqpoint{1.703571in}{0.250000in}}%
\pgfpathlineto{\pgfqpoint{1.703571in}{1.800000in}}%
\pgfusepath{stroke}%
\end{pgfscope}%
\begin{pgfscope}%
\pgfsetbuttcap%
\pgfsetroundjoin%
\definecolor{currentfill}{rgb}{0.150000,0.150000,0.150000}%
\pgfsetfillcolor{currentfill}%
\pgfsetlinewidth{0.803000pt}%
\definecolor{currentstroke}{rgb}{0.150000,0.150000,0.150000}%
\pgfsetstrokecolor{currentstroke}%
\pgfsetdash{}{0pt}%
\pgfsys@defobject{currentmarker}{\pgfqpoint{0.000000in}{0.000000in}}{\pgfqpoint{0.000000in}{0.000000in}}{%
\pgfpathmoveto{\pgfqpoint{0.000000in}{0.000000in}}%
\pgfpathlineto{\pgfqpoint{0.000000in}{0.000000in}}%
\pgfusepath{stroke,fill}%
}%
\begin{pgfscope}%
\pgfsys@transformshift{1.703571in}{0.250000in}%
\pgfsys@useobject{currentmarker}{}%
\end{pgfscope}%
\end{pgfscope}%
\begin{pgfscope}%
\pgfsetbuttcap%
\pgfsetroundjoin%
\definecolor{currentfill}{rgb}{0.150000,0.150000,0.150000}%
\pgfsetfillcolor{currentfill}%
\pgfsetlinewidth{0.803000pt}%
\definecolor{currentstroke}{rgb}{0.150000,0.150000,0.150000}%
\pgfsetstrokecolor{currentstroke}%
\pgfsetdash{}{0pt}%
\pgfsys@defobject{currentmarker}{\pgfqpoint{0.000000in}{0.000000in}}{\pgfqpoint{0.000000in}{0.000000in}}{%
\pgfpathmoveto{\pgfqpoint{0.000000in}{0.000000in}}%
\pgfpathlineto{\pgfqpoint{0.000000in}{0.000000in}}%
\pgfusepath{stroke,fill}%
}%
\begin{pgfscope}%
\pgfsys@transformshift{1.703571in}{1.800000in}%
\pgfsys@useobject{currentmarker}{}%
\end{pgfscope}%
\end{pgfscope}%
\begin{pgfscope}%
\definecolor{textcolor}{rgb}{0.150000,0.150000,0.150000}%
\pgfsetstrokecolor{textcolor}%
\pgfsetfillcolor{textcolor}%
\pgftext[x=1.703571in,y=0.172222in,,top]{\color{textcolor}\sffamily\fontsize{8.000000}{9.600000}\selectfont 4}%
\end{pgfscope}%
\begin{pgfscope}%
\pgfpathrectangle{\pgfqpoint{0.375000in}{0.250000in}}{\pgfqpoint{2.325000in}{1.550000in}} %
\pgfusepath{clip}%
\pgfsetroundcap%
\pgfsetroundjoin%
\pgfsetlinewidth{0.803000pt}%
\definecolor{currentstroke}{rgb}{1.000000,1.000000,1.000000}%
\pgfsetstrokecolor{currentstroke}%
\pgfsetdash{}{0pt}%
\pgfpathmoveto{\pgfqpoint{2.035714in}{0.250000in}}%
\pgfpathlineto{\pgfqpoint{2.035714in}{1.800000in}}%
\pgfusepath{stroke}%
\end{pgfscope}%
\begin{pgfscope}%
\pgfsetbuttcap%
\pgfsetroundjoin%
\definecolor{currentfill}{rgb}{0.150000,0.150000,0.150000}%
\pgfsetfillcolor{currentfill}%
\pgfsetlinewidth{0.803000pt}%
\definecolor{currentstroke}{rgb}{0.150000,0.150000,0.150000}%
\pgfsetstrokecolor{currentstroke}%
\pgfsetdash{}{0pt}%
\pgfsys@defobject{currentmarker}{\pgfqpoint{0.000000in}{0.000000in}}{\pgfqpoint{0.000000in}{0.000000in}}{%
\pgfpathmoveto{\pgfqpoint{0.000000in}{0.000000in}}%
\pgfpathlineto{\pgfqpoint{0.000000in}{0.000000in}}%
\pgfusepath{stroke,fill}%
}%
\begin{pgfscope}%
\pgfsys@transformshift{2.035714in}{0.250000in}%
\pgfsys@useobject{currentmarker}{}%
\end{pgfscope}%
\end{pgfscope}%
\begin{pgfscope}%
\pgfsetbuttcap%
\pgfsetroundjoin%
\definecolor{currentfill}{rgb}{0.150000,0.150000,0.150000}%
\pgfsetfillcolor{currentfill}%
\pgfsetlinewidth{0.803000pt}%
\definecolor{currentstroke}{rgb}{0.150000,0.150000,0.150000}%
\pgfsetstrokecolor{currentstroke}%
\pgfsetdash{}{0pt}%
\pgfsys@defobject{currentmarker}{\pgfqpoint{0.000000in}{0.000000in}}{\pgfqpoint{0.000000in}{0.000000in}}{%
\pgfpathmoveto{\pgfqpoint{0.000000in}{0.000000in}}%
\pgfpathlineto{\pgfqpoint{0.000000in}{0.000000in}}%
\pgfusepath{stroke,fill}%
}%
\begin{pgfscope}%
\pgfsys@transformshift{2.035714in}{1.800000in}%
\pgfsys@useobject{currentmarker}{}%
\end{pgfscope}%
\end{pgfscope}%
\begin{pgfscope}%
\definecolor{textcolor}{rgb}{0.150000,0.150000,0.150000}%
\pgfsetstrokecolor{textcolor}%
\pgfsetfillcolor{textcolor}%
\pgftext[x=2.035714in,y=0.172222in,,top]{\color{textcolor}\sffamily\fontsize{8.000000}{9.600000}\selectfont 5}%
\end{pgfscope}%
\begin{pgfscope}%
\pgfpathrectangle{\pgfqpoint{0.375000in}{0.250000in}}{\pgfqpoint{2.325000in}{1.550000in}} %
\pgfusepath{clip}%
\pgfsetroundcap%
\pgfsetroundjoin%
\pgfsetlinewidth{0.803000pt}%
\definecolor{currentstroke}{rgb}{1.000000,1.000000,1.000000}%
\pgfsetstrokecolor{currentstroke}%
\pgfsetdash{}{0pt}%
\pgfpathmoveto{\pgfqpoint{2.367857in}{0.250000in}}%
\pgfpathlineto{\pgfqpoint{2.367857in}{1.800000in}}%
\pgfusepath{stroke}%
\end{pgfscope}%
\begin{pgfscope}%
\pgfsetbuttcap%
\pgfsetroundjoin%
\definecolor{currentfill}{rgb}{0.150000,0.150000,0.150000}%
\pgfsetfillcolor{currentfill}%
\pgfsetlinewidth{0.803000pt}%
\definecolor{currentstroke}{rgb}{0.150000,0.150000,0.150000}%
\pgfsetstrokecolor{currentstroke}%
\pgfsetdash{}{0pt}%
\pgfsys@defobject{currentmarker}{\pgfqpoint{0.000000in}{0.000000in}}{\pgfqpoint{0.000000in}{0.000000in}}{%
\pgfpathmoveto{\pgfqpoint{0.000000in}{0.000000in}}%
\pgfpathlineto{\pgfqpoint{0.000000in}{0.000000in}}%
\pgfusepath{stroke,fill}%
}%
\begin{pgfscope}%
\pgfsys@transformshift{2.367857in}{0.250000in}%
\pgfsys@useobject{currentmarker}{}%
\end{pgfscope}%
\end{pgfscope}%
\begin{pgfscope}%
\pgfsetbuttcap%
\pgfsetroundjoin%
\definecolor{currentfill}{rgb}{0.150000,0.150000,0.150000}%
\pgfsetfillcolor{currentfill}%
\pgfsetlinewidth{0.803000pt}%
\definecolor{currentstroke}{rgb}{0.150000,0.150000,0.150000}%
\pgfsetstrokecolor{currentstroke}%
\pgfsetdash{}{0pt}%
\pgfsys@defobject{currentmarker}{\pgfqpoint{0.000000in}{0.000000in}}{\pgfqpoint{0.000000in}{0.000000in}}{%
\pgfpathmoveto{\pgfqpoint{0.000000in}{0.000000in}}%
\pgfpathlineto{\pgfqpoint{0.000000in}{0.000000in}}%
\pgfusepath{stroke,fill}%
}%
\begin{pgfscope}%
\pgfsys@transformshift{2.367857in}{1.800000in}%
\pgfsys@useobject{currentmarker}{}%
\end{pgfscope}%
\end{pgfscope}%
\begin{pgfscope}%
\definecolor{textcolor}{rgb}{0.150000,0.150000,0.150000}%
\pgfsetstrokecolor{textcolor}%
\pgfsetfillcolor{textcolor}%
\pgftext[x=2.367857in,y=0.172222in,,top]{\color{textcolor}\sffamily\fontsize{8.000000}{9.600000}\selectfont 6}%
\end{pgfscope}%
\begin{pgfscope}%
\pgfpathrectangle{\pgfqpoint{0.375000in}{0.250000in}}{\pgfqpoint{2.325000in}{1.550000in}} %
\pgfusepath{clip}%
\pgfsetroundcap%
\pgfsetroundjoin%
\pgfsetlinewidth{0.803000pt}%
\definecolor{currentstroke}{rgb}{1.000000,1.000000,1.000000}%
\pgfsetstrokecolor{currentstroke}%
\pgfsetdash{}{0pt}%
\pgfpathmoveto{\pgfqpoint{2.700000in}{0.250000in}}%
\pgfpathlineto{\pgfqpoint{2.700000in}{1.800000in}}%
\pgfusepath{stroke}%
\end{pgfscope}%
\begin{pgfscope}%
\pgfsetbuttcap%
\pgfsetroundjoin%
\definecolor{currentfill}{rgb}{0.150000,0.150000,0.150000}%
\pgfsetfillcolor{currentfill}%
\pgfsetlinewidth{0.803000pt}%
\definecolor{currentstroke}{rgb}{0.150000,0.150000,0.150000}%
\pgfsetstrokecolor{currentstroke}%
\pgfsetdash{}{0pt}%
\pgfsys@defobject{currentmarker}{\pgfqpoint{0.000000in}{0.000000in}}{\pgfqpoint{0.000000in}{0.000000in}}{%
\pgfpathmoveto{\pgfqpoint{0.000000in}{0.000000in}}%
\pgfpathlineto{\pgfqpoint{0.000000in}{0.000000in}}%
\pgfusepath{stroke,fill}%
}%
\begin{pgfscope}%
\pgfsys@transformshift{2.700000in}{0.250000in}%
\pgfsys@useobject{currentmarker}{}%
\end{pgfscope}%
\end{pgfscope}%
\begin{pgfscope}%
\pgfsetbuttcap%
\pgfsetroundjoin%
\definecolor{currentfill}{rgb}{0.150000,0.150000,0.150000}%
\pgfsetfillcolor{currentfill}%
\pgfsetlinewidth{0.803000pt}%
\definecolor{currentstroke}{rgb}{0.150000,0.150000,0.150000}%
\pgfsetstrokecolor{currentstroke}%
\pgfsetdash{}{0pt}%
\pgfsys@defobject{currentmarker}{\pgfqpoint{0.000000in}{0.000000in}}{\pgfqpoint{0.000000in}{0.000000in}}{%
\pgfpathmoveto{\pgfqpoint{0.000000in}{0.000000in}}%
\pgfpathlineto{\pgfqpoint{0.000000in}{0.000000in}}%
\pgfusepath{stroke,fill}%
}%
\begin{pgfscope}%
\pgfsys@transformshift{2.700000in}{1.800000in}%
\pgfsys@useobject{currentmarker}{}%
\end{pgfscope}%
\end{pgfscope}%
\begin{pgfscope}%
\definecolor{textcolor}{rgb}{0.150000,0.150000,0.150000}%
\pgfsetstrokecolor{textcolor}%
\pgfsetfillcolor{textcolor}%
\pgftext[x=2.700000in,y=0.172222in,,top]{\color{textcolor}\sffamily\fontsize{8.000000}{9.600000}\selectfont 7}%
\end{pgfscope}%
\begin{pgfscope}%
\pgfpathrectangle{\pgfqpoint{0.375000in}{0.250000in}}{\pgfqpoint{2.325000in}{1.550000in}} %
\pgfusepath{clip}%
\pgfsetroundcap%
\pgfsetroundjoin%
\pgfsetlinewidth{0.803000pt}%
\definecolor{currentstroke}{rgb}{1.000000,1.000000,1.000000}%
\pgfsetstrokecolor{currentstroke}%
\pgfsetdash{}{0pt}%
\pgfpathmoveto{\pgfqpoint{0.375000in}{0.250000in}}%
\pgfpathlineto{\pgfqpoint{2.700000in}{0.250000in}}%
\pgfusepath{stroke}%
\end{pgfscope}%
\begin{pgfscope}%
\pgfsetbuttcap%
\pgfsetroundjoin%
\definecolor{currentfill}{rgb}{0.150000,0.150000,0.150000}%
\pgfsetfillcolor{currentfill}%
\pgfsetlinewidth{0.803000pt}%
\definecolor{currentstroke}{rgb}{0.150000,0.150000,0.150000}%
\pgfsetstrokecolor{currentstroke}%
\pgfsetdash{}{0pt}%
\pgfsys@defobject{currentmarker}{\pgfqpoint{0.000000in}{0.000000in}}{\pgfqpoint{0.000000in}{0.000000in}}{%
\pgfpathmoveto{\pgfqpoint{0.000000in}{0.000000in}}%
\pgfpathlineto{\pgfqpoint{0.000000in}{0.000000in}}%
\pgfusepath{stroke,fill}%
}%
\begin{pgfscope}%
\pgfsys@transformshift{0.375000in}{0.250000in}%
\pgfsys@useobject{currentmarker}{}%
\end{pgfscope}%
\end{pgfscope}%
\begin{pgfscope}%
\pgfsetbuttcap%
\pgfsetroundjoin%
\definecolor{currentfill}{rgb}{0.150000,0.150000,0.150000}%
\pgfsetfillcolor{currentfill}%
\pgfsetlinewidth{0.803000pt}%
\definecolor{currentstroke}{rgb}{0.150000,0.150000,0.150000}%
\pgfsetstrokecolor{currentstroke}%
\pgfsetdash{}{0pt}%
\pgfsys@defobject{currentmarker}{\pgfqpoint{0.000000in}{0.000000in}}{\pgfqpoint{0.000000in}{0.000000in}}{%
\pgfpathmoveto{\pgfqpoint{0.000000in}{0.000000in}}%
\pgfpathlineto{\pgfqpoint{0.000000in}{0.000000in}}%
\pgfusepath{stroke,fill}%
}%
\begin{pgfscope}%
\pgfsys@transformshift{2.700000in}{0.250000in}%
\pgfsys@useobject{currentmarker}{}%
\end{pgfscope}%
\end{pgfscope}%
\begin{pgfscope}%
\definecolor{textcolor}{rgb}{0.150000,0.150000,0.150000}%
\pgfsetstrokecolor{textcolor}%
\pgfsetfillcolor{textcolor}%
\pgftext[x=0.297222in,y=0.250000in,right,]{\color{textcolor}\sffamily\fontsize{8.000000}{9.600000}\selectfont −2}%
\end{pgfscope}%
\begin{pgfscope}%
\pgfpathrectangle{\pgfqpoint{0.375000in}{0.250000in}}{\pgfqpoint{2.325000in}{1.550000in}} %
\pgfusepath{clip}%
\pgfsetroundcap%
\pgfsetroundjoin%
\pgfsetlinewidth{0.803000pt}%
\definecolor{currentstroke}{rgb}{1.000000,1.000000,1.000000}%
\pgfsetstrokecolor{currentstroke}%
\pgfsetdash{}{0pt}%
\pgfpathmoveto{\pgfqpoint{0.375000in}{0.443750in}}%
\pgfpathlineto{\pgfqpoint{2.700000in}{0.443750in}}%
\pgfusepath{stroke}%
\end{pgfscope}%
\begin{pgfscope}%
\pgfsetbuttcap%
\pgfsetroundjoin%
\definecolor{currentfill}{rgb}{0.150000,0.150000,0.150000}%
\pgfsetfillcolor{currentfill}%
\pgfsetlinewidth{0.803000pt}%
\definecolor{currentstroke}{rgb}{0.150000,0.150000,0.150000}%
\pgfsetstrokecolor{currentstroke}%
\pgfsetdash{}{0pt}%
\pgfsys@defobject{currentmarker}{\pgfqpoint{0.000000in}{0.000000in}}{\pgfqpoint{0.000000in}{0.000000in}}{%
\pgfpathmoveto{\pgfqpoint{0.000000in}{0.000000in}}%
\pgfpathlineto{\pgfqpoint{0.000000in}{0.000000in}}%
\pgfusepath{stroke,fill}%
}%
\begin{pgfscope}%
\pgfsys@transformshift{0.375000in}{0.443750in}%
\pgfsys@useobject{currentmarker}{}%
\end{pgfscope}%
\end{pgfscope}%
\begin{pgfscope}%
\pgfsetbuttcap%
\pgfsetroundjoin%
\definecolor{currentfill}{rgb}{0.150000,0.150000,0.150000}%
\pgfsetfillcolor{currentfill}%
\pgfsetlinewidth{0.803000pt}%
\definecolor{currentstroke}{rgb}{0.150000,0.150000,0.150000}%
\pgfsetstrokecolor{currentstroke}%
\pgfsetdash{}{0pt}%
\pgfsys@defobject{currentmarker}{\pgfqpoint{0.000000in}{0.000000in}}{\pgfqpoint{0.000000in}{0.000000in}}{%
\pgfpathmoveto{\pgfqpoint{0.000000in}{0.000000in}}%
\pgfpathlineto{\pgfqpoint{0.000000in}{0.000000in}}%
\pgfusepath{stroke,fill}%
}%
\begin{pgfscope}%
\pgfsys@transformshift{2.700000in}{0.443750in}%
\pgfsys@useobject{currentmarker}{}%
\end{pgfscope}%
\end{pgfscope}%
\begin{pgfscope}%
\definecolor{textcolor}{rgb}{0.150000,0.150000,0.150000}%
\pgfsetstrokecolor{textcolor}%
\pgfsetfillcolor{textcolor}%
\pgftext[x=0.297222in,y=0.443750in,right,]{\color{textcolor}\sffamily\fontsize{8.000000}{9.600000}\selectfont 0}%
\end{pgfscope}%
\begin{pgfscope}%
\pgfpathrectangle{\pgfqpoint{0.375000in}{0.250000in}}{\pgfqpoint{2.325000in}{1.550000in}} %
\pgfusepath{clip}%
\pgfsetroundcap%
\pgfsetroundjoin%
\pgfsetlinewidth{0.803000pt}%
\definecolor{currentstroke}{rgb}{1.000000,1.000000,1.000000}%
\pgfsetstrokecolor{currentstroke}%
\pgfsetdash{}{0pt}%
\pgfpathmoveto{\pgfqpoint{0.375000in}{0.637500in}}%
\pgfpathlineto{\pgfqpoint{2.700000in}{0.637500in}}%
\pgfusepath{stroke}%
\end{pgfscope}%
\begin{pgfscope}%
\pgfsetbuttcap%
\pgfsetroundjoin%
\definecolor{currentfill}{rgb}{0.150000,0.150000,0.150000}%
\pgfsetfillcolor{currentfill}%
\pgfsetlinewidth{0.803000pt}%
\definecolor{currentstroke}{rgb}{0.150000,0.150000,0.150000}%
\pgfsetstrokecolor{currentstroke}%
\pgfsetdash{}{0pt}%
\pgfsys@defobject{currentmarker}{\pgfqpoint{0.000000in}{0.000000in}}{\pgfqpoint{0.000000in}{0.000000in}}{%
\pgfpathmoveto{\pgfqpoint{0.000000in}{0.000000in}}%
\pgfpathlineto{\pgfqpoint{0.000000in}{0.000000in}}%
\pgfusepath{stroke,fill}%
}%
\begin{pgfscope}%
\pgfsys@transformshift{0.375000in}{0.637500in}%
\pgfsys@useobject{currentmarker}{}%
\end{pgfscope}%
\end{pgfscope}%
\begin{pgfscope}%
\pgfsetbuttcap%
\pgfsetroundjoin%
\definecolor{currentfill}{rgb}{0.150000,0.150000,0.150000}%
\pgfsetfillcolor{currentfill}%
\pgfsetlinewidth{0.803000pt}%
\definecolor{currentstroke}{rgb}{0.150000,0.150000,0.150000}%
\pgfsetstrokecolor{currentstroke}%
\pgfsetdash{}{0pt}%
\pgfsys@defobject{currentmarker}{\pgfqpoint{0.000000in}{0.000000in}}{\pgfqpoint{0.000000in}{0.000000in}}{%
\pgfpathmoveto{\pgfqpoint{0.000000in}{0.000000in}}%
\pgfpathlineto{\pgfqpoint{0.000000in}{0.000000in}}%
\pgfusepath{stroke,fill}%
}%
\begin{pgfscope}%
\pgfsys@transformshift{2.700000in}{0.637500in}%
\pgfsys@useobject{currentmarker}{}%
\end{pgfscope}%
\end{pgfscope}%
\begin{pgfscope}%
\definecolor{textcolor}{rgb}{0.150000,0.150000,0.150000}%
\pgfsetstrokecolor{textcolor}%
\pgfsetfillcolor{textcolor}%
\pgftext[x=0.297222in,y=0.637500in,right,]{\color{textcolor}\sffamily\fontsize{8.000000}{9.600000}\selectfont 2}%
\end{pgfscope}%
\begin{pgfscope}%
\pgfpathrectangle{\pgfqpoint{0.375000in}{0.250000in}}{\pgfqpoint{2.325000in}{1.550000in}} %
\pgfusepath{clip}%
\pgfsetroundcap%
\pgfsetroundjoin%
\pgfsetlinewidth{0.803000pt}%
\definecolor{currentstroke}{rgb}{1.000000,1.000000,1.000000}%
\pgfsetstrokecolor{currentstroke}%
\pgfsetdash{}{0pt}%
\pgfpathmoveto{\pgfqpoint{0.375000in}{0.831250in}}%
\pgfpathlineto{\pgfqpoint{2.700000in}{0.831250in}}%
\pgfusepath{stroke}%
\end{pgfscope}%
\begin{pgfscope}%
\pgfsetbuttcap%
\pgfsetroundjoin%
\definecolor{currentfill}{rgb}{0.150000,0.150000,0.150000}%
\pgfsetfillcolor{currentfill}%
\pgfsetlinewidth{0.803000pt}%
\definecolor{currentstroke}{rgb}{0.150000,0.150000,0.150000}%
\pgfsetstrokecolor{currentstroke}%
\pgfsetdash{}{0pt}%
\pgfsys@defobject{currentmarker}{\pgfqpoint{0.000000in}{0.000000in}}{\pgfqpoint{0.000000in}{0.000000in}}{%
\pgfpathmoveto{\pgfqpoint{0.000000in}{0.000000in}}%
\pgfpathlineto{\pgfqpoint{0.000000in}{0.000000in}}%
\pgfusepath{stroke,fill}%
}%
\begin{pgfscope}%
\pgfsys@transformshift{0.375000in}{0.831250in}%
\pgfsys@useobject{currentmarker}{}%
\end{pgfscope}%
\end{pgfscope}%
\begin{pgfscope}%
\pgfsetbuttcap%
\pgfsetroundjoin%
\definecolor{currentfill}{rgb}{0.150000,0.150000,0.150000}%
\pgfsetfillcolor{currentfill}%
\pgfsetlinewidth{0.803000pt}%
\definecolor{currentstroke}{rgb}{0.150000,0.150000,0.150000}%
\pgfsetstrokecolor{currentstroke}%
\pgfsetdash{}{0pt}%
\pgfsys@defobject{currentmarker}{\pgfqpoint{0.000000in}{0.000000in}}{\pgfqpoint{0.000000in}{0.000000in}}{%
\pgfpathmoveto{\pgfqpoint{0.000000in}{0.000000in}}%
\pgfpathlineto{\pgfqpoint{0.000000in}{0.000000in}}%
\pgfusepath{stroke,fill}%
}%
\begin{pgfscope}%
\pgfsys@transformshift{2.700000in}{0.831250in}%
\pgfsys@useobject{currentmarker}{}%
\end{pgfscope}%
\end{pgfscope}%
\begin{pgfscope}%
\definecolor{textcolor}{rgb}{0.150000,0.150000,0.150000}%
\pgfsetstrokecolor{textcolor}%
\pgfsetfillcolor{textcolor}%
\pgftext[x=0.297222in,y=0.831250in,right,]{\color{textcolor}\sffamily\fontsize{8.000000}{9.600000}\selectfont 4}%
\end{pgfscope}%
\begin{pgfscope}%
\pgfpathrectangle{\pgfqpoint{0.375000in}{0.250000in}}{\pgfqpoint{2.325000in}{1.550000in}} %
\pgfusepath{clip}%
\pgfsetroundcap%
\pgfsetroundjoin%
\pgfsetlinewidth{0.803000pt}%
\definecolor{currentstroke}{rgb}{1.000000,1.000000,1.000000}%
\pgfsetstrokecolor{currentstroke}%
\pgfsetdash{}{0pt}%
\pgfpathmoveto{\pgfqpoint{0.375000in}{1.025000in}}%
\pgfpathlineto{\pgfqpoint{2.700000in}{1.025000in}}%
\pgfusepath{stroke}%
\end{pgfscope}%
\begin{pgfscope}%
\pgfsetbuttcap%
\pgfsetroundjoin%
\definecolor{currentfill}{rgb}{0.150000,0.150000,0.150000}%
\pgfsetfillcolor{currentfill}%
\pgfsetlinewidth{0.803000pt}%
\definecolor{currentstroke}{rgb}{0.150000,0.150000,0.150000}%
\pgfsetstrokecolor{currentstroke}%
\pgfsetdash{}{0pt}%
\pgfsys@defobject{currentmarker}{\pgfqpoint{0.000000in}{0.000000in}}{\pgfqpoint{0.000000in}{0.000000in}}{%
\pgfpathmoveto{\pgfqpoint{0.000000in}{0.000000in}}%
\pgfpathlineto{\pgfqpoint{0.000000in}{0.000000in}}%
\pgfusepath{stroke,fill}%
}%
\begin{pgfscope}%
\pgfsys@transformshift{0.375000in}{1.025000in}%
\pgfsys@useobject{currentmarker}{}%
\end{pgfscope}%
\end{pgfscope}%
\begin{pgfscope}%
\pgfsetbuttcap%
\pgfsetroundjoin%
\definecolor{currentfill}{rgb}{0.150000,0.150000,0.150000}%
\pgfsetfillcolor{currentfill}%
\pgfsetlinewidth{0.803000pt}%
\definecolor{currentstroke}{rgb}{0.150000,0.150000,0.150000}%
\pgfsetstrokecolor{currentstroke}%
\pgfsetdash{}{0pt}%
\pgfsys@defobject{currentmarker}{\pgfqpoint{0.000000in}{0.000000in}}{\pgfqpoint{0.000000in}{0.000000in}}{%
\pgfpathmoveto{\pgfqpoint{0.000000in}{0.000000in}}%
\pgfpathlineto{\pgfqpoint{0.000000in}{0.000000in}}%
\pgfusepath{stroke,fill}%
}%
\begin{pgfscope}%
\pgfsys@transformshift{2.700000in}{1.025000in}%
\pgfsys@useobject{currentmarker}{}%
\end{pgfscope}%
\end{pgfscope}%
\begin{pgfscope}%
\definecolor{textcolor}{rgb}{0.150000,0.150000,0.150000}%
\pgfsetstrokecolor{textcolor}%
\pgfsetfillcolor{textcolor}%
\pgftext[x=0.297222in,y=1.025000in,right,]{\color{textcolor}\sffamily\fontsize{8.000000}{9.600000}\selectfont 6}%
\end{pgfscope}%
\begin{pgfscope}%
\pgfpathrectangle{\pgfqpoint{0.375000in}{0.250000in}}{\pgfqpoint{2.325000in}{1.550000in}} %
\pgfusepath{clip}%
\pgfsetroundcap%
\pgfsetroundjoin%
\pgfsetlinewidth{0.803000pt}%
\definecolor{currentstroke}{rgb}{1.000000,1.000000,1.000000}%
\pgfsetstrokecolor{currentstroke}%
\pgfsetdash{}{0pt}%
\pgfpathmoveto{\pgfqpoint{0.375000in}{1.218750in}}%
\pgfpathlineto{\pgfqpoint{2.700000in}{1.218750in}}%
\pgfusepath{stroke}%
\end{pgfscope}%
\begin{pgfscope}%
\pgfsetbuttcap%
\pgfsetroundjoin%
\definecolor{currentfill}{rgb}{0.150000,0.150000,0.150000}%
\pgfsetfillcolor{currentfill}%
\pgfsetlinewidth{0.803000pt}%
\definecolor{currentstroke}{rgb}{0.150000,0.150000,0.150000}%
\pgfsetstrokecolor{currentstroke}%
\pgfsetdash{}{0pt}%
\pgfsys@defobject{currentmarker}{\pgfqpoint{0.000000in}{0.000000in}}{\pgfqpoint{0.000000in}{0.000000in}}{%
\pgfpathmoveto{\pgfqpoint{0.000000in}{0.000000in}}%
\pgfpathlineto{\pgfqpoint{0.000000in}{0.000000in}}%
\pgfusepath{stroke,fill}%
}%
\begin{pgfscope}%
\pgfsys@transformshift{0.375000in}{1.218750in}%
\pgfsys@useobject{currentmarker}{}%
\end{pgfscope}%
\end{pgfscope}%
\begin{pgfscope}%
\pgfsetbuttcap%
\pgfsetroundjoin%
\definecolor{currentfill}{rgb}{0.150000,0.150000,0.150000}%
\pgfsetfillcolor{currentfill}%
\pgfsetlinewidth{0.803000pt}%
\definecolor{currentstroke}{rgb}{0.150000,0.150000,0.150000}%
\pgfsetstrokecolor{currentstroke}%
\pgfsetdash{}{0pt}%
\pgfsys@defobject{currentmarker}{\pgfqpoint{0.000000in}{0.000000in}}{\pgfqpoint{0.000000in}{0.000000in}}{%
\pgfpathmoveto{\pgfqpoint{0.000000in}{0.000000in}}%
\pgfpathlineto{\pgfqpoint{0.000000in}{0.000000in}}%
\pgfusepath{stroke,fill}%
}%
\begin{pgfscope}%
\pgfsys@transformshift{2.700000in}{1.218750in}%
\pgfsys@useobject{currentmarker}{}%
\end{pgfscope}%
\end{pgfscope}%
\begin{pgfscope}%
\definecolor{textcolor}{rgb}{0.150000,0.150000,0.150000}%
\pgfsetstrokecolor{textcolor}%
\pgfsetfillcolor{textcolor}%
\pgftext[x=0.297222in,y=1.218750in,right,]{\color{textcolor}\sffamily\fontsize{8.000000}{9.600000}\selectfont 8}%
\end{pgfscope}%
\begin{pgfscope}%
\pgfpathrectangle{\pgfqpoint{0.375000in}{0.250000in}}{\pgfqpoint{2.325000in}{1.550000in}} %
\pgfusepath{clip}%
\pgfsetroundcap%
\pgfsetroundjoin%
\pgfsetlinewidth{0.803000pt}%
\definecolor{currentstroke}{rgb}{1.000000,1.000000,1.000000}%
\pgfsetstrokecolor{currentstroke}%
\pgfsetdash{}{0pt}%
\pgfpathmoveto{\pgfqpoint{0.375000in}{1.412500in}}%
\pgfpathlineto{\pgfqpoint{2.700000in}{1.412500in}}%
\pgfusepath{stroke}%
\end{pgfscope}%
\begin{pgfscope}%
\pgfsetbuttcap%
\pgfsetroundjoin%
\definecolor{currentfill}{rgb}{0.150000,0.150000,0.150000}%
\pgfsetfillcolor{currentfill}%
\pgfsetlinewidth{0.803000pt}%
\definecolor{currentstroke}{rgb}{0.150000,0.150000,0.150000}%
\pgfsetstrokecolor{currentstroke}%
\pgfsetdash{}{0pt}%
\pgfsys@defobject{currentmarker}{\pgfqpoint{0.000000in}{0.000000in}}{\pgfqpoint{0.000000in}{0.000000in}}{%
\pgfpathmoveto{\pgfqpoint{0.000000in}{0.000000in}}%
\pgfpathlineto{\pgfqpoint{0.000000in}{0.000000in}}%
\pgfusepath{stroke,fill}%
}%
\begin{pgfscope}%
\pgfsys@transformshift{0.375000in}{1.412500in}%
\pgfsys@useobject{currentmarker}{}%
\end{pgfscope}%
\end{pgfscope}%
\begin{pgfscope}%
\pgfsetbuttcap%
\pgfsetroundjoin%
\definecolor{currentfill}{rgb}{0.150000,0.150000,0.150000}%
\pgfsetfillcolor{currentfill}%
\pgfsetlinewidth{0.803000pt}%
\definecolor{currentstroke}{rgb}{0.150000,0.150000,0.150000}%
\pgfsetstrokecolor{currentstroke}%
\pgfsetdash{}{0pt}%
\pgfsys@defobject{currentmarker}{\pgfqpoint{0.000000in}{0.000000in}}{\pgfqpoint{0.000000in}{0.000000in}}{%
\pgfpathmoveto{\pgfqpoint{0.000000in}{0.000000in}}%
\pgfpathlineto{\pgfqpoint{0.000000in}{0.000000in}}%
\pgfusepath{stroke,fill}%
}%
\begin{pgfscope}%
\pgfsys@transformshift{2.700000in}{1.412500in}%
\pgfsys@useobject{currentmarker}{}%
\end{pgfscope}%
\end{pgfscope}%
\begin{pgfscope}%
\definecolor{textcolor}{rgb}{0.150000,0.150000,0.150000}%
\pgfsetstrokecolor{textcolor}%
\pgfsetfillcolor{textcolor}%
\pgftext[x=0.297222in,y=1.412500in,right,]{\color{textcolor}\sffamily\fontsize{8.000000}{9.600000}\selectfont 10}%
\end{pgfscope}%
\begin{pgfscope}%
\pgfpathrectangle{\pgfqpoint{0.375000in}{0.250000in}}{\pgfqpoint{2.325000in}{1.550000in}} %
\pgfusepath{clip}%
\pgfsetroundcap%
\pgfsetroundjoin%
\pgfsetlinewidth{0.803000pt}%
\definecolor{currentstroke}{rgb}{1.000000,1.000000,1.000000}%
\pgfsetstrokecolor{currentstroke}%
\pgfsetdash{}{0pt}%
\pgfpathmoveto{\pgfqpoint{0.375000in}{1.606250in}}%
\pgfpathlineto{\pgfqpoint{2.700000in}{1.606250in}}%
\pgfusepath{stroke}%
\end{pgfscope}%
\begin{pgfscope}%
\pgfsetbuttcap%
\pgfsetroundjoin%
\definecolor{currentfill}{rgb}{0.150000,0.150000,0.150000}%
\pgfsetfillcolor{currentfill}%
\pgfsetlinewidth{0.803000pt}%
\definecolor{currentstroke}{rgb}{0.150000,0.150000,0.150000}%
\pgfsetstrokecolor{currentstroke}%
\pgfsetdash{}{0pt}%
\pgfsys@defobject{currentmarker}{\pgfqpoint{0.000000in}{0.000000in}}{\pgfqpoint{0.000000in}{0.000000in}}{%
\pgfpathmoveto{\pgfqpoint{0.000000in}{0.000000in}}%
\pgfpathlineto{\pgfqpoint{0.000000in}{0.000000in}}%
\pgfusepath{stroke,fill}%
}%
\begin{pgfscope}%
\pgfsys@transformshift{0.375000in}{1.606250in}%
\pgfsys@useobject{currentmarker}{}%
\end{pgfscope}%
\end{pgfscope}%
\begin{pgfscope}%
\pgfsetbuttcap%
\pgfsetroundjoin%
\definecolor{currentfill}{rgb}{0.150000,0.150000,0.150000}%
\pgfsetfillcolor{currentfill}%
\pgfsetlinewidth{0.803000pt}%
\definecolor{currentstroke}{rgb}{0.150000,0.150000,0.150000}%
\pgfsetstrokecolor{currentstroke}%
\pgfsetdash{}{0pt}%
\pgfsys@defobject{currentmarker}{\pgfqpoint{0.000000in}{0.000000in}}{\pgfqpoint{0.000000in}{0.000000in}}{%
\pgfpathmoveto{\pgfqpoint{0.000000in}{0.000000in}}%
\pgfpathlineto{\pgfqpoint{0.000000in}{0.000000in}}%
\pgfusepath{stroke,fill}%
}%
\begin{pgfscope}%
\pgfsys@transformshift{2.700000in}{1.606250in}%
\pgfsys@useobject{currentmarker}{}%
\end{pgfscope}%
\end{pgfscope}%
\begin{pgfscope}%
\definecolor{textcolor}{rgb}{0.150000,0.150000,0.150000}%
\pgfsetstrokecolor{textcolor}%
\pgfsetfillcolor{textcolor}%
\pgftext[x=0.297222in,y=1.606250in,right,]{\color{textcolor}\sffamily\fontsize{8.000000}{9.600000}\selectfont 12}%
\end{pgfscope}%
\begin{pgfscope}%
\pgfpathrectangle{\pgfqpoint{0.375000in}{0.250000in}}{\pgfqpoint{2.325000in}{1.550000in}} %
\pgfusepath{clip}%
\pgfsetroundcap%
\pgfsetroundjoin%
\pgfsetlinewidth{0.803000pt}%
\definecolor{currentstroke}{rgb}{1.000000,1.000000,1.000000}%
\pgfsetstrokecolor{currentstroke}%
\pgfsetdash{}{0pt}%
\pgfpathmoveto{\pgfqpoint{0.375000in}{1.800000in}}%
\pgfpathlineto{\pgfqpoint{2.700000in}{1.800000in}}%
\pgfusepath{stroke}%
\end{pgfscope}%
\begin{pgfscope}%
\pgfsetbuttcap%
\pgfsetroundjoin%
\definecolor{currentfill}{rgb}{0.150000,0.150000,0.150000}%
\pgfsetfillcolor{currentfill}%
\pgfsetlinewidth{0.803000pt}%
\definecolor{currentstroke}{rgb}{0.150000,0.150000,0.150000}%
\pgfsetstrokecolor{currentstroke}%
\pgfsetdash{}{0pt}%
\pgfsys@defobject{currentmarker}{\pgfqpoint{0.000000in}{0.000000in}}{\pgfqpoint{0.000000in}{0.000000in}}{%
\pgfpathmoveto{\pgfqpoint{0.000000in}{0.000000in}}%
\pgfpathlineto{\pgfqpoint{0.000000in}{0.000000in}}%
\pgfusepath{stroke,fill}%
}%
\begin{pgfscope}%
\pgfsys@transformshift{0.375000in}{1.800000in}%
\pgfsys@useobject{currentmarker}{}%
\end{pgfscope}%
\end{pgfscope}%
\begin{pgfscope}%
\pgfsetbuttcap%
\pgfsetroundjoin%
\definecolor{currentfill}{rgb}{0.150000,0.150000,0.150000}%
\pgfsetfillcolor{currentfill}%
\pgfsetlinewidth{0.803000pt}%
\definecolor{currentstroke}{rgb}{0.150000,0.150000,0.150000}%
\pgfsetstrokecolor{currentstroke}%
\pgfsetdash{}{0pt}%
\pgfsys@defobject{currentmarker}{\pgfqpoint{0.000000in}{0.000000in}}{\pgfqpoint{0.000000in}{0.000000in}}{%
\pgfpathmoveto{\pgfqpoint{0.000000in}{0.000000in}}%
\pgfpathlineto{\pgfqpoint{0.000000in}{0.000000in}}%
\pgfusepath{stroke,fill}%
}%
\begin{pgfscope}%
\pgfsys@transformshift{2.700000in}{1.800000in}%
\pgfsys@useobject{currentmarker}{}%
\end{pgfscope}%
\end{pgfscope}%
\begin{pgfscope}%
\definecolor{textcolor}{rgb}{0.150000,0.150000,0.150000}%
\pgfsetstrokecolor{textcolor}%
\pgfsetfillcolor{textcolor}%
\pgftext[x=0.297222in,y=1.800000in,right,]{\color{textcolor}\sffamily\fontsize{8.000000}{9.600000}\selectfont 14}%
\end{pgfscope}%
\begin{pgfscope}%
\pgfpathrectangle{\pgfqpoint{0.375000in}{0.250000in}}{\pgfqpoint{2.325000in}{1.550000in}} %
\pgfusepath{clip}%
\pgfsetbuttcap%
\pgfsetmiterjoin%
\definecolor{currentfill}{rgb}{0.447059,0.623529,0.811765}%
\pgfsetfillcolor{currentfill}%
\pgfsetfillopacity{0.300000}%
\pgfsetlinewidth{0.240900pt}%
\definecolor{currentstroke}{rgb}{0.447059,0.623529,0.811765}%
\pgfsetstrokecolor{currentstroke}%
\pgfsetstrokeopacity{0.300000}%
\pgfsetdash{}{0pt}%
\pgfpathmoveto{\pgfqpoint{0.462984in}{0.560362in}}%
\pgfpathlineto{\pgfqpoint{0.493765in}{0.499938in}}%
\pgfpathlineto{\pgfqpoint{0.526258in}{0.449054in}}%
\pgfpathlineto{\pgfqpoint{0.704818in}{0.651292in}}%
\pgfpathlineto{\pgfqpoint{0.709273in}{0.636804in}}%
\pgfpathlineto{\pgfqpoint{0.912504in}{0.808203in}}%
\pgfpathlineto{\pgfqpoint{1.136212in}{0.932695in}}%
\pgfpathlineto{\pgfqpoint{1.147478in}{0.905681in}}%
\pgfpathlineto{\pgfqpoint{1.235284in}{0.890207in}}%
\pgfpathlineto{\pgfqpoint{1.254878in}{0.946064in}}%
\pgfpathlineto{\pgfqpoint{1.364408in}{1.048148in}}%
\pgfpathlineto{\pgfqpoint{1.452386in}{1.002184in}}%
\pgfpathlineto{\pgfqpoint{1.489712in}{1.123184in}}%
\pgfpathlineto{\pgfqpoint{1.727314in}{1.301751in}}%
\pgfpathlineto{\pgfqpoint{1.868909in}{1.226201in}}%
\pgfpathlineto{\pgfqpoint{1.890244in}{1.251988in}}%
\pgfpathlineto{\pgfqpoint{2.042485in}{1.399630in}}%
\pgfpathlineto{\pgfqpoint{2.112283in}{1.378489in}}%
\pgfpathlineto{\pgfqpoint{2.232189in}{1.598548in}}%
\pgfpathlineto{\pgfqpoint{2.450813in}{1.731446in}}%
\pgfpathlineto{\pgfqpoint{2.450813in}{1.719592in}}%
\pgfpathlineto{\pgfqpoint{2.232189in}{1.581564in}}%
\pgfpathlineto{\pgfqpoint{2.112283in}{1.376139in}}%
\pgfpathlineto{\pgfqpoint{2.042485in}{1.396272in}}%
\pgfpathlineto{\pgfqpoint{1.890244in}{1.241700in}}%
\pgfpathlineto{\pgfqpoint{1.868909in}{1.210456in}}%
\pgfpathlineto{\pgfqpoint{1.727314in}{1.284232in}}%
\pgfpathlineto{\pgfqpoint{1.489712in}{1.106047in}}%
\pgfpathlineto{\pgfqpoint{1.452386in}{0.997429in}}%
\pgfpathlineto{\pgfqpoint{1.364408in}{1.031946in}}%
\pgfpathlineto{\pgfqpoint{1.254878in}{0.936012in}}%
\pgfpathlineto{\pgfqpoint{1.235284in}{0.872952in}}%
\pgfpathlineto{\pgfqpoint{1.147478in}{0.896890in}}%
\pgfpathlineto{\pgfqpoint{1.136212in}{0.916727in}}%
\pgfpathlineto{\pgfqpoint{0.912504in}{0.791651in}}%
\pgfpathlineto{\pgfqpoint{0.709273in}{0.628970in}}%
\pgfpathlineto{\pgfqpoint{0.704818in}{0.633747in}}%
\pgfpathlineto{\pgfqpoint{0.526258in}{0.438040in}}%
\pgfpathlineto{\pgfqpoint{0.493765in}{0.482247in}}%
\pgfpathlineto{\pgfqpoint{0.462984in}{0.546248in}}%
\pgfpathclose%
\pgfusepath{stroke,fill}%
\end{pgfscope}%
\begin{pgfscope}%
\pgfpathrectangle{\pgfqpoint{0.375000in}{0.250000in}}{\pgfqpoint{2.325000in}{1.550000in}} %
\pgfusepath{clip}%
\pgfsetroundcap%
\pgfsetroundjoin%
\pgfsetlinewidth{2.007500pt}%
\definecolor{currentstroke}{rgb}{0.125490,0.290196,0.529412}%
\pgfsetstrokecolor{currentstroke}%
\pgfsetdash{}{0pt}%
\pgfpathmoveto{\pgfqpoint{0.462984in}{0.553305in}}%
\pgfpathlineto{\pgfqpoint{0.493765in}{0.491093in}}%
\pgfpathlineto{\pgfqpoint{0.526258in}{0.443547in}}%
\pgfpathlineto{\pgfqpoint{0.704818in}{0.642519in}}%
\pgfpathlineto{\pgfqpoint{0.709273in}{0.632887in}}%
\pgfpathlineto{\pgfqpoint{0.912504in}{0.799927in}}%
\pgfpathlineto{\pgfqpoint{1.136212in}{0.924711in}}%
\pgfpathlineto{\pgfqpoint{1.147478in}{0.901286in}}%
\pgfpathlineto{\pgfqpoint{1.235284in}{0.881579in}}%
\pgfpathlineto{\pgfqpoint{1.254878in}{0.941038in}}%
\pgfpathlineto{\pgfqpoint{1.364408in}{1.040047in}}%
\pgfpathlineto{\pgfqpoint{1.452386in}{0.999806in}}%
\pgfpathlineto{\pgfqpoint{1.489712in}{1.114615in}}%
\pgfpathlineto{\pgfqpoint{1.727314in}{1.292992in}}%
\pgfpathlineto{\pgfqpoint{1.868909in}{1.218329in}}%
\pgfpathlineto{\pgfqpoint{1.890244in}{1.246844in}}%
\pgfpathlineto{\pgfqpoint{2.042485in}{1.397951in}}%
\pgfpathlineto{\pgfqpoint{2.112283in}{1.377314in}}%
\pgfpathlineto{\pgfqpoint{2.232189in}{1.590056in}}%
\pgfpathlineto{\pgfqpoint{2.450813in}{1.725519in}}%
\pgfusepath{stroke}%
\end{pgfscope}%
\begin{pgfscope}%
\pgfpathrectangle{\pgfqpoint{0.375000in}{0.250000in}}{\pgfqpoint{2.325000in}{1.550000in}} %
\pgfusepath{clip}%
\pgfsetbuttcap%
\pgfsetbeveljoin%
\definecolor{currentfill}{rgb}{0.125490,0.290196,0.529412}%
\pgfsetfillcolor{currentfill}%
\pgfsetlinewidth{0.000000pt}%
\definecolor{currentstroke}{rgb}{0.000000,0.000000,0.000000}%
\pgfsetstrokecolor{currentstroke}%
\pgfsetdash{}{0pt}%
\pgfsys@defobject{currentmarker}{\pgfqpoint{-0.036986in}{-0.031462in}}{\pgfqpoint{0.036986in}{0.038889in}}{%
\pgfpathmoveto{\pgfqpoint{0.000000in}{0.038889in}}%
\pgfpathlineto{\pgfqpoint{-0.008731in}{0.012017in}}%
\pgfpathlineto{\pgfqpoint{-0.036986in}{0.012017in}}%
\pgfpathlineto{\pgfqpoint{-0.014127in}{-0.004590in}}%
\pgfpathlineto{\pgfqpoint{-0.022858in}{-0.031462in}}%
\pgfpathlineto{\pgfqpoint{-0.000000in}{-0.014854in}}%
\pgfpathlineto{\pgfqpoint{0.022858in}{-0.031462in}}%
\pgfpathlineto{\pgfqpoint{0.014127in}{-0.004590in}}%
\pgfpathlineto{\pgfqpoint{0.036986in}{0.012017in}}%
\pgfpathlineto{\pgfqpoint{0.008731in}{0.012017in}}%
\pgfpathclose%
\pgfusepath{fill}%
}%
\begin{pgfscope}%
\pgfsys@transformshift{0.462984in}{0.553305in}%
\pgfsys@useobject{currentmarker}{}%
\end{pgfscope}%
\begin{pgfscope}%
\pgfsys@transformshift{0.493765in}{0.491093in}%
\pgfsys@useobject{currentmarker}{}%
\end{pgfscope}%
\begin{pgfscope}%
\pgfsys@transformshift{0.526258in}{0.443547in}%
\pgfsys@useobject{currentmarker}{}%
\end{pgfscope}%
\begin{pgfscope}%
\pgfsys@transformshift{0.704818in}{0.642519in}%
\pgfsys@useobject{currentmarker}{}%
\end{pgfscope}%
\begin{pgfscope}%
\pgfsys@transformshift{0.709273in}{0.632887in}%
\pgfsys@useobject{currentmarker}{}%
\end{pgfscope}%
\begin{pgfscope}%
\pgfsys@transformshift{0.912504in}{0.799927in}%
\pgfsys@useobject{currentmarker}{}%
\end{pgfscope}%
\begin{pgfscope}%
\pgfsys@transformshift{1.136212in}{0.924711in}%
\pgfsys@useobject{currentmarker}{}%
\end{pgfscope}%
\begin{pgfscope}%
\pgfsys@transformshift{1.147478in}{0.901286in}%
\pgfsys@useobject{currentmarker}{}%
\end{pgfscope}%
\begin{pgfscope}%
\pgfsys@transformshift{1.235284in}{0.881579in}%
\pgfsys@useobject{currentmarker}{}%
\end{pgfscope}%
\begin{pgfscope}%
\pgfsys@transformshift{1.254878in}{0.941038in}%
\pgfsys@useobject{currentmarker}{}%
\end{pgfscope}%
\begin{pgfscope}%
\pgfsys@transformshift{1.364408in}{1.040047in}%
\pgfsys@useobject{currentmarker}{}%
\end{pgfscope}%
\begin{pgfscope}%
\pgfsys@transformshift{1.452386in}{0.999806in}%
\pgfsys@useobject{currentmarker}{}%
\end{pgfscope}%
\begin{pgfscope}%
\pgfsys@transformshift{1.489712in}{1.114615in}%
\pgfsys@useobject{currentmarker}{}%
\end{pgfscope}%
\begin{pgfscope}%
\pgfsys@transformshift{1.727314in}{1.292992in}%
\pgfsys@useobject{currentmarker}{}%
\end{pgfscope}%
\begin{pgfscope}%
\pgfsys@transformshift{1.868909in}{1.218329in}%
\pgfsys@useobject{currentmarker}{}%
\end{pgfscope}%
\begin{pgfscope}%
\pgfsys@transformshift{1.890244in}{1.246844in}%
\pgfsys@useobject{currentmarker}{}%
\end{pgfscope}%
\begin{pgfscope}%
\pgfsys@transformshift{2.042485in}{1.397951in}%
\pgfsys@useobject{currentmarker}{}%
\end{pgfscope}%
\begin{pgfscope}%
\pgfsys@transformshift{2.112283in}{1.377314in}%
\pgfsys@useobject{currentmarker}{}%
\end{pgfscope}%
\begin{pgfscope}%
\pgfsys@transformshift{2.232189in}{1.590056in}%
\pgfsys@useobject{currentmarker}{}%
\end{pgfscope}%
\begin{pgfscope}%
\pgfsys@transformshift{2.450813in}{1.725519in}%
\pgfsys@useobject{currentmarker}{}%
\end{pgfscope}%
\end{pgfscope}%
\begin{pgfscope}%
\pgfpathrectangle{\pgfqpoint{0.375000in}{0.250000in}}{\pgfqpoint{2.325000in}{1.550000in}} %
\pgfusepath{clip}%
\pgfsetbuttcap%
\pgfsetbeveljoin%
\definecolor{currentfill}{rgb}{1.000000,0.000000,0.000000}%
\pgfsetfillcolor{currentfill}%
\pgfsetlinewidth{0.000000pt}%
\definecolor{currentstroke}{rgb}{0.000000,0.000000,0.000000}%
\pgfsetstrokecolor{currentstroke}%
\pgfsetdash{}{0pt}%
\pgfsys@defobject{currentmarker}{\pgfqpoint{-0.036986in}{-0.031462in}}{\pgfqpoint{0.036986in}{0.038889in}}{%
\pgfpathmoveto{\pgfqpoint{0.000000in}{0.038889in}}%
\pgfpathlineto{\pgfqpoint{-0.008731in}{0.012017in}}%
\pgfpathlineto{\pgfqpoint{-0.036986in}{0.012017in}}%
\pgfpathlineto{\pgfqpoint{-0.014127in}{-0.004590in}}%
\pgfpathlineto{\pgfqpoint{-0.022858in}{-0.031462in}}%
\pgfpathlineto{\pgfqpoint{-0.000000in}{-0.014854in}}%
\pgfpathlineto{\pgfqpoint{0.022858in}{-0.031462in}}%
\pgfpathlineto{\pgfqpoint{0.014127in}{-0.004590in}}%
\pgfpathlineto{\pgfqpoint{0.036986in}{0.012017in}}%
\pgfpathlineto{\pgfqpoint{0.008731in}{0.012017in}}%
\pgfpathclose%
\pgfusepath{fill}%
}%
\begin{pgfscope}%
\pgfsys@transformshift{0.462984in}{0.553305in}%
\pgfsys@useobject{currentmarker}{}%
\end{pgfscope}%
\begin{pgfscope}%
\pgfsys@transformshift{0.493765in}{0.491093in}%
\pgfsys@useobject{currentmarker}{}%
\end{pgfscope}%
\begin{pgfscope}%
\pgfsys@transformshift{0.526258in}{0.443547in}%
\pgfsys@useobject{currentmarker}{}%
\end{pgfscope}%
\begin{pgfscope}%
\pgfsys@transformshift{0.704818in}{0.642519in}%
\pgfsys@useobject{currentmarker}{}%
\end{pgfscope}%
\begin{pgfscope}%
\pgfsys@transformshift{0.709273in}{0.632887in}%
\pgfsys@useobject{currentmarker}{}%
\end{pgfscope}%
\begin{pgfscope}%
\pgfsys@transformshift{0.912504in}{0.799927in}%
\pgfsys@useobject{currentmarker}{}%
\end{pgfscope}%
\begin{pgfscope}%
\pgfsys@transformshift{1.136212in}{0.924711in}%
\pgfsys@useobject{currentmarker}{}%
\end{pgfscope}%
\begin{pgfscope}%
\pgfsys@transformshift{1.147478in}{0.901286in}%
\pgfsys@useobject{currentmarker}{}%
\end{pgfscope}%
\begin{pgfscope}%
\pgfsys@transformshift{1.235284in}{0.881579in}%
\pgfsys@useobject{currentmarker}{}%
\end{pgfscope}%
\begin{pgfscope}%
\pgfsys@transformshift{1.254878in}{0.941038in}%
\pgfsys@useobject{currentmarker}{}%
\end{pgfscope}%
\begin{pgfscope}%
\pgfsys@transformshift{1.364408in}{1.040047in}%
\pgfsys@useobject{currentmarker}{}%
\end{pgfscope}%
\begin{pgfscope}%
\pgfsys@transformshift{1.452386in}{0.999806in}%
\pgfsys@useobject{currentmarker}{}%
\end{pgfscope}%
\begin{pgfscope}%
\pgfsys@transformshift{1.489712in}{1.114615in}%
\pgfsys@useobject{currentmarker}{}%
\end{pgfscope}%
\begin{pgfscope}%
\pgfsys@transformshift{1.727314in}{1.292992in}%
\pgfsys@useobject{currentmarker}{}%
\end{pgfscope}%
\begin{pgfscope}%
\pgfsys@transformshift{1.868909in}{1.218329in}%
\pgfsys@useobject{currentmarker}{}%
\end{pgfscope}%
\begin{pgfscope}%
\pgfsys@transformshift{1.890244in}{1.246844in}%
\pgfsys@useobject{currentmarker}{}%
\end{pgfscope}%
\begin{pgfscope}%
\pgfsys@transformshift{2.042485in}{1.397951in}%
\pgfsys@useobject{currentmarker}{}%
\end{pgfscope}%
\begin{pgfscope}%
\pgfsys@transformshift{2.112283in}{1.377314in}%
\pgfsys@useobject{currentmarker}{}%
\end{pgfscope}%
\begin{pgfscope}%
\pgfsys@transformshift{2.232189in}{1.590056in}%
\pgfsys@useobject{currentmarker}{}%
\end{pgfscope}%
\begin{pgfscope}%
\pgfsys@transformshift{2.450813in}{1.725519in}%
\pgfsys@useobject{currentmarker}{}%
\end{pgfscope}%
\end{pgfscope}%
\begin{pgfscope}%
\pgfpathrectangle{\pgfqpoint{0.375000in}{0.250000in}}{\pgfqpoint{2.325000in}{1.550000in}} %
\pgfusepath{clip}%
\pgfsetroundcap%
\pgfsetroundjoin%
\pgfsetlinewidth{0.200750pt}%
\definecolor{currentstroke}{rgb}{0.125490,0.290196,0.529412}%
\pgfsetstrokecolor{currentstroke}%
\pgfsetdash{}{0pt}%
\pgfpathmoveto{\pgfqpoint{0.462984in}{0.560362in}}%
\pgfpathlineto{\pgfqpoint{0.493765in}{0.499938in}}%
\pgfpathlineto{\pgfqpoint{0.526258in}{0.449054in}}%
\pgfpathlineto{\pgfqpoint{0.704818in}{0.651292in}}%
\pgfpathlineto{\pgfqpoint{0.709273in}{0.636804in}}%
\pgfpathlineto{\pgfqpoint{0.912504in}{0.808203in}}%
\pgfpathlineto{\pgfqpoint{1.136212in}{0.932695in}}%
\pgfpathlineto{\pgfqpoint{1.147478in}{0.905681in}}%
\pgfpathlineto{\pgfqpoint{1.235284in}{0.890207in}}%
\pgfpathlineto{\pgfqpoint{1.254878in}{0.946064in}}%
\pgfpathlineto{\pgfqpoint{1.364408in}{1.048148in}}%
\pgfpathlineto{\pgfqpoint{1.452386in}{1.002184in}}%
\pgfpathlineto{\pgfqpoint{1.489712in}{1.123184in}}%
\pgfpathlineto{\pgfqpoint{1.727314in}{1.301751in}}%
\pgfpathlineto{\pgfqpoint{1.868909in}{1.226201in}}%
\pgfpathlineto{\pgfqpoint{1.890244in}{1.251988in}}%
\pgfpathlineto{\pgfqpoint{2.042485in}{1.399630in}}%
\pgfpathlineto{\pgfqpoint{2.112283in}{1.378489in}}%
\pgfpathlineto{\pgfqpoint{2.232189in}{1.598548in}}%
\pgfpathlineto{\pgfqpoint{2.450813in}{1.731446in}}%
\pgfusepath{stroke}%
\end{pgfscope}%
\begin{pgfscope}%
\pgfpathrectangle{\pgfqpoint{0.375000in}{0.250000in}}{\pgfqpoint{2.325000in}{1.550000in}} %
\pgfusepath{clip}%
\pgfsetroundcap%
\pgfsetroundjoin%
\pgfsetlinewidth{0.200750pt}%
\definecolor{currentstroke}{rgb}{0.125490,0.290196,0.529412}%
\pgfsetstrokecolor{currentstroke}%
\pgfsetdash{}{0pt}%
\pgfpathmoveto{\pgfqpoint{0.462984in}{0.546248in}}%
\pgfpathlineto{\pgfqpoint{0.493765in}{0.482247in}}%
\pgfpathlineto{\pgfqpoint{0.526258in}{0.438040in}}%
\pgfpathlineto{\pgfqpoint{0.704818in}{0.633747in}}%
\pgfpathlineto{\pgfqpoint{0.709273in}{0.628970in}}%
\pgfpathlineto{\pgfqpoint{0.912504in}{0.791651in}}%
\pgfpathlineto{\pgfqpoint{1.136212in}{0.916727in}}%
\pgfpathlineto{\pgfqpoint{1.147478in}{0.896890in}}%
\pgfpathlineto{\pgfqpoint{1.235284in}{0.872952in}}%
\pgfpathlineto{\pgfqpoint{1.254878in}{0.936012in}}%
\pgfpathlineto{\pgfqpoint{1.364408in}{1.031946in}}%
\pgfpathlineto{\pgfqpoint{1.452386in}{0.997429in}}%
\pgfpathlineto{\pgfqpoint{1.489712in}{1.106047in}}%
\pgfpathlineto{\pgfqpoint{1.727314in}{1.284232in}}%
\pgfpathlineto{\pgfqpoint{1.868909in}{1.210456in}}%
\pgfpathlineto{\pgfqpoint{1.890244in}{1.241700in}}%
\pgfpathlineto{\pgfqpoint{2.042485in}{1.396272in}}%
\pgfpathlineto{\pgfqpoint{2.112283in}{1.376139in}}%
\pgfpathlineto{\pgfqpoint{2.232189in}{1.581564in}}%
\pgfpathlineto{\pgfqpoint{2.450813in}{1.719592in}}%
\pgfusepath{stroke}%
\end{pgfscope}%
\begin{pgfscope}%
\pgfsetrectcap%
\pgfsetmiterjoin%
\pgfsetlinewidth{0.000000pt}%
\definecolor{currentstroke}{rgb}{1.000000,1.000000,1.000000}%
\pgfsetstrokecolor{currentstroke}%
\pgfsetdash{}{0pt}%
\pgfpathmoveto{\pgfqpoint{2.700000in}{0.250000in}}%
\pgfpathlineto{\pgfqpoint{2.700000in}{1.800000in}}%
\pgfusepath{}%
\end{pgfscope}%
\begin{pgfscope}%
\pgfsetrectcap%
\pgfsetmiterjoin%
\pgfsetlinewidth{0.000000pt}%
\definecolor{currentstroke}{rgb}{1.000000,1.000000,1.000000}%
\pgfsetstrokecolor{currentstroke}%
\pgfsetdash{}{0pt}%
\pgfpathmoveto{\pgfqpoint{0.375000in}{1.800000in}}%
\pgfpathlineto{\pgfqpoint{2.700000in}{1.800000in}}%
\pgfusepath{}%
\end{pgfscope}%
\begin{pgfscope}%
\pgfsetrectcap%
\pgfsetmiterjoin%
\pgfsetlinewidth{0.000000pt}%
\definecolor{currentstroke}{rgb}{1.000000,1.000000,1.000000}%
\pgfsetstrokecolor{currentstroke}%
\pgfsetdash{}{0pt}%
\pgfpathmoveto{\pgfqpoint{0.375000in}{0.250000in}}%
\pgfpathlineto{\pgfqpoint{2.700000in}{0.250000in}}%
\pgfusepath{}%
\end{pgfscope}%
\begin{pgfscope}%
\pgfsetrectcap%
\pgfsetmiterjoin%
\pgfsetlinewidth{0.000000pt}%
\definecolor{currentstroke}{rgb}{1.000000,1.000000,1.000000}%
\pgfsetstrokecolor{currentstroke}%
\pgfsetdash{}{0pt}%
\pgfpathmoveto{\pgfqpoint{0.375000in}{0.250000in}}%
\pgfpathlineto{\pgfqpoint{0.375000in}{1.800000in}}%
\pgfusepath{}%
\end{pgfscope}%
\end{pgfpicture}%
\makeatother%
\endgroup%

    \caption{Predictions for model trained with $150$ samples.}
    \label{fig_predftest}
  \end{subfigure}
  \begin{subfigure}[h]{.5\linewidth}
    %% Creator: Matplotlib, PGF backend
%%
%% To include the figure in your LaTeX document, write
%%   \input{<filename>.pgf}
%%
%% Make sure the required packages are loaded in your preamble
%%   \usepackage{pgf}
%%
%% Figures using additional raster images can only be included by \input if
%% they are in the same directory as the main LaTeX file. For loading figures
%% from other directories you can use the `import` package
%%   \usepackage{import}
%% and then include the figures with
%%   \import{<path to file>}{<filename>.pgf}
%%
%% Matplotlib used the following preamble
%%   \usepackage[utf8x]{inputenc}
%%   \usepackage[T1]{fontenc}
%%   \usepackage{cmbright}
%%
\begingroup%
\makeatletter%
\begin{pgfpicture}%
\pgfpathrectangle{\pgfpointorigin}{\pgfqpoint{3.000000in}{2.000000in}}%
\pgfusepath{use as bounding box, clip}%
\begin{pgfscope}%
\pgfsetbuttcap%
\pgfsetmiterjoin%
\definecolor{currentfill}{rgb}{1.000000,1.000000,1.000000}%
\pgfsetfillcolor{currentfill}%
\pgfsetlinewidth{0.000000pt}%
\definecolor{currentstroke}{rgb}{1.000000,1.000000,1.000000}%
\pgfsetstrokecolor{currentstroke}%
\pgfsetdash{}{0pt}%
\pgfpathmoveto{\pgfqpoint{0.000000in}{0.000000in}}%
\pgfpathlineto{\pgfqpoint{3.000000in}{0.000000in}}%
\pgfpathlineto{\pgfqpoint{3.000000in}{2.000000in}}%
\pgfpathlineto{\pgfqpoint{0.000000in}{2.000000in}}%
\pgfpathclose%
\pgfusepath{fill}%
\end{pgfscope}%
\begin{pgfscope}%
\pgfsetbuttcap%
\pgfsetmiterjoin%
\definecolor{currentfill}{rgb}{0.917647,0.917647,0.949020}%
\pgfsetfillcolor{currentfill}%
\pgfsetlinewidth{0.000000pt}%
\definecolor{currentstroke}{rgb}{0.000000,0.000000,0.000000}%
\pgfsetstrokecolor{currentstroke}%
\pgfsetstrokeopacity{0.000000}%
\pgfsetdash{}{0pt}%
\pgfpathmoveto{\pgfqpoint{0.375000in}{0.250000in}}%
\pgfpathlineto{\pgfqpoint{2.700000in}{0.250000in}}%
\pgfpathlineto{\pgfqpoint{2.700000in}{1.800000in}}%
\pgfpathlineto{\pgfqpoint{0.375000in}{1.800000in}}%
\pgfpathclose%
\pgfusepath{fill}%
\end{pgfscope}%
\begin{pgfscope}%
\pgfpathrectangle{\pgfqpoint{0.375000in}{0.250000in}}{\pgfqpoint{2.325000in}{1.550000in}} %
\pgfusepath{clip}%
\pgfsetroundcap%
\pgfsetroundjoin%
\pgfsetlinewidth{0.803000pt}%
\definecolor{currentstroke}{rgb}{1.000000,1.000000,1.000000}%
\pgfsetstrokecolor{currentstroke}%
\pgfsetdash{}{0pt}%
\pgfpathmoveto{\pgfqpoint{0.375000in}{0.250000in}}%
\pgfpathlineto{\pgfqpoint{0.375000in}{1.800000in}}%
\pgfusepath{stroke}%
\end{pgfscope}%
\begin{pgfscope}%
\pgfsetbuttcap%
\pgfsetroundjoin%
\definecolor{currentfill}{rgb}{0.150000,0.150000,0.150000}%
\pgfsetfillcolor{currentfill}%
\pgfsetlinewidth{0.803000pt}%
\definecolor{currentstroke}{rgb}{0.150000,0.150000,0.150000}%
\pgfsetstrokecolor{currentstroke}%
\pgfsetdash{}{0pt}%
\pgfsys@defobject{currentmarker}{\pgfqpoint{0.000000in}{0.000000in}}{\pgfqpoint{0.000000in}{0.000000in}}{%
\pgfpathmoveto{\pgfqpoint{0.000000in}{0.000000in}}%
\pgfpathlineto{\pgfqpoint{0.000000in}{0.000000in}}%
\pgfusepath{stroke,fill}%
}%
\begin{pgfscope}%
\pgfsys@transformshift{0.375000in}{0.250000in}%
\pgfsys@useobject{currentmarker}{}%
\end{pgfscope}%
\end{pgfscope}%
\begin{pgfscope}%
\pgfsetbuttcap%
\pgfsetroundjoin%
\definecolor{currentfill}{rgb}{0.150000,0.150000,0.150000}%
\pgfsetfillcolor{currentfill}%
\pgfsetlinewidth{0.803000pt}%
\definecolor{currentstroke}{rgb}{0.150000,0.150000,0.150000}%
\pgfsetstrokecolor{currentstroke}%
\pgfsetdash{}{0pt}%
\pgfsys@defobject{currentmarker}{\pgfqpoint{0.000000in}{0.000000in}}{\pgfqpoint{0.000000in}{0.000000in}}{%
\pgfpathmoveto{\pgfqpoint{0.000000in}{0.000000in}}%
\pgfpathlineto{\pgfqpoint{0.000000in}{0.000000in}}%
\pgfusepath{stroke,fill}%
}%
\begin{pgfscope}%
\pgfsys@transformshift{0.375000in}{1.800000in}%
\pgfsys@useobject{currentmarker}{}%
\end{pgfscope}%
\end{pgfscope}%
\begin{pgfscope}%
\definecolor{textcolor}{rgb}{0.150000,0.150000,0.150000}%
\pgfsetstrokecolor{textcolor}%
\pgfsetfillcolor{textcolor}%
\pgftext[x=0.375000in,y=0.172222in,,top]{\color{textcolor}\sffamily\fontsize{8.000000}{9.600000}\selectfont 0}%
\end{pgfscope}%
\begin{pgfscope}%
\pgfpathrectangle{\pgfqpoint{0.375000in}{0.250000in}}{\pgfqpoint{2.325000in}{1.550000in}} %
\pgfusepath{clip}%
\pgfsetroundcap%
\pgfsetroundjoin%
\pgfsetlinewidth{0.803000pt}%
\definecolor{currentstroke}{rgb}{1.000000,1.000000,1.000000}%
\pgfsetstrokecolor{currentstroke}%
\pgfsetdash{}{0pt}%
\pgfpathmoveto{\pgfqpoint{0.707143in}{0.250000in}}%
\pgfpathlineto{\pgfqpoint{0.707143in}{1.800000in}}%
\pgfusepath{stroke}%
\end{pgfscope}%
\begin{pgfscope}%
\pgfsetbuttcap%
\pgfsetroundjoin%
\definecolor{currentfill}{rgb}{0.150000,0.150000,0.150000}%
\pgfsetfillcolor{currentfill}%
\pgfsetlinewidth{0.803000pt}%
\definecolor{currentstroke}{rgb}{0.150000,0.150000,0.150000}%
\pgfsetstrokecolor{currentstroke}%
\pgfsetdash{}{0pt}%
\pgfsys@defobject{currentmarker}{\pgfqpoint{0.000000in}{0.000000in}}{\pgfqpoint{0.000000in}{0.000000in}}{%
\pgfpathmoveto{\pgfqpoint{0.000000in}{0.000000in}}%
\pgfpathlineto{\pgfqpoint{0.000000in}{0.000000in}}%
\pgfusepath{stroke,fill}%
}%
\begin{pgfscope}%
\pgfsys@transformshift{0.707143in}{0.250000in}%
\pgfsys@useobject{currentmarker}{}%
\end{pgfscope}%
\end{pgfscope}%
\begin{pgfscope}%
\pgfsetbuttcap%
\pgfsetroundjoin%
\definecolor{currentfill}{rgb}{0.150000,0.150000,0.150000}%
\pgfsetfillcolor{currentfill}%
\pgfsetlinewidth{0.803000pt}%
\definecolor{currentstroke}{rgb}{0.150000,0.150000,0.150000}%
\pgfsetstrokecolor{currentstroke}%
\pgfsetdash{}{0pt}%
\pgfsys@defobject{currentmarker}{\pgfqpoint{0.000000in}{0.000000in}}{\pgfqpoint{0.000000in}{0.000000in}}{%
\pgfpathmoveto{\pgfqpoint{0.000000in}{0.000000in}}%
\pgfpathlineto{\pgfqpoint{0.000000in}{0.000000in}}%
\pgfusepath{stroke,fill}%
}%
\begin{pgfscope}%
\pgfsys@transformshift{0.707143in}{1.800000in}%
\pgfsys@useobject{currentmarker}{}%
\end{pgfscope}%
\end{pgfscope}%
\begin{pgfscope}%
\definecolor{textcolor}{rgb}{0.150000,0.150000,0.150000}%
\pgfsetstrokecolor{textcolor}%
\pgfsetfillcolor{textcolor}%
\pgftext[x=0.707143in,y=0.172222in,,top]{\color{textcolor}\sffamily\fontsize{8.000000}{9.600000}\selectfont 1}%
\end{pgfscope}%
\begin{pgfscope}%
\pgfpathrectangle{\pgfqpoint{0.375000in}{0.250000in}}{\pgfqpoint{2.325000in}{1.550000in}} %
\pgfusepath{clip}%
\pgfsetroundcap%
\pgfsetroundjoin%
\pgfsetlinewidth{0.803000pt}%
\definecolor{currentstroke}{rgb}{1.000000,1.000000,1.000000}%
\pgfsetstrokecolor{currentstroke}%
\pgfsetdash{}{0pt}%
\pgfpathmoveto{\pgfqpoint{1.039286in}{0.250000in}}%
\pgfpathlineto{\pgfqpoint{1.039286in}{1.800000in}}%
\pgfusepath{stroke}%
\end{pgfscope}%
\begin{pgfscope}%
\pgfsetbuttcap%
\pgfsetroundjoin%
\definecolor{currentfill}{rgb}{0.150000,0.150000,0.150000}%
\pgfsetfillcolor{currentfill}%
\pgfsetlinewidth{0.803000pt}%
\definecolor{currentstroke}{rgb}{0.150000,0.150000,0.150000}%
\pgfsetstrokecolor{currentstroke}%
\pgfsetdash{}{0pt}%
\pgfsys@defobject{currentmarker}{\pgfqpoint{0.000000in}{0.000000in}}{\pgfqpoint{0.000000in}{0.000000in}}{%
\pgfpathmoveto{\pgfqpoint{0.000000in}{0.000000in}}%
\pgfpathlineto{\pgfqpoint{0.000000in}{0.000000in}}%
\pgfusepath{stroke,fill}%
}%
\begin{pgfscope}%
\pgfsys@transformshift{1.039286in}{0.250000in}%
\pgfsys@useobject{currentmarker}{}%
\end{pgfscope}%
\end{pgfscope}%
\begin{pgfscope}%
\pgfsetbuttcap%
\pgfsetroundjoin%
\definecolor{currentfill}{rgb}{0.150000,0.150000,0.150000}%
\pgfsetfillcolor{currentfill}%
\pgfsetlinewidth{0.803000pt}%
\definecolor{currentstroke}{rgb}{0.150000,0.150000,0.150000}%
\pgfsetstrokecolor{currentstroke}%
\pgfsetdash{}{0pt}%
\pgfsys@defobject{currentmarker}{\pgfqpoint{0.000000in}{0.000000in}}{\pgfqpoint{0.000000in}{0.000000in}}{%
\pgfpathmoveto{\pgfqpoint{0.000000in}{0.000000in}}%
\pgfpathlineto{\pgfqpoint{0.000000in}{0.000000in}}%
\pgfusepath{stroke,fill}%
}%
\begin{pgfscope}%
\pgfsys@transformshift{1.039286in}{1.800000in}%
\pgfsys@useobject{currentmarker}{}%
\end{pgfscope}%
\end{pgfscope}%
\begin{pgfscope}%
\definecolor{textcolor}{rgb}{0.150000,0.150000,0.150000}%
\pgfsetstrokecolor{textcolor}%
\pgfsetfillcolor{textcolor}%
\pgftext[x=1.039286in,y=0.172222in,,top]{\color{textcolor}\sffamily\fontsize{8.000000}{9.600000}\selectfont 2}%
\end{pgfscope}%
\begin{pgfscope}%
\pgfpathrectangle{\pgfqpoint{0.375000in}{0.250000in}}{\pgfqpoint{2.325000in}{1.550000in}} %
\pgfusepath{clip}%
\pgfsetroundcap%
\pgfsetroundjoin%
\pgfsetlinewidth{0.803000pt}%
\definecolor{currentstroke}{rgb}{1.000000,1.000000,1.000000}%
\pgfsetstrokecolor{currentstroke}%
\pgfsetdash{}{0pt}%
\pgfpathmoveto{\pgfqpoint{1.371429in}{0.250000in}}%
\pgfpathlineto{\pgfqpoint{1.371429in}{1.800000in}}%
\pgfusepath{stroke}%
\end{pgfscope}%
\begin{pgfscope}%
\pgfsetbuttcap%
\pgfsetroundjoin%
\definecolor{currentfill}{rgb}{0.150000,0.150000,0.150000}%
\pgfsetfillcolor{currentfill}%
\pgfsetlinewidth{0.803000pt}%
\definecolor{currentstroke}{rgb}{0.150000,0.150000,0.150000}%
\pgfsetstrokecolor{currentstroke}%
\pgfsetdash{}{0pt}%
\pgfsys@defobject{currentmarker}{\pgfqpoint{0.000000in}{0.000000in}}{\pgfqpoint{0.000000in}{0.000000in}}{%
\pgfpathmoveto{\pgfqpoint{0.000000in}{0.000000in}}%
\pgfpathlineto{\pgfqpoint{0.000000in}{0.000000in}}%
\pgfusepath{stroke,fill}%
}%
\begin{pgfscope}%
\pgfsys@transformshift{1.371429in}{0.250000in}%
\pgfsys@useobject{currentmarker}{}%
\end{pgfscope}%
\end{pgfscope}%
\begin{pgfscope}%
\pgfsetbuttcap%
\pgfsetroundjoin%
\definecolor{currentfill}{rgb}{0.150000,0.150000,0.150000}%
\pgfsetfillcolor{currentfill}%
\pgfsetlinewidth{0.803000pt}%
\definecolor{currentstroke}{rgb}{0.150000,0.150000,0.150000}%
\pgfsetstrokecolor{currentstroke}%
\pgfsetdash{}{0pt}%
\pgfsys@defobject{currentmarker}{\pgfqpoint{0.000000in}{0.000000in}}{\pgfqpoint{0.000000in}{0.000000in}}{%
\pgfpathmoveto{\pgfqpoint{0.000000in}{0.000000in}}%
\pgfpathlineto{\pgfqpoint{0.000000in}{0.000000in}}%
\pgfusepath{stroke,fill}%
}%
\begin{pgfscope}%
\pgfsys@transformshift{1.371429in}{1.800000in}%
\pgfsys@useobject{currentmarker}{}%
\end{pgfscope}%
\end{pgfscope}%
\begin{pgfscope}%
\definecolor{textcolor}{rgb}{0.150000,0.150000,0.150000}%
\pgfsetstrokecolor{textcolor}%
\pgfsetfillcolor{textcolor}%
\pgftext[x=1.371429in,y=0.172222in,,top]{\color{textcolor}\sffamily\fontsize{8.000000}{9.600000}\selectfont 3}%
\end{pgfscope}%
\begin{pgfscope}%
\pgfpathrectangle{\pgfqpoint{0.375000in}{0.250000in}}{\pgfqpoint{2.325000in}{1.550000in}} %
\pgfusepath{clip}%
\pgfsetroundcap%
\pgfsetroundjoin%
\pgfsetlinewidth{0.803000pt}%
\definecolor{currentstroke}{rgb}{1.000000,1.000000,1.000000}%
\pgfsetstrokecolor{currentstroke}%
\pgfsetdash{}{0pt}%
\pgfpathmoveto{\pgfqpoint{1.703571in}{0.250000in}}%
\pgfpathlineto{\pgfqpoint{1.703571in}{1.800000in}}%
\pgfusepath{stroke}%
\end{pgfscope}%
\begin{pgfscope}%
\pgfsetbuttcap%
\pgfsetroundjoin%
\definecolor{currentfill}{rgb}{0.150000,0.150000,0.150000}%
\pgfsetfillcolor{currentfill}%
\pgfsetlinewidth{0.803000pt}%
\definecolor{currentstroke}{rgb}{0.150000,0.150000,0.150000}%
\pgfsetstrokecolor{currentstroke}%
\pgfsetdash{}{0pt}%
\pgfsys@defobject{currentmarker}{\pgfqpoint{0.000000in}{0.000000in}}{\pgfqpoint{0.000000in}{0.000000in}}{%
\pgfpathmoveto{\pgfqpoint{0.000000in}{0.000000in}}%
\pgfpathlineto{\pgfqpoint{0.000000in}{0.000000in}}%
\pgfusepath{stroke,fill}%
}%
\begin{pgfscope}%
\pgfsys@transformshift{1.703571in}{0.250000in}%
\pgfsys@useobject{currentmarker}{}%
\end{pgfscope}%
\end{pgfscope}%
\begin{pgfscope}%
\pgfsetbuttcap%
\pgfsetroundjoin%
\definecolor{currentfill}{rgb}{0.150000,0.150000,0.150000}%
\pgfsetfillcolor{currentfill}%
\pgfsetlinewidth{0.803000pt}%
\definecolor{currentstroke}{rgb}{0.150000,0.150000,0.150000}%
\pgfsetstrokecolor{currentstroke}%
\pgfsetdash{}{0pt}%
\pgfsys@defobject{currentmarker}{\pgfqpoint{0.000000in}{0.000000in}}{\pgfqpoint{0.000000in}{0.000000in}}{%
\pgfpathmoveto{\pgfqpoint{0.000000in}{0.000000in}}%
\pgfpathlineto{\pgfqpoint{0.000000in}{0.000000in}}%
\pgfusepath{stroke,fill}%
}%
\begin{pgfscope}%
\pgfsys@transformshift{1.703571in}{1.800000in}%
\pgfsys@useobject{currentmarker}{}%
\end{pgfscope}%
\end{pgfscope}%
\begin{pgfscope}%
\definecolor{textcolor}{rgb}{0.150000,0.150000,0.150000}%
\pgfsetstrokecolor{textcolor}%
\pgfsetfillcolor{textcolor}%
\pgftext[x=1.703571in,y=0.172222in,,top]{\color{textcolor}\sffamily\fontsize{8.000000}{9.600000}\selectfont 4}%
\end{pgfscope}%
\begin{pgfscope}%
\pgfpathrectangle{\pgfqpoint{0.375000in}{0.250000in}}{\pgfqpoint{2.325000in}{1.550000in}} %
\pgfusepath{clip}%
\pgfsetroundcap%
\pgfsetroundjoin%
\pgfsetlinewidth{0.803000pt}%
\definecolor{currentstroke}{rgb}{1.000000,1.000000,1.000000}%
\pgfsetstrokecolor{currentstroke}%
\pgfsetdash{}{0pt}%
\pgfpathmoveto{\pgfqpoint{2.035714in}{0.250000in}}%
\pgfpathlineto{\pgfqpoint{2.035714in}{1.800000in}}%
\pgfusepath{stroke}%
\end{pgfscope}%
\begin{pgfscope}%
\pgfsetbuttcap%
\pgfsetroundjoin%
\definecolor{currentfill}{rgb}{0.150000,0.150000,0.150000}%
\pgfsetfillcolor{currentfill}%
\pgfsetlinewidth{0.803000pt}%
\definecolor{currentstroke}{rgb}{0.150000,0.150000,0.150000}%
\pgfsetstrokecolor{currentstroke}%
\pgfsetdash{}{0pt}%
\pgfsys@defobject{currentmarker}{\pgfqpoint{0.000000in}{0.000000in}}{\pgfqpoint{0.000000in}{0.000000in}}{%
\pgfpathmoveto{\pgfqpoint{0.000000in}{0.000000in}}%
\pgfpathlineto{\pgfqpoint{0.000000in}{0.000000in}}%
\pgfusepath{stroke,fill}%
}%
\begin{pgfscope}%
\pgfsys@transformshift{2.035714in}{0.250000in}%
\pgfsys@useobject{currentmarker}{}%
\end{pgfscope}%
\end{pgfscope}%
\begin{pgfscope}%
\pgfsetbuttcap%
\pgfsetroundjoin%
\definecolor{currentfill}{rgb}{0.150000,0.150000,0.150000}%
\pgfsetfillcolor{currentfill}%
\pgfsetlinewidth{0.803000pt}%
\definecolor{currentstroke}{rgb}{0.150000,0.150000,0.150000}%
\pgfsetstrokecolor{currentstroke}%
\pgfsetdash{}{0pt}%
\pgfsys@defobject{currentmarker}{\pgfqpoint{0.000000in}{0.000000in}}{\pgfqpoint{0.000000in}{0.000000in}}{%
\pgfpathmoveto{\pgfqpoint{0.000000in}{0.000000in}}%
\pgfpathlineto{\pgfqpoint{0.000000in}{0.000000in}}%
\pgfusepath{stroke,fill}%
}%
\begin{pgfscope}%
\pgfsys@transformshift{2.035714in}{1.800000in}%
\pgfsys@useobject{currentmarker}{}%
\end{pgfscope}%
\end{pgfscope}%
\begin{pgfscope}%
\definecolor{textcolor}{rgb}{0.150000,0.150000,0.150000}%
\pgfsetstrokecolor{textcolor}%
\pgfsetfillcolor{textcolor}%
\pgftext[x=2.035714in,y=0.172222in,,top]{\color{textcolor}\sffamily\fontsize{8.000000}{9.600000}\selectfont 5}%
\end{pgfscope}%
\begin{pgfscope}%
\pgfpathrectangle{\pgfqpoint{0.375000in}{0.250000in}}{\pgfqpoint{2.325000in}{1.550000in}} %
\pgfusepath{clip}%
\pgfsetroundcap%
\pgfsetroundjoin%
\pgfsetlinewidth{0.803000pt}%
\definecolor{currentstroke}{rgb}{1.000000,1.000000,1.000000}%
\pgfsetstrokecolor{currentstroke}%
\pgfsetdash{}{0pt}%
\pgfpathmoveto{\pgfqpoint{2.367857in}{0.250000in}}%
\pgfpathlineto{\pgfqpoint{2.367857in}{1.800000in}}%
\pgfusepath{stroke}%
\end{pgfscope}%
\begin{pgfscope}%
\pgfsetbuttcap%
\pgfsetroundjoin%
\definecolor{currentfill}{rgb}{0.150000,0.150000,0.150000}%
\pgfsetfillcolor{currentfill}%
\pgfsetlinewidth{0.803000pt}%
\definecolor{currentstroke}{rgb}{0.150000,0.150000,0.150000}%
\pgfsetstrokecolor{currentstroke}%
\pgfsetdash{}{0pt}%
\pgfsys@defobject{currentmarker}{\pgfqpoint{0.000000in}{0.000000in}}{\pgfqpoint{0.000000in}{0.000000in}}{%
\pgfpathmoveto{\pgfqpoint{0.000000in}{0.000000in}}%
\pgfpathlineto{\pgfqpoint{0.000000in}{0.000000in}}%
\pgfusepath{stroke,fill}%
}%
\begin{pgfscope}%
\pgfsys@transformshift{2.367857in}{0.250000in}%
\pgfsys@useobject{currentmarker}{}%
\end{pgfscope}%
\end{pgfscope}%
\begin{pgfscope}%
\pgfsetbuttcap%
\pgfsetroundjoin%
\definecolor{currentfill}{rgb}{0.150000,0.150000,0.150000}%
\pgfsetfillcolor{currentfill}%
\pgfsetlinewidth{0.803000pt}%
\definecolor{currentstroke}{rgb}{0.150000,0.150000,0.150000}%
\pgfsetstrokecolor{currentstroke}%
\pgfsetdash{}{0pt}%
\pgfsys@defobject{currentmarker}{\pgfqpoint{0.000000in}{0.000000in}}{\pgfqpoint{0.000000in}{0.000000in}}{%
\pgfpathmoveto{\pgfqpoint{0.000000in}{0.000000in}}%
\pgfpathlineto{\pgfqpoint{0.000000in}{0.000000in}}%
\pgfusepath{stroke,fill}%
}%
\begin{pgfscope}%
\pgfsys@transformshift{2.367857in}{1.800000in}%
\pgfsys@useobject{currentmarker}{}%
\end{pgfscope}%
\end{pgfscope}%
\begin{pgfscope}%
\definecolor{textcolor}{rgb}{0.150000,0.150000,0.150000}%
\pgfsetstrokecolor{textcolor}%
\pgfsetfillcolor{textcolor}%
\pgftext[x=2.367857in,y=0.172222in,,top]{\color{textcolor}\sffamily\fontsize{8.000000}{9.600000}\selectfont 6}%
\end{pgfscope}%
\begin{pgfscope}%
\pgfpathrectangle{\pgfqpoint{0.375000in}{0.250000in}}{\pgfqpoint{2.325000in}{1.550000in}} %
\pgfusepath{clip}%
\pgfsetroundcap%
\pgfsetroundjoin%
\pgfsetlinewidth{0.803000pt}%
\definecolor{currentstroke}{rgb}{1.000000,1.000000,1.000000}%
\pgfsetstrokecolor{currentstroke}%
\pgfsetdash{}{0pt}%
\pgfpathmoveto{\pgfqpoint{2.700000in}{0.250000in}}%
\pgfpathlineto{\pgfqpoint{2.700000in}{1.800000in}}%
\pgfusepath{stroke}%
\end{pgfscope}%
\begin{pgfscope}%
\pgfsetbuttcap%
\pgfsetroundjoin%
\definecolor{currentfill}{rgb}{0.150000,0.150000,0.150000}%
\pgfsetfillcolor{currentfill}%
\pgfsetlinewidth{0.803000pt}%
\definecolor{currentstroke}{rgb}{0.150000,0.150000,0.150000}%
\pgfsetstrokecolor{currentstroke}%
\pgfsetdash{}{0pt}%
\pgfsys@defobject{currentmarker}{\pgfqpoint{0.000000in}{0.000000in}}{\pgfqpoint{0.000000in}{0.000000in}}{%
\pgfpathmoveto{\pgfqpoint{0.000000in}{0.000000in}}%
\pgfpathlineto{\pgfqpoint{0.000000in}{0.000000in}}%
\pgfusepath{stroke,fill}%
}%
\begin{pgfscope}%
\pgfsys@transformshift{2.700000in}{0.250000in}%
\pgfsys@useobject{currentmarker}{}%
\end{pgfscope}%
\end{pgfscope}%
\begin{pgfscope}%
\pgfsetbuttcap%
\pgfsetroundjoin%
\definecolor{currentfill}{rgb}{0.150000,0.150000,0.150000}%
\pgfsetfillcolor{currentfill}%
\pgfsetlinewidth{0.803000pt}%
\definecolor{currentstroke}{rgb}{0.150000,0.150000,0.150000}%
\pgfsetstrokecolor{currentstroke}%
\pgfsetdash{}{0pt}%
\pgfsys@defobject{currentmarker}{\pgfqpoint{0.000000in}{0.000000in}}{\pgfqpoint{0.000000in}{0.000000in}}{%
\pgfpathmoveto{\pgfqpoint{0.000000in}{0.000000in}}%
\pgfpathlineto{\pgfqpoint{0.000000in}{0.000000in}}%
\pgfusepath{stroke,fill}%
}%
\begin{pgfscope}%
\pgfsys@transformshift{2.700000in}{1.800000in}%
\pgfsys@useobject{currentmarker}{}%
\end{pgfscope}%
\end{pgfscope}%
\begin{pgfscope}%
\definecolor{textcolor}{rgb}{0.150000,0.150000,0.150000}%
\pgfsetstrokecolor{textcolor}%
\pgfsetfillcolor{textcolor}%
\pgftext[x=2.700000in,y=0.172222in,,top]{\color{textcolor}\sffamily\fontsize{8.000000}{9.600000}\selectfont 7}%
\end{pgfscope}%
\begin{pgfscope}%
\pgfpathrectangle{\pgfqpoint{0.375000in}{0.250000in}}{\pgfqpoint{2.325000in}{1.550000in}} %
\pgfusepath{clip}%
\pgfsetroundcap%
\pgfsetroundjoin%
\pgfsetlinewidth{0.803000pt}%
\definecolor{currentstroke}{rgb}{1.000000,1.000000,1.000000}%
\pgfsetstrokecolor{currentstroke}%
\pgfsetdash{}{0pt}%
\pgfpathmoveto{\pgfqpoint{0.375000in}{0.250000in}}%
\pgfpathlineto{\pgfqpoint{2.700000in}{0.250000in}}%
\pgfusepath{stroke}%
\end{pgfscope}%
\begin{pgfscope}%
\pgfsetbuttcap%
\pgfsetroundjoin%
\definecolor{currentfill}{rgb}{0.150000,0.150000,0.150000}%
\pgfsetfillcolor{currentfill}%
\pgfsetlinewidth{0.803000pt}%
\definecolor{currentstroke}{rgb}{0.150000,0.150000,0.150000}%
\pgfsetstrokecolor{currentstroke}%
\pgfsetdash{}{0pt}%
\pgfsys@defobject{currentmarker}{\pgfqpoint{0.000000in}{0.000000in}}{\pgfqpoint{0.000000in}{0.000000in}}{%
\pgfpathmoveto{\pgfqpoint{0.000000in}{0.000000in}}%
\pgfpathlineto{\pgfqpoint{0.000000in}{0.000000in}}%
\pgfusepath{stroke,fill}%
}%
\begin{pgfscope}%
\pgfsys@transformshift{0.375000in}{0.250000in}%
\pgfsys@useobject{currentmarker}{}%
\end{pgfscope}%
\end{pgfscope}%
\begin{pgfscope}%
\pgfsetbuttcap%
\pgfsetroundjoin%
\definecolor{currentfill}{rgb}{0.150000,0.150000,0.150000}%
\pgfsetfillcolor{currentfill}%
\pgfsetlinewidth{0.803000pt}%
\definecolor{currentstroke}{rgb}{0.150000,0.150000,0.150000}%
\pgfsetstrokecolor{currentstroke}%
\pgfsetdash{}{0pt}%
\pgfsys@defobject{currentmarker}{\pgfqpoint{0.000000in}{0.000000in}}{\pgfqpoint{0.000000in}{0.000000in}}{%
\pgfpathmoveto{\pgfqpoint{0.000000in}{0.000000in}}%
\pgfpathlineto{\pgfqpoint{0.000000in}{0.000000in}}%
\pgfusepath{stroke,fill}%
}%
\begin{pgfscope}%
\pgfsys@transformshift{2.700000in}{0.250000in}%
\pgfsys@useobject{currentmarker}{}%
\end{pgfscope}%
\end{pgfscope}%
\begin{pgfscope}%
\definecolor{textcolor}{rgb}{0.150000,0.150000,0.150000}%
\pgfsetstrokecolor{textcolor}%
\pgfsetfillcolor{textcolor}%
\pgftext[x=0.297222in,y=0.250000in,right,]{\color{textcolor}\sffamily\fontsize{8.000000}{9.600000}\selectfont −2}%
\end{pgfscope}%
\begin{pgfscope}%
\pgfpathrectangle{\pgfqpoint{0.375000in}{0.250000in}}{\pgfqpoint{2.325000in}{1.550000in}} %
\pgfusepath{clip}%
\pgfsetroundcap%
\pgfsetroundjoin%
\pgfsetlinewidth{0.803000pt}%
\definecolor{currentstroke}{rgb}{1.000000,1.000000,1.000000}%
\pgfsetstrokecolor{currentstroke}%
\pgfsetdash{}{0pt}%
\pgfpathmoveto{\pgfqpoint{0.375000in}{0.422222in}}%
\pgfpathlineto{\pgfqpoint{2.700000in}{0.422222in}}%
\pgfusepath{stroke}%
\end{pgfscope}%
\begin{pgfscope}%
\pgfsetbuttcap%
\pgfsetroundjoin%
\definecolor{currentfill}{rgb}{0.150000,0.150000,0.150000}%
\pgfsetfillcolor{currentfill}%
\pgfsetlinewidth{0.803000pt}%
\definecolor{currentstroke}{rgb}{0.150000,0.150000,0.150000}%
\pgfsetstrokecolor{currentstroke}%
\pgfsetdash{}{0pt}%
\pgfsys@defobject{currentmarker}{\pgfqpoint{0.000000in}{0.000000in}}{\pgfqpoint{0.000000in}{0.000000in}}{%
\pgfpathmoveto{\pgfqpoint{0.000000in}{0.000000in}}%
\pgfpathlineto{\pgfqpoint{0.000000in}{0.000000in}}%
\pgfusepath{stroke,fill}%
}%
\begin{pgfscope}%
\pgfsys@transformshift{0.375000in}{0.422222in}%
\pgfsys@useobject{currentmarker}{}%
\end{pgfscope}%
\end{pgfscope}%
\begin{pgfscope}%
\pgfsetbuttcap%
\pgfsetroundjoin%
\definecolor{currentfill}{rgb}{0.150000,0.150000,0.150000}%
\pgfsetfillcolor{currentfill}%
\pgfsetlinewidth{0.803000pt}%
\definecolor{currentstroke}{rgb}{0.150000,0.150000,0.150000}%
\pgfsetstrokecolor{currentstroke}%
\pgfsetdash{}{0pt}%
\pgfsys@defobject{currentmarker}{\pgfqpoint{0.000000in}{0.000000in}}{\pgfqpoint{0.000000in}{0.000000in}}{%
\pgfpathmoveto{\pgfqpoint{0.000000in}{0.000000in}}%
\pgfpathlineto{\pgfqpoint{0.000000in}{0.000000in}}%
\pgfusepath{stroke,fill}%
}%
\begin{pgfscope}%
\pgfsys@transformshift{2.700000in}{0.422222in}%
\pgfsys@useobject{currentmarker}{}%
\end{pgfscope}%
\end{pgfscope}%
\begin{pgfscope}%
\definecolor{textcolor}{rgb}{0.150000,0.150000,0.150000}%
\pgfsetstrokecolor{textcolor}%
\pgfsetfillcolor{textcolor}%
\pgftext[x=0.297222in,y=0.422222in,right,]{\color{textcolor}\sffamily\fontsize{8.000000}{9.600000}\selectfont 0}%
\end{pgfscope}%
\begin{pgfscope}%
\pgfpathrectangle{\pgfqpoint{0.375000in}{0.250000in}}{\pgfqpoint{2.325000in}{1.550000in}} %
\pgfusepath{clip}%
\pgfsetroundcap%
\pgfsetroundjoin%
\pgfsetlinewidth{0.803000pt}%
\definecolor{currentstroke}{rgb}{1.000000,1.000000,1.000000}%
\pgfsetstrokecolor{currentstroke}%
\pgfsetdash{}{0pt}%
\pgfpathmoveto{\pgfqpoint{0.375000in}{0.594444in}}%
\pgfpathlineto{\pgfqpoint{2.700000in}{0.594444in}}%
\pgfusepath{stroke}%
\end{pgfscope}%
\begin{pgfscope}%
\pgfsetbuttcap%
\pgfsetroundjoin%
\definecolor{currentfill}{rgb}{0.150000,0.150000,0.150000}%
\pgfsetfillcolor{currentfill}%
\pgfsetlinewidth{0.803000pt}%
\definecolor{currentstroke}{rgb}{0.150000,0.150000,0.150000}%
\pgfsetstrokecolor{currentstroke}%
\pgfsetdash{}{0pt}%
\pgfsys@defobject{currentmarker}{\pgfqpoint{0.000000in}{0.000000in}}{\pgfqpoint{0.000000in}{0.000000in}}{%
\pgfpathmoveto{\pgfqpoint{0.000000in}{0.000000in}}%
\pgfpathlineto{\pgfqpoint{0.000000in}{0.000000in}}%
\pgfusepath{stroke,fill}%
}%
\begin{pgfscope}%
\pgfsys@transformshift{0.375000in}{0.594444in}%
\pgfsys@useobject{currentmarker}{}%
\end{pgfscope}%
\end{pgfscope}%
\begin{pgfscope}%
\pgfsetbuttcap%
\pgfsetroundjoin%
\definecolor{currentfill}{rgb}{0.150000,0.150000,0.150000}%
\pgfsetfillcolor{currentfill}%
\pgfsetlinewidth{0.803000pt}%
\definecolor{currentstroke}{rgb}{0.150000,0.150000,0.150000}%
\pgfsetstrokecolor{currentstroke}%
\pgfsetdash{}{0pt}%
\pgfsys@defobject{currentmarker}{\pgfqpoint{0.000000in}{0.000000in}}{\pgfqpoint{0.000000in}{0.000000in}}{%
\pgfpathmoveto{\pgfqpoint{0.000000in}{0.000000in}}%
\pgfpathlineto{\pgfqpoint{0.000000in}{0.000000in}}%
\pgfusepath{stroke,fill}%
}%
\begin{pgfscope}%
\pgfsys@transformshift{2.700000in}{0.594444in}%
\pgfsys@useobject{currentmarker}{}%
\end{pgfscope}%
\end{pgfscope}%
\begin{pgfscope}%
\definecolor{textcolor}{rgb}{0.150000,0.150000,0.150000}%
\pgfsetstrokecolor{textcolor}%
\pgfsetfillcolor{textcolor}%
\pgftext[x=0.297222in,y=0.594444in,right,]{\color{textcolor}\sffamily\fontsize{8.000000}{9.600000}\selectfont 2}%
\end{pgfscope}%
\begin{pgfscope}%
\pgfpathrectangle{\pgfqpoint{0.375000in}{0.250000in}}{\pgfqpoint{2.325000in}{1.550000in}} %
\pgfusepath{clip}%
\pgfsetroundcap%
\pgfsetroundjoin%
\pgfsetlinewidth{0.803000pt}%
\definecolor{currentstroke}{rgb}{1.000000,1.000000,1.000000}%
\pgfsetstrokecolor{currentstroke}%
\pgfsetdash{}{0pt}%
\pgfpathmoveto{\pgfqpoint{0.375000in}{0.766667in}}%
\pgfpathlineto{\pgfqpoint{2.700000in}{0.766667in}}%
\pgfusepath{stroke}%
\end{pgfscope}%
\begin{pgfscope}%
\pgfsetbuttcap%
\pgfsetroundjoin%
\definecolor{currentfill}{rgb}{0.150000,0.150000,0.150000}%
\pgfsetfillcolor{currentfill}%
\pgfsetlinewidth{0.803000pt}%
\definecolor{currentstroke}{rgb}{0.150000,0.150000,0.150000}%
\pgfsetstrokecolor{currentstroke}%
\pgfsetdash{}{0pt}%
\pgfsys@defobject{currentmarker}{\pgfqpoint{0.000000in}{0.000000in}}{\pgfqpoint{0.000000in}{0.000000in}}{%
\pgfpathmoveto{\pgfqpoint{0.000000in}{0.000000in}}%
\pgfpathlineto{\pgfqpoint{0.000000in}{0.000000in}}%
\pgfusepath{stroke,fill}%
}%
\begin{pgfscope}%
\pgfsys@transformshift{0.375000in}{0.766667in}%
\pgfsys@useobject{currentmarker}{}%
\end{pgfscope}%
\end{pgfscope}%
\begin{pgfscope}%
\pgfsetbuttcap%
\pgfsetroundjoin%
\definecolor{currentfill}{rgb}{0.150000,0.150000,0.150000}%
\pgfsetfillcolor{currentfill}%
\pgfsetlinewidth{0.803000pt}%
\definecolor{currentstroke}{rgb}{0.150000,0.150000,0.150000}%
\pgfsetstrokecolor{currentstroke}%
\pgfsetdash{}{0pt}%
\pgfsys@defobject{currentmarker}{\pgfqpoint{0.000000in}{0.000000in}}{\pgfqpoint{0.000000in}{0.000000in}}{%
\pgfpathmoveto{\pgfqpoint{0.000000in}{0.000000in}}%
\pgfpathlineto{\pgfqpoint{0.000000in}{0.000000in}}%
\pgfusepath{stroke,fill}%
}%
\begin{pgfscope}%
\pgfsys@transformshift{2.700000in}{0.766667in}%
\pgfsys@useobject{currentmarker}{}%
\end{pgfscope}%
\end{pgfscope}%
\begin{pgfscope}%
\definecolor{textcolor}{rgb}{0.150000,0.150000,0.150000}%
\pgfsetstrokecolor{textcolor}%
\pgfsetfillcolor{textcolor}%
\pgftext[x=0.297222in,y=0.766667in,right,]{\color{textcolor}\sffamily\fontsize{8.000000}{9.600000}\selectfont 4}%
\end{pgfscope}%
\begin{pgfscope}%
\pgfpathrectangle{\pgfqpoint{0.375000in}{0.250000in}}{\pgfqpoint{2.325000in}{1.550000in}} %
\pgfusepath{clip}%
\pgfsetroundcap%
\pgfsetroundjoin%
\pgfsetlinewidth{0.803000pt}%
\definecolor{currentstroke}{rgb}{1.000000,1.000000,1.000000}%
\pgfsetstrokecolor{currentstroke}%
\pgfsetdash{}{0pt}%
\pgfpathmoveto{\pgfqpoint{0.375000in}{0.938889in}}%
\pgfpathlineto{\pgfqpoint{2.700000in}{0.938889in}}%
\pgfusepath{stroke}%
\end{pgfscope}%
\begin{pgfscope}%
\pgfsetbuttcap%
\pgfsetroundjoin%
\definecolor{currentfill}{rgb}{0.150000,0.150000,0.150000}%
\pgfsetfillcolor{currentfill}%
\pgfsetlinewidth{0.803000pt}%
\definecolor{currentstroke}{rgb}{0.150000,0.150000,0.150000}%
\pgfsetstrokecolor{currentstroke}%
\pgfsetdash{}{0pt}%
\pgfsys@defobject{currentmarker}{\pgfqpoint{0.000000in}{0.000000in}}{\pgfqpoint{0.000000in}{0.000000in}}{%
\pgfpathmoveto{\pgfqpoint{0.000000in}{0.000000in}}%
\pgfpathlineto{\pgfqpoint{0.000000in}{0.000000in}}%
\pgfusepath{stroke,fill}%
}%
\begin{pgfscope}%
\pgfsys@transformshift{0.375000in}{0.938889in}%
\pgfsys@useobject{currentmarker}{}%
\end{pgfscope}%
\end{pgfscope}%
\begin{pgfscope}%
\pgfsetbuttcap%
\pgfsetroundjoin%
\definecolor{currentfill}{rgb}{0.150000,0.150000,0.150000}%
\pgfsetfillcolor{currentfill}%
\pgfsetlinewidth{0.803000pt}%
\definecolor{currentstroke}{rgb}{0.150000,0.150000,0.150000}%
\pgfsetstrokecolor{currentstroke}%
\pgfsetdash{}{0pt}%
\pgfsys@defobject{currentmarker}{\pgfqpoint{0.000000in}{0.000000in}}{\pgfqpoint{0.000000in}{0.000000in}}{%
\pgfpathmoveto{\pgfqpoint{0.000000in}{0.000000in}}%
\pgfpathlineto{\pgfqpoint{0.000000in}{0.000000in}}%
\pgfusepath{stroke,fill}%
}%
\begin{pgfscope}%
\pgfsys@transformshift{2.700000in}{0.938889in}%
\pgfsys@useobject{currentmarker}{}%
\end{pgfscope}%
\end{pgfscope}%
\begin{pgfscope}%
\definecolor{textcolor}{rgb}{0.150000,0.150000,0.150000}%
\pgfsetstrokecolor{textcolor}%
\pgfsetfillcolor{textcolor}%
\pgftext[x=0.297222in,y=0.938889in,right,]{\color{textcolor}\sffamily\fontsize{8.000000}{9.600000}\selectfont 6}%
\end{pgfscope}%
\begin{pgfscope}%
\pgfpathrectangle{\pgfqpoint{0.375000in}{0.250000in}}{\pgfqpoint{2.325000in}{1.550000in}} %
\pgfusepath{clip}%
\pgfsetroundcap%
\pgfsetroundjoin%
\pgfsetlinewidth{0.803000pt}%
\definecolor{currentstroke}{rgb}{1.000000,1.000000,1.000000}%
\pgfsetstrokecolor{currentstroke}%
\pgfsetdash{}{0pt}%
\pgfpathmoveto{\pgfqpoint{0.375000in}{1.111111in}}%
\pgfpathlineto{\pgfqpoint{2.700000in}{1.111111in}}%
\pgfusepath{stroke}%
\end{pgfscope}%
\begin{pgfscope}%
\pgfsetbuttcap%
\pgfsetroundjoin%
\definecolor{currentfill}{rgb}{0.150000,0.150000,0.150000}%
\pgfsetfillcolor{currentfill}%
\pgfsetlinewidth{0.803000pt}%
\definecolor{currentstroke}{rgb}{0.150000,0.150000,0.150000}%
\pgfsetstrokecolor{currentstroke}%
\pgfsetdash{}{0pt}%
\pgfsys@defobject{currentmarker}{\pgfqpoint{0.000000in}{0.000000in}}{\pgfqpoint{0.000000in}{0.000000in}}{%
\pgfpathmoveto{\pgfqpoint{0.000000in}{0.000000in}}%
\pgfpathlineto{\pgfqpoint{0.000000in}{0.000000in}}%
\pgfusepath{stroke,fill}%
}%
\begin{pgfscope}%
\pgfsys@transformshift{0.375000in}{1.111111in}%
\pgfsys@useobject{currentmarker}{}%
\end{pgfscope}%
\end{pgfscope}%
\begin{pgfscope}%
\pgfsetbuttcap%
\pgfsetroundjoin%
\definecolor{currentfill}{rgb}{0.150000,0.150000,0.150000}%
\pgfsetfillcolor{currentfill}%
\pgfsetlinewidth{0.803000pt}%
\definecolor{currentstroke}{rgb}{0.150000,0.150000,0.150000}%
\pgfsetstrokecolor{currentstroke}%
\pgfsetdash{}{0pt}%
\pgfsys@defobject{currentmarker}{\pgfqpoint{0.000000in}{0.000000in}}{\pgfqpoint{0.000000in}{0.000000in}}{%
\pgfpathmoveto{\pgfqpoint{0.000000in}{0.000000in}}%
\pgfpathlineto{\pgfqpoint{0.000000in}{0.000000in}}%
\pgfusepath{stroke,fill}%
}%
\begin{pgfscope}%
\pgfsys@transformshift{2.700000in}{1.111111in}%
\pgfsys@useobject{currentmarker}{}%
\end{pgfscope}%
\end{pgfscope}%
\begin{pgfscope}%
\definecolor{textcolor}{rgb}{0.150000,0.150000,0.150000}%
\pgfsetstrokecolor{textcolor}%
\pgfsetfillcolor{textcolor}%
\pgftext[x=0.297222in,y=1.111111in,right,]{\color{textcolor}\sffamily\fontsize{8.000000}{9.600000}\selectfont 8}%
\end{pgfscope}%
\begin{pgfscope}%
\pgfpathrectangle{\pgfqpoint{0.375000in}{0.250000in}}{\pgfqpoint{2.325000in}{1.550000in}} %
\pgfusepath{clip}%
\pgfsetroundcap%
\pgfsetroundjoin%
\pgfsetlinewidth{0.803000pt}%
\definecolor{currentstroke}{rgb}{1.000000,1.000000,1.000000}%
\pgfsetstrokecolor{currentstroke}%
\pgfsetdash{}{0pt}%
\pgfpathmoveto{\pgfqpoint{0.375000in}{1.283333in}}%
\pgfpathlineto{\pgfqpoint{2.700000in}{1.283333in}}%
\pgfusepath{stroke}%
\end{pgfscope}%
\begin{pgfscope}%
\pgfsetbuttcap%
\pgfsetroundjoin%
\definecolor{currentfill}{rgb}{0.150000,0.150000,0.150000}%
\pgfsetfillcolor{currentfill}%
\pgfsetlinewidth{0.803000pt}%
\definecolor{currentstroke}{rgb}{0.150000,0.150000,0.150000}%
\pgfsetstrokecolor{currentstroke}%
\pgfsetdash{}{0pt}%
\pgfsys@defobject{currentmarker}{\pgfqpoint{0.000000in}{0.000000in}}{\pgfqpoint{0.000000in}{0.000000in}}{%
\pgfpathmoveto{\pgfqpoint{0.000000in}{0.000000in}}%
\pgfpathlineto{\pgfqpoint{0.000000in}{0.000000in}}%
\pgfusepath{stroke,fill}%
}%
\begin{pgfscope}%
\pgfsys@transformshift{0.375000in}{1.283333in}%
\pgfsys@useobject{currentmarker}{}%
\end{pgfscope}%
\end{pgfscope}%
\begin{pgfscope}%
\pgfsetbuttcap%
\pgfsetroundjoin%
\definecolor{currentfill}{rgb}{0.150000,0.150000,0.150000}%
\pgfsetfillcolor{currentfill}%
\pgfsetlinewidth{0.803000pt}%
\definecolor{currentstroke}{rgb}{0.150000,0.150000,0.150000}%
\pgfsetstrokecolor{currentstroke}%
\pgfsetdash{}{0pt}%
\pgfsys@defobject{currentmarker}{\pgfqpoint{0.000000in}{0.000000in}}{\pgfqpoint{0.000000in}{0.000000in}}{%
\pgfpathmoveto{\pgfqpoint{0.000000in}{0.000000in}}%
\pgfpathlineto{\pgfqpoint{0.000000in}{0.000000in}}%
\pgfusepath{stroke,fill}%
}%
\begin{pgfscope}%
\pgfsys@transformshift{2.700000in}{1.283333in}%
\pgfsys@useobject{currentmarker}{}%
\end{pgfscope}%
\end{pgfscope}%
\begin{pgfscope}%
\definecolor{textcolor}{rgb}{0.150000,0.150000,0.150000}%
\pgfsetstrokecolor{textcolor}%
\pgfsetfillcolor{textcolor}%
\pgftext[x=0.297222in,y=1.283333in,right,]{\color{textcolor}\sffamily\fontsize{8.000000}{9.600000}\selectfont 10}%
\end{pgfscope}%
\begin{pgfscope}%
\pgfpathrectangle{\pgfqpoint{0.375000in}{0.250000in}}{\pgfqpoint{2.325000in}{1.550000in}} %
\pgfusepath{clip}%
\pgfsetroundcap%
\pgfsetroundjoin%
\pgfsetlinewidth{0.803000pt}%
\definecolor{currentstroke}{rgb}{1.000000,1.000000,1.000000}%
\pgfsetstrokecolor{currentstroke}%
\pgfsetdash{}{0pt}%
\pgfpathmoveto{\pgfqpoint{0.375000in}{1.455556in}}%
\pgfpathlineto{\pgfqpoint{2.700000in}{1.455556in}}%
\pgfusepath{stroke}%
\end{pgfscope}%
\begin{pgfscope}%
\pgfsetbuttcap%
\pgfsetroundjoin%
\definecolor{currentfill}{rgb}{0.150000,0.150000,0.150000}%
\pgfsetfillcolor{currentfill}%
\pgfsetlinewidth{0.803000pt}%
\definecolor{currentstroke}{rgb}{0.150000,0.150000,0.150000}%
\pgfsetstrokecolor{currentstroke}%
\pgfsetdash{}{0pt}%
\pgfsys@defobject{currentmarker}{\pgfqpoint{0.000000in}{0.000000in}}{\pgfqpoint{0.000000in}{0.000000in}}{%
\pgfpathmoveto{\pgfqpoint{0.000000in}{0.000000in}}%
\pgfpathlineto{\pgfqpoint{0.000000in}{0.000000in}}%
\pgfusepath{stroke,fill}%
}%
\begin{pgfscope}%
\pgfsys@transformshift{0.375000in}{1.455556in}%
\pgfsys@useobject{currentmarker}{}%
\end{pgfscope}%
\end{pgfscope}%
\begin{pgfscope}%
\pgfsetbuttcap%
\pgfsetroundjoin%
\definecolor{currentfill}{rgb}{0.150000,0.150000,0.150000}%
\pgfsetfillcolor{currentfill}%
\pgfsetlinewidth{0.803000pt}%
\definecolor{currentstroke}{rgb}{0.150000,0.150000,0.150000}%
\pgfsetstrokecolor{currentstroke}%
\pgfsetdash{}{0pt}%
\pgfsys@defobject{currentmarker}{\pgfqpoint{0.000000in}{0.000000in}}{\pgfqpoint{0.000000in}{0.000000in}}{%
\pgfpathmoveto{\pgfqpoint{0.000000in}{0.000000in}}%
\pgfpathlineto{\pgfqpoint{0.000000in}{0.000000in}}%
\pgfusepath{stroke,fill}%
}%
\begin{pgfscope}%
\pgfsys@transformshift{2.700000in}{1.455556in}%
\pgfsys@useobject{currentmarker}{}%
\end{pgfscope}%
\end{pgfscope}%
\begin{pgfscope}%
\definecolor{textcolor}{rgb}{0.150000,0.150000,0.150000}%
\pgfsetstrokecolor{textcolor}%
\pgfsetfillcolor{textcolor}%
\pgftext[x=0.297222in,y=1.455556in,right,]{\color{textcolor}\sffamily\fontsize{8.000000}{9.600000}\selectfont 12}%
\end{pgfscope}%
\begin{pgfscope}%
\pgfpathrectangle{\pgfqpoint{0.375000in}{0.250000in}}{\pgfqpoint{2.325000in}{1.550000in}} %
\pgfusepath{clip}%
\pgfsetroundcap%
\pgfsetroundjoin%
\pgfsetlinewidth{0.803000pt}%
\definecolor{currentstroke}{rgb}{1.000000,1.000000,1.000000}%
\pgfsetstrokecolor{currentstroke}%
\pgfsetdash{}{0pt}%
\pgfpathmoveto{\pgfqpoint{0.375000in}{1.627778in}}%
\pgfpathlineto{\pgfqpoint{2.700000in}{1.627778in}}%
\pgfusepath{stroke}%
\end{pgfscope}%
\begin{pgfscope}%
\pgfsetbuttcap%
\pgfsetroundjoin%
\definecolor{currentfill}{rgb}{0.150000,0.150000,0.150000}%
\pgfsetfillcolor{currentfill}%
\pgfsetlinewidth{0.803000pt}%
\definecolor{currentstroke}{rgb}{0.150000,0.150000,0.150000}%
\pgfsetstrokecolor{currentstroke}%
\pgfsetdash{}{0pt}%
\pgfsys@defobject{currentmarker}{\pgfqpoint{0.000000in}{0.000000in}}{\pgfqpoint{0.000000in}{0.000000in}}{%
\pgfpathmoveto{\pgfqpoint{0.000000in}{0.000000in}}%
\pgfpathlineto{\pgfqpoint{0.000000in}{0.000000in}}%
\pgfusepath{stroke,fill}%
}%
\begin{pgfscope}%
\pgfsys@transformshift{0.375000in}{1.627778in}%
\pgfsys@useobject{currentmarker}{}%
\end{pgfscope}%
\end{pgfscope}%
\begin{pgfscope}%
\pgfsetbuttcap%
\pgfsetroundjoin%
\definecolor{currentfill}{rgb}{0.150000,0.150000,0.150000}%
\pgfsetfillcolor{currentfill}%
\pgfsetlinewidth{0.803000pt}%
\definecolor{currentstroke}{rgb}{0.150000,0.150000,0.150000}%
\pgfsetstrokecolor{currentstroke}%
\pgfsetdash{}{0pt}%
\pgfsys@defobject{currentmarker}{\pgfqpoint{0.000000in}{0.000000in}}{\pgfqpoint{0.000000in}{0.000000in}}{%
\pgfpathmoveto{\pgfqpoint{0.000000in}{0.000000in}}%
\pgfpathlineto{\pgfqpoint{0.000000in}{0.000000in}}%
\pgfusepath{stroke,fill}%
}%
\begin{pgfscope}%
\pgfsys@transformshift{2.700000in}{1.627778in}%
\pgfsys@useobject{currentmarker}{}%
\end{pgfscope}%
\end{pgfscope}%
\begin{pgfscope}%
\definecolor{textcolor}{rgb}{0.150000,0.150000,0.150000}%
\pgfsetstrokecolor{textcolor}%
\pgfsetfillcolor{textcolor}%
\pgftext[x=0.297222in,y=1.627778in,right,]{\color{textcolor}\sffamily\fontsize{8.000000}{9.600000}\selectfont 14}%
\end{pgfscope}%
\begin{pgfscope}%
\pgfpathrectangle{\pgfqpoint{0.375000in}{0.250000in}}{\pgfqpoint{2.325000in}{1.550000in}} %
\pgfusepath{clip}%
\pgfsetroundcap%
\pgfsetroundjoin%
\pgfsetlinewidth{0.803000pt}%
\definecolor{currentstroke}{rgb}{1.000000,1.000000,1.000000}%
\pgfsetstrokecolor{currentstroke}%
\pgfsetdash{}{0pt}%
\pgfpathmoveto{\pgfqpoint{0.375000in}{1.800000in}}%
\pgfpathlineto{\pgfqpoint{2.700000in}{1.800000in}}%
\pgfusepath{stroke}%
\end{pgfscope}%
\begin{pgfscope}%
\pgfsetbuttcap%
\pgfsetroundjoin%
\definecolor{currentfill}{rgb}{0.150000,0.150000,0.150000}%
\pgfsetfillcolor{currentfill}%
\pgfsetlinewidth{0.803000pt}%
\definecolor{currentstroke}{rgb}{0.150000,0.150000,0.150000}%
\pgfsetstrokecolor{currentstroke}%
\pgfsetdash{}{0pt}%
\pgfsys@defobject{currentmarker}{\pgfqpoint{0.000000in}{0.000000in}}{\pgfqpoint{0.000000in}{0.000000in}}{%
\pgfpathmoveto{\pgfqpoint{0.000000in}{0.000000in}}%
\pgfpathlineto{\pgfqpoint{0.000000in}{0.000000in}}%
\pgfusepath{stroke,fill}%
}%
\begin{pgfscope}%
\pgfsys@transformshift{0.375000in}{1.800000in}%
\pgfsys@useobject{currentmarker}{}%
\end{pgfscope}%
\end{pgfscope}%
\begin{pgfscope}%
\pgfsetbuttcap%
\pgfsetroundjoin%
\definecolor{currentfill}{rgb}{0.150000,0.150000,0.150000}%
\pgfsetfillcolor{currentfill}%
\pgfsetlinewidth{0.803000pt}%
\definecolor{currentstroke}{rgb}{0.150000,0.150000,0.150000}%
\pgfsetstrokecolor{currentstroke}%
\pgfsetdash{}{0pt}%
\pgfsys@defobject{currentmarker}{\pgfqpoint{0.000000in}{0.000000in}}{\pgfqpoint{0.000000in}{0.000000in}}{%
\pgfpathmoveto{\pgfqpoint{0.000000in}{0.000000in}}%
\pgfpathlineto{\pgfqpoint{0.000000in}{0.000000in}}%
\pgfusepath{stroke,fill}%
}%
\begin{pgfscope}%
\pgfsys@transformshift{2.700000in}{1.800000in}%
\pgfsys@useobject{currentmarker}{}%
\end{pgfscope}%
\end{pgfscope}%
\begin{pgfscope}%
\definecolor{textcolor}{rgb}{0.150000,0.150000,0.150000}%
\pgfsetstrokecolor{textcolor}%
\pgfsetfillcolor{textcolor}%
\pgftext[x=0.297222in,y=1.800000in,right,]{\color{textcolor}\sffamily\fontsize{8.000000}{9.600000}\selectfont 16}%
\end{pgfscope}%
\begin{pgfscope}%
\pgfpathrectangle{\pgfqpoint{0.375000in}{0.250000in}}{\pgfqpoint{2.325000in}{1.550000in}} %
\pgfusepath{clip}%
\pgfsetbuttcap%
\pgfsetmiterjoin%
\definecolor{currentfill}{rgb}{0.447059,0.623529,0.811765}%
\pgfsetfillcolor{currentfill}%
\pgfsetfillopacity{0.300000}%
\pgfsetlinewidth{0.240900pt}%
\definecolor{currentstroke}{rgb}{0.447059,0.623529,0.811765}%
\pgfsetstrokecolor{currentstroke}%
\pgfsetstrokeopacity{0.300000}%
\pgfsetdash{}{0pt}%
\pgfpathmoveto{\pgfqpoint{0.484620in}{0.555104in}}%
\pgfpathlineto{\pgfqpoint{0.660687in}{0.661583in}}%
\pgfpathlineto{\pgfqpoint{0.682136in}{0.691824in}}%
\pgfpathlineto{\pgfqpoint{0.726616in}{0.585133in}}%
\pgfpathlineto{\pgfqpoint{0.786616in}{0.641642in}}%
\pgfpathlineto{\pgfqpoint{0.805072in}{0.622762in}}%
\pgfpathlineto{\pgfqpoint{0.814014in}{0.661083in}}%
\pgfpathlineto{\pgfqpoint{0.833078in}{0.746612in}}%
\pgfpathlineto{\pgfqpoint{0.895876in}{0.808104in}}%
\pgfpathlineto{\pgfqpoint{1.013910in}{0.734015in}}%
\pgfpathlineto{\pgfqpoint{1.195378in}{0.846398in}}%
\pgfpathlineto{\pgfqpoint{1.206225in}{0.851095in}}%
\pgfpathlineto{\pgfqpoint{1.232895in}{0.830596in}}%
\pgfpathlineto{\pgfqpoint{1.351711in}{1.021362in}}%
\pgfpathlineto{\pgfqpoint{1.640943in}{1.068985in}}%
\pgfpathlineto{\pgfqpoint{1.783095in}{1.256284in}}%
\pgfpathlineto{\pgfqpoint{1.820691in}{1.183154in}}%
\pgfpathlineto{\pgfqpoint{2.284749in}{1.414872in}}%
\pgfpathlineto{\pgfqpoint{2.426526in}{1.634781in}}%
\pgfpathlineto{\pgfqpoint{2.433853in}{1.639340in}}%
\pgfpathlineto{\pgfqpoint{2.433853in}{1.493617in}}%
\pgfpathlineto{\pgfqpoint{2.426526in}{1.482185in}}%
\pgfpathlineto{\pgfqpoint{2.284749in}{1.265046in}}%
\pgfpathlineto{\pgfqpoint{1.820691in}{1.135039in}}%
\pgfpathlineto{\pgfqpoint{1.783095in}{1.108331in}}%
\pgfpathlineto{\pgfqpoint{1.640943in}{0.923436in}}%
\pgfpathlineto{\pgfqpoint{1.351711in}{0.873877in}}%
\pgfpathlineto{\pgfqpoint{1.232895in}{0.787899in}}%
\pgfpathlineto{\pgfqpoint{1.206225in}{0.701966in}}%
\pgfpathlineto{\pgfqpoint{1.195378in}{0.701820in}}%
\pgfpathlineto{\pgfqpoint{1.013910in}{0.670659in}}%
\pgfpathlineto{\pgfqpoint{0.895876in}{0.699439in}}%
\pgfpathlineto{\pgfqpoint{0.833078in}{0.607012in}}%
\pgfpathlineto{\pgfqpoint{0.814014in}{0.607579in}}%
\pgfpathlineto{\pgfqpoint{0.805072in}{0.609263in}}%
\pgfpathlineto{\pgfqpoint{0.786616in}{0.520601in}}%
\pgfpathlineto{\pgfqpoint{0.726616in}{0.525336in}}%
\pgfpathlineto{\pgfqpoint{0.682136in}{0.574144in}}%
\pgfpathlineto{\pgfqpoint{0.660687in}{0.644534in}}%
\pgfpathlineto{\pgfqpoint{0.484620in}{0.404665in}}%
\pgfpathclose%
\pgfusepath{stroke,fill}%
\end{pgfscope}%
\begin{pgfscope}%
\pgfpathrectangle{\pgfqpoint{0.375000in}{0.250000in}}{\pgfqpoint{2.325000in}{1.550000in}} %
\pgfusepath{clip}%
\pgfsetroundcap%
\pgfsetroundjoin%
\pgfsetlinewidth{2.007500pt}%
\definecolor{currentstroke}{rgb}{0.125490,0.290196,0.529412}%
\pgfsetstrokecolor{currentstroke}%
\pgfsetdash{}{0pt}%
\pgfpathmoveto{\pgfqpoint{0.484620in}{0.479884in}}%
\pgfpathlineto{\pgfqpoint{0.660687in}{0.653059in}}%
\pgfpathlineto{\pgfqpoint{0.682136in}{0.632984in}}%
\pgfpathlineto{\pgfqpoint{0.726616in}{0.555235in}}%
\pgfpathlineto{\pgfqpoint{0.786616in}{0.581121in}}%
\pgfpathlineto{\pgfqpoint{0.805072in}{0.616013in}}%
\pgfpathlineto{\pgfqpoint{0.814014in}{0.634331in}}%
\pgfpathlineto{\pgfqpoint{0.833078in}{0.676812in}}%
\pgfpathlineto{\pgfqpoint{0.895876in}{0.753771in}}%
\pgfpathlineto{\pgfqpoint{1.013910in}{0.702337in}}%
\pgfpathlineto{\pgfqpoint{1.195378in}{0.774109in}}%
\pgfpathlineto{\pgfqpoint{1.206225in}{0.776530in}}%
\pgfpathlineto{\pgfqpoint{1.232895in}{0.809247in}}%
\pgfpathlineto{\pgfqpoint{1.351711in}{0.947619in}}%
\pgfpathlineto{\pgfqpoint{1.640943in}{0.996210in}}%
\pgfpathlineto{\pgfqpoint{1.783095in}{1.182308in}}%
\pgfpathlineto{\pgfqpoint{1.820691in}{1.159096in}}%
\pgfpathlineto{\pgfqpoint{2.284749in}{1.339959in}}%
\pgfpathlineto{\pgfqpoint{2.426526in}{1.558483in}}%
\pgfpathlineto{\pgfqpoint{2.433853in}{1.566479in}}%
\pgfusepath{stroke}%
\end{pgfscope}%
\begin{pgfscope}%
\pgfpathrectangle{\pgfqpoint{0.375000in}{0.250000in}}{\pgfqpoint{2.325000in}{1.550000in}} %
\pgfusepath{clip}%
\pgfsetbuttcap%
\pgfsetbeveljoin%
\definecolor{currentfill}{rgb}{0.125490,0.290196,0.529412}%
\pgfsetfillcolor{currentfill}%
\pgfsetlinewidth{0.000000pt}%
\definecolor{currentstroke}{rgb}{0.000000,0.000000,0.000000}%
\pgfsetstrokecolor{currentstroke}%
\pgfsetdash{}{0pt}%
\pgfsys@defobject{currentmarker}{\pgfqpoint{-0.036986in}{-0.031462in}}{\pgfqpoint{0.036986in}{0.038889in}}{%
\pgfpathmoveto{\pgfqpoint{0.000000in}{0.038889in}}%
\pgfpathlineto{\pgfqpoint{-0.008731in}{0.012017in}}%
\pgfpathlineto{\pgfqpoint{-0.036986in}{0.012017in}}%
\pgfpathlineto{\pgfqpoint{-0.014127in}{-0.004590in}}%
\pgfpathlineto{\pgfqpoint{-0.022858in}{-0.031462in}}%
\pgfpathlineto{\pgfqpoint{-0.000000in}{-0.014854in}}%
\pgfpathlineto{\pgfqpoint{0.022858in}{-0.031462in}}%
\pgfpathlineto{\pgfqpoint{0.014127in}{-0.004590in}}%
\pgfpathlineto{\pgfqpoint{0.036986in}{0.012017in}}%
\pgfpathlineto{\pgfqpoint{0.008731in}{0.012017in}}%
\pgfpathclose%
\pgfusepath{fill}%
}%
\begin{pgfscope}%
\pgfsys@transformshift{0.484620in}{0.479884in}%
\pgfsys@useobject{currentmarker}{}%
\end{pgfscope}%
\begin{pgfscope}%
\pgfsys@transformshift{0.660687in}{0.653059in}%
\pgfsys@useobject{currentmarker}{}%
\end{pgfscope}%
\begin{pgfscope}%
\pgfsys@transformshift{0.682136in}{0.632984in}%
\pgfsys@useobject{currentmarker}{}%
\end{pgfscope}%
\begin{pgfscope}%
\pgfsys@transformshift{0.726616in}{0.555235in}%
\pgfsys@useobject{currentmarker}{}%
\end{pgfscope}%
\begin{pgfscope}%
\pgfsys@transformshift{0.786616in}{0.581121in}%
\pgfsys@useobject{currentmarker}{}%
\end{pgfscope}%
\begin{pgfscope}%
\pgfsys@transformshift{0.805072in}{0.616013in}%
\pgfsys@useobject{currentmarker}{}%
\end{pgfscope}%
\begin{pgfscope}%
\pgfsys@transformshift{0.814014in}{0.634331in}%
\pgfsys@useobject{currentmarker}{}%
\end{pgfscope}%
\begin{pgfscope}%
\pgfsys@transformshift{0.833078in}{0.676812in}%
\pgfsys@useobject{currentmarker}{}%
\end{pgfscope}%
\begin{pgfscope}%
\pgfsys@transformshift{0.895876in}{0.753771in}%
\pgfsys@useobject{currentmarker}{}%
\end{pgfscope}%
\begin{pgfscope}%
\pgfsys@transformshift{1.013910in}{0.702337in}%
\pgfsys@useobject{currentmarker}{}%
\end{pgfscope}%
\begin{pgfscope}%
\pgfsys@transformshift{1.195378in}{0.774109in}%
\pgfsys@useobject{currentmarker}{}%
\end{pgfscope}%
\begin{pgfscope}%
\pgfsys@transformshift{1.206225in}{0.776530in}%
\pgfsys@useobject{currentmarker}{}%
\end{pgfscope}%
\begin{pgfscope}%
\pgfsys@transformshift{1.232895in}{0.809247in}%
\pgfsys@useobject{currentmarker}{}%
\end{pgfscope}%
\begin{pgfscope}%
\pgfsys@transformshift{1.351711in}{0.947619in}%
\pgfsys@useobject{currentmarker}{}%
\end{pgfscope}%
\begin{pgfscope}%
\pgfsys@transformshift{1.640943in}{0.996210in}%
\pgfsys@useobject{currentmarker}{}%
\end{pgfscope}%
\begin{pgfscope}%
\pgfsys@transformshift{1.783095in}{1.182308in}%
\pgfsys@useobject{currentmarker}{}%
\end{pgfscope}%
\begin{pgfscope}%
\pgfsys@transformshift{1.820691in}{1.159096in}%
\pgfsys@useobject{currentmarker}{}%
\end{pgfscope}%
\begin{pgfscope}%
\pgfsys@transformshift{2.284749in}{1.339959in}%
\pgfsys@useobject{currentmarker}{}%
\end{pgfscope}%
\begin{pgfscope}%
\pgfsys@transformshift{2.426526in}{1.558483in}%
\pgfsys@useobject{currentmarker}{}%
\end{pgfscope}%
\begin{pgfscope}%
\pgfsys@transformshift{2.433853in}{1.566479in}%
\pgfsys@useobject{currentmarker}{}%
\end{pgfscope}%
\end{pgfscope}%
\begin{pgfscope}%
\pgfpathrectangle{\pgfqpoint{0.375000in}{0.250000in}}{\pgfqpoint{2.325000in}{1.550000in}} %
\pgfusepath{clip}%
\pgfsetbuttcap%
\pgfsetbeveljoin%
\definecolor{currentfill}{rgb}{1.000000,0.000000,0.000000}%
\pgfsetfillcolor{currentfill}%
\pgfsetlinewidth{0.000000pt}%
\definecolor{currentstroke}{rgb}{0.000000,0.000000,0.000000}%
\pgfsetstrokecolor{currentstroke}%
\pgfsetdash{}{0pt}%
\pgfsys@defobject{currentmarker}{\pgfqpoint{-0.036986in}{-0.031462in}}{\pgfqpoint{0.036986in}{0.038889in}}{%
\pgfpathmoveto{\pgfqpoint{0.000000in}{0.038889in}}%
\pgfpathlineto{\pgfqpoint{-0.008731in}{0.012017in}}%
\pgfpathlineto{\pgfqpoint{-0.036986in}{0.012017in}}%
\pgfpathlineto{\pgfqpoint{-0.014127in}{-0.004590in}}%
\pgfpathlineto{\pgfqpoint{-0.022858in}{-0.031462in}}%
\pgfpathlineto{\pgfqpoint{-0.000000in}{-0.014854in}}%
\pgfpathlineto{\pgfqpoint{0.022858in}{-0.031462in}}%
\pgfpathlineto{\pgfqpoint{0.014127in}{-0.004590in}}%
\pgfpathlineto{\pgfqpoint{0.036986in}{0.012017in}}%
\pgfpathlineto{\pgfqpoint{0.008731in}{0.012017in}}%
\pgfpathclose%
\pgfusepath{fill}%
}%
\begin{pgfscope}%
\pgfsys@transformshift{0.484620in}{0.479884in}%
\pgfsys@useobject{currentmarker}{}%
\end{pgfscope}%
\begin{pgfscope}%
\pgfsys@transformshift{0.660687in}{0.653059in}%
\pgfsys@useobject{currentmarker}{}%
\end{pgfscope}%
\begin{pgfscope}%
\pgfsys@transformshift{0.682136in}{0.632984in}%
\pgfsys@useobject{currentmarker}{}%
\end{pgfscope}%
\begin{pgfscope}%
\pgfsys@transformshift{0.726616in}{0.555235in}%
\pgfsys@useobject{currentmarker}{}%
\end{pgfscope}%
\begin{pgfscope}%
\pgfsys@transformshift{0.786616in}{0.581121in}%
\pgfsys@useobject{currentmarker}{}%
\end{pgfscope}%
\begin{pgfscope}%
\pgfsys@transformshift{0.805072in}{0.616013in}%
\pgfsys@useobject{currentmarker}{}%
\end{pgfscope}%
\begin{pgfscope}%
\pgfsys@transformshift{0.814014in}{0.634331in}%
\pgfsys@useobject{currentmarker}{}%
\end{pgfscope}%
\begin{pgfscope}%
\pgfsys@transformshift{0.833078in}{0.676812in}%
\pgfsys@useobject{currentmarker}{}%
\end{pgfscope}%
\begin{pgfscope}%
\pgfsys@transformshift{0.895876in}{0.753771in}%
\pgfsys@useobject{currentmarker}{}%
\end{pgfscope}%
\begin{pgfscope}%
\pgfsys@transformshift{1.013910in}{0.702337in}%
\pgfsys@useobject{currentmarker}{}%
\end{pgfscope}%
\begin{pgfscope}%
\pgfsys@transformshift{1.195378in}{0.774109in}%
\pgfsys@useobject{currentmarker}{}%
\end{pgfscope}%
\begin{pgfscope}%
\pgfsys@transformshift{1.206225in}{0.776530in}%
\pgfsys@useobject{currentmarker}{}%
\end{pgfscope}%
\begin{pgfscope}%
\pgfsys@transformshift{1.232895in}{0.809247in}%
\pgfsys@useobject{currentmarker}{}%
\end{pgfscope}%
\begin{pgfscope}%
\pgfsys@transformshift{1.351711in}{0.947619in}%
\pgfsys@useobject{currentmarker}{}%
\end{pgfscope}%
\begin{pgfscope}%
\pgfsys@transformshift{1.640943in}{0.996210in}%
\pgfsys@useobject{currentmarker}{}%
\end{pgfscope}%
\begin{pgfscope}%
\pgfsys@transformshift{1.783095in}{1.182308in}%
\pgfsys@useobject{currentmarker}{}%
\end{pgfscope}%
\begin{pgfscope}%
\pgfsys@transformshift{1.820691in}{1.159096in}%
\pgfsys@useobject{currentmarker}{}%
\end{pgfscope}%
\begin{pgfscope}%
\pgfsys@transformshift{2.284749in}{1.339959in}%
\pgfsys@useobject{currentmarker}{}%
\end{pgfscope}%
\begin{pgfscope}%
\pgfsys@transformshift{2.426526in}{1.558483in}%
\pgfsys@useobject{currentmarker}{}%
\end{pgfscope}%
\begin{pgfscope}%
\pgfsys@transformshift{2.433853in}{1.566479in}%
\pgfsys@useobject{currentmarker}{}%
\end{pgfscope}%
\end{pgfscope}%
\begin{pgfscope}%
\pgfpathrectangle{\pgfqpoint{0.375000in}{0.250000in}}{\pgfqpoint{2.325000in}{1.550000in}} %
\pgfusepath{clip}%
\pgfsetroundcap%
\pgfsetroundjoin%
\pgfsetlinewidth{0.200750pt}%
\definecolor{currentstroke}{rgb}{0.125490,0.290196,0.529412}%
\pgfsetstrokecolor{currentstroke}%
\pgfsetdash{}{0pt}%
\pgfpathmoveto{\pgfqpoint{0.484620in}{0.555104in}}%
\pgfpathlineto{\pgfqpoint{0.660687in}{0.661583in}}%
\pgfpathlineto{\pgfqpoint{0.682136in}{0.691824in}}%
\pgfpathlineto{\pgfqpoint{0.726616in}{0.585133in}}%
\pgfpathlineto{\pgfqpoint{0.786616in}{0.641642in}}%
\pgfpathlineto{\pgfqpoint{0.805072in}{0.622762in}}%
\pgfpathlineto{\pgfqpoint{0.814014in}{0.661083in}}%
\pgfpathlineto{\pgfqpoint{0.833078in}{0.746612in}}%
\pgfpathlineto{\pgfqpoint{0.895876in}{0.808104in}}%
\pgfpathlineto{\pgfqpoint{1.013910in}{0.734015in}}%
\pgfpathlineto{\pgfqpoint{1.195378in}{0.846398in}}%
\pgfpathlineto{\pgfqpoint{1.206225in}{0.851095in}}%
\pgfpathlineto{\pgfqpoint{1.232895in}{0.830596in}}%
\pgfpathlineto{\pgfqpoint{1.351711in}{1.021362in}}%
\pgfpathlineto{\pgfqpoint{1.640943in}{1.068985in}}%
\pgfpathlineto{\pgfqpoint{1.783095in}{1.256284in}}%
\pgfpathlineto{\pgfqpoint{1.820691in}{1.183154in}}%
\pgfpathlineto{\pgfqpoint{2.284749in}{1.414872in}}%
\pgfpathlineto{\pgfqpoint{2.426526in}{1.634781in}}%
\pgfpathlineto{\pgfqpoint{2.433853in}{1.639340in}}%
\pgfusepath{stroke}%
\end{pgfscope}%
\begin{pgfscope}%
\pgfpathrectangle{\pgfqpoint{0.375000in}{0.250000in}}{\pgfqpoint{2.325000in}{1.550000in}} %
\pgfusepath{clip}%
\pgfsetroundcap%
\pgfsetroundjoin%
\pgfsetlinewidth{0.200750pt}%
\definecolor{currentstroke}{rgb}{0.125490,0.290196,0.529412}%
\pgfsetstrokecolor{currentstroke}%
\pgfsetdash{}{0pt}%
\pgfpathmoveto{\pgfqpoint{0.484620in}{0.404665in}}%
\pgfpathlineto{\pgfqpoint{0.660687in}{0.644534in}}%
\pgfpathlineto{\pgfqpoint{0.682136in}{0.574144in}}%
\pgfpathlineto{\pgfqpoint{0.726616in}{0.525336in}}%
\pgfpathlineto{\pgfqpoint{0.786616in}{0.520601in}}%
\pgfpathlineto{\pgfqpoint{0.805072in}{0.609263in}}%
\pgfpathlineto{\pgfqpoint{0.814014in}{0.607579in}}%
\pgfpathlineto{\pgfqpoint{0.833078in}{0.607012in}}%
\pgfpathlineto{\pgfqpoint{0.895876in}{0.699439in}}%
\pgfpathlineto{\pgfqpoint{1.013910in}{0.670659in}}%
\pgfpathlineto{\pgfqpoint{1.195378in}{0.701820in}}%
\pgfpathlineto{\pgfqpoint{1.206225in}{0.701966in}}%
\pgfpathlineto{\pgfqpoint{1.232895in}{0.787899in}}%
\pgfpathlineto{\pgfqpoint{1.351711in}{0.873877in}}%
\pgfpathlineto{\pgfqpoint{1.640943in}{0.923436in}}%
\pgfpathlineto{\pgfqpoint{1.783095in}{1.108331in}}%
\pgfpathlineto{\pgfqpoint{1.820691in}{1.135039in}}%
\pgfpathlineto{\pgfqpoint{2.284749in}{1.265046in}}%
\pgfpathlineto{\pgfqpoint{2.426526in}{1.482185in}}%
\pgfpathlineto{\pgfqpoint{2.433853in}{1.493617in}}%
\pgfusepath{stroke}%
\end{pgfscope}%
\begin{pgfscope}%
\pgfsetrectcap%
\pgfsetmiterjoin%
\pgfsetlinewidth{0.000000pt}%
\definecolor{currentstroke}{rgb}{1.000000,1.000000,1.000000}%
\pgfsetstrokecolor{currentstroke}%
\pgfsetdash{}{0pt}%
\pgfpathmoveto{\pgfqpoint{0.375000in}{0.250000in}}%
\pgfpathlineto{\pgfqpoint{2.700000in}{0.250000in}}%
\pgfusepath{}%
\end{pgfscope}%
\begin{pgfscope}%
\pgfsetrectcap%
\pgfsetmiterjoin%
\pgfsetlinewidth{0.000000pt}%
\definecolor{currentstroke}{rgb}{1.000000,1.000000,1.000000}%
\pgfsetstrokecolor{currentstroke}%
\pgfsetdash{}{0pt}%
\pgfpathmoveto{\pgfqpoint{0.375000in}{0.250000in}}%
\pgfpathlineto{\pgfqpoint{0.375000in}{1.800000in}}%
\pgfusepath{}%
\end{pgfscope}%
\begin{pgfscope}%
\pgfsetrectcap%
\pgfsetmiterjoin%
\pgfsetlinewidth{0.000000pt}%
\definecolor{currentstroke}{rgb}{1.000000,1.000000,1.000000}%
\pgfsetstrokecolor{currentstroke}%
\pgfsetdash{}{0pt}%
\pgfpathmoveto{\pgfqpoint{0.375000in}{1.800000in}}%
\pgfpathlineto{\pgfqpoint{2.700000in}{1.800000in}}%
\pgfusepath{}%
\end{pgfscope}%
\begin{pgfscope}%
\pgfsetrectcap%
\pgfsetmiterjoin%
\pgfsetlinewidth{0.000000pt}%
\definecolor{currentstroke}{rgb}{1.000000,1.000000,1.000000}%
\pgfsetstrokecolor{currentstroke}%
\pgfsetdash{}{0pt}%
\pgfpathmoveto{\pgfqpoint{2.700000in}{0.250000in}}%
\pgfpathlineto{\pgfqpoint{2.700000in}{1.800000in}}%
\pgfusepath{}%
\end{pgfscope}%
\end{pgfpicture}%
\makeatother%
\endgroup%

    \caption{Predictions for model trained with $30$ samples.}
    \label{fig_predftest2}
  \end{subfigure}
  \caption{GP predictions for data sampled from $f(x) = \sin(3\pi x)$ of $20$
   points (in red) using a \emph{Matern32} kernel.}
  \label{fig_gppredftest}
\end{figure}
