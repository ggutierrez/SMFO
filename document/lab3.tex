% -*- root: main.tex -*-
\section{Lab session 3}

In the previous section several designs of experiments (Doe) were implemented and analyzed in order to determine which was the best given some quality measures. Thus, for this case a latin hypercube design was used to generate 40 experiments which corresponds to one helicopter. Each helicopter was thrown 2 times from 5 to 6 meters height, 2 falling times for each throw were properly registered in seconds by different subjects.

\subsection{Analyzing Data}

The collected data was further analyzed in order to determine the observation noise, and other correlations of the variables used by the model. Inititally the 4 falling times were plotted, in fig \ref{fig_EX1_EX1} and \ref{fig_EX1_EX2} are illustrated two scatter plots wich represents the measurements from two observers for the same throw and from the same observer for different throws. Between different observers there is not a lot of noise as in the measurement for different throws. despite of that fact, in a first setting a Gaussian Process model was fitted using only two falling times, the average time of the first throw and the average time of the second throw.

\paragraph{}
In that setting the leave one out function had to be modified in order to retrieve the two observations for a same set of parameters or helicopter, then a $Q2$ quality measure was used to measure the prediction power of the proposed model, however the $Q2$ was very low, aproximately 0.1, the recommended value to state that the model was correct was at least 0.5, the model is better when the $Q2$ is close to one.

\paragraph{}
One of the assumptions for the $Q2$ to not be as high as expected was that the gathered data was very noisy and didn't expose the correlations between some of the variables and the falling time, according to some prior assumptions made, for example, that the wing angle an the falling time were linear increasingly correlated, the wing length to tail ratio was also expected to have a linear correlation with falling time. In fig ure
\ref{fig_wtr_vs_obs2} the linear correlation between the ratio and the falling time is not as clear as in the figure \ref{fig_wtr_vs_avg4}. While in the first figure only the measurment of one observer was ploted in the second figure the average of all four measurements was taken, exposing a better linear correlation, as expected.



\begin{figure}
	\begin{subfigure}[h]{.5\linewidth}
		%% Creator: Matplotlib, PGF backend
%%
%% To include the figure in your LaTeX document, write
%%   \input{<filename>.pgf}
%%
%% Make sure the required packages are loaded in your preamble
%%   \usepackage{pgf}
%%
%% Figures using additional raster images can only be included by \input if
%% they are in the same directory as the main LaTeX file. For loading figures
%% from other directories you can use the `import` package
%%   \usepackage{import}
%% and then include the figures with
%%   \import{<path to file>}{<filename>.pgf}
%%
%% Matplotlib used the following preamble
%%   \usepackage[utf8x]{inputenc}
%%   \usepackage[T1]{fontenc}
%%   \usepackage{cmbright}
%%
\begingroup%
\makeatletter%
\begin{pgfpicture}%
\pgfpathrectangle{\pgfpointorigin}{\pgfqpoint{2.500000in}{2.500000in}}%
\pgfusepath{use as bounding box, clip}%
\begin{pgfscope}%
\pgfsetbuttcap%
\pgfsetmiterjoin%
\definecolor{currentfill}{rgb}{1.000000,1.000000,1.000000}%
\pgfsetfillcolor{currentfill}%
\pgfsetlinewidth{0.000000pt}%
\definecolor{currentstroke}{rgb}{1.000000,1.000000,1.000000}%
\pgfsetstrokecolor{currentstroke}%
\pgfsetdash{}{0pt}%
\pgfpathmoveto{\pgfqpoint{0.000000in}{0.000000in}}%
\pgfpathlineto{\pgfqpoint{2.500000in}{0.000000in}}%
\pgfpathlineto{\pgfqpoint{2.500000in}{2.500000in}}%
\pgfpathlineto{\pgfqpoint{0.000000in}{2.500000in}}%
\pgfpathclose%
\pgfusepath{fill}%
\end{pgfscope}%
\begin{pgfscope}%
\pgfsetbuttcap%
\pgfsetmiterjoin%
\definecolor{currentfill}{rgb}{0.917647,0.917647,0.949020}%
\pgfsetfillcolor{currentfill}%
\pgfsetlinewidth{0.000000pt}%
\definecolor{currentstroke}{rgb}{0.000000,0.000000,0.000000}%
\pgfsetstrokecolor{currentstroke}%
\pgfsetstrokeopacity{0.000000}%
\pgfsetdash{}{0pt}%
\pgfpathmoveto{\pgfqpoint{0.556847in}{0.516222in}}%
\pgfpathlineto{\pgfqpoint{2.279437in}{0.516222in}}%
\pgfpathlineto{\pgfqpoint{2.279437in}{2.299750in}}%
\pgfpathlineto{\pgfqpoint{0.556847in}{2.299750in}}%
\pgfpathclose%
\pgfusepath{fill}%
\end{pgfscope}%
\begin{pgfscope}%
\pgfpathrectangle{\pgfqpoint{0.556847in}{0.516222in}}{\pgfqpoint{1.722590in}{1.783528in}} %
\pgfusepath{clip}%
\pgfsetroundcap%
\pgfsetroundjoin%
\pgfsetlinewidth{0.803000pt}%
\definecolor{currentstroke}{rgb}{1.000000,1.000000,1.000000}%
\pgfsetstrokecolor{currentstroke}%
\pgfsetdash{}{0pt}%
\pgfpathmoveto{\pgfqpoint{0.556847in}{0.516222in}}%
\pgfpathlineto{\pgfqpoint{0.556847in}{2.299750in}}%
\pgfusepath{stroke}%
\end{pgfscope}%
\begin{pgfscope}%
\pgfsetbuttcap%
\pgfsetroundjoin%
\definecolor{currentfill}{rgb}{0.150000,0.150000,0.150000}%
\pgfsetfillcolor{currentfill}%
\pgfsetlinewidth{0.803000pt}%
\definecolor{currentstroke}{rgb}{0.150000,0.150000,0.150000}%
\pgfsetstrokecolor{currentstroke}%
\pgfsetdash{}{0pt}%
\pgfsys@defobject{currentmarker}{\pgfqpoint{0.000000in}{0.000000in}}{\pgfqpoint{0.000000in}{0.000000in}}{%
\pgfpathmoveto{\pgfqpoint{0.000000in}{0.000000in}}%
\pgfpathlineto{\pgfqpoint{0.000000in}{0.000000in}}%
\pgfusepath{stroke,fill}%
}%
\begin{pgfscope}%
\pgfsys@transformshift{0.556847in}{0.516222in}%
\pgfsys@useobject{currentmarker}{}%
\end{pgfscope}%
\end{pgfscope}%
\begin{pgfscope}%
\definecolor{textcolor}{rgb}{0.150000,0.150000,0.150000}%
\pgfsetstrokecolor{textcolor}%
\pgfsetfillcolor{textcolor}%
\pgftext[x=0.556847in,y=0.438444in,,top]{\color{textcolor}\sffamily\fontsize{8.000000}{9.600000}\selectfont 2.0}%
\end{pgfscope}%
\begin{pgfscope}%
\pgfpathrectangle{\pgfqpoint{0.556847in}{0.516222in}}{\pgfqpoint{1.722590in}{1.783528in}} %
\pgfusepath{clip}%
\pgfsetroundcap%
\pgfsetroundjoin%
\pgfsetlinewidth{0.803000pt}%
\definecolor{currentstroke}{rgb}{1.000000,1.000000,1.000000}%
\pgfsetstrokecolor{currentstroke}%
\pgfsetdash{}{0pt}%
\pgfpathmoveto{\pgfqpoint{0.748246in}{0.516222in}}%
\pgfpathlineto{\pgfqpoint{0.748246in}{2.299750in}}%
\pgfusepath{stroke}%
\end{pgfscope}%
\begin{pgfscope}%
\pgfsetbuttcap%
\pgfsetroundjoin%
\definecolor{currentfill}{rgb}{0.150000,0.150000,0.150000}%
\pgfsetfillcolor{currentfill}%
\pgfsetlinewidth{0.803000pt}%
\definecolor{currentstroke}{rgb}{0.150000,0.150000,0.150000}%
\pgfsetstrokecolor{currentstroke}%
\pgfsetdash{}{0pt}%
\pgfsys@defobject{currentmarker}{\pgfqpoint{0.000000in}{0.000000in}}{\pgfqpoint{0.000000in}{0.000000in}}{%
\pgfpathmoveto{\pgfqpoint{0.000000in}{0.000000in}}%
\pgfpathlineto{\pgfqpoint{0.000000in}{0.000000in}}%
\pgfusepath{stroke,fill}%
}%
\begin{pgfscope}%
\pgfsys@transformshift{0.748246in}{0.516222in}%
\pgfsys@useobject{currentmarker}{}%
\end{pgfscope}%
\end{pgfscope}%
\begin{pgfscope}%
\definecolor{textcolor}{rgb}{0.150000,0.150000,0.150000}%
\pgfsetstrokecolor{textcolor}%
\pgfsetfillcolor{textcolor}%
\pgftext[x=0.748246in,y=0.438444in,,top]{\color{textcolor}\sffamily\fontsize{8.000000}{9.600000}\selectfont 2.5}%
\end{pgfscope}%
\begin{pgfscope}%
\pgfpathrectangle{\pgfqpoint{0.556847in}{0.516222in}}{\pgfqpoint{1.722590in}{1.783528in}} %
\pgfusepath{clip}%
\pgfsetroundcap%
\pgfsetroundjoin%
\pgfsetlinewidth{0.803000pt}%
\definecolor{currentstroke}{rgb}{1.000000,1.000000,1.000000}%
\pgfsetstrokecolor{currentstroke}%
\pgfsetdash{}{0pt}%
\pgfpathmoveto{\pgfqpoint{0.939645in}{0.516222in}}%
\pgfpathlineto{\pgfqpoint{0.939645in}{2.299750in}}%
\pgfusepath{stroke}%
\end{pgfscope}%
\begin{pgfscope}%
\pgfsetbuttcap%
\pgfsetroundjoin%
\definecolor{currentfill}{rgb}{0.150000,0.150000,0.150000}%
\pgfsetfillcolor{currentfill}%
\pgfsetlinewidth{0.803000pt}%
\definecolor{currentstroke}{rgb}{0.150000,0.150000,0.150000}%
\pgfsetstrokecolor{currentstroke}%
\pgfsetdash{}{0pt}%
\pgfsys@defobject{currentmarker}{\pgfqpoint{0.000000in}{0.000000in}}{\pgfqpoint{0.000000in}{0.000000in}}{%
\pgfpathmoveto{\pgfqpoint{0.000000in}{0.000000in}}%
\pgfpathlineto{\pgfqpoint{0.000000in}{0.000000in}}%
\pgfusepath{stroke,fill}%
}%
\begin{pgfscope}%
\pgfsys@transformshift{0.939645in}{0.516222in}%
\pgfsys@useobject{currentmarker}{}%
\end{pgfscope}%
\end{pgfscope}%
\begin{pgfscope}%
\definecolor{textcolor}{rgb}{0.150000,0.150000,0.150000}%
\pgfsetstrokecolor{textcolor}%
\pgfsetfillcolor{textcolor}%
\pgftext[x=0.939645in,y=0.438444in,,top]{\color{textcolor}\sffamily\fontsize{8.000000}{9.600000}\selectfont 3.0}%
\end{pgfscope}%
\begin{pgfscope}%
\pgfpathrectangle{\pgfqpoint{0.556847in}{0.516222in}}{\pgfqpoint{1.722590in}{1.783528in}} %
\pgfusepath{clip}%
\pgfsetroundcap%
\pgfsetroundjoin%
\pgfsetlinewidth{0.803000pt}%
\definecolor{currentstroke}{rgb}{1.000000,1.000000,1.000000}%
\pgfsetstrokecolor{currentstroke}%
\pgfsetdash{}{0pt}%
\pgfpathmoveto{\pgfqpoint{1.131044in}{0.516222in}}%
\pgfpathlineto{\pgfqpoint{1.131044in}{2.299750in}}%
\pgfusepath{stroke}%
\end{pgfscope}%
\begin{pgfscope}%
\pgfsetbuttcap%
\pgfsetroundjoin%
\definecolor{currentfill}{rgb}{0.150000,0.150000,0.150000}%
\pgfsetfillcolor{currentfill}%
\pgfsetlinewidth{0.803000pt}%
\definecolor{currentstroke}{rgb}{0.150000,0.150000,0.150000}%
\pgfsetstrokecolor{currentstroke}%
\pgfsetdash{}{0pt}%
\pgfsys@defobject{currentmarker}{\pgfqpoint{0.000000in}{0.000000in}}{\pgfqpoint{0.000000in}{0.000000in}}{%
\pgfpathmoveto{\pgfqpoint{0.000000in}{0.000000in}}%
\pgfpathlineto{\pgfqpoint{0.000000in}{0.000000in}}%
\pgfusepath{stroke,fill}%
}%
\begin{pgfscope}%
\pgfsys@transformshift{1.131044in}{0.516222in}%
\pgfsys@useobject{currentmarker}{}%
\end{pgfscope}%
\end{pgfscope}%
\begin{pgfscope}%
\definecolor{textcolor}{rgb}{0.150000,0.150000,0.150000}%
\pgfsetstrokecolor{textcolor}%
\pgfsetfillcolor{textcolor}%
\pgftext[x=1.131044in,y=0.438444in,,top]{\color{textcolor}\sffamily\fontsize{8.000000}{9.600000}\selectfont 3.5}%
\end{pgfscope}%
\begin{pgfscope}%
\pgfpathrectangle{\pgfqpoint{0.556847in}{0.516222in}}{\pgfqpoint{1.722590in}{1.783528in}} %
\pgfusepath{clip}%
\pgfsetroundcap%
\pgfsetroundjoin%
\pgfsetlinewidth{0.803000pt}%
\definecolor{currentstroke}{rgb}{1.000000,1.000000,1.000000}%
\pgfsetstrokecolor{currentstroke}%
\pgfsetdash{}{0pt}%
\pgfpathmoveto{\pgfqpoint{1.322443in}{0.516222in}}%
\pgfpathlineto{\pgfqpoint{1.322443in}{2.299750in}}%
\pgfusepath{stroke}%
\end{pgfscope}%
\begin{pgfscope}%
\pgfsetbuttcap%
\pgfsetroundjoin%
\definecolor{currentfill}{rgb}{0.150000,0.150000,0.150000}%
\pgfsetfillcolor{currentfill}%
\pgfsetlinewidth{0.803000pt}%
\definecolor{currentstroke}{rgb}{0.150000,0.150000,0.150000}%
\pgfsetstrokecolor{currentstroke}%
\pgfsetdash{}{0pt}%
\pgfsys@defobject{currentmarker}{\pgfqpoint{0.000000in}{0.000000in}}{\pgfqpoint{0.000000in}{0.000000in}}{%
\pgfpathmoveto{\pgfqpoint{0.000000in}{0.000000in}}%
\pgfpathlineto{\pgfqpoint{0.000000in}{0.000000in}}%
\pgfusepath{stroke,fill}%
}%
\begin{pgfscope}%
\pgfsys@transformshift{1.322443in}{0.516222in}%
\pgfsys@useobject{currentmarker}{}%
\end{pgfscope}%
\end{pgfscope}%
\begin{pgfscope}%
\definecolor{textcolor}{rgb}{0.150000,0.150000,0.150000}%
\pgfsetstrokecolor{textcolor}%
\pgfsetfillcolor{textcolor}%
\pgftext[x=1.322443in,y=0.438444in,,top]{\color{textcolor}\sffamily\fontsize{8.000000}{9.600000}\selectfont 4.0}%
\end{pgfscope}%
\begin{pgfscope}%
\pgfpathrectangle{\pgfqpoint{0.556847in}{0.516222in}}{\pgfqpoint{1.722590in}{1.783528in}} %
\pgfusepath{clip}%
\pgfsetroundcap%
\pgfsetroundjoin%
\pgfsetlinewidth{0.803000pt}%
\definecolor{currentstroke}{rgb}{1.000000,1.000000,1.000000}%
\pgfsetstrokecolor{currentstroke}%
\pgfsetdash{}{0pt}%
\pgfpathmoveto{\pgfqpoint{1.513842in}{0.516222in}}%
\pgfpathlineto{\pgfqpoint{1.513842in}{2.299750in}}%
\pgfusepath{stroke}%
\end{pgfscope}%
\begin{pgfscope}%
\pgfsetbuttcap%
\pgfsetroundjoin%
\definecolor{currentfill}{rgb}{0.150000,0.150000,0.150000}%
\pgfsetfillcolor{currentfill}%
\pgfsetlinewidth{0.803000pt}%
\definecolor{currentstroke}{rgb}{0.150000,0.150000,0.150000}%
\pgfsetstrokecolor{currentstroke}%
\pgfsetdash{}{0pt}%
\pgfsys@defobject{currentmarker}{\pgfqpoint{0.000000in}{0.000000in}}{\pgfqpoint{0.000000in}{0.000000in}}{%
\pgfpathmoveto{\pgfqpoint{0.000000in}{0.000000in}}%
\pgfpathlineto{\pgfqpoint{0.000000in}{0.000000in}}%
\pgfusepath{stroke,fill}%
}%
\begin{pgfscope}%
\pgfsys@transformshift{1.513842in}{0.516222in}%
\pgfsys@useobject{currentmarker}{}%
\end{pgfscope}%
\end{pgfscope}%
\begin{pgfscope}%
\definecolor{textcolor}{rgb}{0.150000,0.150000,0.150000}%
\pgfsetstrokecolor{textcolor}%
\pgfsetfillcolor{textcolor}%
\pgftext[x=1.513842in,y=0.438444in,,top]{\color{textcolor}\sffamily\fontsize{8.000000}{9.600000}\selectfont 4.5}%
\end{pgfscope}%
\begin{pgfscope}%
\pgfpathrectangle{\pgfqpoint{0.556847in}{0.516222in}}{\pgfqpoint{1.722590in}{1.783528in}} %
\pgfusepath{clip}%
\pgfsetroundcap%
\pgfsetroundjoin%
\pgfsetlinewidth{0.803000pt}%
\definecolor{currentstroke}{rgb}{1.000000,1.000000,1.000000}%
\pgfsetstrokecolor{currentstroke}%
\pgfsetdash{}{0pt}%
\pgfpathmoveto{\pgfqpoint{1.705241in}{0.516222in}}%
\pgfpathlineto{\pgfqpoint{1.705241in}{2.299750in}}%
\pgfusepath{stroke}%
\end{pgfscope}%
\begin{pgfscope}%
\pgfsetbuttcap%
\pgfsetroundjoin%
\definecolor{currentfill}{rgb}{0.150000,0.150000,0.150000}%
\pgfsetfillcolor{currentfill}%
\pgfsetlinewidth{0.803000pt}%
\definecolor{currentstroke}{rgb}{0.150000,0.150000,0.150000}%
\pgfsetstrokecolor{currentstroke}%
\pgfsetdash{}{0pt}%
\pgfsys@defobject{currentmarker}{\pgfqpoint{0.000000in}{0.000000in}}{\pgfqpoint{0.000000in}{0.000000in}}{%
\pgfpathmoveto{\pgfqpoint{0.000000in}{0.000000in}}%
\pgfpathlineto{\pgfqpoint{0.000000in}{0.000000in}}%
\pgfusepath{stroke,fill}%
}%
\begin{pgfscope}%
\pgfsys@transformshift{1.705241in}{0.516222in}%
\pgfsys@useobject{currentmarker}{}%
\end{pgfscope}%
\end{pgfscope}%
\begin{pgfscope}%
\definecolor{textcolor}{rgb}{0.150000,0.150000,0.150000}%
\pgfsetstrokecolor{textcolor}%
\pgfsetfillcolor{textcolor}%
\pgftext[x=1.705241in,y=0.438444in,,top]{\color{textcolor}\sffamily\fontsize{8.000000}{9.600000}\selectfont 5.0}%
\end{pgfscope}%
\begin{pgfscope}%
\pgfpathrectangle{\pgfqpoint{0.556847in}{0.516222in}}{\pgfqpoint{1.722590in}{1.783528in}} %
\pgfusepath{clip}%
\pgfsetroundcap%
\pgfsetroundjoin%
\pgfsetlinewidth{0.803000pt}%
\definecolor{currentstroke}{rgb}{1.000000,1.000000,1.000000}%
\pgfsetstrokecolor{currentstroke}%
\pgfsetdash{}{0pt}%
\pgfpathmoveto{\pgfqpoint{1.896640in}{0.516222in}}%
\pgfpathlineto{\pgfqpoint{1.896640in}{2.299750in}}%
\pgfusepath{stroke}%
\end{pgfscope}%
\begin{pgfscope}%
\pgfsetbuttcap%
\pgfsetroundjoin%
\definecolor{currentfill}{rgb}{0.150000,0.150000,0.150000}%
\pgfsetfillcolor{currentfill}%
\pgfsetlinewidth{0.803000pt}%
\definecolor{currentstroke}{rgb}{0.150000,0.150000,0.150000}%
\pgfsetstrokecolor{currentstroke}%
\pgfsetdash{}{0pt}%
\pgfsys@defobject{currentmarker}{\pgfqpoint{0.000000in}{0.000000in}}{\pgfqpoint{0.000000in}{0.000000in}}{%
\pgfpathmoveto{\pgfqpoint{0.000000in}{0.000000in}}%
\pgfpathlineto{\pgfqpoint{0.000000in}{0.000000in}}%
\pgfusepath{stroke,fill}%
}%
\begin{pgfscope}%
\pgfsys@transformshift{1.896640in}{0.516222in}%
\pgfsys@useobject{currentmarker}{}%
\end{pgfscope}%
\end{pgfscope}%
\begin{pgfscope}%
\definecolor{textcolor}{rgb}{0.150000,0.150000,0.150000}%
\pgfsetstrokecolor{textcolor}%
\pgfsetfillcolor{textcolor}%
\pgftext[x=1.896640in,y=0.438444in,,top]{\color{textcolor}\sffamily\fontsize{8.000000}{9.600000}\selectfont 5.5}%
\end{pgfscope}%
\begin{pgfscope}%
\pgfpathrectangle{\pgfqpoint{0.556847in}{0.516222in}}{\pgfqpoint{1.722590in}{1.783528in}} %
\pgfusepath{clip}%
\pgfsetroundcap%
\pgfsetroundjoin%
\pgfsetlinewidth{0.803000pt}%
\definecolor{currentstroke}{rgb}{1.000000,1.000000,1.000000}%
\pgfsetstrokecolor{currentstroke}%
\pgfsetdash{}{0pt}%
\pgfpathmoveto{\pgfqpoint{2.088039in}{0.516222in}}%
\pgfpathlineto{\pgfqpoint{2.088039in}{2.299750in}}%
\pgfusepath{stroke}%
\end{pgfscope}%
\begin{pgfscope}%
\pgfsetbuttcap%
\pgfsetroundjoin%
\definecolor{currentfill}{rgb}{0.150000,0.150000,0.150000}%
\pgfsetfillcolor{currentfill}%
\pgfsetlinewidth{0.803000pt}%
\definecolor{currentstroke}{rgb}{0.150000,0.150000,0.150000}%
\pgfsetstrokecolor{currentstroke}%
\pgfsetdash{}{0pt}%
\pgfsys@defobject{currentmarker}{\pgfqpoint{0.000000in}{0.000000in}}{\pgfqpoint{0.000000in}{0.000000in}}{%
\pgfpathmoveto{\pgfqpoint{0.000000in}{0.000000in}}%
\pgfpathlineto{\pgfqpoint{0.000000in}{0.000000in}}%
\pgfusepath{stroke,fill}%
}%
\begin{pgfscope}%
\pgfsys@transformshift{2.088039in}{0.516222in}%
\pgfsys@useobject{currentmarker}{}%
\end{pgfscope}%
\end{pgfscope}%
\begin{pgfscope}%
\definecolor{textcolor}{rgb}{0.150000,0.150000,0.150000}%
\pgfsetstrokecolor{textcolor}%
\pgfsetfillcolor{textcolor}%
\pgftext[x=2.088039in,y=0.438444in,,top]{\color{textcolor}\sffamily\fontsize{8.000000}{9.600000}\selectfont 6.0}%
\end{pgfscope}%
\begin{pgfscope}%
\pgfpathrectangle{\pgfqpoint{0.556847in}{0.516222in}}{\pgfqpoint{1.722590in}{1.783528in}} %
\pgfusepath{clip}%
\pgfsetroundcap%
\pgfsetroundjoin%
\pgfsetlinewidth{0.803000pt}%
\definecolor{currentstroke}{rgb}{1.000000,1.000000,1.000000}%
\pgfsetstrokecolor{currentstroke}%
\pgfsetdash{}{0pt}%
\pgfpathmoveto{\pgfqpoint{2.279437in}{0.516222in}}%
\pgfpathlineto{\pgfqpoint{2.279437in}{2.299750in}}%
\pgfusepath{stroke}%
\end{pgfscope}%
\begin{pgfscope}%
\pgfsetbuttcap%
\pgfsetroundjoin%
\definecolor{currentfill}{rgb}{0.150000,0.150000,0.150000}%
\pgfsetfillcolor{currentfill}%
\pgfsetlinewidth{0.803000pt}%
\definecolor{currentstroke}{rgb}{0.150000,0.150000,0.150000}%
\pgfsetstrokecolor{currentstroke}%
\pgfsetdash{}{0pt}%
\pgfsys@defobject{currentmarker}{\pgfqpoint{0.000000in}{0.000000in}}{\pgfqpoint{0.000000in}{0.000000in}}{%
\pgfpathmoveto{\pgfqpoint{0.000000in}{0.000000in}}%
\pgfpathlineto{\pgfqpoint{0.000000in}{0.000000in}}%
\pgfusepath{stroke,fill}%
}%
\begin{pgfscope}%
\pgfsys@transformshift{2.279437in}{0.516222in}%
\pgfsys@useobject{currentmarker}{}%
\end{pgfscope}%
\end{pgfscope}%
\begin{pgfscope}%
\definecolor{textcolor}{rgb}{0.150000,0.150000,0.150000}%
\pgfsetstrokecolor{textcolor}%
\pgfsetfillcolor{textcolor}%
\pgftext[x=2.279437in,y=0.438444in,,top]{\color{textcolor}\sffamily\fontsize{8.000000}{9.600000}\selectfont 6.5}%
\end{pgfscope}%
\begin{pgfscope}%
\definecolor{textcolor}{rgb}{0.150000,0.150000,0.150000}%
\pgfsetstrokecolor{textcolor}%
\pgfsetfillcolor{textcolor}%
\pgftext[x=1.418142in,y=0.273321in,,top]{\color{textcolor}\sffamily\fontsize{8.800000}{10.560000}\selectfont Falling time realization 1 obs 1}%
\end{pgfscope}%
\begin{pgfscope}%
\pgfpathrectangle{\pgfqpoint{0.556847in}{0.516222in}}{\pgfqpoint{1.722590in}{1.783528in}} %
\pgfusepath{clip}%
\pgfsetroundcap%
\pgfsetroundjoin%
\pgfsetlinewidth{0.803000pt}%
\definecolor{currentstroke}{rgb}{1.000000,1.000000,1.000000}%
\pgfsetstrokecolor{currentstroke}%
\pgfsetdash{}{0pt}%
\pgfpathmoveto{\pgfqpoint{0.556847in}{0.516222in}}%
\pgfpathlineto{\pgfqpoint{2.279437in}{0.516222in}}%
\pgfusepath{stroke}%
\end{pgfscope}%
\begin{pgfscope}%
\pgfsetbuttcap%
\pgfsetroundjoin%
\definecolor{currentfill}{rgb}{0.150000,0.150000,0.150000}%
\pgfsetfillcolor{currentfill}%
\pgfsetlinewidth{0.803000pt}%
\definecolor{currentstroke}{rgb}{0.150000,0.150000,0.150000}%
\pgfsetstrokecolor{currentstroke}%
\pgfsetdash{}{0pt}%
\pgfsys@defobject{currentmarker}{\pgfqpoint{0.000000in}{0.000000in}}{\pgfqpoint{0.000000in}{0.000000in}}{%
\pgfpathmoveto{\pgfqpoint{0.000000in}{0.000000in}}%
\pgfpathlineto{\pgfqpoint{0.000000in}{0.000000in}}%
\pgfusepath{stroke,fill}%
}%
\begin{pgfscope}%
\pgfsys@transformshift{0.556847in}{0.516222in}%
\pgfsys@useobject{currentmarker}{}%
\end{pgfscope}%
\end{pgfscope}%
\begin{pgfscope}%
\definecolor{textcolor}{rgb}{0.150000,0.150000,0.150000}%
\pgfsetstrokecolor{textcolor}%
\pgfsetfillcolor{textcolor}%
\pgftext[x=0.479069in,y=0.516222in,right,]{\color{textcolor}\sffamily\fontsize{8.000000}{9.600000}\selectfont 2.5}%
\end{pgfscope}%
\begin{pgfscope}%
\pgfpathrectangle{\pgfqpoint{0.556847in}{0.516222in}}{\pgfqpoint{1.722590in}{1.783528in}} %
\pgfusepath{clip}%
\pgfsetroundcap%
\pgfsetroundjoin%
\pgfsetlinewidth{0.803000pt}%
\definecolor{currentstroke}{rgb}{1.000000,1.000000,1.000000}%
\pgfsetstrokecolor{currentstroke}%
\pgfsetdash{}{0pt}%
\pgfpathmoveto{\pgfqpoint{0.556847in}{0.739163in}}%
\pgfpathlineto{\pgfqpoint{2.279437in}{0.739163in}}%
\pgfusepath{stroke}%
\end{pgfscope}%
\begin{pgfscope}%
\pgfsetbuttcap%
\pgfsetroundjoin%
\definecolor{currentfill}{rgb}{0.150000,0.150000,0.150000}%
\pgfsetfillcolor{currentfill}%
\pgfsetlinewidth{0.803000pt}%
\definecolor{currentstroke}{rgb}{0.150000,0.150000,0.150000}%
\pgfsetstrokecolor{currentstroke}%
\pgfsetdash{}{0pt}%
\pgfsys@defobject{currentmarker}{\pgfqpoint{0.000000in}{0.000000in}}{\pgfqpoint{0.000000in}{0.000000in}}{%
\pgfpathmoveto{\pgfqpoint{0.000000in}{0.000000in}}%
\pgfpathlineto{\pgfqpoint{0.000000in}{0.000000in}}%
\pgfusepath{stroke,fill}%
}%
\begin{pgfscope}%
\pgfsys@transformshift{0.556847in}{0.739163in}%
\pgfsys@useobject{currentmarker}{}%
\end{pgfscope}%
\end{pgfscope}%
\begin{pgfscope}%
\definecolor{textcolor}{rgb}{0.150000,0.150000,0.150000}%
\pgfsetstrokecolor{textcolor}%
\pgfsetfillcolor{textcolor}%
\pgftext[x=0.479069in,y=0.739163in,right,]{\color{textcolor}\sffamily\fontsize{8.000000}{9.600000}\selectfont 3.0}%
\end{pgfscope}%
\begin{pgfscope}%
\pgfpathrectangle{\pgfqpoint{0.556847in}{0.516222in}}{\pgfqpoint{1.722590in}{1.783528in}} %
\pgfusepath{clip}%
\pgfsetroundcap%
\pgfsetroundjoin%
\pgfsetlinewidth{0.803000pt}%
\definecolor{currentstroke}{rgb}{1.000000,1.000000,1.000000}%
\pgfsetstrokecolor{currentstroke}%
\pgfsetdash{}{0pt}%
\pgfpathmoveto{\pgfqpoint{0.556847in}{0.962104in}}%
\pgfpathlineto{\pgfqpoint{2.279437in}{0.962104in}}%
\pgfusepath{stroke}%
\end{pgfscope}%
\begin{pgfscope}%
\pgfsetbuttcap%
\pgfsetroundjoin%
\definecolor{currentfill}{rgb}{0.150000,0.150000,0.150000}%
\pgfsetfillcolor{currentfill}%
\pgfsetlinewidth{0.803000pt}%
\definecolor{currentstroke}{rgb}{0.150000,0.150000,0.150000}%
\pgfsetstrokecolor{currentstroke}%
\pgfsetdash{}{0pt}%
\pgfsys@defobject{currentmarker}{\pgfqpoint{0.000000in}{0.000000in}}{\pgfqpoint{0.000000in}{0.000000in}}{%
\pgfpathmoveto{\pgfqpoint{0.000000in}{0.000000in}}%
\pgfpathlineto{\pgfqpoint{0.000000in}{0.000000in}}%
\pgfusepath{stroke,fill}%
}%
\begin{pgfscope}%
\pgfsys@transformshift{0.556847in}{0.962104in}%
\pgfsys@useobject{currentmarker}{}%
\end{pgfscope}%
\end{pgfscope}%
\begin{pgfscope}%
\definecolor{textcolor}{rgb}{0.150000,0.150000,0.150000}%
\pgfsetstrokecolor{textcolor}%
\pgfsetfillcolor{textcolor}%
\pgftext[x=0.479069in,y=0.962104in,right,]{\color{textcolor}\sffamily\fontsize{8.000000}{9.600000}\selectfont 3.5}%
\end{pgfscope}%
\begin{pgfscope}%
\pgfpathrectangle{\pgfqpoint{0.556847in}{0.516222in}}{\pgfqpoint{1.722590in}{1.783528in}} %
\pgfusepath{clip}%
\pgfsetroundcap%
\pgfsetroundjoin%
\pgfsetlinewidth{0.803000pt}%
\definecolor{currentstroke}{rgb}{1.000000,1.000000,1.000000}%
\pgfsetstrokecolor{currentstroke}%
\pgfsetdash{}{0pt}%
\pgfpathmoveto{\pgfqpoint{0.556847in}{1.185045in}}%
\pgfpathlineto{\pgfqpoint{2.279437in}{1.185045in}}%
\pgfusepath{stroke}%
\end{pgfscope}%
\begin{pgfscope}%
\pgfsetbuttcap%
\pgfsetroundjoin%
\definecolor{currentfill}{rgb}{0.150000,0.150000,0.150000}%
\pgfsetfillcolor{currentfill}%
\pgfsetlinewidth{0.803000pt}%
\definecolor{currentstroke}{rgb}{0.150000,0.150000,0.150000}%
\pgfsetstrokecolor{currentstroke}%
\pgfsetdash{}{0pt}%
\pgfsys@defobject{currentmarker}{\pgfqpoint{0.000000in}{0.000000in}}{\pgfqpoint{0.000000in}{0.000000in}}{%
\pgfpathmoveto{\pgfqpoint{0.000000in}{0.000000in}}%
\pgfpathlineto{\pgfqpoint{0.000000in}{0.000000in}}%
\pgfusepath{stroke,fill}%
}%
\begin{pgfscope}%
\pgfsys@transformshift{0.556847in}{1.185045in}%
\pgfsys@useobject{currentmarker}{}%
\end{pgfscope}%
\end{pgfscope}%
\begin{pgfscope}%
\definecolor{textcolor}{rgb}{0.150000,0.150000,0.150000}%
\pgfsetstrokecolor{textcolor}%
\pgfsetfillcolor{textcolor}%
\pgftext[x=0.479069in,y=1.185045in,right,]{\color{textcolor}\sffamily\fontsize{8.000000}{9.600000}\selectfont 4.0}%
\end{pgfscope}%
\begin{pgfscope}%
\pgfpathrectangle{\pgfqpoint{0.556847in}{0.516222in}}{\pgfqpoint{1.722590in}{1.783528in}} %
\pgfusepath{clip}%
\pgfsetroundcap%
\pgfsetroundjoin%
\pgfsetlinewidth{0.803000pt}%
\definecolor{currentstroke}{rgb}{1.000000,1.000000,1.000000}%
\pgfsetstrokecolor{currentstroke}%
\pgfsetdash{}{0pt}%
\pgfpathmoveto{\pgfqpoint{0.556847in}{1.407986in}}%
\pgfpathlineto{\pgfqpoint{2.279437in}{1.407986in}}%
\pgfusepath{stroke}%
\end{pgfscope}%
\begin{pgfscope}%
\pgfsetbuttcap%
\pgfsetroundjoin%
\definecolor{currentfill}{rgb}{0.150000,0.150000,0.150000}%
\pgfsetfillcolor{currentfill}%
\pgfsetlinewidth{0.803000pt}%
\definecolor{currentstroke}{rgb}{0.150000,0.150000,0.150000}%
\pgfsetstrokecolor{currentstroke}%
\pgfsetdash{}{0pt}%
\pgfsys@defobject{currentmarker}{\pgfqpoint{0.000000in}{0.000000in}}{\pgfqpoint{0.000000in}{0.000000in}}{%
\pgfpathmoveto{\pgfqpoint{0.000000in}{0.000000in}}%
\pgfpathlineto{\pgfqpoint{0.000000in}{0.000000in}}%
\pgfusepath{stroke,fill}%
}%
\begin{pgfscope}%
\pgfsys@transformshift{0.556847in}{1.407986in}%
\pgfsys@useobject{currentmarker}{}%
\end{pgfscope}%
\end{pgfscope}%
\begin{pgfscope}%
\definecolor{textcolor}{rgb}{0.150000,0.150000,0.150000}%
\pgfsetstrokecolor{textcolor}%
\pgfsetfillcolor{textcolor}%
\pgftext[x=0.479069in,y=1.407986in,right,]{\color{textcolor}\sffamily\fontsize{8.000000}{9.600000}\selectfont 4.5}%
\end{pgfscope}%
\begin{pgfscope}%
\pgfpathrectangle{\pgfqpoint{0.556847in}{0.516222in}}{\pgfqpoint{1.722590in}{1.783528in}} %
\pgfusepath{clip}%
\pgfsetroundcap%
\pgfsetroundjoin%
\pgfsetlinewidth{0.803000pt}%
\definecolor{currentstroke}{rgb}{1.000000,1.000000,1.000000}%
\pgfsetstrokecolor{currentstroke}%
\pgfsetdash{}{0pt}%
\pgfpathmoveto{\pgfqpoint{0.556847in}{1.630927in}}%
\pgfpathlineto{\pgfqpoint{2.279437in}{1.630927in}}%
\pgfusepath{stroke}%
\end{pgfscope}%
\begin{pgfscope}%
\pgfsetbuttcap%
\pgfsetroundjoin%
\definecolor{currentfill}{rgb}{0.150000,0.150000,0.150000}%
\pgfsetfillcolor{currentfill}%
\pgfsetlinewidth{0.803000pt}%
\definecolor{currentstroke}{rgb}{0.150000,0.150000,0.150000}%
\pgfsetstrokecolor{currentstroke}%
\pgfsetdash{}{0pt}%
\pgfsys@defobject{currentmarker}{\pgfqpoint{0.000000in}{0.000000in}}{\pgfqpoint{0.000000in}{0.000000in}}{%
\pgfpathmoveto{\pgfqpoint{0.000000in}{0.000000in}}%
\pgfpathlineto{\pgfqpoint{0.000000in}{0.000000in}}%
\pgfusepath{stroke,fill}%
}%
\begin{pgfscope}%
\pgfsys@transformshift{0.556847in}{1.630927in}%
\pgfsys@useobject{currentmarker}{}%
\end{pgfscope}%
\end{pgfscope}%
\begin{pgfscope}%
\definecolor{textcolor}{rgb}{0.150000,0.150000,0.150000}%
\pgfsetstrokecolor{textcolor}%
\pgfsetfillcolor{textcolor}%
\pgftext[x=0.479069in,y=1.630927in,right,]{\color{textcolor}\sffamily\fontsize{8.000000}{9.600000}\selectfont 5.0}%
\end{pgfscope}%
\begin{pgfscope}%
\pgfpathrectangle{\pgfqpoint{0.556847in}{0.516222in}}{\pgfqpoint{1.722590in}{1.783528in}} %
\pgfusepath{clip}%
\pgfsetroundcap%
\pgfsetroundjoin%
\pgfsetlinewidth{0.803000pt}%
\definecolor{currentstroke}{rgb}{1.000000,1.000000,1.000000}%
\pgfsetstrokecolor{currentstroke}%
\pgfsetdash{}{0pt}%
\pgfpathmoveto{\pgfqpoint{0.556847in}{1.853868in}}%
\pgfpathlineto{\pgfqpoint{2.279437in}{1.853868in}}%
\pgfusepath{stroke}%
\end{pgfscope}%
\begin{pgfscope}%
\pgfsetbuttcap%
\pgfsetroundjoin%
\definecolor{currentfill}{rgb}{0.150000,0.150000,0.150000}%
\pgfsetfillcolor{currentfill}%
\pgfsetlinewidth{0.803000pt}%
\definecolor{currentstroke}{rgb}{0.150000,0.150000,0.150000}%
\pgfsetstrokecolor{currentstroke}%
\pgfsetdash{}{0pt}%
\pgfsys@defobject{currentmarker}{\pgfqpoint{0.000000in}{0.000000in}}{\pgfqpoint{0.000000in}{0.000000in}}{%
\pgfpathmoveto{\pgfqpoint{0.000000in}{0.000000in}}%
\pgfpathlineto{\pgfqpoint{0.000000in}{0.000000in}}%
\pgfusepath{stroke,fill}%
}%
\begin{pgfscope}%
\pgfsys@transformshift{0.556847in}{1.853868in}%
\pgfsys@useobject{currentmarker}{}%
\end{pgfscope}%
\end{pgfscope}%
\begin{pgfscope}%
\definecolor{textcolor}{rgb}{0.150000,0.150000,0.150000}%
\pgfsetstrokecolor{textcolor}%
\pgfsetfillcolor{textcolor}%
\pgftext[x=0.479069in,y=1.853868in,right,]{\color{textcolor}\sffamily\fontsize{8.000000}{9.600000}\selectfont 5.5}%
\end{pgfscope}%
\begin{pgfscope}%
\pgfpathrectangle{\pgfqpoint{0.556847in}{0.516222in}}{\pgfqpoint{1.722590in}{1.783528in}} %
\pgfusepath{clip}%
\pgfsetroundcap%
\pgfsetroundjoin%
\pgfsetlinewidth{0.803000pt}%
\definecolor{currentstroke}{rgb}{1.000000,1.000000,1.000000}%
\pgfsetstrokecolor{currentstroke}%
\pgfsetdash{}{0pt}%
\pgfpathmoveto{\pgfqpoint{0.556847in}{2.076809in}}%
\pgfpathlineto{\pgfqpoint{2.279437in}{2.076809in}}%
\pgfusepath{stroke}%
\end{pgfscope}%
\begin{pgfscope}%
\pgfsetbuttcap%
\pgfsetroundjoin%
\definecolor{currentfill}{rgb}{0.150000,0.150000,0.150000}%
\pgfsetfillcolor{currentfill}%
\pgfsetlinewidth{0.803000pt}%
\definecolor{currentstroke}{rgb}{0.150000,0.150000,0.150000}%
\pgfsetstrokecolor{currentstroke}%
\pgfsetdash{}{0pt}%
\pgfsys@defobject{currentmarker}{\pgfqpoint{0.000000in}{0.000000in}}{\pgfqpoint{0.000000in}{0.000000in}}{%
\pgfpathmoveto{\pgfqpoint{0.000000in}{0.000000in}}%
\pgfpathlineto{\pgfqpoint{0.000000in}{0.000000in}}%
\pgfusepath{stroke,fill}%
}%
\begin{pgfscope}%
\pgfsys@transformshift{0.556847in}{2.076809in}%
\pgfsys@useobject{currentmarker}{}%
\end{pgfscope}%
\end{pgfscope}%
\begin{pgfscope}%
\definecolor{textcolor}{rgb}{0.150000,0.150000,0.150000}%
\pgfsetstrokecolor{textcolor}%
\pgfsetfillcolor{textcolor}%
\pgftext[x=0.479069in,y=2.076809in,right,]{\color{textcolor}\sffamily\fontsize{8.000000}{9.600000}\selectfont 6.0}%
\end{pgfscope}%
\begin{pgfscope}%
\pgfpathrectangle{\pgfqpoint{0.556847in}{0.516222in}}{\pgfqpoint{1.722590in}{1.783528in}} %
\pgfusepath{clip}%
\pgfsetroundcap%
\pgfsetroundjoin%
\pgfsetlinewidth{0.803000pt}%
\definecolor{currentstroke}{rgb}{1.000000,1.000000,1.000000}%
\pgfsetstrokecolor{currentstroke}%
\pgfsetdash{}{0pt}%
\pgfpathmoveto{\pgfqpoint{0.556847in}{2.299750in}}%
\pgfpathlineto{\pgfqpoint{2.279437in}{2.299750in}}%
\pgfusepath{stroke}%
\end{pgfscope}%
\begin{pgfscope}%
\pgfsetbuttcap%
\pgfsetroundjoin%
\definecolor{currentfill}{rgb}{0.150000,0.150000,0.150000}%
\pgfsetfillcolor{currentfill}%
\pgfsetlinewidth{0.803000pt}%
\definecolor{currentstroke}{rgb}{0.150000,0.150000,0.150000}%
\pgfsetstrokecolor{currentstroke}%
\pgfsetdash{}{0pt}%
\pgfsys@defobject{currentmarker}{\pgfqpoint{0.000000in}{0.000000in}}{\pgfqpoint{0.000000in}{0.000000in}}{%
\pgfpathmoveto{\pgfqpoint{0.000000in}{0.000000in}}%
\pgfpathlineto{\pgfqpoint{0.000000in}{0.000000in}}%
\pgfusepath{stroke,fill}%
}%
\begin{pgfscope}%
\pgfsys@transformshift{0.556847in}{2.299750in}%
\pgfsys@useobject{currentmarker}{}%
\end{pgfscope}%
\end{pgfscope}%
\begin{pgfscope}%
\definecolor{textcolor}{rgb}{0.150000,0.150000,0.150000}%
\pgfsetstrokecolor{textcolor}%
\pgfsetfillcolor{textcolor}%
\pgftext[x=0.479069in,y=2.299750in,right,]{\color{textcolor}\sffamily\fontsize{8.000000}{9.600000}\selectfont 6.5}%
\end{pgfscope}%
\begin{pgfscope}%
\definecolor{textcolor}{rgb}{0.150000,0.150000,0.150000}%
\pgfsetstrokecolor{textcolor}%
\pgfsetfillcolor{textcolor}%
\pgftext[x=0.251677in,y=1.407986in,,bottom,rotate=90.000000]{\color{textcolor}\sffamily\fontsize{8.800000}{10.560000}\selectfont Falling time realization 1 obs 2}%
\end{pgfscope}%
\begin{pgfscope}%
\pgfpathrectangle{\pgfqpoint{0.556847in}{0.516222in}}{\pgfqpoint{1.722590in}{1.783528in}} %
\pgfusepath{clip}%
\pgfsetbuttcap%
\pgfsetroundjoin%
\definecolor{currentfill}{rgb}{0.298039,0.447059,0.690196}%
\pgfsetfillcolor{currentfill}%
\pgfsetlinewidth{0.240900pt}%
\definecolor{currentstroke}{rgb}{1.000000,1.000000,1.000000}%
\pgfsetstrokecolor{currentstroke}%
\pgfsetdash{}{0pt}%
\pgfpathmoveto{\pgfqpoint{1.575089in}{1.599871in}}%
\pgfpathcurveto{\pgfqpoint{1.583326in}{1.599871in}}{\pgfqpoint{1.591226in}{1.603143in}}{\pgfqpoint{1.597050in}{1.608967in}}%
\pgfpathcurveto{\pgfqpoint{1.602874in}{1.614791in}}{\pgfqpoint{1.606146in}{1.622691in}}{\pgfqpoint{1.606146in}{1.630927in}}%
\pgfpathcurveto{\pgfqpoint{1.606146in}{1.639163in}}{\pgfqpoint{1.602874in}{1.647063in}}{\pgfqpoint{1.597050in}{1.652887in}}%
\pgfpathcurveto{\pgfqpoint{1.591226in}{1.658711in}}{\pgfqpoint{1.583326in}{1.661984in}}{\pgfqpoint{1.575089in}{1.661984in}}%
\pgfpathcurveto{\pgfqpoint{1.566853in}{1.661984in}}{\pgfqpoint{1.558953in}{1.658711in}}{\pgfqpoint{1.553129in}{1.652887in}}%
\pgfpathcurveto{\pgfqpoint{1.547305in}{1.647063in}}{\pgfqpoint{1.544033in}{1.639163in}}{\pgfqpoint{1.544033in}{1.630927in}}%
\pgfpathcurveto{\pgfqpoint{1.544033in}{1.622691in}}{\pgfqpoint{1.547305in}{1.614791in}}{\pgfqpoint{1.553129in}{1.608967in}}%
\pgfpathcurveto{\pgfqpoint{1.558953in}{1.603143in}}{\pgfqpoint{1.566853in}{1.599871in}}{\pgfqpoint{1.575089in}{1.599871in}}%
\pgfpathclose%
\pgfusepath{stroke,fill}%
\end{pgfscope}%
\begin{pgfscope}%
\pgfpathrectangle{\pgfqpoint{0.556847in}{0.516222in}}{\pgfqpoint{1.722590in}{1.783528in}} %
\pgfusepath{clip}%
\pgfsetbuttcap%
\pgfsetroundjoin%
\definecolor{currentfill}{rgb}{0.298039,0.447059,0.690196}%
\pgfsetfillcolor{currentfill}%
\pgfsetlinewidth{0.240900pt}%
\definecolor{currentstroke}{rgb}{1.000000,1.000000,1.000000}%
\pgfsetstrokecolor{currentstroke}%
\pgfsetdash{}{0pt}%
\pgfpathmoveto{\pgfqpoint{1.287991in}{1.162906in}}%
\pgfpathcurveto{\pgfqpoint{1.296227in}{1.162906in}}{\pgfqpoint{1.304127in}{1.166179in}}{\pgfqpoint{1.309951in}{1.172003in}}%
\pgfpathcurveto{\pgfqpoint{1.315775in}{1.177826in}}{\pgfqpoint{1.319048in}{1.185726in}}{\pgfqpoint{1.319048in}{1.193963in}}%
\pgfpathcurveto{\pgfqpoint{1.319048in}{1.202199in}}{\pgfqpoint{1.315775in}{1.210099in}}{\pgfqpoint{1.309951in}{1.215923in}}%
\pgfpathcurveto{\pgfqpoint{1.304127in}{1.221747in}}{\pgfqpoint{1.296227in}{1.225019in}}{\pgfqpoint{1.287991in}{1.225019in}}%
\pgfpathcurveto{\pgfqpoint{1.279755in}{1.225019in}}{\pgfqpoint{1.271855in}{1.221747in}}{\pgfqpoint{1.266031in}{1.215923in}}%
\pgfpathcurveto{\pgfqpoint{1.260207in}{1.210099in}}{\pgfqpoint{1.256935in}{1.202199in}}{\pgfqpoint{1.256935in}{1.193963in}}%
\pgfpathcurveto{\pgfqpoint{1.256935in}{1.185726in}}{\pgfqpoint{1.260207in}{1.177826in}}{\pgfqpoint{1.266031in}{1.172003in}}%
\pgfpathcurveto{\pgfqpoint{1.271855in}{1.166179in}}{\pgfqpoint{1.279755in}{1.162906in}}{\pgfqpoint{1.287991in}{1.162906in}}%
\pgfpathclose%
\pgfusepath{stroke,fill}%
\end{pgfscope}%
\begin{pgfscope}%
\pgfpathrectangle{\pgfqpoint{0.556847in}{0.516222in}}{\pgfqpoint{1.722590in}{1.783528in}} %
\pgfusepath{clip}%
\pgfsetbuttcap%
\pgfsetroundjoin%
\definecolor{currentfill}{rgb}{0.298039,0.447059,0.690196}%
\pgfsetfillcolor{currentfill}%
\pgfsetlinewidth{0.240900pt}%
\definecolor{currentstroke}{rgb}{1.000000,1.000000,1.000000}%
\pgfsetstrokecolor{currentstroke}%
\pgfsetdash{}{0pt}%
\pgfpathmoveto{\pgfqpoint{1.287991in}{1.131695in}}%
\pgfpathcurveto{\pgfqpoint{1.296227in}{1.131695in}}{\pgfqpoint{1.304127in}{1.134967in}}{\pgfqpoint{1.309951in}{1.140791in}}%
\pgfpathcurveto{\pgfqpoint{1.315775in}{1.146615in}}{\pgfqpoint{1.319048in}{1.154515in}}{\pgfqpoint{1.319048in}{1.162751in}}%
\pgfpathcurveto{\pgfqpoint{1.319048in}{1.170987in}}{\pgfqpoint{1.315775in}{1.178887in}}{\pgfqpoint{1.309951in}{1.184711in}}%
\pgfpathcurveto{\pgfqpoint{1.304127in}{1.190535in}}{\pgfqpoint{1.296227in}{1.193808in}}{\pgfqpoint{1.287991in}{1.193808in}}%
\pgfpathcurveto{\pgfqpoint{1.279755in}{1.193808in}}{\pgfqpoint{1.271855in}{1.190535in}}{\pgfqpoint{1.266031in}{1.184711in}}%
\pgfpathcurveto{\pgfqpoint{1.260207in}{1.178887in}}{\pgfqpoint{1.256935in}{1.170987in}}{\pgfqpoint{1.256935in}{1.162751in}}%
\pgfpathcurveto{\pgfqpoint{1.256935in}{1.154515in}}{\pgfqpoint{1.260207in}{1.146615in}}{\pgfqpoint{1.266031in}{1.140791in}}%
\pgfpathcurveto{\pgfqpoint{1.271855in}{1.134967in}}{\pgfqpoint{1.279755in}{1.131695in}}{\pgfqpoint{1.287991in}{1.131695in}}%
\pgfpathclose%
\pgfusepath{stroke,fill}%
\end{pgfscope}%
\begin{pgfscope}%
\pgfpathrectangle{\pgfqpoint{0.556847in}{0.516222in}}{\pgfqpoint{1.722590in}{1.783528in}} %
\pgfusepath{clip}%
\pgfsetbuttcap%
\pgfsetroundjoin%
\definecolor{currentfill}{rgb}{0.298039,0.447059,0.690196}%
\pgfsetfillcolor{currentfill}%
\pgfsetlinewidth{0.240900pt}%
\definecolor{currentstroke}{rgb}{1.000000,1.000000,1.000000}%
\pgfsetstrokecolor{currentstroke}%
\pgfsetdash{}{0pt}%
\pgfpathmoveto{\pgfqpoint{0.901365in}{0.841871in}}%
\pgfpathcurveto{\pgfqpoint{0.909602in}{0.841871in}}{\pgfqpoint{0.917502in}{0.845144in}}{\pgfqpoint{0.923326in}{0.850968in}}%
\pgfpathcurveto{\pgfqpoint{0.929149in}{0.856791in}}{\pgfqpoint{0.932422in}{0.864691in}}{\pgfqpoint{0.932422in}{0.872928in}}%
\pgfpathcurveto{\pgfqpoint{0.932422in}{0.881164in}}{\pgfqpoint{0.929149in}{0.889064in}}{\pgfqpoint{0.923326in}{0.894888in}}%
\pgfpathcurveto{\pgfqpoint{0.917502in}{0.900712in}}{\pgfqpoint{0.909602in}{0.903984in}}{\pgfqpoint{0.901365in}{0.903984in}}%
\pgfpathcurveto{\pgfqpoint{0.893129in}{0.903984in}}{\pgfqpoint{0.885229in}{0.900712in}}{\pgfqpoint{0.879405in}{0.894888in}}%
\pgfpathcurveto{\pgfqpoint{0.873581in}{0.889064in}}{\pgfqpoint{0.870309in}{0.881164in}}{\pgfqpoint{0.870309in}{0.872928in}}%
\pgfpathcurveto{\pgfqpoint{0.870309in}{0.864691in}}{\pgfqpoint{0.873581in}{0.856791in}}{\pgfqpoint{0.879405in}{0.850968in}}%
\pgfpathcurveto{\pgfqpoint{0.885229in}{0.845144in}}{\pgfqpoint{0.893129in}{0.841871in}}{\pgfqpoint{0.901365in}{0.841871in}}%
\pgfpathclose%
\pgfusepath{stroke,fill}%
\end{pgfscope}%
\begin{pgfscope}%
\pgfpathrectangle{\pgfqpoint{0.556847in}{0.516222in}}{\pgfqpoint{1.722590in}{1.783528in}} %
\pgfusepath{clip}%
\pgfsetbuttcap%
\pgfsetroundjoin%
\definecolor{currentfill}{rgb}{0.298039,0.447059,0.690196}%
\pgfsetfillcolor{currentfill}%
\pgfsetlinewidth{0.240900pt}%
\definecolor{currentstroke}{rgb}{1.000000,1.000000,1.000000}%
\pgfsetstrokecolor{currentstroke}%
\pgfsetdash{}{0pt}%
\pgfpathmoveto{\pgfqpoint{1.012377in}{0.841871in}}%
\pgfpathcurveto{\pgfqpoint{1.020613in}{0.841871in}}{\pgfqpoint{1.028513in}{0.845144in}}{\pgfqpoint{1.034337in}{0.850968in}}%
\pgfpathcurveto{\pgfqpoint{1.040161in}{0.856791in}}{\pgfqpoint{1.043433in}{0.864691in}}{\pgfqpoint{1.043433in}{0.872928in}}%
\pgfpathcurveto{\pgfqpoint{1.043433in}{0.881164in}}{\pgfqpoint{1.040161in}{0.889064in}}{\pgfqpoint{1.034337in}{0.894888in}}%
\pgfpathcurveto{\pgfqpoint{1.028513in}{0.900712in}}{\pgfqpoint{1.020613in}{0.903984in}}{\pgfqpoint{1.012377in}{0.903984in}}%
\pgfpathcurveto{\pgfqpoint{1.004140in}{0.903984in}}{\pgfqpoint{0.996240in}{0.900712in}}{\pgfqpoint{0.990416in}{0.894888in}}%
\pgfpathcurveto{\pgfqpoint{0.984592in}{0.889064in}}{\pgfqpoint{0.981320in}{0.881164in}}{\pgfqpoint{0.981320in}{0.872928in}}%
\pgfpathcurveto{\pgfqpoint{0.981320in}{0.864691in}}{\pgfqpoint{0.984592in}{0.856791in}}{\pgfqpoint{0.990416in}{0.850968in}}%
\pgfpathcurveto{\pgfqpoint{0.996240in}{0.845144in}}{\pgfqpoint{1.004140in}{0.841871in}}{\pgfqpoint{1.012377in}{0.841871in}}%
\pgfpathclose%
\pgfusepath{stroke,fill}%
\end{pgfscope}%
\begin{pgfscope}%
\pgfpathrectangle{\pgfqpoint{0.556847in}{0.516222in}}{\pgfqpoint{1.722590in}{1.783528in}} %
\pgfusepath{clip}%
\pgfsetbuttcap%
\pgfsetroundjoin%
\definecolor{currentfill}{rgb}{0.298039,0.447059,0.690196}%
\pgfsetfillcolor{currentfill}%
\pgfsetlinewidth{0.240900pt}%
\definecolor{currentstroke}{rgb}{1.000000,1.000000,1.000000}%
\pgfsetstrokecolor{currentstroke}%
\pgfsetdash{}{0pt}%
\pgfpathmoveto{\pgfqpoint{0.755902in}{0.618930in}}%
\pgfpathcurveto{\pgfqpoint{0.764138in}{0.618930in}}{\pgfqpoint{0.772038in}{0.622203in}}{\pgfqpoint{0.777862in}{0.628027in}}%
\pgfpathcurveto{\pgfqpoint{0.783686in}{0.633850in}}{\pgfqpoint{0.786959in}{0.641751in}}{\pgfqpoint{0.786959in}{0.649987in}}%
\pgfpathcurveto{\pgfqpoint{0.786959in}{0.658223in}}{\pgfqpoint{0.783686in}{0.666123in}}{\pgfqpoint{0.777862in}{0.671947in}}%
\pgfpathcurveto{\pgfqpoint{0.772038in}{0.677771in}}{\pgfqpoint{0.764138in}{0.681043in}}{\pgfqpoint{0.755902in}{0.681043in}}%
\pgfpathcurveto{\pgfqpoint{0.747666in}{0.681043in}}{\pgfqpoint{0.739766in}{0.677771in}}{\pgfqpoint{0.733942in}{0.671947in}}%
\pgfpathcurveto{\pgfqpoint{0.728118in}{0.666123in}}{\pgfqpoint{0.724846in}{0.658223in}}{\pgfqpoint{0.724846in}{0.649987in}}%
\pgfpathcurveto{\pgfqpoint{0.724846in}{0.641751in}}{\pgfqpoint{0.728118in}{0.633850in}}{\pgfqpoint{0.733942in}{0.628027in}}%
\pgfpathcurveto{\pgfqpoint{0.739766in}{0.622203in}}{\pgfqpoint{0.747666in}{0.618930in}}{\pgfqpoint{0.755902in}{0.618930in}}%
\pgfpathclose%
\pgfusepath{stroke,fill}%
\end{pgfscope}%
\begin{pgfscope}%
\pgfpathrectangle{\pgfqpoint{0.556847in}{0.516222in}}{\pgfqpoint{1.722590in}{1.783528in}} %
\pgfusepath{clip}%
\pgfsetbuttcap%
\pgfsetroundjoin%
\definecolor{currentfill}{rgb}{0.298039,0.447059,0.690196}%
\pgfsetfillcolor{currentfill}%
\pgfsetlinewidth{0.240900pt}%
\definecolor{currentstroke}{rgb}{1.000000,1.000000,1.000000}%
\pgfsetstrokecolor{currentstroke}%
\pgfsetdash{}{0pt}%
\pgfpathmoveto{\pgfqpoint{0.985581in}{1.055895in}}%
\pgfpathcurveto{\pgfqpoint{0.993817in}{1.055895in}}{\pgfqpoint{1.001717in}{1.059167in}}{\pgfqpoint{1.007541in}{1.064991in}}%
\pgfpathcurveto{\pgfqpoint{1.013365in}{1.070815in}}{\pgfqpoint{1.016637in}{1.078715in}}{\pgfqpoint{1.016637in}{1.086951in}}%
\pgfpathcurveto{\pgfqpoint{1.016637in}{1.095187in}}{\pgfqpoint{1.013365in}{1.103087in}}{\pgfqpoint{1.007541in}{1.108911in}}%
\pgfpathcurveto{\pgfqpoint{1.001717in}{1.114735in}}{\pgfqpoint{0.993817in}{1.118008in}}{\pgfqpoint{0.985581in}{1.118008in}}%
\pgfpathcurveto{\pgfqpoint{0.977345in}{1.118008in}}{\pgfqpoint{0.969444in}{1.114735in}}{\pgfqpoint{0.963621in}{1.108911in}}%
\pgfpathcurveto{\pgfqpoint{0.957797in}{1.103087in}}{\pgfqpoint{0.954524in}{1.095187in}}{\pgfqpoint{0.954524in}{1.086951in}}%
\pgfpathcurveto{\pgfqpoint{0.954524in}{1.078715in}}{\pgfqpoint{0.957797in}{1.070815in}}{\pgfqpoint{0.963621in}{1.064991in}}%
\pgfpathcurveto{\pgfqpoint{0.969444in}{1.059167in}}{\pgfqpoint{0.977345in}{1.055895in}}{\pgfqpoint{0.985581in}{1.055895in}}%
\pgfpathclose%
\pgfusepath{stroke,fill}%
\end{pgfscope}%
\begin{pgfscope}%
\pgfpathrectangle{\pgfqpoint{0.556847in}{0.516222in}}{\pgfqpoint{1.722590in}{1.783528in}} %
\pgfusepath{clip}%
\pgfsetbuttcap%
\pgfsetroundjoin%
\definecolor{currentfill}{rgb}{0.298039,0.447059,0.690196}%
\pgfsetfillcolor{currentfill}%
\pgfsetlinewidth{0.240900pt}%
\definecolor{currentstroke}{rgb}{1.000000,1.000000,1.000000}%
\pgfsetstrokecolor{currentstroke}%
\pgfsetdash{}{0pt}%
\pgfpathmoveto{\pgfqpoint{1.460250in}{1.448271in}}%
\pgfpathcurveto{\pgfqpoint{1.468486in}{1.448271in}}{\pgfqpoint{1.476386in}{1.451543in}}{\pgfqpoint{1.482210in}{1.457367in}}%
\pgfpathcurveto{\pgfqpoint{1.488034in}{1.463191in}}{\pgfqpoint{1.491307in}{1.471091in}}{\pgfqpoint{1.491307in}{1.479327in}}%
\pgfpathcurveto{\pgfqpoint{1.491307in}{1.487564in}}{\pgfqpoint{1.488034in}{1.495464in}}{\pgfqpoint{1.482210in}{1.501287in}}%
\pgfpathcurveto{\pgfqpoint{1.476386in}{1.507111in}}{\pgfqpoint{1.468486in}{1.510384in}}{\pgfqpoint{1.460250in}{1.510384in}}%
\pgfpathcurveto{\pgfqpoint{1.452014in}{1.510384in}}{\pgfqpoint{1.444114in}{1.507111in}}{\pgfqpoint{1.438290in}{1.501287in}}%
\pgfpathcurveto{\pgfqpoint{1.432466in}{1.495464in}}{\pgfqpoint{1.429194in}{1.487564in}}{\pgfqpoint{1.429194in}{1.479327in}}%
\pgfpathcurveto{\pgfqpoint{1.429194in}{1.471091in}}{\pgfqpoint{1.432466in}{1.463191in}}{\pgfqpoint{1.438290in}{1.457367in}}%
\pgfpathcurveto{\pgfqpoint{1.444114in}{1.451543in}}{\pgfqpoint{1.452014in}{1.448271in}}{\pgfqpoint{1.460250in}{1.448271in}}%
\pgfpathclose%
\pgfusepath{stroke,fill}%
\end{pgfscope}%
\begin{pgfscope}%
\pgfpathrectangle{\pgfqpoint{0.556847in}{0.516222in}}{\pgfqpoint{1.722590in}{1.783528in}} %
\pgfusepath{clip}%
\pgfsetbuttcap%
\pgfsetroundjoin%
\definecolor{currentfill}{rgb}{0.298039,0.447059,0.690196}%
\pgfsetfillcolor{currentfill}%
\pgfsetlinewidth{0.240900pt}%
\definecolor{currentstroke}{rgb}{1.000000,1.000000,1.000000}%
\pgfsetstrokecolor{currentstroke}%
\pgfsetdash{}{0pt}%
\pgfpathmoveto{\pgfqpoint{1.850704in}{1.590953in}}%
\pgfpathcurveto{\pgfqpoint{1.858940in}{1.590953in}}{\pgfqpoint{1.866840in}{1.594225in}}{\pgfqpoint{1.872664in}{1.600049in}}%
\pgfpathcurveto{\pgfqpoint{1.878488in}{1.605873in}}{\pgfqpoint{1.881760in}{1.613773in}}{\pgfqpoint{1.881760in}{1.622009in}}%
\pgfpathcurveto{\pgfqpoint{1.881760in}{1.630246in}}{\pgfqpoint{1.878488in}{1.638146in}}{\pgfqpoint{1.872664in}{1.643970in}}%
\pgfpathcurveto{\pgfqpoint{1.866840in}{1.649794in}}{\pgfqpoint{1.858940in}{1.653066in}}{\pgfqpoint{1.850704in}{1.653066in}}%
\pgfpathcurveto{\pgfqpoint{1.842468in}{1.653066in}}{\pgfqpoint{1.834568in}{1.649794in}}{\pgfqpoint{1.828744in}{1.643970in}}%
\pgfpathcurveto{\pgfqpoint{1.822920in}{1.638146in}}{\pgfqpoint{1.819647in}{1.630246in}}{\pgfqpoint{1.819647in}{1.622009in}}%
\pgfpathcurveto{\pgfqpoint{1.819647in}{1.613773in}}{\pgfqpoint{1.822920in}{1.605873in}}{\pgfqpoint{1.828744in}{1.600049in}}%
\pgfpathcurveto{\pgfqpoint{1.834568in}{1.594225in}}{\pgfqpoint{1.842468in}{1.590953in}}{\pgfqpoint{1.850704in}{1.590953in}}%
\pgfpathclose%
\pgfusepath{stroke,fill}%
\end{pgfscope}%
\begin{pgfscope}%
\pgfpathrectangle{\pgfqpoint{0.556847in}{0.516222in}}{\pgfqpoint{1.722590in}{1.783528in}} %
\pgfusepath{clip}%
\pgfsetbuttcap%
\pgfsetroundjoin%
\definecolor{currentfill}{rgb}{0.298039,0.447059,0.690196}%
\pgfsetfillcolor{currentfill}%
\pgfsetlinewidth{0.240900pt}%
\definecolor{currentstroke}{rgb}{1.000000,1.000000,1.000000}%
\pgfsetstrokecolor{currentstroke}%
\pgfsetdash{}{0pt}%
\pgfpathmoveto{\pgfqpoint{1.150184in}{0.989012in}}%
\pgfpathcurveto{\pgfqpoint{1.158420in}{0.989012in}}{\pgfqpoint{1.166320in}{0.992285in}}{\pgfqpoint{1.172144in}{0.998109in}}%
\pgfpathcurveto{\pgfqpoint{1.177968in}{1.003932in}}{\pgfqpoint{1.181240in}{1.011833in}}{\pgfqpoint{1.181240in}{1.020069in}}%
\pgfpathcurveto{\pgfqpoint{1.181240in}{1.028305in}}{\pgfqpoint{1.177968in}{1.036205in}}{\pgfqpoint{1.172144in}{1.042029in}}%
\pgfpathcurveto{\pgfqpoint{1.166320in}{1.047853in}}{\pgfqpoint{1.158420in}{1.051125in}}{\pgfqpoint{1.150184in}{1.051125in}}%
\pgfpathcurveto{\pgfqpoint{1.141948in}{1.051125in}}{\pgfqpoint{1.134048in}{1.047853in}}{\pgfqpoint{1.128224in}{1.042029in}}%
\pgfpathcurveto{\pgfqpoint{1.122400in}{1.036205in}}{\pgfqpoint{1.119127in}{1.028305in}}{\pgfqpoint{1.119127in}{1.020069in}}%
\pgfpathcurveto{\pgfqpoint{1.119127in}{1.011833in}}{\pgfqpoint{1.122400in}{1.003932in}}{\pgfqpoint{1.128224in}{0.998109in}}%
\pgfpathcurveto{\pgfqpoint{1.134048in}{0.992285in}}{\pgfqpoint{1.141948in}{0.989012in}}{\pgfqpoint{1.150184in}{0.989012in}}%
\pgfpathclose%
\pgfusepath{stroke,fill}%
\end{pgfscope}%
\begin{pgfscope}%
\pgfpathrectangle{\pgfqpoint{0.556847in}{0.516222in}}{\pgfqpoint{1.722590in}{1.783528in}} %
\pgfusepath{clip}%
\pgfsetbuttcap%
\pgfsetroundjoin%
\definecolor{currentfill}{rgb}{0.298039,0.447059,0.690196}%
\pgfsetfillcolor{currentfill}%
\pgfsetlinewidth{0.240900pt}%
\definecolor{currentstroke}{rgb}{1.000000,1.000000,1.000000}%
\pgfsetstrokecolor{currentstroke}%
\pgfsetdash{}{0pt}%
\pgfpathmoveto{\pgfqpoint{1.154012in}{1.002389in}}%
\pgfpathcurveto{\pgfqpoint{1.162248in}{1.002389in}}{\pgfqpoint{1.170148in}{1.005661in}}{\pgfqpoint{1.175972in}{1.011485in}}%
\pgfpathcurveto{\pgfqpoint{1.181796in}{1.017309in}}{\pgfqpoint{1.185068in}{1.025209in}}{\pgfqpoint{1.185068in}{1.033445in}}%
\pgfpathcurveto{\pgfqpoint{1.185068in}{1.041682in}}{\pgfqpoint{1.181796in}{1.049582in}}{\pgfqpoint{1.175972in}{1.055406in}}%
\pgfpathcurveto{\pgfqpoint{1.170148in}{1.061229in}}{\pgfqpoint{1.162248in}{1.064502in}}{\pgfqpoint{1.154012in}{1.064502in}}%
\pgfpathcurveto{\pgfqpoint{1.145776in}{1.064502in}}{\pgfqpoint{1.137876in}{1.061229in}}{\pgfqpoint{1.132052in}{1.055406in}}%
\pgfpathcurveto{\pgfqpoint{1.126228in}{1.049582in}}{\pgfqpoint{1.122955in}{1.041682in}}{\pgfqpoint{1.122955in}{1.033445in}}%
\pgfpathcurveto{\pgfqpoint{1.122955in}{1.025209in}}{\pgfqpoint{1.126228in}{1.017309in}}{\pgfqpoint{1.132052in}{1.011485in}}%
\pgfpathcurveto{\pgfqpoint{1.137876in}{1.005661in}}{\pgfqpoint{1.145776in}{1.002389in}}{\pgfqpoint{1.154012in}{1.002389in}}%
\pgfpathclose%
\pgfusepath{stroke,fill}%
\end{pgfscope}%
\begin{pgfscope}%
\pgfpathrectangle{\pgfqpoint{0.556847in}{0.516222in}}{\pgfqpoint{1.722590in}{1.783528in}} %
\pgfusepath{clip}%
\pgfsetbuttcap%
\pgfsetroundjoin%
\definecolor{currentfill}{rgb}{0.298039,0.447059,0.690196}%
\pgfsetfillcolor{currentfill}%
\pgfsetlinewidth{0.240900pt}%
\definecolor{currentstroke}{rgb}{1.000000,1.000000,1.000000}%
\pgfsetstrokecolor{currentstroke}%
\pgfsetdash{}{0pt}%
\pgfpathmoveto{\pgfqpoint{0.790354in}{0.690271in}}%
\pgfpathcurveto{\pgfqpoint{0.798590in}{0.690271in}}{\pgfqpoint{0.806490in}{0.693544in}}{\pgfqpoint{0.812314in}{0.699368in}}%
\pgfpathcurveto{\pgfqpoint{0.818138in}{0.705192in}}{\pgfqpoint{0.821410in}{0.713092in}}{\pgfqpoint{0.821410in}{0.721328in}}%
\pgfpathcurveto{\pgfqpoint{0.821410in}{0.729564in}}{\pgfqpoint{0.818138in}{0.737464in}}{\pgfqpoint{0.812314in}{0.743288in}}%
\pgfpathcurveto{\pgfqpoint{0.806490in}{0.749112in}}{\pgfqpoint{0.798590in}{0.752384in}}{\pgfqpoint{0.790354in}{0.752384in}}%
\pgfpathcurveto{\pgfqpoint{0.782118in}{0.752384in}}{\pgfqpoint{0.774218in}{0.749112in}}{\pgfqpoint{0.768394in}{0.743288in}}%
\pgfpathcurveto{\pgfqpoint{0.762570in}{0.737464in}}{\pgfqpoint{0.759297in}{0.729564in}}{\pgfqpoint{0.759297in}{0.721328in}}%
\pgfpathcurveto{\pgfqpoint{0.759297in}{0.713092in}}{\pgfqpoint{0.762570in}{0.705192in}}{\pgfqpoint{0.768394in}{0.699368in}}%
\pgfpathcurveto{\pgfqpoint{0.774218in}{0.693544in}}{\pgfqpoint{0.782118in}{0.690271in}}{\pgfqpoint{0.790354in}{0.690271in}}%
\pgfpathclose%
\pgfusepath{stroke,fill}%
\end{pgfscope}%
\begin{pgfscope}%
\pgfpathrectangle{\pgfqpoint{0.556847in}{0.516222in}}{\pgfqpoint{1.722590in}{1.783528in}} %
\pgfusepath{clip}%
\pgfsetbuttcap%
\pgfsetroundjoin%
\definecolor{currentfill}{rgb}{0.298039,0.447059,0.690196}%
\pgfsetfillcolor{currentfill}%
\pgfsetlinewidth{0.240900pt}%
\definecolor{currentstroke}{rgb}{1.000000,1.000000,1.000000}%
\pgfsetstrokecolor{currentstroke}%
\pgfsetdash{}{0pt}%
\pgfpathmoveto{\pgfqpoint{1.487046in}{1.755929in}}%
\pgfpathcurveto{\pgfqpoint{1.495282in}{1.755929in}}{\pgfqpoint{1.503182in}{1.759202in}}{\pgfqpoint{1.509006in}{1.765026in}}%
\pgfpathcurveto{\pgfqpoint{1.514830in}{1.770849in}}{\pgfqpoint{1.518102in}{1.778749in}}{\pgfqpoint{1.518102in}{1.786986in}}%
\pgfpathcurveto{\pgfqpoint{1.518102in}{1.795222in}}{\pgfqpoint{1.514830in}{1.803122in}}{\pgfqpoint{1.509006in}{1.808946in}}%
\pgfpathcurveto{\pgfqpoint{1.503182in}{1.814770in}}{\pgfqpoint{1.495282in}{1.818042in}}{\pgfqpoint{1.487046in}{1.818042in}}%
\pgfpathcurveto{\pgfqpoint{1.478810in}{1.818042in}}{\pgfqpoint{1.470910in}{1.814770in}}{\pgfqpoint{1.465086in}{1.808946in}}%
\pgfpathcurveto{\pgfqpoint{1.459262in}{1.803122in}}{\pgfqpoint{1.455989in}{1.795222in}}{\pgfqpoint{1.455989in}{1.786986in}}%
\pgfpathcurveto{\pgfqpoint{1.455989in}{1.778749in}}{\pgfqpoint{1.459262in}{1.770849in}}{\pgfqpoint{1.465086in}{1.765026in}}%
\pgfpathcurveto{\pgfqpoint{1.470910in}{1.759202in}}{\pgfqpoint{1.478810in}{1.755929in}}{\pgfqpoint{1.487046in}{1.755929in}}%
\pgfpathclose%
\pgfusepath{stroke,fill}%
\end{pgfscope}%
\begin{pgfscope}%
\pgfpathrectangle{\pgfqpoint{0.556847in}{0.516222in}}{\pgfqpoint{1.722590in}{1.783528in}} %
\pgfusepath{clip}%
\pgfsetbuttcap%
\pgfsetroundjoin%
\definecolor{currentfill}{rgb}{0.298039,0.447059,0.690196}%
\pgfsetfillcolor{currentfill}%
\pgfsetlinewidth{0.240900pt}%
\definecolor{currentstroke}{rgb}{1.000000,1.000000,1.000000}%
\pgfsetstrokecolor{currentstroke}%
\pgfsetdash{}{0pt}%
\pgfpathmoveto{\pgfqpoint{1.161668in}{0.980095in}}%
\pgfpathcurveto{\pgfqpoint{1.169904in}{0.980095in}}{\pgfqpoint{1.177804in}{0.983367in}}{\pgfqpoint{1.183628in}{0.989191in}}%
\pgfpathcurveto{\pgfqpoint{1.189452in}{0.995015in}}{\pgfqpoint{1.192724in}{1.002915in}}{\pgfqpoint{1.192724in}{1.011151in}}%
\pgfpathcurveto{\pgfqpoint{1.192724in}{1.019387in}}{\pgfqpoint{1.189452in}{1.027288in}}{\pgfqpoint{1.183628in}{1.033111in}}%
\pgfpathcurveto{\pgfqpoint{1.177804in}{1.038935in}}{\pgfqpoint{1.169904in}{1.042208in}}{\pgfqpoint{1.161668in}{1.042208in}}%
\pgfpathcurveto{\pgfqpoint{1.153432in}{1.042208in}}{\pgfqpoint{1.145531in}{1.038935in}}{\pgfqpoint{1.139708in}{1.033111in}}%
\pgfpathcurveto{\pgfqpoint{1.133884in}{1.027288in}}{\pgfqpoint{1.130611in}{1.019387in}}{\pgfqpoint{1.130611in}{1.011151in}}%
\pgfpathcurveto{\pgfqpoint{1.130611in}{1.002915in}}{\pgfqpoint{1.133884in}{0.995015in}}{\pgfqpoint{1.139708in}{0.989191in}}%
\pgfpathcurveto{\pgfqpoint{1.145531in}{0.983367in}}{\pgfqpoint{1.153432in}{0.980095in}}{\pgfqpoint{1.161668in}{0.980095in}}%
\pgfpathclose%
\pgfusepath{stroke,fill}%
\end{pgfscope}%
\begin{pgfscope}%
\pgfpathrectangle{\pgfqpoint{0.556847in}{0.516222in}}{\pgfqpoint{1.722590in}{1.783528in}} %
\pgfusepath{clip}%
\pgfsetbuttcap%
\pgfsetroundjoin%
\definecolor{currentfill}{rgb}{0.298039,0.447059,0.690196}%
\pgfsetfillcolor{currentfill}%
\pgfsetlinewidth{0.240900pt}%
\definecolor{currentstroke}{rgb}{1.000000,1.000000,1.000000}%
\pgfsetstrokecolor{currentstroke}%
\pgfsetdash{}{0pt}%
\pgfpathmoveto{\pgfqpoint{0.721450in}{0.779448in}}%
\pgfpathcurveto{\pgfqpoint{0.729687in}{0.779448in}}{\pgfqpoint{0.737587in}{0.782720in}}{\pgfqpoint{0.743411in}{0.788544in}}%
\pgfpathcurveto{\pgfqpoint{0.749234in}{0.794368in}}{\pgfqpoint{0.752507in}{0.802268in}}{\pgfqpoint{0.752507in}{0.810504in}}%
\pgfpathcurveto{\pgfqpoint{0.752507in}{0.818741in}}{\pgfqpoint{0.749234in}{0.826641in}}{\pgfqpoint{0.743411in}{0.832465in}}%
\pgfpathcurveto{\pgfqpoint{0.737587in}{0.838288in}}{\pgfqpoint{0.729687in}{0.841561in}}{\pgfqpoint{0.721450in}{0.841561in}}%
\pgfpathcurveto{\pgfqpoint{0.713214in}{0.841561in}}{\pgfqpoint{0.705314in}{0.838288in}}{\pgfqpoint{0.699490in}{0.832465in}}%
\pgfpathcurveto{\pgfqpoint{0.693666in}{0.826641in}}{\pgfqpoint{0.690394in}{0.818741in}}{\pgfqpoint{0.690394in}{0.810504in}}%
\pgfpathcurveto{\pgfqpoint{0.690394in}{0.802268in}}{\pgfqpoint{0.693666in}{0.794368in}}{\pgfqpoint{0.699490in}{0.788544in}}%
\pgfpathcurveto{\pgfqpoint{0.705314in}{0.782720in}}{\pgfqpoint{0.713214in}{0.779448in}}{\pgfqpoint{0.721450in}{0.779448in}}%
\pgfpathclose%
\pgfusepath{stroke,fill}%
\end{pgfscope}%
\begin{pgfscope}%
\pgfpathrectangle{\pgfqpoint{0.556847in}{0.516222in}}{\pgfqpoint{1.722590in}{1.783528in}} %
\pgfusepath{clip}%
\pgfsetbuttcap%
\pgfsetroundjoin%
\definecolor{currentfill}{rgb}{0.298039,0.447059,0.690196}%
\pgfsetfillcolor{currentfill}%
\pgfsetlinewidth{0.240900pt}%
\definecolor{currentstroke}{rgb}{1.000000,1.000000,1.000000}%
\pgfsetstrokecolor{currentstroke}%
\pgfsetdash{}{0pt}%
\pgfpathmoveto{\pgfqpoint{1.008549in}{0.792824in}}%
\pgfpathcurveto{\pgfqpoint{1.016785in}{0.792824in}}{\pgfqpoint{1.024685in}{0.796097in}}{\pgfqpoint{1.030509in}{0.801921in}}%
\pgfpathcurveto{\pgfqpoint{1.036333in}{0.807744in}}{\pgfqpoint{1.039605in}{0.815644in}}{\pgfqpoint{1.039605in}{0.823881in}}%
\pgfpathcurveto{\pgfqpoint{1.039605in}{0.832117in}}{\pgfqpoint{1.036333in}{0.840017in}}{\pgfqpoint{1.030509in}{0.845841in}}%
\pgfpathcurveto{\pgfqpoint{1.024685in}{0.851665in}}{\pgfqpoint{1.016785in}{0.854937in}}{\pgfqpoint{1.008549in}{0.854937in}}%
\pgfpathcurveto{\pgfqpoint{1.000312in}{0.854937in}}{\pgfqpoint{0.992412in}{0.851665in}}{\pgfqpoint{0.986588in}{0.845841in}}%
\pgfpathcurveto{\pgfqpoint{0.980764in}{0.840017in}}{\pgfqpoint{0.977492in}{0.832117in}}{\pgfqpoint{0.977492in}{0.823881in}}%
\pgfpathcurveto{\pgfqpoint{0.977492in}{0.815644in}}{\pgfqpoint{0.980764in}{0.807744in}}{\pgfqpoint{0.986588in}{0.801921in}}%
\pgfpathcurveto{\pgfqpoint{0.992412in}{0.796097in}}{\pgfqpoint{1.000312in}{0.792824in}}{\pgfqpoint{1.008549in}{0.792824in}}%
\pgfpathclose%
\pgfusepath{stroke,fill}%
\end{pgfscope}%
\begin{pgfscope}%
\pgfpathrectangle{\pgfqpoint{0.556847in}{0.516222in}}{\pgfqpoint{1.722590in}{1.783528in}} %
\pgfusepath{clip}%
\pgfsetbuttcap%
\pgfsetroundjoin%
\definecolor{currentfill}{rgb}{0.298039,0.447059,0.690196}%
\pgfsetfillcolor{currentfill}%
\pgfsetlinewidth{0.240900pt}%
\definecolor{currentstroke}{rgb}{1.000000,1.000000,1.000000}%
\pgfsetstrokecolor{currentstroke}%
\pgfsetdash{}{0pt}%
\pgfpathmoveto{\pgfqpoint{1.663133in}{1.671212in}}%
\pgfpathcurveto{\pgfqpoint{1.671369in}{1.671212in}}{\pgfqpoint{1.679269in}{1.674484in}}{\pgfqpoint{1.685093in}{1.680308in}}%
\pgfpathcurveto{\pgfqpoint{1.690917in}{1.686132in}}{\pgfqpoint{1.694189in}{1.694032in}}{\pgfqpoint{1.694189in}{1.702268in}}%
\pgfpathcurveto{\pgfqpoint{1.694189in}{1.710504in}}{\pgfqpoint{1.690917in}{1.718405in}}{\pgfqpoint{1.685093in}{1.724228in}}%
\pgfpathcurveto{\pgfqpoint{1.679269in}{1.730052in}}{\pgfqpoint{1.671369in}{1.733325in}}{\pgfqpoint{1.663133in}{1.733325in}}%
\pgfpathcurveto{\pgfqpoint{1.654897in}{1.733325in}}{\pgfqpoint{1.646997in}{1.730052in}}{\pgfqpoint{1.641173in}{1.724228in}}%
\pgfpathcurveto{\pgfqpoint{1.635349in}{1.718405in}}{\pgfqpoint{1.632076in}{1.710504in}}{\pgfqpoint{1.632076in}{1.702268in}}%
\pgfpathcurveto{\pgfqpoint{1.632076in}{1.694032in}}{\pgfqpoint{1.635349in}{1.686132in}}{\pgfqpoint{1.641173in}{1.680308in}}%
\pgfpathcurveto{\pgfqpoint{1.646997in}{1.674484in}}{\pgfqpoint{1.654897in}{1.671212in}}{\pgfqpoint{1.663133in}{1.671212in}}%
\pgfpathclose%
\pgfusepath{stroke,fill}%
\end{pgfscope}%
\begin{pgfscope}%
\pgfpathrectangle{\pgfqpoint{0.556847in}{0.516222in}}{\pgfqpoint{1.722590in}{1.783528in}} %
\pgfusepath{clip}%
\pgfsetbuttcap%
\pgfsetroundjoin%
\definecolor{currentfill}{rgb}{0.298039,0.447059,0.690196}%
\pgfsetfillcolor{currentfill}%
\pgfsetlinewidth{0.240900pt}%
\definecolor{currentstroke}{rgb}{1.000000,1.000000,1.000000}%
\pgfsetstrokecolor{currentstroke}%
\pgfsetdash{}{0pt}%
\pgfpathmoveto{\pgfqpoint{0.870741in}{0.659060in}}%
\pgfpathcurveto{\pgfqpoint{0.878978in}{0.659060in}}{\pgfqpoint{0.886878in}{0.662332in}}{\pgfqpoint{0.892702in}{0.668156in}}%
\pgfpathcurveto{\pgfqpoint{0.898526in}{0.673980in}}{\pgfqpoint{0.901798in}{0.681880in}}{\pgfqpoint{0.901798in}{0.690116in}}%
\pgfpathcurveto{\pgfqpoint{0.901798in}{0.698352in}}{\pgfqpoint{0.898526in}{0.706253in}}{\pgfqpoint{0.892702in}{0.712076in}}%
\pgfpathcurveto{\pgfqpoint{0.886878in}{0.717900in}}{\pgfqpoint{0.878978in}{0.721173in}}{\pgfqpoint{0.870741in}{0.721173in}}%
\pgfpathcurveto{\pgfqpoint{0.862505in}{0.721173in}}{\pgfqpoint{0.854605in}{0.717900in}}{\pgfqpoint{0.848781in}{0.712076in}}%
\pgfpathcurveto{\pgfqpoint{0.842957in}{0.706253in}}{\pgfqpoint{0.839685in}{0.698352in}}{\pgfqpoint{0.839685in}{0.690116in}}%
\pgfpathcurveto{\pgfqpoint{0.839685in}{0.681880in}}{\pgfqpoint{0.842957in}{0.673980in}}{\pgfqpoint{0.848781in}{0.668156in}}%
\pgfpathcurveto{\pgfqpoint{0.854605in}{0.662332in}}{\pgfqpoint{0.862505in}{0.659060in}}{\pgfqpoint{0.870741in}{0.659060in}}%
\pgfpathclose%
\pgfusepath{stroke,fill}%
\end{pgfscope}%
\begin{pgfscope}%
\pgfpathrectangle{\pgfqpoint{0.556847in}{0.516222in}}{\pgfqpoint{1.722590in}{1.783528in}} %
\pgfusepath{clip}%
\pgfsetbuttcap%
\pgfsetroundjoin%
\definecolor{currentfill}{rgb}{0.298039,0.447059,0.690196}%
\pgfsetfillcolor{currentfill}%
\pgfsetlinewidth{0.240900pt}%
\definecolor{currentstroke}{rgb}{1.000000,1.000000,1.000000}%
\pgfsetstrokecolor{currentstroke}%
\pgfsetdash{}{0pt}%
\pgfpathmoveto{\pgfqpoint{0.843946in}{0.766071in}}%
\pgfpathcurveto{\pgfqpoint{0.852182in}{0.766071in}}{\pgfqpoint{0.860082in}{0.769344in}}{\pgfqpoint{0.865906in}{0.775168in}}%
\pgfpathcurveto{\pgfqpoint{0.871730in}{0.780992in}}{\pgfqpoint{0.875002in}{0.788892in}}{\pgfqpoint{0.875002in}{0.797128in}}%
\pgfpathcurveto{\pgfqpoint{0.875002in}{0.805364in}}{\pgfqpoint{0.871730in}{0.813264in}}{\pgfqpoint{0.865906in}{0.819088in}}%
\pgfpathcurveto{\pgfqpoint{0.860082in}{0.824912in}}{\pgfqpoint{0.852182in}{0.828184in}}{\pgfqpoint{0.843946in}{0.828184in}}%
\pgfpathcurveto{\pgfqpoint{0.835709in}{0.828184in}}{\pgfqpoint{0.827809in}{0.824912in}}{\pgfqpoint{0.821985in}{0.819088in}}%
\pgfpathcurveto{\pgfqpoint{0.816161in}{0.813264in}}{\pgfqpoint{0.812889in}{0.805364in}}{\pgfqpoint{0.812889in}{0.797128in}}%
\pgfpathcurveto{\pgfqpoint{0.812889in}{0.788892in}}{\pgfqpoint{0.816161in}{0.780992in}}{\pgfqpoint{0.821985in}{0.775168in}}%
\pgfpathcurveto{\pgfqpoint{0.827809in}{0.769344in}}{\pgfqpoint{0.835709in}{0.766071in}}{\pgfqpoint{0.843946in}{0.766071in}}%
\pgfpathclose%
\pgfusepath{stroke,fill}%
\end{pgfscope}%
\begin{pgfscope}%
\pgfpathrectangle{\pgfqpoint{0.556847in}{0.516222in}}{\pgfqpoint{1.722590in}{1.783528in}} %
\pgfusepath{clip}%
\pgfsetbuttcap%
\pgfsetroundjoin%
\definecolor{currentfill}{rgb}{0.298039,0.447059,0.690196}%
\pgfsetfillcolor{currentfill}%
\pgfsetlinewidth{0.240900pt}%
\definecolor{currentstroke}{rgb}{1.000000,1.000000,1.000000}%
\pgfsetstrokecolor{currentstroke}%
\pgfsetdash{}{0pt}%
\pgfpathmoveto{\pgfqpoint{1.261195in}{1.100483in}}%
\pgfpathcurveto{\pgfqpoint{1.269432in}{1.100483in}}{\pgfqpoint{1.277332in}{1.103755in}}{\pgfqpoint{1.283156in}{1.109579in}}%
\pgfpathcurveto{\pgfqpoint{1.288979in}{1.115403in}}{\pgfqpoint{1.292252in}{1.123303in}}{\pgfqpoint{1.292252in}{1.131539in}}%
\pgfpathcurveto{\pgfqpoint{1.292252in}{1.139776in}}{\pgfqpoint{1.288979in}{1.147676in}}{\pgfqpoint{1.283156in}{1.153500in}}%
\pgfpathcurveto{\pgfqpoint{1.277332in}{1.159323in}}{\pgfqpoint{1.269432in}{1.162596in}}{\pgfqpoint{1.261195in}{1.162596in}}%
\pgfpathcurveto{\pgfqpoint{1.252959in}{1.162596in}}{\pgfqpoint{1.245059in}{1.159323in}}{\pgfqpoint{1.239235in}{1.153500in}}%
\pgfpathcurveto{\pgfqpoint{1.233411in}{1.147676in}}{\pgfqpoint{1.230139in}{1.139776in}}{\pgfqpoint{1.230139in}{1.131539in}}%
\pgfpathcurveto{\pgfqpoint{1.230139in}{1.123303in}}{\pgfqpoint{1.233411in}{1.115403in}}{\pgfqpoint{1.239235in}{1.109579in}}%
\pgfpathcurveto{\pgfqpoint{1.245059in}{1.103755in}}{\pgfqpoint{1.252959in}{1.100483in}}{\pgfqpoint{1.261195in}{1.100483in}}%
\pgfpathclose%
\pgfusepath{stroke,fill}%
\end{pgfscope}%
\begin{pgfscope}%
\pgfpathrectangle{\pgfqpoint{0.556847in}{0.516222in}}{\pgfqpoint{1.722590in}{1.783528in}} %
\pgfusepath{clip}%
\pgfsetbuttcap%
\pgfsetroundjoin%
\definecolor{currentfill}{rgb}{0.298039,0.447059,0.690196}%
\pgfsetfillcolor{currentfill}%
\pgfsetlinewidth{0.240900pt}%
\definecolor{currentstroke}{rgb}{1.000000,1.000000,1.000000}%
\pgfsetstrokecolor{currentstroke}%
\pgfsetdash{}{0pt}%
\pgfpathmoveto{\pgfqpoint{1.115732in}{0.855248in}}%
\pgfpathcurveto{\pgfqpoint{1.123968in}{0.855248in}}{\pgfqpoint{1.131868in}{0.858520in}}{\pgfqpoint{1.137692in}{0.864344in}}%
\pgfpathcurveto{\pgfqpoint{1.143516in}{0.870168in}}{\pgfqpoint{1.146789in}{0.878068in}}{\pgfqpoint{1.146789in}{0.886304in}}%
\pgfpathcurveto{\pgfqpoint{1.146789in}{0.894541in}}{\pgfqpoint{1.143516in}{0.902441in}}{\pgfqpoint{1.137692in}{0.908264in}}%
\pgfpathcurveto{\pgfqpoint{1.131868in}{0.914088in}}{\pgfqpoint{1.123968in}{0.917361in}}{\pgfqpoint{1.115732in}{0.917361in}}%
\pgfpathcurveto{\pgfqpoint{1.107496in}{0.917361in}}{\pgfqpoint{1.099596in}{0.914088in}}{\pgfqpoint{1.093772in}{0.908264in}}%
\pgfpathcurveto{\pgfqpoint{1.087948in}{0.902441in}}{\pgfqpoint{1.084676in}{0.894541in}}{\pgfqpoint{1.084676in}{0.886304in}}%
\pgfpathcurveto{\pgfqpoint{1.084676in}{0.878068in}}{\pgfqpoint{1.087948in}{0.870168in}}{\pgfqpoint{1.093772in}{0.864344in}}%
\pgfpathcurveto{\pgfqpoint{1.099596in}{0.858520in}}{\pgfqpoint{1.107496in}{0.855248in}}{\pgfqpoint{1.115732in}{0.855248in}}%
\pgfpathclose%
\pgfusepath{stroke,fill}%
\end{pgfscope}%
\begin{pgfscope}%
\pgfpathrectangle{\pgfqpoint{0.556847in}{0.516222in}}{\pgfqpoint{1.722590in}{1.783528in}} %
\pgfusepath{clip}%
\pgfsetbuttcap%
\pgfsetroundjoin%
\definecolor{currentfill}{rgb}{0.298039,0.447059,0.690196}%
\pgfsetfillcolor{currentfill}%
\pgfsetlinewidth{0.240900pt}%
\definecolor{currentstroke}{rgb}{1.000000,1.000000,1.000000}%
\pgfsetstrokecolor{currentstroke}%
\pgfsetdash{}{0pt}%
\pgfpathmoveto{\pgfqpoint{1.127216in}{1.033601in}}%
\pgfpathcurveto{\pgfqpoint{1.135452in}{1.033601in}}{\pgfqpoint{1.143352in}{1.036873in}}{\pgfqpoint{1.149176in}{1.042697in}}%
\pgfpathcurveto{\pgfqpoint{1.155000in}{1.048521in}}{\pgfqpoint{1.158273in}{1.056421in}}{\pgfqpoint{1.158273in}{1.064657in}}%
\pgfpathcurveto{\pgfqpoint{1.158273in}{1.072893in}}{\pgfqpoint{1.155000in}{1.080793in}}{\pgfqpoint{1.149176in}{1.086617in}}%
\pgfpathcurveto{\pgfqpoint{1.143352in}{1.092441in}}{\pgfqpoint{1.135452in}{1.095714in}}{\pgfqpoint{1.127216in}{1.095714in}}%
\pgfpathcurveto{\pgfqpoint{1.118980in}{1.095714in}}{\pgfqpoint{1.111080in}{1.092441in}}{\pgfqpoint{1.105256in}{1.086617in}}%
\pgfpathcurveto{\pgfqpoint{1.099432in}{1.080793in}}{\pgfqpoint{1.096160in}{1.072893in}}{\pgfqpoint{1.096160in}{1.064657in}}%
\pgfpathcurveto{\pgfqpoint{1.096160in}{1.056421in}}{\pgfqpoint{1.099432in}{1.048521in}}{\pgfqpoint{1.105256in}{1.042697in}}%
\pgfpathcurveto{\pgfqpoint{1.111080in}{1.036873in}}{\pgfqpoint{1.118980in}{1.033601in}}{\pgfqpoint{1.127216in}{1.033601in}}%
\pgfpathclose%
\pgfusepath{stroke,fill}%
\end{pgfscope}%
\begin{pgfscope}%
\pgfpathrectangle{\pgfqpoint{0.556847in}{0.516222in}}{\pgfqpoint{1.722590in}{1.783528in}} %
\pgfusepath{clip}%
\pgfsetbuttcap%
\pgfsetroundjoin%
\definecolor{currentfill}{rgb}{0.298039,0.447059,0.690196}%
\pgfsetfillcolor{currentfill}%
\pgfsetlinewidth{0.240900pt}%
\definecolor{currentstroke}{rgb}{1.000000,1.000000,1.000000}%
\pgfsetstrokecolor{currentstroke}%
\pgfsetdash{}{0pt}%
\pgfpathmoveto{\pgfqpoint{1.318615in}{1.363553in}}%
\pgfpathcurveto{\pgfqpoint{1.326851in}{1.363553in}}{\pgfqpoint{1.334751in}{1.366825in}}{\pgfqpoint{1.340575in}{1.372649in}}%
\pgfpathcurveto{\pgfqpoint{1.346399in}{1.378473in}}{\pgfqpoint{1.349671in}{1.386373in}}{\pgfqpoint{1.349671in}{1.394610in}}%
\pgfpathcurveto{\pgfqpoint{1.349671in}{1.402846in}}{\pgfqpoint{1.346399in}{1.410746in}}{\pgfqpoint{1.340575in}{1.416570in}}%
\pgfpathcurveto{\pgfqpoint{1.334751in}{1.422394in}}{\pgfqpoint{1.326851in}{1.425666in}}{\pgfqpoint{1.318615in}{1.425666in}}%
\pgfpathcurveto{\pgfqpoint{1.310379in}{1.425666in}}{\pgfqpoint{1.302479in}{1.422394in}}{\pgfqpoint{1.296655in}{1.416570in}}%
\pgfpathcurveto{\pgfqpoint{1.290831in}{1.410746in}}{\pgfqpoint{1.287558in}{1.402846in}}{\pgfqpoint{1.287558in}{1.394610in}}%
\pgfpathcurveto{\pgfqpoint{1.287558in}{1.386373in}}{\pgfqpoint{1.290831in}{1.378473in}}{\pgfqpoint{1.296655in}{1.372649in}}%
\pgfpathcurveto{\pgfqpoint{1.302479in}{1.366825in}}{\pgfqpoint{1.310379in}{1.363553in}}{\pgfqpoint{1.318615in}{1.363553in}}%
\pgfpathclose%
\pgfusepath{stroke,fill}%
\end{pgfscope}%
\begin{pgfscope}%
\pgfpathrectangle{\pgfqpoint{0.556847in}{0.516222in}}{\pgfqpoint{1.722590in}{1.783528in}} %
\pgfusepath{clip}%
\pgfsetbuttcap%
\pgfsetroundjoin%
\definecolor{currentfill}{rgb}{0.298039,0.447059,0.690196}%
\pgfsetfillcolor{currentfill}%
\pgfsetlinewidth{0.240900pt}%
\definecolor{currentstroke}{rgb}{1.000000,1.000000,1.000000}%
\pgfsetstrokecolor{currentstroke}%
\pgfsetdash{}{0pt}%
\pgfpathmoveto{\pgfqpoint{1.444938in}{1.399224in}}%
\pgfpathcurveto{\pgfqpoint{1.453174in}{1.399224in}}{\pgfqpoint{1.461075in}{1.402496in}}{\pgfqpoint{1.466898in}{1.408320in}}%
\pgfpathcurveto{\pgfqpoint{1.472722in}{1.414144in}}{\pgfqpoint{1.475995in}{1.422044in}}{\pgfqpoint{1.475995in}{1.430280in}}%
\pgfpathcurveto{\pgfqpoint{1.475995in}{1.438516in}}{\pgfqpoint{1.472722in}{1.446417in}}{\pgfqpoint{1.466898in}{1.452240in}}%
\pgfpathcurveto{\pgfqpoint{1.461075in}{1.458064in}}{\pgfqpoint{1.453174in}{1.461337in}}{\pgfqpoint{1.444938in}{1.461337in}}%
\pgfpathcurveto{\pgfqpoint{1.436702in}{1.461337in}}{\pgfqpoint{1.428802in}{1.458064in}}{\pgfqpoint{1.422978in}{1.452240in}}%
\pgfpathcurveto{\pgfqpoint{1.417154in}{1.446417in}}{\pgfqpoint{1.413882in}{1.438516in}}{\pgfqpoint{1.413882in}{1.430280in}}%
\pgfpathcurveto{\pgfqpoint{1.413882in}{1.422044in}}{\pgfqpoint{1.417154in}{1.414144in}}{\pgfqpoint{1.422978in}{1.408320in}}%
\pgfpathcurveto{\pgfqpoint{1.428802in}{1.402496in}}{\pgfqpoint{1.436702in}{1.399224in}}{\pgfqpoint{1.444938in}{1.399224in}}%
\pgfpathclose%
\pgfusepath{stroke,fill}%
\end{pgfscope}%
\begin{pgfscope}%
\pgfpathrectangle{\pgfqpoint{0.556847in}{0.516222in}}{\pgfqpoint{1.722590in}{1.783528in}} %
\pgfusepath{clip}%
\pgfsetbuttcap%
\pgfsetroundjoin%
\definecolor{currentfill}{rgb}{0.298039,0.447059,0.690196}%
\pgfsetfillcolor{currentfill}%
\pgfsetlinewidth{0.240900pt}%
\definecolor{currentstroke}{rgb}{1.000000,1.000000,1.000000}%
\pgfsetstrokecolor{currentstroke}%
\pgfsetdash{}{0pt}%
\pgfpathmoveto{\pgfqpoint{1.188464in}{1.127236in}}%
\pgfpathcurveto{\pgfqpoint{1.196700in}{1.127236in}}{\pgfqpoint{1.204600in}{1.130508in}}{\pgfqpoint{1.210424in}{1.136332in}}%
\pgfpathcurveto{\pgfqpoint{1.216248in}{1.142156in}}{\pgfqpoint{1.219520in}{1.150056in}}{\pgfqpoint{1.219520in}{1.158292in}}%
\pgfpathcurveto{\pgfqpoint{1.219520in}{1.166529in}}{\pgfqpoint{1.216248in}{1.174429in}}{\pgfqpoint{1.210424in}{1.180252in}}%
\pgfpathcurveto{\pgfqpoint{1.204600in}{1.186076in}}{\pgfqpoint{1.196700in}{1.189349in}}{\pgfqpoint{1.188464in}{1.189349in}}%
\pgfpathcurveto{\pgfqpoint{1.180227in}{1.189349in}}{\pgfqpoint{1.172327in}{1.186076in}}{\pgfqpoint{1.166503in}{1.180252in}}%
\pgfpathcurveto{\pgfqpoint{1.160679in}{1.174429in}}{\pgfqpoint{1.157407in}{1.166529in}}{\pgfqpoint{1.157407in}{1.158292in}}%
\pgfpathcurveto{\pgfqpoint{1.157407in}{1.150056in}}{\pgfqpoint{1.160679in}{1.142156in}}{\pgfqpoint{1.166503in}{1.136332in}}%
\pgfpathcurveto{\pgfqpoint{1.172327in}{1.130508in}}{\pgfqpoint{1.180227in}{1.127236in}}{\pgfqpoint{1.188464in}{1.127236in}}%
\pgfpathclose%
\pgfusepath{stroke,fill}%
\end{pgfscope}%
\begin{pgfscope}%
\pgfpathrectangle{\pgfqpoint{0.556847in}{0.516222in}}{\pgfqpoint{1.722590in}{1.783528in}} %
\pgfusepath{clip}%
\pgfsetbuttcap%
\pgfsetroundjoin%
\definecolor{currentfill}{rgb}{0.298039,0.447059,0.690196}%
\pgfsetfillcolor{currentfill}%
\pgfsetlinewidth{0.240900pt}%
\definecolor{currentstroke}{rgb}{1.000000,1.000000,1.000000}%
\pgfsetstrokecolor{currentstroke}%
\pgfsetdash{}{0pt}%
\pgfpathmoveto{\pgfqpoint{1.800940in}{1.693506in}}%
\pgfpathcurveto{\pgfqpoint{1.809176in}{1.693506in}}{\pgfqpoint{1.817077in}{1.696778in}}{\pgfqpoint{1.822900in}{1.702602in}}%
\pgfpathcurveto{\pgfqpoint{1.828724in}{1.708426in}}{\pgfqpoint{1.831997in}{1.716326in}}{\pgfqpoint{1.831997in}{1.724562in}}%
\pgfpathcurveto{\pgfqpoint{1.831997in}{1.732799in}}{\pgfqpoint{1.828724in}{1.740699in}}{\pgfqpoint{1.822900in}{1.746523in}}%
\pgfpathcurveto{\pgfqpoint{1.817077in}{1.752346in}}{\pgfqpoint{1.809176in}{1.755619in}}{\pgfqpoint{1.800940in}{1.755619in}}%
\pgfpathcurveto{\pgfqpoint{1.792704in}{1.755619in}}{\pgfqpoint{1.784804in}{1.752346in}}{\pgfqpoint{1.778980in}{1.746523in}}%
\pgfpathcurveto{\pgfqpoint{1.773156in}{1.740699in}}{\pgfqpoint{1.769884in}{1.732799in}}{\pgfqpoint{1.769884in}{1.724562in}}%
\pgfpathcurveto{\pgfqpoint{1.769884in}{1.716326in}}{\pgfqpoint{1.773156in}{1.708426in}}{\pgfqpoint{1.778980in}{1.702602in}}%
\pgfpathcurveto{\pgfqpoint{1.784804in}{1.696778in}}{\pgfqpoint{1.792704in}{1.693506in}}{\pgfqpoint{1.800940in}{1.693506in}}%
\pgfpathclose%
\pgfusepath{stroke,fill}%
\end{pgfscope}%
\begin{pgfscope}%
\pgfpathrectangle{\pgfqpoint{0.556847in}{0.516222in}}{\pgfqpoint{1.722590in}{1.783528in}} %
\pgfusepath{clip}%
\pgfsetbuttcap%
\pgfsetroundjoin%
\definecolor{currentfill}{rgb}{0.298039,0.447059,0.690196}%
\pgfsetfillcolor{currentfill}%
\pgfsetlinewidth{0.240900pt}%
\definecolor{currentstroke}{rgb}{1.000000,1.000000,1.000000}%
\pgfsetstrokecolor{currentstroke}%
\pgfsetdash{}{0pt}%
\pgfpathmoveto{\pgfqpoint{1.188464in}{1.185200in}}%
\pgfpathcurveto{\pgfqpoint{1.196700in}{1.185200in}}{\pgfqpoint{1.204600in}{1.188473in}}{\pgfqpoint{1.210424in}{1.194297in}}%
\pgfpathcurveto{\pgfqpoint{1.216248in}{1.200121in}}{\pgfqpoint{1.219520in}{1.208021in}}{\pgfqpoint{1.219520in}{1.216257in}}%
\pgfpathcurveto{\pgfqpoint{1.219520in}{1.224493in}}{\pgfqpoint{1.216248in}{1.232393in}}{\pgfqpoint{1.210424in}{1.238217in}}%
\pgfpathcurveto{\pgfqpoint{1.204600in}{1.244041in}}{\pgfqpoint{1.196700in}{1.247313in}}{\pgfqpoint{1.188464in}{1.247313in}}%
\pgfpathcurveto{\pgfqpoint{1.180227in}{1.247313in}}{\pgfqpoint{1.172327in}{1.244041in}}{\pgfqpoint{1.166503in}{1.238217in}}%
\pgfpathcurveto{\pgfqpoint{1.160679in}{1.232393in}}{\pgfqpoint{1.157407in}{1.224493in}}{\pgfqpoint{1.157407in}{1.216257in}}%
\pgfpathcurveto{\pgfqpoint{1.157407in}{1.208021in}}{\pgfqpoint{1.160679in}{1.200121in}}{\pgfqpoint{1.166503in}{1.194297in}}%
\pgfpathcurveto{\pgfqpoint{1.172327in}{1.188473in}}{\pgfqpoint{1.180227in}{1.185200in}}{\pgfqpoint{1.188464in}{1.185200in}}%
\pgfpathclose%
\pgfusepath{stroke,fill}%
\end{pgfscope}%
\begin{pgfscope}%
\pgfpathrectangle{\pgfqpoint{0.556847in}{0.516222in}}{\pgfqpoint{1.722590in}{1.783528in}} %
\pgfusepath{clip}%
\pgfsetbuttcap%
\pgfsetroundjoin%
\definecolor{currentfill}{rgb}{0.298039,0.447059,0.690196}%
\pgfsetfillcolor{currentfill}%
\pgfsetlinewidth{0.240900pt}%
\definecolor{currentstroke}{rgb}{1.000000,1.000000,1.000000}%
\pgfsetstrokecolor{currentstroke}%
\pgfsetdash{}{0pt}%
\pgfpathmoveto{\pgfqpoint{1.108076in}{1.060353in}}%
\pgfpathcurveto{\pgfqpoint{1.116312in}{1.060353in}}{\pgfqpoint{1.124212in}{1.063626in}}{\pgfqpoint{1.130036in}{1.069450in}}%
\pgfpathcurveto{\pgfqpoint{1.135860in}{1.075274in}}{\pgfqpoint{1.139133in}{1.083174in}}{\pgfqpoint{1.139133in}{1.091410in}}%
\pgfpathcurveto{\pgfqpoint{1.139133in}{1.099646in}}{\pgfqpoint{1.135860in}{1.107546in}}{\pgfqpoint{1.130036in}{1.113370in}}%
\pgfpathcurveto{\pgfqpoint{1.124212in}{1.119194in}}{\pgfqpoint{1.116312in}{1.122466in}}{\pgfqpoint{1.108076in}{1.122466in}}%
\pgfpathcurveto{\pgfqpoint{1.099840in}{1.122466in}}{\pgfqpoint{1.091940in}{1.119194in}}{\pgfqpoint{1.086116in}{1.113370in}}%
\pgfpathcurveto{\pgfqpoint{1.080292in}{1.107546in}}{\pgfqpoint{1.077020in}{1.099646in}}{\pgfqpoint{1.077020in}{1.091410in}}%
\pgfpathcurveto{\pgfqpoint{1.077020in}{1.083174in}}{\pgfqpoint{1.080292in}{1.075274in}}{\pgfqpoint{1.086116in}{1.069450in}}%
\pgfpathcurveto{\pgfqpoint{1.091940in}{1.063626in}}{\pgfqpoint{1.099840in}{1.060353in}}{\pgfqpoint{1.108076in}{1.060353in}}%
\pgfpathclose%
\pgfusepath{stroke,fill}%
\end{pgfscope}%
\begin{pgfscope}%
\pgfpathrectangle{\pgfqpoint{0.556847in}{0.516222in}}{\pgfqpoint{1.722590in}{1.783528in}} %
\pgfusepath{clip}%
\pgfsetbuttcap%
\pgfsetroundjoin%
\definecolor{currentfill}{rgb}{0.298039,0.447059,0.690196}%
\pgfsetfillcolor{currentfill}%
\pgfsetlinewidth{0.240900pt}%
\definecolor{currentstroke}{rgb}{1.000000,1.000000,1.000000}%
\pgfsetstrokecolor{currentstroke}%
\pgfsetdash{}{0pt}%
\pgfpathmoveto{\pgfqpoint{1.276507in}{1.082648in}}%
\pgfpathcurveto{\pgfqpoint{1.284743in}{1.082648in}}{\pgfqpoint{1.292643in}{1.085920in}}{\pgfqpoint{1.298467in}{1.091744in}}%
\pgfpathcurveto{\pgfqpoint{1.304291in}{1.097568in}}{\pgfqpoint{1.307564in}{1.105468in}}{\pgfqpoint{1.307564in}{1.113704in}}%
\pgfpathcurveto{\pgfqpoint{1.307564in}{1.121940in}}{\pgfqpoint{1.304291in}{1.129840in}}{\pgfqpoint{1.298467in}{1.135664in}}%
\pgfpathcurveto{\pgfqpoint{1.292643in}{1.141488in}}{\pgfqpoint{1.284743in}{1.144761in}}{\pgfqpoint{1.276507in}{1.144761in}}%
\pgfpathcurveto{\pgfqpoint{1.268271in}{1.144761in}}{\pgfqpoint{1.260371in}{1.141488in}}{\pgfqpoint{1.254547in}{1.135664in}}%
\pgfpathcurveto{\pgfqpoint{1.248723in}{1.129840in}}{\pgfqpoint{1.245451in}{1.121940in}}{\pgfqpoint{1.245451in}{1.113704in}}%
\pgfpathcurveto{\pgfqpoint{1.245451in}{1.105468in}}{\pgfqpoint{1.248723in}{1.097568in}}{\pgfqpoint{1.254547in}{1.091744in}}%
\pgfpathcurveto{\pgfqpoint{1.260371in}{1.085920in}}{\pgfqpoint{1.268271in}{1.082648in}}{\pgfqpoint{1.276507in}{1.082648in}}%
\pgfpathclose%
\pgfusepath{stroke,fill}%
\end{pgfscope}%
\begin{pgfscope}%
\pgfpathrectangle{\pgfqpoint{0.556847in}{0.516222in}}{\pgfqpoint{1.722590in}{1.783528in}} %
\pgfusepath{clip}%
\pgfsetbuttcap%
\pgfsetroundjoin%
\definecolor{currentfill}{rgb}{0.298039,0.447059,0.690196}%
\pgfsetfillcolor{currentfill}%
\pgfsetlinewidth{0.240900pt}%
\definecolor{currentstroke}{rgb}{1.000000,1.000000,1.000000}%
\pgfsetstrokecolor{currentstroke}%
\pgfsetdash{}{0pt}%
\pgfpathmoveto{\pgfqpoint{1.513842in}{1.590953in}}%
\pgfpathcurveto{\pgfqpoint{1.522078in}{1.590953in}}{\pgfqpoint{1.529978in}{1.594225in}}{\pgfqpoint{1.535802in}{1.600049in}}%
\pgfpathcurveto{\pgfqpoint{1.541626in}{1.605873in}}{\pgfqpoint{1.544898in}{1.613773in}}{\pgfqpoint{1.544898in}{1.622009in}}%
\pgfpathcurveto{\pgfqpoint{1.544898in}{1.630246in}}{\pgfqpoint{1.541626in}{1.638146in}}{\pgfqpoint{1.535802in}{1.643970in}}%
\pgfpathcurveto{\pgfqpoint{1.529978in}{1.649794in}}{\pgfqpoint{1.522078in}{1.653066in}}{\pgfqpoint{1.513842in}{1.653066in}}%
\pgfpathcurveto{\pgfqpoint{1.505606in}{1.653066in}}{\pgfqpoint{1.497705in}{1.649794in}}{\pgfqpoint{1.491882in}{1.643970in}}%
\pgfpathcurveto{\pgfqpoint{1.486058in}{1.638146in}}{\pgfqpoint{1.482785in}{1.630246in}}{\pgfqpoint{1.482785in}{1.622009in}}%
\pgfpathcurveto{\pgfqpoint{1.482785in}{1.613773in}}{\pgfqpoint{1.486058in}{1.605873in}}{\pgfqpoint{1.491882in}{1.600049in}}%
\pgfpathcurveto{\pgfqpoint{1.497705in}{1.594225in}}{\pgfqpoint{1.505606in}{1.590953in}}{\pgfqpoint{1.513842in}{1.590953in}}%
\pgfpathclose%
\pgfusepath{stroke,fill}%
\end{pgfscope}%
\begin{pgfscope}%
\pgfpathrectangle{\pgfqpoint{0.556847in}{0.516222in}}{\pgfqpoint{1.722590in}{1.783528in}} %
\pgfusepath{clip}%
\pgfsetbuttcap%
\pgfsetroundjoin%
\definecolor{currentfill}{rgb}{0.298039,0.447059,0.690196}%
\pgfsetfillcolor{currentfill}%
\pgfsetlinewidth{0.240900pt}%
\definecolor{currentstroke}{rgb}{1.000000,1.000000,1.000000}%
\pgfsetstrokecolor{currentstroke}%
\pgfsetdash{}{0pt}%
\pgfpathmoveto{\pgfqpoint{1.142528in}{0.989012in}}%
\pgfpathcurveto{\pgfqpoint{1.150764in}{0.989012in}}{\pgfqpoint{1.158664in}{0.992285in}}{\pgfqpoint{1.164488in}{0.998109in}}%
\pgfpathcurveto{\pgfqpoint{1.170312in}{1.003932in}}{\pgfqpoint{1.173584in}{1.011833in}}{\pgfqpoint{1.173584in}{1.020069in}}%
\pgfpathcurveto{\pgfqpoint{1.173584in}{1.028305in}}{\pgfqpoint{1.170312in}{1.036205in}}{\pgfqpoint{1.164488in}{1.042029in}}%
\pgfpathcurveto{\pgfqpoint{1.158664in}{1.047853in}}{\pgfqpoint{1.150764in}{1.051125in}}{\pgfqpoint{1.142528in}{1.051125in}}%
\pgfpathcurveto{\pgfqpoint{1.134292in}{1.051125in}}{\pgfqpoint{1.126392in}{1.047853in}}{\pgfqpoint{1.120568in}{1.042029in}}%
\pgfpathcurveto{\pgfqpoint{1.114744in}{1.036205in}}{\pgfqpoint{1.111471in}{1.028305in}}{\pgfqpoint{1.111471in}{1.020069in}}%
\pgfpathcurveto{\pgfqpoint{1.111471in}{1.011833in}}{\pgfqpoint{1.114744in}{1.003932in}}{\pgfqpoint{1.120568in}{0.998109in}}%
\pgfpathcurveto{\pgfqpoint{1.126392in}{0.992285in}}{\pgfqpoint{1.134292in}{0.989012in}}{\pgfqpoint{1.142528in}{0.989012in}}%
\pgfpathclose%
\pgfusepath{stroke,fill}%
\end{pgfscope}%
\begin{pgfscope}%
\pgfpathrectangle{\pgfqpoint{0.556847in}{0.516222in}}{\pgfqpoint{1.722590in}{1.783528in}} %
\pgfusepath{clip}%
\pgfsetbuttcap%
\pgfsetroundjoin%
\definecolor{currentfill}{rgb}{0.298039,0.447059,0.690196}%
\pgfsetfillcolor{currentfill}%
\pgfsetlinewidth{0.240900pt}%
\definecolor{currentstroke}{rgb}{1.000000,1.000000,1.000000}%
\pgfsetstrokecolor{currentstroke}%
\pgfsetdash{}{0pt}%
\pgfpathmoveto{\pgfqpoint{1.900468in}{1.831729in}}%
\pgfpathcurveto{\pgfqpoint{1.908704in}{1.831729in}}{\pgfqpoint{1.916604in}{1.835002in}}{\pgfqpoint{1.922428in}{1.840825in}}%
\pgfpathcurveto{\pgfqpoint{1.928252in}{1.846649in}}{\pgfqpoint{1.931524in}{1.854549in}}{\pgfqpoint{1.931524in}{1.862786in}}%
\pgfpathcurveto{\pgfqpoint{1.931524in}{1.871022in}}{\pgfqpoint{1.928252in}{1.878922in}}{\pgfqpoint{1.922428in}{1.884746in}}%
\pgfpathcurveto{\pgfqpoint{1.916604in}{1.890570in}}{\pgfqpoint{1.908704in}{1.893842in}}{\pgfqpoint{1.900468in}{1.893842in}}%
\pgfpathcurveto{\pgfqpoint{1.892231in}{1.893842in}}{\pgfqpoint{1.884331in}{1.890570in}}{\pgfqpoint{1.878507in}{1.884746in}}%
\pgfpathcurveto{\pgfqpoint{1.872683in}{1.878922in}}{\pgfqpoint{1.869411in}{1.871022in}}{\pgfqpoint{1.869411in}{1.862786in}}%
\pgfpathcurveto{\pgfqpoint{1.869411in}{1.854549in}}{\pgfqpoint{1.872683in}{1.846649in}}{\pgfqpoint{1.878507in}{1.840825in}}%
\pgfpathcurveto{\pgfqpoint{1.884331in}{1.835002in}}{\pgfqpoint{1.892231in}{1.831729in}}{\pgfqpoint{1.900468in}{1.831729in}}%
\pgfpathclose%
\pgfusepath{stroke,fill}%
\end{pgfscope}%
\begin{pgfscope}%
\pgfpathrectangle{\pgfqpoint{0.556847in}{0.516222in}}{\pgfqpoint{1.722590in}{1.783528in}} %
\pgfusepath{clip}%
\pgfsetbuttcap%
\pgfsetroundjoin%
\definecolor{currentfill}{rgb}{0.298039,0.447059,0.690196}%
\pgfsetfillcolor{currentfill}%
\pgfsetlinewidth{0.240900pt}%
\definecolor{currentstroke}{rgb}{1.000000,1.000000,1.000000}%
\pgfsetstrokecolor{currentstroke}%
\pgfsetdash{}{0pt}%
\pgfpathmoveto{\pgfqpoint{1.525326in}{1.501777in}}%
\pgfpathcurveto{\pgfqpoint{1.533562in}{1.501777in}}{\pgfqpoint{1.541462in}{1.505049in}}{\pgfqpoint{1.547286in}{1.510873in}}%
\pgfpathcurveto{\pgfqpoint{1.553110in}{1.516697in}}{\pgfqpoint{1.556382in}{1.524597in}}{\pgfqpoint{1.556382in}{1.532833in}}%
\pgfpathcurveto{\pgfqpoint{1.556382in}{1.541069in}}{\pgfqpoint{1.553110in}{1.548969in}}{\pgfqpoint{1.547286in}{1.554793in}}%
\pgfpathcurveto{\pgfqpoint{1.541462in}{1.560617in}}{\pgfqpoint{1.533562in}{1.563890in}}{\pgfqpoint{1.525326in}{1.563890in}}%
\pgfpathcurveto{\pgfqpoint{1.517089in}{1.563890in}}{\pgfqpoint{1.509189in}{1.560617in}}{\pgfqpoint{1.503365in}{1.554793in}}%
\pgfpathcurveto{\pgfqpoint{1.497542in}{1.548969in}}{\pgfqpoint{1.494269in}{1.541069in}}{\pgfqpoint{1.494269in}{1.532833in}}%
\pgfpathcurveto{\pgfqpoint{1.494269in}{1.524597in}}{\pgfqpoint{1.497542in}{1.516697in}}{\pgfqpoint{1.503365in}{1.510873in}}%
\pgfpathcurveto{\pgfqpoint{1.509189in}{1.505049in}}{\pgfqpoint{1.517089in}{1.501777in}}{\pgfqpoint{1.525326in}{1.501777in}}%
\pgfpathclose%
\pgfusepath{stroke,fill}%
\end{pgfscope}%
\begin{pgfscope}%
\pgfpathrectangle{\pgfqpoint{0.556847in}{0.516222in}}{\pgfqpoint{1.722590in}{1.783528in}} %
\pgfusepath{clip}%
\pgfsetbuttcap%
\pgfsetroundjoin%
\definecolor{currentfill}{rgb}{0.298039,0.447059,0.690196}%
\pgfsetfillcolor{currentfill}%
\pgfsetlinewidth{0.240900pt}%
\definecolor{currentstroke}{rgb}{1.000000,1.000000,1.000000}%
\pgfsetstrokecolor{currentstroke}%
\pgfsetdash{}{0pt}%
\pgfpathmoveto{\pgfqpoint{1.785628in}{1.706882in}}%
\pgfpathcurveto{\pgfqpoint{1.793865in}{1.706882in}}{\pgfqpoint{1.801765in}{1.710155in}}{\pgfqpoint{1.807589in}{1.715978in}}%
\pgfpathcurveto{\pgfqpoint{1.813412in}{1.721802in}}{\pgfqpoint{1.816685in}{1.729702in}}{\pgfqpoint{1.816685in}{1.737939in}}%
\pgfpathcurveto{\pgfqpoint{1.816685in}{1.746175in}}{\pgfqpoint{1.813412in}{1.754075in}}{\pgfqpoint{1.807589in}{1.759899in}}%
\pgfpathcurveto{\pgfqpoint{1.801765in}{1.765723in}}{\pgfqpoint{1.793865in}{1.768995in}}{\pgfqpoint{1.785628in}{1.768995in}}%
\pgfpathcurveto{\pgfqpoint{1.777392in}{1.768995in}}{\pgfqpoint{1.769492in}{1.765723in}}{\pgfqpoint{1.763668in}{1.759899in}}%
\pgfpathcurveto{\pgfqpoint{1.757844in}{1.754075in}}{\pgfqpoint{1.754572in}{1.746175in}}{\pgfqpoint{1.754572in}{1.737939in}}%
\pgfpathcurveto{\pgfqpoint{1.754572in}{1.729702in}}{\pgfqpoint{1.757844in}{1.721802in}}{\pgfqpoint{1.763668in}{1.715978in}}%
\pgfpathcurveto{\pgfqpoint{1.769492in}{1.710155in}}{\pgfqpoint{1.777392in}{1.706882in}}{\pgfqpoint{1.785628in}{1.706882in}}%
\pgfpathclose%
\pgfusepath{stroke,fill}%
\end{pgfscope}%
\begin{pgfscope}%
\pgfpathrectangle{\pgfqpoint{0.556847in}{0.516222in}}{\pgfqpoint{1.722590in}{1.783528in}} %
\pgfusepath{clip}%
\pgfsetbuttcap%
\pgfsetroundjoin%
\definecolor{currentfill}{rgb}{0.298039,0.447059,0.690196}%
\pgfsetfillcolor{currentfill}%
\pgfsetlinewidth{0.240900pt}%
\definecolor{currentstroke}{rgb}{1.000000,1.000000,1.000000}%
\pgfsetstrokecolor{currentstroke}%
\pgfsetdash{}{0pt}%
\pgfpathmoveto{\pgfqpoint{2.091867in}{1.983329in}}%
\pgfpathcurveto{\pgfqpoint{2.100103in}{1.983329in}}{\pgfqpoint{2.108003in}{1.986601in}}{\pgfqpoint{2.113827in}{1.992425in}}%
\pgfpathcurveto{\pgfqpoint{2.119651in}{1.998249in}}{\pgfqpoint{2.122923in}{2.006149in}}{\pgfqpoint{2.122923in}{2.014386in}}%
\pgfpathcurveto{\pgfqpoint{2.122923in}{2.022622in}}{\pgfqpoint{2.119651in}{2.030522in}}{\pgfqpoint{2.113827in}{2.036346in}}%
\pgfpathcurveto{\pgfqpoint{2.108003in}{2.042170in}}{\pgfqpoint{2.100103in}{2.045442in}}{\pgfqpoint{2.091867in}{2.045442in}}%
\pgfpathcurveto{\pgfqpoint{2.083630in}{2.045442in}}{\pgfqpoint{2.075730in}{2.042170in}}{\pgfqpoint{2.069906in}{2.036346in}}%
\pgfpathcurveto{\pgfqpoint{2.064082in}{2.030522in}}{\pgfqpoint{2.060810in}{2.022622in}}{\pgfqpoint{2.060810in}{2.014386in}}%
\pgfpathcurveto{\pgfqpoint{2.060810in}{2.006149in}}{\pgfqpoint{2.064082in}{1.998249in}}{\pgfqpoint{2.069906in}{1.992425in}}%
\pgfpathcurveto{\pgfqpoint{2.075730in}{1.986601in}}{\pgfqpoint{2.083630in}{1.983329in}}{\pgfqpoint{2.091867in}{1.983329in}}%
\pgfpathclose%
\pgfusepath{stroke,fill}%
\end{pgfscope}%
\begin{pgfscope}%
\pgfpathrectangle{\pgfqpoint{0.556847in}{0.516222in}}{\pgfqpoint{1.722590in}{1.783528in}} %
\pgfusepath{clip}%
\pgfsetbuttcap%
\pgfsetroundjoin%
\definecolor{currentfill}{rgb}{0.298039,0.447059,0.690196}%
\pgfsetfillcolor{currentfill}%
\pgfsetlinewidth{0.240900pt}%
\definecolor{currentstroke}{rgb}{1.000000,1.000000,1.000000}%
\pgfsetstrokecolor{currentstroke}%
\pgfsetdash{}{0pt}%
\pgfpathmoveto{\pgfqpoint{1.131044in}{1.011306in}}%
\pgfpathcurveto{\pgfqpoint{1.139280in}{1.011306in}}{\pgfqpoint{1.147180in}{1.014579in}}{\pgfqpoint{1.153004in}{1.020403in}}%
\pgfpathcurveto{\pgfqpoint{1.158828in}{1.026227in}}{\pgfqpoint{1.162100in}{1.034127in}}{\pgfqpoint{1.162100in}{1.042363in}}%
\pgfpathcurveto{\pgfqpoint{1.162100in}{1.050599in}}{\pgfqpoint{1.158828in}{1.058499in}}{\pgfqpoint{1.153004in}{1.064323in}}%
\pgfpathcurveto{\pgfqpoint{1.147180in}{1.070147in}}{\pgfqpoint{1.139280in}{1.073419in}}{\pgfqpoint{1.131044in}{1.073419in}}%
\pgfpathcurveto{\pgfqpoint{1.122808in}{1.073419in}}{\pgfqpoint{1.114908in}{1.070147in}}{\pgfqpoint{1.109084in}{1.064323in}}%
\pgfpathcurveto{\pgfqpoint{1.103260in}{1.058499in}}{\pgfqpoint{1.099987in}{1.050599in}}{\pgfqpoint{1.099987in}{1.042363in}}%
\pgfpathcurveto{\pgfqpoint{1.099987in}{1.034127in}}{\pgfqpoint{1.103260in}{1.026227in}}{\pgfqpoint{1.109084in}{1.020403in}}%
\pgfpathcurveto{\pgfqpoint{1.114908in}{1.014579in}}{\pgfqpoint{1.122808in}{1.011306in}}{\pgfqpoint{1.131044in}{1.011306in}}%
\pgfpathclose%
\pgfusepath{stroke,fill}%
\end{pgfscope}%
\begin{pgfscope}%
\pgfpathrectangle{\pgfqpoint{0.556847in}{0.516222in}}{\pgfqpoint{1.722590in}{1.783528in}} %
\pgfusepath{clip}%
\pgfsetbuttcap%
\pgfsetroundjoin%
\definecolor{currentfill}{rgb}{0.298039,0.447059,0.690196}%
\pgfsetfillcolor{currentfill}%
\pgfsetlinewidth{0.240900pt}%
\definecolor{currentstroke}{rgb}{1.000000,1.000000,1.000000}%
\pgfsetstrokecolor{currentstroke}%
\pgfsetdash{}{0pt}%
\pgfpathmoveto{\pgfqpoint{1.885156in}{1.827270in}}%
\pgfpathcurveto{\pgfqpoint{1.893392in}{1.827270in}}{\pgfqpoint{1.901292in}{1.830543in}}{\pgfqpoint{1.907116in}{1.836367in}}%
\pgfpathcurveto{\pgfqpoint{1.912940in}{1.842191in}}{\pgfqpoint{1.916212in}{1.850091in}}{\pgfqpoint{1.916212in}{1.858327in}}%
\pgfpathcurveto{\pgfqpoint{1.916212in}{1.866563in}}{\pgfqpoint{1.912940in}{1.874463in}}{\pgfqpoint{1.907116in}{1.880287in}}%
\pgfpathcurveto{\pgfqpoint{1.901292in}{1.886111in}}{\pgfqpoint{1.893392in}{1.889383in}}{\pgfqpoint{1.885156in}{1.889383in}}%
\pgfpathcurveto{\pgfqpoint{1.876919in}{1.889383in}}{\pgfqpoint{1.869019in}{1.886111in}}{\pgfqpoint{1.863195in}{1.880287in}}%
\pgfpathcurveto{\pgfqpoint{1.857372in}{1.874463in}}{\pgfqpoint{1.854099in}{1.866563in}}{\pgfqpoint{1.854099in}{1.858327in}}%
\pgfpathcurveto{\pgfqpoint{1.854099in}{1.850091in}}{\pgfqpoint{1.857372in}{1.842191in}}{\pgfqpoint{1.863195in}{1.836367in}}%
\pgfpathcurveto{\pgfqpoint{1.869019in}{1.830543in}}{\pgfqpoint{1.876919in}{1.827270in}}{\pgfqpoint{1.885156in}{1.827270in}}%
\pgfpathclose%
\pgfusepath{stroke,fill}%
\end{pgfscope}%
\begin{pgfscope}%
\pgfpathrectangle{\pgfqpoint{0.556847in}{0.516222in}}{\pgfqpoint{1.722590in}{1.783528in}} %
\pgfusepath{clip}%
\pgfsetbuttcap%
\pgfsetroundjoin%
\definecolor{currentfill}{rgb}{0.298039,0.447059,0.690196}%
\pgfsetfillcolor{currentfill}%
\pgfsetlinewidth{0.240900pt}%
\definecolor{currentstroke}{rgb}{1.000000,1.000000,1.000000}%
\pgfsetstrokecolor{currentstroke}%
\pgfsetdash{}{0pt}%
\pgfpathmoveto{\pgfqpoint{1.670789in}{1.671212in}}%
\pgfpathcurveto{\pgfqpoint{1.679025in}{1.671212in}}{\pgfqpoint{1.686925in}{1.674484in}}{\pgfqpoint{1.692749in}{1.680308in}}%
\pgfpathcurveto{\pgfqpoint{1.698573in}{1.686132in}}{\pgfqpoint{1.701845in}{1.694032in}}{\pgfqpoint{1.701845in}{1.702268in}}%
\pgfpathcurveto{\pgfqpoint{1.701845in}{1.710504in}}{\pgfqpoint{1.698573in}{1.718405in}}{\pgfqpoint{1.692749in}{1.724228in}}%
\pgfpathcurveto{\pgfqpoint{1.686925in}{1.730052in}}{\pgfqpoint{1.679025in}{1.733325in}}{\pgfqpoint{1.670789in}{1.733325in}}%
\pgfpathcurveto{\pgfqpoint{1.662553in}{1.733325in}}{\pgfqpoint{1.654653in}{1.730052in}}{\pgfqpoint{1.648829in}{1.724228in}}%
\pgfpathcurveto{\pgfqpoint{1.643005in}{1.718405in}}{\pgfqpoint{1.639732in}{1.710504in}}{\pgfqpoint{1.639732in}{1.702268in}}%
\pgfpathcurveto{\pgfqpoint{1.639732in}{1.694032in}}{\pgfqpoint{1.643005in}{1.686132in}}{\pgfqpoint{1.648829in}{1.680308in}}%
\pgfpathcurveto{\pgfqpoint{1.654653in}{1.674484in}}{\pgfqpoint{1.662553in}{1.671212in}}{\pgfqpoint{1.670789in}{1.671212in}}%
\pgfpathclose%
\pgfusepath{stroke,fill}%
\end{pgfscope}%
\begin{pgfscope}%
\pgfpathrectangle{\pgfqpoint{0.556847in}{0.516222in}}{\pgfqpoint{1.722590in}{1.783528in}} %
\pgfusepath{clip}%
\pgfsetbuttcap%
\pgfsetroundjoin%
\definecolor{currentfill}{rgb}{0.298039,0.447059,0.690196}%
\pgfsetfillcolor{currentfill}%
\pgfsetlinewidth{0.240900pt}%
\definecolor{currentstroke}{rgb}{1.000000,1.000000,1.000000}%
\pgfsetstrokecolor{currentstroke}%
\pgfsetdash{}{0pt}%
\pgfpathmoveto{\pgfqpoint{1.303303in}{1.225330in}}%
\pgfpathcurveto{\pgfqpoint{1.311539in}{1.225330in}}{\pgfqpoint{1.319439in}{1.228602in}}{\pgfqpoint{1.325263in}{1.234426in}}%
\pgfpathcurveto{\pgfqpoint{1.331087in}{1.240250in}}{\pgfqpoint{1.334360in}{1.248150in}}{\pgfqpoint{1.334360in}{1.256386in}}%
\pgfpathcurveto{\pgfqpoint{1.334360in}{1.264623in}}{\pgfqpoint{1.331087in}{1.272523in}}{\pgfqpoint{1.325263in}{1.278347in}}%
\pgfpathcurveto{\pgfqpoint{1.319439in}{1.284170in}}{\pgfqpoint{1.311539in}{1.287443in}}{\pgfqpoint{1.303303in}{1.287443in}}%
\pgfpathcurveto{\pgfqpoint{1.295067in}{1.287443in}}{\pgfqpoint{1.287167in}{1.284170in}}{\pgfqpoint{1.281343in}{1.278347in}}%
\pgfpathcurveto{\pgfqpoint{1.275519in}{1.272523in}}{\pgfqpoint{1.272247in}{1.264623in}}{\pgfqpoint{1.272247in}{1.256386in}}%
\pgfpathcurveto{\pgfqpoint{1.272247in}{1.248150in}}{\pgfqpoint{1.275519in}{1.240250in}}{\pgfqpoint{1.281343in}{1.234426in}}%
\pgfpathcurveto{\pgfqpoint{1.287167in}{1.228602in}}{\pgfqpoint{1.295067in}{1.225330in}}{\pgfqpoint{1.303303in}{1.225330in}}%
\pgfpathclose%
\pgfusepath{stroke,fill}%
\end{pgfscope}%
\begin{pgfscope}%
\pgfpathrectangle{\pgfqpoint{0.556847in}{0.516222in}}{\pgfqpoint{1.722590in}{1.783528in}} %
\pgfusepath{clip}%
\pgfsetbuttcap%
\pgfsetroundjoin%
\definecolor{currentfill}{rgb}{0.298039,0.447059,0.690196}%
\pgfsetfillcolor{currentfill}%
\pgfsetlinewidth{0.240900pt}%
\definecolor{currentstroke}{rgb}{1.000000,1.000000,1.000000}%
\pgfsetstrokecolor{currentstroke}%
\pgfsetdash{}{0pt}%
\pgfpathmoveto{\pgfqpoint{1.552122in}{1.510694in}}%
\pgfpathcurveto{\pgfqpoint{1.560358in}{1.510694in}}{\pgfqpoint{1.568258in}{1.513967in}}{\pgfqpoint{1.574082in}{1.519790in}}%
\pgfpathcurveto{\pgfqpoint{1.579906in}{1.525614in}}{\pgfqpoint{1.583178in}{1.533514in}}{\pgfqpoint{1.583178in}{1.541751in}}%
\pgfpathcurveto{\pgfqpoint{1.583178in}{1.549987in}}{\pgfqpoint{1.579906in}{1.557887in}}{\pgfqpoint{1.574082in}{1.563711in}}%
\pgfpathcurveto{\pgfqpoint{1.568258in}{1.569535in}}{\pgfqpoint{1.560358in}{1.572807in}}{\pgfqpoint{1.552122in}{1.572807in}}%
\pgfpathcurveto{\pgfqpoint{1.543885in}{1.572807in}}{\pgfqpoint{1.535985in}{1.569535in}}{\pgfqpoint{1.530161in}{1.563711in}}%
\pgfpathcurveto{\pgfqpoint{1.524337in}{1.557887in}}{\pgfqpoint{1.521065in}{1.549987in}}{\pgfqpoint{1.521065in}{1.541751in}}%
\pgfpathcurveto{\pgfqpoint{1.521065in}{1.533514in}}{\pgfqpoint{1.524337in}{1.525614in}}{\pgfqpoint{1.530161in}{1.519790in}}%
\pgfpathcurveto{\pgfqpoint{1.535985in}{1.513967in}}{\pgfqpoint{1.543885in}{1.510694in}}{\pgfqpoint{1.552122in}{1.510694in}}%
\pgfpathclose%
\pgfusepath{stroke,fill}%
\end{pgfscope}%
\begin{pgfscope}%
\pgfsetrectcap%
\pgfsetmiterjoin%
\pgfsetlinewidth{0.000000pt}%
\definecolor{currentstroke}{rgb}{1.000000,1.000000,1.000000}%
\pgfsetstrokecolor{currentstroke}%
\pgfsetdash{}{0pt}%
\pgfpathmoveto{\pgfqpoint{0.556847in}{0.516222in}}%
\pgfpathlineto{\pgfqpoint{0.556847in}{2.299750in}}%
\pgfusepath{}%
\end{pgfscope}%
\begin{pgfscope}%
\pgfsetrectcap%
\pgfsetmiterjoin%
\pgfsetlinewidth{0.000000pt}%
\definecolor{currentstroke}{rgb}{1.000000,1.000000,1.000000}%
\pgfsetstrokecolor{currentstroke}%
\pgfsetdash{}{0pt}%
\pgfpathmoveto{\pgfqpoint{0.556847in}{0.516222in}}%
\pgfpathlineto{\pgfqpoint{2.279437in}{0.516222in}}%
\pgfusepath{}%
\end{pgfscope}%
\end{pgfpicture}%
\makeatother%
\endgroup%

		\caption{Comparison between the two times registered for one throw.}
		\label{fig_EX1_EX1}
	\end{subfigure}
	\begin{subfigure}[h]{.5\linewidth}
		%% Creator: Matplotlib, PGF backend
%%
%% To include the figure in your LaTeX document, write
%%   \input{<filename>.pgf}
%%
%% Make sure the required packages are loaded in your preamble
%%   \usepackage{pgf}
%%
%% Figures using additional raster images can only be included by \input if
%% they are in the same directory as the main LaTeX file. For loading figures
%% from other directories you can use the `import` package
%%   \usepackage{import}
%% and then include the figures with
%%   \import{<path to file>}{<filename>.pgf}
%%
%% Matplotlib used the following preamble
%%   \usepackage[utf8x]{inputenc}
%%   \usepackage[T1]{fontenc}
%%   \usepackage{cmbright}
%%
\begingroup%
\makeatletter%
\begin{pgfpicture}%
\pgfpathrectangle{\pgfpointorigin}{\pgfqpoint{2.500000in}{2.500000in}}%
\pgfusepath{use as bounding box, clip}%
\begin{pgfscope}%
\pgfsetbuttcap%
\pgfsetmiterjoin%
\definecolor{currentfill}{rgb}{1.000000,1.000000,1.000000}%
\pgfsetfillcolor{currentfill}%
\pgfsetlinewidth{0.000000pt}%
\definecolor{currentstroke}{rgb}{1.000000,1.000000,1.000000}%
\pgfsetstrokecolor{currentstroke}%
\pgfsetdash{}{0pt}%
\pgfpathmoveto{\pgfqpoint{0.000000in}{0.000000in}}%
\pgfpathlineto{\pgfqpoint{2.500000in}{0.000000in}}%
\pgfpathlineto{\pgfqpoint{2.500000in}{2.500000in}}%
\pgfpathlineto{\pgfqpoint{0.000000in}{2.500000in}}%
\pgfpathclose%
\pgfusepath{fill}%
\end{pgfscope}%
\begin{pgfscope}%
\pgfsetbuttcap%
\pgfsetmiterjoin%
\definecolor{currentfill}{rgb}{0.917647,0.917647,0.949020}%
\pgfsetfillcolor{currentfill}%
\pgfsetlinewidth{0.000000pt}%
\definecolor{currentstroke}{rgb}{0.000000,0.000000,0.000000}%
\pgfsetstrokecolor{currentstroke}%
\pgfsetstrokeopacity{0.000000}%
\pgfsetdash{}{0pt}%
\pgfpathmoveto{\pgfqpoint{0.556847in}{0.516222in}}%
\pgfpathlineto{\pgfqpoint{2.279437in}{0.516222in}}%
\pgfpathlineto{\pgfqpoint{2.279437in}{2.299750in}}%
\pgfpathlineto{\pgfqpoint{0.556847in}{2.299750in}}%
\pgfpathclose%
\pgfusepath{fill}%
\end{pgfscope}%
\begin{pgfscope}%
\pgfpathrectangle{\pgfqpoint{0.556847in}{0.516222in}}{\pgfqpoint{1.722590in}{1.783528in}} %
\pgfusepath{clip}%
\pgfsetroundcap%
\pgfsetroundjoin%
\pgfsetlinewidth{0.803000pt}%
\definecolor{currentstroke}{rgb}{1.000000,1.000000,1.000000}%
\pgfsetstrokecolor{currentstroke}%
\pgfsetdash{}{0pt}%
\pgfpathmoveto{\pgfqpoint{0.556847in}{0.516222in}}%
\pgfpathlineto{\pgfqpoint{0.556847in}{2.299750in}}%
\pgfusepath{stroke}%
\end{pgfscope}%
\begin{pgfscope}%
\pgfsetbuttcap%
\pgfsetroundjoin%
\definecolor{currentfill}{rgb}{0.150000,0.150000,0.150000}%
\pgfsetfillcolor{currentfill}%
\pgfsetlinewidth{0.803000pt}%
\definecolor{currentstroke}{rgb}{0.150000,0.150000,0.150000}%
\pgfsetstrokecolor{currentstroke}%
\pgfsetdash{}{0pt}%
\pgfsys@defobject{currentmarker}{\pgfqpoint{0.000000in}{0.000000in}}{\pgfqpoint{0.000000in}{0.000000in}}{%
\pgfpathmoveto{\pgfqpoint{0.000000in}{0.000000in}}%
\pgfpathlineto{\pgfqpoint{0.000000in}{0.000000in}}%
\pgfusepath{stroke,fill}%
}%
\begin{pgfscope}%
\pgfsys@transformshift{0.556847in}{0.516222in}%
\pgfsys@useobject{currentmarker}{}%
\end{pgfscope}%
\end{pgfscope}%
\begin{pgfscope}%
\definecolor{textcolor}{rgb}{0.150000,0.150000,0.150000}%
\pgfsetstrokecolor{textcolor}%
\pgfsetfillcolor{textcolor}%
\pgftext[x=0.556847in,y=0.438444in,,top]{\color{textcolor}\sffamily\fontsize{8.000000}{9.600000}\selectfont 2.0}%
\end{pgfscope}%
\begin{pgfscope}%
\pgfpathrectangle{\pgfqpoint{0.556847in}{0.516222in}}{\pgfqpoint{1.722590in}{1.783528in}} %
\pgfusepath{clip}%
\pgfsetroundcap%
\pgfsetroundjoin%
\pgfsetlinewidth{0.803000pt}%
\definecolor{currentstroke}{rgb}{1.000000,1.000000,1.000000}%
\pgfsetstrokecolor{currentstroke}%
\pgfsetdash{}{0pt}%
\pgfpathmoveto{\pgfqpoint{0.748246in}{0.516222in}}%
\pgfpathlineto{\pgfqpoint{0.748246in}{2.299750in}}%
\pgfusepath{stroke}%
\end{pgfscope}%
\begin{pgfscope}%
\pgfsetbuttcap%
\pgfsetroundjoin%
\definecolor{currentfill}{rgb}{0.150000,0.150000,0.150000}%
\pgfsetfillcolor{currentfill}%
\pgfsetlinewidth{0.803000pt}%
\definecolor{currentstroke}{rgb}{0.150000,0.150000,0.150000}%
\pgfsetstrokecolor{currentstroke}%
\pgfsetdash{}{0pt}%
\pgfsys@defobject{currentmarker}{\pgfqpoint{0.000000in}{0.000000in}}{\pgfqpoint{0.000000in}{0.000000in}}{%
\pgfpathmoveto{\pgfqpoint{0.000000in}{0.000000in}}%
\pgfpathlineto{\pgfqpoint{0.000000in}{0.000000in}}%
\pgfusepath{stroke,fill}%
}%
\begin{pgfscope}%
\pgfsys@transformshift{0.748246in}{0.516222in}%
\pgfsys@useobject{currentmarker}{}%
\end{pgfscope}%
\end{pgfscope}%
\begin{pgfscope}%
\definecolor{textcolor}{rgb}{0.150000,0.150000,0.150000}%
\pgfsetstrokecolor{textcolor}%
\pgfsetfillcolor{textcolor}%
\pgftext[x=0.748246in,y=0.438444in,,top]{\color{textcolor}\sffamily\fontsize{8.000000}{9.600000}\selectfont 2.5}%
\end{pgfscope}%
\begin{pgfscope}%
\pgfpathrectangle{\pgfqpoint{0.556847in}{0.516222in}}{\pgfqpoint{1.722590in}{1.783528in}} %
\pgfusepath{clip}%
\pgfsetroundcap%
\pgfsetroundjoin%
\pgfsetlinewidth{0.803000pt}%
\definecolor{currentstroke}{rgb}{1.000000,1.000000,1.000000}%
\pgfsetstrokecolor{currentstroke}%
\pgfsetdash{}{0pt}%
\pgfpathmoveto{\pgfqpoint{0.939645in}{0.516222in}}%
\pgfpathlineto{\pgfqpoint{0.939645in}{2.299750in}}%
\pgfusepath{stroke}%
\end{pgfscope}%
\begin{pgfscope}%
\pgfsetbuttcap%
\pgfsetroundjoin%
\definecolor{currentfill}{rgb}{0.150000,0.150000,0.150000}%
\pgfsetfillcolor{currentfill}%
\pgfsetlinewidth{0.803000pt}%
\definecolor{currentstroke}{rgb}{0.150000,0.150000,0.150000}%
\pgfsetstrokecolor{currentstroke}%
\pgfsetdash{}{0pt}%
\pgfsys@defobject{currentmarker}{\pgfqpoint{0.000000in}{0.000000in}}{\pgfqpoint{0.000000in}{0.000000in}}{%
\pgfpathmoveto{\pgfqpoint{0.000000in}{0.000000in}}%
\pgfpathlineto{\pgfqpoint{0.000000in}{0.000000in}}%
\pgfusepath{stroke,fill}%
}%
\begin{pgfscope}%
\pgfsys@transformshift{0.939645in}{0.516222in}%
\pgfsys@useobject{currentmarker}{}%
\end{pgfscope}%
\end{pgfscope}%
\begin{pgfscope}%
\definecolor{textcolor}{rgb}{0.150000,0.150000,0.150000}%
\pgfsetstrokecolor{textcolor}%
\pgfsetfillcolor{textcolor}%
\pgftext[x=0.939645in,y=0.438444in,,top]{\color{textcolor}\sffamily\fontsize{8.000000}{9.600000}\selectfont 3.0}%
\end{pgfscope}%
\begin{pgfscope}%
\pgfpathrectangle{\pgfqpoint{0.556847in}{0.516222in}}{\pgfqpoint{1.722590in}{1.783528in}} %
\pgfusepath{clip}%
\pgfsetroundcap%
\pgfsetroundjoin%
\pgfsetlinewidth{0.803000pt}%
\definecolor{currentstroke}{rgb}{1.000000,1.000000,1.000000}%
\pgfsetstrokecolor{currentstroke}%
\pgfsetdash{}{0pt}%
\pgfpathmoveto{\pgfqpoint{1.131044in}{0.516222in}}%
\pgfpathlineto{\pgfqpoint{1.131044in}{2.299750in}}%
\pgfusepath{stroke}%
\end{pgfscope}%
\begin{pgfscope}%
\pgfsetbuttcap%
\pgfsetroundjoin%
\definecolor{currentfill}{rgb}{0.150000,0.150000,0.150000}%
\pgfsetfillcolor{currentfill}%
\pgfsetlinewidth{0.803000pt}%
\definecolor{currentstroke}{rgb}{0.150000,0.150000,0.150000}%
\pgfsetstrokecolor{currentstroke}%
\pgfsetdash{}{0pt}%
\pgfsys@defobject{currentmarker}{\pgfqpoint{0.000000in}{0.000000in}}{\pgfqpoint{0.000000in}{0.000000in}}{%
\pgfpathmoveto{\pgfqpoint{0.000000in}{0.000000in}}%
\pgfpathlineto{\pgfqpoint{0.000000in}{0.000000in}}%
\pgfusepath{stroke,fill}%
}%
\begin{pgfscope}%
\pgfsys@transformshift{1.131044in}{0.516222in}%
\pgfsys@useobject{currentmarker}{}%
\end{pgfscope}%
\end{pgfscope}%
\begin{pgfscope}%
\definecolor{textcolor}{rgb}{0.150000,0.150000,0.150000}%
\pgfsetstrokecolor{textcolor}%
\pgfsetfillcolor{textcolor}%
\pgftext[x=1.131044in,y=0.438444in,,top]{\color{textcolor}\sffamily\fontsize{8.000000}{9.600000}\selectfont 3.5}%
\end{pgfscope}%
\begin{pgfscope}%
\pgfpathrectangle{\pgfqpoint{0.556847in}{0.516222in}}{\pgfqpoint{1.722590in}{1.783528in}} %
\pgfusepath{clip}%
\pgfsetroundcap%
\pgfsetroundjoin%
\pgfsetlinewidth{0.803000pt}%
\definecolor{currentstroke}{rgb}{1.000000,1.000000,1.000000}%
\pgfsetstrokecolor{currentstroke}%
\pgfsetdash{}{0pt}%
\pgfpathmoveto{\pgfqpoint{1.322443in}{0.516222in}}%
\pgfpathlineto{\pgfqpoint{1.322443in}{2.299750in}}%
\pgfusepath{stroke}%
\end{pgfscope}%
\begin{pgfscope}%
\pgfsetbuttcap%
\pgfsetroundjoin%
\definecolor{currentfill}{rgb}{0.150000,0.150000,0.150000}%
\pgfsetfillcolor{currentfill}%
\pgfsetlinewidth{0.803000pt}%
\definecolor{currentstroke}{rgb}{0.150000,0.150000,0.150000}%
\pgfsetstrokecolor{currentstroke}%
\pgfsetdash{}{0pt}%
\pgfsys@defobject{currentmarker}{\pgfqpoint{0.000000in}{0.000000in}}{\pgfqpoint{0.000000in}{0.000000in}}{%
\pgfpathmoveto{\pgfqpoint{0.000000in}{0.000000in}}%
\pgfpathlineto{\pgfqpoint{0.000000in}{0.000000in}}%
\pgfusepath{stroke,fill}%
}%
\begin{pgfscope}%
\pgfsys@transformshift{1.322443in}{0.516222in}%
\pgfsys@useobject{currentmarker}{}%
\end{pgfscope}%
\end{pgfscope}%
\begin{pgfscope}%
\definecolor{textcolor}{rgb}{0.150000,0.150000,0.150000}%
\pgfsetstrokecolor{textcolor}%
\pgfsetfillcolor{textcolor}%
\pgftext[x=1.322443in,y=0.438444in,,top]{\color{textcolor}\sffamily\fontsize{8.000000}{9.600000}\selectfont 4.0}%
\end{pgfscope}%
\begin{pgfscope}%
\pgfpathrectangle{\pgfqpoint{0.556847in}{0.516222in}}{\pgfqpoint{1.722590in}{1.783528in}} %
\pgfusepath{clip}%
\pgfsetroundcap%
\pgfsetroundjoin%
\pgfsetlinewidth{0.803000pt}%
\definecolor{currentstroke}{rgb}{1.000000,1.000000,1.000000}%
\pgfsetstrokecolor{currentstroke}%
\pgfsetdash{}{0pt}%
\pgfpathmoveto{\pgfqpoint{1.513842in}{0.516222in}}%
\pgfpathlineto{\pgfqpoint{1.513842in}{2.299750in}}%
\pgfusepath{stroke}%
\end{pgfscope}%
\begin{pgfscope}%
\pgfsetbuttcap%
\pgfsetroundjoin%
\definecolor{currentfill}{rgb}{0.150000,0.150000,0.150000}%
\pgfsetfillcolor{currentfill}%
\pgfsetlinewidth{0.803000pt}%
\definecolor{currentstroke}{rgb}{0.150000,0.150000,0.150000}%
\pgfsetstrokecolor{currentstroke}%
\pgfsetdash{}{0pt}%
\pgfsys@defobject{currentmarker}{\pgfqpoint{0.000000in}{0.000000in}}{\pgfqpoint{0.000000in}{0.000000in}}{%
\pgfpathmoveto{\pgfqpoint{0.000000in}{0.000000in}}%
\pgfpathlineto{\pgfqpoint{0.000000in}{0.000000in}}%
\pgfusepath{stroke,fill}%
}%
\begin{pgfscope}%
\pgfsys@transformshift{1.513842in}{0.516222in}%
\pgfsys@useobject{currentmarker}{}%
\end{pgfscope}%
\end{pgfscope}%
\begin{pgfscope}%
\definecolor{textcolor}{rgb}{0.150000,0.150000,0.150000}%
\pgfsetstrokecolor{textcolor}%
\pgfsetfillcolor{textcolor}%
\pgftext[x=1.513842in,y=0.438444in,,top]{\color{textcolor}\sffamily\fontsize{8.000000}{9.600000}\selectfont 4.5}%
\end{pgfscope}%
\begin{pgfscope}%
\pgfpathrectangle{\pgfqpoint{0.556847in}{0.516222in}}{\pgfqpoint{1.722590in}{1.783528in}} %
\pgfusepath{clip}%
\pgfsetroundcap%
\pgfsetroundjoin%
\pgfsetlinewidth{0.803000pt}%
\definecolor{currentstroke}{rgb}{1.000000,1.000000,1.000000}%
\pgfsetstrokecolor{currentstroke}%
\pgfsetdash{}{0pt}%
\pgfpathmoveto{\pgfqpoint{1.705241in}{0.516222in}}%
\pgfpathlineto{\pgfqpoint{1.705241in}{2.299750in}}%
\pgfusepath{stroke}%
\end{pgfscope}%
\begin{pgfscope}%
\pgfsetbuttcap%
\pgfsetroundjoin%
\definecolor{currentfill}{rgb}{0.150000,0.150000,0.150000}%
\pgfsetfillcolor{currentfill}%
\pgfsetlinewidth{0.803000pt}%
\definecolor{currentstroke}{rgb}{0.150000,0.150000,0.150000}%
\pgfsetstrokecolor{currentstroke}%
\pgfsetdash{}{0pt}%
\pgfsys@defobject{currentmarker}{\pgfqpoint{0.000000in}{0.000000in}}{\pgfqpoint{0.000000in}{0.000000in}}{%
\pgfpathmoveto{\pgfqpoint{0.000000in}{0.000000in}}%
\pgfpathlineto{\pgfqpoint{0.000000in}{0.000000in}}%
\pgfusepath{stroke,fill}%
}%
\begin{pgfscope}%
\pgfsys@transformshift{1.705241in}{0.516222in}%
\pgfsys@useobject{currentmarker}{}%
\end{pgfscope}%
\end{pgfscope}%
\begin{pgfscope}%
\definecolor{textcolor}{rgb}{0.150000,0.150000,0.150000}%
\pgfsetstrokecolor{textcolor}%
\pgfsetfillcolor{textcolor}%
\pgftext[x=1.705241in,y=0.438444in,,top]{\color{textcolor}\sffamily\fontsize{8.000000}{9.600000}\selectfont 5.0}%
\end{pgfscope}%
\begin{pgfscope}%
\pgfpathrectangle{\pgfqpoint{0.556847in}{0.516222in}}{\pgfqpoint{1.722590in}{1.783528in}} %
\pgfusepath{clip}%
\pgfsetroundcap%
\pgfsetroundjoin%
\pgfsetlinewidth{0.803000pt}%
\definecolor{currentstroke}{rgb}{1.000000,1.000000,1.000000}%
\pgfsetstrokecolor{currentstroke}%
\pgfsetdash{}{0pt}%
\pgfpathmoveto{\pgfqpoint{1.896640in}{0.516222in}}%
\pgfpathlineto{\pgfqpoint{1.896640in}{2.299750in}}%
\pgfusepath{stroke}%
\end{pgfscope}%
\begin{pgfscope}%
\pgfsetbuttcap%
\pgfsetroundjoin%
\definecolor{currentfill}{rgb}{0.150000,0.150000,0.150000}%
\pgfsetfillcolor{currentfill}%
\pgfsetlinewidth{0.803000pt}%
\definecolor{currentstroke}{rgb}{0.150000,0.150000,0.150000}%
\pgfsetstrokecolor{currentstroke}%
\pgfsetdash{}{0pt}%
\pgfsys@defobject{currentmarker}{\pgfqpoint{0.000000in}{0.000000in}}{\pgfqpoint{0.000000in}{0.000000in}}{%
\pgfpathmoveto{\pgfqpoint{0.000000in}{0.000000in}}%
\pgfpathlineto{\pgfqpoint{0.000000in}{0.000000in}}%
\pgfusepath{stroke,fill}%
}%
\begin{pgfscope}%
\pgfsys@transformshift{1.896640in}{0.516222in}%
\pgfsys@useobject{currentmarker}{}%
\end{pgfscope}%
\end{pgfscope}%
\begin{pgfscope}%
\definecolor{textcolor}{rgb}{0.150000,0.150000,0.150000}%
\pgfsetstrokecolor{textcolor}%
\pgfsetfillcolor{textcolor}%
\pgftext[x=1.896640in,y=0.438444in,,top]{\color{textcolor}\sffamily\fontsize{8.000000}{9.600000}\selectfont 5.5}%
\end{pgfscope}%
\begin{pgfscope}%
\pgfpathrectangle{\pgfqpoint{0.556847in}{0.516222in}}{\pgfqpoint{1.722590in}{1.783528in}} %
\pgfusepath{clip}%
\pgfsetroundcap%
\pgfsetroundjoin%
\pgfsetlinewidth{0.803000pt}%
\definecolor{currentstroke}{rgb}{1.000000,1.000000,1.000000}%
\pgfsetstrokecolor{currentstroke}%
\pgfsetdash{}{0pt}%
\pgfpathmoveto{\pgfqpoint{2.088039in}{0.516222in}}%
\pgfpathlineto{\pgfqpoint{2.088039in}{2.299750in}}%
\pgfusepath{stroke}%
\end{pgfscope}%
\begin{pgfscope}%
\pgfsetbuttcap%
\pgfsetroundjoin%
\definecolor{currentfill}{rgb}{0.150000,0.150000,0.150000}%
\pgfsetfillcolor{currentfill}%
\pgfsetlinewidth{0.803000pt}%
\definecolor{currentstroke}{rgb}{0.150000,0.150000,0.150000}%
\pgfsetstrokecolor{currentstroke}%
\pgfsetdash{}{0pt}%
\pgfsys@defobject{currentmarker}{\pgfqpoint{0.000000in}{0.000000in}}{\pgfqpoint{0.000000in}{0.000000in}}{%
\pgfpathmoveto{\pgfqpoint{0.000000in}{0.000000in}}%
\pgfpathlineto{\pgfqpoint{0.000000in}{0.000000in}}%
\pgfusepath{stroke,fill}%
}%
\begin{pgfscope}%
\pgfsys@transformshift{2.088039in}{0.516222in}%
\pgfsys@useobject{currentmarker}{}%
\end{pgfscope}%
\end{pgfscope}%
\begin{pgfscope}%
\definecolor{textcolor}{rgb}{0.150000,0.150000,0.150000}%
\pgfsetstrokecolor{textcolor}%
\pgfsetfillcolor{textcolor}%
\pgftext[x=2.088039in,y=0.438444in,,top]{\color{textcolor}\sffamily\fontsize{8.000000}{9.600000}\selectfont 6.0}%
\end{pgfscope}%
\begin{pgfscope}%
\pgfpathrectangle{\pgfqpoint{0.556847in}{0.516222in}}{\pgfqpoint{1.722590in}{1.783528in}} %
\pgfusepath{clip}%
\pgfsetroundcap%
\pgfsetroundjoin%
\pgfsetlinewidth{0.803000pt}%
\definecolor{currentstroke}{rgb}{1.000000,1.000000,1.000000}%
\pgfsetstrokecolor{currentstroke}%
\pgfsetdash{}{0pt}%
\pgfpathmoveto{\pgfqpoint{2.279437in}{0.516222in}}%
\pgfpathlineto{\pgfqpoint{2.279437in}{2.299750in}}%
\pgfusepath{stroke}%
\end{pgfscope}%
\begin{pgfscope}%
\pgfsetbuttcap%
\pgfsetroundjoin%
\definecolor{currentfill}{rgb}{0.150000,0.150000,0.150000}%
\pgfsetfillcolor{currentfill}%
\pgfsetlinewidth{0.803000pt}%
\definecolor{currentstroke}{rgb}{0.150000,0.150000,0.150000}%
\pgfsetstrokecolor{currentstroke}%
\pgfsetdash{}{0pt}%
\pgfsys@defobject{currentmarker}{\pgfqpoint{0.000000in}{0.000000in}}{\pgfqpoint{0.000000in}{0.000000in}}{%
\pgfpathmoveto{\pgfqpoint{0.000000in}{0.000000in}}%
\pgfpathlineto{\pgfqpoint{0.000000in}{0.000000in}}%
\pgfusepath{stroke,fill}%
}%
\begin{pgfscope}%
\pgfsys@transformshift{2.279437in}{0.516222in}%
\pgfsys@useobject{currentmarker}{}%
\end{pgfscope}%
\end{pgfscope}%
\begin{pgfscope}%
\definecolor{textcolor}{rgb}{0.150000,0.150000,0.150000}%
\pgfsetstrokecolor{textcolor}%
\pgfsetfillcolor{textcolor}%
\pgftext[x=2.279437in,y=0.438444in,,top]{\color{textcolor}\sffamily\fontsize{8.000000}{9.600000}\selectfont 6.5}%
\end{pgfscope}%
\begin{pgfscope}%
\definecolor{textcolor}{rgb}{0.150000,0.150000,0.150000}%
\pgfsetstrokecolor{textcolor}%
\pgfsetfillcolor{textcolor}%
\pgftext[x=1.418142in,y=0.273321in,,top]{\color{textcolor}\sffamily\fontsize{8.800000}{10.560000}\selectfont Falling time realization 1 obs 1}%
\end{pgfscope}%
\begin{pgfscope}%
\pgfpathrectangle{\pgfqpoint{0.556847in}{0.516222in}}{\pgfqpoint{1.722590in}{1.783528in}} %
\pgfusepath{clip}%
\pgfsetroundcap%
\pgfsetroundjoin%
\pgfsetlinewidth{0.803000pt}%
\definecolor{currentstroke}{rgb}{1.000000,1.000000,1.000000}%
\pgfsetstrokecolor{currentstroke}%
\pgfsetdash{}{0pt}%
\pgfpathmoveto{\pgfqpoint{0.556847in}{0.516222in}}%
\pgfpathlineto{\pgfqpoint{2.279437in}{0.516222in}}%
\pgfusepath{stroke}%
\end{pgfscope}%
\begin{pgfscope}%
\pgfsetbuttcap%
\pgfsetroundjoin%
\definecolor{currentfill}{rgb}{0.150000,0.150000,0.150000}%
\pgfsetfillcolor{currentfill}%
\pgfsetlinewidth{0.803000pt}%
\definecolor{currentstroke}{rgb}{0.150000,0.150000,0.150000}%
\pgfsetstrokecolor{currentstroke}%
\pgfsetdash{}{0pt}%
\pgfsys@defobject{currentmarker}{\pgfqpoint{0.000000in}{0.000000in}}{\pgfqpoint{0.000000in}{0.000000in}}{%
\pgfpathmoveto{\pgfqpoint{0.000000in}{0.000000in}}%
\pgfpathlineto{\pgfqpoint{0.000000in}{0.000000in}}%
\pgfusepath{stroke,fill}%
}%
\begin{pgfscope}%
\pgfsys@transformshift{0.556847in}{0.516222in}%
\pgfsys@useobject{currentmarker}{}%
\end{pgfscope}%
\end{pgfscope}%
\begin{pgfscope}%
\definecolor{textcolor}{rgb}{0.150000,0.150000,0.150000}%
\pgfsetstrokecolor{textcolor}%
\pgfsetfillcolor{textcolor}%
\pgftext[x=0.479069in,y=0.516222in,right,]{\color{textcolor}\sffamily\fontsize{8.000000}{9.600000}\selectfont 2.5}%
\end{pgfscope}%
\begin{pgfscope}%
\pgfpathrectangle{\pgfqpoint{0.556847in}{0.516222in}}{\pgfqpoint{1.722590in}{1.783528in}} %
\pgfusepath{clip}%
\pgfsetroundcap%
\pgfsetroundjoin%
\pgfsetlinewidth{0.803000pt}%
\definecolor{currentstroke}{rgb}{1.000000,1.000000,1.000000}%
\pgfsetstrokecolor{currentstroke}%
\pgfsetdash{}{0pt}%
\pgfpathmoveto{\pgfqpoint{0.556847in}{0.739163in}}%
\pgfpathlineto{\pgfqpoint{2.279437in}{0.739163in}}%
\pgfusepath{stroke}%
\end{pgfscope}%
\begin{pgfscope}%
\pgfsetbuttcap%
\pgfsetroundjoin%
\definecolor{currentfill}{rgb}{0.150000,0.150000,0.150000}%
\pgfsetfillcolor{currentfill}%
\pgfsetlinewidth{0.803000pt}%
\definecolor{currentstroke}{rgb}{0.150000,0.150000,0.150000}%
\pgfsetstrokecolor{currentstroke}%
\pgfsetdash{}{0pt}%
\pgfsys@defobject{currentmarker}{\pgfqpoint{0.000000in}{0.000000in}}{\pgfqpoint{0.000000in}{0.000000in}}{%
\pgfpathmoveto{\pgfqpoint{0.000000in}{0.000000in}}%
\pgfpathlineto{\pgfqpoint{0.000000in}{0.000000in}}%
\pgfusepath{stroke,fill}%
}%
\begin{pgfscope}%
\pgfsys@transformshift{0.556847in}{0.739163in}%
\pgfsys@useobject{currentmarker}{}%
\end{pgfscope}%
\end{pgfscope}%
\begin{pgfscope}%
\definecolor{textcolor}{rgb}{0.150000,0.150000,0.150000}%
\pgfsetstrokecolor{textcolor}%
\pgfsetfillcolor{textcolor}%
\pgftext[x=0.479069in,y=0.739163in,right,]{\color{textcolor}\sffamily\fontsize{8.000000}{9.600000}\selectfont 3.0}%
\end{pgfscope}%
\begin{pgfscope}%
\pgfpathrectangle{\pgfqpoint{0.556847in}{0.516222in}}{\pgfqpoint{1.722590in}{1.783528in}} %
\pgfusepath{clip}%
\pgfsetroundcap%
\pgfsetroundjoin%
\pgfsetlinewidth{0.803000pt}%
\definecolor{currentstroke}{rgb}{1.000000,1.000000,1.000000}%
\pgfsetstrokecolor{currentstroke}%
\pgfsetdash{}{0pt}%
\pgfpathmoveto{\pgfqpoint{0.556847in}{0.962104in}}%
\pgfpathlineto{\pgfqpoint{2.279437in}{0.962104in}}%
\pgfusepath{stroke}%
\end{pgfscope}%
\begin{pgfscope}%
\pgfsetbuttcap%
\pgfsetroundjoin%
\definecolor{currentfill}{rgb}{0.150000,0.150000,0.150000}%
\pgfsetfillcolor{currentfill}%
\pgfsetlinewidth{0.803000pt}%
\definecolor{currentstroke}{rgb}{0.150000,0.150000,0.150000}%
\pgfsetstrokecolor{currentstroke}%
\pgfsetdash{}{0pt}%
\pgfsys@defobject{currentmarker}{\pgfqpoint{0.000000in}{0.000000in}}{\pgfqpoint{0.000000in}{0.000000in}}{%
\pgfpathmoveto{\pgfqpoint{0.000000in}{0.000000in}}%
\pgfpathlineto{\pgfqpoint{0.000000in}{0.000000in}}%
\pgfusepath{stroke,fill}%
}%
\begin{pgfscope}%
\pgfsys@transformshift{0.556847in}{0.962104in}%
\pgfsys@useobject{currentmarker}{}%
\end{pgfscope}%
\end{pgfscope}%
\begin{pgfscope}%
\definecolor{textcolor}{rgb}{0.150000,0.150000,0.150000}%
\pgfsetstrokecolor{textcolor}%
\pgfsetfillcolor{textcolor}%
\pgftext[x=0.479069in,y=0.962104in,right,]{\color{textcolor}\sffamily\fontsize{8.000000}{9.600000}\selectfont 3.5}%
\end{pgfscope}%
\begin{pgfscope}%
\pgfpathrectangle{\pgfqpoint{0.556847in}{0.516222in}}{\pgfqpoint{1.722590in}{1.783528in}} %
\pgfusepath{clip}%
\pgfsetroundcap%
\pgfsetroundjoin%
\pgfsetlinewidth{0.803000pt}%
\definecolor{currentstroke}{rgb}{1.000000,1.000000,1.000000}%
\pgfsetstrokecolor{currentstroke}%
\pgfsetdash{}{0pt}%
\pgfpathmoveto{\pgfqpoint{0.556847in}{1.185045in}}%
\pgfpathlineto{\pgfqpoint{2.279437in}{1.185045in}}%
\pgfusepath{stroke}%
\end{pgfscope}%
\begin{pgfscope}%
\pgfsetbuttcap%
\pgfsetroundjoin%
\definecolor{currentfill}{rgb}{0.150000,0.150000,0.150000}%
\pgfsetfillcolor{currentfill}%
\pgfsetlinewidth{0.803000pt}%
\definecolor{currentstroke}{rgb}{0.150000,0.150000,0.150000}%
\pgfsetstrokecolor{currentstroke}%
\pgfsetdash{}{0pt}%
\pgfsys@defobject{currentmarker}{\pgfqpoint{0.000000in}{0.000000in}}{\pgfqpoint{0.000000in}{0.000000in}}{%
\pgfpathmoveto{\pgfqpoint{0.000000in}{0.000000in}}%
\pgfpathlineto{\pgfqpoint{0.000000in}{0.000000in}}%
\pgfusepath{stroke,fill}%
}%
\begin{pgfscope}%
\pgfsys@transformshift{0.556847in}{1.185045in}%
\pgfsys@useobject{currentmarker}{}%
\end{pgfscope}%
\end{pgfscope}%
\begin{pgfscope}%
\definecolor{textcolor}{rgb}{0.150000,0.150000,0.150000}%
\pgfsetstrokecolor{textcolor}%
\pgfsetfillcolor{textcolor}%
\pgftext[x=0.479069in,y=1.185045in,right,]{\color{textcolor}\sffamily\fontsize{8.000000}{9.600000}\selectfont 4.0}%
\end{pgfscope}%
\begin{pgfscope}%
\pgfpathrectangle{\pgfqpoint{0.556847in}{0.516222in}}{\pgfqpoint{1.722590in}{1.783528in}} %
\pgfusepath{clip}%
\pgfsetroundcap%
\pgfsetroundjoin%
\pgfsetlinewidth{0.803000pt}%
\definecolor{currentstroke}{rgb}{1.000000,1.000000,1.000000}%
\pgfsetstrokecolor{currentstroke}%
\pgfsetdash{}{0pt}%
\pgfpathmoveto{\pgfqpoint{0.556847in}{1.407986in}}%
\pgfpathlineto{\pgfqpoint{2.279437in}{1.407986in}}%
\pgfusepath{stroke}%
\end{pgfscope}%
\begin{pgfscope}%
\pgfsetbuttcap%
\pgfsetroundjoin%
\definecolor{currentfill}{rgb}{0.150000,0.150000,0.150000}%
\pgfsetfillcolor{currentfill}%
\pgfsetlinewidth{0.803000pt}%
\definecolor{currentstroke}{rgb}{0.150000,0.150000,0.150000}%
\pgfsetstrokecolor{currentstroke}%
\pgfsetdash{}{0pt}%
\pgfsys@defobject{currentmarker}{\pgfqpoint{0.000000in}{0.000000in}}{\pgfqpoint{0.000000in}{0.000000in}}{%
\pgfpathmoveto{\pgfqpoint{0.000000in}{0.000000in}}%
\pgfpathlineto{\pgfqpoint{0.000000in}{0.000000in}}%
\pgfusepath{stroke,fill}%
}%
\begin{pgfscope}%
\pgfsys@transformshift{0.556847in}{1.407986in}%
\pgfsys@useobject{currentmarker}{}%
\end{pgfscope}%
\end{pgfscope}%
\begin{pgfscope}%
\definecolor{textcolor}{rgb}{0.150000,0.150000,0.150000}%
\pgfsetstrokecolor{textcolor}%
\pgfsetfillcolor{textcolor}%
\pgftext[x=0.479069in,y=1.407986in,right,]{\color{textcolor}\sffamily\fontsize{8.000000}{9.600000}\selectfont 4.5}%
\end{pgfscope}%
\begin{pgfscope}%
\pgfpathrectangle{\pgfqpoint{0.556847in}{0.516222in}}{\pgfqpoint{1.722590in}{1.783528in}} %
\pgfusepath{clip}%
\pgfsetroundcap%
\pgfsetroundjoin%
\pgfsetlinewidth{0.803000pt}%
\definecolor{currentstroke}{rgb}{1.000000,1.000000,1.000000}%
\pgfsetstrokecolor{currentstroke}%
\pgfsetdash{}{0pt}%
\pgfpathmoveto{\pgfqpoint{0.556847in}{1.630927in}}%
\pgfpathlineto{\pgfqpoint{2.279437in}{1.630927in}}%
\pgfusepath{stroke}%
\end{pgfscope}%
\begin{pgfscope}%
\pgfsetbuttcap%
\pgfsetroundjoin%
\definecolor{currentfill}{rgb}{0.150000,0.150000,0.150000}%
\pgfsetfillcolor{currentfill}%
\pgfsetlinewidth{0.803000pt}%
\definecolor{currentstroke}{rgb}{0.150000,0.150000,0.150000}%
\pgfsetstrokecolor{currentstroke}%
\pgfsetdash{}{0pt}%
\pgfsys@defobject{currentmarker}{\pgfqpoint{0.000000in}{0.000000in}}{\pgfqpoint{0.000000in}{0.000000in}}{%
\pgfpathmoveto{\pgfqpoint{0.000000in}{0.000000in}}%
\pgfpathlineto{\pgfqpoint{0.000000in}{0.000000in}}%
\pgfusepath{stroke,fill}%
}%
\begin{pgfscope}%
\pgfsys@transformshift{0.556847in}{1.630927in}%
\pgfsys@useobject{currentmarker}{}%
\end{pgfscope}%
\end{pgfscope}%
\begin{pgfscope}%
\definecolor{textcolor}{rgb}{0.150000,0.150000,0.150000}%
\pgfsetstrokecolor{textcolor}%
\pgfsetfillcolor{textcolor}%
\pgftext[x=0.479069in,y=1.630927in,right,]{\color{textcolor}\sffamily\fontsize{8.000000}{9.600000}\selectfont 5.0}%
\end{pgfscope}%
\begin{pgfscope}%
\pgfpathrectangle{\pgfqpoint{0.556847in}{0.516222in}}{\pgfqpoint{1.722590in}{1.783528in}} %
\pgfusepath{clip}%
\pgfsetroundcap%
\pgfsetroundjoin%
\pgfsetlinewidth{0.803000pt}%
\definecolor{currentstroke}{rgb}{1.000000,1.000000,1.000000}%
\pgfsetstrokecolor{currentstroke}%
\pgfsetdash{}{0pt}%
\pgfpathmoveto{\pgfqpoint{0.556847in}{1.853868in}}%
\pgfpathlineto{\pgfqpoint{2.279437in}{1.853868in}}%
\pgfusepath{stroke}%
\end{pgfscope}%
\begin{pgfscope}%
\pgfsetbuttcap%
\pgfsetroundjoin%
\definecolor{currentfill}{rgb}{0.150000,0.150000,0.150000}%
\pgfsetfillcolor{currentfill}%
\pgfsetlinewidth{0.803000pt}%
\definecolor{currentstroke}{rgb}{0.150000,0.150000,0.150000}%
\pgfsetstrokecolor{currentstroke}%
\pgfsetdash{}{0pt}%
\pgfsys@defobject{currentmarker}{\pgfqpoint{0.000000in}{0.000000in}}{\pgfqpoint{0.000000in}{0.000000in}}{%
\pgfpathmoveto{\pgfqpoint{0.000000in}{0.000000in}}%
\pgfpathlineto{\pgfqpoint{0.000000in}{0.000000in}}%
\pgfusepath{stroke,fill}%
}%
\begin{pgfscope}%
\pgfsys@transformshift{0.556847in}{1.853868in}%
\pgfsys@useobject{currentmarker}{}%
\end{pgfscope}%
\end{pgfscope}%
\begin{pgfscope}%
\definecolor{textcolor}{rgb}{0.150000,0.150000,0.150000}%
\pgfsetstrokecolor{textcolor}%
\pgfsetfillcolor{textcolor}%
\pgftext[x=0.479069in,y=1.853868in,right,]{\color{textcolor}\sffamily\fontsize{8.000000}{9.600000}\selectfont 5.5}%
\end{pgfscope}%
\begin{pgfscope}%
\pgfpathrectangle{\pgfqpoint{0.556847in}{0.516222in}}{\pgfqpoint{1.722590in}{1.783528in}} %
\pgfusepath{clip}%
\pgfsetroundcap%
\pgfsetroundjoin%
\pgfsetlinewidth{0.803000pt}%
\definecolor{currentstroke}{rgb}{1.000000,1.000000,1.000000}%
\pgfsetstrokecolor{currentstroke}%
\pgfsetdash{}{0pt}%
\pgfpathmoveto{\pgfqpoint{0.556847in}{2.076809in}}%
\pgfpathlineto{\pgfqpoint{2.279437in}{2.076809in}}%
\pgfusepath{stroke}%
\end{pgfscope}%
\begin{pgfscope}%
\pgfsetbuttcap%
\pgfsetroundjoin%
\definecolor{currentfill}{rgb}{0.150000,0.150000,0.150000}%
\pgfsetfillcolor{currentfill}%
\pgfsetlinewidth{0.803000pt}%
\definecolor{currentstroke}{rgb}{0.150000,0.150000,0.150000}%
\pgfsetstrokecolor{currentstroke}%
\pgfsetdash{}{0pt}%
\pgfsys@defobject{currentmarker}{\pgfqpoint{0.000000in}{0.000000in}}{\pgfqpoint{0.000000in}{0.000000in}}{%
\pgfpathmoveto{\pgfqpoint{0.000000in}{0.000000in}}%
\pgfpathlineto{\pgfqpoint{0.000000in}{0.000000in}}%
\pgfusepath{stroke,fill}%
}%
\begin{pgfscope}%
\pgfsys@transformshift{0.556847in}{2.076809in}%
\pgfsys@useobject{currentmarker}{}%
\end{pgfscope}%
\end{pgfscope}%
\begin{pgfscope}%
\definecolor{textcolor}{rgb}{0.150000,0.150000,0.150000}%
\pgfsetstrokecolor{textcolor}%
\pgfsetfillcolor{textcolor}%
\pgftext[x=0.479069in,y=2.076809in,right,]{\color{textcolor}\sffamily\fontsize{8.000000}{9.600000}\selectfont 6.0}%
\end{pgfscope}%
\begin{pgfscope}%
\pgfpathrectangle{\pgfqpoint{0.556847in}{0.516222in}}{\pgfqpoint{1.722590in}{1.783528in}} %
\pgfusepath{clip}%
\pgfsetroundcap%
\pgfsetroundjoin%
\pgfsetlinewidth{0.803000pt}%
\definecolor{currentstroke}{rgb}{1.000000,1.000000,1.000000}%
\pgfsetstrokecolor{currentstroke}%
\pgfsetdash{}{0pt}%
\pgfpathmoveto{\pgfqpoint{0.556847in}{2.299750in}}%
\pgfpathlineto{\pgfqpoint{2.279437in}{2.299750in}}%
\pgfusepath{stroke}%
\end{pgfscope}%
\begin{pgfscope}%
\pgfsetbuttcap%
\pgfsetroundjoin%
\definecolor{currentfill}{rgb}{0.150000,0.150000,0.150000}%
\pgfsetfillcolor{currentfill}%
\pgfsetlinewidth{0.803000pt}%
\definecolor{currentstroke}{rgb}{0.150000,0.150000,0.150000}%
\pgfsetstrokecolor{currentstroke}%
\pgfsetdash{}{0pt}%
\pgfsys@defobject{currentmarker}{\pgfqpoint{0.000000in}{0.000000in}}{\pgfqpoint{0.000000in}{0.000000in}}{%
\pgfpathmoveto{\pgfqpoint{0.000000in}{0.000000in}}%
\pgfpathlineto{\pgfqpoint{0.000000in}{0.000000in}}%
\pgfusepath{stroke,fill}%
}%
\begin{pgfscope}%
\pgfsys@transformshift{0.556847in}{2.299750in}%
\pgfsys@useobject{currentmarker}{}%
\end{pgfscope}%
\end{pgfscope}%
\begin{pgfscope}%
\definecolor{textcolor}{rgb}{0.150000,0.150000,0.150000}%
\pgfsetstrokecolor{textcolor}%
\pgfsetfillcolor{textcolor}%
\pgftext[x=0.479069in,y=2.299750in,right,]{\color{textcolor}\sffamily\fontsize{8.000000}{9.600000}\selectfont 6.5}%
\end{pgfscope}%
\begin{pgfscope}%
\definecolor{textcolor}{rgb}{0.150000,0.150000,0.150000}%
\pgfsetstrokecolor{textcolor}%
\pgfsetfillcolor{textcolor}%
\pgftext[x=0.251677in,y=1.407986in,,bottom,rotate=90.000000]{\color{textcolor}\sffamily\fontsize{8.800000}{10.560000}\selectfont Falling time realization 2 obs 1}%
\end{pgfscope}%
\begin{pgfscope}%
\pgfpathrectangle{\pgfqpoint{0.556847in}{0.516222in}}{\pgfqpoint{1.722590in}{1.783528in}} %
\pgfusepath{clip}%
\pgfsetbuttcap%
\pgfsetroundjoin%
\definecolor{currentfill}{rgb}{0.298039,0.447059,0.690196}%
\pgfsetfillcolor{currentfill}%
\pgfsetlinewidth{0.240900pt}%
\definecolor{currentstroke}{rgb}{1.000000,1.000000,1.000000}%
\pgfsetstrokecolor{currentstroke}%
\pgfsetdash{}{0pt}%
\pgfpathmoveto{\pgfqpoint{1.575089in}{1.180742in}}%
\pgfpathcurveto{\pgfqpoint{1.583326in}{1.180742in}}{\pgfqpoint{1.591226in}{1.184014in}}{\pgfqpoint{1.597050in}{1.189838in}}%
\pgfpathcurveto{\pgfqpoint{1.602874in}{1.195662in}}{\pgfqpoint{1.606146in}{1.203562in}}{\pgfqpoint{1.606146in}{1.211798in}}%
\pgfpathcurveto{\pgfqpoint{1.606146in}{1.220034in}}{\pgfqpoint{1.602874in}{1.227934in}}{\pgfqpoint{1.597050in}{1.233758in}}%
\pgfpathcurveto{\pgfqpoint{1.591226in}{1.239582in}}{\pgfqpoint{1.583326in}{1.242855in}}{\pgfqpoint{1.575089in}{1.242855in}}%
\pgfpathcurveto{\pgfqpoint{1.566853in}{1.242855in}}{\pgfqpoint{1.558953in}{1.239582in}}{\pgfqpoint{1.553129in}{1.233758in}}%
\pgfpathcurveto{\pgfqpoint{1.547305in}{1.227934in}}{\pgfqpoint{1.544033in}{1.220034in}}{\pgfqpoint{1.544033in}{1.211798in}}%
\pgfpathcurveto{\pgfqpoint{1.544033in}{1.203562in}}{\pgfqpoint{1.547305in}{1.195662in}}{\pgfqpoint{1.553129in}{1.189838in}}%
\pgfpathcurveto{\pgfqpoint{1.558953in}{1.184014in}}{\pgfqpoint{1.566853in}{1.180742in}}{\pgfqpoint{1.575089in}{1.180742in}}%
\pgfpathclose%
\pgfusepath{stroke,fill}%
\end{pgfscope}%
\begin{pgfscope}%
\pgfpathrectangle{\pgfqpoint{0.556847in}{0.516222in}}{\pgfqpoint{1.722590in}{1.783528in}} %
\pgfusepath{clip}%
\pgfsetbuttcap%
\pgfsetroundjoin%
\definecolor{currentfill}{rgb}{0.298039,0.447059,0.690196}%
\pgfsetfillcolor{currentfill}%
\pgfsetlinewidth{0.240900pt}%
\definecolor{currentstroke}{rgb}{1.000000,1.000000,1.000000}%
\pgfsetstrokecolor{currentstroke}%
\pgfsetdash{}{0pt}%
\pgfpathmoveto{\pgfqpoint{1.287991in}{1.064812in}}%
\pgfpathcurveto{\pgfqpoint{1.296227in}{1.064812in}}{\pgfqpoint{1.304127in}{1.068085in}}{\pgfqpoint{1.309951in}{1.073908in}}%
\pgfpathcurveto{\pgfqpoint{1.315775in}{1.079732in}}{\pgfqpoint{1.319048in}{1.087632in}}{\pgfqpoint{1.319048in}{1.095869in}}%
\pgfpathcurveto{\pgfqpoint{1.319048in}{1.104105in}}{\pgfqpoint{1.315775in}{1.112005in}}{\pgfqpoint{1.309951in}{1.117829in}}%
\pgfpathcurveto{\pgfqpoint{1.304127in}{1.123653in}}{\pgfqpoint{1.296227in}{1.126925in}}{\pgfqpoint{1.287991in}{1.126925in}}%
\pgfpathcurveto{\pgfqpoint{1.279755in}{1.126925in}}{\pgfqpoint{1.271855in}{1.123653in}}{\pgfqpoint{1.266031in}{1.117829in}}%
\pgfpathcurveto{\pgfqpoint{1.260207in}{1.112005in}}{\pgfqpoint{1.256935in}{1.104105in}}{\pgfqpoint{1.256935in}{1.095869in}}%
\pgfpathcurveto{\pgfqpoint{1.256935in}{1.087632in}}{\pgfqpoint{1.260207in}{1.079732in}}{\pgfqpoint{1.266031in}{1.073908in}}%
\pgfpathcurveto{\pgfqpoint{1.271855in}{1.068085in}}{\pgfqpoint{1.279755in}{1.064812in}}{\pgfqpoint{1.287991in}{1.064812in}}%
\pgfpathclose%
\pgfusepath{stroke,fill}%
\end{pgfscope}%
\begin{pgfscope}%
\pgfpathrectangle{\pgfqpoint{0.556847in}{0.516222in}}{\pgfqpoint{1.722590in}{1.783528in}} %
\pgfusepath{clip}%
\pgfsetbuttcap%
\pgfsetroundjoin%
\definecolor{currentfill}{rgb}{0.298039,0.447059,0.690196}%
\pgfsetfillcolor{currentfill}%
\pgfsetlinewidth{0.240900pt}%
\definecolor{currentstroke}{rgb}{1.000000,1.000000,1.000000}%
\pgfsetstrokecolor{currentstroke}%
\pgfsetdash{}{0pt}%
\pgfpathmoveto{\pgfqpoint{1.287991in}{1.015765in}}%
\pgfpathcurveto{\pgfqpoint{1.296227in}{1.015765in}}{\pgfqpoint{1.304127in}{1.019038in}}{\pgfqpoint{1.309951in}{1.024861in}}%
\pgfpathcurveto{\pgfqpoint{1.315775in}{1.030685in}}{\pgfqpoint{1.319048in}{1.038585in}}{\pgfqpoint{1.319048in}{1.046822in}}%
\pgfpathcurveto{\pgfqpoint{1.319048in}{1.055058in}}{\pgfqpoint{1.315775in}{1.062958in}}{\pgfqpoint{1.309951in}{1.068782in}}%
\pgfpathcurveto{\pgfqpoint{1.304127in}{1.074606in}}{\pgfqpoint{1.296227in}{1.077878in}}{\pgfqpoint{1.287991in}{1.077878in}}%
\pgfpathcurveto{\pgfqpoint{1.279755in}{1.077878in}}{\pgfqpoint{1.271855in}{1.074606in}}{\pgfqpoint{1.266031in}{1.068782in}}%
\pgfpathcurveto{\pgfqpoint{1.260207in}{1.062958in}}{\pgfqpoint{1.256935in}{1.055058in}}{\pgfqpoint{1.256935in}{1.046822in}}%
\pgfpathcurveto{\pgfqpoint{1.256935in}{1.038585in}}{\pgfqpoint{1.260207in}{1.030685in}}{\pgfqpoint{1.266031in}{1.024861in}}%
\pgfpathcurveto{\pgfqpoint{1.271855in}{1.019038in}}{\pgfqpoint{1.279755in}{1.015765in}}{\pgfqpoint{1.287991in}{1.015765in}}%
\pgfpathclose%
\pgfusepath{stroke,fill}%
\end{pgfscope}%
\begin{pgfscope}%
\pgfpathrectangle{\pgfqpoint{0.556847in}{0.516222in}}{\pgfqpoint{1.722590in}{1.783528in}} %
\pgfusepath{clip}%
\pgfsetbuttcap%
\pgfsetroundjoin%
\definecolor{currentfill}{rgb}{0.298039,0.447059,0.690196}%
\pgfsetfillcolor{currentfill}%
\pgfsetlinewidth{0.240900pt}%
\definecolor{currentstroke}{rgb}{1.000000,1.000000,1.000000}%
\pgfsetstrokecolor{currentstroke}%
\pgfsetdash{}{0pt}%
\pgfpathmoveto{\pgfqpoint{0.901365in}{0.953342in}}%
\pgfpathcurveto{\pgfqpoint{0.909602in}{0.953342in}}{\pgfqpoint{0.917502in}{0.956614in}}{\pgfqpoint{0.923326in}{0.962438in}}%
\pgfpathcurveto{\pgfqpoint{0.929149in}{0.968262in}}{\pgfqpoint{0.932422in}{0.976162in}}{\pgfqpoint{0.932422in}{0.984398in}}%
\pgfpathcurveto{\pgfqpoint{0.932422in}{0.992635in}}{\pgfqpoint{0.929149in}{1.000535in}}{\pgfqpoint{0.923326in}{1.006359in}}%
\pgfpathcurveto{\pgfqpoint{0.917502in}{1.012182in}}{\pgfqpoint{0.909602in}{1.015455in}}{\pgfqpoint{0.901365in}{1.015455in}}%
\pgfpathcurveto{\pgfqpoint{0.893129in}{1.015455in}}{\pgfqpoint{0.885229in}{1.012182in}}{\pgfqpoint{0.879405in}{1.006359in}}%
\pgfpathcurveto{\pgfqpoint{0.873581in}{1.000535in}}{\pgfqpoint{0.870309in}{0.992635in}}{\pgfqpoint{0.870309in}{0.984398in}}%
\pgfpathcurveto{\pgfqpoint{0.870309in}{0.976162in}}{\pgfqpoint{0.873581in}{0.968262in}}{\pgfqpoint{0.879405in}{0.962438in}}%
\pgfpathcurveto{\pgfqpoint{0.885229in}{0.956614in}}{\pgfqpoint{0.893129in}{0.953342in}}{\pgfqpoint{0.901365in}{0.953342in}}%
\pgfpathclose%
\pgfusepath{stroke,fill}%
\end{pgfscope}%
\begin{pgfscope}%
\pgfpathrectangle{\pgfqpoint{0.556847in}{0.516222in}}{\pgfqpoint{1.722590in}{1.783528in}} %
\pgfusepath{clip}%
\pgfsetbuttcap%
\pgfsetroundjoin%
\definecolor{currentfill}{rgb}{0.298039,0.447059,0.690196}%
\pgfsetfillcolor{currentfill}%
\pgfsetlinewidth{0.240900pt}%
\definecolor{currentstroke}{rgb}{1.000000,1.000000,1.000000}%
\pgfsetstrokecolor{currentstroke}%
\pgfsetdash{}{0pt}%
\pgfpathmoveto{\pgfqpoint{1.012377in}{0.574342in}}%
\pgfpathcurveto{\pgfqpoint{1.020613in}{0.574342in}}{\pgfqpoint{1.028513in}{0.577614in}}{\pgfqpoint{1.034337in}{0.583438in}}%
\pgfpathcurveto{\pgfqpoint{1.040161in}{0.589262in}}{\pgfqpoint{1.043433in}{0.597162in}}{\pgfqpoint{1.043433in}{0.605399in}}%
\pgfpathcurveto{\pgfqpoint{1.043433in}{0.613635in}}{\pgfqpoint{1.040161in}{0.621535in}}{\pgfqpoint{1.034337in}{0.627359in}}%
\pgfpathcurveto{\pgfqpoint{1.028513in}{0.633183in}}{\pgfqpoint{1.020613in}{0.636455in}}{\pgfqpoint{1.012377in}{0.636455in}}%
\pgfpathcurveto{\pgfqpoint{1.004140in}{0.636455in}}{\pgfqpoint{0.996240in}{0.633183in}}{\pgfqpoint{0.990416in}{0.627359in}}%
\pgfpathcurveto{\pgfqpoint{0.984592in}{0.621535in}}{\pgfqpoint{0.981320in}{0.613635in}}{\pgfqpoint{0.981320in}{0.605399in}}%
\pgfpathcurveto{\pgfqpoint{0.981320in}{0.597162in}}{\pgfqpoint{0.984592in}{0.589262in}}{\pgfqpoint{0.990416in}{0.583438in}}%
\pgfpathcurveto{\pgfqpoint{0.996240in}{0.577614in}}{\pgfqpoint{1.004140in}{0.574342in}}{\pgfqpoint{1.012377in}{0.574342in}}%
\pgfpathclose%
\pgfusepath{stroke,fill}%
\end{pgfscope}%
\begin{pgfscope}%
\pgfpathrectangle{\pgfqpoint{0.556847in}{0.516222in}}{\pgfqpoint{1.722590in}{1.783528in}} %
\pgfusepath{clip}%
\pgfsetbuttcap%
\pgfsetroundjoin%
\definecolor{currentfill}{rgb}{0.298039,0.447059,0.690196}%
\pgfsetfillcolor{currentfill}%
\pgfsetlinewidth{0.240900pt}%
\definecolor{currentstroke}{rgb}{1.000000,1.000000,1.000000}%
\pgfsetstrokecolor{currentstroke}%
\pgfsetdash{}{0pt}%
\pgfpathmoveto{\pgfqpoint{0.755902in}{0.873083in}}%
\pgfpathcurveto{\pgfqpoint{0.764138in}{0.873083in}}{\pgfqpoint{0.772038in}{0.876355in}}{\pgfqpoint{0.777862in}{0.882179in}}%
\pgfpathcurveto{\pgfqpoint{0.783686in}{0.888003in}}{\pgfqpoint{0.786959in}{0.895903in}}{\pgfqpoint{0.786959in}{0.904140in}}%
\pgfpathcurveto{\pgfqpoint{0.786959in}{0.912376in}}{\pgfqpoint{0.783686in}{0.920276in}}{\pgfqpoint{0.777862in}{0.926100in}}%
\pgfpathcurveto{\pgfqpoint{0.772038in}{0.931924in}}{\pgfqpoint{0.764138in}{0.935196in}}{\pgfqpoint{0.755902in}{0.935196in}}%
\pgfpathcurveto{\pgfqpoint{0.747666in}{0.935196in}}{\pgfqpoint{0.739766in}{0.931924in}}{\pgfqpoint{0.733942in}{0.926100in}}%
\pgfpathcurveto{\pgfqpoint{0.728118in}{0.920276in}}{\pgfqpoint{0.724846in}{0.912376in}}{\pgfqpoint{0.724846in}{0.904140in}}%
\pgfpathcurveto{\pgfqpoint{0.724846in}{0.895903in}}{\pgfqpoint{0.728118in}{0.888003in}}{\pgfqpoint{0.733942in}{0.882179in}}%
\pgfpathcurveto{\pgfqpoint{0.739766in}{0.876355in}}{\pgfqpoint{0.747666in}{0.873083in}}{\pgfqpoint{0.755902in}{0.873083in}}%
\pgfpathclose%
\pgfusepath{stroke,fill}%
\end{pgfscope}%
\begin{pgfscope}%
\pgfpathrectangle{\pgfqpoint{0.556847in}{0.516222in}}{\pgfqpoint{1.722590in}{1.783528in}} %
\pgfusepath{clip}%
\pgfsetbuttcap%
\pgfsetroundjoin%
\definecolor{currentfill}{rgb}{0.298039,0.447059,0.690196}%
\pgfsetfillcolor{currentfill}%
\pgfsetlinewidth{0.240900pt}%
\definecolor{currentstroke}{rgb}{1.000000,1.000000,1.000000}%
\pgfsetstrokecolor{currentstroke}%
\pgfsetdash{}{0pt}%
\pgfpathmoveto{\pgfqpoint{0.985581in}{1.087106in}}%
\pgfpathcurveto{\pgfqpoint{0.993817in}{1.087106in}}{\pgfqpoint{1.001717in}{1.090379in}}{\pgfqpoint{1.007541in}{1.096203in}}%
\pgfpathcurveto{\pgfqpoint{1.013365in}{1.102027in}}{\pgfqpoint{1.016637in}{1.109927in}}{\pgfqpoint{1.016637in}{1.118163in}}%
\pgfpathcurveto{\pgfqpoint{1.016637in}{1.126399in}}{\pgfqpoint{1.013365in}{1.134299in}}{\pgfqpoint{1.007541in}{1.140123in}}%
\pgfpathcurveto{\pgfqpoint{1.001717in}{1.145947in}}{\pgfqpoint{0.993817in}{1.149219in}}{\pgfqpoint{0.985581in}{1.149219in}}%
\pgfpathcurveto{\pgfqpoint{0.977345in}{1.149219in}}{\pgfqpoint{0.969444in}{1.145947in}}{\pgfqpoint{0.963621in}{1.140123in}}%
\pgfpathcurveto{\pgfqpoint{0.957797in}{1.134299in}}{\pgfqpoint{0.954524in}{1.126399in}}{\pgfqpoint{0.954524in}{1.118163in}}%
\pgfpathcurveto{\pgfqpoint{0.954524in}{1.109927in}}{\pgfqpoint{0.957797in}{1.102027in}}{\pgfqpoint{0.963621in}{1.096203in}}%
\pgfpathcurveto{\pgfqpoint{0.969444in}{1.090379in}}{\pgfqpoint{0.977345in}{1.087106in}}{\pgfqpoint{0.985581in}{1.087106in}}%
\pgfpathclose%
\pgfusepath{stroke,fill}%
\end{pgfscope}%
\begin{pgfscope}%
\pgfpathrectangle{\pgfqpoint{0.556847in}{0.516222in}}{\pgfqpoint{1.722590in}{1.783528in}} %
\pgfusepath{clip}%
\pgfsetbuttcap%
\pgfsetroundjoin%
\definecolor{currentfill}{rgb}{0.298039,0.447059,0.690196}%
\pgfsetfillcolor{currentfill}%
\pgfsetlinewidth{0.240900pt}%
\definecolor{currentstroke}{rgb}{1.000000,1.000000,1.000000}%
\pgfsetstrokecolor{currentstroke}%
\pgfsetdash{}{0pt}%
\pgfpathmoveto{\pgfqpoint{1.460250in}{1.376930in}}%
\pgfpathcurveto{\pgfqpoint{1.468486in}{1.376930in}}{\pgfqpoint{1.476386in}{1.380202in}}{\pgfqpoint{1.482210in}{1.386026in}}%
\pgfpathcurveto{\pgfqpoint{1.488034in}{1.391850in}}{\pgfqpoint{1.491307in}{1.399750in}}{\pgfqpoint{1.491307in}{1.407986in}}%
\pgfpathcurveto{\pgfqpoint{1.491307in}{1.416222in}}{\pgfqpoint{1.488034in}{1.424122in}}{\pgfqpoint{1.482210in}{1.429946in}}%
\pgfpathcurveto{\pgfqpoint{1.476386in}{1.435770in}}{\pgfqpoint{1.468486in}{1.439043in}}{\pgfqpoint{1.460250in}{1.439043in}}%
\pgfpathcurveto{\pgfqpoint{1.452014in}{1.439043in}}{\pgfqpoint{1.444114in}{1.435770in}}{\pgfqpoint{1.438290in}{1.429946in}}%
\pgfpathcurveto{\pgfqpoint{1.432466in}{1.424122in}}{\pgfqpoint{1.429194in}{1.416222in}}{\pgfqpoint{1.429194in}{1.407986in}}%
\pgfpathcurveto{\pgfqpoint{1.429194in}{1.399750in}}{\pgfqpoint{1.432466in}{1.391850in}}{\pgfqpoint{1.438290in}{1.386026in}}%
\pgfpathcurveto{\pgfqpoint{1.444114in}{1.380202in}}{\pgfqpoint{1.452014in}{1.376930in}}{\pgfqpoint{1.460250in}{1.376930in}}%
\pgfpathclose%
\pgfusepath{stroke,fill}%
\end{pgfscope}%
\begin{pgfscope}%
\pgfpathrectangle{\pgfqpoint{0.556847in}{0.516222in}}{\pgfqpoint{1.722590in}{1.783528in}} %
\pgfusepath{clip}%
\pgfsetbuttcap%
\pgfsetroundjoin%
\definecolor{currentfill}{rgb}{0.298039,0.447059,0.690196}%
\pgfsetfillcolor{currentfill}%
\pgfsetlinewidth{0.240900pt}%
\definecolor{currentstroke}{rgb}{1.000000,1.000000,1.000000}%
\pgfsetstrokecolor{currentstroke}%
\pgfsetdash{}{0pt}%
\pgfpathmoveto{\pgfqpoint{1.850704in}{1.372471in}}%
\pgfpathcurveto{\pgfqpoint{1.858940in}{1.372471in}}{\pgfqpoint{1.866840in}{1.375743in}}{\pgfqpoint{1.872664in}{1.381567in}}%
\pgfpathcurveto{\pgfqpoint{1.878488in}{1.387391in}}{\pgfqpoint{1.881760in}{1.395291in}}{\pgfqpoint{1.881760in}{1.403527in}}%
\pgfpathcurveto{\pgfqpoint{1.881760in}{1.411764in}}{\pgfqpoint{1.878488in}{1.419664in}}{\pgfqpoint{1.872664in}{1.425488in}}%
\pgfpathcurveto{\pgfqpoint{1.866840in}{1.431311in}}{\pgfqpoint{1.858940in}{1.434584in}}{\pgfqpoint{1.850704in}{1.434584in}}%
\pgfpathcurveto{\pgfqpoint{1.842468in}{1.434584in}}{\pgfqpoint{1.834568in}{1.431311in}}{\pgfqpoint{1.828744in}{1.425488in}}%
\pgfpathcurveto{\pgfqpoint{1.822920in}{1.419664in}}{\pgfqpoint{1.819647in}{1.411764in}}{\pgfqpoint{1.819647in}{1.403527in}}%
\pgfpathcurveto{\pgfqpoint{1.819647in}{1.395291in}}{\pgfqpoint{1.822920in}{1.387391in}}{\pgfqpoint{1.828744in}{1.381567in}}%
\pgfpathcurveto{\pgfqpoint{1.834568in}{1.375743in}}{\pgfqpoint{1.842468in}{1.372471in}}{\pgfqpoint{1.850704in}{1.372471in}}%
\pgfpathclose%
\pgfusepath{stroke,fill}%
\end{pgfscope}%
\begin{pgfscope}%
\pgfpathrectangle{\pgfqpoint{0.556847in}{0.516222in}}{\pgfqpoint{1.722590in}{1.783528in}} %
\pgfusepath{clip}%
\pgfsetbuttcap%
\pgfsetroundjoin%
\definecolor{currentfill}{rgb}{0.298039,0.447059,0.690196}%
\pgfsetfillcolor{currentfill}%
\pgfsetlinewidth{0.240900pt}%
\definecolor{currentstroke}{rgb}{1.000000,1.000000,1.000000}%
\pgfsetstrokecolor{currentstroke}%
\pgfsetdash{}{0pt}%
\pgfpathmoveto{\pgfqpoint{1.150184in}{0.788365in}}%
\pgfpathcurveto{\pgfqpoint{1.158420in}{0.788365in}}{\pgfqpoint{1.166320in}{0.791638in}}{\pgfqpoint{1.172144in}{0.797462in}}%
\pgfpathcurveto{\pgfqpoint{1.177968in}{0.803286in}}{\pgfqpoint{1.181240in}{0.811186in}}{\pgfqpoint{1.181240in}{0.819422in}}%
\pgfpathcurveto{\pgfqpoint{1.181240in}{0.827658in}}{\pgfqpoint{1.177968in}{0.835558in}}{\pgfqpoint{1.172144in}{0.841382in}}%
\pgfpathcurveto{\pgfqpoint{1.166320in}{0.847206in}}{\pgfqpoint{1.158420in}{0.850478in}}{\pgfqpoint{1.150184in}{0.850478in}}%
\pgfpathcurveto{\pgfqpoint{1.141948in}{0.850478in}}{\pgfqpoint{1.134048in}{0.847206in}}{\pgfqpoint{1.128224in}{0.841382in}}%
\pgfpathcurveto{\pgfqpoint{1.122400in}{0.835558in}}{\pgfqpoint{1.119127in}{0.827658in}}{\pgfqpoint{1.119127in}{0.819422in}}%
\pgfpathcurveto{\pgfqpoint{1.119127in}{0.811186in}}{\pgfqpoint{1.122400in}{0.803286in}}{\pgfqpoint{1.128224in}{0.797462in}}%
\pgfpathcurveto{\pgfqpoint{1.134048in}{0.791638in}}{\pgfqpoint{1.141948in}{0.788365in}}{\pgfqpoint{1.150184in}{0.788365in}}%
\pgfpathclose%
\pgfusepath{stroke,fill}%
\end{pgfscope}%
\begin{pgfscope}%
\pgfpathrectangle{\pgfqpoint{0.556847in}{0.516222in}}{\pgfqpoint{1.722590in}{1.783528in}} %
\pgfusepath{clip}%
\pgfsetbuttcap%
\pgfsetroundjoin%
\definecolor{currentfill}{rgb}{0.298039,0.447059,0.690196}%
\pgfsetfillcolor{currentfill}%
\pgfsetlinewidth{0.240900pt}%
\definecolor{currentstroke}{rgb}{1.000000,1.000000,1.000000}%
\pgfsetstrokecolor{currentstroke}%
\pgfsetdash{}{0pt}%
\pgfpathmoveto{\pgfqpoint{1.154012in}{1.176283in}}%
\pgfpathcurveto{\pgfqpoint{1.162248in}{1.176283in}}{\pgfqpoint{1.170148in}{1.179555in}}{\pgfqpoint{1.175972in}{1.185379in}}%
\pgfpathcurveto{\pgfqpoint{1.181796in}{1.191203in}}{\pgfqpoint{1.185068in}{1.199103in}}{\pgfqpoint{1.185068in}{1.207339in}}%
\pgfpathcurveto{\pgfqpoint{1.185068in}{1.215576in}}{\pgfqpoint{1.181796in}{1.223476in}}{\pgfqpoint{1.175972in}{1.229299in}}%
\pgfpathcurveto{\pgfqpoint{1.170148in}{1.235123in}}{\pgfqpoint{1.162248in}{1.238396in}}{\pgfqpoint{1.154012in}{1.238396in}}%
\pgfpathcurveto{\pgfqpoint{1.145776in}{1.238396in}}{\pgfqpoint{1.137876in}{1.235123in}}{\pgfqpoint{1.132052in}{1.229299in}}%
\pgfpathcurveto{\pgfqpoint{1.126228in}{1.223476in}}{\pgfqpoint{1.122955in}{1.215576in}}{\pgfqpoint{1.122955in}{1.207339in}}%
\pgfpathcurveto{\pgfqpoint{1.122955in}{1.199103in}}{\pgfqpoint{1.126228in}{1.191203in}}{\pgfqpoint{1.132052in}{1.185379in}}%
\pgfpathcurveto{\pgfqpoint{1.137876in}{1.179555in}}{\pgfqpoint{1.145776in}{1.176283in}}{\pgfqpoint{1.154012in}{1.176283in}}%
\pgfpathclose%
\pgfusepath{stroke,fill}%
\end{pgfscope}%
\begin{pgfscope}%
\pgfpathrectangle{\pgfqpoint{0.556847in}{0.516222in}}{\pgfqpoint{1.722590in}{1.783528in}} %
\pgfusepath{clip}%
\pgfsetbuttcap%
\pgfsetroundjoin%
\definecolor{currentfill}{rgb}{0.298039,0.447059,0.690196}%
\pgfsetfillcolor{currentfill}%
\pgfsetlinewidth{0.240900pt}%
\definecolor{currentstroke}{rgb}{1.000000,1.000000,1.000000}%
\pgfsetstrokecolor{currentstroke}%
\pgfsetdash{}{0pt}%
\pgfpathmoveto{\pgfqpoint{0.790354in}{0.717024in}}%
\pgfpathcurveto{\pgfqpoint{0.798590in}{0.717024in}}{\pgfqpoint{0.806490in}{0.720297in}}{\pgfqpoint{0.812314in}{0.726121in}}%
\pgfpathcurveto{\pgfqpoint{0.818138in}{0.731945in}}{\pgfqpoint{0.821410in}{0.739845in}}{\pgfqpoint{0.821410in}{0.748081in}}%
\pgfpathcurveto{\pgfqpoint{0.821410in}{0.756317in}}{\pgfqpoint{0.818138in}{0.764217in}}{\pgfqpoint{0.812314in}{0.770041in}}%
\pgfpathcurveto{\pgfqpoint{0.806490in}{0.775865in}}{\pgfqpoint{0.798590in}{0.779137in}}{\pgfqpoint{0.790354in}{0.779137in}}%
\pgfpathcurveto{\pgfqpoint{0.782118in}{0.779137in}}{\pgfqpoint{0.774218in}{0.775865in}}{\pgfqpoint{0.768394in}{0.770041in}}%
\pgfpathcurveto{\pgfqpoint{0.762570in}{0.764217in}}{\pgfqpoint{0.759297in}{0.756317in}}{\pgfqpoint{0.759297in}{0.748081in}}%
\pgfpathcurveto{\pgfqpoint{0.759297in}{0.739845in}}{\pgfqpoint{0.762570in}{0.731945in}}{\pgfqpoint{0.768394in}{0.726121in}}%
\pgfpathcurveto{\pgfqpoint{0.774218in}{0.720297in}}{\pgfqpoint{0.782118in}{0.717024in}}{\pgfqpoint{0.790354in}{0.717024in}}%
\pgfpathclose%
\pgfusepath{stroke,fill}%
\end{pgfscope}%
\begin{pgfscope}%
\pgfpathrectangle{\pgfqpoint{0.556847in}{0.516222in}}{\pgfqpoint{1.722590in}{1.783528in}} %
\pgfusepath{clip}%
\pgfsetbuttcap%
\pgfsetroundjoin%
\definecolor{currentfill}{rgb}{0.298039,0.447059,0.690196}%
\pgfsetfillcolor{currentfill}%
\pgfsetlinewidth{0.240900pt}%
\definecolor{currentstroke}{rgb}{1.000000,1.000000,1.000000}%
\pgfsetstrokecolor{currentstroke}%
\pgfsetdash{}{0pt}%
\pgfpathmoveto{\pgfqpoint{1.487046in}{1.403683in}}%
\pgfpathcurveto{\pgfqpoint{1.495282in}{1.403683in}}{\pgfqpoint{1.503182in}{1.406955in}}{\pgfqpoint{1.509006in}{1.412779in}}%
\pgfpathcurveto{\pgfqpoint{1.514830in}{1.418603in}}{\pgfqpoint{1.518102in}{1.426503in}}{\pgfqpoint{1.518102in}{1.434739in}}%
\pgfpathcurveto{\pgfqpoint{1.518102in}{1.442975in}}{\pgfqpoint{1.514830in}{1.450875in}}{\pgfqpoint{1.509006in}{1.456699in}}%
\pgfpathcurveto{\pgfqpoint{1.503182in}{1.462523in}}{\pgfqpoint{1.495282in}{1.465796in}}{\pgfqpoint{1.487046in}{1.465796in}}%
\pgfpathcurveto{\pgfqpoint{1.478810in}{1.465796in}}{\pgfqpoint{1.470910in}{1.462523in}}{\pgfqpoint{1.465086in}{1.456699in}}%
\pgfpathcurveto{\pgfqpoint{1.459262in}{1.450875in}}{\pgfqpoint{1.455989in}{1.442975in}}{\pgfqpoint{1.455989in}{1.434739in}}%
\pgfpathcurveto{\pgfqpoint{1.455989in}{1.426503in}}{\pgfqpoint{1.459262in}{1.418603in}}{\pgfqpoint{1.465086in}{1.412779in}}%
\pgfpathcurveto{\pgfqpoint{1.470910in}{1.406955in}}{\pgfqpoint{1.478810in}{1.403683in}}{\pgfqpoint{1.487046in}{1.403683in}}%
\pgfpathclose%
\pgfusepath{stroke,fill}%
\end{pgfscope}%
\begin{pgfscope}%
\pgfpathrectangle{\pgfqpoint{0.556847in}{0.516222in}}{\pgfqpoint{1.722590in}{1.783528in}} %
\pgfusepath{clip}%
\pgfsetbuttcap%
\pgfsetroundjoin%
\definecolor{currentfill}{rgb}{0.298039,0.447059,0.690196}%
\pgfsetfillcolor{currentfill}%
\pgfsetlinewidth{0.240900pt}%
\definecolor{currentstroke}{rgb}{1.000000,1.000000,1.000000}%
\pgfsetstrokecolor{currentstroke}%
\pgfsetdash{}{0pt}%
\pgfpathmoveto{\pgfqpoint{1.161668in}{0.873083in}}%
\pgfpathcurveto{\pgfqpoint{1.169904in}{0.873083in}}{\pgfqpoint{1.177804in}{0.876355in}}{\pgfqpoint{1.183628in}{0.882179in}}%
\pgfpathcurveto{\pgfqpoint{1.189452in}{0.888003in}}{\pgfqpoint{1.192724in}{0.895903in}}{\pgfqpoint{1.192724in}{0.904140in}}%
\pgfpathcurveto{\pgfqpoint{1.192724in}{0.912376in}}{\pgfqpoint{1.189452in}{0.920276in}}{\pgfqpoint{1.183628in}{0.926100in}}%
\pgfpathcurveto{\pgfqpoint{1.177804in}{0.931924in}}{\pgfqpoint{1.169904in}{0.935196in}}{\pgfqpoint{1.161668in}{0.935196in}}%
\pgfpathcurveto{\pgfqpoint{1.153432in}{0.935196in}}{\pgfqpoint{1.145531in}{0.931924in}}{\pgfqpoint{1.139708in}{0.926100in}}%
\pgfpathcurveto{\pgfqpoint{1.133884in}{0.920276in}}{\pgfqpoint{1.130611in}{0.912376in}}{\pgfqpoint{1.130611in}{0.904140in}}%
\pgfpathcurveto{\pgfqpoint{1.130611in}{0.895903in}}{\pgfqpoint{1.133884in}{0.888003in}}{\pgfqpoint{1.139708in}{0.882179in}}%
\pgfpathcurveto{\pgfqpoint{1.145531in}{0.876355in}}{\pgfqpoint{1.153432in}{0.873083in}}{\pgfqpoint{1.161668in}{0.873083in}}%
\pgfpathclose%
\pgfusepath{stroke,fill}%
\end{pgfscope}%
\begin{pgfscope}%
\pgfpathrectangle{\pgfqpoint{0.556847in}{0.516222in}}{\pgfqpoint{1.722590in}{1.783528in}} %
\pgfusepath{clip}%
\pgfsetbuttcap%
\pgfsetroundjoin%
\definecolor{currentfill}{rgb}{0.298039,0.447059,0.690196}%
\pgfsetfillcolor{currentfill}%
\pgfsetlinewidth{0.240900pt}%
\definecolor{currentstroke}{rgb}{1.000000,1.000000,1.000000}%
\pgfsetstrokecolor{currentstroke}%
\pgfsetdash{}{0pt}%
\pgfpathmoveto{\pgfqpoint{0.721450in}{0.971177in}}%
\pgfpathcurveto{\pgfqpoint{0.729687in}{0.971177in}}{\pgfqpoint{0.737587in}{0.974449in}}{\pgfqpoint{0.743411in}{0.980273in}}%
\pgfpathcurveto{\pgfqpoint{0.749234in}{0.986097in}}{\pgfqpoint{0.752507in}{0.993997in}}{\pgfqpoint{0.752507in}{1.002234in}}%
\pgfpathcurveto{\pgfqpoint{0.752507in}{1.010470in}}{\pgfqpoint{0.749234in}{1.018370in}}{\pgfqpoint{0.743411in}{1.024194in}}%
\pgfpathcurveto{\pgfqpoint{0.737587in}{1.030018in}}{\pgfqpoint{0.729687in}{1.033290in}}{\pgfqpoint{0.721450in}{1.033290in}}%
\pgfpathcurveto{\pgfqpoint{0.713214in}{1.033290in}}{\pgfqpoint{0.705314in}{1.030018in}}{\pgfqpoint{0.699490in}{1.024194in}}%
\pgfpathcurveto{\pgfqpoint{0.693666in}{1.018370in}}{\pgfqpoint{0.690394in}{1.010470in}}{\pgfqpoint{0.690394in}{1.002234in}}%
\pgfpathcurveto{\pgfqpoint{0.690394in}{0.993997in}}{\pgfqpoint{0.693666in}{0.986097in}}{\pgfqpoint{0.699490in}{0.980273in}}%
\pgfpathcurveto{\pgfqpoint{0.705314in}{0.974449in}}{\pgfqpoint{0.713214in}{0.971177in}}{\pgfqpoint{0.721450in}{0.971177in}}%
\pgfpathclose%
\pgfusepath{stroke,fill}%
\end{pgfscope}%
\begin{pgfscope}%
\pgfpathrectangle{\pgfqpoint{0.556847in}{0.516222in}}{\pgfqpoint{1.722590in}{1.783528in}} %
\pgfusepath{clip}%
\pgfsetbuttcap%
\pgfsetroundjoin%
\definecolor{currentfill}{rgb}{0.298039,0.447059,0.690196}%
\pgfsetfillcolor{currentfill}%
\pgfsetlinewidth{0.240900pt}%
\definecolor{currentstroke}{rgb}{1.000000,1.000000,1.000000}%
\pgfsetstrokecolor{currentstroke}%
\pgfsetdash{}{0pt}%
\pgfpathmoveto{\pgfqpoint{1.008549in}{1.403683in}}%
\pgfpathcurveto{\pgfqpoint{1.016785in}{1.403683in}}{\pgfqpoint{1.024685in}{1.406955in}}{\pgfqpoint{1.030509in}{1.412779in}}%
\pgfpathcurveto{\pgfqpoint{1.036333in}{1.418603in}}{\pgfqpoint{1.039605in}{1.426503in}}{\pgfqpoint{1.039605in}{1.434739in}}%
\pgfpathcurveto{\pgfqpoint{1.039605in}{1.442975in}}{\pgfqpoint{1.036333in}{1.450875in}}{\pgfqpoint{1.030509in}{1.456699in}}%
\pgfpathcurveto{\pgfqpoint{1.024685in}{1.462523in}}{\pgfqpoint{1.016785in}{1.465796in}}{\pgfqpoint{1.008549in}{1.465796in}}%
\pgfpathcurveto{\pgfqpoint{1.000312in}{1.465796in}}{\pgfqpoint{0.992412in}{1.462523in}}{\pgfqpoint{0.986588in}{1.456699in}}%
\pgfpathcurveto{\pgfqpoint{0.980764in}{1.450875in}}{\pgfqpoint{0.977492in}{1.442975in}}{\pgfqpoint{0.977492in}{1.434739in}}%
\pgfpathcurveto{\pgfqpoint{0.977492in}{1.426503in}}{\pgfqpoint{0.980764in}{1.418603in}}{\pgfqpoint{0.986588in}{1.412779in}}%
\pgfpathcurveto{\pgfqpoint{0.992412in}{1.406955in}}{\pgfqpoint{1.000312in}{1.403683in}}{\pgfqpoint{1.008549in}{1.403683in}}%
\pgfpathclose%
\pgfusepath{stroke,fill}%
\end{pgfscope}%
\begin{pgfscope}%
\pgfpathrectangle{\pgfqpoint{0.556847in}{0.516222in}}{\pgfqpoint{1.722590in}{1.783528in}} %
\pgfusepath{clip}%
\pgfsetbuttcap%
\pgfsetroundjoin%
\definecolor{currentfill}{rgb}{0.298039,0.447059,0.690196}%
\pgfsetfillcolor{currentfill}%
\pgfsetlinewidth{0.240900pt}%
\definecolor{currentstroke}{rgb}{1.000000,1.000000,1.000000}%
\pgfsetstrokecolor{currentstroke}%
\pgfsetdash{}{0pt}%
\pgfpathmoveto{\pgfqpoint{1.663133in}{1.537447in}}%
\pgfpathcurveto{\pgfqpoint{1.671369in}{1.537447in}}{\pgfqpoint{1.679269in}{1.540719in}}{\pgfqpoint{1.685093in}{1.546543in}}%
\pgfpathcurveto{\pgfqpoint{1.690917in}{1.552367in}}{\pgfqpoint{1.694189in}{1.560267in}}{\pgfqpoint{1.694189in}{1.568504in}}%
\pgfpathcurveto{\pgfqpoint{1.694189in}{1.576740in}}{\pgfqpoint{1.690917in}{1.584640in}}{\pgfqpoint{1.685093in}{1.590464in}}%
\pgfpathcurveto{\pgfqpoint{1.679269in}{1.596288in}}{\pgfqpoint{1.671369in}{1.599560in}}{\pgfqpoint{1.663133in}{1.599560in}}%
\pgfpathcurveto{\pgfqpoint{1.654897in}{1.599560in}}{\pgfqpoint{1.646997in}{1.596288in}}{\pgfqpoint{1.641173in}{1.590464in}}%
\pgfpathcurveto{\pgfqpoint{1.635349in}{1.584640in}}{\pgfqpoint{1.632076in}{1.576740in}}{\pgfqpoint{1.632076in}{1.568504in}}%
\pgfpathcurveto{\pgfqpoint{1.632076in}{1.560267in}}{\pgfqpoint{1.635349in}{1.552367in}}{\pgfqpoint{1.641173in}{1.546543in}}%
\pgfpathcurveto{\pgfqpoint{1.646997in}{1.540719in}}{\pgfqpoint{1.654897in}{1.537447in}}{\pgfqpoint{1.663133in}{1.537447in}}%
\pgfpathclose%
\pgfusepath{stroke,fill}%
\end{pgfscope}%
\begin{pgfscope}%
\pgfpathrectangle{\pgfqpoint{0.556847in}{0.516222in}}{\pgfqpoint{1.722590in}{1.783528in}} %
\pgfusepath{clip}%
\pgfsetbuttcap%
\pgfsetroundjoin%
\definecolor{currentfill}{rgb}{0.298039,0.447059,0.690196}%
\pgfsetfillcolor{currentfill}%
\pgfsetlinewidth{0.240900pt}%
\definecolor{currentstroke}{rgb}{1.000000,1.000000,1.000000}%
\pgfsetstrokecolor{currentstroke}%
\pgfsetdash{}{0pt}%
\pgfpathmoveto{\pgfqpoint{0.870741in}{0.877542in}}%
\pgfpathcurveto{\pgfqpoint{0.878978in}{0.877542in}}{\pgfqpoint{0.886878in}{0.880814in}}{\pgfqpoint{0.892702in}{0.886638in}}%
\pgfpathcurveto{\pgfqpoint{0.898526in}{0.892462in}}{\pgfqpoint{0.901798in}{0.900362in}}{\pgfqpoint{0.901798in}{0.908598in}}%
\pgfpathcurveto{\pgfqpoint{0.901798in}{0.916835in}}{\pgfqpoint{0.898526in}{0.924735in}}{\pgfqpoint{0.892702in}{0.930559in}}%
\pgfpathcurveto{\pgfqpoint{0.886878in}{0.936383in}}{\pgfqpoint{0.878978in}{0.939655in}}{\pgfqpoint{0.870741in}{0.939655in}}%
\pgfpathcurveto{\pgfqpoint{0.862505in}{0.939655in}}{\pgfqpoint{0.854605in}{0.936383in}}{\pgfqpoint{0.848781in}{0.930559in}}%
\pgfpathcurveto{\pgfqpoint{0.842957in}{0.924735in}}{\pgfqpoint{0.839685in}{0.916835in}}{\pgfqpoint{0.839685in}{0.908598in}}%
\pgfpathcurveto{\pgfqpoint{0.839685in}{0.900362in}}{\pgfqpoint{0.842957in}{0.892462in}}{\pgfqpoint{0.848781in}{0.886638in}}%
\pgfpathcurveto{\pgfqpoint{0.854605in}{0.880814in}}{\pgfqpoint{0.862505in}{0.877542in}}{\pgfqpoint{0.870741in}{0.877542in}}%
\pgfpathclose%
\pgfusepath{stroke,fill}%
\end{pgfscope}%
\begin{pgfscope}%
\pgfpathrectangle{\pgfqpoint{0.556847in}{0.516222in}}{\pgfqpoint{1.722590in}{1.783528in}} %
\pgfusepath{clip}%
\pgfsetbuttcap%
\pgfsetroundjoin%
\definecolor{currentfill}{rgb}{0.298039,0.447059,0.690196}%
\pgfsetfillcolor{currentfill}%
\pgfsetlinewidth{0.240900pt}%
\definecolor{currentstroke}{rgb}{1.000000,1.000000,1.000000}%
\pgfsetstrokecolor{currentstroke}%
\pgfsetdash{}{0pt}%
\pgfpathmoveto{\pgfqpoint{0.843946in}{0.815118in}}%
\pgfpathcurveto{\pgfqpoint{0.852182in}{0.815118in}}{\pgfqpoint{0.860082in}{0.818391in}}{\pgfqpoint{0.865906in}{0.824215in}}%
\pgfpathcurveto{\pgfqpoint{0.871730in}{0.830039in}}{\pgfqpoint{0.875002in}{0.837939in}}{\pgfqpoint{0.875002in}{0.846175in}}%
\pgfpathcurveto{\pgfqpoint{0.875002in}{0.854411in}}{\pgfqpoint{0.871730in}{0.862311in}}{\pgfqpoint{0.865906in}{0.868135in}}%
\pgfpathcurveto{\pgfqpoint{0.860082in}{0.873959in}}{\pgfqpoint{0.852182in}{0.877231in}}{\pgfqpoint{0.843946in}{0.877231in}}%
\pgfpathcurveto{\pgfqpoint{0.835709in}{0.877231in}}{\pgfqpoint{0.827809in}{0.873959in}}{\pgfqpoint{0.821985in}{0.868135in}}%
\pgfpathcurveto{\pgfqpoint{0.816161in}{0.862311in}}{\pgfqpoint{0.812889in}{0.854411in}}{\pgfqpoint{0.812889in}{0.846175in}}%
\pgfpathcurveto{\pgfqpoint{0.812889in}{0.837939in}}{\pgfqpoint{0.816161in}{0.830039in}}{\pgfqpoint{0.821985in}{0.824215in}}%
\pgfpathcurveto{\pgfqpoint{0.827809in}{0.818391in}}{\pgfqpoint{0.835709in}{0.815118in}}{\pgfqpoint{0.843946in}{0.815118in}}%
\pgfpathclose%
\pgfusepath{stroke,fill}%
\end{pgfscope}%
\begin{pgfscope}%
\pgfpathrectangle{\pgfqpoint{0.556847in}{0.516222in}}{\pgfqpoint{1.722590in}{1.783528in}} %
\pgfusepath{clip}%
\pgfsetbuttcap%
\pgfsetroundjoin%
\definecolor{currentfill}{rgb}{0.298039,0.447059,0.690196}%
\pgfsetfillcolor{currentfill}%
\pgfsetlinewidth{0.240900pt}%
\definecolor{currentstroke}{rgb}{1.000000,1.000000,1.000000}%
\pgfsetstrokecolor{currentstroke}%
\pgfsetdash{}{0pt}%
\pgfpathmoveto{\pgfqpoint{1.261195in}{0.993471in}}%
\pgfpathcurveto{\pgfqpoint{1.269432in}{0.993471in}}{\pgfqpoint{1.277332in}{0.996743in}}{\pgfqpoint{1.283156in}{1.002567in}}%
\pgfpathcurveto{\pgfqpoint{1.288979in}{1.008391in}}{\pgfqpoint{1.292252in}{1.016291in}}{\pgfqpoint{1.292252in}{1.024528in}}%
\pgfpathcurveto{\pgfqpoint{1.292252in}{1.032764in}}{\pgfqpoint{1.288979in}{1.040664in}}{\pgfqpoint{1.283156in}{1.046488in}}%
\pgfpathcurveto{\pgfqpoint{1.277332in}{1.052312in}}{\pgfqpoint{1.269432in}{1.055584in}}{\pgfqpoint{1.261195in}{1.055584in}}%
\pgfpathcurveto{\pgfqpoint{1.252959in}{1.055584in}}{\pgfqpoint{1.245059in}{1.052312in}}{\pgfqpoint{1.239235in}{1.046488in}}%
\pgfpathcurveto{\pgfqpoint{1.233411in}{1.040664in}}{\pgfqpoint{1.230139in}{1.032764in}}{\pgfqpoint{1.230139in}{1.024528in}}%
\pgfpathcurveto{\pgfqpoint{1.230139in}{1.016291in}}{\pgfqpoint{1.233411in}{1.008391in}}{\pgfqpoint{1.239235in}{1.002567in}}%
\pgfpathcurveto{\pgfqpoint{1.245059in}{0.996743in}}{\pgfqpoint{1.252959in}{0.993471in}}{\pgfqpoint{1.261195in}{0.993471in}}%
\pgfpathclose%
\pgfusepath{stroke,fill}%
\end{pgfscope}%
\begin{pgfscope}%
\pgfpathrectangle{\pgfqpoint{0.556847in}{0.516222in}}{\pgfqpoint{1.722590in}{1.783528in}} %
\pgfusepath{clip}%
\pgfsetbuttcap%
\pgfsetroundjoin%
\definecolor{currentfill}{rgb}{0.298039,0.447059,0.690196}%
\pgfsetfillcolor{currentfill}%
\pgfsetlinewidth{0.240900pt}%
\definecolor{currentstroke}{rgb}{1.000000,1.000000,1.000000}%
\pgfsetstrokecolor{currentstroke}%
\pgfsetdash{}{0pt}%
\pgfpathmoveto{\pgfqpoint{1.115732in}{0.819577in}}%
\pgfpathcurveto{\pgfqpoint{1.123968in}{0.819577in}}{\pgfqpoint{1.131868in}{0.822849in}}{\pgfqpoint{1.137692in}{0.828673in}}%
\pgfpathcurveto{\pgfqpoint{1.143516in}{0.834497in}}{\pgfqpoint{1.146789in}{0.842397in}}{\pgfqpoint{1.146789in}{0.850634in}}%
\pgfpathcurveto{\pgfqpoint{1.146789in}{0.858870in}}{\pgfqpoint{1.143516in}{0.866770in}}{\pgfqpoint{1.137692in}{0.872594in}}%
\pgfpathcurveto{\pgfqpoint{1.131868in}{0.878418in}}{\pgfqpoint{1.123968in}{0.881690in}}{\pgfqpoint{1.115732in}{0.881690in}}%
\pgfpathcurveto{\pgfqpoint{1.107496in}{0.881690in}}{\pgfqpoint{1.099596in}{0.878418in}}{\pgfqpoint{1.093772in}{0.872594in}}%
\pgfpathcurveto{\pgfqpoint{1.087948in}{0.866770in}}{\pgfqpoint{1.084676in}{0.858870in}}{\pgfqpoint{1.084676in}{0.850634in}}%
\pgfpathcurveto{\pgfqpoint{1.084676in}{0.842397in}}{\pgfqpoint{1.087948in}{0.834497in}}{\pgfqpoint{1.093772in}{0.828673in}}%
\pgfpathcurveto{\pgfqpoint{1.099596in}{0.822849in}}{\pgfqpoint{1.107496in}{0.819577in}}{\pgfqpoint{1.115732in}{0.819577in}}%
\pgfpathclose%
\pgfusepath{stroke,fill}%
\end{pgfscope}%
\begin{pgfscope}%
\pgfpathrectangle{\pgfqpoint{0.556847in}{0.516222in}}{\pgfqpoint{1.722590in}{1.783528in}} %
\pgfusepath{clip}%
\pgfsetbuttcap%
\pgfsetroundjoin%
\definecolor{currentfill}{rgb}{0.298039,0.447059,0.690196}%
\pgfsetfillcolor{currentfill}%
\pgfsetlinewidth{0.240900pt}%
\definecolor{currentstroke}{rgb}{1.000000,1.000000,1.000000}%
\pgfsetstrokecolor{currentstroke}%
\pgfsetdash{}{0pt}%
\pgfpathmoveto{\pgfqpoint{1.127216in}{1.390306in}}%
\pgfpathcurveto{\pgfqpoint{1.135452in}{1.390306in}}{\pgfqpoint{1.143352in}{1.393578in}}{\pgfqpoint{1.149176in}{1.399402in}}%
\pgfpathcurveto{\pgfqpoint{1.155000in}{1.405226in}}{\pgfqpoint{1.158273in}{1.413126in}}{\pgfqpoint{1.158273in}{1.421363in}}%
\pgfpathcurveto{\pgfqpoint{1.158273in}{1.429599in}}{\pgfqpoint{1.155000in}{1.437499in}}{\pgfqpoint{1.149176in}{1.443323in}}%
\pgfpathcurveto{\pgfqpoint{1.143352in}{1.449147in}}{\pgfqpoint{1.135452in}{1.452419in}}{\pgfqpoint{1.127216in}{1.452419in}}%
\pgfpathcurveto{\pgfqpoint{1.118980in}{1.452419in}}{\pgfqpoint{1.111080in}{1.449147in}}{\pgfqpoint{1.105256in}{1.443323in}}%
\pgfpathcurveto{\pgfqpoint{1.099432in}{1.437499in}}{\pgfqpoint{1.096160in}{1.429599in}}{\pgfqpoint{1.096160in}{1.421363in}}%
\pgfpathcurveto{\pgfqpoint{1.096160in}{1.413126in}}{\pgfqpoint{1.099432in}{1.405226in}}{\pgfqpoint{1.105256in}{1.399402in}}%
\pgfpathcurveto{\pgfqpoint{1.111080in}{1.393578in}}{\pgfqpoint{1.118980in}{1.390306in}}{\pgfqpoint{1.127216in}{1.390306in}}%
\pgfpathclose%
\pgfusepath{stroke,fill}%
\end{pgfscope}%
\begin{pgfscope}%
\pgfpathrectangle{\pgfqpoint{0.556847in}{0.516222in}}{\pgfqpoint{1.722590in}{1.783528in}} %
\pgfusepath{clip}%
\pgfsetbuttcap%
\pgfsetroundjoin%
\definecolor{currentfill}{rgb}{0.298039,0.447059,0.690196}%
\pgfsetfillcolor{currentfill}%
\pgfsetlinewidth{0.240900pt}%
\definecolor{currentstroke}{rgb}{1.000000,1.000000,1.000000}%
\pgfsetstrokecolor{currentstroke}%
\pgfsetdash{}{0pt}%
\pgfpathmoveto{\pgfqpoint{1.318615in}{1.488400in}}%
\pgfpathcurveto{\pgfqpoint{1.326851in}{1.488400in}}{\pgfqpoint{1.334751in}{1.491672in}}{\pgfqpoint{1.340575in}{1.497496in}}%
\pgfpathcurveto{\pgfqpoint{1.346399in}{1.503320in}}{\pgfqpoint{1.349671in}{1.511220in}}{\pgfqpoint{1.349671in}{1.519457in}}%
\pgfpathcurveto{\pgfqpoint{1.349671in}{1.527693in}}{\pgfqpoint{1.346399in}{1.535593in}}{\pgfqpoint{1.340575in}{1.541417in}}%
\pgfpathcurveto{\pgfqpoint{1.334751in}{1.547241in}}{\pgfqpoint{1.326851in}{1.550513in}}{\pgfqpoint{1.318615in}{1.550513in}}%
\pgfpathcurveto{\pgfqpoint{1.310379in}{1.550513in}}{\pgfqpoint{1.302479in}{1.547241in}}{\pgfqpoint{1.296655in}{1.541417in}}%
\pgfpathcurveto{\pgfqpoint{1.290831in}{1.535593in}}{\pgfqpoint{1.287558in}{1.527693in}}{\pgfqpoint{1.287558in}{1.519457in}}%
\pgfpathcurveto{\pgfqpoint{1.287558in}{1.511220in}}{\pgfqpoint{1.290831in}{1.503320in}}{\pgfqpoint{1.296655in}{1.497496in}}%
\pgfpathcurveto{\pgfqpoint{1.302479in}{1.491672in}}{\pgfqpoint{1.310379in}{1.488400in}}{\pgfqpoint{1.318615in}{1.488400in}}%
\pgfpathclose%
\pgfusepath{stroke,fill}%
\end{pgfscope}%
\begin{pgfscope}%
\pgfpathrectangle{\pgfqpoint{0.556847in}{0.516222in}}{\pgfqpoint{1.722590in}{1.783528in}} %
\pgfusepath{clip}%
\pgfsetbuttcap%
\pgfsetroundjoin%
\definecolor{currentfill}{rgb}{0.298039,0.447059,0.690196}%
\pgfsetfillcolor{currentfill}%
\pgfsetlinewidth{0.240900pt}%
\definecolor{currentstroke}{rgb}{1.000000,1.000000,1.000000}%
\pgfsetstrokecolor{currentstroke}%
\pgfsetdash{}{0pt}%
\pgfpathmoveto{\pgfqpoint{1.444938in}{1.457188in}}%
\pgfpathcurveto{\pgfqpoint{1.453174in}{1.457188in}}{\pgfqpoint{1.461075in}{1.460461in}}{\pgfqpoint{1.466898in}{1.466285in}}%
\pgfpathcurveto{\pgfqpoint{1.472722in}{1.472109in}}{\pgfqpoint{1.475995in}{1.480009in}}{\pgfqpoint{1.475995in}{1.488245in}}%
\pgfpathcurveto{\pgfqpoint{1.475995in}{1.496481in}}{\pgfqpoint{1.472722in}{1.504381in}}{\pgfqpoint{1.466898in}{1.510205in}}%
\pgfpathcurveto{\pgfqpoint{1.461075in}{1.516029in}}{\pgfqpoint{1.453174in}{1.519301in}}{\pgfqpoint{1.444938in}{1.519301in}}%
\pgfpathcurveto{\pgfqpoint{1.436702in}{1.519301in}}{\pgfqpoint{1.428802in}{1.516029in}}{\pgfqpoint{1.422978in}{1.510205in}}%
\pgfpathcurveto{\pgfqpoint{1.417154in}{1.504381in}}{\pgfqpoint{1.413882in}{1.496481in}}{\pgfqpoint{1.413882in}{1.488245in}}%
\pgfpathcurveto{\pgfqpoint{1.413882in}{1.480009in}}{\pgfqpoint{1.417154in}{1.472109in}}{\pgfqpoint{1.422978in}{1.466285in}}%
\pgfpathcurveto{\pgfqpoint{1.428802in}{1.460461in}}{\pgfqpoint{1.436702in}{1.457188in}}{\pgfqpoint{1.444938in}{1.457188in}}%
\pgfpathclose%
\pgfusepath{stroke,fill}%
\end{pgfscope}%
\begin{pgfscope}%
\pgfpathrectangle{\pgfqpoint{0.556847in}{0.516222in}}{\pgfqpoint{1.722590in}{1.783528in}} %
\pgfusepath{clip}%
\pgfsetbuttcap%
\pgfsetroundjoin%
\definecolor{currentfill}{rgb}{0.298039,0.447059,0.690196}%
\pgfsetfillcolor{currentfill}%
\pgfsetlinewidth{0.240900pt}%
\definecolor{currentstroke}{rgb}{1.000000,1.000000,1.000000}%
\pgfsetstrokecolor{currentstroke}%
\pgfsetdash{}{0pt}%
\pgfpathmoveto{\pgfqpoint{1.188464in}{1.153989in}}%
\pgfpathcurveto{\pgfqpoint{1.196700in}{1.153989in}}{\pgfqpoint{1.204600in}{1.157261in}}{\pgfqpoint{1.210424in}{1.163085in}}%
\pgfpathcurveto{\pgfqpoint{1.216248in}{1.168909in}}{\pgfqpoint{1.219520in}{1.176809in}}{\pgfqpoint{1.219520in}{1.185045in}}%
\pgfpathcurveto{\pgfqpoint{1.219520in}{1.193281in}}{\pgfqpoint{1.216248in}{1.201181in}}{\pgfqpoint{1.210424in}{1.207005in}}%
\pgfpathcurveto{\pgfqpoint{1.204600in}{1.212829in}}{\pgfqpoint{1.196700in}{1.216102in}}{\pgfqpoint{1.188464in}{1.216102in}}%
\pgfpathcurveto{\pgfqpoint{1.180227in}{1.216102in}}{\pgfqpoint{1.172327in}{1.212829in}}{\pgfqpoint{1.166503in}{1.207005in}}%
\pgfpathcurveto{\pgfqpoint{1.160679in}{1.201181in}}{\pgfqpoint{1.157407in}{1.193281in}}{\pgfqpoint{1.157407in}{1.185045in}}%
\pgfpathcurveto{\pgfqpoint{1.157407in}{1.176809in}}{\pgfqpoint{1.160679in}{1.168909in}}{\pgfqpoint{1.166503in}{1.163085in}}%
\pgfpathcurveto{\pgfqpoint{1.172327in}{1.157261in}}{\pgfqpoint{1.180227in}{1.153989in}}{\pgfqpoint{1.188464in}{1.153989in}}%
\pgfpathclose%
\pgfusepath{stroke,fill}%
\end{pgfscope}%
\begin{pgfscope}%
\pgfpathrectangle{\pgfqpoint{0.556847in}{0.516222in}}{\pgfqpoint{1.722590in}{1.783528in}} %
\pgfusepath{clip}%
\pgfsetbuttcap%
\pgfsetroundjoin%
\definecolor{currentfill}{rgb}{0.298039,0.447059,0.690196}%
\pgfsetfillcolor{currentfill}%
\pgfsetlinewidth{0.240900pt}%
\definecolor{currentstroke}{rgb}{1.000000,1.000000,1.000000}%
\pgfsetstrokecolor{currentstroke}%
\pgfsetdash{}{0pt}%
\pgfpathmoveto{\pgfqpoint{1.800940in}{1.368012in}}%
\pgfpathcurveto{\pgfqpoint{1.809176in}{1.368012in}}{\pgfqpoint{1.817077in}{1.371284in}}{\pgfqpoint{1.822900in}{1.377108in}}%
\pgfpathcurveto{\pgfqpoint{1.828724in}{1.382932in}}{\pgfqpoint{1.831997in}{1.390832in}}{\pgfqpoint{1.831997in}{1.399068in}}%
\pgfpathcurveto{\pgfqpoint{1.831997in}{1.407305in}}{\pgfqpoint{1.828724in}{1.415205in}}{\pgfqpoint{1.822900in}{1.421029in}}%
\pgfpathcurveto{\pgfqpoint{1.817077in}{1.426853in}}{\pgfqpoint{1.809176in}{1.430125in}}{\pgfqpoint{1.800940in}{1.430125in}}%
\pgfpathcurveto{\pgfqpoint{1.792704in}{1.430125in}}{\pgfqpoint{1.784804in}{1.426853in}}{\pgfqpoint{1.778980in}{1.421029in}}%
\pgfpathcurveto{\pgfqpoint{1.773156in}{1.415205in}}{\pgfqpoint{1.769884in}{1.407305in}}{\pgfqpoint{1.769884in}{1.399068in}}%
\pgfpathcurveto{\pgfqpoint{1.769884in}{1.390832in}}{\pgfqpoint{1.773156in}{1.382932in}}{\pgfqpoint{1.778980in}{1.377108in}}%
\pgfpathcurveto{\pgfqpoint{1.784804in}{1.371284in}}{\pgfqpoint{1.792704in}{1.368012in}}{\pgfqpoint{1.800940in}{1.368012in}}%
\pgfpathclose%
\pgfusepath{stroke,fill}%
\end{pgfscope}%
\begin{pgfscope}%
\pgfpathrectangle{\pgfqpoint{0.556847in}{0.516222in}}{\pgfqpoint{1.722590in}{1.783528in}} %
\pgfusepath{clip}%
\pgfsetbuttcap%
\pgfsetroundjoin%
\definecolor{currentfill}{rgb}{0.298039,0.447059,0.690196}%
\pgfsetfillcolor{currentfill}%
\pgfsetlinewidth{0.240900pt}%
\definecolor{currentstroke}{rgb}{1.000000,1.000000,1.000000}%
\pgfsetstrokecolor{currentstroke}%
\pgfsetdash{}{0pt}%
\pgfpathmoveto{\pgfqpoint{1.188464in}{1.359094in}}%
\pgfpathcurveto{\pgfqpoint{1.196700in}{1.359094in}}{\pgfqpoint{1.204600in}{1.362367in}}{\pgfqpoint{1.210424in}{1.368191in}}%
\pgfpathcurveto{\pgfqpoint{1.216248in}{1.374015in}}{\pgfqpoint{1.219520in}{1.381915in}}{\pgfqpoint{1.219520in}{1.390151in}}%
\pgfpathcurveto{\pgfqpoint{1.219520in}{1.398387in}}{\pgfqpoint{1.216248in}{1.406287in}}{\pgfqpoint{1.210424in}{1.412111in}}%
\pgfpathcurveto{\pgfqpoint{1.204600in}{1.417935in}}{\pgfqpoint{1.196700in}{1.421207in}}{\pgfqpoint{1.188464in}{1.421207in}}%
\pgfpathcurveto{\pgfqpoint{1.180227in}{1.421207in}}{\pgfqpoint{1.172327in}{1.417935in}}{\pgfqpoint{1.166503in}{1.412111in}}%
\pgfpathcurveto{\pgfqpoint{1.160679in}{1.406287in}}{\pgfqpoint{1.157407in}{1.398387in}}{\pgfqpoint{1.157407in}{1.390151in}}%
\pgfpathcurveto{\pgfqpoint{1.157407in}{1.381915in}}{\pgfqpoint{1.160679in}{1.374015in}}{\pgfqpoint{1.166503in}{1.368191in}}%
\pgfpathcurveto{\pgfqpoint{1.172327in}{1.362367in}}{\pgfqpoint{1.180227in}{1.359094in}}{\pgfqpoint{1.188464in}{1.359094in}}%
\pgfpathclose%
\pgfusepath{stroke,fill}%
\end{pgfscope}%
\begin{pgfscope}%
\pgfpathrectangle{\pgfqpoint{0.556847in}{0.516222in}}{\pgfqpoint{1.722590in}{1.783528in}} %
\pgfusepath{clip}%
\pgfsetbuttcap%
\pgfsetroundjoin%
\definecolor{currentfill}{rgb}{0.298039,0.447059,0.690196}%
\pgfsetfillcolor{currentfill}%
\pgfsetlinewidth{0.240900pt}%
\definecolor{currentstroke}{rgb}{1.000000,1.000000,1.000000}%
\pgfsetstrokecolor{currentstroke}%
\pgfsetdash{}{0pt}%
\pgfpathmoveto{\pgfqpoint{1.108076in}{1.278836in}}%
\pgfpathcurveto{\pgfqpoint{1.116312in}{1.278836in}}{\pgfqpoint{1.124212in}{1.282108in}}{\pgfqpoint{1.130036in}{1.287932in}}%
\pgfpathcurveto{\pgfqpoint{1.135860in}{1.293756in}}{\pgfqpoint{1.139133in}{1.301656in}}{\pgfqpoint{1.139133in}{1.309892in}}%
\pgfpathcurveto{\pgfqpoint{1.139133in}{1.318128in}}{\pgfqpoint{1.135860in}{1.326028in}}{\pgfqpoint{1.130036in}{1.331852in}}%
\pgfpathcurveto{\pgfqpoint{1.124212in}{1.337676in}}{\pgfqpoint{1.116312in}{1.340949in}}{\pgfqpoint{1.108076in}{1.340949in}}%
\pgfpathcurveto{\pgfqpoint{1.099840in}{1.340949in}}{\pgfqpoint{1.091940in}{1.337676in}}{\pgfqpoint{1.086116in}{1.331852in}}%
\pgfpathcurveto{\pgfqpoint{1.080292in}{1.326028in}}{\pgfqpoint{1.077020in}{1.318128in}}{\pgfqpoint{1.077020in}{1.309892in}}%
\pgfpathcurveto{\pgfqpoint{1.077020in}{1.301656in}}{\pgfqpoint{1.080292in}{1.293756in}}{\pgfqpoint{1.086116in}{1.287932in}}%
\pgfpathcurveto{\pgfqpoint{1.091940in}{1.282108in}}{\pgfqpoint{1.099840in}{1.278836in}}{\pgfqpoint{1.108076in}{1.278836in}}%
\pgfpathclose%
\pgfusepath{stroke,fill}%
\end{pgfscope}%
\begin{pgfscope}%
\pgfpathrectangle{\pgfqpoint{0.556847in}{0.516222in}}{\pgfqpoint{1.722590in}{1.783528in}} %
\pgfusepath{clip}%
\pgfsetbuttcap%
\pgfsetroundjoin%
\definecolor{currentfill}{rgb}{0.298039,0.447059,0.690196}%
\pgfsetfillcolor{currentfill}%
\pgfsetlinewidth{0.240900pt}%
\definecolor{currentstroke}{rgb}{1.000000,1.000000,1.000000}%
\pgfsetstrokecolor{currentstroke}%
\pgfsetdash{}{0pt}%
\pgfpathmoveto{\pgfqpoint{1.276507in}{0.931048in}}%
\pgfpathcurveto{\pgfqpoint{1.284743in}{0.931048in}}{\pgfqpoint{1.292643in}{0.934320in}}{\pgfqpoint{1.298467in}{0.940144in}}%
\pgfpathcurveto{\pgfqpoint{1.304291in}{0.945968in}}{\pgfqpoint{1.307564in}{0.953868in}}{\pgfqpoint{1.307564in}{0.962104in}}%
\pgfpathcurveto{\pgfqpoint{1.307564in}{0.970340in}}{\pgfqpoint{1.304291in}{0.978240in}}{\pgfqpoint{1.298467in}{0.984064in}}%
\pgfpathcurveto{\pgfqpoint{1.292643in}{0.989888in}}{\pgfqpoint{1.284743in}{0.993161in}}{\pgfqpoint{1.276507in}{0.993161in}}%
\pgfpathcurveto{\pgfqpoint{1.268271in}{0.993161in}}{\pgfqpoint{1.260371in}{0.989888in}}{\pgfqpoint{1.254547in}{0.984064in}}%
\pgfpathcurveto{\pgfqpoint{1.248723in}{0.978240in}}{\pgfqpoint{1.245451in}{0.970340in}}{\pgfqpoint{1.245451in}{0.962104in}}%
\pgfpathcurveto{\pgfqpoint{1.245451in}{0.953868in}}{\pgfqpoint{1.248723in}{0.945968in}}{\pgfqpoint{1.254547in}{0.940144in}}%
\pgfpathcurveto{\pgfqpoint{1.260371in}{0.934320in}}{\pgfqpoint{1.268271in}{0.931048in}}{\pgfqpoint{1.276507in}{0.931048in}}%
\pgfpathclose%
\pgfusepath{stroke,fill}%
\end{pgfscope}%
\begin{pgfscope}%
\pgfpathrectangle{\pgfqpoint{0.556847in}{0.516222in}}{\pgfqpoint{1.722590in}{1.783528in}} %
\pgfusepath{clip}%
\pgfsetbuttcap%
\pgfsetroundjoin%
\definecolor{currentfill}{rgb}{0.298039,0.447059,0.690196}%
\pgfsetfillcolor{currentfill}%
\pgfsetlinewidth{0.240900pt}%
\definecolor{currentstroke}{rgb}{1.000000,1.000000,1.000000}%
\pgfsetstrokecolor{currentstroke}%
\pgfsetdash{}{0pt}%
\pgfpathmoveto{\pgfqpoint{1.513842in}{2.090341in}}%
\pgfpathcurveto{\pgfqpoint{1.522078in}{2.090341in}}{\pgfqpoint{1.529978in}{2.093613in}}{\pgfqpoint{1.535802in}{2.099437in}}%
\pgfpathcurveto{\pgfqpoint{1.541626in}{2.105261in}}{\pgfqpoint{1.544898in}{2.113161in}}{\pgfqpoint{1.544898in}{2.121397in}}%
\pgfpathcurveto{\pgfqpoint{1.544898in}{2.129634in}}{\pgfqpoint{1.541626in}{2.137534in}}{\pgfqpoint{1.535802in}{2.143357in}}%
\pgfpathcurveto{\pgfqpoint{1.529978in}{2.149181in}}{\pgfqpoint{1.522078in}{2.152454in}}{\pgfqpoint{1.513842in}{2.152454in}}%
\pgfpathcurveto{\pgfqpoint{1.505606in}{2.152454in}}{\pgfqpoint{1.497705in}{2.149181in}}{\pgfqpoint{1.491882in}{2.143357in}}%
\pgfpathcurveto{\pgfqpoint{1.486058in}{2.137534in}}{\pgfqpoint{1.482785in}{2.129634in}}{\pgfqpoint{1.482785in}{2.121397in}}%
\pgfpathcurveto{\pgfqpoint{1.482785in}{2.113161in}}{\pgfqpoint{1.486058in}{2.105261in}}{\pgfqpoint{1.491882in}{2.099437in}}%
\pgfpathcurveto{\pgfqpoint{1.497705in}{2.093613in}}{\pgfqpoint{1.505606in}{2.090341in}}{\pgfqpoint{1.513842in}{2.090341in}}%
\pgfpathclose%
\pgfusepath{stroke,fill}%
\end{pgfscope}%
\begin{pgfscope}%
\pgfpathrectangle{\pgfqpoint{0.556847in}{0.516222in}}{\pgfqpoint{1.722590in}{1.783528in}} %
\pgfusepath{clip}%
\pgfsetbuttcap%
\pgfsetroundjoin%
\definecolor{currentfill}{rgb}{0.298039,0.447059,0.690196}%
\pgfsetfillcolor{currentfill}%
\pgfsetlinewidth{0.240900pt}%
\definecolor{currentstroke}{rgb}{1.000000,1.000000,1.000000}%
\pgfsetstrokecolor{currentstroke}%
\pgfsetdash{}{0pt}%
\pgfpathmoveto{\pgfqpoint{1.142528in}{0.810660in}}%
\pgfpathcurveto{\pgfqpoint{1.150764in}{0.810660in}}{\pgfqpoint{1.158664in}{0.813932in}}{\pgfqpoint{1.164488in}{0.819756in}}%
\pgfpathcurveto{\pgfqpoint{1.170312in}{0.825580in}}{\pgfqpoint{1.173584in}{0.833480in}}{\pgfqpoint{1.173584in}{0.841716in}}%
\pgfpathcurveto{\pgfqpoint{1.173584in}{0.849952in}}{\pgfqpoint{1.170312in}{0.857852in}}{\pgfqpoint{1.164488in}{0.863676in}}%
\pgfpathcurveto{\pgfqpoint{1.158664in}{0.869500in}}{\pgfqpoint{1.150764in}{0.872773in}}{\pgfqpoint{1.142528in}{0.872773in}}%
\pgfpathcurveto{\pgfqpoint{1.134292in}{0.872773in}}{\pgfqpoint{1.126392in}{0.869500in}}{\pgfqpoint{1.120568in}{0.863676in}}%
\pgfpathcurveto{\pgfqpoint{1.114744in}{0.857852in}}{\pgfqpoint{1.111471in}{0.849952in}}{\pgfqpoint{1.111471in}{0.841716in}}%
\pgfpathcurveto{\pgfqpoint{1.111471in}{0.833480in}}{\pgfqpoint{1.114744in}{0.825580in}}{\pgfqpoint{1.120568in}{0.819756in}}%
\pgfpathcurveto{\pgfqpoint{1.126392in}{0.813932in}}{\pgfqpoint{1.134292in}{0.810660in}}{\pgfqpoint{1.142528in}{0.810660in}}%
\pgfpathclose%
\pgfusepath{stroke,fill}%
\end{pgfscope}%
\begin{pgfscope}%
\pgfpathrectangle{\pgfqpoint{0.556847in}{0.516222in}}{\pgfqpoint{1.722590in}{1.783528in}} %
\pgfusepath{clip}%
\pgfsetbuttcap%
\pgfsetroundjoin%
\definecolor{currentfill}{rgb}{0.298039,0.447059,0.690196}%
\pgfsetfillcolor{currentfill}%
\pgfsetlinewidth{0.240900pt}%
\definecolor{currentstroke}{rgb}{1.000000,1.000000,1.000000}%
\pgfsetstrokecolor{currentstroke}%
\pgfsetdash{}{0pt}%
\pgfpathmoveto{\pgfqpoint{1.900468in}{1.265459in}}%
\pgfpathcurveto{\pgfqpoint{1.908704in}{1.265459in}}{\pgfqpoint{1.916604in}{1.268731in}}{\pgfqpoint{1.922428in}{1.274555in}}%
\pgfpathcurveto{\pgfqpoint{1.928252in}{1.280379in}}{\pgfqpoint{1.931524in}{1.288279in}}{\pgfqpoint{1.931524in}{1.296516in}}%
\pgfpathcurveto{\pgfqpoint{1.931524in}{1.304752in}}{\pgfqpoint{1.928252in}{1.312652in}}{\pgfqpoint{1.922428in}{1.318476in}}%
\pgfpathcurveto{\pgfqpoint{1.916604in}{1.324300in}}{\pgfqpoint{1.908704in}{1.327572in}}{\pgfqpoint{1.900468in}{1.327572in}}%
\pgfpathcurveto{\pgfqpoint{1.892231in}{1.327572in}}{\pgfqpoint{1.884331in}{1.324300in}}{\pgfqpoint{1.878507in}{1.318476in}}%
\pgfpathcurveto{\pgfqpoint{1.872683in}{1.312652in}}{\pgfqpoint{1.869411in}{1.304752in}}{\pgfqpoint{1.869411in}{1.296516in}}%
\pgfpathcurveto{\pgfqpoint{1.869411in}{1.288279in}}{\pgfqpoint{1.872683in}{1.280379in}}{\pgfqpoint{1.878507in}{1.274555in}}%
\pgfpathcurveto{\pgfqpoint{1.884331in}{1.268731in}}{\pgfqpoint{1.892231in}{1.265459in}}{\pgfqpoint{1.900468in}{1.265459in}}%
\pgfpathclose%
\pgfusepath{stroke,fill}%
\end{pgfscope}%
\begin{pgfscope}%
\pgfpathrectangle{\pgfqpoint{0.556847in}{0.516222in}}{\pgfqpoint{1.722590in}{1.783528in}} %
\pgfusepath{clip}%
\pgfsetbuttcap%
\pgfsetroundjoin%
\definecolor{currentfill}{rgb}{0.298039,0.447059,0.690196}%
\pgfsetfillcolor{currentfill}%
\pgfsetlinewidth{0.240900pt}%
\definecolor{currentstroke}{rgb}{1.000000,1.000000,1.000000}%
\pgfsetstrokecolor{currentstroke}%
\pgfsetdash{}{0pt}%
\pgfpathmoveto{\pgfqpoint{1.525326in}{1.211953in}}%
\pgfpathcurveto{\pgfqpoint{1.533562in}{1.211953in}}{\pgfqpoint{1.541462in}{1.215226in}}{\pgfqpoint{1.547286in}{1.221050in}}%
\pgfpathcurveto{\pgfqpoint{1.553110in}{1.226873in}}{\pgfqpoint{1.556382in}{1.234774in}}{\pgfqpoint{1.556382in}{1.243010in}}%
\pgfpathcurveto{\pgfqpoint{1.556382in}{1.251246in}}{\pgfqpoint{1.553110in}{1.259146in}}{\pgfqpoint{1.547286in}{1.264970in}}%
\pgfpathcurveto{\pgfqpoint{1.541462in}{1.270794in}}{\pgfqpoint{1.533562in}{1.274066in}}{\pgfqpoint{1.525326in}{1.274066in}}%
\pgfpathcurveto{\pgfqpoint{1.517089in}{1.274066in}}{\pgfqpoint{1.509189in}{1.270794in}}{\pgfqpoint{1.503365in}{1.264970in}}%
\pgfpathcurveto{\pgfqpoint{1.497542in}{1.259146in}}{\pgfqpoint{1.494269in}{1.251246in}}{\pgfqpoint{1.494269in}{1.243010in}}%
\pgfpathcurveto{\pgfqpoint{1.494269in}{1.234774in}}{\pgfqpoint{1.497542in}{1.226873in}}{\pgfqpoint{1.503365in}{1.221050in}}%
\pgfpathcurveto{\pgfqpoint{1.509189in}{1.215226in}}{\pgfqpoint{1.517089in}{1.211953in}}{\pgfqpoint{1.525326in}{1.211953in}}%
\pgfpathclose%
\pgfusepath{stroke,fill}%
\end{pgfscope}%
\begin{pgfscope}%
\pgfpathrectangle{\pgfqpoint{0.556847in}{0.516222in}}{\pgfqpoint{1.722590in}{1.783528in}} %
\pgfusepath{clip}%
\pgfsetbuttcap%
\pgfsetroundjoin%
\definecolor{currentfill}{rgb}{0.298039,0.447059,0.690196}%
\pgfsetfillcolor{currentfill}%
\pgfsetlinewidth{0.240900pt}%
\definecolor{currentstroke}{rgb}{1.000000,1.000000,1.000000}%
\pgfsetstrokecolor{currentstroke}%
\pgfsetdash{}{0pt}%
\pgfpathmoveto{\pgfqpoint{1.785628in}{1.559741in}}%
\pgfpathcurveto{\pgfqpoint{1.793865in}{1.559741in}}{\pgfqpoint{1.801765in}{1.563014in}}{\pgfqpoint{1.807589in}{1.568837in}}%
\pgfpathcurveto{\pgfqpoint{1.813412in}{1.574661in}}{\pgfqpoint{1.816685in}{1.582561in}}{\pgfqpoint{1.816685in}{1.590798in}}%
\pgfpathcurveto{\pgfqpoint{1.816685in}{1.599034in}}{\pgfqpoint{1.813412in}{1.606934in}}{\pgfqpoint{1.807589in}{1.612758in}}%
\pgfpathcurveto{\pgfqpoint{1.801765in}{1.618582in}}{\pgfqpoint{1.793865in}{1.621854in}}{\pgfqpoint{1.785628in}{1.621854in}}%
\pgfpathcurveto{\pgfqpoint{1.777392in}{1.621854in}}{\pgfqpoint{1.769492in}{1.618582in}}{\pgfqpoint{1.763668in}{1.612758in}}%
\pgfpathcurveto{\pgfqpoint{1.757844in}{1.606934in}}{\pgfqpoint{1.754572in}{1.599034in}}{\pgfqpoint{1.754572in}{1.590798in}}%
\pgfpathcurveto{\pgfqpoint{1.754572in}{1.582561in}}{\pgfqpoint{1.757844in}{1.574661in}}{\pgfqpoint{1.763668in}{1.568837in}}%
\pgfpathcurveto{\pgfqpoint{1.769492in}{1.563014in}}{\pgfqpoint{1.777392in}{1.559741in}}{\pgfqpoint{1.785628in}{1.559741in}}%
\pgfpathclose%
\pgfusepath{stroke,fill}%
\end{pgfscope}%
\begin{pgfscope}%
\pgfpathrectangle{\pgfqpoint{0.556847in}{0.516222in}}{\pgfqpoint{1.722590in}{1.783528in}} %
\pgfusepath{clip}%
\pgfsetbuttcap%
\pgfsetroundjoin%
\definecolor{currentfill}{rgb}{0.298039,0.447059,0.690196}%
\pgfsetfillcolor{currentfill}%
\pgfsetlinewidth{0.240900pt}%
\definecolor{currentstroke}{rgb}{1.000000,1.000000,1.000000}%
\pgfsetstrokecolor{currentstroke}%
\pgfsetdash{}{0pt}%
\pgfpathmoveto{\pgfqpoint{2.091867in}{0.939965in}}%
\pgfpathcurveto{\pgfqpoint{2.100103in}{0.939965in}}{\pgfqpoint{2.108003in}{0.943238in}}{\pgfqpoint{2.113827in}{0.949062in}}%
\pgfpathcurveto{\pgfqpoint{2.119651in}{0.954885in}}{\pgfqpoint{2.122923in}{0.962786in}}{\pgfqpoint{2.122923in}{0.971022in}}%
\pgfpathcurveto{\pgfqpoint{2.122923in}{0.979258in}}{\pgfqpoint{2.119651in}{0.987158in}}{\pgfqpoint{2.113827in}{0.992982in}}%
\pgfpathcurveto{\pgfqpoint{2.108003in}{0.998806in}}{\pgfqpoint{2.100103in}{1.002078in}}{\pgfqpoint{2.091867in}{1.002078in}}%
\pgfpathcurveto{\pgfqpoint{2.083630in}{1.002078in}}{\pgfqpoint{2.075730in}{0.998806in}}{\pgfqpoint{2.069906in}{0.992982in}}%
\pgfpathcurveto{\pgfqpoint{2.064082in}{0.987158in}}{\pgfqpoint{2.060810in}{0.979258in}}{\pgfqpoint{2.060810in}{0.971022in}}%
\pgfpathcurveto{\pgfqpoint{2.060810in}{0.962786in}}{\pgfqpoint{2.064082in}{0.954885in}}{\pgfqpoint{2.069906in}{0.949062in}}%
\pgfpathcurveto{\pgfqpoint{2.075730in}{0.943238in}}{\pgfqpoint{2.083630in}{0.939965in}}{\pgfqpoint{2.091867in}{0.939965in}}%
\pgfpathclose%
\pgfusepath{stroke,fill}%
\end{pgfscope}%
\begin{pgfscope}%
\pgfpathrectangle{\pgfqpoint{0.556847in}{0.516222in}}{\pgfqpoint{1.722590in}{1.783528in}} %
\pgfusepath{clip}%
\pgfsetbuttcap%
\pgfsetroundjoin%
\definecolor{currentfill}{rgb}{0.298039,0.447059,0.690196}%
\pgfsetfillcolor{currentfill}%
\pgfsetlinewidth{0.240900pt}%
\definecolor{currentstroke}{rgb}{1.000000,1.000000,1.000000}%
\pgfsetstrokecolor{currentstroke}%
\pgfsetdash{}{0pt}%
\pgfpathmoveto{\pgfqpoint{1.131044in}{1.158447in}}%
\pgfpathcurveto{\pgfqpoint{1.139280in}{1.158447in}}{\pgfqpoint{1.147180in}{1.161720in}}{\pgfqpoint{1.153004in}{1.167544in}}%
\pgfpathcurveto{\pgfqpoint{1.158828in}{1.173368in}}{\pgfqpoint{1.162100in}{1.181268in}}{\pgfqpoint{1.162100in}{1.189504in}}%
\pgfpathcurveto{\pgfqpoint{1.162100in}{1.197740in}}{\pgfqpoint{1.158828in}{1.205640in}}{\pgfqpoint{1.153004in}{1.211464in}}%
\pgfpathcurveto{\pgfqpoint{1.147180in}{1.217288in}}{\pgfqpoint{1.139280in}{1.220560in}}{\pgfqpoint{1.131044in}{1.220560in}}%
\pgfpathcurveto{\pgfqpoint{1.122808in}{1.220560in}}{\pgfqpoint{1.114908in}{1.217288in}}{\pgfqpoint{1.109084in}{1.211464in}}%
\pgfpathcurveto{\pgfqpoint{1.103260in}{1.205640in}}{\pgfqpoint{1.099987in}{1.197740in}}{\pgfqpoint{1.099987in}{1.189504in}}%
\pgfpathcurveto{\pgfqpoint{1.099987in}{1.181268in}}{\pgfqpoint{1.103260in}{1.173368in}}{\pgfqpoint{1.109084in}{1.167544in}}%
\pgfpathcurveto{\pgfqpoint{1.114908in}{1.161720in}}{\pgfqpoint{1.122808in}{1.158447in}}{\pgfqpoint{1.131044in}{1.158447in}}%
\pgfpathclose%
\pgfusepath{stroke,fill}%
\end{pgfscope}%
\begin{pgfscope}%
\pgfpathrectangle{\pgfqpoint{0.556847in}{0.516222in}}{\pgfqpoint{1.722590in}{1.783528in}} %
\pgfusepath{clip}%
\pgfsetbuttcap%
\pgfsetroundjoin%
\definecolor{currentfill}{rgb}{0.298039,0.447059,0.690196}%
\pgfsetfillcolor{currentfill}%
\pgfsetlinewidth{0.240900pt}%
\definecolor{currentstroke}{rgb}{1.000000,1.000000,1.000000}%
\pgfsetstrokecolor{currentstroke}%
\pgfsetdash{}{0pt}%
\pgfpathmoveto{\pgfqpoint{1.885156in}{1.265459in}}%
\pgfpathcurveto{\pgfqpoint{1.893392in}{1.265459in}}{\pgfqpoint{1.901292in}{1.268731in}}{\pgfqpoint{1.907116in}{1.274555in}}%
\pgfpathcurveto{\pgfqpoint{1.912940in}{1.280379in}}{\pgfqpoint{1.916212in}{1.288279in}}{\pgfqpoint{1.916212in}{1.296516in}}%
\pgfpathcurveto{\pgfqpoint{1.916212in}{1.304752in}}{\pgfqpoint{1.912940in}{1.312652in}}{\pgfqpoint{1.907116in}{1.318476in}}%
\pgfpathcurveto{\pgfqpoint{1.901292in}{1.324300in}}{\pgfqpoint{1.893392in}{1.327572in}}{\pgfqpoint{1.885156in}{1.327572in}}%
\pgfpathcurveto{\pgfqpoint{1.876919in}{1.327572in}}{\pgfqpoint{1.869019in}{1.324300in}}{\pgfqpoint{1.863195in}{1.318476in}}%
\pgfpathcurveto{\pgfqpoint{1.857372in}{1.312652in}}{\pgfqpoint{1.854099in}{1.304752in}}{\pgfqpoint{1.854099in}{1.296516in}}%
\pgfpathcurveto{\pgfqpoint{1.854099in}{1.288279in}}{\pgfqpoint{1.857372in}{1.280379in}}{\pgfqpoint{1.863195in}{1.274555in}}%
\pgfpathcurveto{\pgfqpoint{1.869019in}{1.268731in}}{\pgfqpoint{1.876919in}{1.265459in}}{\pgfqpoint{1.885156in}{1.265459in}}%
\pgfpathclose%
\pgfusepath{stroke,fill}%
\end{pgfscope}%
\begin{pgfscope}%
\pgfpathrectangle{\pgfqpoint{0.556847in}{0.516222in}}{\pgfqpoint{1.722590in}{1.783528in}} %
\pgfusepath{clip}%
\pgfsetbuttcap%
\pgfsetroundjoin%
\definecolor{currentfill}{rgb}{0.298039,0.447059,0.690196}%
\pgfsetfillcolor{currentfill}%
\pgfsetlinewidth{0.240900pt}%
\definecolor{currentstroke}{rgb}{1.000000,1.000000,1.000000}%
\pgfsetstrokecolor{currentstroke}%
\pgfsetdash{}{0pt}%
\pgfpathmoveto{\pgfqpoint{1.670789in}{1.920906in}}%
\pgfpathcurveto{\pgfqpoint{1.679025in}{1.920906in}}{\pgfqpoint{1.686925in}{1.924178in}}{\pgfqpoint{1.692749in}{1.930002in}}%
\pgfpathcurveto{\pgfqpoint{1.698573in}{1.935826in}}{\pgfqpoint{1.701845in}{1.943726in}}{\pgfqpoint{1.701845in}{1.951962in}}%
\pgfpathcurveto{\pgfqpoint{1.701845in}{1.960198in}}{\pgfqpoint{1.698573in}{1.968098in}}{\pgfqpoint{1.692749in}{1.973922in}}%
\pgfpathcurveto{\pgfqpoint{1.686925in}{1.979746in}}{\pgfqpoint{1.679025in}{1.983019in}}{\pgfqpoint{1.670789in}{1.983019in}}%
\pgfpathcurveto{\pgfqpoint{1.662553in}{1.983019in}}{\pgfqpoint{1.654653in}{1.979746in}}{\pgfqpoint{1.648829in}{1.973922in}}%
\pgfpathcurveto{\pgfqpoint{1.643005in}{1.968098in}}{\pgfqpoint{1.639732in}{1.960198in}}{\pgfqpoint{1.639732in}{1.951962in}}%
\pgfpathcurveto{\pgfqpoint{1.639732in}{1.943726in}}{\pgfqpoint{1.643005in}{1.935826in}}{\pgfqpoint{1.648829in}{1.930002in}}%
\pgfpathcurveto{\pgfqpoint{1.654653in}{1.924178in}}{\pgfqpoint{1.662553in}{1.920906in}}{\pgfqpoint{1.670789in}{1.920906in}}%
\pgfpathclose%
\pgfusepath{stroke,fill}%
\end{pgfscope}%
\begin{pgfscope}%
\pgfpathrectangle{\pgfqpoint{0.556847in}{0.516222in}}{\pgfqpoint{1.722590in}{1.783528in}} %
\pgfusepath{clip}%
\pgfsetbuttcap%
\pgfsetroundjoin%
\definecolor{currentfill}{rgb}{0.298039,0.447059,0.690196}%
\pgfsetfillcolor{currentfill}%
\pgfsetlinewidth{0.240900pt}%
\definecolor{currentstroke}{rgb}{1.000000,1.000000,1.000000}%
\pgfsetstrokecolor{currentstroke}%
\pgfsetdash{}{0pt}%
\pgfpathmoveto{\pgfqpoint{1.303303in}{1.051436in}}%
\pgfpathcurveto{\pgfqpoint{1.311539in}{1.051436in}}{\pgfqpoint{1.319439in}{1.054708in}}{\pgfqpoint{1.325263in}{1.060532in}}%
\pgfpathcurveto{\pgfqpoint{1.331087in}{1.066356in}}{\pgfqpoint{1.334360in}{1.074256in}}{\pgfqpoint{1.334360in}{1.082492in}}%
\pgfpathcurveto{\pgfqpoint{1.334360in}{1.090729in}}{\pgfqpoint{1.331087in}{1.098629in}}{\pgfqpoint{1.325263in}{1.104453in}}%
\pgfpathcurveto{\pgfqpoint{1.319439in}{1.110276in}}{\pgfqpoint{1.311539in}{1.113549in}}{\pgfqpoint{1.303303in}{1.113549in}}%
\pgfpathcurveto{\pgfqpoint{1.295067in}{1.113549in}}{\pgfqpoint{1.287167in}{1.110276in}}{\pgfqpoint{1.281343in}{1.104453in}}%
\pgfpathcurveto{\pgfqpoint{1.275519in}{1.098629in}}{\pgfqpoint{1.272247in}{1.090729in}}{\pgfqpoint{1.272247in}{1.082492in}}%
\pgfpathcurveto{\pgfqpoint{1.272247in}{1.074256in}}{\pgfqpoint{1.275519in}{1.066356in}}{\pgfqpoint{1.281343in}{1.060532in}}%
\pgfpathcurveto{\pgfqpoint{1.287167in}{1.054708in}}{\pgfqpoint{1.295067in}{1.051436in}}{\pgfqpoint{1.303303in}{1.051436in}}%
\pgfpathclose%
\pgfusepath{stroke,fill}%
\end{pgfscope}%
\begin{pgfscope}%
\pgfpathrectangle{\pgfqpoint{0.556847in}{0.516222in}}{\pgfqpoint{1.722590in}{1.783528in}} %
\pgfusepath{clip}%
\pgfsetbuttcap%
\pgfsetroundjoin%
\definecolor{currentfill}{rgb}{0.298039,0.447059,0.690196}%
\pgfsetfillcolor{currentfill}%
\pgfsetlinewidth{0.240900pt}%
\definecolor{currentstroke}{rgb}{1.000000,1.000000,1.000000}%
\pgfsetstrokecolor{currentstroke}%
\pgfsetdash{}{0pt}%
\pgfpathmoveto{\pgfqpoint{1.552122in}{1.345718in}}%
\pgfpathcurveto{\pgfqpoint{1.560358in}{1.345718in}}{\pgfqpoint{1.568258in}{1.348990in}}{\pgfqpoint{1.574082in}{1.354814in}}%
\pgfpathcurveto{\pgfqpoint{1.579906in}{1.360638in}}{\pgfqpoint{1.583178in}{1.368538in}}{\pgfqpoint{1.583178in}{1.376774in}}%
\pgfpathcurveto{\pgfqpoint{1.583178in}{1.385011in}}{\pgfqpoint{1.579906in}{1.392911in}}{\pgfqpoint{1.574082in}{1.398735in}}%
\pgfpathcurveto{\pgfqpoint{1.568258in}{1.404559in}}{\pgfqpoint{1.560358in}{1.407831in}}{\pgfqpoint{1.552122in}{1.407831in}}%
\pgfpathcurveto{\pgfqpoint{1.543885in}{1.407831in}}{\pgfqpoint{1.535985in}{1.404559in}}{\pgfqpoint{1.530161in}{1.398735in}}%
\pgfpathcurveto{\pgfqpoint{1.524337in}{1.392911in}}{\pgfqpoint{1.521065in}{1.385011in}}{\pgfqpoint{1.521065in}{1.376774in}}%
\pgfpathcurveto{\pgfqpoint{1.521065in}{1.368538in}}{\pgfqpoint{1.524337in}{1.360638in}}{\pgfqpoint{1.530161in}{1.354814in}}%
\pgfpathcurveto{\pgfqpoint{1.535985in}{1.348990in}}{\pgfqpoint{1.543885in}{1.345718in}}{\pgfqpoint{1.552122in}{1.345718in}}%
\pgfpathclose%
\pgfusepath{stroke,fill}%
\end{pgfscope}%
\begin{pgfscope}%
\pgfsetrectcap%
\pgfsetmiterjoin%
\pgfsetlinewidth{0.000000pt}%
\definecolor{currentstroke}{rgb}{1.000000,1.000000,1.000000}%
\pgfsetstrokecolor{currentstroke}%
\pgfsetdash{}{0pt}%
\pgfpathmoveto{\pgfqpoint{0.556847in}{0.516222in}}%
\pgfpathlineto{\pgfqpoint{0.556847in}{2.299750in}}%
\pgfusepath{}%
\end{pgfscope}%
\begin{pgfscope}%
\pgfsetrectcap%
\pgfsetmiterjoin%
\pgfsetlinewidth{0.000000pt}%
\definecolor{currentstroke}{rgb}{1.000000,1.000000,1.000000}%
\pgfsetstrokecolor{currentstroke}%
\pgfsetdash{}{0pt}%
\pgfpathmoveto{\pgfqpoint{0.556847in}{0.516222in}}%
\pgfpathlineto{\pgfqpoint{2.279437in}{0.516222in}}%
\pgfusepath{}%
\end{pgfscope}%
\end{pgfpicture}%
\makeatother%
\endgroup%

		\caption{Comparison between the two times from different throws.}
		\label{fig_EX1_EX2}
	\end{subfigure}
	\caption{Plots of time measurments made by two observers(obs)}
\end{figure}

\subsection{Fitting a GP model with average data}

Another Assumption made to explain the $Q2$ value was the simplicity of the model, initially a GP with a Matern 32 kernel was used. Thus, after modifying the time variable to be the average of the four measurements and adding some other 2 RBF kernels the $Q2$ measure increased in 0.4 giving exactly a value of $0.58$ aproximately. The $Q2$ measure was enough for us to follow with the optimization stage. With the new data was not neccesary to modify the leave one out function to delete the repeated point with the other time observation.

\begin{figure}
	\begin{subfigure}[h]{.5\linewidth}
		%% Creator: Matplotlib, PGF backend
%%
%% To include the figure in your LaTeX document, write
%%   \input{<filename>.pgf}
%%
%% Make sure the required packages are loaded in your preamble
%%   \usepackage{pgf}
%%
%% Figures using additional raster images can only be included by \input if
%% they are in the same directory as the main LaTeX file. For loading figures
%% from other directories you can use the `import` package
%%   \usepackage{import}
%% and then include the figures with
%%   \import{<path to file>}{<filename>.pgf}
%%
%% Matplotlib used the following preamble
%%   \usepackage[utf8x]{inputenc}
%%   \usepackage[T1]{fontenc}
%%   \usepackage{cmbright}
%%
\begingroup%
\makeatletter%
\begin{pgfpicture}%
\pgfpathrectangle{\pgfpointorigin}{\pgfqpoint{2.500000in}{2.500000in}}%
\pgfusepath{use as bounding box, clip}%
\begin{pgfscope}%
\pgfsetbuttcap%
\pgfsetmiterjoin%
\definecolor{currentfill}{rgb}{1.000000,1.000000,1.000000}%
\pgfsetfillcolor{currentfill}%
\pgfsetlinewidth{0.000000pt}%
\definecolor{currentstroke}{rgb}{1.000000,1.000000,1.000000}%
\pgfsetstrokecolor{currentstroke}%
\pgfsetdash{}{0pt}%
\pgfpathmoveto{\pgfqpoint{0.000000in}{0.000000in}}%
\pgfpathlineto{\pgfqpoint{2.500000in}{0.000000in}}%
\pgfpathlineto{\pgfqpoint{2.500000in}{2.500000in}}%
\pgfpathlineto{\pgfqpoint{0.000000in}{2.500000in}}%
\pgfpathclose%
\pgfusepath{fill}%
\end{pgfscope}%
\begin{pgfscope}%
\pgfsetbuttcap%
\pgfsetmiterjoin%
\definecolor{currentfill}{rgb}{0.917647,0.917647,0.949020}%
\pgfsetfillcolor{currentfill}%
\pgfsetlinewidth{0.000000pt}%
\definecolor{currentstroke}{rgb}{0.000000,0.000000,0.000000}%
\pgfsetstrokecolor{currentstroke}%
\pgfsetstrokeopacity{0.000000}%
\pgfsetdash{}{0pt}%
\pgfpathmoveto{\pgfqpoint{0.548058in}{0.516222in}}%
\pgfpathlineto{\pgfqpoint{2.287641in}{0.516222in}}%
\pgfpathlineto{\pgfqpoint{2.287641in}{2.299750in}}%
\pgfpathlineto{\pgfqpoint{0.548058in}{2.299750in}}%
\pgfpathclose%
\pgfusepath{fill}%
\end{pgfscope}%
\begin{pgfscope}%
\pgfpathrectangle{\pgfqpoint{0.548058in}{0.516222in}}{\pgfqpoint{1.739582in}{1.783528in}} %
\pgfusepath{clip}%
\pgfsetroundcap%
\pgfsetroundjoin%
\pgfsetlinewidth{0.803000pt}%
\definecolor{currentstroke}{rgb}{1.000000,1.000000,1.000000}%
\pgfsetstrokecolor{currentstroke}%
\pgfsetdash{}{0pt}%
\pgfpathmoveto{\pgfqpoint{0.548058in}{0.516222in}}%
\pgfpathlineto{\pgfqpoint{0.548058in}{2.299750in}}%
\pgfusepath{stroke}%
\end{pgfscope}%
\begin{pgfscope}%
\pgfsetbuttcap%
\pgfsetroundjoin%
\definecolor{currentfill}{rgb}{0.150000,0.150000,0.150000}%
\pgfsetfillcolor{currentfill}%
\pgfsetlinewidth{0.803000pt}%
\definecolor{currentstroke}{rgb}{0.150000,0.150000,0.150000}%
\pgfsetstrokecolor{currentstroke}%
\pgfsetdash{}{0pt}%
\pgfsys@defobject{currentmarker}{\pgfqpoint{0.000000in}{0.000000in}}{\pgfqpoint{0.000000in}{0.000000in}}{%
\pgfpathmoveto{\pgfqpoint{0.000000in}{0.000000in}}%
\pgfpathlineto{\pgfqpoint{0.000000in}{0.000000in}}%
\pgfusepath{stroke,fill}%
}%
\begin{pgfscope}%
\pgfsys@transformshift{0.548058in}{0.516222in}%
\pgfsys@useobject{currentmarker}{}%
\end{pgfscope}%
\end{pgfscope}%
\begin{pgfscope}%
\definecolor{textcolor}{rgb}{0.150000,0.150000,0.150000}%
\pgfsetstrokecolor{textcolor}%
\pgfsetfillcolor{textcolor}%
\pgftext[x=0.548058in,y=0.438444in,,top]{\color{textcolor}\sffamily\fontsize{8.000000}{9.600000}\selectfont −0.2}%
\end{pgfscope}%
\begin{pgfscope}%
\pgfpathrectangle{\pgfqpoint{0.548058in}{0.516222in}}{\pgfqpoint{1.739582in}{1.783528in}} %
\pgfusepath{clip}%
\pgfsetroundcap%
\pgfsetroundjoin%
\pgfsetlinewidth{0.803000pt}%
\definecolor{currentstroke}{rgb}{1.000000,1.000000,1.000000}%
\pgfsetstrokecolor{currentstroke}%
\pgfsetdash{}{0pt}%
\pgfpathmoveto{\pgfqpoint{0.796570in}{0.516222in}}%
\pgfpathlineto{\pgfqpoint{0.796570in}{2.299750in}}%
\pgfusepath{stroke}%
\end{pgfscope}%
\begin{pgfscope}%
\pgfsetbuttcap%
\pgfsetroundjoin%
\definecolor{currentfill}{rgb}{0.150000,0.150000,0.150000}%
\pgfsetfillcolor{currentfill}%
\pgfsetlinewidth{0.803000pt}%
\definecolor{currentstroke}{rgb}{0.150000,0.150000,0.150000}%
\pgfsetstrokecolor{currentstroke}%
\pgfsetdash{}{0pt}%
\pgfsys@defobject{currentmarker}{\pgfqpoint{0.000000in}{0.000000in}}{\pgfqpoint{0.000000in}{0.000000in}}{%
\pgfpathmoveto{\pgfqpoint{0.000000in}{0.000000in}}%
\pgfpathlineto{\pgfqpoint{0.000000in}{0.000000in}}%
\pgfusepath{stroke,fill}%
}%
\begin{pgfscope}%
\pgfsys@transformshift{0.796570in}{0.516222in}%
\pgfsys@useobject{currentmarker}{}%
\end{pgfscope}%
\end{pgfscope}%
\begin{pgfscope}%
\definecolor{textcolor}{rgb}{0.150000,0.150000,0.150000}%
\pgfsetstrokecolor{textcolor}%
\pgfsetfillcolor{textcolor}%
\pgftext[x=0.796570in,y=0.438444in,,top]{\color{textcolor}\sffamily\fontsize{8.000000}{9.600000}\selectfont 0.0}%
\end{pgfscope}%
\begin{pgfscope}%
\pgfpathrectangle{\pgfqpoint{0.548058in}{0.516222in}}{\pgfqpoint{1.739582in}{1.783528in}} %
\pgfusepath{clip}%
\pgfsetroundcap%
\pgfsetroundjoin%
\pgfsetlinewidth{0.803000pt}%
\definecolor{currentstroke}{rgb}{1.000000,1.000000,1.000000}%
\pgfsetstrokecolor{currentstroke}%
\pgfsetdash{}{0pt}%
\pgfpathmoveto{\pgfqpoint{1.045082in}{0.516222in}}%
\pgfpathlineto{\pgfqpoint{1.045082in}{2.299750in}}%
\pgfusepath{stroke}%
\end{pgfscope}%
\begin{pgfscope}%
\pgfsetbuttcap%
\pgfsetroundjoin%
\definecolor{currentfill}{rgb}{0.150000,0.150000,0.150000}%
\pgfsetfillcolor{currentfill}%
\pgfsetlinewidth{0.803000pt}%
\definecolor{currentstroke}{rgb}{0.150000,0.150000,0.150000}%
\pgfsetstrokecolor{currentstroke}%
\pgfsetdash{}{0pt}%
\pgfsys@defobject{currentmarker}{\pgfqpoint{0.000000in}{0.000000in}}{\pgfqpoint{0.000000in}{0.000000in}}{%
\pgfpathmoveto{\pgfqpoint{0.000000in}{0.000000in}}%
\pgfpathlineto{\pgfqpoint{0.000000in}{0.000000in}}%
\pgfusepath{stroke,fill}%
}%
\begin{pgfscope}%
\pgfsys@transformshift{1.045082in}{0.516222in}%
\pgfsys@useobject{currentmarker}{}%
\end{pgfscope}%
\end{pgfscope}%
\begin{pgfscope}%
\definecolor{textcolor}{rgb}{0.150000,0.150000,0.150000}%
\pgfsetstrokecolor{textcolor}%
\pgfsetfillcolor{textcolor}%
\pgftext[x=1.045082in,y=0.438444in,,top]{\color{textcolor}\sffamily\fontsize{8.000000}{9.600000}\selectfont 0.2}%
\end{pgfscope}%
\begin{pgfscope}%
\pgfpathrectangle{\pgfqpoint{0.548058in}{0.516222in}}{\pgfqpoint{1.739582in}{1.783528in}} %
\pgfusepath{clip}%
\pgfsetroundcap%
\pgfsetroundjoin%
\pgfsetlinewidth{0.803000pt}%
\definecolor{currentstroke}{rgb}{1.000000,1.000000,1.000000}%
\pgfsetstrokecolor{currentstroke}%
\pgfsetdash{}{0pt}%
\pgfpathmoveto{\pgfqpoint{1.293594in}{0.516222in}}%
\pgfpathlineto{\pgfqpoint{1.293594in}{2.299750in}}%
\pgfusepath{stroke}%
\end{pgfscope}%
\begin{pgfscope}%
\pgfsetbuttcap%
\pgfsetroundjoin%
\definecolor{currentfill}{rgb}{0.150000,0.150000,0.150000}%
\pgfsetfillcolor{currentfill}%
\pgfsetlinewidth{0.803000pt}%
\definecolor{currentstroke}{rgb}{0.150000,0.150000,0.150000}%
\pgfsetstrokecolor{currentstroke}%
\pgfsetdash{}{0pt}%
\pgfsys@defobject{currentmarker}{\pgfqpoint{0.000000in}{0.000000in}}{\pgfqpoint{0.000000in}{0.000000in}}{%
\pgfpathmoveto{\pgfqpoint{0.000000in}{0.000000in}}%
\pgfpathlineto{\pgfqpoint{0.000000in}{0.000000in}}%
\pgfusepath{stroke,fill}%
}%
\begin{pgfscope}%
\pgfsys@transformshift{1.293594in}{0.516222in}%
\pgfsys@useobject{currentmarker}{}%
\end{pgfscope}%
\end{pgfscope}%
\begin{pgfscope}%
\definecolor{textcolor}{rgb}{0.150000,0.150000,0.150000}%
\pgfsetstrokecolor{textcolor}%
\pgfsetfillcolor{textcolor}%
\pgftext[x=1.293594in,y=0.438444in,,top]{\color{textcolor}\sffamily\fontsize{8.000000}{9.600000}\selectfont 0.4}%
\end{pgfscope}%
\begin{pgfscope}%
\pgfpathrectangle{\pgfqpoint{0.548058in}{0.516222in}}{\pgfqpoint{1.739582in}{1.783528in}} %
\pgfusepath{clip}%
\pgfsetroundcap%
\pgfsetroundjoin%
\pgfsetlinewidth{0.803000pt}%
\definecolor{currentstroke}{rgb}{1.000000,1.000000,1.000000}%
\pgfsetstrokecolor{currentstroke}%
\pgfsetdash{}{0pt}%
\pgfpathmoveto{\pgfqpoint{1.542105in}{0.516222in}}%
\pgfpathlineto{\pgfqpoint{1.542105in}{2.299750in}}%
\pgfusepath{stroke}%
\end{pgfscope}%
\begin{pgfscope}%
\pgfsetbuttcap%
\pgfsetroundjoin%
\definecolor{currentfill}{rgb}{0.150000,0.150000,0.150000}%
\pgfsetfillcolor{currentfill}%
\pgfsetlinewidth{0.803000pt}%
\definecolor{currentstroke}{rgb}{0.150000,0.150000,0.150000}%
\pgfsetstrokecolor{currentstroke}%
\pgfsetdash{}{0pt}%
\pgfsys@defobject{currentmarker}{\pgfqpoint{0.000000in}{0.000000in}}{\pgfqpoint{0.000000in}{0.000000in}}{%
\pgfpathmoveto{\pgfqpoint{0.000000in}{0.000000in}}%
\pgfpathlineto{\pgfqpoint{0.000000in}{0.000000in}}%
\pgfusepath{stroke,fill}%
}%
\begin{pgfscope}%
\pgfsys@transformshift{1.542105in}{0.516222in}%
\pgfsys@useobject{currentmarker}{}%
\end{pgfscope}%
\end{pgfscope}%
\begin{pgfscope}%
\definecolor{textcolor}{rgb}{0.150000,0.150000,0.150000}%
\pgfsetstrokecolor{textcolor}%
\pgfsetfillcolor{textcolor}%
\pgftext[x=1.542105in,y=0.438444in,,top]{\color{textcolor}\sffamily\fontsize{8.000000}{9.600000}\selectfont 0.6}%
\end{pgfscope}%
\begin{pgfscope}%
\pgfpathrectangle{\pgfqpoint{0.548058in}{0.516222in}}{\pgfqpoint{1.739582in}{1.783528in}} %
\pgfusepath{clip}%
\pgfsetroundcap%
\pgfsetroundjoin%
\pgfsetlinewidth{0.803000pt}%
\definecolor{currentstroke}{rgb}{1.000000,1.000000,1.000000}%
\pgfsetstrokecolor{currentstroke}%
\pgfsetdash{}{0pt}%
\pgfpathmoveto{\pgfqpoint{1.790617in}{0.516222in}}%
\pgfpathlineto{\pgfqpoint{1.790617in}{2.299750in}}%
\pgfusepath{stroke}%
\end{pgfscope}%
\begin{pgfscope}%
\pgfsetbuttcap%
\pgfsetroundjoin%
\definecolor{currentfill}{rgb}{0.150000,0.150000,0.150000}%
\pgfsetfillcolor{currentfill}%
\pgfsetlinewidth{0.803000pt}%
\definecolor{currentstroke}{rgb}{0.150000,0.150000,0.150000}%
\pgfsetstrokecolor{currentstroke}%
\pgfsetdash{}{0pt}%
\pgfsys@defobject{currentmarker}{\pgfqpoint{0.000000in}{0.000000in}}{\pgfqpoint{0.000000in}{0.000000in}}{%
\pgfpathmoveto{\pgfqpoint{0.000000in}{0.000000in}}%
\pgfpathlineto{\pgfqpoint{0.000000in}{0.000000in}}%
\pgfusepath{stroke,fill}%
}%
\begin{pgfscope}%
\pgfsys@transformshift{1.790617in}{0.516222in}%
\pgfsys@useobject{currentmarker}{}%
\end{pgfscope}%
\end{pgfscope}%
\begin{pgfscope}%
\definecolor{textcolor}{rgb}{0.150000,0.150000,0.150000}%
\pgfsetstrokecolor{textcolor}%
\pgfsetfillcolor{textcolor}%
\pgftext[x=1.790617in,y=0.438444in,,top]{\color{textcolor}\sffamily\fontsize{8.000000}{9.600000}\selectfont 0.8}%
\end{pgfscope}%
\begin{pgfscope}%
\pgfpathrectangle{\pgfqpoint{0.548058in}{0.516222in}}{\pgfqpoint{1.739582in}{1.783528in}} %
\pgfusepath{clip}%
\pgfsetroundcap%
\pgfsetroundjoin%
\pgfsetlinewidth{0.803000pt}%
\definecolor{currentstroke}{rgb}{1.000000,1.000000,1.000000}%
\pgfsetstrokecolor{currentstroke}%
\pgfsetdash{}{0pt}%
\pgfpathmoveto{\pgfqpoint{2.039129in}{0.516222in}}%
\pgfpathlineto{\pgfqpoint{2.039129in}{2.299750in}}%
\pgfusepath{stroke}%
\end{pgfscope}%
\begin{pgfscope}%
\pgfsetbuttcap%
\pgfsetroundjoin%
\definecolor{currentfill}{rgb}{0.150000,0.150000,0.150000}%
\pgfsetfillcolor{currentfill}%
\pgfsetlinewidth{0.803000pt}%
\definecolor{currentstroke}{rgb}{0.150000,0.150000,0.150000}%
\pgfsetstrokecolor{currentstroke}%
\pgfsetdash{}{0pt}%
\pgfsys@defobject{currentmarker}{\pgfqpoint{0.000000in}{0.000000in}}{\pgfqpoint{0.000000in}{0.000000in}}{%
\pgfpathmoveto{\pgfqpoint{0.000000in}{0.000000in}}%
\pgfpathlineto{\pgfqpoint{0.000000in}{0.000000in}}%
\pgfusepath{stroke,fill}%
}%
\begin{pgfscope}%
\pgfsys@transformshift{2.039129in}{0.516222in}%
\pgfsys@useobject{currentmarker}{}%
\end{pgfscope}%
\end{pgfscope}%
\begin{pgfscope}%
\definecolor{textcolor}{rgb}{0.150000,0.150000,0.150000}%
\pgfsetstrokecolor{textcolor}%
\pgfsetfillcolor{textcolor}%
\pgftext[x=2.039129in,y=0.438444in,,top]{\color{textcolor}\sffamily\fontsize{8.000000}{9.600000}\selectfont 1.0}%
\end{pgfscope}%
\begin{pgfscope}%
\pgfpathrectangle{\pgfqpoint{0.548058in}{0.516222in}}{\pgfqpoint{1.739582in}{1.783528in}} %
\pgfusepath{clip}%
\pgfsetroundcap%
\pgfsetroundjoin%
\pgfsetlinewidth{0.803000pt}%
\definecolor{currentstroke}{rgb}{1.000000,1.000000,1.000000}%
\pgfsetstrokecolor{currentstroke}%
\pgfsetdash{}{0pt}%
\pgfpathmoveto{\pgfqpoint{2.287641in}{0.516222in}}%
\pgfpathlineto{\pgfqpoint{2.287641in}{2.299750in}}%
\pgfusepath{stroke}%
\end{pgfscope}%
\begin{pgfscope}%
\pgfsetbuttcap%
\pgfsetroundjoin%
\definecolor{currentfill}{rgb}{0.150000,0.150000,0.150000}%
\pgfsetfillcolor{currentfill}%
\pgfsetlinewidth{0.803000pt}%
\definecolor{currentstroke}{rgb}{0.150000,0.150000,0.150000}%
\pgfsetstrokecolor{currentstroke}%
\pgfsetdash{}{0pt}%
\pgfsys@defobject{currentmarker}{\pgfqpoint{0.000000in}{0.000000in}}{\pgfqpoint{0.000000in}{0.000000in}}{%
\pgfpathmoveto{\pgfqpoint{0.000000in}{0.000000in}}%
\pgfpathlineto{\pgfqpoint{0.000000in}{0.000000in}}%
\pgfusepath{stroke,fill}%
}%
\begin{pgfscope}%
\pgfsys@transformshift{2.287641in}{0.516222in}%
\pgfsys@useobject{currentmarker}{}%
\end{pgfscope}%
\end{pgfscope}%
\begin{pgfscope}%
\definecolor{textcolor}{rgb}{0.150000,0.150000,0.150000}%
\pgfsetstrokecolor{textcolor}%
\pgfsetfillcolor{textcolor}%
\pgftext[x=2.287641in,y=0.438444in,,top]{\color{textcolor}\sffamily\fontsize{8.000000}{9.600000}\selectfont 1.2}%
\end{pgfscope}%
\begin{pgfscope}%
\definecolor{textcolor}{rgb}{0.150000,0.150000,0.150000}%
\pgfsetstrokecolor{textcolor}%
\pgfsetfillcolor{textcolor}%
\pgftext[x=1.417849in,y=0.273321in,,top]{\color{textcolor}\sffamily\fontsize{8.800000}{10.560000}\selectfont wing tail ratio}%
\end{pgfscope}%
\begin{pgfscope}%
\pgfpathrectangle{\pgfqpoint{0.548058in}{0.516222in}}{\pgfqpoint{1.739582in}{1.783528in}} %
\pgfusepath{clip}%
\pgfsetroundcap%
\pgfsetroundjoin%
\pgfsetlinewidth{0.803000pt}%
\definecolor{currentstroke}{rgb}{1.000000,1.000000,1.000000}%
\pgfsetstrokecolor{currentstroke}%
\pgfsetdash{}{0pt}%
\pgfpathmoveto{\pgfqpoint{0.548058in}{0.516222in}}%
\pgfpathlineto{\pgfqpoint{2.287641in}{0.516222in}}%
\pgfusepath{stroke}%
\end{pgfscope}%
\begin{pgfscope}%
\pgfsetbuttcap%
\pgfsetroundjoin%
\definecolor{currentfill}{rgb}{0.150000,0.150000,0.150000}%
\pgfsetfillcolor{currentfill}%
\pgfsetlinewidth{0.803000pt}%
\definecolor{currentstroke}{rgb}{0.150000,0.150000,0.150000}%
\pgfsetstrokecolor{currentstroke}%
\pgfsetdash{}{0pt}%
\pgfsys@defobject{currentmarker}{\pgfqpoint{0.000000in}{0.000000in}}{\pgfqpoint{0.000000in}{0.000000in}}{%
\pgfpathmoveto{\pgfqpoint{0.000000in}{0.000000in}}%
\pgfpathlineto{\pgfqpoint{0.000000in}{0.000000in}}%
\pgfusepath{stroke,fill}%
}%
\begin{pgfscope}%
\pgfsys@transformshift{0.548058in}{0.516222in}%
\pgfsys@useobject{currentmarker}{}%
\end{pgfscope}%
\end{pgfscope}%
\begin{pgfscope}%
\definecolor{textcolor}{rgb}{0.150000,0.150000,0.150000}%
\pgfsetstrokecolor{textcolor}%
\pgfsetfillcolor{textcolor}%
\pgftext[x=0.470280in,y=0.516222in,right,]{\color{textcolor}\sffamily\fontsize{8.000000}{9.600000}\selectfont 2.5}%
\end{pgfscope}%
\begin{pgfscope}%
\pgfpathrectangle{\pgfqpoint{0.548058in}{0.516222in}}{\pgfqpoint{1.739582in}{1.783528in}} %
\pgfusepath{clip}%
\pgfsetroundcap%
\pgfsetroundjoin%
\pgfsetlinewidth{0.803000pt}%
\definecolor{currentstroke}{rgb}{1.000000,1.000000,1.000000}%
\pgfsetstrokecolor{currentstroke}%
\pgfsetdash{}{0pt}%
\pgfpathmoveto{\pgfqpoint{0.548058in}{0.739163in}}%
\pgfpathlineto{\pgfqpoint{2.287641in}{0.739163in}}%
\pgfusepath{stroke}%
\end{pgfscope}%
\begin{pgfscope}%
\pgfsetbuttcap%
\pgfsetroundjoin%
\definecolor{currentfill}{rgb}{0.150000,0.150000,0.150000}%
\pgfsetfillcolor{currentfill}%
\pgfsetlinewidth{0.803000pt}%
\definecolor{currentstroke}{rgb}{0.150000,0.150000,0.150000}%
\pgfsetstrokecolor{currentstroke}%
\pgfsetdash{}{0pt}%
\pgfsys@defobject{currentmarker}{\pgfqpoint{0.000000in}{0.000000in}}{\pgfqpoint{0.000000in}{0.000000in}}{%
\pgfpathmoveto{\pgfqpoint{0.000000in}{0.000000in}}%
\pgfpathlineto{\pgfqpoint{0.000000in}{0.000000in}}%
\pgfusepath{stroke,fill}%
}%
\begin{pgfscope}%
\pgfsys@transformshift{0.548058in}{0.739163in}%
\pgfsys@useobject{currentmarker}{}%
\end{pgfscope}%
\end{pgfscope}%
\begin{pgfscope}%
\definecolor{textcolor}{rgb}{0.150000,0.150000,0.150000}%
\pgfsetstrokecolor{textcolor}%
\pgfsetfillcolor{textcolor}%
\pgftext[x=0.470280in,y=0.739163in,right,]{\color{textcolor}\sffamily\fontsize{8.000000}{9.600000}\selectfont 3.0}%
\end{pgfscope}%
\begin{pgfscope}%
\pgfpathrectangle{\pgfqpoint{0.548058in}{0.516222in}}{\pgfqpoint{1.739582in}{1.783528in}} %
\pgfusepath{clip}%
\pgfsetroundcap%
\pgfsetroundjoin%
\pgfsetlinewidth{0.803000pt}%
\definecolor{currentstroke}{rgb}{1.000000,1.000000,1.000000}%
\pgfsetstrokecolor{currentstroke}%
\pgfsetdash{}{0pt}%
\pgfpathmoveto{\pgfqpoint{0.548058in}{0.962104in}}%
\pgfpathlineto{\pgfqpoint{2.287641in}{0.962104in}}%
\pgfusepath{stroke}%
\end{pgfscope}%
\begin{pgfscope}%
\pgfsetbuttcap%
\pgfsetroundjoin%
\definecolor{currentfill}{rgb}{0.150000,0.150000,0.150000}%
\pgfsetfillcolor{currentfill}%
\pgfsetlinewidth{0.803000pt}%
\definecolor{currentstroke}{rgb}{0.150000,0.150000,0.150000}%
\pgfsetstrokecolor{currentstroke}%
\pgfsetdash{}{0pt}%
\pgfsys@defobject{currentmarker}{\pgfqpoint{0.000000in}{0.000000in}}{\pgfqpoint{0.000000in}{0.000000in}}{%
\pgfpathmoveto{\pgfqpoint{0.000000in}{0.000000in}}%
\pgfpathlineto{\pgfqpoint{0.000000in}{0.000000in}}%
\pgfusepath{stroke,fill}%
}%
\begin{pgfscope}%
\pgfsys@transformshift{0.548058in}{0.962104in}%
\pgfsys@useobject{currentmarker}{}%
\end{pgfscope}%
\end{pgfscope}%
\begin{pgfscope}%
\definecolor{textcolor}{rgb}{0.150000,0.150000,0.150000}%
\pgfsetstrokecolor{textcolor}%
\pgfsetfillcolor{textcolor}%
\pgftext[x=0.470280in,y=0.962104in,right,]{\color{textcolor}\sffamily\fontsize{8.000000}{9.600000}\selectfont 3.5}%
\end{pgfscope}%
\begin{pgfscope}%
\pgfpathrectangle{\pgfqpoint{0.548058in}{0.516222in}}{\pgfqpoint{1.739582in}{1.783528in}} %
\pgfusepath{clip}%
\pgfsetroundcap%
\pgfsetroundjoin%
\pgfsetlinewidth{0.803000pt}%
\definecolor{currentstroke}{rgb}{1.000000,1.000000,1.000000}%
\pgfsetstrokecolor{currentstroke}%
\pgfsetdash{}{0pt}%
\pgfpathmoveto{\pgfqpoint{0.548058in}{1.185045in}}%
\pgfpathlineto{\pgfqpoint{2.287641in}{1.185045in}}%
\pgfusepath{stroke}%
\end{pgfscope}%
\begin{pgfscope}%
\pgfsetbuttcap%
\pgfsetroundjoin%
\definecolor{currentfill}{rgb}{0.150000,0.150000,0.150000}%
\pgfsetfillcolor{currentfill}%
\pgfsetlinewidth{0.803000pt}%
\definecolor{currentstroke}{rgb}{0.150000,0.150000,0.150000}%
\pgfsetstrokecolor{currentstroke}%
\pgfsetdash{}{0pt}%
\pgfsys@defobject{currentmarker}{\pgfqpoint{0.000000in}{0.000000in}}{\pgfqpoint{0.000000in}{0.000000in}}{%
\pgfpathmoveto{\pgfqpoint{0.000000in}{0.000000in}}%
\pgfpathlineto{\pgfqpoint{0.000000in}{0.000000in}}%
\pgfusepath{stroke,fill}%
}%
\begin{pgfscope}%
\pgfsys@transformshift{0.548058in}{1.185045in}%
\pgfsys@useobject{currentmarker}{}%
\end{pgfscope}%
\end{pgfscope}%
\begin{pgfscope}%
\definecolor{textcolor}{rgb}{0.150000,0.150000,0.150000}%
\pgfsetstrokecolor{textcolor}%
\pgfsetfillcolor{textcolor}%
\pgftext[x=0.470280in,y=1.185045in,right,]{\color{textcolor}\sffamily\fontsize{8.000000}{9.600000}\selectfont 4.0}%
\end{pgfscope}%
\begin{pgfscope}%
\pgfpathrectangle{\pgfqpoint{0.548058in}{0.516222in}}{\pgfqpoint{1.739582in}{1.783528in}} %
\pgfusepath{clip}%
\pgfsetroundcap%
\pgfsetroundjoin%
\pgfsetlinewidth{0.803000pt}%
\definecolor{currentstroke}{rgb}{1.000000,1.000000,1.000000}%
\pgfsetstrokecolor{currentstroke}%
\pgfsetdash{}{0pt}%
\pgfpathmoveto{\pgfqpoint{0.548058in}{1.407986in}}%
\pgfpathlineto{\pgfqpoint{2.287641in}{1.407986in}}%
\pgfusepath{stroke}%
\end{pgfscope}%
\begin{pgfscope}%
\pgfsetbuttcap%
\pgfsetroundjoin%
\definecolor{currentfill}{rgb}{0.150000,0.150000,0.150000}%
\pgfsetfillcolor{currentfill}%
\pgfsetlinewidth{0.803000pt}%
\definecolor{currentstroke}{rgb}{0.150000,0.150000,0.150000}%
\pgfsetstrokecolor{currentstroke}%
\pgfsetdash{}{0pt}%
\pgfsys@defobject{currentmarker}{\pgfqpoint{0.000000in}{0.000000in}}{\pgfqpoint{0.000000in}{0.000000in}}{%
\pgfpathmoveto{\pgfqpoint{0.000000in}{0.000000in}}%
\pgfpathlineto{\pgfqpoint{0.000000in}{0.000000in}}%
\pgfusepath{stroke,fill}%
}%
\begin{pgfscope}%
\pgfsys@transformshift{0.548058in}{1.407986in}%
\pgfsys@useobject{currentmarker}{}%
\end{pgfscope}%
\end{pgfscope}%
\begin{pgfscope}%
\definecolor{textcolor}{rgb}{0.150000,0.150000,0.150000}%
\pgfsetstrokecolor{textcolor}%
\pgfsetfillcolor{textcolor}%
\pgftext[x=0.470280in,y=1.407986in,right,]{\color{textcolor}\sffamily\fontsize{8.000000}{9.600000}\selectfont 4.5}%
\end{pgfscope}%
\begin{pgfscope}%
\pgfpathrectangle{\pgfqpoint{0.548058in}{0.516222in}}{\pgfqpoint{1.739582in}{1.783528in}} %
\pgfusepath{clip}%
\pgfsetroundcap%
\pgfsetroundjoin%
\pgfsetlinewidth{0.803000pt}%
\definecolor{currentstroke}{rgb}{1.000000,1.000000,1.000000}%
\pgfsetstrokecolor{currentstroke}%
\pgfsetdash{}{0pt}%
\pgfpathmoveto{\pgfqpoint{0.548058in}{1.630927in}}%
\pgfpathlineto{\pgfqpoint{2.287641in}{1.630927in}}%
\pgfusepath{stroke}%
\end{pgfscope}%
\begin{pgfscope}%
\pgfsetbuttcap%
\pgfsetroundjoin%
\definecolor{currentfill}{rgb}{0.150000,0.150000,0.150000}%
\pgfsetfillcolor{currentfill}%
\pgfsetlinewidth{0.803000pt}%
\definecolor{currentstroke}{rgb}{0.150000,0.150000,0.150000}%
\pgfsetstrokecolor{currentstroke}%
\pgfsetdash{}{0pt}%
\pgfsys@defobject{currentmarker}{\pgfqpoint{0.000000in}{0.000000in}}{\pgfqpoint{0.000000in}{0.000000in}}{%
\pgfpathmoveto{\pgfqpoint{0.000000in}{0.000000in}}%
\pgfpathlineto{\pgfqpoint{0.000000in}{0.000000in}}%
\pgfusepath{stroke,fill}%
}%
\begin{pgfscope}%
\pgfsys@transformshift{0.548058in}{1.630927in}%
\pgfsys@useobject{currentmarker}{}%
\end{pgfscope}%
\end{pgfscope}%
\begin{pgfscope}%
\definecolor{textcolor}{rgb}{0.150000,0.150000,0.150000}%
\pgfsetstrokecolor{textcolor}%
\pgfsetfillcolor{textcolor}%
\pgftext[x=0.470280in,y=1.630927in,right,]{\color{textcolor}\sffamily\fontsize{8.000000}{9.600000}\selectfont 5.0}%
\end{pgfscope}%
\begin{pgfscope}%
\pgfpathrectangle{\pgfqpoint{0.548058in}{0.516222in}}{\pgfqpoint{1.739582in}{1.783528in}} %
\pgfusepath{clip}%
\pgfsetroundcap%
\pgfsetroundjoin%
\pgfsetlinewidth{0.803000pt}%
\definecolor{currentstroke}{rgb}{1.000000,1.000000,1.000000}%
\pgfsetstrokecolor{currentstroke}%
\pgfsetdash{}{0pt}%
\pgfpathmoveto{\pgfqpoint{0.548058in}{1.853868in}}%
\pgfpathlineto{\pgfqpoint{2.287641in}{1.853868in}}%
\pgfusepath{stroke}%
\end{pgfscope}%
\begin{pgfscope}%
\pgfsetbuttcap%
\pgfsetroundjoin%
\definecolor{currentfill}{rgb}{0.150000,0.150000,0.150000}%
\pgfsetfillcolor{currentfill}%
\pgfsetlinewidth{0.803000pt}%
\definecolor{currentstroke}{rgb}{0.150000,0.150000,0.150000}%
\pgfsetstrokecolor{currentstroke}%
\pgfsetdash{}{0pt}%
\pgfsys@defobject{currentmarker}{\pgfqpoint{0.000000in}{0.000000in}}{\pgfqpoint{0.000000in}{0.000000in}}{%
\pgfpathmoveto{\pgfqpoint{0.000000in}{0.000000in}}%
\pgfpathlineto{\pgfqpoint{0.000000in}{0.000000in}}%
\pgfusepath{stroke,fill}%
}%
\begin{pgfscope}%
\pgfsys@transformshift{0.548058in}{1.853868in}%
\pgfsys@useobject{currentmarker}{}%
\end{pgfscope}%
\end{pgfscope}%
\begin{pgfscope}%
\definecolor{textcolor}{rgb}{0.150000,0.150000,0.150000}%
\pgfsetstrokecolor{textcolor}%
\pgfsetfillcolor{textcolor}%
\pgftext[x=0.470280in,y=1.853868in,right,]{\color{textcolor}\sffamily\fontsize{8.000000}{9.600000}\selectfont 5.5}%
\end{pgfscope}%
\begin{pgfscope}%
\pgfpathrectangle{\pgfqpoint{0.548058in}{0.516222in}}{\pgfqpoint{1.739582in}{1.783528in}} %
\pgfusepath{clip}%
\pgfsetroundcap%
\pgfsetroundjoin%
\pgfsetlinewidth{0.803000pt}%
\definecolor{currentstroke}{rgb}{1.000000,1.000000,1.000000}%
\pgfsetstrokecolor{currentstroke}%
\pgfsetdash{}{0pt}%
\pgfpathmoveto{\pgfqpoint{0.548058in}{2.076809in}}%
\pgfpathlineto{\pgfqpoint{2.287641in}{2.076809in}}%
\pgfusepath{stroke}%
\end{pgfscope}%
\begin{pgfscope}%
\pgfsetbuttcap%
\pgfsetroundjoin%
\definecolor{currentfill}{rgb}{0.150000,0.150000,0.150000}%
\pgfsetfillcolor{currentfill}%
\pgfsetlinewidth{0.803000pt}%
\definecolor{currentstroke}{rgb}{0.150000,0.150000,0.150000}%
\pgfsetstrokecolor{currentstroke}%
\pgfsetdash{}{0pt}%
\pgfsys@defobject{currentmarker}{\pgfqpoint{0.000000in}{0.000000in}}{\pgfqpoint{0.000000in}{0.000000in}}{%
\pgfpathmoveto{\pgfqpoint{0.000000in}{0.000000in}}%
\pgfpathlineto{\pgfqpoint{0.000000in}{0.000000in}}%
\pgfusepath{stroke,fill}%
}%
\begin{pgfscope}%
\pgfsys@transformshift{0.548058in}{2.076809in}%
\pgfsys@useobject{currentmarker}{}%
\end{pgfscope}%
\end{pgfscope}%
\begin{pgfscope}%
\definecolor{textcolor}{rgb}{0.150000,0.150000,0.150000}%
\pgfsetstrokecolor{textcolor}%
\pgfsetfillcolor{textcolor}%
\pgftext[x=0.470280in,y=2.076809in,right,]{\color{textcolor}\sffamily\fontsize{8.000000}{9.600000}\selectfont 6.0}%
\end{pgfscope}%
\begin{pgfscope}%
\pgfpathrectangle{\pgfqpoint{0.548058in}{0.516222in}}{\pgfqpoint{1.739582in}{1.783528in}} %
\pgfusepath{clip}%
\pgfsetroundcap%
\pgfsetroundjoin%
\pgfsetlinewidth{0.803000pt}%
\definecolor{currentstroke}{rgb}{1.000000,1.000000,1.000000}%
\pgfsetstrokecolor{currentstroke}%
\pgfsetdash{}{0pt}%
\pgfpathmoveto{\pgfqpoint{0.548058in}{2.299750in}}%
\pgfpathlineto{\pgfqpoint{2.287641in}{2.299750in}}%
\pgfusepath{stroke}%
\end{pgfscope}%
\begin{pgfscope}%
\pgfsetbuttcap%
\pgfsetroundjoin%
\definecolor{currentfill}{rgb}{0.150000,0.150000,0.150000}%
\pgfsetfillcolor{currentfill}%
\pgfsetlinewidth{0.803000pt}%
\definecolor{currentstroke}{rgb}{0.150000,0.150000,0.150000}%
\pgfsetstrokecolor{currentstroke}%
\pgfsetdash{}{0pt}%
\pgfsys@defobject{currentmarker}{\pgfqpoint{0.000000in}{0.000000in}}{\pgfqpoint{0.000000in}{0.000000in}}{%
\pgfpathmoveto{\pgfqpoint{0.000000in}{0.000000in}}%
\pgfpathlineto{\pgfqpoint{0.000000in}{0.000000in}}%
\pgfusepath{stroke,fill}%
}%
\begin{pgfscope}%
\pgfsys@transformshift{0.548058in}{2.299750in}%
\pgfsys@useobject{currentmarker}{}%
\end{pgfscope}%
\end{pgfscope}%
\begin{pgfscope}%
\definecolor{textcolor}{rgb}{0.150000,0.150000,0.150000}%
\pgfsetstrokecolor{textcolor}%
\pgfsetfillcolor{textcolor}%
\pgftext[x=0.470280in,y=2.299750in,right,]{\color{textcolor}\sffamily\fontsize{8.000000}{9.600000}\selectfont 6.5}%
\end{pgfscope}%
\begin{pgfscope}%
\definecolor{textcolor}{rgb}{0.150000,0.150000,0.150000}%
\pgfsetstrokecolor{textcolor}%
\pgfsetfillcolor{textcolor}%
\pgftext[x=0.242888in,y=1.407986in,,bottom,rotate=90.000000]{\color{textcolor}\sffamily\fontsize{8.800000}{10.560000}\selectfont fall time exp 2 obs 1}%
\end{pgfscope}%
\begin{pgfscope}%
\pgfpathrectangle{\pgfqpoint{0.548058in}{0.516222in}}{\pgfqpoint{1.739582in}{1.783528in}} %
\pgfusepath{clip}%
\pgfsetbuttcap%
\pgfsetroundjoin%
\definecolor{currentfill}{rgb}{0.298039,0.447059,0.690196}%
\pgfsetfillcolor{currentfill}%
\pgfsetlinewidth{0.240900pt}%
\definecolor{currentstroke}{rgb}{1.000000,1.000000,1.000000}%
\pgfsetstrokecolor{currentstroke}%
\pgfsetdash{}{0pt}%
\pgfpathmoveto{\pgfqpoint{0.796570in}{1.180742in}}%
\pgfpathcurveto{\pgfqpoint{0.804806in}{1.180742in}}{\pgfqpoint{0.812706in}{1.184014in}}{\pgfqpoint{0.818530in}{1.189838in}}%
\pgfpathcurveto{\pgfqpoint{0.824354in}{1.195662in}}{\pgfqpoint{0.827626in}{1.203562in}}{\pgfqpoint{0.827626in}{1.211798in}}%
\pgfpathcurveto{\pgfqpoint{0.827626in}{1.220034in}}{\pgfqpoint{0.824354in}{1.227934in}}{\pgfqpoint{0.818530in}{1.233758in}}%
\pgfpathcurveto{\pgfqpoint{0.812706in}{1.239582in}}{\pgfqpoint{0.804806in}{1.242855in}}{\pgfqpoint{0.796570in}{1.242855in}}%
\pgfpathcurveto{\pgfqpoint{0.788334in}{1.242855in}}{\pgfqpoint{0.780434in}{1.239582in}}{\pgfqpoint{0.774610in}{1.233758in}}%
\pgfpathcurveto{\pgfqpoint{0.768786in}{1.227934in}}{\pgfqpoint{0.765513in}{1.220034in}}{\pgfqpoint{0.765513in}{1.211798in}}%
\pgfpathcurveto{\pgfqpoint{0.765513in}{1.203562in}}{\pgfqpoint{0.768786in}{1.195662in}}{\pgfqpoint{0.774610in}{1.189838in}}%
\pgfpathcurveto{\pgfqpoint{0.780434in}{1.184014in}}{\pgfqpoint{0.788334in}{1.180742in}}{\pgfqpoint{0.796570in}{1.180742in}}%
\pgfpathlineto{\pgfqpoint{0.796570in}{1.180742in}}%
\pgfusepath{stroke,fill}%
\end{pgfscope}%
\begin{pgfscope}%
\pgfpathrectangle{\pgfqpoint{0.548058in}{0.516222in}}{\pgfqpoint{1.739582in}{1.783528in}} %
\pgfusepath{clip}%
\pgfsetbuttcap%
\pgfsetroundjoin%
\definecolor{currentfill}{rgb}{0.298039,0.447059,0.690196}%
\pgfsetfillcolor{currentfill}%
\pgfsetlinewidth{0.240900pt}%
\definecolor{currentstroke}{rgb}{1.000000,1.000000,1.000000}%
\pgfsetstrokecolor{currentstroke}%
\pgfsetdash{}{0pt}%
\pgfpathmoveto{\pgfqpoint{1.019593in}{1.064812in}}%
\pgfpathcurveto{\pgfqpoint{1.027830in}{1.064812in}}{\pgfqpoint{1.035730in}{1.068085in}}{\pgfqpoint{1.041554in}{1.073908in}}%
\pgfpathcurveto{\pgfqpoint{1.047378in}{1.079732in}}{\pgfqpoint{1.050650in}{1.087632in}}{\pgfqpoint{1.050650in}{1.095869in}}%
\pgfpathcurveto{\pgfqpoint{1.050650in}{1.104105in}}{\pgfqpoint{1.047378in}{1.112005in}}{\pgfqpoint{1.041554in}{1.117829in}}%
\pgfpathcurveto{\pgfqpoint{1.035730in}{1.123653in}}{\pgfqpoint{1.027830in}{1.126925in}}{\pgfqpoint{1.019593in}{1.126925in}}%
\pgfpathcurveto{\pgfqpoint{1.011357in}{1.126925in}}{\pgfqpoint{1.003457in}{1.123653in}}{\pgfqpoint{0.997633in}{1.117829in}}%
\pgfpathcurveto{\pgfqpoint{0.991809in}{1.112005in}}{\pgfqpoint{0.988537in}{1.104105in}}{\pgfqpoint{0.988537in}{1.095869in}}%
\pgfpathcurveto{\pgfqpoint{0.988537in}{1.087632in}}{\pgfqpoint{0.991809in}{1.079732in}}{\pgfqpoint{0.997633in}{1.073908in}}%
\pgfpathcurveto{\pgfqpoint{1.003457in}{1.068085in}}{\pgfqpoint{1.011357in}{1.064812in}}{\pgfqpoint{1.019593in}{1.064812in}}%
\pgfpathlineto{\pgfqpoint{1.019593in}{1.064812in}}%
\pgfusepath{stroke,fill}%
\end{pgfscope}%
\begin{pgfscope}%
\pgfpathrectangle{\pgfqpoint{0.548058in}{0.516222in}}{\pgfqpoint{1.739582in}{1.783528in}} %
\pgfusepath{clip}%
\pgfsetbuttcap%
\pgfsetroundjoin%
\definecolor{currentfill}{rgb}{0.298039,0.447059,0.690196}%
\pgfsetfillcolor{currentfill}%
\pgfsetlinewidth{0.240900pt}%
\definecolor{currentstroke}{rgb}{1.000000,1.000000,1.000000}%
\pgfsetstrokecolor{currentstroke}%
\pgfsetdash{}{0pt}%
\pgfpathmoveto{\pgfqpoint{1.847966in}{1.015765in}}%
\pgfpathcurveto{\pgfqpoint{1.856202in}{1.015765in}}{\pgfqpoint{1.864102in}{1.019038in}}{\pgfqpoint{1.869926in}{1.024861in}}%
\pgfpathcurveto{\pgfqpoint{1.875750in}{1.030685in}}{\pgfqpoint{1.879022in}{1.038585in}}{\pgfqpoint{1.879022in}{1.046822in}}%
\pgfpathcurveto{\pgfqpoint{1.879022in}{1.055058in}}{\pgfqpoint{1.875750in}{1.062958in}}{\pgfqpoint{1.869926in}{1.068782in}}%
\pgfpathcurveto{\pgfqpoint{1.864102in}{1.074606in}}{\pgfqpoint{1.856202in}{1.077878in}}{\pgfqpoint{1.847966in}{1.077878in}}%
\pgfpathcurveto{\pgfqpoint{1.839730in}{1.077878in}}{\pgfqpoint{1.831830in}{1.074606in}}{\pgfqpoint{1.826006in}{1.068782in}}%
\pgfpathcurveto{\pgfqpoint{1.820182in}{1.062958in}}{\pgfqpoint{1.816909in}{1.055058in}}{\pgfqpoint{1.816909in}{1.046822in}}%
\pgfpathcurveto{\pgfqpoint{1.816909in}{1.038585in}}{\pgfqpoint{1.820182in}{1.030685in}}{\pgfqpoint{1.826006in}{1.024861in}}%
\pgfpathcurveto{\pgfqpoint{1.831830in}{1.019038in}}{\pgfqpoint{1.839730in}{1.015765in}}{\pgfqpoint{1.847966in}{1.015765in}}%
\pgfpathlineto{\pgfqpoint{1.847966in}{1.015765in}}%
\pgfusepath{stroke,fill}%
\end{pgfscope}%
\begin{pgfscope}%
\pgfpathrectangle{\pgfqpoint{0.548058in}{0.516222in}}{\pgfqpoint{1.739582in}{1.783528in}} %
\pgfusepath{clip}%
\pgfsetbuttcap%
\pgfsetroundjoin%
\definecolor{currentfill}{rgb}{0.298039,0.447059,0.690196}%
\pgfsetfillcolor{currentfill}%
\pgfsetlinewidth{0.240900pt}%
\definecolor{currentstroke}{rgb}{1.000000,1.000000,1.000000}%
\pgfsetstrokecolor{currentstroke}%
\pgfsetdash{}{0pt}%
\pgfpathmoveto{\pgfqpoint{2.039129in}{0.953342in}}%
\pgfpathcurveto{\pgfqpoint{2.047365in}{0.953342in}}{\pgfqpoint{2.055265in}{0.956614in}}{\pgfqpoint{2.061089in}{0.962438in}}%
\pgfpathcurveto{\pgfqpoint{2.066913in}{0.968262in}}{\pgfqpoint{2.070185in}{0.976162in}}{\pgfqpoint{2.070185in}{0.984398in}}%
\pgfpathcurveto{\pgfqpoint{2.070185in}{0.992635in}}{\pgfqpoint{2.066913in}{1.000535in}}{\pgfqpoint{2.061089in}{1.006359in}}%
\pgfpathcurveto{\pgfqpoint{2.055265in}{1.012182in}}{\pgfqpoint{2.047365in}{1.015455in}}{\pgfqpoint{2.039129in}{1.015455in}}%
\pgfpathcurveto{\pgfqpoint{2.030893in}{1.015455in}}{\pgfqpoint{2.022993in}{1.012182in}}{\pgfqpoint{2.017169in}{1.006359in}}%
\pgfpathcurveto{\pgfqpoint{2.011345in}{1.000535in}}{\pgfqpoint{2.008072in}{0.992635in}}{\pgfqpoint{2.008072in}{0.984398in}}%
\pgfpathcurveto{\pgfqpoint{2.008072in}{0.976162in}}{\pgfqpoint{2.011345in}{0.968262in}}{\pgfqpoint{2.017169in}{0.962438in}}%
\pgfpathcurveto{\pgfqpoint{2.022993in}{0.956614in}}{\pgfqpoint{2.030893in}{0.953342in}}{\pgfqpoint{2.039129in}{0.953342in}}%
\pgfpathlineto{\pgfqpoint{2.039129in}{0.953342in}}%
\pgfusepath{stroke,fill}%
\end{pgfscope}%
\begin{pgfscope}%
\pgfpathrectangle{\pgfqpoint{0.548058in}{0.516222in}}{\pgfqpoint{1.739582in}{1.783528in}} %
\pgfusepath{clip}%
\pgfsetbuttcap%
\pgfsetroundjoin%
\definecolor{currentfill}{rgb}{0.298039,0.447059,0.690196}%
\pgfsetfillcolor{currentfill}%
\pgfsetlinewidth{0.240900pt}%
\definecolor{currentstroke}{rgb}{1.000000,1.000000,1.000000}%
\pgfsetstrokecolor{currentstroke}%
\pgfsetdash{}{0pt}%
\pgfpathmoveto{\pgfqpoint{1.656803in}{0.574342in}}%
\pgfpathcurveto{\pgfqpoint{1.665039in}{0.574342in}}{\pgfqpoint{1.672939in}{0.577614in}}{\pgfqpoint{1.678763in}{0.583438in}}%
\pgfpathcurveto{\pgfqpoint{1.684587in}{0.589262in}}{\pgfqpoint{1.687860in}{0.597162in}}{\pgfqpoint{1.687860in}{0.605399in}}%
\pgfpathcurveto{\pgfqpoint{1.687860in}{0.613635in}}{\pgfqpoint{1.684587in}{0.621535in}}{\pgfqpoint{1.678763in}{0.627359in}}%
\pgfpathcurveto{\pgfqpoint{1.672939in}{0.633183in}}{\pgfqpoint{1.665039in}{0.636455in}}{\pgfqpoint{1.656803in}{0.636455in}}%
\pgfpathcurveto{\pgfqpoint{1.648567in}{0.636455in}}{\pgfqpoint{1.640667in}{0.633183in}}{\pgfqpoint{1.634843in}{0.627359in}}%
\pgfpathcurveto{\pgfqpoint{1.629019in}{0.621535in}}{\pgfqpoint{1.625747in}{0.613635in}}{\pgfqpoint{1.625747in}{0.605399in}}%
\pgfpathcurveto{\pgfqpoint{1.625747in}{0.597162in}}{\pgfqpoint{1.629019in}{0.589262in}}{\pgfqpoint{1.634843in}{0.583438in}}%
\pgfpathcurveto{\pgfqpoint{1.640667in}{0.577614in}}{\pgfqpoint{1.648567in}{0.574342in}}{\pgfqpoint{1.656803in}{0.574342in}}%
\pgfpathlineto{\pgfqpoint{1.656803in}{0.574342in}}%
\pgfusepath{stroke,fill}%
\end{pgfscope}%
\begin{pgfscope}%
\pgfpathrectangle{\pgfqpoint{0.548058in}{0.516222in}}{\pgfqpoint{1.739582in}{1.783528in}} %
\pgfusepath{clip}%
\pgfsetbuttcap%
\pgfsetroundjoin%
\definecolor{currentfill}{rgb}{0.298039,0.447059,0.690196}%
\pgfsetfillcolor{currentfill}%
\pgfsetlinewidth{0.240900pt}%
\definecolor{currentstroke}{rgb}{1.000000,1.000000,1.000000}%
\pgfsetstrokecolor{currentstroke}%
\pgfsetdash{}{0pt}%
\pgfpathmoveto{\pgfqpoint{1.816105in}{0.873083in}}%
\pgfpathcurveto{\pgfqpoint{1.824342in}{0.873083in}}{\pgfqpoint{1.832242in}{0.876355in}}{\pgfqpoint{1.838066in}{0.882179in}}%
\pgfpathcurveto{\pgfqpoint{1.843890in}{0.888003in}}{\pgfqpoint{1.847162in}{0.895903in}}{\pgfqpoint{1.847162in}{0.904140in}}%
\pgfpathcurveto{\pgfqpoint{1.847162in}{0.912376in}}{\pgfqpoint{1.843890in}{0.920276in}}{\pgfqpoint{1.838066in}{0.926100in}}%
\pgfpathcurveto{\pgfqpoint{1.832242in}{0.931924in}}{\pgfqpoint{1.824342in}{0.935196in}}{\pgfqpoint{1.816105in}{0.935196in}}%
\pgfpathcurveto{\pgfqpoint{1.807869in}{0.935196in}}{\pgfqpoint{1.799969in}{0.931924in}}{\pgfqpoint{1.794145in}{0.926100in}}%
\pgfpathcurveto{\pgfqpoint{1.788321in}{0.920276in}}{\pgfqpoint{1.785049in}{0.912376in}}{\pgfqpoint{1.785049in}{0.904140in}}%
\pgfpathcurveto{\pgfqpoint{1.785049in}{0.895903in}}{\pgfqpoint{1.788321in}{0.888003in}}{\pgfqpoint{1.794145in}{0.882179in}}%
\pgfpathcurveto{\pgfqpoint{1.799969in}{0.876355in}}{\pgfqpoint{1.807869in}{0.873083in}}{\pgfqpoint{1.816105in}{0.873083in}}%
\pgfpathlineto{\pgfqpoint{1.816105in}{0.873083in}}%
\pgfusepath{stroke,fill}%
\end{pgfscope}%
\begin{pgfscope}%
\pgfpathrectangle{\pgfqpoint{0.548058in}{0.516222in}}{\pgfqpoint{1.739582in}{1.783528in}} %
\pgfusepath{clip}%
\pgfsetbuttcap%
\pgfsetroundjoin%
\definecolor{currentfill}{rgb}{0.298039,0.447059,0.690196}%
\pgfsetfillcolor{currentfill}%
\pgfsetlinewidth{0.240900pt}%
\definecolor{currentstroke}{rgb}{1.000000,1.000000,1.000000}%
\pgfsetstrokecolor{currentstroke}%
\pgfsetdash{}{0pt}%
\pgfpathmoveto{\pgfqpoint{1.051454in}{1.087106in}}%
\pgfpathcurveto{\pgfqpoint{1.059690in}{1.087106in}}{\pgfqpoint{1.067590in}{1.090379in}}{\pgfqpoint{1.073414in}{1.096203in}}%
\pgfpathcurveto{\pgfqpoint{1.079238in}{1.102027in}}{\pgfqpoint{1.082510in}{1.109927in}}{\pgfqpoint{1.082510in}{1.118163in}}%
\pgfpathcurveto{\pgfqpoint{1.082510in}{1.126399in}}{\pgfqpoint{1.079238in}{1.134299in}}{\pgfqpoint{1.073414in}{1.140123in}}%
\pgfpathcurveto{\pgfqpoint{1.067590in}{1.145947in}}{\pgfqpoint{1.059690in}{1.149219in}}{\pgfqpoint{1.051454in}{1.149219in}}%
\pgfpathcurveto{\pgfqpoint{1.043218in}{1.149219in}}{\pgfqpoint{1.035317in}{1.145947in}}{\pgfqpoint{1.029494in}{1.140123in}}%
\pgfpathcurveto{\pgfqpoint{1.023670in}{1.134299in}}{\pgfqpoint{1.020397in}{1.126399in}}{\pgfqpoint{1.020397in}{1.118163in}}%
\pgfpathcurveto{\pgfqpoint{1.020397in}{1.109927in}}{\pgfqpoint{1.023670in}{1.102027in}}{\pgfqpoint{1.029494in}{1.096203in}}%
\pgfpathcurveto{\pgfqpoint{1.035317in}{1.090379in}}{\pgfqpoint{1.043218in}{1.087106in}}{\pgfqpoint{1.051454in}{1.087106in}}%
\pgfpathlineto{\pgfqpoint{1.051454in}{1.087106in}}%
\pgfusepath{stroke,fill}%
\end{pgfscope}%
\begin{pgfscope}%
\pgfpathrectangle{\pgfqpoint{0.548058in}{0.516222in}}{\pgfqpoint{1.739582in}{1.783528in}} %
\pgfusepath{clip}%
\pgfsetbuttcap%
\pgfsetroundjoin%
\definecolor{currentfill}{rgb}{0.298039,0.447059,0.690196}%
\pgfsetfillcolor{currentfill}%
\pgfsetlinewidth{0.240900pt}%
\definecolor{currentstroke}{rgb}{1.000000,1.000000,1.000000}%
\pgfsetstrokecolor{currentstroke}%
\pgfsetdash{}{0pt}%
\pgfpathmoveto{\pgfqpoint{1.433780in}{1.376930in}}%
\pgfpathcurveto{\pgfqpoint{1.442016in}{1.376930in}}{\pgfqpoint{1.449916in}{1.380202in}}{\pgfqpoint{1.455740in}{1.386026in}}%
\pgfpathcurveto{\pgfqpoint{1.461564in}{1.391850in}}{\pgfqpoint{1.464836in}{1.399750in}}{\pgfqpoint{1.464836in}{1.407986in}}%
\pgfpathcurveto{\pgfqpoint{1.464836in}{1.416222in}}{\pgfqpoint{1.461564in}{1.424122in}}{\pgfqpoint{1.455740in}{1.429946in}}%
\pgfpathcurveto{\pgfqpoint{1.449916in}{1.435770in}}{\pgfqpoint{1.442016in}{1.439043in}}{\pgfqpoint{1.433780in}{1.439043in}}%
\pgfpathcurveto{\pgfqpoint{1.425543in}{1.439043in}}{\pgfqpoint{1.417643in}{1.435770in}}{\pgfqpoint{1.411819in}{1.429946in}}%
\pgfpathcurveto{\pgfqpoint{1.405995in}{1.424122in}}{\pgfqpoint{1.402723in}{1.416222in}}{\pgfqpoint{1.402723in}{1.407986in}}%
\pgfpathcurveto{\pgfqpoint{1.402723in}{1.399750in}}{\pgfqpoint{1.405995in}{1.391850in}}{\pgfqpoint{1.411819in}{1.386026in}}%
\pgfpathcurveto{\pgfqpoint{1.417643in}{1.380202in}}{\pgfqpoint{1.425543in}{1.376930in}}{\pgfqpoint{1.433780in}{1.376930in}}%
\pgfpathlineto{\pgfqpoint{1.433780in}{1.376930in}}%
\pgfusepath{stroke,fill}%
\end{pgfscope}%
\begin{pgfscope}%
\pgfpathrectangle{\pgfqpoint{0.548058in}{0.516222in}}{\pgfqpoint{1.739582in}{1.783528in}} %
\pgfusepath{clip}%
\pgfsetbuttcap%
\pgfsetroundjoin%
\definecolor{currentfill}{rgb}{0.298039,0.447059,0.690196}%
\pgfsetfillcolor{currentfill}%
\pgfsetlinewidth{0.240900pt}%
\definecolor{currentstroke}{rgb}{1.000000,1.000000,1.000000}%
\pgfsetstrokecolor{currentstroke}%
\pgfsetdash{}{0pt}%
\pgfpathmoveto{\pgfqpoint{1.370059in}{1.372471in}}%
\pgfpathcurveto{\pgfqpoint{1.378295in}{1.372471in}}{\pgfqpoint{1.386195in}{1.375743in}}{\pgfqpoint{1.392019in}{1.381567in}}%
\pgfpathcurveto{\pgfqpoint{1.397843in}{1.387391in}}{\pgfqpoint{1.401115in}{1.395291in}}{\pgfqpoint{1.401115in}{1.403527in}}%
\pgfpathcurveto{\pgfqpoint{1.401115in}{1.411764in}}{\pgfqpoint{1.397843in}{1.419664in}}{\pgfqpoint{1.392019in}{1.425488in}}%
\pgfpathcurveto{\pgfqpoint{1.386195in}{1.431311in}}{\pgfqpoint{1.378295in}{1.434584in}}{\pgfqpoint{1.370059in}{1.434584in}}%
\pgfpathcurveto{\pgfqpoint{1.361822in}{1.434584in}}{\pgfqpoint{1.353922in}{1.431311in}}{\pgfqpoint{1.348098in}{1.425488in}}%
\pgfpathcurveto{\pgfqpoint{1.342274in}{1.419664in}}{\pgfqpoint{1.339002in}{1.411764in}}{\pgfqpoint{1.339002in}{1.403527in}}%
\pgfpathcurveto{\pgfqpoint{1.339002in}{1.395291in}}{\pgfqpoint{1.342274in}{1.387391in}}{\pgfqpoint{1.348098in}{1.381567in}}%
\pgfpathcurveto{\pgfqpoint{1.353922in}{1.375743in}}{\pgfqpoint{1.361822in}{1.372471in}}{\pgfqpoint{1.370059in}{1.372471in}}%
\pgfpathlineto{\pgfqpoint{1.370059in}{1.372471in}}%
\pgfusepath{stroke,fill}%
\end{pgfscope}%
\begin{pgfscope}%
\pgfpathrectangle{\pgfqpoint{0.548058in}{0.516222in}}{\pgfqpoint{1.739582in}{1.783528in}} %
\pgfusepath{clip}%
\pgfsetbuttcap%
\pgfsetroundjoin%
\definecolor{currentfill}{rgb}{0.298039,0.447059,0.690196}%
\pgfsetfillcolor{currentfill}%
\pgfsetlinewidth{0.240900pt}%
\definecolor{currentstroke}{rgb}{1.000000,1.000000,1.000000}%
\pgfsetstrokecolor{currentstroke}%
\pgfsetdash{}{0pt}%
\pgfpathmoveto{\pgfqpoint{1.083314in}{0.788365in}}%
\pgfpathcurveto{\pgfqpoint{1.091551in}{0.788365in}}{\pgfqpoint{1.099451in}{0.791638in}}{\pgfqpoint{1.105275in}{0.797462in}}%
\pgfpathcurveto{\pgfqpoint{1.111098in}{0.803286in}}{\pgfqpoint{1.114371in}{0.811186in}}{\pgfqpoint{1.114371in}{0.819422in}}%
\pgfpathcurveto{\pgfqpoint{1.114371in}{0.827658in}}{\pgfqpoint{1.111098in}{0.835558in}}{\pgfqpoint{1.105275in}{0.841382in}}%
\pgfpathcurveto{\pgfqpoint{1.099451in}{0.847206in}}{\pgfqpoint{1.091551in}{0.850478in}}{\pgfqpoint{1.083314in}{0.850478in}}%
\pgfpathcurveto{\pgfqpoint{1.075078in}{0.850478in}}{\pgfqpoint{1.067178in}{0.847206in}}{\pgfqpoint{1.061354in}{0.841382in}}%
\pgfpathcurveto{\pgfqpoint{1.055530in}{0.835558in}}{\pgfqpoint{1.052258in}{0.827658in}}{\pgfqpoint{1.052258in}{0.819422in}}%
\pgfpathcurveto{\pgfqpoint{1.052258in}{0.811186in}}{\pgfqpoint{1.055530in}{0.803286in}}{\pgfqpoint{1.061354in}{0.797462in}}%
\pgfpathcurveto{\pgfqpoint{1.067178in}{0.791638in}}{\pgfqpoint{1.075078in}{0.788365in}}{\pgfqpoint{1.083314in}{0.788365in}}%
\pgfpathlineto{\pgfqpoint{1.083314in}{0.788365in}}%
\pgfusepath{stroke,fill}%
\end{pgfscope}%
\begin{pgfscope}%
\pgfpathrectangle{\pgfqpoint{0.548058in}{0.516222in}}{\pgfqpoint{1.739582in}{1.783528in}} %
\pgfusepath{clip}%
\pgfsetbuttcap%
\pgfsetroundjoin%
\definecolor{currentfill}{rgb}{0.298039,0.447059,0.690196}%
\pgfsetfillcolor{currentfill}%
\pgfsetlinewidth{0.240900pt}%
\definecolor{currentstroke}{rgb}{1.000000,1.000000,1.000000}%
\pgfsetstrokecolor{currentstroke}%
\pgfsetdash{}{0pt}%
\pgfpathmoveto{\pgfqpoint{1.497501in}{1.176283in}}%
\pgfpathcurveto{\pgfqpoint{1.505737in}{1.176283in}}{\pgfqpoint{1.513637in}{1.179555in}}{\pgfqpoint{1.519461in}{1.185379in}}%
\pgfpathcurveto{\pgfqpoint{1.525285in}{1.191203in}}{\pgfqpoint{1.528557in}{1.199103in}}{\pgfqpoint{1.528557in}{1.207339in}}%
\pgfpathcurveto{\pgfqpoint{1.528557in}{1.215576in}}{\pgfqpoint{1.525285in}{1.223476in}}{\pgfqpoint{1.519461in}{1.229299in}}%
\pgfpathcurveto{\pgfqpoint{1.513637in}{1.235123in}}{\pgfqpoint{1.505737in}{1.238396in}}{\pgfqpoint{1.497501in}{1.238396in}}%
\pgfpathcurveto{\pgfqpoint{1.489264in}{1.238396in}}{\pgfqpoint{1.481364in}{1.235123in}}{\pgfqpoint{1.475540in}{1.229299in}}%
\pgfpathcurveto{\pgfqpoint{1.469716in}{1.223476in}}{\pgfqpoint{1.466444in}{1.215576in}}{\pgfqpoint{1.466444in}{1.207339in}}%
\pgfpathcurveto{\pgfqpoint{1.466444in}{1.199103in}}{\pgfqpoint{1.469716in}{1.191203in}}{\pgfqpoint{1.475540in}{1.185379in}}%
\pgfpathcurveto{\pgfqpoint{1.481364in}{1.179555in}}{\pgfqpoint{1.489264in}{1.176283in}}{\pgfqpoint{1.497501in}{1.176283in}}%
\pgfpathlineto{\pgfqpoint{1.497501in}{1.176283in}}%
\pgfusepath{stroke,fill}%
\end{pgfscope}%
\begin{pgfscope}%
\pgfpathrectangle{\pgfqpoint{0.548058in}{0.516222in}}{\pgfqpoint{1.739582in}{1.783528in}} %
\pgfusepath{clip}%
\pgfsetbuttcap%
\pgfsetroundjoin%
\definecolor{currentfill}{rgb}{0.298039,0.447059,0.690196}%
\pgfsetfillcolor{currentfill}%
\pgfsetlinewidth{0.240900pt}%
\definecolor{currentstroke}{rgb}{1.000000,1.000000,1.000000}%
\pgfsetstrokecolor{currentstroke}%
\pgfsetdash{}{0pt}%
\pgfpathmoveto{\pgfqpoint{1.688664in}{0.717024in}}%
\pgfpathcurveto{\pgfqpoint{1.696900in}{0.717024in}}{\pgfqpoint{1.704800in}{0.720297in}}{\pgfqpoint{1.710624in}{0.726121in}}%
\pgfpathcurveto{\pgfqpoint{1.716448in}{0.731945in}}{\pgfqpoint{1.719720in}{0.739845in}}{\pgfqpoint{1.719720in}{0.748081in}}%
\pgfpathcurveto{\pgfqpoint{1.719720in}{0.756317in}}{\pgfqpoint{1.716448in}{0.764217in}}{\pgfqpoint{1.710624in}{0.770041in}}%
\pgfpathcurveto{\pgfqpoint{1.704800in}{0.775865in}}{\pgfqpoint{1.696900in}{0.779137in}}{\pgfqpoint{1.688664in}{0.779137in}}%
\pgfpathcurveto{\pgfqpoint{1.680427in}{0.779137in}}{\pgfqpoint{1.672527in}{0.775865in}}{\pgfqpoint{1.666703in}{0.770041in}}%
\pgfpathcurveto{\pgfqpoint{1.660879in}{0.764217in}}{\pgfqpoint{1.657607in}{0.756317in}}{\pgfqpoint{1.657607in}{0.748081in}}%
\pgfpathcurveto{\pgfqpoint{1.657607in}{0.739845in}}{\pgfqpoint{1.660879in}{0.731945in}}{\pgfqpoint{1.666703in}{0.726121in}}%
\pgfpathcurveto{\pgfqpoint{1.672527in}{0.720297in}}{\pgfqpoint{1.680427in}{0.717024in}}{\pgfqpoint{1.688664in}{0.717024in}}%
\pgfpathlineto{\pgfqpoint{1.688664in}{0.717024in}}%
\pgfusepath{stroke,fill}%
\end{pgfscope}%
\begin{pgfscope}%
\pgfpathrectangle{\pgfqpoint{0.548058in}{0.516222in}}{\pgfqpoint{1.739582in}{1.783528in}} %
\pgfusepath{clip}%
\pgfsetbuttcap%
\pgfsetroundjoin%
\definecolor{currentfill}{rgb}{0.298039,0.447059,0.690196}%
\pgfsetfillcolor{currentfill}%
\pgfsetlinewidth{0.240900pt}%
\definecolor{currentstroke}{rgb}{1.000000,1.000000,1.000000}%
\pgfsetstrokecolor{currentstroke}%
\pgfsetdash{}{0pt}%
\pgfpathmoveto{\pgfqpoint{0.987733in}{1.403683in}}%
\pgfpathcurveto{\pgfqpoint{0.995969in}{1.403683in}}{\pgfqpoint{1.003869in}{1.406955in}}{\pgfqpoint{1.009693in}{1.412779in}}%
\pgfpathcurveto{\pgfqpoint{1.015517in}{1.418603in}}{\pgfqpoint{1.018789in}{1.426503in}}{\pgfqpoint{1.018789in}{1.434739in}}%
\pgfpathcurveto{\pgfqpoint{1.018789in}{1.442975in}}{\pgfqpoint{1.015517in}{1.450875in}}{\pgfqpoint{1.009693in}{1.456699in}}%
\pgfpathcurveto{\pgfqpoint{1.003869in}{1.462523in}}{\pgfqpoint{0.995969in}{1.465796in}}{\pgfqpoint{0.987733in}{1.465796in}}%
\pgfpathcurveto{\pgfqpoint{0.979497in}{1.465796in}}{\pgfqpoint{0.971597in}{1.462523in}}{\pgfqpoint{0.965773in}{1.456699in}}%
\pgfpathcurveto{\pgfqpoint{0.959949in}{1.450875in}}{\pgfqpoint{0.956676in}{1.442975in}}{\pgfqpoint{0.956676in}{1.434739in}}%
\pgfpathcurveto{\pgfqpoint{0.956676in}{1.426503in}}{\pgfqpoint{0.959949in}{1.418603in}}{\pgfqpoint{0.965773in}{1.412779in}}%
\pgfpathcurveto{\pgfqpoint{0.971597in}{1.406955in}}{\pgfqpoint{0.979497in}{1.403683in}}{\pgfqpoint{0.987733in}{1.403683in}}%
\pgfpathlineto{\pgfqpoint{0.987733in}{1.403683in}}%
\pgfusepath{stroke,fill}%
\end{pgfscope}%
\begin{pgfscope}%
\pgfpathrectangle{\pgfqpoint{0.548058in}{0.516222in}}{\pgfqpoint{1.739582in}{1.783528in}} %
\pgfusepath{clip}%
\pgfsetbuttcap%
\pgfsetroundjoin%
\definecolor{currentfill}{rgb}{0.298039,0.447059,0.690196}%
\pgfsetfillcolor{currentfill}%
\pgfsetlinewidth{0.240900pt}%
\definecolor{currentstroke}{rgb}{1.000000,1.000000,1.000000}%
\pgfsetstrokecolor{currentstroke}%
\pgfsetdash{}{0pt}%
\pgfpathmoveto{\pgfqpoint{1.401919in}{0.873083in}}%
\pgfpathcurveto{\pgfqpoint{1.410155in}{0.873083in}}{\pgfqpoint{1.418055in}{0.876355in}}{\pgfqpoint{1.423879in}{0.882179in}}%
\pgfpathcurveto{\pgfqpoint{1.429703in}{0.888003in}}{\pgfqpoint{1.432976in}{0.895903in}}{\pgfqpoint{1.432976in}{0.904140in}}%
\pgfpathcurveto{\pgfqpoint{1.432976in}{0.912376in}}{\pgfqpoint{1.429703in}{0.920276in}}{\pgfqpoint{1.423879in}{0.926100in}}%
\pgfpathcurveto{\pgfqpoint{1.418055in}{0.931924in}}{\pgfqpoint{1.410155in}{0.935196in}}{\pgfqpoint{1.401919in}{0.935196in}}%
\pgfpathcurveto{\pgfqpoint{1.393683in}{0.935196in}}{\pgfqpoint{1.385783in}{0.931924in}}{\pgfqpoint{1.379959in}{0.926100in}}%
\pgfpathcurveto{\pgfqpoint{1.374135in}{0.920276in}}{\pgfqpoint{1.370863in}{0.912376in}}{\pgfqpoint{1.370863in}{0.904140in}}%
\pgfpathcurveto{\pgfqpoint{1.370863in}{0.895903in}}{\pgfqpoint{1.374135in}{0.888003in}}{\pgfqpoint{1.379959in}{0.882179in}}%
\pgfpathcurveto{\pgfqpoint{1.385783in}{0.876355in}}{\pgfqpoint{1.393683in}{0.873083in}}{\pgfqpoint{1.401919in}{0.873083in}}%
\pgfpathlineto{\pgfqpoint{1.401919in}{0.873083in}}%
\pgfusepath{stroke,fill}%
\end{pgfscope}%
\begin{pgfscope}%
\pgfpathrectangle{\pgfqpoint{0.548058in}{0.516222in}}{\pgfqpoint{1.739582in}{1.783528in}} %
\pgfusepath{clip}%
\pgfsetbuttcap%
\pgfsetroundjoin%
\definecolor{currentfill}{rgb}{0.298039,0.447059,0.690196}%
\pgfsetfillcolor{currentfill}%
\pgfsetlinewidth{0.240900pt}%
\definecolor{currentstroke}{rgb}{1.000000,1.000000,1.000000}%
\pgfsetstrokecolor{currentstroke}%
\pgfsetdash{}{0pt}%
\pgfpathmoveto{\pgfqpoint{1.943547in}{0.971177in}}%
\pgfpathcurveto{\pgfqpoint{1.951784in}{0.971177in}}{\pgfqpoint{1.959684in}{0.974449in}}{\pgfqpoint{1.965508in}{0.980273in}}%
\pgfpathcurveto{\pgfqpoint{1.971332in}{0.986097in}}{\pgfqpoint{1.974604in}{0.993997in}}{\pgfqpoint{1.974604in}{1.002234in}}%
\pgfpathcurveto{\pgfqpoint{1.974604in}{1.010470in}}{\pgfqpoint{1.971332in}{1.018370in}}{\pgfqpoint{1.965508in}{1.024194in}}%
\pgfpathcurveto{\pgfqpoint{1.959684in}{1.030018in}}{\pgfqpoint{1.951784in}{1.033290in}}{\pgfqpoint{1.943547in}{1.033290in}}%
\pgfpathcurveto{\pgfqpoint{1.935311in}{1.033290in}}{\pgfqpoint{1.927411in}{1.030018in}}{\pgfqpoint{1.921587in}{1.024194in}}%
\pgfpathcurveto{\pgfqpoint{1.915763in}{1.018370in}}{\pgfqpoint{1.912491in}{1.010470in}}{\pgfqpoint{1.912491in}{1.002234in}}%
\pgfpathcurveto{\pgfqpoint{1.912491in}{0.993997in}}{\pgfqpoint{1.915763in}{0.986097in}}{\pgfqpoint{1.921587in}{0.980273in}}%
\pgfpathcurveto{\pgfqpoint{1.927411in}{0.974449in}}{\pgfqpoint{1.935311in}{0.971177in}}{\pgfqpoint{1.943547in}{0.971177in}}%
\pgfpathlineto{\pgfqpoint{1.943547in}{0.971177in}}%
\pgfusepath{stroke,fill}%
\end{pgfscope}%
\begin{pgfscope}%
\pgfpathrectangle{\pgfqpoint{0.548058in}{0.516222in}}{\pgfqpoint{1.739582in}{1.783528in}} %
\pgfusepath{clip}%
\pgfsetbuttcap%
\pgfsetroundjoin%
\definecolor{currentfill}{rgb}{0.298039,0.447059,0.690196}%
\pgfsetfillcolor{currentfill}%
\pgfsetlinewidth{0.240900pt}%
\definecolor{currentstroke}{rgb}{1.000000,1.000000,1.000000}%
\pgfsetstrokecolor{currentstroke}%
\pgfsetdash{}{0pt}%
\pgfpathmoveto{\pgfqpoint{1.242617in}{1.403683in}}%
\pgfpathcurveto{\pgfqpoint{1.250853in}{1.403683in}}{\pgfqpoint{1.258753in}{1.406955in}}{\pgfqpoint{1.264577in}{1.412779in}}%
\pgfpathcurveto{\pgfqpoint{1.270401in}{1.418603in}}{\pgfqpoint{1.273673in}{1.426503in}}{\pgfqpoint{1.273673in}{1.434739in}}%
\pgfpathcurveto{\pgfqpoint{1.273673in}{1.442975in}}{\pgfqpoint{1.270401in}{1.450875in}}{\pgfqpoint{1.264577in}{1.456699in}}%
\pgfpathcurveto{\pgfqpoint{1.258753in}{1.462523in}}{\pgfqpoint{1.250853in}{1.465796in}}{\pgfqpoint{1.242617in}{1.465796in}}%
\pgfpathcurveto{\pgfqpoint{1.234380in}{1.465796in}}{\pgfqpoint{1.226480in}{1.462523in}}{\pgfqpoint{1.220656in}{1.456699in}}%
\pgfpathcurveto{\pgfqpoint{1.214833in}{1.450875in}}{\pgfqpoint{1.211560in}{1.442975in}}{\pgfqpoint{1.211560in}{1.434739in}}%
\pgfpathcurveto{\pgfqpoint{1.211560in}{1.426503in}}{\pgfqpoint{1.214833in}{1.418603in}}{\pgfqpoint{1.220656in}{1.412779in}}%
\pgfpathcurveto{\pgfqpoint{1.226480in}{1.406955in}}{\pgfqpoint{1.234380in}{1.403683in}}{\pgfqpoint{1.242617in}{1.403683in}}%
\pgfpathlineto{\pgfqpoint{1.242617in}{1.403683in}}%
\pgfusepath{stroke,fill}%
\end{pgfscope}%
\begin{pgfscope}%
\pgfpathrectangle{\pgfqpoint{0.548058in}{0.516222in}}{\pgfqpoint{1.739582in}{1.783528in}} %
\pgfusepath{clip}%
\pgfsetbuttcap%
\pgfsetroundjoin%
\definecolor{currentfill}{rgb}{0.298039,0.447059,0.690196}%
\pgfsetfillcolor{currentfill}%
\pgfsetlinewidth{0.240900pt}%
\definecolor{currentstroke}{rgb}{1.000000,1.000000,1.000000}%
\pgfsetstrokecolor{currentstroke}%
\pgfsetdash{}{0pt}%
\pgfpathmoveto{\pgfqpoint{0.828430in}{1.537447in}}%
\pgfpathcurveto{\pgfqpoint{0.836667in}{1.537447in}}{\pgfqpoint{0.844567in}{1.540719in}}{\pgfqpoint{0.850391in}{1.546543in}}%
\pgfpathcurveto{\pgfqpoint{0.856215in}{1.552367in}}{\pgfqpoint{0.859487in}{1.560267in}}{\pgfqpoint{0.859487in}{1.568504in}}%
\pgfpathcurveto{\pgfqpoint{0.859487in}{1.576740in}}{\pgfqpoint{0.856215in}{1.584640in}}{\pgfqpoint{0.850391in}{1.590464in}}%
\pgfpathcurveto{\pgfqpoint{0.844567in}{1.596288in}}{\pgfqpoint{0.836667in}{1.599560in}}{\pgfqpoint{0.828430in}{1.599560in}}%
\pgfpathcurveto{\pgfqpoint{0.820194in}{1.599560in}}{\pgfqpoint{0.812294in}{1.596288in}}{\pgfqpoint{0.806470in}{1.590464in}}%
\pgfpathcurveto{\pgfqpoint{0.800646in}{1.584640in}}{\pgfqpoint{0.797374in}{1.576740in}}{\pgfqpoint{0.797374in}{1.568504in}}%
\pgfpathcurveto{\pgfqpoint{0.797374in}{1.560267in}}{\pgfqpoint{0.800646in}{1.552367in}}{\pgfqpoint{0.806470in}{1.546543in}}%
\pgfpathcurveto{\pgfqpoint{0.812294in}{1.540719in}}{\pgfqpoint{0.820194in}{1.537447in}}{\pgfqpoint{0.828430in}{1.537447in}}%
\pgfpathlineto{\pgfqpoint{0.828430in}{1.537447in}}%
\pgfusepath{stroke,fill}%
\end{pgfscope}%
\begin{pgfscope}%
\pgfpathrectangle{\pgfqpoint{0.548058in}{0.516222in}}{\pgfqpoint{1.739582in}{1.783528in}} %
\pgfusepath{clip}%
\pgfsetbuttcap%
\pgfsetroundjoin%
\definecolor{currentfill}{rgb}{0.298039,0.447059,0.690196}%
\pgfsetfillcolor{currentfill}%
\pgfsetlinewidth{0.240900pt}%
\definecolor{currentstroke}{rgb}{1.000000,1.000000,1.000000}%
\pgfsetstrokecolor{currentstroke}%
\pgfsetdash{}{0pt}%
\pgfpathmoveto{\pgfqpoint{1.338198in}{0.877542in}}%
\pgfpathcurveto{\pgfqpoint{1.346434in}{0.877542in}}{\pgfqpoint{1.354335in}{0.880814in}}{\pgfqpoint{1.360158in}{0.886638in}}%
\pgfpathcurveto{\pgfqpoint{1.365982in}{0.892462in}}{\pgfqpoint{1.369255in}{0.900362in}}{\pgfqpoint{1.369255in}{0.908598in}}%
\pgfpathcurveto{\pgfqpoint{1.369255in}{0.916835in}}{\pgfqpoint{1.365982in}{0.924735in}}{\pgfqpoint{1.360158in}{0.930559in}}%
\pgfpathcurveto{\pgfqpoint{1.354335in}{0.936383in}}{\pgfqpoint{1.346434in}{0.939655in}}{\pgfqpoint{1.338198in}{0.939655in}}%
\pgfpathcurveto{\pgfqpoint{1.329962in}{0.939655in}}{\pgfqpoint{1.322062in}{0.936383in}}{\pgfqpoint{1.316238in}{0.930559in}}%
\pgfpathcurveto{\pgfqpoint{1.310414in}{0.924735in}}{\pgfqpoint{1.307142in}{0.916835in}}{\pgfqpoint{1.307142in}{0.908598in}}%
\pgfpathcurveto{\pgfqpoint{1.307142in}{0.900362in}}{\pgfqpoint{1.310414in}{0.892462in}}{\pgfqpoint{1.316238in}{0.886638in}}%
\pgfpathcurveto{\pgfqpoint{1.322062in}{0.880814in}}{\pgfqpoint{1.329962in}{0.877542in}}{\pgfqpoint{1.338198in}{0.877542in}}%
\pgfpathlineto{\pgfqpoint{1.338198in}{0.877542in}}%
\pgfusepath{stroke,fill}%
\end{pgfscope}%
\begin{pgfscope}%
\pgfpathrectangle{\pgfqpoint{0.548058in}{0.516222in}}{\pgfqpoint{1.739582in}{1.783528in}} %
\pgfusepath{clip}%
\pgfsetbuttcap%
\pgfsetroundjoin%
\definecolor{currentfill}{rgb}{0.298039,0.447059,0.690196}%
\pgfsetfillcolor{currentfill}%
\pgfsetlinewidth{0.240900pt}%
\definecolor{currentstroke}{rgb}{1.000000,1.000000,1.000000}%
\pgfsetstrokecolor{currentstroke}%
\pgfsetdash{}{0pt}%
\pgfpathmoveto{\pgfqpoint{1.911687in}{0.815118in}}%
\pgfpathcurveto{\pgfqpoint{1.919923in}{0.815118in}}{\pgfqpoint{1.927823in}{0.818391in}}{\pgfqpoint{1.933647in}{0.824215in}}%
\pgfpathcurveto{\pgfqpoint{1.939471in}{0.830039in}}{\pgfqpoint{1.942743in}{0.837939in}}{\pgfqpoint{1.942743in}{0.846175in}}%
\pgfpathcurveto{\pgfqpoint{1.942743in}{0.854411in}}{\pgfqpoint{1.939471in}{0.862311in}}{\pgfqpoint{1.933647in}{0.868135in}}%
\pgfpathcurveto{\pgfqpoint{1.927823in}{0.873959in}}{\pgfqpoint{1.919923in}{0.877231in}}{\pgfqpoint{1.911687in}{0.877231in}}%
\pgfpathcurveto{\pgfqpoint{1.903451in}{0.877231in}}{\pgfqpoint{1.895551in}{0.873959in}}{\pgfqpoint{1.889727in}{0.868135in}}%
\pgfpathcurveto{\pgfqpoint{1.883903in}{0.862311in}}{\pgfqpoint{1.880630in}{0.854411in}}{\pgfqpoint{1.880630in}{0.846175in}}%
\pgfpathcurveto{\pgfqpoint{1.880630in}{0.837939in}}{\pgfqpoint{1.883903in}{0.830039in}}{\pgfqpoint{1.889727in}{0.824215in}}%
\pgfpathcurveto{\pgfqpoint{1.895551in}{0.818391in}}{\pgfqpoint{1.903451in}{0.815118in}}{\pgfqpoint{1.911687in}{0.815118in}}%
\pgfpathlineto{\pgfqpoint{1.911687in}{0.815118in}}%
\pgfusepath{stroke,fill}%
\end{pgfscope}%
\begin{pgfscope}%
\pgfpathrectangle{\pgfqpoint{0.548058in}{0.516222in}}{\pgfqpoint{1.739582in}{1.783528in}} %
\pgfusepath{clip}%
\pgfsetbuttcap%
\pgfsetroundjoin%
\definecolor{currentfill}{rgb}{0.298039,0.447059,0.690196}%
\pgfsetfillcolor{currentfill}%
\pgfsetlinewidth{0.240900pt}%
\definecolor{currentstroke}{rgb}{1.000000,1.000000,1.000000}%
\pgfsetstrokecolor{currentstroke}%
\pgfsetdash{}{0pt}%
\pgfpathmoveto{\pgfqpoint{1.720524in}{0.993471in}}%
\pgfpathcurveto{\pgfqpoint{1.728760in}{0.993471in}}{\pgfqpoint{1.736660in}{0.996743in}}{\pgfqpoint{1.742484in}{1.002567in}}%
\pgfpathcurveto{\pgfqpoint{1.748308in}{1.008391in}}{\pgfqpoint{1.751580in}{1.016291in}}{\pgfqpoint{1.751580in}{1.024528in}}%
\pgfpathcurveto{\pgfqpoint{1.751580in}{1.032764in}}{\pgfqpoint{1.748308in}{1.040664in}}{\pgfqpoint{1.742484in}{1.046488in}}%
\pgfpathcurveto{\pgfqpoint{1.736660in}{1.052312in}}{\pgfqpoint{1.728760in}{1.055584in}}{\pgfqpoint{1.720524in}{1.055584in}}%
\pgfpathcurveto{\pgfqpoint{1.712288in}{1.055584in}}{\pgfqpoint{1.704388in}{1.052312in}}{\pgfqpoint{1.698564in}{1.046488in}}%
\pgfpathcurveto{\pgfqpoint{1.692740in}{1.040664in}}{\pgfqpoint{1.689467in}{1.032764in}}{\pgfqpoint{1.689467in}{1.024528in}}%
\pgfpathcurveto{\pgfqpoint{1.689467in}{1.016291in}}{\pgfqpoint{1.692740in}{1.008391in}}{\pgfqpoint{1.698564in}{1.002567in}}%
\pgfpathcurveto{\pgfqpoint{1.704388in}{0.996743in}}{\pgfqpoint{1.712288in}{0.993471in}}{\pgfqpoint{1.720524in}{0.993471in}}%
\pgfpathlineto{\pgfqpoint{1.720524in}{0.993471in}}%
\pgfusepath{stroke,fill}%
\end{pgfscope}%
\begin{pgfscope}%
\pgfpathrectangle{\pgfqpoint{0.548058in}{0.516222in}}{\pgfqpoint{1.739582in}{1.783528in}} %
\pgfusepath{clip}%
\pgfsetbuttcap%
\pgfsetroundjoin%
\definecolor{currentfill}{rgb}{0.298039,0.447059,0.690196}%
\pgfsetfillcolor{currentfill}%
\pgfsetlinewidth{0.240900pt}%
\definecolor{currentstroke}{rgb}{1.000000,1.000000,1.000000}%
\pgfsetstrokecolor{currentstroke}%
\pgfsetdash{}{0pt}%
\pgfpathmoveto{\pgfqpoint{1.465640in}{0.819577in}}%
\pgfpathcurveto{\pgfqpoint{1.473876in}{0.819577in}}{\pgfqpoint{1.481776in}{0.822849in}}{\pgfqpoint{1.487600in}{0.828673in}}%
\pgfpathcurveto{\pgfqpoint{1.493424in}{0.834497in}}{\pgfqpoint{1.496697in}{0.842397in}}{\pgfqpoint{1.496697in}{0.850634in}}%
\pgfpathcurveto{\pgfqpoint{1.496697in}{0.858870in}}{\pgfqpoint{1.493424in}{0.866770in}}{\pgfqpoint{1.487600in}{0.872594in}}%
\pgfpathcurveto{\pgfqpoint{1.481776in}{0.878418in}}{\pgfqpoint{1.473876in}{0.881690in}}{\pgfqpoint{1.465640in}{0.881690in}}%
\pgfpathcurveto{\pgfqpoint{1.457404in}{0.881690in}}{\pgfqpoint{1.449504in}{0.878418in}}{\pgfqpoint{1.443680in}{0.872594in}}%
\pgfpathcurveto{\pgfqpoint{1.437856in}{0.866770in}}{\pgfqpoint{1.434584in}{0.858870in}}{\pgfqpoint{1.434584in}{0.850634in}}%
\pgfpathcurveto{\pgfqpoint{1.434584in}{0.842397in}}{\pgfqpoint{1.437856in}{0.834497in}}{\pgfqpoint{1.443680in}{0.828673in}}%
\pgfpathcurveto{\pgfqpoint{1.449504in}{0.822849in}}{\pgfqpoint{1.457404in}{0.819577in}}{\pgfqpoint{1.465640in}{0.819577in}}%
\pgfpathlineto{\pgfqpoint{1.465640in}{0.819577in}}%
\pgfusepath{stroke,fill}%
\end{pgfscope}%
\begin{pgfscope}%
\pgfpathrectangle{\pgfqpoint{0.548058in}{0.516222in}}{\pgfqpoint{1.739582in}{1.783528in}} %
\pgfusepath{clip}%
\pgfsetbuttcap%
\pgfsetroundjoin%
\definecolor{currentfill}{rgb}{0.298039,0.447059,0.690196}%
\pgfsetfillcolor{currentfill}%
\pgfsetlinewidth{0.240900pt}%
\definecolor{currentstroke}{rgb}{1.000000,1.000000,1.000000}%
\pgfsetstrokecolor{currentstroke}%
\pgfsetdash{}{0pt}%
\pgfpathmoveto{\pgfqpoint{0.955872in}{1.390306in}}%
\pgfpathcurveto{\pgfqpoint{0.964109in}{1.390306in}}{\pgfqpoint{0.972009in}{1.393578in}}{\pgfqpoint{0.977833in}{1.399402in}}%
\pgfpathcurveto{\pgfqpoint{0.983657in}{1.405226in}}{\pgfqpoint{0.986929in}{1.413126in}}{\pgfqpoint{0.986929in}{1.421363in}}%
\pgfpathcurveto{\pgfqpoint{0.986929in}{1.429599in}}{\pgfqpoint{0.983657in}{1.437499in}}{\pgfqpoint{0.977833in}{1.443323in}}%
\pgfpathcurveto{\pgfqpoint{0.972009in}{1.449147in}}{\pgfqpoint{0.964109in}{1.452419in}}{\pgfqpoint{0.955872in}{1.452419in}}%
\pgfpathcurveto{\pgfqpoint{0.947636in}{1.452419in}}{\pgfqpoint{0.939736in}{1.449147in}}{\pgfqpoint{0.933912in}{1.443323in}}%
\pgfpathcurveto{\pgfqpoint{0.928088in}{1.437499in}}{\pgfqpoint{0.924816in}{1.429599in}}{\pgfqpoint{0.924816in}{1.421363in}}%
\pgfpathcurveto{\pgfqpoint{0.924816in}{1.413126in}}{\pgfqpoint{0.928088in}{1.405226in}}{\pgfqpoint{0.933912in}{1.399402in}}%
\pgfpathcurveto{\pgfqpoint{0.939736in}{1.393578in}}{\pgfqpoint{0.947636in}{1.390306in}}{\pgfqpoint{0.955872in}{1.390306in}}%
\pgfpathlineto{\pgfqpoint{0.955872in}{1.390306in}}%
\pgfusepath{stroke,fill}%
\end{pgfscope}%
\begin{pgfscope}%
\pgfpathrectangle{\pgfqpoint{0.548058in}{0.516222in}}{\pgfqpoint{1.739582in}{1.783528in}} %
\pgfusepath{clip}%
\pgfsetbuttcap%
\pgfsetroundjoin%
\definecolor{currentfill}{rgb}{0.298039,0.447059,0.690196}%
\pgfsetfillcolor{currentfill}%
\pgfsetlinewidth{0.240900pt}%
\definecolor{currentstroke}{rgb}{1.000000,1.000000,1.000000}%
\pgfsetstrokecolor{currentstroke}%
\pgfsetdash{}{0pt}%
\pgfpathmoveto{\pgfqpoint{1.306338in}{1.488400in}}%
\pgfpathcurveto{\pgfqpoint{1.314574in}{1.488400in}}{\pgfqpoint{1.322474in}{1.491672in}}{\pgfqpoint{1.328298in}{1.497496in}}%
\pgfpathcurveto{\pgfqpoint{1.334122in}{1.503320in}}{\pgfqpoint{1.337394in}{1.511220in}}{\pgfqpoint{1.337394in}{1.519457in}}%
\pgfpathcurveto{\pgfqpoint{1.337394in}{1.527693in}}{\pgfqpoint{1.334122in}{1.535593in}}{\pgfqpoint{1.328298in}{1.541417in}}%
\pgfpathcurveto{\pgfqpoint{1.322474in}{1.547241in}}{\pgfqpoint{1.314574in}{1.550513in}}{\pgfqpoint{1.306338in}{1.550513in}}%
\pgfpathcurveto{\pgfqpoint{1.298101in}{1.550513in}}{\pgfqpoint{1.290201in}{1.547241in}}{\pgfqpoint{1.284377in}{1.541417in}}%
\pgfpathcurveto{\pgfqpoint{1.278554in}{1.535593in}}{\pgfqpoint{1.275281in}{1.527693in}}{\pgfqpoint{1.275281in}{1.519457in}}%
\pgfpathcurveto{\pgfqpoint{1.275281in}{1.511220in}}{\pgfqpoint{1.278554in}{1.503320in}}{\pgfqpoint{1.284377in}{1.497496in}}%
\pgfpathcurveto{\pgfqpoint{1.290201in}{1.491672in}}{\pgfqpoint{1.298101in}{1.488400in}}{\pgfqpoint{1.306338in}{1.488400in}}%
\pgfpathlineto{\pgfqpoint{1.306338in}{1.488400in}}%
\pgfusepath{stroke,fill}%
\end{pgfscope}%
\begin{pgfscope}%
\pgfpathrectangle{\pgfqpoint{0.548058in}{0.516222in}}{\pgfqpoint{1.739582in}{1.783528in}} %
\pgfusepath{clip}%
\pgfsetbuttcap%
\pgfsetroundjoin%
\definecolor{currentfill}{rgb}{0.298039,0.447059,0.690196}%
\pgfsetfillcolor{currentfill}%
\pgfsetlinewidth{0.240900pt}%
\definecolor{currentstroke}{rgb}{1.000000,1.000000,1.000000}%
\pgfsetstrokecolor{currentstroke}%
\pgfsetdash{}{0pt}%
\pgfpathmoveto{\pgfqpoint{0.860291in}{1.457188in}}%
\pgfpathcurveto{\pgfqpoint{0.868527in}{1.457188in}}{\pgfqpoint{0.876427in}{1.460461in}}{\pgfqpoint{0.882251in}{1.466285in}}%
\pgfpathcurveto{\pgfqpoint{0.888075in}{1.472109in}}{\pgfqpoint{0.891347in}{1.480009in}}{\pgfqpoint{0.891347in}{1.488245in}}%
\pgfpathcurveto{\pgfqpoint{0.891347in}{1.496481in}}{\pgfqpoint{0.888075in}{1.504381in}}{\pgfqpoint{0.882251in}{1.510205in}}%
\pgfpathcurveto{\pgfqpoint{0.876427in}{1.516029in}}{\pgfqpoint{0.868527in}{1.519301in}}{\pgfqpoint{0.860291in}{1.519301in}}%
\pgfpathcurveto{\pgfqpoint{0.852055in}{1.519301in}}{\pgfqpoint{0.844155in}{1.516029in}}{\pgfqpoint{0.838331in}{1.510205in}}%
\pgfpathcurveto{\pgfqpoint{0.832507in}{1.504381in}}{\pgfqpoint{0.829234in}{1.496481in}}{\pgfqpoint{0.829234in}{1.488245in}}%
\pgfpathcurveto{\pgfqpoint{0.829234in}{1.480009in}}{\pgfqpoint{0.832507in}{1.472109in}}{\pgfqpoint{0.838331in}{1.466285in}}%
\pgfpathcurveto{\pgfqpoint{0.844155in}{1.460461in}}{\pgfqpoint{0.852055in}{1.457188in}}{\pgfqpoint{0.860291in}{1.457188in}}%
\pgfpathlineto{\pgfqpoint{0.860291in}{1.457188in}}%
\pgfusepath{stroke,fill}%
\end{pgfscope}%
\begin{pgfscope}%
\pgfpathrectangle{\pgfqpoint{0.548058in}{0.516222in}}{\pgfqpoint{1.739582in}{1.783528in}} %
\pgfusepath{clip}%
\pgfsetbuttcap%
\pgfsetroundjoin%
\definecolor{currentfill}{rgb}{0.298039,0.447059,0.690196}%
\pgfsetfillcolor{currentfill}%
\pgfsetlinewidth{0.240900pt}%
\definecolor{currentstroke}{rgb}{1.000000,1.000000,1.000000}%
\pgfsetstrokecolor{currentstroke}%
\pgfsetdash{}{0pt}%
\pgfpathmoveto{\pgfqpoint{1.752384in}{1.153989in}}%
\pgfpathcurveto{\pgfqpoint{1.760621in}{1.153989in}}{\pgfqpoint{1.768521in}{1.157261in}}{\pgfqpoint{1.774345in}{1.163085in}}%
\pgfpathcurveto{\pgfqpoint{1.780169in}{1.168909in}}{\pgfqpoint{1.783441in}{1.176809in}}{\pgfqpoint{1.783441in}{1.185045in}}%
\pgfpathcurveto{\pgfqpoint{1.783441in}{1.193281in}}{\pgfqpoint{1.780169in}{1.201181in}}{\pgfqpoint{1.774345in}{1.207005in}}%
\pgfpathcurveto{\pgfqpoint{1.768521in}{1.212829in}}{\pgfqpoint{1.760621in}{1.216102in}}{\pgfqpoint{1.752384in}{1.216102in}}%
\pgfpathcurveto{\pgfqpoint{1.744148in}{1.216102in}}{\pgfqpoint{1.736248in}{1.212829in}}{\pgfqpoint{1.730424in}{1.207005in}}%
\pgfpathcurveto{\pgfqpoint{1.724600in}{1.201181in}}{\pgfqpoint{1.721328in}{1.193281in}}{\pgfqpoint{1.721328in}{1.185045in}}%
\pgfpathcurveto{\pgfqpoint{1.721328in}{1.176809in}}{\pgfqpoint{1.724600in}{1.168909in}}{\pgfqpoint{1.730424in}{1.163085in}}%
\pgfpathcurveto{\pgfqpoint{1.736248in}{1.157261in}}{\pgfqpoint{1.744148in}{1.153989in}}{\pgfqpoint{1.752384in}{1.153989in}}%
\pgfpathlineto{\pgfqpoint{1.752384in}{1.153989in}}%
\pgfusepath{stroke,fill}%
\end{pgfscope}%
\begin{pgfscope}%
\pgfpathrectangle{\pgfqpoint{0.548058in}{0.516222in}}{\pgfqpoint{1.739582in}{1.783528in}} %
\pgfusepath{clip}%
\pgfsetbuttcap%
\pgfsetroundjoin%
\definecolor{currentfill}{rgb}{0.298039,0.447059,0.690196}%
\pgfsetfillcolor{currentfill}%
\pgfsetlinewidth{0.240900pt}%
\definecolor{currentstroke}{rgb}{1.000000,1.000000,1.000000}%
\pgfsetstrokecolor{currentstroke}%
\pgfsetdash{}{0pt}%
\pgfpathmoveto{\pgfqpoint{1.147035in}{1.368012in}}%
\pgfpathcurveto{\pgfqpoint{1.155272in}{1.368012in}}{\pgfqpoint{1.163172in}{1.371284in}}{\pgfqpoint{1.168996in}{1.377108in}}%
\pgfpathcurveto{\pgfqpoint{1.174819in}{1.382932in}}{\pgfqpoint{1.178092in}{1.390832in}}{\pgfqpoint{1.178092in}{1.399068in}}%
\pgfpathcurveto{\pgfqpoint{1.178092in}{1.407305in}}{\pgfqpoint{1.174819in}{1.415205in}}{\pgfqpoint{1.168996in}{1.421029in}}%
\pgfpathcurveto{\pgfqpoint{1.163172in}{1.426853in}}{\pgfqpoint{1.155272in}{1.430125in}}{\pgfqpoint{1.147035in}{1.430125in}}%
\pgfpathcurveto{\pgfqpoint{1.138799in}{1.430125in}}{\pgfqpoint{1.130899in}{1.426853in}}{\pgfqpoint{1.125075in}{1.421029in}}%
\pgfpathcurveto{\pgfqpoint{1.119251in}{1.415205in}}{\pgfqpoint{1.115979in}{1.407305in}}{\pgfqpoint{1.115979in}{1.399068in}}%
\pgfpathcurveto{\pgfqpoint{1.115979in}{1.390832in}}{\pgfqpoint{1.119251in}{1.382932in}}{\pgfqpoint{1.125075in}{1.377108in}}%
\pgfpathcurveto{\pgfqpoint{1.130899in}{1.371284in}}{\pgfqpoint{1.138799in}{1.368012in}}{\pgfqpoint{1.147035in}{1.368012in}}%
\pgfpathlineto{\pgfqpoint{1.147035in}{1.368012in}}%
\pgfusepath{stroke,fill}%
\end{pgfscope}%
\begin{pgfscope}%
\pgfpathrectangle{\pgfqpoint{0.548058in}{0.516222in}}{\pgfqpoint{1.739582in}{1.783528in}} %
\pgfusepath{clip}%
\pgfsetbuttcap%
\pgfsetroundjoin%
\definecolor{currentfill}{rgb}{0.298039,0.447059,0.690196}%
\pgfsetfillcolor{currentfill}%
\pgfsetlinewidth{0.240900pt}%
\definecolor{currentstroke}{rgb}{1.000000,1.000000,1.000000}%
\pgfsetstrokecolor{currentstroke}%
\pgfsetdash{}{0pt}%
\pgfpathmoveto{\pgfqpoint{1.178896in}{1.359094in}}%
\pgfpathcurveto{\pgfqpoint{1.187132in}{1.359094in}}{\pgfqpoint{1.195032in}{1.362367in}}{\pgfqpoint{1.200856in}{1.368191in}}%
\pgfpathcurveto{\pgfqpoint{1.206680in}{1.374015in}}{\pgfqpoint{1.209952in}{1.381915in}}{\pgfqpoint{1.209952in}{1.390151in}}%
\pgfpathcurveto{\pgfqpoint{1.209952in}{1.398387in}}{\pgfqpoint{1.206680in}{1.406287in}}{\pgfqpoint{1.200856in}{1.412111in}}%
\pgfpathcurveto{\pgfqpoint{1.195032in}{1.417935in}}{\pgfqpoint{1.187132in}{1.421207in}}{\pgfqpoint{1.178896in}{1.421207in}}%
\pgfpathcurveto{\pgfqpoint{1.170659in}{1.421207in}}{\pgfqpoint{1.162759in}{1.417935in}}{\pgfqpoint{1.156935in}{1.412111in}}%
\pgfpathcurveto{\pgfqpoint{1.151112in}{1.406287in}}{\pgfqpoint{1.147839in}{1.398387in}}{\pgfqpoint{1.147839in}{1.390151in}}%
\pgfpathcurveto{\pgfqpoint{1.147839in}{1.381915in}}{\pgfqpoint{1.151112in}{1.374015in}}{\pgfqpoint{1.156935in}{1.368191in}}%
\pgfpathcurveto{\pgfqpoint{1.162759in}{1.362367in}}{\pgfqpoint{1.170659in}{1.359094in}}{\pgfqpoint{1.178896in}{1.359094in}}%
\pgfpathlineto{\pgfqpoint{1.178896in}{1.359094in}}%
\pgfusepath{stroke,fill}%
\end{pgfscope}%
\begin{pgfscope}%
\pgfpathrectangle{\pgfqpoint{0.548058in}{0.516222in}}{\pgfqpoint{1.739582in}{1.783528in}} %
\pgfusepath{clip}%
\pgfsetbuttcap%
\pgfsetroundjoin%
\definecolor{currentfill}{rgb}{0.298039,0.447059,0.690196}%
\pgfsetfillcolor{currentfill}%
\pgfsetlinewidth{0.240900pt}%
\definecolor{currentstroke}{rgb}{1.000000,1.000000,1.000000}%
\pgfsetstrokecolor{currentstroke}%
\pgfsetdash{}{0pt}%
\pgfpathmoveto{\pgfqpoint{2.007268in}{1.278836in}}%
\pgfpathcurveto{\pgfqpoint{2.015505in}{1.278836in}}{\pgfqpoint{2.023405in}{1.282108in}}{\pgfqpoint{2.029229in}{1.287932in}}%
\pgfpathcurveto{\pgfqpoint{2.035053in}{1.293756in}}{\pgfqpoint{2.038325in}{1.301656in}}{\pgfqpoint{2.038325in}{1.309892in}}%
\pgfpathcurveto{\pgfqpoint{2.038325in}{1.318128in}}{\pgfqpoint{2.035053in}{1.326028in}}{\pgfqpoint{2.029229in}{1.331852in}}%
\pgfpathcurveto{\pgfqpoint{2.023405in}{1.337676in}}{\pgfqpoint{2.015505in}{1.340949in}}{\pgfqpoint{2.007268in}{1.340949in}}%
\pgfpathcurveto{\pgfqpoint{1.999032in}{1.340949in}}{\pgfqpoint{1.991132in}{1.337676in}}{\pgfqpoint{1.985308in}{1.331852in}}%
\pgfpathcurveto{\pgfqpoint{1.979484in}{1.326028in}}{\pgfqpoint{1.976212in}{1.318128in}}{\pgfqpoint{1.976212in}{1.309892in}}%
\pgfpathcurveto{\pgfqpoint{1.976212in}{1.301656in}}{\pgfqpoint{1.979484in}{1.293756in}}{\pgfqpoint{1.985308in}{1.287932in}}%
\pgfpathcurveto{\pgfqpoint{1.991132in}{1.282108in}}{\pgfqpoint{1.999032in}{1.278836in}}{\pgfqpoint{2.007268in}{1.278836in}}%
\pgfpathlineto{\pgfqpoint{2.007268in}{1.278836in}}%
\pgfusepath{stroke,fill}%
\end{pgfscope}%
\begin{pgfscope}%
\pgfpathrectangle{\pgfqpoint{0.548058in}{0.516222in}}{\pgfqpoint{1.739582in}{1.783528in}} %
\pgfusepath{clip}%
\pgfsetbuttcap%
\pgfsetroundjoin%
\definecolor{currentfill}{rgb}{0.298039,0.447059,0.690196}%
\pgfsetfillcolor{currentfill}%
\pgfsetlinewidth{0.240900pt}%
\definecolor{currentstroke}{rgb}{1.000000,1.000000,1.000000}%
\pgfsetstrokecolor{currentstroke}%
\pgfsetdash{}{0pt}%
\pgfpathmoveto{\pgfqpoint{1.784245in}{0.931048in}}%
\pgfpathcurveto{\pgfqpoint{1.792481in}{0.931048in}}{\pgfqpoint{1.800381in}{0.934320in}}{\pgfqpoint{1.806205in}{0.940144in}}%
\pgfpathcurveto{\pgfqpoint{1.812029in}{0.945968in}}{\pgfqpoint{1.815301in}{0.953868in}}{\pgfqpoint{1.815301in}{0.962104in}}%
\pgfpathcurveto{\pgfqpoint{1.815301in}{0.970340in}}{\pgfqpoint{1.812029in}{0.978240in}}{\pgfqpoint{1.806205in}{0.984064in}}%
\pgfpathcurveto{\pgfqpoint{1.800381in}{0.989888in}}{\pgfqpoint{1.792481in}{0.993161in}}{\pgfqpoint{1.784245in}{0.993161in}}%
\pgfpathcurveto{\pgfqpoint{1.776009in}{0.993161in}}{\pgfqpoint{1.768109in}{0.989888in}}{\pgfqpoint{1.762285in}{0.984064in}}%
\pgfpathcurveto{\pgfqpoint{1.756461in}{0.978240in}}{\pgfqpoint{1.753188in}{0.970340in}}{\pgfqpoint{1.753188in}{0.962104in}}%
\pgfpathcurveto{\pgfqpoint{1.753188in}{0.953868in}}{\pgfqpoint{1.756461in}{0.945968in}}{\pgfqpoint{1.762285in}{0.940144in}}%
\pgfpathcurveto{\pgfqpoint{1.768109in}{0.934320in}}{\pgfqpoint{1.776009in}{0.931048in}}{\pgfqpoint{1.784245in}{0.931048in}}%
\pgfpathlineto{\pgfqpoint{1.784245in}{0.931048in}}%
\pgfusepath{stroke,fill}%
\end{pgfscope}%
\begin{pgfscope}%
\pgfpathrectangle{\pgfqpoint{0.548058in}{0.516222in}}{\pgfqpoint{1.739582in}{1.783528in}} %
\pgfusepath{clip}%
\pgfsetbuttcap%
\pgfsetroundjoin%
\definecolor{currentfill}{rgb}{0.298039,0.447059,0.690196}%
\pgfsetfillcolor{currentfill}%
\pgfsetlinewidth{0.240900pt}%
\definecolor{currentstroke}{rgb}{1.000000,1.000000,1.000000}%
\pgfsetstrokecolor{currentstroke}%
\pgfsetdash{}{0pt}%
\pgfpathmoveto{\pgfqpoint{0.924012in}{2.090341in}}%
\pgfpathcurveto{\pgfqpoint{0.932248in}{2.090341in}}{\pgfqpoint{0.940148in}{2.093613in}}{\pgfqpoint{0.945972in}{2.099437in}}%
\pgfpathcurveto{\pgfqpoint{0.951796in}{2.105261in}}{\pgfqpoint{0.955068in}{2.113161in}}{\pgfqpoint{0.955068in}{2.121397in}}%
\pgfpathcurveto{\pgfqpoint{0.955068in}{2.129634in}}{\pgfqpoint{0.951796in}{2.137534in}}{\pgfqpoint{0.945972in}{2.143357in}}%
\pgfpathcurveto{\pgfqpoint{0.940148in}{2.149181in}}{\pgfqpoint{0.932248in}{2.152454in}}{\pgfqpoint{0.924012in}{2.152454in}}%
\pgfpathcurveto{\pgfqpoint{0.915776in}{2.152454in}}{\pgfqpoint{0.907876in}{2.149181in}}{\pgfqpoint{0.902052in}{2.143357in}}%
\pgfpathcurveto{\pgfqpoint{0.896228in}{2.137534in}}{\pgfqpoint{0.892955in}{2.129634in}}{\pgfqpoint{0.892955in}{2.121397in}}%
\pgfpathcurveto{\pgfqpoint{0.892955in}{2.113161in}}{\pgfqpoint{0.896228in}{2.105261in}}{\pgfqpoint{0.902052in}{2.099437in}}%
\pgfpathcurveto{\pgfqpoint{0.907876in}{2.093613in}}{\pgfqpoint{0.915776in}{2.090341in}}{\pgfqpoint{0.924012in}{2.090341in}}%
\pgfpathlineto{\pgfqpoint{0.924012in}{2.090341in}}%
\pgfusepath{stroke,fill}%
\end{pgfscope}%
\begin{pgfscope}%
\pgfpathrectangle{\pgfqpoint{0.548058in}{0.516222in}}{\pgfqpoint{1.739582in}{1.783528in}} %
\pgfusepath{clip}%
\pgfsetbuttcap%
\pgfsetroundjoin%
\definecolor{currentfill}{rgb}{0.298039,0.447059,0.690196}%
\pgfsetfillcolor{currentfill}%
\pgfsetlinewidth{0.240900pt}%
\definecolor{currentstroke}{rgb}{1.000000,1.000000,1.000000}%
\pgfsetstrokecolor{currentstroke}%
\pgfsetdash{}{0pt}%
\pgfpathmoveto{\pgfqpoint{1.975408in}{0.810660in}}%
\pgfpathcurveto{\pgfqpoint{1.983644in}{0.810660in}}{\pgfqpoint{1.991544in}{0.813932in}}{\pgfqpoint{1.997368in}{0.819756in}}%
\pgfpathcurveto{\pgfqpoint{2.003192in}{0.825580in}}{\pgfqpoint{2.006464in}{0.833480in}}{\pgfqpoint{2.006464in}{0.841716in}}%
\pgfpathcurveto{\pgfqpoint{2.006464in}{0.849952in}}{\pgfqpoint{2.003192in}{0.857852in}}{\pgfqpoint{1.997368in}{0.863676in}}%
\pgfpathcurveto{\pgfqpoint{1.991544in}{0.869500in}}{\pgfqpoint{1.983644in}{0.872773in}}{\pgfqpoint{1.975408in}{0.872773in}}%
\pgfpathcurveto{\pgfqpoint{1.967172in}{0.872773in}}{\pgfqpoint{1.959272in}{0.869500in}}{\pgfqpoint{1.953448in}{0.863676in}}%
\pgfpathcurveto{\pgfqpoint{1.947624in}{0.857852in}}{\pgfqpoint{1.944351in}{0.849952in}}{\pgfqpoint{1.944351in}{0.841716in}}%
\pgfpathcurveto{\pgfqpoint{1.944351in}{0.833480in}}{\pgfqpoint{1.947624in}{0.825580in}}{\pgfqpoint{1.953448in}{0.819756in}}%
\pgfpathcurveto{\pgfqpoint{1.959272in}{0.813932in}}{\pgfqpoint{1.967172in}{0.810660in}}{\pgfqpoint{1.975408in}{0.810660in}}%
\pgfpathlineto{\pgfqpoint{1.975408in}{0.810660in}}%
\pgfusepath{stroke,fill}%
\end{pgfscope}%
\begin{pgfscope}%
\pgfpathrectangle{\pgfqpoint{0.548058in}{0.516222in}}{\pgfqpoint{1.739582in}{1.783528in}} %
\pgfusepath{clip}%
\pgfsetbuttcap%
\pgfsetroundjoin%
\definecolor{currentfill}{rgb}{0.298039,0.447059,0.690196}%
\pgfsetfillcolor{currentfill}%
\pgfsetlinewidth{0.240900pt}%
\definecolor{currentstroke}{rgb}{1.000000,1.000000,1.000000}%
\pgfsetstrokecolor{currentstroke}%
\pgfsetdash{}{0pt}%
\pgfpathmoveto{\pgfqpoint{1.624943in}{1.265459in}}%
\pgfpathcurveto{\pgfqpoint{1.633179in}{1.265459in}}{\pgfqpoint{1.641079in}{1.268731in}}{\pgfqpoint{1.646903in}{1.274555in}}%
\pgfpathcurveto{\pgfqpoint{1.652727in}{1.280379in}}{\pgfqpoint{1.655999in}{1.288279in}}{\pgfqpoint{1.655999in}{1.296516in}}%
\pgfpathcurveto{\pgfqpoint{1.655999in}{1.304752in}}{\pgfqpoint{1.652727in}{1.312652in}}{\pgfqpoint{1.646903in}{1.318476in}}%
\pgfpathcurveto{\pgfqpoint{1.641079in}{1.324300in}}{\pgfqpoint{1.633179in}{1.327572in}}{\pgfqpoint{1.624943in}{1.327572in}}%
\pgfpathcurveto{\pgfqpoint{1.616706in}{1.327572in}}{\pgfqpoint{1.608806in}{1.324300in}}{\pgfqpoint{1.602982in}{1.318476in}}%
\pgfpathcurveto{\pgfqpoint{1.597158in}{1.312652in}}{\pgfqpoint{1.593886in}{1.304752in}}{\pgfqpoint{1.593886in}{1.296516in}}%
\pgfpathcurveto{\pgfqpoint{1.593886in}{1.288279in}}{\pgfqpoint{1.597158in}{1.280379in}}{\pgfqpoint{1.602982in}{1.274555in}}%
\pgfpathcurveto{\pgfqpoint{1.608806in}{1.268731in}}{\pgfqpoint{1.616706in}{1.265459in}}{\pgfqpoint{1.624943in}{1.265459in}}%
\pgfpathlineto{\pgfqpoint{1.624943in}{1.265459in}}%
\pgfusepath{stroke,fill}%
\end{pgfscope}%
\begin{pgfscope}%
\pgfpathrectangle{\pgfqpoint{0.548058in}{0.516222in}}{\pgfqpoint{1.739582in}{1.783528in}} %
\pgfusepath{clip}%
\pgfsetbuttcap%
\pgfsetroundjoin%
\definecolor{currentfill}{rgb}{0.298039,0.447059,0.690196}%
\pgfsetfillcolor{currentfill}%
\pgfsetlinewidth{0.240900pt}%
\definecolor{currentstroke}{rgb}{1.000000,1.000000,1.000000}%
\pgfsetstrokecolor{currentstroke}%
\pgfsetdash{}{0pt}%
\pgfpathmoveto{\pgfqpoint{1.879826in}{1.211953in}}%
\pgfpathcurveto{\pgfqpoint{1.888063in}{1.211953in}}{\pgfqpoint{1.895963in}{1.215226in}}{\pgfqpoint{1.901787in}{1.221050in}}%
\pgfpathcurveto{\pgfqpoint{1.907611in}{1.226873in}}{\pgfqpoint{1.910883in}{1.234774in}}{\pgfqpoint{1.910883in}{1.243010in}}%
\pgfpathcurveto{\pgfqpoint{1.910883in}{1.251246in}}{\pgfqpoint{1.907611in}{1.259146in}}{\pgfqpoint{1.901787in}{1.264970in}}%
\pgfpathcurveto{\pgfqpoint{1.895963in}{1.270794in}}{\pgfqpoint{1.888063in}{1.274066in}}{\pgfqpoint{1.879826in}{1.274066in}}%
\pgfpathcurveto{\pgfqpoint{1.871590in}{1.274066in}}{\pgfqpoint{1.863690in}{1.270794in}}{\pgfqpoint{1.857866in}{1.264970in}}%
\pgfpathcurveto{\pgfqpoint{1.852042in}{1.259146in}}{\pgfqpoint{1.848770in}{1.251246in}}{\pgfqpoint{1.848770in}{1.243010in}}%
\pgfpathcurveto{\pgfqpoint{1.848770in}{1.234774in}}{\pgfqpoint{1.852042in}{1.226873in}}{\pgfqpoint{1.857866in}{1.221050in}}%
\pgfpathcurveto{\pgfqpoint{1.863690in}{1.215226in}}{\pgfqpoint{1.871590in}{1.211953in}}{\pgfqpoint{1.879826in}{1.211953in}}%
\pgfpathlineto{\pgfqpoint{1.879826in}{1.211953in}}%
\pgfusepath{stroke,fill}%
\end{pgfscope}%
\begin{pgfscope}%
\pgfpathrectangle{\pgfqpoint{0.548058in}{0.516222in}}{\pgfqpoint{1.739582in}{1.783528in}} %
\pgfusepath{clip}%
\pgfsetbuttcap%
\pgfsetroundjoin%
\definecolor{currentfill}{rgb}{0.298039,0.447059,0.690196}%
\pgfsetfillcolor{currentfill}%
\pgfsetlinewidth{0.240900pt}%
\definecolor{currentstroke}{rgb}{1.000000,1.000000,1.000000}%
\pgfsetstrokecolor{currentstroke}%
\pgfsetdash{}{0pt}%
\pgfpathmoveto{\pgfqpoint{1.593082in}{1.559741in}}%
\pgfpathcurveto{\pgfqpoint{1.601318in}{1.559741in}}{\pgfqpoint{1.609218in}{1.563014in}}{\pgfqpoint{1.615042in}{1.568837in}}%
\pgfpathcurveto{\pgfqpoint{1.620866in}{1.574661in}}{\pgfqpoint{1.624139in}{1.582561in}}{\pgfqpoint{1.624139in}{1.590798in}}%
\pgfpathcurveto{\pgfqpoint{1.624139in}{1.599034in}}{\pgfqpoint{1.620866in}{1.606934in}}{\pgfqpoint{1.615042in}{1.612758in}}%
\pgfpathcurveto{\pgfqpoint{1.609218in}{1.618582in}}{\pgfqpoint{1.601318in}{1.621854in}}{\pgfqpoint{1.593082in}{1.621854in}}%
\pgfpathcurveto{\pgfqpoint{1.584846in}{1.621854in}}{\pgfqpoint{1.576946in}{1.618582in}}{\pgfqpoint{1.571122in}{1.612758in}}%
\pgfpathcurveto{\pgfqpoint{1.565298in}{1.606934in}}{\pgfqpoint{1.562026in}{1.599034in}}{\pgfqpoint{1.562026in}{1.590798in}}%
\pgfpathcurveto{\pgfqpoint{1.562026in}{1.582561in}}{\pgfqpoint{1.565298in}{1.574661in}}{\pgfqpoint{1.571122in}{1.568837in}}%
\pgfpathcurveto{\pgfqpoint{1.576946in}{1.563014in}}{\pgfqpoint{1.584846in}{1.559741in}}{\pgfqpoint{1.593082in}{1.559741in}}%
\pgfpathlineto{\pgfqpoint{1.593082in}{1.559741in}}%
\pgfusepath{stroke,fill}%
\end{pgfscope}%
\begin{pgfscope}%
\pgfpathrectangle{\pgfqpoint{0.548058in}{0.516222in}}{\pgfqpoint{1.739582in}{1.783528in}} %
\pgfusepath{clip}%
\pgfsetbuttcap%
\pgfsetroundjoin%
\definecolor{currentfill}{rgb}{0.298039,0.447059,0.690196}%
\pgfsetfillcolor{currentfill}%
\pgfsetlinewidth{0.240900pt}%
\definecolor{currentstroke}{rgb}{1.000000,1.000000,1.000000}%
\pgfsetstrokecolor{currentstroke}%
\pgfsetdash{}{0pt}%
\pgfpathmoveto{\pgfqpoint{1.529361in}{0.939965in}}%
\pgfpathcurveto{\pgfqpoint{1.537597in}{0.939965in}}{\pgfqpoint{1.545497in}{0.943238in}}{\pgfqpoint{1.551321in}{0.949062in}}%
\pgfpathcurveto{\pgfqpoint{1.557145in}{0.954885in}}{\pgfqpoint{1.560418in}{0.962786in}}{\pgfqpoint{1.560418in}{0.971022in}}%
\pgfpathcurveto{\pgfqpoint{1.560418in}{0.979258in}}{\pgfqpoint{1.557145in}{0.987158in}}{\pgfqpoint{1.551321in}{0.992982in}}%
\pgfpathcurveto{\pgfqpoint{1.545497in}{0.998806in}}{\pgfqpoint{1.537597in}{1.002078in}}{\pgfqpoint{1.529361in}{1.002078in}}%
\pgfpathcurveto{\pgfqpoint{1.521125in}{1.002078in}}{\pgfqpoint{1.513225in}{0.998806in}}{\pgfqpoint{1.507401in}{0.992982in}}%
\pgfpathcurveto{\pgfqpoint{1.501577in}{0.987158in}}{\pgfqpoint{1.498305in}{0.979258in}}{\pgfqpoint{1.498305in}{0.971022in}}%
\pgfpathcurveto{\pgfqpoint{1.498305in}{0.962786in}}{\pgfqpoint{1.501577in}{0.954885in}}{\pgfqpoint{1.507401in}{0.949062in}}%
\pgfpathcurveto{\pgfqpoint{1.513225in}{0.943238in}}{\pgfqpoint{1.521125in}{0.939965in}}{\pgfqpoint{1.529361in}{0.939965in}}%
\pgfpathlineto{\pgfqpoint{1.529361in}{0.939965in}}%
\pgfusepath{stroke,fill}%
\end{pgfscope}%
\begin{pgfscope}%
\pgfpathrectangle{\pgfqpoint{0.548058in}{0.516222in}}{\pgfqpoint{1.739582in}{1.783528in}} %
\pgfusepath{clip}%
\pgfsetbuttcap%
\pgfsetroundjoin%
\definecolor{currentfill}{rgb}{0.298039,0.447059,0.690196}%
\pgfsetfillcolor{currentfill}%
\pgfsetlinewidth{0.240900pt}%
\definecolor{currentstroke}{rgb}{1.000000,1.000000,1.000000}%
\pgfsetstrokecolor{currentstroke}%
\pgfsetdash{}{0pt}%
\pgfpathmoveto{\pgfqpoint{1.115175in}{1.158447in}}%
\pgfpathcurveto{\pgfqpoint{1.123411in}{1.158447in}}{\pgfqpoint{1.131311in}{1.161720in}}{\pgfqpoint{1.137135in}{1.167544in}}%
\pgfpathcurveto{\pgfqpoint{1.142959in}{1.173368in}}{\pgfqpoint{1.146231in}{1.181268in}}{\pgfqpoint{1.146231in}{1.189504in}}%
\pgfpathcurveto{\pgfqpoint{1.146231in}{1.197740in}}{\pgfqpoint{1.142959in}{1.205640in}}{\pgfqpoint{1.137135in}{1.211464in}}%
\pgfpathcurveto{\pgfqpoint{1.131311in}{1.217288in}}{\pgfqpoint{1.123411in}{1.220560in}}{\pgfqpoint{1.115175in}{1.220560in}}%
\pgfpathcurveto{\pgfqpoint{1.106939in}{1.220560in}}{\pgfqpoint{1.099038in}{1.217288in}}{\pgfqpoint{1.093215in}{1.211464in}}%
\pgfpathcurveto{\pgfqpoint{1.087391in}{1.205640in}}{\pgfqpoint{1.084118in}{1.197740in}}{\pgfqpoint{1.084118in}{1.189504in}}%
\pgfpathcurveto{\pgfqpoint{1.084118in}{1.181268in}}{\pgfqpoint{1.087391in}{1.173368in}}{\pgfqpoint{1.093215in}{1.167544in}}%
\pgfpathcurveto{\pgfqpoint{1.099038in}{1.161720in}}{\pgfqpoint{1.106939in}{1.158447in}}{\pgfqpoint{1.115175in}{1.158447in}}%
\pgfpathlineto{\pgfqpoint{1.115175in}{1.158447in}}%
\pgfusepath{stroke,fill}%
\end{pgfscope}%
\begin{pgfscope}%
\pgfpathrectangle{\pgfqpoint{0.548058in}{0.516222in}}{\pgfqpoint{1.739582in}{1.783528in}} %
\pgfusepath{clip}%
\pgfsetbuttcap%
\pgfsetroundjoin%
\definecolor{currentfill}{rgb}{0.298039,0.447059,0.690196}%
\pgfsetfillcolor{currentfill}%
\pgfsetlinewidth{0.240900pt}%
\definecolor{currentstroke}{rgb}{1.000000,1.000000,1.000000}%
\pgfsetstrokecolor{currentstroke}%
\pgfsetdash{}{0pt}%
\pgfpathmoveto{\pgfqpoint{1.274477in}{1.265459in}}%
\pgfpathcurveto{\pgfqpoint{1.282713in}{1.265459in}}{\pgfqpoint{1.290614in}{1.268731in}}{\pgfqpoint{1.296437in}{1.274555in}}%
\pgfpathcurveto{\pgfqpoint{1.302261in}{1.280379in}}{\pgfqpoint{1.305534in}{1.288279in}}{\pgfqpoint{1.305534in}{1.296516in}}%
\pgfpathcurveto{\pgfqpoint{1.305534in}{1.304752in}}{\pgfqpoint{1.302261in}{1.312652in}}{\pgfqpoint{1.296437in}{1.318476in}}%
\pgfpathcurveto{\pgfqpoint{1.290614in}{1.324300in}}{\pgfqpoint{1.282713in}{1.327572in}}{\pgfqpoint{1.274477in}{1.327572in}}%
\pgfpathcurveto{\pgfqpoint{1.266241in}{1.327572in}}{\pgfqpoint{1.258341in}{1.324300in}}{\pgfqpoint{1.252517in}{1.318476in}}%
\pgfpathcurveto{\pgfqpoint{1.246693in}{1.312652in}}{\pgfqpoint{1.243421in}{1.304752in}}{\pgfqpoint{1.243421in}{1.296516in}}%
\pgfpathcurveto{\pgfqpoint{1.243421in}{1.288279in}}{\pgfqpoint{1.246693in}{1.280379in}}{\pgfqpoint{1.252517in}{1.274555in}}%
\pgfpathcurveto{\pgfqpoint{1.258341in}{1.268731in}}{\pgfqpoint{1.266241in}{1.265459in}}{\pgfqpoint{1.274477in}{1.265459in}}%
\pgfpathlineto{\pgfqpoint{1.274477in}{1.265459in}}%
\pgfusepath{stroke,fill}%
\end{pgfscope}%
\begin{pgfscope}%
\pgfpathrectangle{\pgfqpoint{0.548058in}{0.516222in}}{\pgfqpoint{1.739582in}{1.783528in}} %
\pgfusepath{clip}%
\pgfsetbuttcap%
\pgfsetroundjoin%
\definecolor{currentfill}{rgb}{0.298039,0.447059,0.690196}%
\pgfsetfillcolor{currentfill}%
\pgfsetlinewidth{0.240900pt}%
\definecolor{currentstroke}{rgb}{1.000000,1.000000,1.000000}%
\pgfsetstrokecolor{currentstroke}%
\pgfsetdash{}{0pt}%
\pgfpathmoveto{\pgfqpoint{1.561222in}{1.920906in}}%
\pgfpathcurveto{\pgfqpoint{1.569458in}{1.920906in}}{\pgfqpoint{1.577358in}{1.924178in}}{\pgfqpoint{1.583182in}{1.930002in}}%
\pgfpathcurveto{\pgfqpoint{1.589006in}{1.935826in}}{\pgfqpoint{1.592278in}{1.943726in}}{\pgfqpoint{1.592278in}{1.951962in}}%
\pgfpathcurveto{\pgfqpoint{1.592278in}{1.960198in}}{\pgfqpoint{1.589006in}{1.968098in}}{\pgfqpoint{1.583182in}{1.973922in}}%
\pgfpathcurveto{\pgfqpoint{1.577358in}{1.979746in}}{\pgfqpoint{1.569458in}{1.983019in}}{\pgfqpoint{1.561222in}{1.983019in}}%
\pgfpathcurveto{\pgfqpoint{1.552985in}{1.983019in}}{\pgfqpoint{1.545085in}{1.979746in}}{\pgfqpoint{1.539261in}{1.973922in}}%
\pgfpathcurveto{\pgfqpoint{1.533437in}{1.968098in}}{\pgfqpoint{1.530165in}{1.960198in}}{\pgfqpoint{1.530165in}{1.951962in}}%
\pgfpathcurveto{\pgfqpoint{1.530165in}{1.943726in}}{\pgfqpoint{1.533437in}{1.935826in}}{\pgfqpoint{1.539261in}{1.930002in}}%
\pgfpathcurveto{\pgfqpoint{1.545085in}{1.924178in}}{\pgfqpoint{1.552985in}{1.920906in}}{\pgfqpoint{1.561222in}{1.920906in}}%
\pgfpathlineto{\pgfqpoint{1.561222in}{1.920906in}}%
\pgfusepath{stroke,fill}%
\end{pgfscope}%
\begin{pgfscope}%
\pgfpathrectangle{\pgfqpoint{0.548058in}{0.516222in}}{\pgfqpoint{1.739582in}{1.783528in}} %
\pgfusepath{clip}%
\pgfsetbuttcap%
\pgfsetroundjoin%
\definecolor{currentfill}{rgb}{0.298039,0.447059,0.690196}%
\pgfsetfillcolor{currentfill}%
\pgfsetlinewidth{0.240900pt}%
\definecolor{currentstroke}{rgb}{1.000000,1.000000,1.000000}%
\pgfsetstrokecolor{currentstroke}%
\pgfsetdash{}{0pt}%
\pgfpathmoveto{\pgfqpoint{0.892151in}{1.051436in}}%
\pgfpathcurveto{\pgfqpoint{0.900388in}{1.051436in}}{\pgfqpoint{0.908288in}{1.054708in}}{\pgfqpoint{0.914112in}{1.060532in}}%
\pgfpathcurveto{\pgfqpoint{0.919936in}{1.066356in}}{\pgfqpoint{0.923208in}{1.074256in}}{\pgfqpoint{0.923208in}{1.082492in}}%
\pgfpathcurveto{\pgfqpoint{0.923208in}{1.090729in}}{\pgfqpoint{0.919936in}{1.098629in}}{\pgfqpoint{0.914112in}{1.104453in}}%
\pgfpathcurveto{\pgfqpoint{0.908288in}{1.110276in}}{\pgfqpoint{0.900388in}{1.113549in}}{\pgfqpoint{0.892151in}{1.113549in}}%
\pgfpathcurveto{\pgfqpoint{0.883915in}{1.113549in}}{\pgfqpoint{0.876015in}{1.110276in}}{\pgfqpoint{0.870191in}{1.104453in}}%
\pgfpathcurveto{\pgfqpoint{0.864367in}{1.098629in}}{\pgfqpoint{0.861095in}{1.090729in}}{\pgfqpoint{0.861095in}{1.082492in}}%
\pgfpathcurveto{\pgfqpoint{0.861095in}{1.074256in}}{\pgfqpoint{0.864367in}{1.066356in}}{\pgfqpoint{0.870191in}{1.060532in}}%
\pgfpathcurveto{\pgfqpoint{0.876015in}{1.054708in}}{\pgfqpoint{0.883915in}{1.051436in}}{\pgfqpoint{0.892151in}{1.051436in}}%
\pgfpathlineto{\pgfqpoint{0.892151in}{1.051436in}}%
\pgfusepath{stroke,fill}%
\end{pgfscope}%
\begin{pgfscope}%
\pgfpathrectangle{\pgfqpoint{0.548058in}{0.516222in}}{\pgfqpoint{1.739582in}{1.783528in}} %
\pgfusepath{clip}%
\pgfsetbuttcap%
\pgfsetroundjoin%
\definecolor{currentfill}{rgb}{0.298039,0.447059,0.690196}%
\pgfsetfillcolor{currentfill}%
\pgfsetlinewidth{0.240900pt}%
\definecolor{currentstroke}{rgb}{1.000000,1.000000,1.000000}%
\pgfsetstrokecolor{currentstroke}%
\pgfsetdash{}{0pt}%
\pgfpathmoveto{\pgfqpoint{1.210756in}{1.345718in}}%
\pgfpathcurveto{\pgfqpoint{1.218993in}{1.345718in}}{\pgfqpoint{1.226893in}{1.348990in}}{\pgfqpoint{1.232717in}{1.354814in}}%
\pgfpathcurveto{\pgfqpoint{1.238540in}{1.360638in}}{\pgfqpoint{1.241813in}{1.368538in}}{\pgfqpoint{1.241813in}{1.376774in}}%
\pgfpathcurveto{\pgfqpoint{1.241813in}{1.385011in}}{\pgfqpoint{1.238540in}{1.392911in}}{\pgfqpoint{1.232717in}{1.398735in}}%
\pgfpathcurveto{\pgfqpoint{1.226893in}{1.404559in}}{\pgfqpoint{1.218993in}{1.407831in}}{\pgfqpoint{1.210756in}{1.407831in}}%
\pgfpathcurveto{\pgfqpoint{1.202520in}{1.407831in}}{\pgfqpoint{1.194620in}{1.404559in}}{\pgfqpoint{1.188796in}{1.398735in}}%
\pgfpathcurveto{\pgfqpoint{1.182972in}{1.392911in}}{\pgfqpoint{1.179700in}{1.385011in}}{\pgfqpoint{1.179700in}{1.376774in}}%
\pgfpathcurveto{\pgfqpoint{1.179700in}{1.368538in}}{\pgfqpoint{1.182972in}{1.360638in}}{\pgfqpoint{1.188796in}{1.354814in}}%
\pgfpathcurveto{\pgfqpoint{1.194620in}{1.348990in}}{\pgfqpoint{1.202520in}{1.345718in}}{\pgfqpoint{1.210756in}{1.345718in}}%
\pgfpathlineto{\pgfqpoint{1.210756in}{1.345718in}}%
\pgfusepath{stroke,fill}%
\end{pgfscope}%
\begin{pgfscope}%
\pgfsetrectcap%
\pgfsetmiterjoin%
\pgfsetlinewidth{0.000000pt}%
\definecolor{currentstroke}{rgb}{1.000000,1.000000,1.000000}%
\pgfsetstrokecolor{currentstroke}%
\pgfsetdash{}{0pt}%
\pgfpathmoveto{\pgfqpoint{0.548058in}{0.516222in}}%
\pgfpathlineto{\pgfqpoint{2.287641in}{0.516222in}}%
\pgfusepath{}%
\end{pgfscope}%
\begin{pgfscope}%
\pgfsetrectcap%
\pgfsetmiterjoin%
\pgfsetlinewidth{0.000000pt}%
\definecolor{currentstroke}{rgb}{1.000000,1.000000,1.000000}%
\pgfsetstrokecolor{currentstroke}%
\pgfsetdash{}{0pt}%
\pgfpathmoveto{\pgfqpoint{0.548058in}{0.516222in}}%
\pgfpathlineto{\pgfqpoint{0.548058in}{2.299750in}}%
\pgfusepath{}%
\end{pgfscope}%
\end{pgfpicture}%
\makeatother%
\endgroup%

		\caption{Comparison between the two times registered for one throw.}
		\label{fig_wtr_vs_obs2}
	\end{subfigure}
	\begin{subfigure}[h]{.5\linewidth}
		%% Creator: Matplotlib, PGF backend
%%
%% To include the figure in your LaTeX document, write
%%   \input{<filename>.pgf}
%%
%% Make sure the required packages are loaded in your preamble
%%   \usepackage{pgf}
%%
%% Figures using additional raster images can only be included by \input if
%% they are in the same directory as the main LaTeX file. For loading figures
%% from other directories you can use the `import` package
%%   \usepackage{import}
%% and then include the figures with
%%   \import{<path to file>}{<filename>.pgf}
%%
%% Matplotlib used the following preamble
%%   \usepackage[utf8x]{inputenc}
%%   \usepackage[T1]{fontenc}
%%   \usepackage{cmbright}
%%
\begingroup%
\makeatletter%
\begin{pgfpicture}%
\pgfpathrectangle{\pgfpointorigin}{\pgfqpoint{2.500000in}{2.500000in}}%
\pgfusepath{use as bounding box, clip}%
\begin{pgfscope}%
\pgfsetbuttcap%
\pgfsetmiterjoin%
\definecolor{currentfill}{rgb}{1.000000,1.000000,1.000000}%
\pgfsetfillcolor{currentfill}%
\pgfsetlinewidth{0.000000pt}%
\definecolor{currentstroke}{rgb}{1.000000,1.000000,1.000000}%
\pgfsetstrokecolor{currentstroke}%
\pgfsetdash{}{0pt}%
\pgfpathmoveto{\pgfqpoint{0.000000in}{0.000000in}}%
\pgfpathlineto{\pgfqpoint{2.500000in}{0.000000in}}%
\pgfpathlineto{\pgfqpoint{2.500000in}{2.500000in}}%
\pgfpathlineto{\pgfqpoint{0.000000in}{2.500000in}}%
\pgfpathclose%
\pgfusepath{fill}%
\end{pgfscope}%
\begin{pgfscope}%
\pgfsetbuttcap%
\pgfsetmiterjoin%
\definecolor{currentfill}{rgb}{0.917647,0.917647,0.949020}%
\pgfsetfillcolor{currentfill}%
\pgfsetlinewidth{0.000000pt}%
\definecolor{currentstroke}{rgb}{0.000000,0.000000,0.000000}%
\pgfsetstrokecolor{currentstroke}%
\pgfsetstrokeopacity{0.000000}%
\pgfsetdash{}{0pt}%
\pgfpathmoveto{\pgfqpoint{0.548058in}{0.516222in}}%
\pgfpathlineto{\pgfqpoint{2.287641in}{0.516222in}}%
\pgfpathlineto{\pgfqpoint{2.287641in}{2.299750in}}%
\pgfpathlineto{\pgfqpoint{0.548058in}{2.299750in}}%
\pgfpathclose%
\pgfusepath{fill}%
\end{pgfscope}%
\begin{pgfscope}%
\pgfpathrectangle{\pgfqpoint{0.548058in}{0.516222in}}{\pgfqpoint{1.739582in}{1.783528in}} %
\pgfusepath{clip}%
\pgfsetroundcap%
\pgfsetroundjoin%
\pgfsetlinewidth{0.803000pt}%
\definecolor{currentstroke}{rgb}{1.000000,1.000000,1.000000}%
\pgfsetstrokecolor{currentstroke}%
\pgfsetdash{}{0pt}%
\pgfpathmoveto{\pgfqpoint{0.548058in}{0.516222in}}%
\pgfpathlineto{\pgfqpoint{0.548058in}{2.299750in}}%
\pgfusepath{stroke}%
\end{pgfscope}%
\begin{pgfscope}%
\pgfsetbuttcap%
\pgfsetroundjoin%
\definecolor{currentfill}{rgb}{0.150000,0.150000,0.150000}%
\pgfsetfillcolor{currentfill}%
\pgfsetlinewidth{0.803000pt}%
\definecolor{currentstroke}{rgb}{0.150000,0.150000,0.150000}%
\pgfsetstrokecolor{currentstroke}%
\pgfsetdash{}{0pt}%
\pgfsys@defobject{currentmarker}{\pgfqpoint{0.000000in}{0.000000in}}{\pgfqpoint{0.000000in}{0.000000in}}{%
\pgfpathmoveto{\pgfqpoint{0.000000in}{0.000000in}}%
\pgfpathlineto{\pgfqpoint{0.000000in}{0.000000in}}%
\pgfusepath{stroke,fill}%
}%
\begin{pgfscope}%
\pgfsys@transformshift{0.548058in}{0.516222in}%
\pgfsys@useobject{currentmarker}{}%
\end{pgfscope}%
\end{pgfscope}%
\begin{pgfscope}%
\definecolor{textcolor}{rgb}{0.150000,0.150000,0.150000}%
\pgfsetstrokecolor{textcolor}%
\pgfsetfillcolor{textcolor}%
\pgftext[x=0.548058in,y=0.438444in,,top]{\color{textcolor}\sffamily\fontsize{8.000000}{9.600000}\selectfont −0.2}%
\end{pgfscope}%
\begin{pgfscope}%
\pgfpathrectangle{\pgfqpoint{0.548058in}{0.516222in}}{\pgfqpoint{1.739582in}{1.783528in}} %
\pgfusepath{clip}%
\pgfsetroundcap%
\pgfsetroundjoin%
\pgfsetlinewidth{0.803000pt}%
\definecolor{currentstroke}{rgb}{1.000000,1.000000,1.000000}%
\pgfsetstrokecolor{currentstroke}%
\pgfsetdash{}{0pt}%
\pgfpathmoveto{\pgfqpoint{0.796570in}{0.516222in}}%
\pgfpathlineto{\pgfqpoint{0.796570in}{2.299750in}}%
\pgfusepath{stroke}%
\end{pgfscope}%
\begin{pgfscope}%
\pgfsetbuttcap%
\pgfsetroundjoin%
\definecolor{currentfill}{rgb}{0.150000,0.150000,0.150000}%
\pgfsetfillcolor{currentfill}%
\pgfsetlinewidth{0.803000pt}%
\definecolor{currentstroke}{rgb}{0.150000,0.150000,0.150000}%
\pgfsetstrokecolor{currentstroke}%
\pgfsetdash{}{0pt}%
\pgfsys@defobject{currentmarker}{\pgfqpoint{0.000000in}{0.000000in}}{\pgfqpoint{0.000000in}{0.000000in}}{%
\pgfpathmoveto{\pgfqpoint{0.000000in}{0.000000in}}%
\pgfpathlineto{\pgfqpoint{0.000000in}{0.000000in}}%
\pgfusepath{stroke,fill}%
}%
\begin{pgfscope}%
\pgfsys@transformshift{0.796570in}{0.516222in}%
\pgfsys@useobject{currentmarker}{}%
\end{pgfscope}%
\end{pgfscope}%
\begin{pgfscope}%
\definecolor{textcolor}{rgb}{0.150000,0.150000,0.150000}%
\pgfsetstrokecolor{textcolor}%
\pgfsetfillcolor{textcolor}%
\pgftext[x=0.796570in,y=0.438444in,,top]{\color{textcolor}\sffamily\fontsize{8.000000}{9.600000}\selectfont 0.0}%
\end{pgfscope}%
\begin{pgfscope}%
\pgfpathrectangle{\pgfqpoint{0.548058in}{0.516222in}}{\pgfqpoint{1.739582in}{1.783528in}} %
\pgfusepath{clip}%
\pgfsetroundcap%
\pgfsetroundjoin%
\pgfsetlinewidth{0.803000pt}%
\definecolor{currentstroke}{rgb}{1.000000,1.000000,1.000000}%
\pgfsetstrokecolor{currentstroke}%
\pgfsetdash{}{0pt}%
\pgfpathmoveto{\pgfqpoint{1.045082in}{0.516222in}}%
\pgfpathlineto{\pgfqpoint{1.045082in}{2.299750in}}%
\pgfusepath{stroke}%
\end{pgfscope}%
\begin{pgfscope}%
\pgfsetbuttcap%
\pgfsetroundjoin%
\definecolor{currentfill}{rgb}{0.150000,0.150000,0.150000}%
\pgfsetfillcolor{currentfill}%
\pgfsetlinewidth{0.803000pt}%
\definecolor{currentstroke}{rgb}{0.150000,0.150000,0.150000}%
\pgfsetstrokecolor{currentstroke}%
\pgfsetdash{}{0pt}%
\pgfsys@defobject{currentmarker}{\pgfqpoint{0.000000in}{0.000000in}}{\pgfqpoint{0.000000in}{0.000000in}}{%
\pgfpathmoveto{\pgfqpoint{0.000000in}{0.000000in}}%
\pgfpathlineto{\pgfqpoint{0.000000in}{0.000000in}}%
\pgfusepath{stroke,fill}%
}%
\begin{pgfscope}%
\pgfsys@transformshift{1.045082in}{0.516222in}%
\pgfsys@useobject{currentmarker}{}%
\end{pgfscope}%
\end{pgfscope}%
\begin{pgfscope}%
\definecolor{textcolor}{rgb}{0.150000,0.150000,0.150000}%
\pgfsetstrokecolor{textcolor}%
\pgfsetfillcolor{textcolor}%
\pgftext[x=1.045082in,y=0.438444in,,top]{\color{textcolor}\sffamily\fontsize{8.000000}{9.600000}\selectfont 0.2}%
\end{pgfscope}%
\begin{pgfscope}%
\pgfpathrectangle{\pgfqpoint{0.548058in}{0.516222in}}{\pgfqpoint{1.739582in}{1.783528in}} %
\pgfusepath{clip}%
\pgfsetroundcap%
\pgfsetroundjoin%
\pgfsetlinewidth{0.803000pt}%
\definecolor{currentstroke}{rgb}{1.000000,1.000000,1.000000}%
\pgfsetstrokecolor{currentstroke}%
\pgfsetdash{}{0pt}%
\pgfpathmoveto{\pgfqpoint{1.293594in}{0.516222in}}%
\pgfpathlineto{\pgfqpoint{1.293594in}{2.299750in}}%
\pgfusepath{stroke}%
\end{pgfscope}%
\begin{pgfscope}%
\pgfsetbuttcap%
\pgfsetroundjoin%
\definecolor{currentfill}{rgb}{0.150000,0.150000,0.150000}%
\pgfsetfillcolor{currentfill}%
\pgfsetlinewidth{0.803000pt}%
\definecolor{currentstroke}{rgb}{0.150000,0.150000,0.150000}%
\pgfsetstrokecolor{currentstroke}%
\pgfsetdash{}{0pt}%
\pgfsys@defobject{currentmarker}{\pgfqpoint{0.000000in}{0.000000in}}{\pgfqpoint{0.000000in}{0.000000in}}{%
\pgfpathmoveto{\pgfqpoint{0.000000in}{0.000000in}}%
\pgfpathlineto{\pgfqpoint{0.000000in}{0.000000in}}%
\pgfusepath{stroke,fill}%
}%
\begin{pgfscope}%
\pgfsys@transformshift{1.293594in}{0.516222in}%
\pgfsys@useobject{currentmarker}{}%
\end{pgfscope}%
\end{pgfscope}%
\begin{pgfscope}%
\definecolor{textcolor}{rgb}{0.150000,0.150000,0.150000}%
\pgfsetstrokecolor{textcolor}%
\pgfsetfillcolor{textcolor}%
\pgftext[x=1.293594in,y=0.438444in,,top]{\color{textcolor}\sffamily\fontsize{8.000000}{9.600000}\selectfont 0.4}%
\end{pgfscope}%
\begin{pgfscope}%
\pgfpathrectangle{\pgfqpoint{0.548058in}{0.516222in}}{\pgfqpoint{1.739582in}{1.783528in}} %
\pgfusepath{clip}%
\pgfsetroundcap%
\pgfsetroundjoin%
\pgfsetlinewidth{0.803000pt}%
\definecolor{currentstroke}{rgb}{1.000000,1.000000,1.000000}%
\pgfsetstrokecolor{currentstroke}%
\pgfsetdash{}{0pt}%
\pgfpathmoveto{\pgfqpoint{1.542105in}{0.516222in}}%
\pgfpathlineto{\pgfqpoint{1.542105in}{2.299750in}}%
\pgfusepath{stroke}%
\end{pgfscope}%
\begin{pgfscope}%
\pgfsetbuttcap%
\pgfsetroundjoin%
\definecolor{currentfill}{rgb}{0.150000,0.150000,0.150000}%
\pgfsetfillcolor{currentfill}%
\pgfsetlinewidth{0.803000pt}%
\definecolor{currentstroke}{rgb}{0.150000,0.150000,0.150000}%
\pgfsetstrokecolor{currentstroke}%
\pgfsetdash{}{0pt}%
\pgfsys@defobject{currentmarker}{\pgfqpoint{0.000000in}{0.000000in}}{\pgfqpoint{0.000000in}{0.000000in}}{%
\pgfpathmoveto{\pgfqpoint{0.000000in}{0.000000in}}%
\pgfpathlineto{\pgfqpoint{0.000000in}{0.000000in}}%
\pgfusepath{stroke,fill}%
}%
\begin{pgfscope}%
\pgfsys@transformshift{1.542105in}{0.516222in}%
\pgfsys@useobject{currentmarker}{}%
\end{pgfscope}%
\end{pgfscope}%
\begin{pgfscope}%
\definecolor{textcolor}{rgb}{0.150000,0.150000,0.150000}%
\pgfsetstrokecolor{textcolor}%
\pgfsetfillcolor{textcolor}%
\pgftext[x=1.542105in,y=0.438444in,,top]{\color{textcolor}\sffamily\fontsize{8.000000}{9.600000}\selectfont 0.6}%
\end{pgfscope}%
\begin{pgfscope}%
\pgfpathrectangle{\pgfqpoint{0.548058in}{0.516222in}}{\pgfqpoint{1.739582in}{1.783528in}} %
\pgfusepath{clip}%
\pgfsetroundcap%
\pgfsetroundjoin%
\pgfsetlinewidth{0.803000pt}%
\definecolor{currentstroke}{rgb}{1.000000,1.000000,1.000000}%
\pgfsetstrokecolor{currentstroke}%
\pgfsetdash{}{0pt}%
\pgfpathmoveto{\pgfqpoint{1.790617in}{0.516222in}}%
\pgfpathlineto{\pgfqpoint{1.790617in}{2.299750in}}%
\pgfusepath{stroke}%
\end{pgfscope}%
\begin{pgfscope}%
\pgfsetbuttcap%
\pgfsetroundjoin%
\definecolor{currentfill}{rgb}{0.150000,0.150000,0.150000}%
\pgfsetfillcolor{currentfill}%
\pgfsetlinewidth{0.803000pt}%
\definecolor{currentstroke}{rgb}{0.150000,0.150000,0.150000}%
\pgfsetstrokecolor{currentstroke}%
\pgfsetdash{}{0pt}%
\pgfsys@defobject{currentmarker}{\pgfqpoint{0.000000in}{0.000000in}}{\pgfqpoint{0.000000in}{0.000000in}}{%
\pgfpathmoveto{\pgfqpoint{0.000000in}{0.000000in}}%
\pgfpathlineto{\pgfqpoint{0.000000in}{0.000000in}}%
\pgfusepath{stroke,fill}%
}%
\begin{pgfscope}%
\pgfsys@transformshift{1.790617in}{0.516222in}%
\pgfsys@useobject{currentmarker}{}%
\end{pgfscope}%
\end{pgfscope}%
\begin{pgfscope}%
\definecolor{textcolor}{rgb}{0.150000,0.150000,0.150000}%
\pgfsetstrokecolor{textcolor}%
\pgfsetfillcolor{textcolor}%
\pgftext[x=1.790617in,y=0.438444in,,top]{\color{textcolor}\sffamily\fontsize{8.000000}{9.600000}\selectfont 0.8}%
\end{pgfscope}%
\begin{pgfscope}%
\pgfpathrectangle{\pgfqpoint{0.548058in}{0.516222in}}{\pgfqpoint{1.739582in}{1.783528in}} %
\pgfusepath{clip}%
\pgfsetroundcap%
\pgfsetroundjoin%
\pgfsetlinewidth{0.803000pt}%
\definecolor{currentstroke}{rgb}{1.000000,1.000000,1.000000}%
\pgfsetstrokecolor{currentstroke}%
\pgfsetdash{}{0pt}%
\pgfpathmoveto{\pgfqpoint{2.039129in}{0.516222in}}%
\pgfpathlineto{\pgfqpoint{2.039129in}{2.299750in}}%
\pgfusepath{stroke}%
\end{pgfscope}%
\begin{pgfscope}%
\pgfsetbuttcap%
\pgfsetroundjoin%
\definecolor{currentfill}{rgb}{0.150000,0.150000,0.150000}%
\pgfsetfillcolor{currentfill}%
\pgfsetlinewidth{0.803000pt}%
\definecolor{currentstroke}{rgb}{0.150000,0.150000,0.150000}%
\pgfsetstrokecolor{currentstroke}%
\pgfsetdash{}{0pt}%
\pgfsys@defobject{currentmarker}{\pgfqpoint{0.000000in}{0.000000in}}{\pgfqpoint{0.000000in}{0.000000in}}{%
\pgfpathmoveto{\pgfqpoint{0.000000in}{0.000000in}}%
\pgfpathlineto{\pgfqpoint{0.000000in}{0.000000in}}%
\pgfusepath{stroke,fill}%
}%
\begin{pgfscope}%
\pgfsys@transformshift{2.039129in}{0.516222in}%
\pgfsys@useobject{currentmarker}{}%
\end{pgfscope}%
\end{pgfscope}%
\begin{pgfscope}%
\definecolor{textcolor}{rgb}{0.150000,0.150000,0.150000}%
\pgfsetstrokecolor{textcolor}%
\pgfsetfillcolor{textcolor}%
\pgftext[x=2.039129in,y=0.438444in,,top]{\color{textcolor}\sffamily\fontsize{8.000000}{9.600000}\selectfont 1.0}%
\end{pgfscope}%
\begin{pgfscope}%
\pgfpathrectangle{\pgfqpoint{0.548058in}{0.516222in}}{\pgfqpoint{1.739582in}{1.783528in}} %
\pgfusepath{clip}%
\pgfsetroundcap%
\pgfsetroundjoin%
\pgfsetlinewidth{0.803000pt}%
\definecolor{currentstroke}{rgb}{1.000000,1.000000,1.000000}%
\pgfsetstrokecolor{currentstroke}%
\pgfsetdash{}{0pt}%
\pgfpathmoveto{\pgfqpoint{2.287641in}{0.516222in}}%
\pgfpathlineto{\pgfqpoint{2.287641in}{2.299750in}}%
\pgfusepath{stroke}%
\end{pgfscope}%
\begin{pgfscope}%
\pgfsetbuttcap%
\pgfsetroundjoin%
\definecolor{currentfill}{rgb}{0.150000,0.150000,0.150000}%
\pgfsetfillcolor{currentfill}%
\pgfsetlinewidth{0.803000pt}%
\definecolor{currentstroke}{rgb}{0.150000,0.150000,0.150000}%
\pgfsetstrokecolor{currentstroke}%
\pgfsetdash{}{0pt}%
\pgfsys@defobject{currentmarker}{\pgfqpoint{0.000000in}{0.000000in}}{\pgfqpoint{0.000000in}{0.000000in}}{%
\pgfpathmoveto{\pgfqpoint{0.000000in}{0.000000in}}%
\pgfpathlineto{\pgfqpoint{0.000000in}{0.000000in}}%
\pgfusepath{stroke,fill}%
}%
\begin{pgfscope}%
\pgfsys@transformshift{2.287641in}{0.516222in}%
\pgfsys@useobject{currentmarker}{}%
\end{pgfscope}%
\end{pgfscope}%
\begin{pgfscope}%
\definecolor{textcolor}{rgb}{0.150000,0.150000,0.150000}%
\pgfsetstrokecolor{textcolor}%
\pgfsetfillcolor{textcolor}%
\pgftext[x=2.287641in,y=0.438444in,,top]{\color{textcolor}\sffamily\fontsize{8.000000}{9.600000}\selectfont 1.2}%
\end{pgfscope}%
\begin{pgfscope}%
\definecolor{textcolor}{rgb}{0.150000,0.150000,0.150000}%
\pgfsetstrokecolor{textcolor}%
\pgfsetfillcolor{textcolor}%
\pgftext[x=1.417849in,y=0.273321in,,top]{\color{textcolor}\sffamily\fontsize{8.800000}{10.560000}\selectfont wing tail ratio}%
\end{pgfscope}%
\begin{pgfscope}%
\pgfpathrectangle{\pgfqpoint{0.548058in}{0.516222in}}{\pgfqpoint{1.739582in}{1.783528in}} %
\pgfusepath{clip}%
\pgfsetroundcap%
\pgfsetroundjoin%
\pgfsetlinewidth{0.803000pt}%
\definecolor{currentstroke}{rgb}{1.000000,1.000000,1.000000}%
\pgfsetstrokecolor{currentstroke}%
\pgfsetdash{}{0pt}%
\pgfpathmoveto{\pgfqpoint{0.548058in}{0.516222in}}%
\pgfpathlineto{\pgfqpoint{2.287641in}{0.516222in}}%
\pgfusepath{stroke}%
\end{pgfscope}%
\begin{pgfscope}%
\pgfsetbuttcap%
\pgfsetroundjoin%
\definecolor{currentfill}{rgb}{0.150000,0.150000,0.150000}%
\pgfsetfillcolor{currentfill}%
\pgfsetlinewidth{0.803000pt}%
\definecolor{currentstroke}{rgb}{0.150000,0.150000,0.150000}%
\pgfsetstrokecolor{currentstroke}%
\pgfsetdash{}{0pt}%
\pgfsys@defobject{currentmarker}{\pgfqpoint{0.000000in}{0.000000in}}{\pgfqpoint{0.000000in}{0.000000in}}{%
\pgfpathmoveto{\pgfqpoint{0.000000in}{0.000000in}}%
\pgfpathlineto{\pgfqpoint{0.000000in}{0.000000in}}%
\pgfusepath{stroke,fill}%
}%
\begin{pgfscope}%
\pgfsys@transformshift{0.548058in}{0.516222in}%
\pgfsys@useobject{currentmarker}{}%
\end{pgfscope}%
\end{pgfscope}%
\begin{pgfscope}%
\definecolor{textcolor}{rgb}{0.150000,0.150000,0.150000}%
\pgfsetstrokecolor{textcolor}%
\pgfsetfillcolor{textcolor}%
\pgftext[x=0.470280in,y=0.516222in,right,]{\color{textcolor}\sffamily\fontsize{8.000000}{9.600000}\selectfont −1.5}%
\end{pgfscope}%
\begin{pgfscope}%
\pgfpathrectangle{\pgfqpoint{0.548058in}{0.516222in}}{\pgfqpoint{1.739582in}{1.783528in}} %
\pgfusepath{clip}%
\pgfsetroundcap%
\pgfsetroundjoin%
\pgfsetlinewidth{0.803000pt}%
\definecolor{currentstroke}{rgb}{1.000000,1.000000,1.000000}%
\pgfsetstrokecolor{currentstroke}%
\pgfsetdash{}{0pt}%
\pgfpathmoveto{\pgfqpoint{0.548058in}{0.771012in}}%
\pgfpathlineto{\pgfqpoint{2.287641in}{0.771012in}}%
\pgfusepath{stroke}%
\end{pgfscope}%
\begin{pgfscope}%
\pgfsetbuttcap%
\pgfsetroundjoin%
\definecolor{currentfill}{rgb}{0.150000,0.150000,0.150000}%
\pgfsetfillcolor{currentfill}%
\pgfsetlinewidth{0.803000pt}%
\definecolor{currentstroke}{rgb}{0.150000,0.150000,0.150000}%
\pgfsetstrokecolor{currentstroke}%
\pgfsetdash{}{0pt}%
\pgfsys@defobject{currentmarker}{\pgfqpoint{0.000000in}{0.000000in}}{\pgfqpoint{0.000000in}{0.000000in}}{%
\pgfpathmoveto{\pgfqpoint{0.000000in}{0.000000in}}%
\pgfpathlineto{\pgfqpoint{0.000000in}{0.000000in}}%
\pgfusepath{stroke,fill}%
}%
\begin{pgfscope}%
\pgfsys@transformshift{0.548058in}{0.771012in}%
\pgfsys@useobject{currentmarker}{}%
\end{pgfscope}%
\end{pgfscope}%
\begin{pgfscope}%
\definecolor{textcolor}{rgb}{0.150000,0.150000,0.150000}%
\pgfsetstrokecolor{textcolor}%
\pgfsetfillcolor{textcolor}%
\pgftext[x=0.470280in,y=0.771012in,right,]{\color{textcolor}\sffamily\fontsize{8.000000}{9.600000}\selectfont −1.0}%
\end{pgfscope}%
\begin{pgfscope}%
\pgfpathrectangle{\pgfqpoint{0.548058in}{0.516222in}}{\pgfqpoint{1.739582in}{1.783528in}} %
\pgfusepath{clip}%
\pgfsetroundcap%
\pgfsetroundjoin%
\pgfsetlinewidth{0.803000pt}%
\definecolor{currentstroke}{rgb}{1.000000,1.000000,1.000000}%
\pgfsetstrokecolor{currentstroke}%
\pgfsetdash{}{0pt}%
\pgfpathmoveto{\pgfqpoint{0.548058in}{1.025802in}}%
\pgfpathlineto{\pgfqpoint{2.287641in}{1.025802in}}%
\pgfusepath{stroke}%
\end{pgfscope}%
\begin{pgfscope}%
\pgfsetbuttcap%
\pgfsetroundjoin%
\definecolor{currentfill}{rgb}{0.150000,0.150000,0.150000}%
\pgfsetfillcolor{currentfill}%
\pgfsetlinewidth{0.803000pt}%
\definecolor{currentstroke}{rgb}{0.150000,0.150000,0.150000}%
\pgfsetstrokecolor{currentstroke}%
\pgfsetdash{}{0pt}%
\pgfsys@defobject{currentmarker}{\pgfqpoint{0.000000in}{0.000000in}}{\pgfqpoint{0.000000in}{0.000000in}}{%
\pgfpathmoveto{\pgfqpoint{0.000000in}{0.000000in}}%
\pgfpathlineto{\pgfqpoint{0.000000in}{0.000000in}}%
\pgfusepath{stroke,fill}%
}%
\begin{pgfscope}%
\pgfsys@transformshift{0.548058in}{1.025802in}%
\pgfsys@useobject{currentmarker}{}%
\end{pgfscope}%
\end{pgfscope}%
\begin{pgfscope}%
\definecolor{textcolor}{rgb}{0.150000,0.150000,0.150000}%
\pgfsetstrokecolor{textcolor}%
\pgfsetfillcolor{textcolor}%
\pgftext[x=0.470280in,y=1.025802in,right,]{\color{textcolor}\sffamily\fontsize{8.000000}{9.600000}\selectfont −0.5}%
\end{pgfscope}%
\begin{pgfscope}%
\pgfpathrectangle{\pgfqpoint{0.548058in}{0.516222in}}{\pgfqpoint{1.739582in}{1.783528in}} %
\pgfusepath{clip}%
\pgfsetroundcap%
\pgfsetroundjoin%
\pgfsetlinewidth{0.803000pt}%
\definecolor{currentstroke}{rgb}{1.000000,1.000000,1.000000}%
\pgfsetstrokecolor{currentstroke}%
\pgfsetdash{}{0pt}%
\pgfpathmoveto{\pgfqpoint{0.548058in}{1.280591in}}%
\pgfpathlineto{\pgfqpoint{2.287641in}{1.280591in}}%
\pgfusepath{stroke}%
\end{pgfscope}%
\begin{pgfscope}%
\pgfsetbuttcap%
\pgfsetroundjoin%
\definecolor{currentfill}{rgb}{0.150000,0.150000,0.150000}%
\pgfsetfillcolor{currentfill}%
\pgfsetlinewidth{0.803000pt}%
\definecolor{currentstroke}{rgb}{0.150000,0.150000,0.150000}%
\pgfsetstrokecolor{currentstroke}%
\pgfsetdash{}{0pt}%
\pgfsys@defobject{currentmarker}{\pgfqpoint{0.000000in}{0.000000in}}{\pgfqpoint{0.000000in}{0.000000in}}{%
\pgfpathmoveto{\pgfqpoint{0.000000in}{0.000000in}}%
\pgfpathlineto{\pgfqpoint{0.000000in}{0.000000in}}%
\pgfusepath{stroke,fill}%
}%
\begin{pgfscope}%
\pgfsys@transformshift{0.548058in}{1.280591in}%
\pgfsys@useobject{currentmarker}{}%
\end{pgfscope}%
\end{pgfscope}%
\begin{pgfscope}%
\definecolor{textcolor}{rgb}{0.150000,0.150000,0.150000}%
\pgfsetstrokecolor{textcolor}%
\pgfsetfillcolor{textcolor}%
\pgftext[x=0.470280in,y=1.280591in,right,]{\color{textcolor}\sffamily\fontsize{8.000000}{9.600000}\selectfont 0.0}%
\end{pgfscope}%
\begin{pgfscope}%
\pgfpathrectangle{\pgfqpoint{0.548058in}{0.516222in}}{\pgfqpoint{1.739582in}{1.783528in}} %
\pgfusepath{clip}%
\pgfsetroundcap%
\pgfsetroundjoin%
\pgfsetlinewidth{0.803000pt}%
\definecolor{currentstroke}{rgb}{1.000000,1.000000,1.000000}%
\pgfsetstrokecolor{currentstroke}%
\pgfsetdash{}{0pt}%
\pgfpathmoveto{\pgfqpoint{0.548058in}{1.535381in}}%
\pgfpathlineto{\pgfqpoint{2.287641in}{1.535381in}}%
\pgfusepath{stroke}%
\end{pgfscope}%
\begin{pgfscope}%
\pgfsetbuttcap%
\pgfsetroundjoin%
\definecolor{currentfill}{rgb}{0.150000,0.150000,0.150000}%
\pgfsetfillcolor{currentfill}%
\pgfsetlinewidth{0.803000pt}%
\definecolor{currentstroke}{rgb}{0.150000,0.150000,0.150000}%
\pgfsetstrokecolor{currentstroke}%
\pgfsetdash{}{0pt}%
\pgfsys@defobject{currentmarker}{\pgfqpoint{0.000000in}{0.000000in}}{\pgfqpoint{0.000000in}{0.000000in}}{%
\pgfpathmoveto{\pgfqpoint{0.000000in}{0.000000in}}%
\pgfpathlineto{\pgfqpoint{0.000000in}{0.000000in}}%
\pgfusepath{stroke,fill}%
}%
\begin{pgfscope}%
\pgfsys@transformshift{0.548058in}{1.535381in}%
\pgfsys@useobject{currentmarker}{}%
\end{pgfscope}%
\end{pgfscope}%
\begin{pgfscope}%
\definecolor{textcolor}{rgb}{0.150000,0.150000,0.150000}%
\pgfsetstrokecolor{textcolor}%
\pgfsetfillcolor{textcolor}%
\pgftext[x=0.470280in,y=1.535381in,right,]{\color{textcolor}\sffamily\fontsize{8.000000}{9.600000}\selectfont 0.5}%
\end{pgfscope}%
\begin{pgfscope}%
\pgfpathrectangle{\pgfqpoint{0.548058in}{0.516222in}}{\pgfqpoint{1.739582in}{1.783528in}} %
\pgfusepath{clip}%
\pgfsetroundcap%
\pgfsetroundjoin%
\pgfsetlinewidth{0.803000pt}%
\definecolor{currentstroke}{rgb}{1.000000,1.000000,1.000000}%
\pgfsetstrokecolor{currentstroke}%
\pgfsetdash{}{0pt}%
\pgfpathmoveto{\pgfqpoint{0.548058in}{1.790171in}}%
\pgfpathlineto{\pgfqpoint{2.287641in}{1.790171in}}%
\pgfusepath{stroke}%
\end{pgfscope}%
\begin{pgfscope}%
\pgfsetbuttcap%
\pgfsetroundjoin%
\definecolor{currentfill}{rgb}{0.150000,0.150000,0.150000}%
\pgfsetfillcolor{currentfill}%
\pgfsetlinewidth{0.803000pt}%
\definecolor{currentstroke}{rgb}{0.150000,0.150000,0.150000}%
\pgfsetstrokecolor{currentstroke}%
\pgfsetdash{}{0pt}%
\pgfsys@defobject{currentmarker}{\pgfqpoint{0.000000in}{0.000000in}}{\pgfqpoint{0.000000in}{0.000000in}}{%
\pgfpathmoveto{\pgfqpoint{0.000000in}{0.000000in}}%
\pgfpathlineto{\pgfqpoint{0.000000in}{0.000000in}}%
\pgfusepath{stroke,fill}%
}%
\begin{pgfscope}%
\pgfsys@transformshift{0.548058in}{1.790171in}%
\pgfsys@useobject{currentmarker}{}%
\end{pgfscope}%
\end{pgfscope}%
\begin{pgfscope}%
\definecolor{textcolor}{rgb}{0.150000,0.150000,0.150000}%
\pgfsetstrokecolor{textcolor}%
\pgfsetfillcolor{textcolor}%
\pgftext[x=0.470280in,y=1.790171in,right,]{\color{textcolor}\sffamily\fontsize{8.000000}{9.600000}\selectfont 1.0}%
\end{pgfscope}%
\begin{pgfscope}%
\pgfpathrectangle{\pgfqpoint{0.548058in}{0.516222in}}{\pgfqpoint{1.739582in}{1.783528in}} %
\pgfusepath{clip}%
\pgfsetroundcap%
\pgfsetroundjoin%
\pgfsetlinewidth{0.803000pt}%
\definecolor{currentstroke}{rgb}{1.000000,1.000000,1.000000}%
\pgfsetstrokecolor{currentstroke}%
\pgfsetdash{}{0pt}%
\pgfpathmoveto{\pgfqpoint{0.548058in}{2.044960in}}%
\pgfpathlineto{\pgfqpoint{2.287641in}{2.044960in}}%
\pgfusepath{stroke}%
\end{pgfscope}%
\begin{pgfscope}%
\pgfsetbuttcap%
\pgfsetroundjoin%
\definecolor{currentfill}{rgb}{0.150000,0.150000,0.150000}%
\pgfsetfillcolor{currentfill}%
\pgfsetlinewidth{0.803000pt}%
\definecolor{currentstroke}{rgb}{0.150000,0.150000,0.150000}%
\pgfsetstrokecolor{currentstroke}%
\pgfsetdash{}{0pt}%
\pgfsys@defobject{currentmarker}{\pgfqpoint{0.000000in}{0.000000in}}{\pgfqpoint{0.000000in}{0.000000in}}{%
\pgfpathmoveto{\pgfqpoint{0.000000in}{0.000000in}}%
\pgfpathlineto{\pgfqpoint{0.000000in}{0.000000in}}%
\pgfusepath{stroke,fill}%
}%
\begin{pgfscope}%
\pgfsys@transformshift{0.548058in}{2.044960in}%
\pgfsys@useobject{currentmarker}{}%
\end{pgfscope}%
\end{pgfscope}%
\begin{pgfscope}%
\definecolor{textcolor}{rgb}{0.150000,0.150000,0.150000}%
\pgfsetstrokecolor{textcolor}%
\pgfsetfillcolor{textcolor}%
\pgftext[x=0.470280in,y=2.044960in,right,]{\color{textcolor}\sffamily\fontsize{8.000000}{9.600000}\selectfont 1.5}%
\end{pgfscope}%
\begin{pgfscope}%
\pgfpathrectangle{\pgfqpoint{0.548058in}{0.516222in}}{\pgfqpoint{1.739582in}{1.783528in}} %
\pgfusepath{clip}%
\pgfsetroundcap%
\pgfsetroundjoin%
\pgfsetlinewidth{0.803000pt}%
\definecolor{currentstroke}{rgb}{1.000000,1.000000,1.000000}%
\pgfsetstrokecolor{currentstroke}%
\pgfsetdash{}{0pt}%
\pgfpathmoveto{\pgfqpoint{0.548058in}{2.299750in}}%
\pgfpathlineto{\pgfqpoint{2.287641in}{2.299750in}}%
\pgfusepath{stroke}%
\end{pgfscope}%
\begin{pgfscope}%
\pgfsetbuttcap%
\pgfsetroundjoin%
\definecolor{currentfill}{rgb}{0.150000,0.150000,0.150000}%
\pgfsetfillcolor{currentfill}%
\pgfsetlinewidth{0.803000pt}%
\definecolor{currentstroke}{rgb}{0.150000,0.150000,0.150000}%
\pgfsetstrokecolor{currentstroke}%
\pgfsetdash{}{0pt}%
\pgfsys@defobject{currentmarker}{\pgfqpoint{0.000000in}{0.000000in}}{\pgfqpoint{0.000000in}{0.000000in}}{%
\pgfpathmoveto{\pgfqpoint{0.000000in}{0.000000in}}%
\pgfpathlineto{\pgfqpoint{0.000000in}{0.000000in}}%
\pgfusepath{stroke,fill}%
}%
\begin{pgfscope}%
\pgfsys@transformshift{0.548058in}{2.299750in}%
\pgfsys@useobject{currentmarker}{}%
\end{pgfscope}%
\end{pgfscope}%
\begin{pgfscope}%
\definecolor{textcolor}{rgb}{0.150000,0.150000,0.150000}%
\pgfsetstrokecolor{textcolor}%
\pgfsetfillcolor{textcolor}%
\pgftext[x=0.470280in,y=2.299750in,right,]{\color{textcolor}\sffamily\fontsize{8.000000}{9.600000}\selectfont 2.0}%
\end{pgfscope}%
\begin{pgfscope}%
\definecolor{textcolor}{rgb}{0.150000,0.150000,0.150000}%
\pgfsetstrokecolor{textcolor}%
\pgfsetfillcolor{textcolor}%
\pgftext[x=0.151066in,y=1.407986in,,bottom,rotate=90.000000]{\color{textcolor}\sffamily\fontsize{8.800000}{10.560000}\selectfont Average of four falling times}%
\end{pgfscope}%
\begin{pgfscope}%
\pgfpathrectangle{\pgfqpoint{0.548058in}{0.516222in}}{\pgfqpoint{1.739582in}{1.783528in}} %
\pgfusepath{clip}%
\pgfsetbuttcap%
\pgfsetroundjoin%
\definecolor{currentfill}{rgb}{0.298039,0.447059,0.690196}%
\pgfsetfillcolor{currentfill}%
\pgfsetlinewidth{0.240900pt}%
\definecolor{currentstroke}{rgb}{1.000000,1.000000,1.000000}%
\pgfsetstrokecolor{currentstroke}%
\pgfsetdash{}{0pt}%
\pgfpathmoveto{\pgfqpoint{0.796570in}{1.476106in}}%
\pgfpathcurveto{\pgfqpoint{0.804806in}{1.476106in}}{\pgfqpoint{0.812706in}{1.479379in}}{\pgfqpoint{0.818530in}{1.485203in}}%
\pgfpathcurveto{\pgfqpoint{0.824354in}{1.491027in}}{\pgfqpoint{0.827626in}{1.498927in}}{\pgfqpoint{0.827626in}{1.507163in}}%
\pgfpathcurveto{\pgfqpoint{0.827626in}{1.515399in}}{\pgfqpoint{0.824354in}{1.523299in}}{\pgfqpoint{0.818530in}{1.529123in}}%
\pgfpathcurveto{\pgfqpoint{0.812706in}{1.534947in}}{\pgfqpoint{0.804806in}{1.538219in}}{\pgfqpoint{0.796570in}{1.538219in}}%
\pgfpathcurveto{\pgfqpoint{0.788334in}{1.538219in}}{\pgfqpoint{0.780434in}{1.534947in}}{\pgfqpoint{0.774610in}{1.529123in}}%
\pgfpathcurveto{\pgfqpoint{0.768786in}{1.523299in}}{\pgfqpoint{0.765513in}{1.515399in}}{\pgfqpoint{0.765513in}{1.507163in}}%
\pgfpathcurveto{\pgfqpoint{0.765513in}{1.498927in}}{\pgfqpoint{0.768786in}{1.491027in}}{\pgfqpoint{0.774610in}{1.485203in}}%
\pgfpathcurveto{\pgfqpoint{0.780434in}{1.479379in}}{\pgfqpoint{0.788334in}{1.476106in}}{\pgfqpoint{0.796570in}{1.476106in}}%
\pgfpathlineto{\pgfqpoint{0.796570in}{1.476106in}}%
\pgfusepath{stroke,fill}%
\end{pgfscope}%
\begin{pgfscope}%
\pgfpathrectangle{\pgfqpoint{0.548058in}{0.516222in}}{\pgfqpoint{1.739582in}{1.783528in}} %
\pgfusepath{clip}%
\pgfsetbuttcap%
\pgfsetroundjoin%
\definecolor{currentfill}{rgb}{0.298039,0.447059,0.690196}%
\pgfsetfillcolor{currentfill}%
\pgfsetlinewidth{0.240900pt}%
\definecolor{currentstroke}{rgb}{1.000000,1.000000,1.000000}%
\pgfsetstrokecolor{currentstroke}%
\pgfsetdash{}{0pt}%
\pgfpathmoveto{\pgfqpoint{1.019593in}{1.148702in}}%
\pgfpathcurveto{\pgfqpoint{1.027830in}{1.148702in}}{\pgfqpoint{1.035730in}{1.151974in}}{\pgfqpoint{1.041554in}{1.157798in}}%
\pgfpathcurveto{\pgfqpoint{1.047378in}{1.163622in}}{\pgfqpoint{1.050650in}{1.171522in}}{\pgfqpoint{1.050650in}{1.179758in}}%
\pgfpathcurveto{\pgfqpoint{1.050650in}{1.187995in}}{\pgfqpoint{1.047378in}{1.195895in}}{\pgfqpoint{1.041554in}{1.201719in}}%
\pgfpathcurveto{\pgfqpoint{1.035730in}{1.207542in}}{\pgfqpoint{1.027830in}{1.210815in}}{\pgfqpoint{1.019593in}{1.210815in}}%
\pgfpathcurveto{\pgfqpoint{1.011357in}{1.210815in}}{\pgfqpoint{1.003457in}{1.207542in}}{\pgfqpoint{0.997633in}{1.201719in}}%
\pgfpathcurveto{\pgfqpoint{0.991809in}{1.195895in}}{\pgfqpoint{0.988537in}{1.187995in}}{\pgfqpoint{0.988537in}{1.179758in}}%
\pgfpathcurveto{\pgfqpoint{0.988537in}{1.171522in}}{\pgfqpoint{0.991809in}{1.163622in}}{\pgfqpoint{0.997633in}{1.157798in}}%
\pgfpathcurveto{\pgfqpoint{1.003457in}{1.151974in}}{\pgfqpoint{1.011357in}{1.148702in}}{\pgfqpoint{1.019593in}{1.148702in}}%
\pgfpathlineto{\pgfqpoint{1.019593in}{1.148702in}}%
\pgfusepath{stroke,fill}%
\end{pgfscope}%
\begin{pgfscope}%
\pgfpathrectangle{\pgfqpoint{0.548058in}{0.516222in}}{\pgfqpoint{1.739582in}{1.783528in}} %
\pgfusepath{clip}%
\pgfsetbuttcap%
\pgfsetroundjoin%
\definecolor{currentfill}{rgb}{0.298039,0.447059,0.690196}%
\pgfsetfillcolor{currentfill}%
\pgfsetlinewidth{0.240900pt}%
\definecolor{currentstroke}{rgb}{1.000000,1.000000,1.000000}%
\pgfsetstrokecolor{currentstroke}%
\pgfsetdash{}{0pt}%
\pgfpathmoveto{\pgfqpoint{1.847966in}{1.114305in}}%
\pgfpathcurveto{\pgfqpoint{1.856202in}{1.114305in}}{\pgfqpoint{1.864102in}{1.117577in}}{\pgfqpoint{1.869926in}{1.123401in}}%
\pgfpathcurveto{\pgfqpoint{1.875750in}{1.129225in}}{\pgfqpoint{1.879022in}{1.137125in}}{\pgfqpoint{1.879022in}{1.145362in}}%
\pgfpathcurveto{\pgfqpoint{1.879022in}{1.153598in}}{\pgfqpoint{1.875750in}{1.161498in}}{\pgfqpoint{1.869926in}{1.167322in}}%
\pgfpathcurveto{\pgfqpoint{1.864102in}{1.173146in}}{\pgfqpoint{1.856202in}{1.176418in}}{\pgfqpoint{1.847966in}{1.176418in}}%
\pgfpathcurveto{\pgfqpoint{1.839730in}{1.176418in}}{\pgfqpoint{1.831830in}{1.173146in}}{\pgfqpoint{1.826006in}{1.167322in}}%
\pgfpathcurveto{\pgfqpoint{1.820182in}{1.161498in}}{\pgfqpoint{1.816909in}{1.153598in}}{\pgfqpoint{1.816909in}{1.145362in}}%
\pgfpathcurveto{\pgfqpoint{1.816909in}{1.137125in}}{\pgfqpoint{1.820182in}{1.129225in}}{\pgfqpoint{1.826006in}{1.123401in}}%
\pgfpathcurveto{\pgfqpoint{1.831830in}{1.117577in}}{\pgfqpoint{1.839730in}{1.114305in}}{\pgfqpoint{1.847966in}{1.114305in}}%
\pgfpathlineto{\pgfqpoint{1.847966in}{1.114305in}}%
\pgfusepath{stroke,fill}%
\end{pgfscope}%
\begin{pgfscope}%
\pgfpathrectangle{\pgfqpoint{0.548058in}{0.516222in}}{\pgfqpoint{1.739582in}{1.783528in}} %
\pgfusepath{clip}%
\pgfsetbuttcap%
\pgfsetroundjoin%
\definecolor{currentfill}{rgb}{0.298039,0.447059,0.690196}%
\pgfsetfillcolor{currentfill}%
\pgfsetlinewidth{0.240900pt}%
\definecolor{currentstroke}{rgb}{1.000000,1.000000,1.000000}%
\pgfsetstrokecolor{currentstroke}%
\pgfsetdash{}{0pt}%
\pgfpathmoveto{\pgfqpoint{2.039129in}{0.869707in}}%
\pgfpathcurveto{\pgfqpoint{2.047365in}{0.869707in}}{\pgfqpoint{2.055265in}{0.872979in}}{\pgfqpoint{2.061089in}{0.878803in}}%
\pgfpathcurveto{\pgfqpoint{2.066913in}{0.884627in}}{\pgfqpoint{2.070185in}{0.892527in}}{\pgfqpoint{2.070185in}{0.900764in}}%
\pgfpathcurveto{\pgfqpoint{2.070185in}{0.909000in}}{\pgfqpoint{2.066913in}{0.916900in}}{\pgfqpoint{2.061089in}{0.922724in}}%
\pgfpathcurveto{\pgfqpoint{2.055265in}{0.928548in}}{\pgfqpoint{2.047365in}{0.931820in}}{\pgfqpoint{2.039129in}{0.931820in}}%
\pgfpathcurveto{\pgfqpoint{2.030893in}{0.931820in}}{\pgfqpoint{2.022993in}{0.928548in}}{\pgfqpoint{2.017169in}{0.922724in}}%
\pgfpathcurveto{\pgfqpoint{2.011345in}{0.916900in}}{\pgfqpoint{2.008072in}{0.909000in}}{\pgfqpoint{2.008072in}{0.900764in}}%
\pgfpathcurveto{\pgfqpoint{2.008072in}{0.892527in}}{\pgfqpoint{2.011345in}{0.884627in}}{\pgfqpoint{2.017169in}{0.878803in}}%
\pgfpathcurveto{\pgfqpoint{2.022993in}{0.872979in}}{\pgfqpoint{2.030893in}{0.869707in}}{\pgfqpoint{2.039129in}{0.869707in}}%
\pgfpathlineto{\pgfqpoint{2.039129in}{0.869707in}}%
\pgfusepath{stroke,fill}%
\end{pgfscope}%
\begin{pgfscope}%
\pgfpathrectangle{\pgfqpoint{0.548058in}{0.516222in}}{\pgfqpoint{1.739582in}{1.783528in}} %
\pgfusepath{clip}%
\pgfsetbuttcap%
\pgfsetroundjoin%
\definecolor{currentfill}{rgb}{0.298039,0.447059,0.690196}%
\pgfsetfillcolor{currentfill}%
\pgfsetlinewidth{0.240900pt}%
\definecolor{currentstroke}{rgb}{1.000000,1.000000,1.000000}%
\pgfsetstrokecolor{currentstroke}%
\pgfsetdash{}{0pt}%
\pgfpathmoveto{\pgfqpoint{1.656803in}{0.668423in}}%
\pgfpathcurveto{\pgfqpoint{1.665039in}{0.668423in}}{\pgfqpoint{1.672939in}{0.671696in}}{\pgfqpoint{1.678763in}{0.677519in}}%
\pgfpathcurveto{\pgfqpoint{1.684587in}{0.683343in}}{\pgfqpoint{1.687860in}{0.691243in}}{\pgfqpoint{1.687860in}{0.699480in}}%
\pgfpathcurveto{\pgfqpoint{1.687860in}{0.707716in}}{\pgfqpoint{1.684587in}{0.715616in}}{\pgfqpoint{1.678763in}{0.721440in}}%
\pgfpathcurveto{\pgfqpoint{1.672939in}{0.727264in}}{\pgfqpoint{1.665039in}{0.730536in}}{\pgfqpoint{1.656803in}{0.730536in}}%
\pgfpathcurveto{\pgfqpoint{1.648567in}{0.730536in}}{\pgfqpoint{1.640667in}{0.727264in}}{\pgfqpoint{1.634843in}{0.721440in}}%
\pgfpathcurveto{\pgfqpoint{1.629019in}{0.715616in}}{\pgfqpoint{1.625747in}{0.707716in}}{\pgfqpoint{1.625747in}{0.699480in}}%
\pgfpathcurveto{\pgfqpoint{1.625747in}{0.691243in}}{\pgfqpoint{1.629019in}{0.683343in}}{\pgfqpoint{1.634843in}{0.677519in}}%
\pgfpathcurveto{\pgfqpoint{1.640667in}{0.671696in}}{\pgfqpoint{1.648567in}{0.668423in}}{\pgfqpoint{1.656803in}{0.668423in}}%
\pgfpathlineto{\pgfqpoint{1.656803in}{0.668423in}}%
\pgfusepath{stroke,fill}%
\end{pgfscope}%
\begin{pgfscope}%
\pgfpathrectangle{\pgfqpoint{0.548058in}{0.516222in}}{\pgfqpoint{1.739582in}{1.783528in}} %
\pgfusepath{clip}%
\pgfsetbuttcap%
\pgfsetroundjoin%
\definecolor{currentfill}{rgb}{0.298039,0.447059,0.690196}%
\pgfsetfillcolor{currentfill}%
\pgfsetlinewidth{0.240900pt}%
\definecolor{currentstroke}{rgb}{1.000000,1.000000,1.000000}%
\pgfsetstrokecolor{currentstroke}%
\pgfsetdash{}{0pt}%
\pgfpathmoveto{\pgfqpoint{1.816105in}{0.697724in}}%
\pgfpathcurveto{\pgfqpoint{1.824342in}{0.697724in}}{\pgfqpoint{1.832242in}{0.700996in}}{\pgfqpoint{1.838066in}{0.706820in}}%
\pgfpathcurveto{\pgfqpoint{1.843890in}{0.712644in}}{\pgfqpoint{1.847162in}{0.720544in}}{\pgfqpoint{1.847162in}{0.728781in}}%
\pgfpathcurveto{\pgfqpoint{1.847162in}{0.737017in}}{\pgfqpoint{1.843890in}{0.744917in}}{\pgfqpoint{1.838066in}{0.750741in}}%
\pgfpathcurveto{\pgfqpoint{1.832242in}{0.756565in}}{\pgfqpoint{1.824342in}{0.759837in}}{\pgfqpoint{1.816105in}{0.759837in}}%
\pgfpathcurveto{\pgfqpoint{1.807869in}{0.759837in}}{\pgfqpoint{1.799969in}{0.756565in}}{\pgfqpoint{1.794145in}{0.750741in}}%
\pgfpathcurveto{\pgfqpoint{1.788321in}{0.744917in}}{\pgfqpoint{1.785049in}{0.737017in}}{\pgfqpoint{1.785049in}{0.728781in}}%
\pgfpathcurveto{\pgfqpoint{1.785049in}{0.720544in}}{\pgfqpoint{1.788321in}{0.712644in}}{\pgfqpoint{1.794145in}{0.706820in}}%
\pgfpathcurveto{\pgfqpoint{1.799969in}{0.700996in}}{\pgfqpoint{1.807869in}{0.697724in}}{\pgfqpoint{1.816105in}{0.697724in}}%
\pgfpathlineto{\pgfqpoint{1.816105in}{0.697724in}}%
\pgfusepath{stroke,fill}%
\end{pgfscope}%
\begin{pgfscope}%
\pgfpathrectangle{\pgfqpoint{0.548058in}{0.516222in}}{\pgfqpoint{1.739582in}{1.783528in}} %
\pgfusepath{clip}%
\pgfsetbuttcap%
\pgfsetroundjoin%
\definecolor{currentfill}{rgb}{0.298039,0.447059,0.690196}%
\pgfsetfillcolor{currentfill}%
\pgfsetlinewidth{0.240900pt}%
\definecolor{currentstroke}{rgb}{1.000000,1.000000,1.000000}%
\pgfsetstrokecolor{currentstroke}%
\pgfsetdash{}{0pt}%
\pgfpathmoveto{\pgfqpoint{1.051454in}{1.032772in}}%
\pgfpathcurveto{\pgfqpoint{1.059690in}{1.032772in}}{\pgfqpoint{1.067590in}{1.036045in}}{\pgfqpoint{1.073414in}{1.041869in}}%
\pgfpathcurveto{\pgfqpoint{1.079238in}{1.047693in}}{\pgfqpoint{1.082510in}{1.055593in}}{\pgfqpoint{1.082510in}{1.063829in}}%
\pgfpathcurveto{\pgfqpoint{1.082510in}{1.072065in}}{\pgfqpoint{1.079238in}{1.079965in}}{\pgfqpoint{1.073414in}{1.085789in}}%
\pgfpathcurveto{\pgfqpoint{1.067590in}{1.091613in}}{\pgfqpoint{1.059690in}{1.094885in}}{\pgfqpoint{1.051454in}{1.094885in}}%
\pgfpathcurveto{\pgfqpoint{1.043218in}{1.094885in}}{\pgfqpoint{1.035317in}{1.091613in}}{\pgfqpoint{1.029494in}{1.085789in}}%
\pgfpathcurveto{\pgfqpoint{1.023670in}{1.079965in}}{\pgfqpoint{1.020397in}{1.072065in}}{\pgfqpoint{1.020397in}{1.063829in}}%
\pgfpathcurveto{\pgfqpoint{1.020397in}{1.055593in}}{\pgfqpoint{1.023670in}{1.047693in}}{\pgfqpoint{1.029494in}{1.041869in}}%
\pgfpathcurveto{\pgfqpoint{1.035317in}{1.036045in}}{\pgfqpoint{1.043218in}{1.032772in}}{\pgfqpoint{1.051454in}{1.032772in}}%
\pgfpathlineto{\pgfqpoint{1.051454in}{1.032772in}}%
\pgfusepath{stroke,fill}%
\end{pgfscope}%
\begin{pgfscope}%
\pgfpathrectangle{\pgfqpoint{0.548058in}{0.516222in}}{\pgfqpoint{1.739582in}{1.783528in}} %
\pgfusepath{clip}%
\pgfsetbuttcap%
\pgfsetroundjoin%
\definecolor{currentfill}{rgb}{0.298039,0.447059,0.690196}%
\pgfsetfillcolor{currentfill}%
\pgfsetlinewidth{0.240900pt}%
\definecolor{currentstroke}{rgb}{1.000000,1.000000,1.000000}%
\pgfsetstrokecolor{currentstroke}%
\pgfsetdash{}{0pt}%
\pgfpathmoveto{\pgfqpoint{1.433780in}{1.450628in}}%
\pgfpathcurveto{\pgfqpoint{1.442016in}{1.450628in}}{\pgfqpoint{1.449916in}{1.453900in}}{\pgfqpoint{1.455740in}{1.459724in}}%
\pgfpathcurveto{\pgfqpoint{1.461564in}{1.465548in}}{\pgfqpoint{1.464836in}{1.473448in}}{\pgfqpoint{1.464836in}{1.481684in}}%
\pgfpathcurveto{\pgfqpoint{1.464836in}{1.489920in}}{\pgfqpoint{1.461564in}{1.497820in}}{\pgfqpoint{1.455740in}{1.503644in}}%
\pgfpathcurveto{\pgfqpoint{1.449916in}{1.509468in}}{\pgfqpoint{1.442016in}{1.512741in}}{\pgfqpoint{1.433780in}{1.512741in}}%
\pgfpathcurveto{\pgfqpoint{1.425543in}{1.512741in}}{\pgfqpoint{1.417643in}{1.509468in}}{\pgfqpoint{1.411819in}{1.503644in}}%
\pgfpathcurveto{\pgfqpoint{1.405995in}{1.497820in}}{\pgfqpoint{1.402723in}{1.489920in}}{\pgfqpoint{1.402723in}{1.481684in}}%
\pgfpathcurveto{\pgfqpoint{1.402723in}{1.473448in}}{\pgfqpoint{1.405995in}{1.465548in}}{\pgfqpoint{1.411819in}{1.459724in}}%
\pgfpathcurveto{\pgfqpoint{1.417643in}{1.453900in}}{\pgfqpoint{1.425543in}{1.450628in}}{\pgfqpoint{1.433780in}{1.450628in}}%
\pgfpathlineto{\pgfqpoint{1.433780in}{1.450628in}}%
\pgfusepath{stroke,fill}%
\end{pgfscope}%
\begin{pgfscope}%
\pgfpathrectangle{\pgfqpoint{0.548058in}{0.516222in}}{\pgfqpoint{1.739582in}{1.783528in}} %
\pgfusepath{clip}%
\pgfsetbuttcap%
\pgfsetroundjoin%
\definecolor{currentfill}{rgb}{0.298039,0.447059,0.690196}%
\pgfsetfillcolor{currentfill}%
\pgfsetlinewidth{0.240900pt}%
\definecolor{currentstroke}{rgb}{1.000000,1.000000,1.000000}%
\pgfsetstrokecolor{currentstroke}%
\pgfsetdash{}{0pt}%
\pgfpathmoveto{\pgfqpoint{1.370059in}{1.604775in}}%
\pgfpathcurveto{\pgfqpoint{1.378295in}{1.604775in}}{\pgfqpoint{1.386195in}{1.608048in}}{\pgfqpoint{1.392019in}{1.613872in}}%
\pgfpathcurveto{\pgfqpoint{1.397843in}{1.619695in}}{\pgfqpoint{1.401115in}{1.627596in}}{\pgfqpoint{1.401115in}{1.635832in}}%
\pgfpathcurveto{\pgfqpoint{1.401115in}{1.644068in}}{\pgfqpoint{1.397843in}{1.651968in}}{\pgfqpoint{1.392019in}{1.657792in}}%
\pgfpathcurveto{\pgfqpoint{1.386195in}{1.663616in}}{\pgfqpoint{1.378295in}{1.666888in}}{\pgfqpoint{1.370059in}{1.666888in}}%
\pgfpathcurveto{\pgfqpoint{1.361822in}{1.666888in}}{\pgfqpoint{1.353922in}{1.663616in}}{\pgfqpoint{1.348098in}{1.657792in}}%
\pgfpathcurveto{\pgfqpoint{1.342274in}{1.651968in}}{\pgfqpoint{1.339002in}{1.644068in}}{\pgfqpoint{1.339002in}{1.635832in}}%
\pgfpathcurveto{\pgfqpoint{1.339002in}{1.627596in}}{\pgfqpoint{1.342274in}{1.619695in}}{\pgfqpoint{1.348098in}{1.613872in}}%
\pgfpathcurveto{\pgfqpoint{1.353922in}{1.608048in}}{\pgfqpoint{1.361822in}{1.604775in}}{\pgfqpoint{1.370059in}{1.604775in}}%
\pgfpathlineto{\pgfqpoint{1.370059in}{1.604775in}}%
\pgfusepath{stroke,fill}%
\end{pgfscope}%
\begin{pgfscope}%
\pgfpathrectangle{\pgfqpoint{0.548058in}{0.516222in}}{\pgfqpoint{1.739582in}{1.783528in}} %
\pgfusepath{clip}%
\pgfsetbuttcap%
\pgfsetroundjoin%
\definecolor{currentfill}{rgb}{0.298039,0.447059,0.690196}%
\pgfsetfillcolor{currentfill}%
\pgfsetlinewidth{0.240900pt}%
\definecolor{currentstroke}{rgb}{1.000000,1.000000,1.000000}%
\pgfsetstrokecolor{currentstroke}%
\pgfsetdash{}{0pt}%
\pgfpathmoveto{\pgfqpoint{1.083314in}{0.927035in}}%
\pgfpathcurveto{\pgfqpoint{1.091551in}{0.927035in}}{\pgfqpoint{1.099451in}{0.930307in}}{\pgfqpoint{1.105275in}{0.936131in}}%
\pgfpathcurveto{\pgfqpoint{1.111098in}{0.941955in}}{\pgfqpoint{1.114371in}{0.949855in}}{\pgfqpoint{1.114371in}{0.958091in}}%
\pgfpathcurveto{\pgfqpoint{1.114371in}{0.966328in}}{\pgfqpoint{1.111098in}{0.974228in}}{\pgfqpoint{1.105275in}{0.980051in}}%
\pgfpathcurveto{\pgfqpoint{1.099451in}{0.985875in}}{\pgfqpoint{1.091551in}{0.989148in}}{\pgfqpoint{1.083314in}{0.989148in}}%
\pgfpathcurveto{\pgfqpoint{1.075078in}{0.989148in}}{\pgfqpoint{1.067178in}{0.985875in}}{\pgfqpoint{1.061354in}{0.980051in}}%
\pgfpathcurveto{\pgfqpoint{1.055530in}{0.974228in}}{\pgfqpoint{1.052258in}{0.966328in}}{\pgfqpoint{1.052258in}{0.958091in}}%
\pgfpathcurveto{\pgfqpoint{1.052258in}{0.949855in}}{\pgfqpoint{1.055530in}{0.941955in}}{\pgfqpoint{1.061354in}{0.936131in}}%
\pgfpathcurveto{\pgfqpoint{1.067178in}{0.930307in}}{\pgfqpoint{1.075078in}{0.927035in}}{\pgfqpoint{1.083314in}{0.927035in}}%
\pgfpathlineto{\pgfqpoint{1.083314in}{0.927035in}}%
\pgfusepath{stroke,fill}%
\end{pgfscope}%
\begin{pgfscope}%
\pgfpathrectangle{\pgfqpoint{0.548058in}{0.516222in}}{\pgfqpoint{1.739582in}{1.783528in}} %
\pgfusepath{clip}%
\pgfsetbuttcap%
\pgfsetroundjoin%
\definecolor{currentfill}{rgb}{0.298039,0.447059,0.690196}%
\pgfsetfillcolor{currentfill}%
\pgfsetlinewidth{0.240900pt}%
\definecolor{currentstroke}{rgb}{1.000000,1.000000,1.000000}%
\pgfsetstrokecolor{currentstroke}%
\pgfsetdash{}{0pt}%
\pgfpathmoveto{\pgfqpoint{1.497501in}{1.102840in}}%
\pgfpathcurveto{\pgfqpoint{1.505737in}{1.102840in}}{\pgfqpoint{1.513637in}{1.106112in}}{\pgfqpoint{1.519461in}{1.111936in}}%
\pgfpathcurveto{\pgfqpoint{1.525285in}{1.117760in}}{\pgfqpoint{1.528557in}{1.125660in}}{\pgfqpoint{1.528557in}{1.133896in}}%
\pgfpathcurveto{\pgfqpoint{1.528557in}{1.142132in}}{\pgfqpoint{1.525285in}{1.150032in}}{\pgfqpoint{1.519461in}{1.155856in}}%
\pgfpathcurveto{\pgfqpoint{1.513637in}{1.161680in}}{\pgfqpoint{1.505737in}{1.164953in}}{\pgfqpoint{1.497501in}{1.164953in}}%
\pgfpathcurveto{\pgfqpoint{1.489264in}{1.164953in}}{\pgfqpoint{1.481364in}{1.161680in}}{\pgfqpoint{1.475540in}{1.155856in}}%
\pgfpathcurveto{\pgfqpoint{1.469716in}{1.150032in}}{\pgfqpoint{1.466444in}{1.142132in}}{\pgfqpoint{1.466444in}{1.133896in}}%
\pgfpathcurveto{\pgfqpoint{1.466444in}{1.125660in}}{\pgfqpoint{1.469716in}{1.117760in}}{\pgfqpoint{1.475540in}{1.111936in}}%
\pgfpathcurveto{\pgfqpoint{1.481364in}{1.106112in}}{\pgfqpoint{1.489264in}{1.102840in}}{\pgfqpoint{1.497501in}{1.102840in}}%
\pgfpathlineto{\pgfqpoint{1.497501in}{1.102840in}}%
\pgfusepath{stroke,fill}%
\end{pgfscope}%
\begin{pgfscope}%
\pgfpathrectangle{\pgfqpoint{0.548058in}{0.516222in}}{\pgfqpoint{1.739582in}{1.783528in}} %
\pgfusepath{clip}%
\pgfsetbuttcap%
\pgfsetroundjoin%
\definecolor{currentfill}{rgb}{0.298039,0.447059,0.690196}%
\pgfsetfillcolor{currentfill}%
\pgfsetlinewidth{0.240900pt}%
\definecolor{currentstroke}{rgb}{1.000000,1.000000,1.000000}%
\pgfsetstrokecolor{currentstroke}%
\pgfsetdash{}{0pt}%
\pgfpathmoveto{\pgfqpoint{1.688664in}{0.695176in}}%
\pgfpathcurveto{\pgfqpoint{1.696900in}{0.695176in}}{\pgfqpoint{1.704800in}{0.698448in}}{\pgfqpoint{1.710624in}{0.704272in}}%
\pgfpathcurveto{\pgfqpoint{1.716448in}{0.710096in}}{\pgfqpoint{1.719720in}{0.717996in}}{\pgfqpoint{1.719720in}{0.726233in}}%
\pgfpathcurveto{\pgfqpoint{1.719720in}{0.734469in}}{\pgfqpoint{1.716448in}{0.742369in}}{\pgfqpoint{1.710624in}{0.748193in}}%
\pgfpathcurveto{\pgfqpoint{1.704800in}{0.754017in}}{\pgfqpoint{1.696900in}{0.757289in}}{\pgfqpoint{1.688664in}{0.757289in}}%
\pgfpathcurveto{\pgfqpoint{1.680427in}{0.757289in}}{\pgfqpoint{1.672527in}{0.754017in}}{\pgfqpoint{1.666703in}{0.748193in}}%
\pgfpathcurveto{\pgfqpoint{1.660879in}{0.742369in}}{\pgfqpoint{1.657607in}{0.734469in}}{\pgfqpoint{1.657607in}{0.726233in}}%
\pgfpathcurveto{\pgfqpoint{1.657607in}{0.717996in}}{\pgfqpoint{1.660879in}{0.710096in}}{\pgfqpoint{1.666703in}{0.704272in}}%
\pgfpathcurveto{\pgfqpoint{1.672527in}{0.698448in}}{\pgfqpoint{1.680427in}{0.695176in}}{\pgfqpoint{1.688664in}{0.695176in}}%
\pgfpathlineto{\pgfqpoint{1.688664in}{0.695176in}}%
\pgfusepath{stroke,fill}%
\end{pgfscope}%
\begin{pgfscope}%
\pgfpathrectangle{\pgfqpoint{0.548058in}{0.516222in}}{\pgfqpoint{1.739582in}{1.783528in}} %
\pgfusepath{clip}%
\pgfsetbuttcap%
\pgfsetroundjoin%
\definecolor{currentfill}{rgb}{0.298039,0.447059,0.690196}%
\pgfsetfillcolor{currentfill}%
\pgfsetlinewidth{0.240900pt}%
\definecolor{currentstroke}{rgb}{1.000000,1.000000,1.000000}%
\pgfsetstrokecolor{currentstroke}%
\pgfsetdash{}{0pt}%
\pgfpathmoveto{\pgfqpoint{0.987733in}{1.594584in}}%
\pgfpathcurveto{\pgfqpoint{0.995969in}{1.594584in}}{\pgfqpoint{1.003869in}{1.597856in}}{\pgfqpoint{1.009693in}{1.603680in}}%
\pgfpathcurveto{\pgfqpoint{1.015517in}{1.609504in}}{\pgfqpoint{1.018789in}{1.617404in}}{\pgfqpoint{1.018789in}{1.625640in}}%
\pgfpathcurveto{\pgfqpoint{1.018789in}{1.633876in}}{\pgfqpoint{1.015517in}{1.641777in}}{\pgfqpoint{1.009693in}{1.647600in}}%
\pgfpathcurveto{\pgfqpoint{1.003869in}{1.653424in}}{\pgfqpoint{0.995969in}{1.656697in}}{\pgfqpoint{0.987733in}{1.656697in}}%
\pgfpathcurveto{\pgfqpoint{0.979497in}{1.656697in}}{\pgfqpoint{0.971597in}{1.653424in}}{\pgfqpoint{0.965773in}{1.647600in}}%
\pgfpathcurveto{\pgfqpoint{0.959949in}{1.641777in}}{\pgfqpoint{0.956676in}{1.633876in}}{\pgfqpoint{0.956676in}{1.625640in}}%
\pgfpathcurveto{\pgfqpoint{0.956676in}{1.617404in}}{\pgfqpoint{0.959949in}{1.609504in}}{\pgfqpoint{0.965773in}{1.603680in}}%
\pgfpathcurveto{\pgfqpoint{0.971597in}{1.597856in}}{\pgfqpoint{0.979497in}{1.594584in}}{\pgfqpoint{0.987733in}{1.594584in}}%
\pgfpathlineto{\pgfqpoint{0.987733in}{1.594584in}}%
\pgfusepath{stroke,fill}%
\end{pgfscope}%
\begin{pgfscope}%
\pgfpathrectangle{\pgfqpoint{0.548058in}{0.516222in}}{\pgfqpoint{1.739582in}{1.783528in}} %
\pgfusepath{clip}%
\pgfsetbuttcap%
\pgfsetroundjoin%
\definecolor{currentfill}{rgb}{0.298039,0.447059,0.690196}%
\pgfsetfillcolor{currentfill}%
\pgfsetlinewidth{0.240900pt}%
\definecolor{currentstroke}{rgb}{1.000000,1.000000,1.000000}%
\pgfsetstrokecolor{currentstroke}%
\pgfsetdash{}{0pt}%
\pgfpathmoveto{\pgfqpoint{1.401919in}{0.921939in}}%
\pgfpathcurveto{\pgfqpoint{1.410155in}{0.921939in}}{\pgfqpoint{1.418055in}{0.925211in}}{\pgfqpoint{1.423879in}{0.931035in}}%
\pgfpathcurveto{\pgfqpoint{1.429703in}{0.936859in}}{\pgfqpoint{1.432976in}{0.944759in}}{\pgfqpoint{1.432976in}{0.952995in}}%
\pgfpathcurveto{\pgfqpoint{1.432976in}{0.961232in}}{\pgfqpoint{1.429703in}{0.969132in}}{\pgfqpoint{1.423879in}{0.974956in}}%
\pgfpathcurveto{\pgfqpoint{1.418055in}{0.980780in}}{\pgfqpoint{1.410155in}{0.984052in}}{\pgfqpoint{1.401919in}{0.984052in}}%
\pgfpathcurveto{\pgfqpoint{1.393683in}{0.984052in}}{\pgfqpoint{1.385783in}{0.980780in}}{\pgfqpoint{1.379959in}{0.974956in}}%
\pgfpathcurveto{\pgfqpoint{1.374135in}{0.969132in}}{\pgfqpoint{1.370863in}{0.961232in}}{\pgfqpoint{1.370863in}{0.952995in}}%
\pgfpathcurveto{\pgfqpoint{1.370863in}{0.944759in}}{\pgfqpoint{1.374135in}{0.936859in}}{\pgfqpoint{1.379959in}{0.931035in}}%
\pgfpathcurveto{\pgfqpoint{1.385783in}{0.925211in}}{\pgfqpoint{1.393683in}{0.921939in}}{\pgfqpoint{1.401919in}{0.921939in}}%
\pgfpathlineto{\pgfqpoint{1.401919in}{0.921939in}}%
\pgfusepath{stroke,fill}%
\end{pgfscope}%
\begin{pgfscope}%
\pgfpathrectangle{\pgfqpoint{0.548058in}{0.516222in}}{\pgfqpoint{1.739582in}{1.783528in}} %
\pgfusepath{clip}%
\pgfsetbuttcap%
\pgfsetroundjoin%
\definecolor{currentfill}{rgb}{0.298039,0.447059,0.690196}%
\pgfsetfillcolor{currentfill}%
\pgfsetlinewidth{0.240900pt}%
\definecolor{currentstroke}{rgb}{1.000000,1.000000,1.000000}%
\pgfsetstrokecolor{currentstroke}%
\pgfsetdash{}{0pt}%
\pgfpathmoveto{\pgfqpoint{1.943547in}{0.798366in}}%
\pgfpathcurveto{\pgfqpoint{1.951784in}{0.798366in}}{\pgfqpoint{1.959684in}{0.801638in}}{\pgfqpoint{1.965508in}{0.807462in}}%
\pgfpathcurveto{\pgfqpoint{1.971332in}{0.813286in}}{\pgfqpoint{1.974604in}{0.821186in}}{\pgfqpoint{1.974604in}{0.829422in}}%
\pgfpathcurveto{\pgfqpoint{1.974604in}{0.837659in}}{\pgfqpoint{1.971332in}{0.845559in}}{\pgfqpoint{1.965508in}{0.851383in}}%
\pgfpathcurveto{\pgfqpoint{1.959684in}{0.857207in}}{\pgfqpoint{1.951784in}{0.860479in}}{\pgfqpoint{1.943547in}{0.860479in}}%
\pgfpathcurveto{\pgfqpoint{1.935311in}{0.860479in}}{\pgfqpoint{1.927411in}{0.857207in}}{\pgfqpoint{1.921587in}{0.851383in}}%
\pgfpathcurveto{\pgfqpoint{1.915763in}{0.845559in}}{\pgfqpoint{1.912491in}{0.837659in}}{\pgfqpoint{1.912491in}{0.829422in}}%
\pgfpathcurveto{\pgfqpoint{1.912491in}{0.821186in}}{\pgfqpoint{1.915763in}{0.813286in}}{\pgfqpoint{1.921587in}{0.807462in}}%
\pgfpathcurveto{\pgfqpoint{1.927411in}{0.801638in}}{\pgfqpoint{1.935311in}{0.798366in}}{\pgfqpoint{1.943547in}{0.798366in}}%
\pgfpathlineto{\pgfqpoint{1.943547in}{0.798366in}}%
\pgfusepath{stroke,fill}%
\end{pgfscope}%
\begin{pgfscope}%
\pgfpathrectangle{\pgfqpoint{0.548058in}{0.516222in}}{\pgfqpoint{1.739582in}{1.783528in}} %
\pgfusepath{clip}%
\pgfsetbuttcap%
\pgfsetroundjoin%
\definecolor{currentfill}{rgb}{0.298039,0.447059,0.690196}%
\pgfsetfillcolor{currentfill}%
\pgfsetlinewidth{0.240900pt}%
\definecolor{currentstroke}{rgb}{1.000000,1.000000,1.000000}%
\pgfsetstrokecolor{currentstroke}%
\pgfsetdash{}{0pt}%
\pgfpathmoveto{\pgfqpoint{1.242617in}{1.146154in}}%
\pgfpathcurveto{\pgfqpoint{1.250853in}{1.146154in}}{\pgfqpoint{1.258753in}{1.149426in}}{\pgfqpoint{1.264577in}{1.155250in}}%
\pgfpathcurveto{\pgfqpoint{1.270401in}{1.161074in}}{\pgfqpoint{1.273673in}{1.168974in}}{\pgfqpoint{1.273673in}{1.177210in}}%
\pgfpathcurveto{\pgfqpoint{1.273673in}{1.185447in}}{\pgfqpoint{1.270401in}{1.193347in}}{\pgfqpoint{1.264577in}{1.199171in}}%
\pgfpathcurveto{\pgfqpoint{1.258753in}{1.204995in}}{\pgfqpoint{1.250853in}{1.208267in}}{\pgfqpoint{1.242617in}{1.208267in}}%
\pgfpathcurveto{\pgfqpoint{1.234380in}{1.208267in}}{\pgfqpoint{1.226480in}{1.204995in}}{\pgfqpoint{1.220656in}{1.199171in}}%
\pgfpathcurveto{\pgfqpoint{1.214833in}{1.193347in}}{\pgfqpoint{1.211560in}{1.185447in}}{\pgfqpoint{1.211560in}{1.177210in}}%
\pgfpathcurveto{\pgfqpoint{1.211560in}{1.168974in}}{\pgfqpoint{1.214833in}{1.161074in}}{\pgfqpoint{1.220656in}{1.155250in}}%
\pgfpathcurveto{\pgfqpoint{1.226480in}{1.149426in}}{\pgfqpoint{1.234380in}{1.146154in}}{\pgfqpoint{1.242617in}{1.146154in}}%
\pgfpathlineto{\pgfqpoint{1.242617in}{1.146154in}}%
\pgfusepath{stroke,fill}%
\end{pgfscope}%
\begin{pgfscope}%
\pgfpathrectangle{\pgfqpoint{0.548058in}{0.516222in}}{\pgfqpoint{1.739582in}{1.783528in}} %
\pgfusepath{clip}%
\pgfsetbuttcap%
\pgfsetroundjoin%
\definecolor{currentfill}{rgb}{0.298039,0.447059,0.690196}%
\pgfsetfillcolor{currentfill}%
\pgfsetlinewidth{0.240900pt}%
\definecolor{currentstroke}{rgb}{1.000000,1.000000,1.000000}%
\pgfsetstrokecolor{currentstroke}%
\pgfsetdash{}{0pt}%
\pgfpathmoveto{\pgfqpoint{0.828430in}{1.730896in}}%
\pgfpathcurveto{\pgfqpoint{0.836667in}{1.730896in}}{\pgfqpoint{0.844567in}{1.734168in}}{\pgfqpoint{0.850391in}{1.739992in}}%
\pgfpathcurveto{\pgfqpoint{0.856215in}{1.745816in}}{\pgfqpoint{0.859487in}{1.753716in}}{\pgfqpoint{0.859487in}{1.761953in}}%
\pgfpathcurveto{\pgfqpoint{0.859487in}{1.770189in}}{\pgfqpoint{0.856215in}{1.778089in}}{\pgfqpoint{0.850391in}{1.783913in}}%
\pgfpathcurveto{\pgfqpoint{0.844567in}{1.789737in}}{\pgfqpoint{0.836667in}{1.793009in}}{\pgfqpoint{0.828430in}{1.793009in}}%
\pgfpathcurveto{\pgfqpoint{0.820194in}{1.793009in}}{\pgfqpoint{0.812294in}{1.789737in}}{\pgfqpoint{0.806470in}{1.783913in}}%
\pgfpathcurveto{\pgfqpoint{0.800646in}{1.778089in}}{\pgfqpoint{0.797374in}{1.770189in}}{\pgfqpoint{0.797374in}{1.761953in}}%
\pgfpathcurveto{\pgfqpoint{0.797374in}{1.753716in}}{\pgfqpoint{0.800646in}{1.745816in}}{\pgfqpoint{0.806470in}{1.739992in}}%
\pgfpathcurveto{\pgfqpoint{0.812294in}{1.734168in}}{\pgfqpoint{0.820194in}{1.730896in}}{\pgfqpoint{0.828430in}{1.730896in}}%
\pgfpathlineto{\pgfqpoint{0.828430in}{1.730896in}}%
\pgfusepath{stroke,fill}%
\end{pgfscope}%
\begin{pgfscope}%
\pgfpathrectangle{\pgfqpoint{0.548058in}{0.516222in}}{\pgfqpoint{1.739582in}{1.783528in}} %
\pgfusepath{clip}%
\pgfsetbuttcap%
\pgfsetroundjoin%
\definecolor{currentfill}{rgb}{0.298039,0.447059,0.690196}%
\pgfsetfillcolor{currentfill}%
\pgfsetlinewidth{0.240900pt}%
\definecolor{currentstroke}{rgb}{1.000000,1.000000,1.000000}%
\pgfsetstrokecolor{currentstroke}%
\pgfsetdash{}{0pt}%
\pgfpathmoveto{\pgfqpoint{1.338198in}{0.807284in}}%
\pgfpathcurveto{\pgfqpoint{1.346434in}{0.807284in}}{\pgfqpoint{1.354335in}{0.810556in}}{\pgfqpoint{1.360158in}{0.816380in}}%
\pgfpathcurveto{\pgfqpoint{1.365982in}{0.822204in}}{\pgfqpoint{1.369255in}{0.830104in}}{\pgfqpoint{1.369255in}{0.838340in}}%
\pgfpathcurveto{\pgfqpoint{1.369255in}{0.846576in}}{\pgfqpoint{1.365982in}{0.854476in}}{\pgfqpoint{1.360158in}{0.860300in}}%
\pgfpathcurveto{\pgfqpoint{1.354335in}{0.866124in}}{\pgfqpoint{1.346434in}{0.869397in}}{\pgfqpoint{1.338198in}{0.869397in}}%
\pgfpathcurveto{\pgfqpoint{1.329962in}{0.869397in}}{\pgfqpoint{1.322062in}{0.866124in}}{\pgfqpoint{1.316238in}{0.860300in}}%
\pgfpathcurveto{\pgfqpoint{1.310414in}{0.854476in}}{\pgfqpoint{1.307142in}{0.846576in}}{\pgfqpoint{1.307142in}{0.838340in}}%
\pgfpathcurveto{\pgfqpoint{1.307142in}{0.830104in}}{\pgfqpoint{1.310414in}{0.822204in}}{\pgfqpoint{1.316238in}{0.816380in}}%
\pgfpathcurveto{\pgfqpoint{1.322062in}{0.810556in}}{\pgfqpoint{1.329962in}{0.807284in}}{\pgfqpoint{1.338198in}{0.807284in}}%
\pgfpathlineto{\pgfqpoint{1.338198in}{0.807284in}}%
\pgfusepath{stroke,fill}%
\end{pgfscope}%
\begin{pgfscope}%
\pgfpathrectangle{\pgfqpoint{0.548058in}{0.516222in}}{\pgfqpoint{1.739582in}{1.783528in}} %
\pgfusepath{clip}%
\pgfsetbuttcap%
\pgfsetroundjoin%
\definecolor{currentfill}{rgb}{0.298039,0.447059,0.690196}%
\pgfsetfillcolor{currentfill}%
\pgfsetlinewidth{0.240900pt}%
\definecolor{currentstroke}{rgb}{1.000000,1.000000,1.000000}%
\pgfsetstrokecolor{currentstroke}%
\pgfsetdash{}{0pt}%
\pgfpathmoveto{\pgfqpoint{1.911687in}{0.776709in}}%
\pgfpathcurveto{\pgfqpoint{1.919923in}{0.776709in}}{\pgfqpoint{1.927823in}{0.779981in}}{\pgfqpoint{1.933647in}{0.785805in}}%
\pgfpathcurveto{\pgfqpoint{1.939471in}{0.791629in}}{\pgfqpoint{1.942743in}{0.799529in}}{\pgfqpoint{1.942743in}{0.807765in}}%
\pgfpathcurveto{\pgfqpoint{1.942743in}{0.816002in}}{\pgfqpoint{1.939471in}{0.823902in}}{\pgfqpoint{1.933647in}{0.829726in}}%
\pgfpathcurveto{\pgfqpoint{1.927823in}{0.835550in}}{\pgfqpoint{1.919923in}{0.838822in}}{\pgfqpoint{1.911687in}{0.838822in}}%
\pgfpathcurveto{\pgfqpoint{1.903451in}{0.838822in}}{\pgfqpoint{1.895551in}{0.835550in}}{\pgfqpoint{1.889727in}{0.829726in}}%
\pgfpathcurveto{\pgfqpoint{1.883903in}{0.823902in}}{\pgfqpoint{1.880630in}{0.816002in}}{\pgfqpoint{1.880630in}{0.807765in}}%
\pgfpathcurveto{\pgfqpoint{1.880630in}{0.799529in}}{\pgfqpoint{1.883903in}{0.791629in}}{\pgfqpoint{1.889727in}{0.785805in}}%
\pgfpathcurveto{\pgfqpoint{1.895551in}{0.779981in}}{\pgfqpoint{1.903451in}{0.776709in}}{\pgfqpoint{1.911687in}{0.776709in}}%
\pgfpathlineto{\pgfqpoint{1.911687in}{0.776709in}}%
\pgfusepath{stroke,fill}%
\end{pgfscope}%
\begin{pgfscope}%
\pgfpathrectangle{\pgfqpoint{0.548058in}{0.516222in}}{\pgfqpoint{1.739582in}{1.783528in}} %
\pgfusepath{clip}%
\pgfsetbuttcap%
\pgfsetroundjoin%
\definecolor{currentfill}{rgb}{0.298039,0.447059,0.690196}%
\pgfsetfillcolor{currentfill}%
\pgfsetlinewidth{0.240900pt}%
\definecolor{currentstroke}{rgb}{1.000000,1.000000,1.000000}%
\pgfsetstrokecolor{currentstroke}%
\pgfsetdash{}{0pt}%
\pgfpathmoveto{\pgfqpoint{1.720524in}{1.116853in}}%
\pgfpathcurveto{\pgfqpoint{1.728760in}{1.116853in}}{\pgfqpoint{1.736660in}{1.120125in}}{\pgfqpoint{1.742484in}{1.125949in}}%
\pgfpathcurveto{\pgfqpoint{1.748308in}{1.131773in}}{\pgfqpoint{1.751580in}{1.139673in}}{\pgfqpoint{1.751580in}{1.147910in}}%
\pgfpathcurveto{\pgfqpoint{1.751580in}{1.156146in}}{\pgfqpoint{1.748308in}{1.164046in}}{\pgfqpoint{1.742484in}{1.169870in}}%
\pgfpathcurveto{\pgfqpoint{1.736660in}{1.175694in}}{\pgfqpoint{1.728760in}{1.178966in}}{\pgfqpoint{1.720524in}{1.178966in}}%
\pgfpathcurveto{\pgfqpoint{1.712288in}{1.178966in}}{\pgfqpoint{1.704388in}{1.175694in}}{\pgfqpoint{1.698564in}{1.169870in}}%
\pgfpathcurveto{\pgfqpoint{1.692740in}{1.164046in}}{\pgfqpoint{1.689467in}{1.156146in}}{\pgfqpoint{1.689467in}{1.147910in}}%
\pgfpathcurveto{\pgfqpoint{1.689467in}{1.139673in}}{\pgfqpoint{1.692740in}{1.131773in}}{\pgfqpoint{1.698564in}{1.125949in}}%
\pgfpathcurveto{\pgfqpoint{1.704388in}{1.120125in}}{\pgfqpoint{1.712288in}{1.116853in}}{\pgfqpoint{1.720524in}{1.116853in}}%
\pgfpathlineto{\pgfqpoint{1.720524in}{1.116853in}}%
\pgfusepath{stroke,fill}%
\end{pgfscope}%
\begin{pgfscope}%
\pgfpathrectangle{\pgfqpoint{0.548058in}{0.516222in}}{\pgfqpoint{1.739582in}{1.783528in}} %
\pgfusepath{clip}%
\pgfsetbuttcap%
\pgfsetroundjoin%
\definecolor{currentfill}{rgb}{0.298039,0.447059,0.690196}%
\pgfsetfillcolor{currentfill}%
\pgfsetlinewidth{0.240900pt}%
\definecolor{currentstroke}{rgb}{1.000000,1.000000,1.000000}%
\pgfsetstrokecolor{currentstroke}%
\pgfsetdash{}{0pt}%
\pgfpathmoveto{\pgfqpoint{1.465640in}{0.911747in}}%
\pgfpathcurveto{\pgfqpoint{1.473876in}{0.911747in}}{\pgfqpoint{1.481776in}{0.915020in}}{\pgfqpoint{1.487600in}{0.920844in}}%
\pgfpathcurveto{\pgfqpoint{1.493424in}{0.926668in}}{\pgfqpoint{1.496697in}{0.934568in}}{\pgfqpoint{1.496697in}{0.942804in}}%
\pgfpathcurveto{\pgfqpoint{1.496697in}{0.951040in}}{\pgfqpoint{1.493424in}{0.958940in}}{\pgfqpoint{1.487600in}{0.964764in}}%
\pgfpathcurveto{\pgfqpoint{1.481776in}{0.970588in}}{\pgfqpoint{1.473876in}{0.973860in}}{\pgfqpoint{1.465640in}{0.973860in}}%
\pgfpathcurveto{\pgfqpoint{1.457404in}{0.973860in}}{\pgfqpoint{1.449504in}{0.970588in}}{\pgfqpoint{1.443680in}{0.964764in}}%
\pgfpathcurveto{\pgfqpoint{1.437856in}{0.958940in}}{\pgfqpoint{1.434584in}{0.951040in}}{\pgfqpoint{1.434584in}{0.942804in}}%
\pgfpathcurveto{\pgfqpoint{1.434584in}{0.934568in}}{\pgfqpoint{1.437856in}{0.926668in}}{\pgfqpoint{1.443680in}{0.920844in}}%
\pgfpathcurveto{\pgfqpoint{1.449504in}{0.915020in}}{\pgfqpoint{1.457404in}{0.911747in}}{\pgfqpoint{1.465640in}{0.911747in}}%
\pgfpathlineto{\pgfqpoint{1.465640in}{0.911747in}}%
\pgfusepath{stroke,fill}%
\end{pgfscope}%
\begin{pgfscope}%
\pgfpathrectangle{\pgfqpoint{0.548058in}{0.516222in}}{\pgfqpoint{1.739582in}{1.783528in}} %
\pgfusepath{clip}%
\pgfsetbuttcap%
\pgfsetroundjoin%
\definecolor{currentfill}{rgb}{0.298039,0.447059,0.690196}%
\pgfsetfillcolor{currentfill}%
\pgfsetlinewidth{0.240900pt}%
\definecolor{currentstroke}{rgb}{1.000000,1.000000,1.000000}%
\pgfsetstrokecolor{currentstroke}%
\pgfsetdash{}{0pt}%
\pgfpathmoveto{\pgfqpoint{0.955872in}{1.272275in}}%
\pgfpathcurveto{\pgfqpoint{0.964109in}{1.272275in}}{\pgfqpoint{0.972009in}{1.275547in}}{\pgfqpoint{0.977833in}{1.281371in}}%
\pgfpathcurveto{\pgfqpoint{0.983657in}{1.287195in}}{\pgfqpoint{0.986929in}{1.295095in}}{\pgfqpoint{0.986929in}{1.303331in}}%
\pgfpathcurveto{\pgfqpoint{0.986929in}{1.311568in}}{\pgfqpoint{0.983657in}{1.319468in}}{\pgfqpoint{0.977833in}{1.325292in}}%
\pgfpathcurveto{\pgfqpoint{0.972009in}{1.331115in}}{\pgfqpoint{0.964109in}{1.334388in}}{\pgfqpoint{0.955872in}{1.334388in}}%
\pgfpathcurveto{\pgfqpoint{0.947636in}{1.334388in}}{\pgfqpoint{0.939736in}{1.331115in}}{\pgfqpoint{0.933912in}{1.325292in}}%
\pgfpathcurveto{\pgfqpoint{0.928088in}{1.319468in}}{\pgfqpoint{0.924816in}{1.311568in}}{\pgfqpoint{0.924816in}{1.303331in}}%
\pgfpathcurveto{\pgfqpoint{0.924816in}{1.295095in}}{\pgfqpoint{0.928088in}{1.287195in}}{\pgfqpoint{0.933912in}{1.281371in}}%
\pgfpathcurveto{\pgfqpoint{0.939736in}{1.275547in}}{\pgfqpoint{0.947636in}{1.272275in}}{\pgfqpoint{0.955872in}{1.272275in}}%
\pgfpathlineto{\pgfqpoint{0.955872in}{1.272275in}}%
\pgfusepath{stroke,fill}%
\end{pgfscope}%
\begin{pgfscope}%
\pgfpathrectangle{\pgfqpoint{0.548058in}{0.516222in}}{\pgfqpoint{1.739582in}{1.783528in}} %
\pgfusepath{clip}%
\pgfsetbuttcap%
\pgfsetroundjoin%
\definecolor{currentfill}{rgb}{0.298039,0.447059,0.690196}%
\pgfsetfillcolor{currentfill}%
\pgfsetlinewidth{0.240900pt}%
\definecolor{currentstroke}{rgb}{1.000000,1.000000,1.000000}%
\pgfsetstrokecolor{currentstroke}%
\pgfsetdash{}{0pt}%
\pgfpathmoveto{\pgfqpoint{1.306338in}{1.501585in}}%
\pgfpathcurveto{\pgfqpoint{1.314574in}{1.501585in}}{\pgfqpoint{1.322474in}{1.504858in}}{\pgfqpoint{1.328298in}{1.510682in}}%
\pgfpathcurveto{\pgfqpoint{1.334122in}{1.516506in}}{\pgfqpoint{1.337394in}{1.524406in}}{\pgfqpoint{1.337394in}{1.532642in}}%
\pgfpathcurveto{\pgfqpoint{1.337394in}{1.540878in}}{\pgfqpoint{1.334122in}{1.548778in}}{\pgfqpoint{1.328298in}{1.554602in}}%
\pgfpathcurveto{\pgfqpoint{1.322474in}{1.560426in}}{\pgfqpoint{1.314574in}{1.563698in}}{\pgfqpoint{1.306338in}{1.563698in}}%
\pgfpathcurveto{\pgfqpoint{1.298101in}{1.563698in}}{\pgfqpoint{1.290201in}{1.560426in}}{\pgfqpoint{1.284377in}{1.554602in}}%
\pgfpathcurveto{\pgfqpoint{1.278554in}{1.548778in}}{\pgfqpoint{1.275281in}{1.540878in}}{\pgfqpoint{1.275281in}{1.532642in}}%
\pgfpathcurveto{\pgfqpoint{1.275281in}{1.524406in}}{\pgfqpoint{1.278554in}{1.516506in}}{\pgfqpoint{1.284377in}{1.510682in}}%
\pgfpathcurveto{\pgfqpoint{1.290201in}{1.504858in}}{\pgfqpoint{1.298101in}{1.501585in}}{\pgfqpoint{1.306338in}{1.501585in}}%
\pgfpathlineto{\pgfqpoint{1.306338in}{1.501585in}}%
\pgfusepath{stroke,fill}%
\end{pgfscope}%
\begin{pgfscope}%
\pgfpathrectangle{\pgfqpoint{0.548058in}{0.516222in}}{\pgfqpoint{1.739582in}{1.783528in}} %
\pgfusepath{clip}%
\pgfsetbuttcap%
\pgfsetroundjoin%
\definecolor{currentfill}{rgb}{0.298039,0.447059,0.690196}%
\pgfsetfillcolor{currentfill}%
\pgfsetlinewidth{0.240900pt}%
\definecolor{currentstroke}{rgb}{1.000000,1.000000,1.000000}%
\pgfsetstrokecolor{currentstroke}%
\pgfsetdash{}{0pt}%
\pgfpathmoveto{\pgfqpoint{0.860291in}{1.506681in}}%
\pgfpathcurveto{\pgfqpoint{0.868527in}{1.506681in}}{\pgfqpoint{0.876427in}{1.509954in}}{\pgfqpoint{0.882251in}{1.515777in}}%
\pgfpathcurveto{\pgfqpoint{0.888075in}{1.521601in}}{\pgfqpoint{0.891347in}{1.529501in}}{\pgfqpoint{0.891347in}{1.537738in}}%
\pgfpathcurveto{\pgfqpoint{0.891347in}{1.545974in}}{\pgfqpoint{0.888075in}{1.553874in}}{\pgfqpoint{0.882251in}{1.559698in}}%
\pgfpathcurveto{\pgfqpoint{0.876427in}{1.565522in}}{\pgfqpoint{0.868527in}{1.568794in}}{\pgfqpoint{0.860291in}{1.568794in}}%
\pgfpathcurveto{\pgfqpoint{0.852055in}{1.568794in}}{\pgfqpoint{0.844155in}{1.565522in}}{\pgfqpoint{0.838331in}{1.559698in}}%
\pgfpathcurveto{\pgfqpoint{0.832507in}{1.553874in}}{\pgfqpoint{0.829234in}{1.545974in}}{\pgfqpoint{0.829234in}{1.537738in}}%
\pgfpathcurveto{\pgfqpoint{0.829234in}{1.529501in}}{\pgfqpoint{0.832507in}{1.521601in}}{\pgfqpoint{0.838331in}{1.515777in}}%
\pgfpathcurveto{\pgfqpoint{0.844155in}{1.509954in}}{\pgfqpoint{0.852055in}{1.506681in}}{\pgfqpoint{0.860291in}{1.506681in}}%
\pgfpathlineto{\pgfqpoint{0.860291in}{1.506681in}}%
\pgfusepath{stroke,fill}%
\end{pgfscope}%
\begin{pgfscope}%
\pgfpathrectangle{\pgfqpoint{0.548058in}{0.516222in}}{\pgfqpoint{1.739582in}{1.783528in}} %
\pgfusepath{clip}%
\pgfsetbuttcap%
\pgfsetroundjoin%
\definecolor{currentfill}{rgb}{0.298039,0.447059,0.690196}%
\pgfsetfillcolor{currentfill}%
\pgfsetlinewidth{0.240900pt}%
\definecolor{currentstroke}{rgb}{1.000000,1.000000,1.000000}%
\pgfsetstrokecolor{currentstroke}%
\pgfsetdash{}{0pt}%
\pgfpathmoveto{\pgfqpoint{1.752384in}{1.161441in}}%
\pgfpathcurveto{\pgfqpoint{1.760621in}{1.161441in}}{\pgfqpoint{1.768521in}{1.164714in}}{\pgfqpoint{1.774345in}{1.170537in}}%
\pgfpathcurveto{\pgfqpoint{1.780169in}{1.176361in}}{\pgfqpoint{1.783441in}{1.184261in}}{\pgfqpoint{1.783441in}{1.192498in}}%
\pgfpathcurveto{\pgfqpoint{1.783441in}{1.200734in}}{\pgfqpoint{1.780169in}{1.208634in}}{\pgfqpoint{1.774345in}{1.214458in}}%
\pgfpathcurveto{\pgfqpoint{1.768521in}{1.220282in}}{\pgfqpoint{1.760621in}{1.223554in}}{\pgfqpoint{1.752384in}{1.223554in}}%
\pgfpathcurveto{\pgfqpoint{1.744148in}{1.223554in}}{\pgfqpoint{1.736248in}{1.220282in}}{\pgfqpoint{1.730424in}{1.214458in}}%
\pgfpathcurveto{\pgfqpoint{1.724600in}{1.208634in}}{\pgfqpoint{1.721328in}{1.200734in}}{\pgfqpoint{1.721328in}{1.192498in}}%
\pgfpathcurveto{\pgfqpoint{1.721328in}{1.184261in}}{\pgfqpoint{1.724600in}{1.176361in}}{\pgfqpoint{1.730424in}{1.170537in}}%
\pgfpathcurveto{\pgfqpoint{1.736248in}{1.164714in}}{\pgfqpoint{1.744148in}{1.161441in}}{\pgfqpoint{1.752384in}{1.161441in}}%
\pgfpathlineto{\pgfqpoint{1.752384in}{1.161441in}}%
\pgfusepath{stroke,fill}%
\end{pgfscope}%
\begin{pgfscope}%
\pgfpathrectangle{\pgfqpoint{0.548058in}{0.516222in}}{\pgfqpoint{1.739582in}{1.783528in}} %
\pgfusepath{clip}%
\pgfsetbuttcap%
\pgfsetroundjoin%
\definecolor{currentfill}{rgb}{0.298039,0.447059,0.690196}%
\pgfsetfillcolor{currentfill}%
\pgfsetlinewidth{0.240900pt}%
\definecolor{currentstroke}{rgb}{1.000000,1.000000,1.000000}%
\pgfsetstrokecolor{currentstroke}%
\pgfsetdash{}{0pt}%
\pgfpathmoveto{\pgfqpoint{1.147035in}{1.655733in}}%
\pgfpathcurveto{\pgfqpoint{1.155272in}{1.655733in}}{\pgfqpoint{1.163172in}{1.659006in}}{\pgfqpoint{1.168996in}{1.664829in}}%
\pgfpathcurveto{\pgfqpoint{1.174819in}{1.670653in}}{\pgfqpoint{1.178092in}{1.678553in}}{\pgfqpoint{1.178092in}{1.686790in}}%
\pgfpathcurveto{\pgfqpoint{1.178092in}{1.695026in}}{\pgfqpoint{1.174819in}{1.702926in}}{\pgfqpoint{1.168996in}{1.708750in}}%
\pgfpathcurveto{\pgfqpoint{1.163172in}{1.714574in}}{\pgfqpoint{1.155272in}{1.717846in}}{\pgfqpoint{1.147035in}{1.717846in}}%
\pgfpathcurveto{\pgfqpoint{1.138799in}{1.717846in}}{\pgfqpoint{1.130899in}{1.714574in}}{\pgfqpoint{1.125075in}{1.708750in}}%
\pgfpathcurveto{\pgfqpoint{1.119251in}{1.702926in}}{\pgfqpoint{1.115979in}{1.695026in}}{\pgfqpoint{1.115979in}{1.686790in}}%
\pgfpathcurveto{\pgfqpoint{1.115979in}{1.678553in}}{\pgfqpoint{1.119251in}{1.670653in}}{\pgfqpoint{1.125075in}{1.664829in}}%
\pgfpathcurveto{\pgfqpoint{1.130899in}{1.659006in}}{\pgfqpoint{1.138799in}{1.655733in}}{\pgfqpoint{1.147035in}{1.655733in}}%
\pgfpathlineto{\pgfqpoint{1.147035in}{1.655733in}}%
\pgfusepath{stroke,fill}%
\end{pgfscope}%
\begin{pgfscope}%
\pgfpathrectangle{\pgfqpoint{0.548058in}{0.516222in}}{\pgfqpoint{1.739582in}{1.783528in}} %
\pgfusepath{clip}%
\pgfsetbuttcap%
\pgfsetroundjoin%
\definecolor{currentfill}{rgb}{0.298039,0.447059,0.690196}%
\pgfsetfillcolor{currentfill}%
\pgfsetlinewidth{0.240900pt}%
\definecolor{currentstroke}{rgb}{1.000000,1.000000,1.000000}%
\pgfsetstrokecolor{currentstroke}%
\pgfsetdash{}{0pt}%
\pgfpathmoveto{\pgfqpoint{1.178896in}{1.264631in}}%
\pgfpathcurveto{\pgfqpoint{1.187132in}{1.264631in}}{\pgfqpoint{1.195032in}{1.267903in}}{\pgfqpoint{1.200856in}{1.273727in}}%
\pgfpathcurveto{\pgfqpoint{1.206680in}{1.279551in}}{\pgfqpoint{1.209952in}{1.287451in}}{\pgfqpoint{1.209952in}{1.295688in}}%
\pgfpathcurveto{\pgfqpoint{1.209952in}{1.303924in}}{\pgfqpoint{1.206680in}{1.311824in}}{\pgfqpoint{1.200856in}{1.317648in}}%
\pgfpathcurveto{\pgfqpoint{1.195032in}{1.323472in}}{\pgfqpoint{1.187132in}{1.326744in}}{\pgfqpoint{1.178896in}{1.326744in}}%
\pgfpathcurveto{\pgfqpoint{1.170659in}{1.326744in}}{\pgfqpoint{1.162759in}{1.323472in}}{\pgfqpoint{1.156935in}{1.317648in}}%
\pgfpathcurveto{\pgfqpoint{1.151112in}{1.311824in}}{\pgfqpoint{1.147839in}{1.303924in}}{\pgfqpoint{1.147839in}{1.295688in}}%
\pgfpathcurveto{\pgfqpoint{1.147839in}{1.287451in}}{\pgfqpoint{1.151112in}{1.279551in}}{\pgfqpoint{1.156935in}{1.273727in}}%
\pgfpathcurveto{\pgfqpoint{1.162759in}{1.267903in}}{\pgfqpoint{1.170659in}{1.264631in}}{\pgfqpoint{1.178896in}{1.264631in}}%
\pgfpathlineto{\pgfqpoint{1.178896in}{1.264631in}}%
\pgfusepath{stroke,fill}%
\end{pgfscope}%
\begin{pgfscope}%
\pgfpathrectangle{\pgfqpoint{0.548058in}{0.516222in}}{\pgfqpoint{1.739582in}{1.783528in}} %
\pgfusepath{clip}%
\pgfsetbuttcap%
\pgfsetroundjoin%
\definecolor{currentfill}{rgb}{0.298039,0.447059,0.690196}%
\pgfsetfillcolor{currentfill}%
\pgfsetlinewidth{0.240900pt}%
\definecolor{currentstroke}{rgb}{1.000000,1.000000,1.000000}%
\pgfsetstrokecolor{currentstroke}%
\pgfsetdash{}{0pt}%
\pgfpathmoveto{\pgfqpoint{2.007268in}{1.180550in}}%
\pgfpathcurveto{\pgfqpoint{2.015505in}{1.180550in}}{\pgfqpoint{2.023405in}{1.183823in}}{\pgfqpoint{2.029229in}{1.189647in}}%
\pgfpathcurveto{\pgfqpoint{2.035053in}{1.195471in}}{\pgfqpoint{2.038325in}{1.203371in}}{\pgfqpoint{2.038325in}{1.211607in}}%
\pgfpathcurveto{\pgfqpoint{2.038325in}{1.219843in}}{\pgfqpoint{2.035053in}{1.227743in}}{\pgfqpoint{2.029229in}{1.233567in}}%
\pgfpathcurveto{\pgfqpoint{2.023405in}{1.239391in}}{\pgfqpoint{2.015505in}{1.242663in}}{\pgfqpoint{2.007268in}{1.242663in}}%
\pgfpathcurveto{\pgfqpoint{1.999032in}{1.242663in}}{\pgfqpoint{1.991132in}{1.239391in}}{\pgfqpoint{1.985308in}{1.233567in}}%
\pgfpathcurveto{\pgfqpoint{1.979484in}{1.227743in}}{\pgfqpoint{1.976212in}{1.219843in}}{\pgfqpoint{1.976212in}{1.211607in}}%
\pgfpathcurveto{\pgfqpoint{1.976212in}{1.203371in}}{\pgfqpoint{1.979484in}{1.195471in}}{\pgfqpoint{1.985308in}{1.189647in}}%
\pgfpathcurveto{\pgfqpoint{1.991132in}{1.183823in}}{\pgfqpoint{1.999032in}{1.180550in}}{\pgfqpoint{2.007268in}{1.180550in}}%
\pgfpathlineto{\pgfqpoint{2.007268in}{1.180550in}}%
\pgfusepath{stroke,fill}%
\end{pgfscope}%
\begin{pgfscope}%
\pgfpathrectangle{\pgfqpoint{0.548058in}{0.516222in}}{\pgfqpoint{1.739582in}{1.783528in}} %
\pgfusepath{clip}%
\pgfsetbuttcap%
\pgfsetroundjoin%
\definecolor{currentfill}{rgb}{0.298039,0.447059,0.690196}%
\pgfsetfillcolor{currentfill}%
\pgfsetlinewidth{0.240900pt}%
\definecolor{currentstroke}{rgb}{1.000000,1.000000,1.000000}%
\pgfsetstrokecolor{currentstroke}%
\pgfsetdash{}{0pt}%
\pgfpathmoveto{\pgfqpoint{1.784245in}{1.041690in}}%
\pgfpathcurveto{\pgfqpoint{1.792481in}{1.041690in}}{\pgfqpoint{1.800381in}{1.044962in}}{\pgfqpoint{1.806205in}{1.050786in}}%
\pgfpathcurveto{\pgfqpoint{1.812029in}{1.056610in}}{\pgfqpoint{1.815301in}{1.064510in}}{\pgfqpoint{1.815301in}{1.072747in}}%
\pgfpathcurveto{\pgfqpoint{1.815301in}{1.080983in}}{\pgfqpoint{1.812029in}{1.088883in}}{\pgfqpoint{1.806205in}{1.094707in}}%
\pgfpathcurveto{\pgfqpoint{1.800381in}{1.100531in}}{\pgfqpoint{1.792481in}{1.103803in}}{\pgfqpoint{1.784245in}{1.103803in}}%
\pgfpathcurveto{\pgfqpoint{1.776009in}{1.103803in}}{\pgfqpoint{1.768109in}{1.100531in}}{\pgfqpoint{1.762285in}{1.094707in}}%
\pgfpathcurveto{\pgfqpoint{1.756461in}{1.088883in}}{\pgfqpoint{1.753188in}{1.080983in}}{\pgfqpoint{1.753188in}{1.072747in}}%
\pgfpathcurveto{\pgfqpoint{1.753188in}{1.064510in}}{\pgfqpoint{1.756461in}{1.056610in}}{\pgfqpoint{1.762285in}{1.050786in}}%
\pgfpathcurveto{\pgfqpoint{1.768109in}{1.044962in}}{\pgfqpoint{1.776009in}{1.041690in}}{\pgfqpoint{1.784245in}{1.041690in}}%
\pgfpathlineto{\pgfqpoint{1.784245in}{1.041690in}}%
\pgfusepath{stroke,fill}%
\end{pgfscope}%
\begin{pgfscope}%
\pgfpathrectangle{\pgfqpoint{0.548058in}{0.516222in}}{\pgfqpoint{1.739582in}{1.783528in}} %
\pgfusepath{clip}%
\pgfsetbuttcap%
\pgfsetroundjoin%
\definecolor{currentfill}{rgb}{0.298039,0.447059,0.690196}%
\pgfsetfillcolor{currentfill}%
\pgfsetlinewidth{0.240900pt}%
\definecolor{currentstroke}{rgb}{1.000000,1.000000,1.000000}%
\pgfsetstrokecolor{currentstroke}%
\pgfsetdash{}{0pt}%
\pgfpathmoveto{\pgfqpoint{0.924012in}{1.970398in}}%
\pgfpathcurveto{\pgfqpoint{0.932248in}{1.970398in}}{\pgfqpoint{0.940148in}{1.973671in}}{\pgfqpoint{0.945972in}{1.979495in}}%
\pgfpathcurveto{\pgfqpoint{0.951796in}{1.985319in}}{\pgfqpoint{0.955068in}{1.993219in}}{\pgfqpoint{0.955068in}{2.001455in}}%
\pgfpathcurveto{\pgfqpoint{0.955068in}{2.009691in}}{\pgfqpoint{0.951796in}{2.017591in}}{\pgfqpoint{0.945972in}{2.023415in}}%
\pgfpathcurveto{\pgfqpoint{0.940148in}{2.029239in}}{\pgfqpoint{0.932248in}{2.032511in}}{\pgfqpoint{0.924012in}{2.032511in}}%
\pgfpathcurveto{\pgfqpoint{0.915776in}{2.032511in}}{\pgfqpoint{0.907876in}{2.029239in}}{\pgfqpoint{0.902052in}{2.023415in}}%
\pgfpathcurveto{\pgfqpoint{0.896228in}{2.017591in}}{\pgfqpoint{0.892955in}{2.009691in}}{\pgfqpoint{0.892955in}{2.001455in}}%
\pgfpathcurveto{\pgfqpoint{0.892955in}{1.993219in}}{\pgfqpoint{0.896228in}{1.985319in}}{\pgfqpoint{0.902052in}{1.979495in}}%
\pgfpathcurveto{\pgfqpoint{0.907876in}{1.973671in}}{\pgfqpoint{0.915776in}{1.970398in}}{\pgfqpoint{0.924012in}{1.970398in}}%
\pgfpathlineto{\pgfqpoint{0.924012in}{1.970398in}}%
\pgfusepath{stroke,fill}%
\end{pgfscope}%
\begin{pgfscope}%
\pgfpathrectangle{\pgfqpoint{0.548058in}{0.516222in}}{\pgfqpoint{1.739582in}{1.783528in}} %
\pgfusepath{clip}%
\pgfsetbuttcap%
\pgfsetroundjoin%
\definecolor{currentfill}{rgb}{0.298039,0.447059,0.690196}%
\pgfsetfillcolor{currentfill}%
\pgfsetlinewidth{0.240900pt}%
\definecolor{currentstroke}{rgb}{1.000000,1.000000,1.000000}%
\pgfsetstrokecolor{currentstroke}%
\pgfsetdash{}{0pt}%
\pgfpathmoveto{\pgfqpoint{1.975408in}{0.943596in}}%
\pgfpathcurveto{\pgfqpoint{1.983644in}{0.943596in}}{\pgfqpoint{1.991544in}{0.946868in}}{\pgfqpoint{1.997368in}{0.952692in}}%
\pgfpathcurveto{\pgfqpoint{2.003192in}{0.958516in}}{\pgfqpoint{2.006464in}{0.966416in}}{\pgfqpoint{2.006464in}{0.974653in}}%
\pgfpathcurveto{\pgfqpoint{2.006464in}{0.982889in}}{\pgfqpoint{2.003192in}{0.990789in}}{\pgfqpoint{1.997368in}{0.996613in}}%
\pgfpathcurveto{\pgfqpoint{1.991544in}{1.002437in}}{\pgfqpoint{1.983644in}{1.005709in}}{\pgfqpoint{1.975408in}{1.005709in}}%
\pgfpathcurveto{\pgfqpoint{1.967172in}{1.005709in}}{\pgfqpoint{1.959272in}{1.002437in}}{\pgfqpoint{1.953448in}{0.996613in}}%
\pgfpathcurveto{\pgfqpoint{1.947624in}{0.990789in}}{\pgfqpoint{1.944351in}{0.982889in}}{\pgfqpoint{1.944351in}{0.974653in}}%
\pgfpathcurveto{\pgfqpoint{1.944351in}{0.966416in}}{\pgfqpoint{1.947624in}{0.958516in}}{\pgfqpoint{1.953448in}{0.952692in}}%
\pgfpathcurveto{\pgfqpoint{1.959272in}{0.946868in}}{\pgfqpoint{1.967172in}{0.943596in}}{\pgfqpoint{1.975408in}{0.943596in}}%
\pgfpathlineto{\pgfqpoint{1.975408in}{0.943596in}}%
\pgfusepath{stroke,fill}%
\end{pgfscope}%
\begin{pgfscope}%
\pgfpathrectangle{\pgfqpoint{0.548058in}{0.516222in}}{\pgfqpoint{1.739582in}{1.783528in}} %
\pgfusepath{clip}%
\pgfsetbuttcap%
\pgfsetroundjoin%
\definecolor{currentfill}{rgb}{0.298039,0.447059,0.690196}%
\pgfsetfillcolor{currentfill}%
\pgfsetlinewidth{0.240900pt}%
\definecolor{currentstroke}{rgb}{1.000000,1.000000,1.000000}%
\pgfsetstrokecolor{currentstroke}%
\pgfsetdash{}{0pt}%
\pgfpathmoveto{\pgfqpoint{1.624943in}{1.695226in}}%
\pgfpathcurveto{\pgfqpoint{1.633179in}{1.695226in}}{\pgfqpoint{1.641079in}{1.698498in}}{\pgfqpoint{1.646903in}{1.704322in}}%
\pgfpathcurveto{\pgfqpoint{1.652727in}{1.710146in}}{\pgfqpoint{1.655999in}{1.718046in}}{\pgfqpoint{1.655999in}{1.726282in}}%
\pgfpathcurveto{\pgfqpoint{1.655999in}{1.734518in}}{\pgfqpoint{1.652727in}{1.742418in}}{\pgfqpoint{1.646903in}{1.748242in}}%
\pgfpathcurveto{\pgfqpoint{1.641079in}{1.754066in}}{\pgfqpoint{1.633179in}{1.757339in}}{\pgfqpoint{1.624943in}{1.757339in}}%
\pgfpathcurveto{\pgfqpoint{1.616706in}{1.757339in}}{\pgfqpoint{1.608806in}{1.754066in}}{\pgfqpoint{1.602982in}{1.748242in}}%
\pgfpathcurveto{\pgfqpoint{1.597158in}{1.742418in}}{\pgfqpoint{1.593886in}{1.734518in}}{\pgfqpoint{1.593886in}{1.726282in}}%
\pgfpathcurveto{\pgfqpoint{1.593886in}{1.718046in}}{\pgfqpoint{1.597158in}{1.710146in}}{\pgfqpoint{1.602982in}{1.704322in}}%
\pgfpathcurveto{\pgfqpoint{1.608806in}{1.698498in}}{\pgfqpoint{1.616706in}{1.695226in}}{\pgfqpoint{1.624943in}{1.695226in}}%
\pgfpathlineto{\pgfqpoint{1.624943in}{1.695226in}}%
\pgfusepath{stroke,fill}%
\end{pgfscope}%
\begin{pgfscope}%
\pgfpathrectangle{\pgfqpoint{0.548058in}{0.516222in}}{\pgfqpoint{1.739582in}{1.783528in}} %
\pgfusepath{clip}%
\pgfsetbuttcap%
\pgfsetroundjoin%
\definecolor{currentfill}{rgb}{0.298039,0.447059,0.690196}%
\pgfsetfillcolor{currentfill}%
\pgfsetlinewidth{0.240900pt}%
\definecolor{currentstroke}{rgb}{1.000000,1.000000,1.000000}%
\pgfsetstrokecolor{currentstroke}%
\pgfsetdash{}{0pt}%
\pgfpathmoveto{\pgfqpoint{1.879826in}{1.378012in}}%
\pgfpathcurveto{\pgfqpoint{1.888063in}{1.378012in}}{\pgfqpoint{1.895963in}{1.381285in}}{\pgfqpoint{1.901787in}{1.387109in}}%
\pgfpathcurveto{\pgfqpoint{1.907611in}{1.392933in}}{\pgfqpoint{1.910883in}{1.400833in}}{\pgfqpoint{1.910883in}{1.409069in}}%
\pgfpathcurveto{\pgfqpoint{1.910883in}{1.417305in}}{\pgfqpoint{1.907611in}{1.425205in}}{\pgfqpoint{1.901787in}{1.431029in}}%
\pgfpathcurveto{\pgfqpoint{1.895963in}{1.436853in}}{\pgfqpoint{1.888063in}{1.440125in}}{\pgfqpoint{1.879826in}{1.440125in}}%
\pgfpathcurveto{\pgfqpoint{1.871590in}{1.440125in}}{\pgfqpoint{1.863690in}{1.436853in}}{\pgfqpoint{1.857866in}{1.431029in}}%
\pgfpathcurveto{\pgfqpoint{1.852042in}{1.425205in}}{\pgfqpoint{1.848770in}{1.417305in}}{\pgfqpoint{1.848770in}{1.409069in}}%
\pgfpathcurveto{\pgfqpoint{1.848770in}{1.400833in}}{\pgfqpoint{1.852042in}{1.392933in}}{\pgfqpoint{1.857866in}{1.387109in}}%
\pgfpathcurveto{\pgfqpoint{1.863690in}{1.381285in}}{\pgfqpoint{1.871590in}{1.378012in}}{\pgfqpoint{1.879826in}{1.378012in}}%
\pgfpathlineto{\pgfqpoint{1.879826in}{1.378012in}}%
\pgfusepath{stroke,fill}%
\end{pgfscope}%
\begin{pgfscope}%
\pgfpathrectangle{\pgfqpoint{0.548058in}{0.516222in}}{\pgfqpoint{1.739582in}{1.783528in}} %
\pgfusepath{clip}%
\pgfsetbuttcap%
\pgfsetroundjoin%
\definecolor{currentfill}{rgb}{0.298039,0.447059,0.690196}%
\pgfsetfillcolor{currentfill}%
\pgfsetlinewidth{0.240900pt}%
\definecolor{currentstroke}{rgb}{1.000000,1.000000,1.000000}%
\pgfsetstrokecolor{currentstroke}%
\pgfsetdash{}{0pt}%
\pgfpathmoveto{\pgfqpoint{1.593082in}{1.743636in}}%
\pgfpathcurveto{\pgfqpoint{1.601318in}{1.743636in}}{\pgfqpoint{1.609218in}{1.746908in}}{\pgfqpoint{1.615042in}{1.752732in}}%
\pgfpathcurveto{\pgfqpoint{1.620866in}{1.758556in}}{\pgfqpoint{1.624139in}{1.766456in}}{\pgfqpoint{1.624139in}{1.774692in}}%
\pgfpathcurveto{\pgfqpoint{1.624139in}{1.782928in}}{\pgfqpoint{1.620866in}{1.790828in}}{\pgfqpoint{1.615042in}{1.796652in}}%
\pgfpathcurveto{\pgfqpoint{1.609218in}{1.802476in}}{\pgfqpoint{1.601318in}{1.805749in}}{\pgfqpoint{1.593082in}{1.805749in}}%
\pgfpathcurveto{\pgfqpoint{1.584846in}{1.805749in}}{\pgfqpoint{1.576946in}{1.802476in}}{\pgfqpoint{1.571122in}{1.796652in}}%
\pgfpathcurveto{\pgfqpoint{1.565298in}{1.790828in}}{\pgfqpoint{1.562026in}{1.782928in}}{\pgfqpoint{1.562026in}{1.774692in}}%
\pgfpathcurveto{\pgfqpoint{1.562026in}{1.766456in}}{\pgfqpoint{1.565298in}{1.758556in}}{\pgfqpoint{1.571122in}{1.752732in}}%
\pgfpathcurveto{\pgfqpoint{1.576946in}{1.746908in}}{\pgfqpoint{1.584846in}{1.743636in}}{\pgfqpoint{1.593082in}{1.743636in}}%
\pgfpathlineto{\pgfqpoint{1.593082in}{1.743636in}}%
\pgfusepath{stroke,fill}%
\end{pgfscope}%
\begin{pgfscope}%
\pgfpathrectangle{\pgfqpoint{0.548058in}{0.516222in}}{\pgfqpoint{1.739582in}{1.783528in}} %
\pgfusepath{clip}%
\pgfsetbuttcap%
\pgfsetroundjoin%
\definecolor{currentfill}{rgb}{0.298039,0.447059,0.690196}%
\pgfsetfillcolor{currentfill}%
\pgfsetlinewidth{0.240900pt}%
\definecolor{currentstroke}{rgb}{1.000000,1.000000,1.000000}%
\pgfsetstrokecolor{currentstroke}%
\pgfsetdash{}{0pt}%
\pgfpathmoveto{\pgfqpoint{1.529361in}{1.720705in}}%
\pgfpathcurveto{\pgfqpoint{1.537597in}{1.720705in}}{\pgfqpoint{1.545497in}{1.723977in}}{\pgfqpoint{1.551321in}{1.729801in}}%
\pgfpathcurveto{\pgfqpoint{1.557145in}{1.735625in}}{\pgfqpoint{1.560418in}{1.743525in}}{\pgfqpoint{1.560418in}{1.751761in}}%
\pgfpathcurveto{\pgfqpoint{1.560418in}{1.759997in}}{\pgfqpoint{1.557145in}{1.767897in}}{\pgfqpoint{1.551321in}{1.773721in}}%
\pgfpathcurveto{\pgfqpoint{1.545497in}{1.779545in}}{\pgfqpoint{1.537597in}{1.782818in}}{\pgfqpoint{1.529361in}{1.782818in}}%
\pgfpathcurveto{\pgfqpoint{1.521125in}{1.782818in}}{\pgfqpoint{1.513225in}{1.779545in}}{\pgfqpoint{1.507401in}{1.773721in}}%
\pgfpathcurveto{\pgfqpoint{1.501577in}{1.767897in}}{\pgfqpoint{1.498305in}{1.759997in}}{\pgfqpoint{1.498305in}{1.751761in}}%
\pgfpathcurveto{\pgfqpoint{1.498305in}{1.743525in}}{\pgfqpoint{1.501577in}{1.735625in}}{\pgfqpoint{1.507401in}{1.729801in}}%
\pgfpathcurveto{\pgfqpoint{1.513225in}{1.723977in}}{\pgfqpoint{1.521125in}{1.720705in}}{\pgfqpoint{1.529361in}{1.720705in}}%
\pgfpathlineto{\pgfqpoint{1.529361in}{1.720705in}}%
\pgfusepath{stroke,fill}%
\end{pgfscope}%
\begin{pgfscope}%
\pgfpathrectangle{\pgfqpoint{0.548058in}{0.516222in}}{\pgfqpoint{1.739582in}{1.783528in}} %
\pgfusepath{clip}%
\pgfsetbuttcap%
\pgfsetroundjoin%
\definecolor{currentfill}{rgb}{0.298039,0.447059,0.690196}%
\pgfsetfillcolor{currentfill}%
\pgfsetlinewidth{0.240900pt}%
\definecolor{currentstroke}{rgb}{1.000000,1.000000,1.000000}%
\pgfsetstrokecolor{currentstroke}%
\pgfsetdash{}{0pt}%
\pgfpathmoveto{\pgfqpoint{1.115175in}{1.096470in}}%
\pgfpathcurveto{\pgfqpoint{1.123411in}{1.096470in}}{\pgfqpoint{1.131311in}{1.099742in}}{\pgfqpoint{1.137135in}{1.105566in}}%
\pgfpathcurveto{\pgfqpoint{1.142959in}{1.111390in}}{\pgfqpoint{1.146231in}{1.119290in}}{\pgfqpoint{1.146231in}{1.127526in}}%
\pgfpathcurveto{\pgfqpoint{1.146231in}{1.135763in}}{\pgfqpoint{1.142959in}{1.143663in}}{\pgfqpoint{1.137135in}{1.149487in}}%
\pgfpathcurveto{\pgfqpoint{1.131311in}{1.155311in}}{\pgfqpoint{1.123411in}{1.158583in}}{\pgfqpoint{1.115175in}{1.158583in}}%
\pgfpathcurveto{\pgfqpoint{1.106939in}{1.158583in}}{\pgfqpoint{1.099038in}{1.155311in}}{\pgfqpoint{1.093215in}{1.149487in}}%
\pgfpathcurveto{\pgfqpoint{1.087391in}{1.143663in}}{\pgfqpoint{1.084118in}{1.135763in}}{\pgfqpoint{1.084118in}{1.127526in}}%
\pgfpathcurveto{\pgfqpoint{1.084118in}{1.119290in}}{\pgfqpoint{1.087391in}{1.111390in}}{\pgfqpoint{1.093215in}{1.105566in}}%
\pgfpathcurveto{\pgfqpoint{1.099038in}{1.099742in}}{\pgfqpoint{1.106939in}{1.096470in}}{\pgfqpoint{1.115175in}{1.096470in}}%
\pgfpathlineto{\pgfqpoint{1.115175in}{1.096470in}}%
\pgfusepath{stroke,fill}%
\end{pgfscope}%
\begin{pgfscope}%
\pgfpathrectangle{\pgfqpoint{0.548058in}{0.516222in}}{\pgfqpoint{1.739582in}{1.783528in}} %
\pgfusepath{clip}%
\pgfsetbuttcap%
\pgfsetroundjoin%
\definecolor{currentfill}{rgb}{0.298039,0.447059,0.690196}%
\pgfsetfillcolor{currentfill}%
\pgfsetlinewidth{0.240900pt}%
\definecolor{currentstroke}{rgb}{1.000000,1.000000,1.000000}%
\pgfsetstrokecolor{currentstroke}%
\pgfsetdash{}{0pt}%
\pgfpathmoveto{\pgfqpoint{1.274477in}{1.686308in}}%
\pgfpathcurveto{\pgfqpoint{1.282713in}{1.686308in}}{\pgfqpoint{1.290614in}{1.689580in}}{\pgfqpoint{1.296437in}{1.695404in}}%
\pgfpathcurveto{\pgfqpoint{1.302261in}{1.701228in}}{\pgfqpoint{1.305534in}{1.709128in}}{\pgfqpoint{1.305534in}{1.717364in}}%
\pgfpathcurveto{\pgfqpoint{1.305534in}{1.725601in}}{\pgfqpoint{1.302261in}{1.733501in}}{\pgfqpoint{1.296437in}{1.739325in}}%
\pgfpathcurveto{\pgfqpoint{1.290614in}{1.745149in}}{\pgfqpoint{1.282713in}{1.748421in}}{\pgfqpoint{1.274477in}{1.748421in}}%
\pgfpathcurveto{\pgfqpoint{1.266241in}{1.748421in}}{\pgfqpoint{1.258341in}{1.745149in}}{\pgfqpoint{1.252517in}{1.739325in}}%
\pgfpathcurveto{\pgfqpoint{1.246693in}{1.733501in}}{\pgfqpoint{1.243421in}{1.725601in}}{\pgfqpoint{1.243421in}{1.717364in}}%
\pgfpathcurveto{\pgfqpoint{1.243421in}{1.709128in}}{\pgfqpoint{1.246693in}{1.701228in}}{\pgfqpoint{1.252517in}{1.695404in}}%
\pgfpathcurveto{\pgfqpoint{1.258341in}{1.689580in}}{\pgfqpoint{1.266241in}{1.686308in}}{\pgfqpoint{1.274477in}{1.686308in}}%
\pgfpathlineto{\pgfqpoint{1.274477in}{1.686308in}}%
\pgfusepath{stroke,fill}%
\end{pgfscope}%
\begin{pgfscope}%
\pgfpathrectangle{\pgfqpoint{0.548058in}{0.516222in}}{\pgfqpoint{1.739582in}{1.783528in}} %
\pgfusepath{clip}%
\pgfsetbuttcap%
\pgfsetroundjoin%
\definecolor{currentfill}{rgb}{0.298039,0.447059,0.690196}%
\pgfsetfillcolor{currentfill}%
\pgfsetlinewidth{0.240900pt}%
\definecolor{currentstroke}{rgb}{1.000000,1.000000,1.000000}%
\pgfsetstrokecolor{currentstroke}%
\pgfsetdash{}{0pt}%
\pgfpathmoveto{\pgfqpoint{1.561222in}{1.910523in}}%
\pgfpathcurveto{\pgfqpoint{1.569458in}{1.910523in}}{\pgfqpoint{1.577358in}{1.913795in}}{\pgfqpoint{1.583182in}{1.919619in}}%
\pgfpathcurveto{\pgfqpoint{1.589006in}{1.925443in}}{\pgfqpoint{1.592278in}{1.933343in}}{\pgfqpoint{1.592278in}{1.941579in}}%
\pgfpathcurveto{\pgfqpoint{1.592278in}{1.949816in}}{\pgfqpoint{1.589006in}{1.957716in}}{\pgfqpoint{1.583182in}{1.963540in}}%
\pgfpathcurveto{\pgfqpoint{1.577358in}{1.969364in}}{\pgfqpoint{1.569458in}{1.972636in}}{\pgfqpoint{1.561222in}{1.972636in}}%
\pgfpathcurveto{\pgfqpoint{1.552985in}{1.972636in}}{\pgfqpoint{1.545085in}{1.969364in}}{\pgfqpoint{1.539261in}{1.963540in}}%
\pgfpathcurveto{\pgfqpoint{1.533437in}{1.957716in}}{\pgfqpoint{1.530165in}{1.949816in}}{\pgfqpoint{1.530165in}{1.941579in}}%
\pgfpathcurveto{\pgfqpoint{1.530165in}{1.933343in}}{\pgfqpoint{1.533437in}{1.925443in}}{\pgfqpoint{1.539261in}{1.919619in}}%
\pgfpathcurveto{\pgfqpoint{1.545085in}{1.913795in}}{\pgfqpoint{1.552985in}{1.910523in}}{\pgfqpoint{1.561222in}{1.910523in}}%
\pgfpathlineto{\pgfqpoint{1.561222in}{1.910523in}}%
\pgfusepath{stroke,fill}%
\end{pgfscope}%
\begin{pgfscope}%
\pgfpathrectangle{\pgfqpoint{0.548058in}{0.516222in}}{\pgfqpoint{1.739582in}{1.783528in}} %
\pgfusepath{clip}%
\pgfsetbuttcap%
\pgfsetroundjoin%
\definecolor{currentfill}{rgb}{0.298039,0.447059,0.690196}%
\pgfsetfillcolor{currentfill}%
\pgfsetlinewidth{0.240900pt}%
\definecolor{currentstroke}{rgb}{1.000000,1.000000,1.000000}%
\pgfsetstrokecolor{currentstroke}%
\pgfsetdash{}{0pt}%
\pgfpathmoveto{\pgfqpoint{0.892151in}{1.137236in}}%
\pgfpathcurveto{\pgfqpoint{0.900388in}{1.137236in}}{\pgfqpoint{0.908288in}{1.140509in}}{\pgfqpoint{0.914112in}{1.146332in}}%
\pgfpathcurveto{\pgfqpoint{0.919936in}{1.152156in}}{\pgfqpoint{0.923208in}{1.160056in}}{\pgfqpoint{0.923208in}{1.168293in}}%
\pgfpathcurveto{\pgfqpoint{0.923208in}{1.176529in}}{\pgfqpoint{0.919936in}{1.184429in}}{\pgfqpoint{0.914112in}{1.190253in}}%
\pgfpathcurveto{\pgfqpoint{0.908288in}{1.196077in}}{\pgfqpoint{0.900388in}{1.199349in}}{\pgfqpoint{0.892151in}{1.199349in}}%
\pgfpathcurveto{\pgfqpoint{0.883915in}{1.199349in}}{\pgfqpoint{0.876015in}{1.196077in}}{\pgfqpoint{0.870191in}{1.190253in}}%
\pgfpathcurveto{\pgfqpoint{0.864367in}{1.184429in}}{\pgfqpoint{0.861095in}{1.176529in}}{\pgfqpoint{0.861095in}{1.168293in}}%
\pgfpathcurveto{\pgfqpoint{0.861095in}{1.160056in}}{\pgfqpoint{0.864367in}{1.152156in}}{\pgfqpoint{0.870191in}{1.146332in}}%
\pgfpathcurveto{\pgfqpoint{0.876015in}{1.140509in}}{\pgfqpoint{0.883915in}{1.137236in}}{\pgfqpoint{0.892151in}{1.137236in}}%
\pgfpathlineto{\pgfqpoint{0.892151in}{1.137236in}}%
\pgfusepath{stroke,fill}%
\end{pgfscope}%
\begin{pgfscope}%
\pgfpathrectangle{\pgfqpoint{0.548058in}{0.516222in}}{\pgfqpoint{1.739582in}{1.783528in}} %
\pgfusepath{clip}%
\pgfsetbuttcap%
\pgfsetroundjoin%
\definecolor{currentfill}{rgb}{0.298039,0.447059,0.690196}%
\pgfsetfillcolor{currentfill}%
\pgfsetlinewidth{0.240900pt}%
\definecolor{currentstroke}{rgb}{1.000000,1.000000,1.000000}%
\pgfsetstrokecolor{currentstroke}%
\pgfsetdash{}{0pt}%
\pgfpathmoveto{\pgfqpoint{1.210756in}{1.521969in}}%
\pgfpathcurveto{\pgfqpoint{1.218993in}{1.521969in}}{\pgfqpoint{1.226893in}{1.525241in}}{\pgfqpoint{1.232717in}{1.531065in}}%
\pgfpathcurveto{\pgfqpoint{1.238540in}{1.536889in}}{\pgfqpoint{1.241813in}{1.544789in}}{\pgfqpoint{1.241813in}{1.553025in}}%
\pgfpathcurveto{\pgfqpoint{1.241813in}{1.561261in}}{\pgfqpoint{1.238540in}{1.569161in}}{\pgfqpoint{1.232717in}{1.574985in}}%
\pgfpathcurveto{\pgfqpoint{1.226893in}{1.580809in}}{\pgfqpoint{1.218993in}{1.584082in}}{\pgfqpoint{1.210756in}{1.584082in}}%
\pgfpathcurveto{\pgfqpoint{1.202520in}{1.584082in}}{\pgfqpoint{1.194620in}{1.580809in}}{\pgfqpoint{1.188796in}{1.574985in}}%
\pgfpathcurveto{\pgfqpoint{1.182972in}{1.569161in}}{\pgfqpoint{1.179700in}{1.561261in}}{\pgfqpoint{1.179700in}{1.553025in}}%
\pgfpathcurveto{\pgfqpoint{1.179700in}{1.544789in}}{\pgfqpoint{1.182972in}{1.536889in}}{\pgfqpoint{1.188796in}{1.531065in}}%
\pgfpathcurveto{\pgfqpoint{1.194620in}{1.525241in}}{\pgfqpoint{1.202520in}{1.521969in}}{\pgfqpoint{1.210756in}{1.521969in}}%
\pgfpathlineto{\pgfqpoint{1.210756in}{1.521969in}}%
\pgfusepath{stroke,fill}%
\end{pgfscope}%
\begin{pgfscope}%
\pgfsetrectcap%
\pgfsetmiterjoin%
\pgfsetlinewidth{0.000000pt}%
\definecolor{currentstroke}{rgb}{1.000000,1.000000,1.000000}%
\pgfsetstrokecolor{currentstroke}%
\pgfsetdash{}{0pt}%
\pgfpathmoveto{\pgfqpoint{0.548058in}{0.516222in}}%
\pgfpathlineto{\pgfqpoint{2.287641in}{0.516222in}}%
\pgfusepath{}%
\end{pgfscope}%
\begin{pgfscope}%
\pgfsetrectcap%
\pgfsetmiterjoin%
\pgfsetlinewidth{0.000000pt}%
\definecolor{currentstroke}{rgb}{1.000000,1.000000,1.000000}%
\pgfsetstrokecolor{currentstroke}%
\pgfsetdash{}{0pt}%
\pgfpathmoveto{\pgfqpoint{0.548058in}{0.516222in}}%
\pgfpathlineto{\pgfqpoint{0.548058in}{2.299750in}}%
\pgfusepath{}%
\end{pgfscope}%
\end{pgfpicture}%
\makeatother%
\endgroup%

		\caption{Comparison between the two times from different throws.}
		\label{fig_wtr_vs_avg4}
	\end{subfigure}
	\caption{Plots of time measurments made by two observers(obs)}
\end{figure}


\begin{figure}[h]{.5\linewidth}
	%% Creator: Matplotlib, PGF backend
%%
%% To include the figure in your LaTeX document, write
%%   \input{<filename>.pgf}
%%
%% Make sure the required packages are loaded in your preamble
%%   \usepackage{pgf}
%%
%% Figures using additional raster images can only be included by \input if
%% they are in the same directory as the main LaTeX file. For loading figures
%% from other directories you can use the `import` package
%%   \usepackage{import}
%% and then include the figures with
%%   \import{<path to file>}{<filename>.pgf}
%%
%% Matplotlib used the following preamble
%%   \usepackage[utf8x]{inputenc}
%%   \usepackage[T1]{fontenc}
%%   \usepackage{cmbright}
%%
\begingroup%
\makeatletter%
\begin{pgfpicture}%
\pgfpathrectangle{\pgfpointorigin}{\pgfqpoint{5.000000in}{3.000000in}}%
\pgfusepath{use as bounding box, clip}%
\begin{pgfscope}%
\pgfsetbuttcap%
\pgfsetmiterjoin%
\definecolor{currentfill}{rgb}{1.000000,1.000000,1.000000}%
\pgfsetfillcolor{currentfill}%
\pgfsetlinewidth{0.000000pt}%
\definecolor{currentstroke}{rgb}{1.000000,1.000000,1.000000}%
\pgfsetstrokecolor{currentstroke}%
\pgfsetdash{}{0pt}%
\pgfpathmoveto{\pgfqpoint{0.000000in}{0.000000in}}%
\pgfpathlineto{\pgfqpoint{5.000000in}{0.000000in}}%
\pgfpathlineto{\pgfqpoint{5.000000in}{3.000000in}}%
\pgfpathlineto{\pgfqpoint{0.000000in}{3.000000in}}%
\pgfpathclose%
\pgfusepath{fill}%
\end{pgfscope}%
\begin{pgfscope}%
\pgfsetbuttcap%
\pgfsetmiterjoin%
\definecolor{currentfill}{rgb}{0.917647,0.917647,0.949020}%
\pgfsetfillcolor{currentfill}%
\pgfsetlinewidth{0.000000pt}%
\definecolor{currentstroke}{rgb}{0.000000,0.000000,0.000000}%
\pgfsetstrokecolor{currentstroke}%
\pgfsetstrokeopacity{0.000000}%
\pgfsetdash{}{0pt}%
\pgfpathmoveto{\pgfqpoint{0.625000in}{0.375000in}}%
\pgfpathlineto{\pgfqpoint{4.500000in}{0.375000in}}%
\pgfpathlineto{\pgfqpoint{4.500000in}{2.700000in}}%
\pgfpathlineto{\pgfqpoint{0.625000in}{2.700000in}}%
\pgfpathclose%
\pgfusepath{fill}%
\end{pgfscope}%
\begin{pgfscope}%
\pgfpathrectangle{\pgfqpoint{0.625000in}{0.375000in}}{\pgfqpoint{3.875000in}{2.325000in}} %
\pgfusepath{clip}%
\pgfsetroundcap%
\pgfsetroundjoin%
\pgfsetlinewidth{0.803000pt}%
\definecolor{currentstroke}{rgb}{1.000000,1.000000,1.000000}%
\pgfsetstrokecolor{currentstroke}%
\pgfsetdash{}{0pt}%
\pgfpathmoveto{\pgfqpoint{0.852941in}{0.375000in}}%
\pgfpathlineto{\pgfqpoint{0.852941in}{2.700000in}}%
\pgfusepath{stroke}%
\end{pgfscope}%
\begin{pgfscope}%
\pgfsetbuttcap%
\pgfsetroundjoin%
\definecolor{currentfill}{rgb}{0.150000,0.150000,0.150000}%
\pgfsetfillcolor{currentfill}%
\pgfsetlinewidth{0.803000pt}%
\definecolor{currentstroke}{rgb}{0.150000,0.150000,0.150000}%
\pgfsetstrokecolor{currentstroke}%
\pgfsetdash{}{0pt}%
\pgfsys@defobject{currentmarker}{\pgfqpoint{0.000000in}{0.000000in}}{\pgfqpoint{0.000000in}{0.000000in}}{%
\pgfpathmoveto{\pgfqpoint{0.000000in}{0.000000in}}%
\pgfpathlineto{\pgfqpoint{0.000000in}{0.000000in}}%
\pgfusepath{stroke,fill}%
}%
\begin{pgfscope}%
\pgfsys@transformshift{0.852941in}{0.375000in}%
\pgfsys@useobject{currentmarker}{}%
\end{pgfscope}%
\end{pgfscope}%
\begin{pgfscope}%
\pgfsetbuttcap%
\pgfsetroundjoin%
\definecolor{currentfill}{rgb}{0.150000,0.150000,0.150000}%
\pgfsetfillcolor{currentfill}%
\pgfsetlinewidth{0.803000pt}%
\definecolor{currentstroke}{rgb}{0.150000,0.150000,0.150000}%
\pgfsetstrokecolor{currentstroke}%
\pgfsetdash{}{0pt}%
\pgfsys@defobject{currentmarker}{\pgfqpoint{0.000000in}{0.000000in}}{\pgfqpoint{0.000000in}{0.000000in}}{%
\pgfpathmoveto{\pgfqpoint{0.000000in}{0.000000in}}%
\pgfpathlineto{\pgfqpoint{0.000000in}{0.000000in}}%
\pgfusepath{stroke,fill}%
}%
\begin{pgfscope}%
\pgfsys@transformshift{0.852941in}{2.700000in}%
\pgfsys@useobject{currentmarker}{}%
\end{pgfscope}%
\end{pgfscope}%
\begin{pgfscope}%
\definecolor{textcolor}{rgb}{0.150000,0.150000,0.150000}%
\pgfsetstrokecolor{textcolor}%
\pgfsetfillcolor{textcolor}%
\pgftext[x=0.852941in,y=0.297222in,,top]{\color{textcolor}\sffamily\fontsize{8.000000}{9.600000}\selectfont −1.5}%
\end{pgfscope}%
\begin{pgfscope}%
\pgfpathrectangle{\pgfqpoint{0.625000in}{0.375000in}}{\pgfqpoint{3.875000in}{2.325000in}} %
\pgfusepath{clip}%
\pgfsetroundcap%
\pgfsetroundjoin%
\pgfsetlinewidth{0.803000pt}%
\definecolor{currentstroke}{rgb}{1.000000,1.000000,1.000000}%
\pgfsetstrokecolor{currentstroke}%
\pgfsetdash{}{0pt}%
\pgfpathmoveto{\pgfqpoint{1.422794in}{0.375000in}}%
\pgfpathlineto{\pgfqpoint{1.422794in}{2.700000in}}%
\pgfusepath{stroke}%
\end{pgfscope}%
\begin{pgfscope}%
\pgfsetbuttcap%
\pgfsetroundjoin%
\definecolor{currentfill}{rgb}{0.150000,0.150000,0.150000}%
\pgfsetfillcolor{currentfill}%
\pgfsetlinewidth{0.803000pt}%
\definecolor{currentstroke}{rgb}{0.150000,0.150000,0.150000}%
\pgfsetstrokecolor{currentstroke}%
\pgfsetdash{}{0pt}%
\pgfsys@defobject{currentmarker}{\pgfqpoint{0.000000in}{0.000000in}}{\pgfqpoint{0.000000in}{0.000000in}}{%
\pgfpathmoveto{\pgfqpoint{0.000000in}{0.000000in}}%
\pgfpathlineto{\pgfqpoint{0.000000in}{0.000000in}}%
\pgfusepath{stroke,fill}%
}%
\begin{pgfscope}%
\pgfsys@transformshift{1.422794in}{0.375000in}%
\pgfsys@useobject{currentmarker}{}%
\end{pgfscope}%
\end{pgfscope}%
\begin{pgfscope}%
\pgfsetbuttcap%
\pgfsetroundjoin%
\definecolor{currentfill}{rgb}{0.150000,0.150000,0.150000}%
\pgfsetfillcolor{currentfill}%
\pgfsetlinewidth{0.803000pt}%
\definecolor{currentstroke}{rgb}{0.150000,0.150000,0.150000}%
\pgfsetstrokecolor{currentstroke}%
\pgfsetdash{}{0pt}%
\pgfsys@defobject{currentmarker}{\pgfqpoint{0.000000in}{0.000000in}}{\pgfqpoint{0.000000in}{0.000000in}}{%
\pgfpathmoveto{\pgfqpoint{0.000000in}{0.000000in}}%
\pgfpathlineto{\pgfqpoint{0.000000in}{0.000000in}}%
\pgfusepath{stroke,fill}%
}%
\begin{pgfscope}%
\pgfsys@transformshift{1.422794in}{2.700000in}%
\pgfsys@useobject{currentmarker}{}%
\end{pgfscope}%
\end{pgfscope}%
\begin{pgfscope}%
\definecolor{textcolor}{rgb}{0.150000,0.150000,0.150000}%
\pgfsetstrokecolor{textcolor}%
\pgfsetfillcolor{textcolor}%
\pgftext[x=1.422794in,y=0.297222in,,top]{\color{textcolor}\sffamily\fontsize{8.000000}{9.600000}\selectfont −1.0}%
\end{pgfscope}%
\begin{pgfscope}%
\pgfpathrectangle{\pgfqpoint{0.625000in}{0.375000in}}{\pgfqpoint{3.875000in}{2.325000in}} %
\pgfusepath{clip}%
\pgfsetroundcap%
\pgfsetroundjoin%
\pgfsetlinewidth{0.803000pt}%
\definecolor{currentstroke}{rgb}{1.000000,1.000000,1.000000}%
\pgfsetstrokecolor{currentstroke}%
\pgfsetdash{}{0pt}%
\pgfpathmoveto{\pgfqpoint{1.992647in}{0.375000in}}%
\pgfpathlineto{\pgfqpoint{1.992647in}{2.700000in}}%
\pgfusepath{stroke}%
\end{pgfscope}%
\begin{pgfscope}%
\pgfsetbuttcap%
\pgfsetroundjoin%
\definecolor{currentfill}{rgb}{0.150000,0.150000,0.150000}%
\pgfsetfillcolor{currentfill}%
\pgfsetlinewidth{0.803000pt}%
\definecolor{currentstroke}{rgb}{0.150000,0.150000,0.150000}%
\pgfsetstrokecolor{currentstroke}%
\pgfsetdash{}{0pt}%
\pgfsys@defobject{currentmarker}{\pgfqpoint{0.000000in}{0.000000in}}{\pgfqpoint{0.000000in}{0.000000in}}{%
\pgfpathmoveto{\pgfqpoint{0.000000in}{0.000000in}}%
\pgfpathlineto{\pgfqpoint{0.000000in}{0.000000in}}%
\pgfusepath{stroke,fill}%
}%
\begin{pgfscope}%
\pgfsys@transformshift{1.992647in}{0.375000in}%
\pgfsys@useobject{currentmarker}{}%
\end{pgfscope}%
\end{pgfscope}%
\begin{pgfscope}%
\pgfsetbuttcap%
\pgfsetroundjoin%
\definecolor{currentfill}{rgb}{0.150000,0.150000,0.150000}%
\pgfsetfillcolor{currentfill}%
\pgfsetlinewidth{0.803000pt}%
\definecolor{currentstroke}{rgb}{0.150000,0.150000,0.150000}%
\pgfsetstrokecolor{currentstroke}%
\pgfsetdash{}{0pt}%
\pgfsys@defobject{currentmarker}{\pgfqpoint{0.000000in}{0.000000in}}{\pgfqpoint{0.000000in}{0.000000in}}{%
\pgfpathmoveto{\pgfqpoint{0.000000in}{0.000000in}}%
\pgfpathlineto{\pgfqpoint{0.000000in}{0.000000in}}%
\pgfusepath{stroke,fill}%
}%
\begin{pgfscope}%
\pgfsys@transformshift{1.992647in}{2.700000in}%
\pgfsys@useobject{currentmarker}{}%
\end{pgfscope}%
\end{pgfscope}%
\begin{pgfscope}%
\definecolor{textcolor}{rgb}{0.150000,0.150000,0.150000}%
\pgfsetstrokecolor{textcolor}%
\pgfsetfillcolor{textcolor}%
\pgftext[x=1.992647in,y=0.297222in,,top]{\color{textcolor}\sffamily\fontsize{8.000000}{9.600000}\selectfont −0.5}%
\end{pgfscope}%
\begin{pgfscope}%
\pgfpathrectangle{\pgfqpoint{0.625000in}{0.375000in}}{\pgfqpoint{3.875000in}{2.325000in}} %
\pgfusepath{clip}%
\pgfsetroundcap%
\pgfsetroundjoin%
\pgfsetlinewidth{0.803000pt}%
\definecolor{currentstroke}{rgb}{1.000000,1.000000,1.000000}%
\pgfsetstrokecolor{currentstroke}%
\pgfsetdash{}{0pt}%
\pgfpathmoveto{\pgfqpoint{2.562500in}{0.375000in}}%
\pgfpathlineto{\pgfqpoint{2.562500in}{2.700000in}}%
\pgfusepath{stroke}%
\end{pgfscope}%
\begin{pgfscope}%
\pgfsetbuttcap%
\pgfsetroundjoin%
\definecolor{currentfill}{rgb}{0.150000,0.150000,0.150000}%
\pgfsetfillcolor{currentfill}%
\pgfsetlinewidth{0.803000pt}%
\definecolor{currentstroke}{rgb}{0.150000,0.150000,0.150000}%
\pgfsetstrokecolor{currentstroke}%
\pgfsetdash{}{0pt}%
\pgfsys@defobject{currentmarker}{\pgfqpoint{0.000000in}{0.000000in}}{\pgfqpoint{0.000000in}{0.000000in}}{%
\pgfpathmoveto{\pgfqpoint{0.000000in}{0.000000in}}%
\pgfpathlineto{\pgfqpoint{0.000000in}{0.000000in}}%
\pgfusepath{stroke,fill}%
}%
\begin{pgfscope}%
\pgfsys@transformshift{2.562500in}{0.375000in}%
\pgfsys@useobject{currentmarker}{}%
\end{pgfscope}%
\end{pgfscope}%
\begin{pgfscope}%
\pgfsetbuttcap%
\pgfsetroundjoin%
\definecolor{currentfill}{rgb}{0.150000,0.150000,0.150000}%
\pgfsetfillcolor{currentfill}%
\pgfsetlinewidth{0.803000pt}%
\definecolor{currentstroke}{rgb}{0.150000,0.150000,0.150000}%
\pgfsetstrokecolor{currentstroke}%
\pgfsetdash{}{0pt}%
\pgfsys@defobject{currentmarker}{\pgfqpoint{0.000000in}{0.000000in}}{\pgfqpoint{0.000000in}{0.000000in}}{%
\pgfpathmoveto{\pgfqpoint{0.000000in}{0.000000in}}%
\pgfpathlineto{\pgfqpoint{0.000000in}{0.000000in}}%
\pgfusepath{stroke,fill}%
}%
\begin{pgfscope}%
\pgfsys@transformshift{2.562500in}{2.700000in}%
\pgfsys@useobject{currentmarker}{}%
\end{pgfscope}%
\end{pgfscope}%
\begin{pgfscope}%
\definecolor{textcolor}{rgb}{0.150000,0.150000,0.150000}%
\pgfsetstrokecolor{textcolor}%
\pgfsetfillcolor{textcolor}%
\pgftext[x=2.562500in,y=0.297222in,,top]{\color{textcolor}\sffamily\fontsize{8.000000}{9.600000}\selectfont 0.0}%
\end{pgfscope}%
\begin{pgfscope}%
\pgfpathrectangle{\pgfqpoint{0.625000in}{0.375000in}}{\pgfqpoint{3.875000in}{2.325000in}} %
\pgfusepath{clip}%
\pgfsetroundcap%
\pgfsetroundjoin%
\pgfsetlinewidth{0.803000pt}%
\definecolor{currentstroke}{rgb}{1.000000,1.000000,1.000000}%
\pgfsetstrokecolor{currentstroke}%
\pgfsetdash{}{0pt}%
\pgfpathmoveto{\pgfqpoint{3.132353in}{0.375000in}}%
\pgfpathlineto{\pgfqpoint{3.132353in}{2.700000in}}%
\pgfusepath{stroke}%
\end{pgfscope}%
\begin{pgfscope}%
\pgfsetbuttcap%
\pgfsetroundjoin%
\definecolor{currentfill}{rgb}{0.150000,0.150000,0.150000}%
\pgfsetfillcolor{currentfill}%
\pgfsetlinewidth{0.803000pt}%
\definecolor{currentstroke}{rgb}{0.150000,0.150000,0.150000}%
\pgfsetstrokecolor{currentstroke}%
\pgfsetdash{}{0pt}%
\pgfsys@defobject{currentmarker}{\pgfqpoint{0.000000in}{0.000000in}}{\pgfqpoint{0.000000in}{0.000000in}}{%
\pgfpathmoveto{\pgfqpoint{0.000000in}{0.000000in}}%
\pgfpathlineto{\pgfqpoint{0.000000in}{0.000000in}}%
\pgfusepath{stroke,fill}%
}%
\begin{pgfscope}%
\pgfsys@transformshift{3.132353in}{0.375000in}%
\pgfsys@useobject{currentmarker}{}%
\end{pgfscope}%
\end{pgfscope}%
\begin{pgfscope}%
\pgfsetbuttcap%
\pgfsetroundjoin%
\definecolor{currentfill}{rgb}{0.150000,0.150000,0.150000}%
\pgfsetfillcolor{currentfill}%
\pgfsetlinewidth{0.803000pt}%
\definecolor{currentstroke}{rgb}{0.150000,0.150000,0.150000}%
\pgfsetstrokecolor{currentstroke}%
\pgfsetdash{}{0pt}%
\pgfsys@defobject{currentmarker}{\pgfqpoint{0.000000in}{0.000000in}}{\pgfqpoint{0.000000in}{0.000000in}}{%
\pgfpathmoveto{\pgfqpoint{0.000000in}{0.000000in}}%
\pgfpathlineto{\pgfqpoint{0.000000in}{0.000000in}}%
\pgfusepath{stroke,fill}%
}%
\begin{pgfscope}%
\pgfsys@transformshift{3.132353in}{2.700000in}%
\pgfsys@useobject{currentmarker}{}%
\end{pgfscope}%
\end{pgfscope}%
\begin{pgfscope}%
\definecolor{textcolor}{rgb}{0.150000,0.150000,0.150000}%
\pgfsetstrokecolor{textcolor}%
\pgfsetfillcolor{textcolor}%
\pgftext[x=3.132353in,y=0.297222in,,top]{\color{textcolor}\sffamily\fontsize{8.000000}{9.600000}\selectfont 0.5}%
\end{pgfscope}%
\begin{pgfscope}%
\pgfpathrectangle{\pgfqpoint{0.625000in}{0.375000in}}{\pgfqpoint{3.875000in}{2.325000in}} %
\pgfusepath{clip}%
\pgfsetroundcap%
\pgfsetroundjoin%
\pgfsetlinewidth{0.803000pt}%
\definecolor{currentstroke}{rgb}{1.000000,1.000000,1.000000}%
\pgfsetstrokecolor{currentstroke}%
\pgfsetdash{}{0pt}%
\pgfpathmoveto{\pgfqpoint{3.702206in}{0.375000in}}%
\pgfpathlineto{\pgfqpoint{3.702206in}{2.700000in}}%
\pgfusepath{stroke}%
\end{pgfscope}%
\begin{pgfscope}%
\pgfsetbuttcap%
\pgfsetroundjoin%
\definecolor{currentfill}{rgb}{0.150000,0.150000,0.150000}%
\pgfsetfillcolor{currentfill}%
\pgfsetlinewidth{0.803000pt}%
\definecolor{currentstroke}{rgb}{0.150000,0.150000,0.150000}%
\pgfsetstrokecolor{currentstroke}%
\pgfsetdash{}{0pt}%
\pgfsys@defobject{currentmarker}{\pgfqpoint{0.000000in}{0.000000in}}{\pgfqpoint{0.000000in}{0.000000in}}{%
\pgfpathmoveto{\pgfqpoint{0.000000in}{0.000000in}}%
\pgfpathlineto{\pgfqpoint{0.000000in}{0.000000in}}%
\pgfusepath{stroke,fill}%
}%
\begin{pgfscope}%
\pgfsys@transformshift{3.702206in}{0.375000in}%
\pgfsys@useobject{currentmarker}{}%
\end{pgfscope}%
\end{pgfscope}%
\begin{pgfscope}%
\pgfsetbuttcap%
\pgfsetroundjoin%
\definecolor{currentfill}{rgb}{0.150000,0.150000,0.150000}%
\pgfsetfillcolor{currentfill}%
\pgfsetlinewidth{0.803000pt}%
\definecolor{currentstroke}{rgb}{0.150000,0.150000,0.150000}%
\pgfsetstrokecolor{currentstroke}%
\pgfsetdash{}{0pt}%
\pgfsys@defobject{currentmarker}{\pgfqpoint{0.000000in}{0.000000in}}{\pgfqpoint{0.000000in}{0.000000in}}{%
\pgfpathmoveto{\pgfqpoint{0.000000in}{0.000000in}}%
\pgfpathlineto{\pgfqpoint{0.000000in}{0.000000in}}%
\pgfusepath{stroke,fill}%
}%
\begin{pgfscope}%
\pgfsys@transformshift{3.702206in}{2.700000in}%
\pgfsys@useobject{currentmarker}{}%
\end{pgfscope}%
\end{pgfscope}%
\begin{pgfscope}%
\definecolor{textcolor}{rgb}{0.150000,0.150000,0.150000}%
\pgfsetstrokecolor{textcolor}%
\pgfsetfillcolor{textcolor}%
\pgftext[x=3.702206in,y=0.297222in,,top]{\color{textcolor}\sffamily\fontsize{8.000000}{9.600000}\selectfont 1.0}%
\end{pgfscope}%
\begin{pgfscope}%
\pgfpathrectangle{\pgfqpoint{0.625000in}{0.375000in}}{\pgfqpoint{3.875000in}{2.325000in}} %
\pgfusepath{clip}%
\pgfsetroundcap%
\pgfsetroundjoin%
\pgfsetlinewidth{0.803000pt}%
\definecolor{currentstroke}{rgb}{1.000000,1.000000,1.000000}%
\pgfsetstrokecolor{currentstroke}%
\pgfsetdash{}{0pt}%
\pgfpathmoveto{\pgfqpoint{4.272059in}{0.375000in}}%
\pgfpathlineto{\pgfqpoint{4.272059in}{2.700000in}}%
\pgfusepath{stroke}%
\end{pgfscope}%
\begin{pgfscope}%
\pgfsetbuttcap%
\pgfsetroundjoin%
\definecolor{currentfill}{rgb}{0.150000,0.150000,0.150000}%
\pgfsetfillcolor{currentfill}%
\pgfsetlinewidth{0.803000pt}%
\definecolor{currentstroke}{rgb}{0.150000,0.150000,0.150000}%
\pgfsetstrokecolor{currentstroke}%
\pgfsetdash{}{0pt}%
\pgfsys@defobject{currentmarker}{\pgfqpoint{0.000000in}{0.000000in}}{\pgfqpoint{0.000000in}{0.000000in}}{%
\pgfpathmoveto{\pgfqpoint{0.000000in}{0.000000in}}%
\pgfpathlineto{\pgfqpoint{0.000000in}{0.000000in}}%
\pgfusepath{stroke,fill}%
}%
\begin{pgfscope}%
\pgfsys@transformshift{4.272059in}{0.375000in}%
\pgfsys@useobject{currentmarker}{}%
\end{pgfscope}%
\end{pgfscope}%
\begin{pgfscope}%
\pgfsetbuttcap%
\pgfsetroundjoin%
\definecolor{currentfill}{rgb}{0.150000,0.150000,0.150000}%
\pgfsetfillcolor{currentfill}%
\pgfsetlinewidth{0.803000pt}%
\definecolor{currentstroke}{rgb}{0.150000,0.150000,0.150000}%
\pgfsetstrokecolor{currentstroke}%
\pgfsetdash{}{0pt}%
\pgfsys@defobject{currentmarker}{\pgfqpoint{0.000000in}{0.000000in}}{\pgfqpoint{0.000000in}{0.000000in}}{%
\pgfpathmoveto{\pgfqpoint{0.000000in}{0.000000in}}%
\pgfpathlineto{\pgfqpoint{0.000000in}{0.000000in}}%
\pgfusepath{stroke,fill}%
}%
\begin{pgfscope}%
\pgfsys@transformshift{4.272059in}{2.700000in}%
\pgfsys@useobject{currentmarker}{}%
\end{pgfscope}%
\end{pgfscope}%
\begin{pgfscope}%
\definecolor{textcolor}{rgb}{0.150000,0.150000,0.150000}%
\pgfsetstrokecolor{textcolor}%
\pgfsetfillcolor{textcolor}%
\pgftext[x=4.272059in,y=0.297222in,,top]{\color{textcolor}\sffamily\fontsize{8.000000}{9.600000}\selectfont 1.5}%
\end{pgfscope}%
\begin{pgfscope}%
\definecolor{textcolor}{rgb}{0.150000,0.150000,0.150000}%
\pgfsetstrokecolor{textcolor}%
\pgfsetfillcolor{textcolor}%
\pgftext[x=2.562500in,y=0.132099in,,top]{\color{textcolor}\sffamily\fontsize{8.800000}{10.560000}\selectfont real values}%
\end{pgfscope}%
\begin{pgfscope}%
\pgfpathrectangle{\pgfqpoint{0.625000in}{0.375000in}}{\pgfqpoint{3.875000in}{2.325000in}} %
\pgfusepath{clip}%
\pgfsetroundcap%
\pgfsetroundjoin%
\pgfsetlinewidth{0.803000pt}%
\definecolor{currentstroke}{rgb}{1.000000,1.000000,1.000000}%
\pgfsetstrokecolor{currentstroke}%
\pgfsetdash{}{0pt}%
\pgfpathmoveto{\pgfqpoint{0.625000in}{0.511765in}}%
\pgfpathlineto{\pgfqpoint{4.500000in}{0.511765in}}%
\pgfusepath{stroke}%
\end{pgfscope}%
\begin{pgfscope}%
\pgfsetbuttcap%
\pgfsetroundjoin%
\definecolor{currentfill}{rgb}{0.150000,0.150000,0.150000}%
\pgfsetfillcolor{currentfill}%
\pgfsetlinewidth{0.803000pt}%
\definecolor{currentstroke}{rgb}{0.150000,0.150000,0.150000}%
\pgfsetstrokecolor{currentstroke}%
\pgfsetdash{}{0pt}%
\pgfsys@defobject{currentmarker}{\pgfqpoint{0.000000in}{0.000000in}}{\pgfqpoint{0.000000in}{0.000000in}}{%
\pgfpathmoveto{\pgfqpoint{0.000000in}{0.000000in}}%
\pgfpathlineto{\pgfqpoint{0.000000in}{0.000000in}}%
\pgfusepath{stroke,fill}%
}%
\begin{pgfscope}%
\pgfsys@transformshift{0.625000in}{0.511765in}%
\pgfsys@useobject{currentmarker}{}%
\end{pgfscope}%
\end{pgfscope}%
\begin{pgfscope}%
\pgfsetbuttcap%
\pgfsetroundjoin%
\definecolor{currentfill}{rgb}{0.150000,0.150000,0.150000}%
\pgfsetfillcolor{currentfill}%
\pgfsetlinewidth{0.803000pt}%
\definecolor{currentstroke}{rgb}{0.150000,0.150000,0.150000}%
\pgfsetstrokecolor{currentstroke}%
\pgfsetdash{}{0pt}%
\pgfsys@defobject{currentmarker}{\pgfqpoint{0.000000in}{0.000000in}}{\pgfqpoint{0.000000in}{0.000000in}}{%
\pgfpathmoveto{\pgfqpoint{0.000000in}{0.000000in}}%
\pgfpathlineto{\pgfqpoint{0.000000in}{0.000000in}}%
\pgfusepath{stroke,fill}%
}%
\begin{pgfscope}%
\pgfsys@transformshift{4.500000in}{0.511765in}%
\pgfsys@useobject{currentmarker}{}%
\end{pgfscope}%
\end{pgfscope}%
\begin{pgfscope}%
\definecolor{textcolor}{rgb}{0.150000,0.150000,0.150000}%
\pgfsetstrokecolor{textcolor}%
\pgfsetfillcolor{textcolor}%
\pgftext[x=0.547222in,y=0.511765in,right,]{\color{textcolor}\sffamily\fontsize{8.000000}{9.600000}\selectfont −1.5}%
\end{pgfscope}%
\begin{pgfscope}%
\pgfpathrectangle{\pgfqpoint{0.625000in}{0.375000in}}{\pgfqpoint{3.875000in}{2.325000in}} %
\pgfusepath{clip}%
\pgfsetroundcap%
\pgfsetroundjoin%
\pgfsetlinewidth{0.803000pt}%
\definecolor{currentstroke}{rgb}{1.000000,1.000000,1.000000}%
\pgfsetstrokecolor{currentstroke}%
\pgfsetdash{}{0pt}%
\pgfpathmoveto{\pgfqpoint{0.625000in}{0.853676in}}%
\pgfpathlineto{\pgfqpoint{4.500000in}{0.853676in}}%
\pgfusepath{stroke}%
\end{pgfscope}%
\begin{pgfscope}%
\pgfsetbuttcap%
\pgfsetroundjoin%
\definecolor{currentfill}{rgb}{0.150000,0.150000,0.150000}%
\pgfsetfillcolor{currentfill}%
\pgfsetlinewidth{0.803000pt}%
\definecolor{currentstroke}{rgb}{0.150000,0.150000,0.150000}%
\pgfsetstrokecolor{currentstroke}%
\pgfsetdash{}{0pt}%
\pgfsys@defobject{currentmarker}{\pgfqpoint{0.000000in}{0.000000in}}{\pgfqpoint{0.000000in}{0.000000in}}{%
\pgfpathmoveto{\pgfqpoint{0.000000in}{0.000000in}}%
\pgfpathlineto{\pgfqpoint{0.000000in}{0.000000in}}%
\pgfusepath{stroke,fill}%
}%
\begin{pgfscope}%
\pgfsys@transformshift{0.625000in}{0.853676in}%
\pgfsys@useobject{currentmarker}{}%
\end{pgfscope}%
\end{pgfscope}%
\begin{pgfscope}%
\pgfsetbuttcap%
\pgfsetroundjoin%
\definecolor{currentfill}{rgb}{0.150000,0.150000,0.150000}%
\pgfsetfillcolor{currentfill}%
\pgfsetlinewidth{0.803000pt}%
\definecolor{currentstroke}{rgb}{0.150000,0.150000,0.150000}%
\pgfsetstrokecolor{currentstroke}%
\pgfsetdash{}{0pt}%
\pgfsys@defobject{currentmarker}{\pgfqpoint{0.000000in}{0.000000in}}{\pgfqpoint{0.000000in}{0.000000in}}{%
\pgfpathmoveto{\pgfqpoint{0.000000in}{0.000000in}}%
\pgfpathlineto{\pgfqpoint{0.000000in}{0.000000in}}%
\pgfusepath{stroke,fill}%
}%
\begin{pgfscope}%
\pgfsys@transformshift{4.500000in}{0.853676in}%
\pgfsys@useobject{currentmarker}{}%
\end{pgfscope}%
\end{pgfscope}%
\begin{pgfscope}%
\definecolor{textcolor}{rgb}{0.150000,0.150000,0.150000}%
\pgfsetstrokecolor{textcolor}%
\pgfsetfillcolor{textcolor}%
\pgftext[x=0.547222in,y=0.853676in,right,]{\color{textcolor}\sffamily\fontsize{8.000000}{9.600000}\selectfont −1.0}%
\end{pgfscope}%
\begin{pgfscope}%
\pgfpathrectangle{\pgfqpoint{0.625000in}{0.375000in}}{\pgfqpoint{3.875000in}{2.325000in}} %
\pgfusepath{clip}%
\pgfsetroundcap%
\pgfsetroundjoin%
\pgfsetlinewidth{0.803000pt}%
\definecolor{currentstroke}{rgb}{1.000000,1.000000,1.000000}%
\pgfsetstrokecolor{currentstroke}%
\pgfsetdash{}{0pt}%
\pgfpathmoveto{\pgfqpoint{0.625000in}{1.195588in}}%
\pgfpathlineto{\pgfqpoint{4.500000in}{1.195588in}}%
\pgfusepath{stroke}%
\end{pgfscope}%
\begin{pgfscope}%
\pgfsetbuttcap%
\pgfsetroundjoin%
\definecolor{currentfill}{rgb}{0.150000,0.150000,0.150000}%
\pgfsetfillcolor{currentfill}%
\pgfsetlinewidth{0.803000pt}%
\definecolor{currentstroke}{rgb}{0.150000,0.150000,0.150000}%
\pgfsetstrokecolor{currentstroke}%
\pgfsetdash{}{0pt}%
\pgfsys@defobject{currentmarker}{\pgfqpoint{0.000000in}{0.000000in}}{\pgfqpoint{0.000000in}{0.000000in}}{%
\pgfpathmoveto{\pgfqpoint{0.000000in}{0.000000in}}%
\pgfpathlineto{\pgfqpoint{0.000000in}{0.000000in}}%
\pgfusepath{stroke,fill}%
}%
\begin{pgfscope}%
\pgfsys@transformshift{0.625000in}{1.195588in}%
\pgfsys@useobject{currentmarker}{}%
\end{pgfscope}%
\end{pgfscope}%
\begin{pgfscope}%
\pgfsetbuttcap%
\pgfsetroundjoin%
\definecolor{currentfill}{rgb}{0.150000,0.150000,0.150000}%
\pgfsetfillcolor{currentfill}%
\pgfsetlinewidth{0.803000pt}%
\definecolor{currentstroke}{rgb}{0.150000,0.150000,0.150000}%
\pgfsetstrokecolor{currentstroke}%
\pgfsetdash{}{0pt}%
\pgfsys@defobject{currentmarker}{\pgfqpoint{0.000000in}{0.000000in}}{\pgfqpoint{0.000000in}{0.000000in}}{%
\pgfpathmoveto{\pgfqpoint{0.000000in}{0.000000in}}%
\pgfpathlineto{\pgfqpoint{0.000000in}{0.000000in}}%
\pgfusepath{stroke,fill}%
}%
\begin{pgfscope}%
\pgfsys@transformshift{4.500000in}{1.195588in}%
\pgfsys@useobject{currentmarker}{}%
\end{pgfscope}%
\end{pgfscope}%
\begin{pgfscope}%
\definecolor{textcolor}{rgb}{0.150000,0.150000,0.150000}%
\pgfsetstrokecolor{textcolor}%
\pgfsetfillcolor{textcolor}%
\pgftext[x=0.547222in,y=1.195588in,right,]{\color{textcolor}\sffamily\fontsize{8.000000}{9.600000}\selectfont −0.5}%
\end{pgfscope}%
\begin{pgfscope}%
\pgfpathrectangle{\pgfqpoint{0.625000in}{0.375000in}}{\pgfqpoint{3.875000in}{2.325000in}} %
\pgfusepath{clip}%
\pgfsetroundcap%
\pgfsetroundjoin%
\pgfsetlinewidth{0.803000pt}%
\definecolor{currentstroke}{rgb}{1.000000,1.000000,1.000000}%
\pgfsetstrokecolor{currentstroke}%
\pgfsetdash{}{0pt}%
\pgfpathmoveto{\pgfqpoint{0.625000in}{1.537500in}}%
\pgfpathlineto{\pgfqpoint{4.500000in}{1.537500in}}%
\pgfusepath{stroke}%
\end{pgfscope}%
\begin{pgfscope}%
\pgfsetbuttcap%
\pgfsetroundjoin%
\definecolor{currentfill}{rgb}{0.150000,0.150000,0.150000}%
\pgfsetfillcolor{currentfill}%
\pgfsetlinewidth{0.803000pt}%
\definecolor{currentstroke}{rgb}{0.150000,0.150000,0.150000}%
\pgfsetstrokecolor{currentstroke}%
\pgfsetdash{}{0pt}%
\pgfsys@defobject{currentmarker}{\pgfqpoint{0.000000in}{0.000000in}}{\pgfqpoint{0.000000in}{0.000000in}}{%
\pgfpathmoveto{\pgfqpoint{0.000000in}{0.000000in}}%
\pgfpathlineto{\pgfqpoint{0.000000in}{0.000000in}}%
\pgfusepath{stroke,fill}%
}%
\begin{pgfscope}%
\pgfsys@transformshift{0.625000in}{1.537500in}%
\pgfsys@useobject{currentmarker}{}%
\end{pgfscope}%
\end{pgfscope}%
\begin{pgfscope}%
\pgfsetbuttcap%
\pgfsetroundjoin%
\definecolor{currentfill}{rgb}{0.150000,0.150000,0.150000}%
\pgfsetfillcolor{currentfill}%
\pgfsetlinewidth{0.803000pt}%
\definecolor{currentstroke}{rgb}{0.150000,0.150000,0.150000}%
\pgfsetstrokecolor{currentstroke}%
\pgfsetdash{}{0pt}%
\pgfsys@defobject{currentmarker}{\pgfqpoint{0.000000in}{0.000000in}}{\pgfqpoint{0.000000in}{0.000000in}}{%
\pgfpathmoveto{\pgfqpoint{0.000000in}{0.000000in}}%
\pgfpathlineto{\pgfqpoint{0.000000in}{0.000000in}}%
\pgfusepath{stroke,fill}%
}%
\begin{pgfscope}%
\pgfsys@transformshift{4.500000in}{1.537500in}%
\pgfsys@useobject{currentmarker}{}%
\end{pgfscope}%
\end{pgfscope}%
\begin{pgfscope}%
\definecolor{textcolor}{rgb}{0.150000,0.150000,0.150000}%
\pgfsetstrokecolor{textcolor}%
\pgfsetfillcolor{textcolor}%
\pgftext[x=0.547222in,y=1.537500in,right,]{\color{textcolor}\sffamily\fontsize{8.000000}{9.600000}\selectfont 0.0}%
\end{pgfscope}%
\begin{pgfscope}%
\pgfpathrectangle{\pgfqpoint{0.625000in}{0.375000in}}{\pgfqpoint{3.875000in}{2.325000in}} %
\pgfusepath{clip}%
\pgfsetroundcap%
\pgfsetroundjoin%
\pgfsetlinewidth{0.803000pt}%
\definecolor{currentstroke}{rgb}{1.000000,1.000000,1.000000}%
\pgfsetstrokecolor{currentstroke}%
\pgfsetdash{}{0pt}%
\pgfpathmoveto{\pgfqpoint{0.625000in}{1.879412in}}%
\pgfpathlineto{\pgfqpoint{4.500000in}{1.879412in}}%
\pgfusepath{stroke}%
\end{pgfscope}%
\begin{pgfscope}%
\pgfsetbuttcap%
\pgfsetroundjoin%
\definecolor{currentfill}{rgb}{0.150000,0.150000,0.150000}%
\pgfsetfillcolor{currentfill}%
\pgfsetlinewidth{0.803000pt}%
\definecolor{currentstroke}{rgb}{0.150000,0.150000,0.150000}%
\pgfsetstrokecolor{currentstroke}%
\pgfsetdash{}{0pt}%
\pgfsys@defobject{currentmarker}{\pgfqpoint{0.000000in}{0.000000in}}{\pgfqpoint{0.000000in}{0.000000in}}{%
\pgfpathmoveto{\pgfqpoint{0.000000in}{0.000000in}}%
\pgfpathlineto{\pgfqpoint{0.000000in}{0.000000in}}%
\pgfusepath{stroke,fill}%
}%
\begin{pgfscope}%
\pgfsys@transformshift{0.625000in}{1.879412in}%
\pgfsys@useobject{currentmarker}{}%
\end{pgfscope}%
\end{pgfscope}%
\begin{pgfscope}%
\pgfsetbuttcap%
\pgfsetroundjoin%
\definecolor{currentfill}{rgb}{0.150000,0.150000,0.150000}%
\pgfsetfillcolor{currentfill}%
\pgfsetlinewidth{0.803000pt}%
\definecolor{currentstroke}{rgb}{0.150000,0.150000,0.150000}%
\pgfsetstrokecolor{currentstroke}%
\pgfsetdash{}{0pt}%
\pgfsys@defobject{currentmarker}{\pgfqpoint{0.000000in}{0.000000in}}{\pgfqpoint{0.000000in}{0.000000in}}{%
\pgfpathmoveto{\pgfqpoint{0.000000in}{0.000000in}}%
\pgfpathlineto{\pgfqpoint{0.000000in}{0.000000in}}%
\pgfusepath{stroke,fill}%
}%
\begin{pgfscope}%
\pgfsys@transformshift{4.500000in}{1.879412in}%
\pgfsys@useobject{currentmarker}{}%
\end{pgfscope}%
\end{pgfscope}%
\begin{pgfscope}%
\definecolor{textcolor}{rgb}{0.150000,0.150000,0.150000}%
\pgfsetstrokecolor{textcolor}%
\pgfsetfillcolor{textcolor}%
\pgftext[x=0.547222in,y=1.879412in,right,]{\color{textcolor}\sffamily\fontsize{8.000000}{9.600000}\selectfont 0.5}%
\end{pgfscope}%
\begin{pgfscope}%
\pgfpathrectangle{\pgfqpoint{0.625000in}{0.375000in}}{\pgfqpoint{3.875000in}{2.325000in}} %
\pgfusepath{clip}%
\pgfsetroundcap%
\pgfsetroundjoin%
\pgfsetlinewidth{0.803000pt}%
\definecolor{currentstroke}{rgb}{1.000000,1.000000,1.000000}%
\pgfsetstrokecolor{currentstroke}%
\pgfsetdash{}{0pt}%
\pgfpathmoveto{\pgfqpoint{0.625000in}{2.221324in}}%
\pgfpathlineto{\pgfqpoint{4.500000in}{2.221324in}}%
\pgfusepath{stroke}%
\end{pgfscope}%
\begin{pgfscope}%
\pgfsetbuttcap%
\pgfsetroundjoin%
\definecolor{currentfill}{rgb}{0.150000,0.150000,0.150000}%
\pgfsetfillcolor{currentfill}%
\pgfsetlinewidth{0.803000pt}%
\definecolor{currentstroke}{rgb}{0.150000,0.150000,0.150000}%
\pgfsetstrokecolor{currentstroke}%
\pgfsetdash{}{0pt}%
\pgfsys@defobject{currentmarker}{\pgfqpoint{0.000000in}{0.000000in}}{\pgfqpoint{0.000000in}{0.000000in}}{%
\pgfpathmoveto{\pgfqpoint{0.000000in}{0.000000in}}%
\pgfpathlineto{\pgfqpoint{0.000000in}{0.000000in}}%
\pgfusepath{stroke,fill}%
}%
\begin{pgfscope}%
\pgfsys@transformshift{0.625000in}{2.221324in}%
\pgfsys@useobject{currentmarker}{}%
\end{pgfscope}%
\end{pgfscope}%
\begin{pgfscope}%
\pgfsetbuttcap%
\pgfsetroundjoin%
\definecolor{currentfill}{rgb}{0.150000,0.150000,0.150000}%
\pgfsetfillcolor{currentfill}%
\pgfsetlinewidth{0.803000pt}%
\definecolor{currentstroke}{rgb}{0.150000,0.150000,0.150000}%
\pgfsetstrokecolor{currentstroke}%
\pgfsetdash{}{0pt}%
\pgfsys@defobject{currentmarker}{\pgfqpoint{0.000000in}{0.000000in}}{\pgfqpoint{0.000000in}{0.000000in}}{%
\pgfpathmoveto{\pgfqpoint{0.000000in}{0.000000in}}%
\pgfpathlineto{\pgfqpoint{0.000000in}{0.000000in}}%
\pgfusepath{stroke,fill}%
}%
\begin{pgfscope}%
\pgfsys@transformshift{4.500000in}{2.221324in}%
\pgfsys@useobject{currentmarker}{}%
\end{pgfscope}%
\end{pgfscope}%
\begin{pgfscope}%
\definecolor{textcolor}{rgb}{0.150000,0.150000,0.150000}%
\pgfsetstrokecolor{textcolor}%
\pgfsetfillcolor{textcolor}%
\pgftext[x=0.547222in,y=2.221324in,right,]{\color{textcolor}\sffamily\fontsize{8.000000}{9.600000}\selectfont 1.0}%
\end{pgfscope}%
\begin{pgfscope}%
\pgfpathrectangle{\pgfqpoint{0.625000in}{0.375000in}}{\pgfqpoint{3.875000in}{2.325000in}} %
\pgfusepath{clip}%
\pgfsetroundcap%
\pgfsetroundjoin%
\pgfsetlinewidth{0.803000pt}%
\definecolor{currentstroke}{rgb}{1.000000,1.000000,1.000000}%
\pgfsetstrokecolor{currentstroke}%
\pgfsetdash{}{0pt}%
\pgfpathmoveto{\pgfqpoint{0.625000in}{2.563235in}}%
\pgfpathlineto{\pgfqpoint{4.500000in}{2.563235in}}%
\pgfusepath{stroke}%
\end{pgfscope}%
\begin{pgfscope}%
\pgfsetbuttcap%
\pgfsetroundjoin%
\definecolor{currentfill}{rgb}{0.150000,0.150000,0.150000}%
\pgfsetfillcolor{currentfill}%
\pgfsetlinewidth{0.803000pt}%
\definecolor{currentstroke}{rgb}{0.150000,0.150000,0.150000}%
\pgfsetstrokecolor{currentstroke}%
\pgfsetdash{}{0pt}%
\pgfsys@defobject{currentmarker}{\pgfqpoint{0.000000in}{0.000000in}}{\pgfqpoint{0.000000in}{0.000000in}}{%
\pgfpathmoveto{\pgfqpoint{0.000000in}{0.000000in}}%
\pgfpathlineto{\pgfqpoint{0.000000in}{0.000000in}}%
\pgfusepath{stroke,fill}%
}%
\begin{pgfscope}%
\pgfsys@transformshift{0.625000in}{2.563235in}%
\pgfsys@useobject{currentmarker}{}%
\end{pgfscope}%
\end{pgfscope}%
\begin{pgfscope}%
\pgfsetbuttcap%
\pgfsetroundjoin%
\definecolor{currentfill}{rgb}{0.150000,0.150000,0.150000}%
\pgfsetfillcolor{currentfill}%
\pgfsetlinewidth{0.803000pt}%
\definecolor{currentstroke}{rgb}{0.150000,0.150000,0.150000}%
\pgfsetstrokecolor{currentstroke}%
\pgfsetdash{}{0pt}%
\pgfsys@defobject{currentmarker}{\pgfqpoint{0.000000in}{0.000000in}}{\pgfqpoint{0.000000in}{0.000000in}}{%
\pgfpathmoveto{\pgfqpoint{0.000000in}{0.000000in}}%
\pgfpathlineto{\pgfqpoint{0.000000in}{0.000000in}}%
\pgfusepath{stroke,fill}%
}%
\begin{pgfscope}%
\pgfsys@transformshift{4.500000in}{2.563235in}%
\pgfsys@useobject{currentmarker}{}%
\end{pgfscope}%
\end{pgfscope}%
\begin{pgfscope}%
\definecolor{textcolor}{rgb}{0.150000,0.150000,0.150000}%
\pgfsetstrokecolor{textcolor}%
\pgfsetfillcolor{textcolor}%
\pgftext[x=0.547222in,y=2.563235in,right,]{\color{textcolor}\sffamily\fontsize{8.000000}{9.600000}\selectfont 1.5}%
\end{pgfscope}%
\begin{pgfscope}%
\definecolor{textcolor}{rgb}{0.150000,0.150000,0.150000}%
\pgfsetstrokecolor{textcolor}%
\pgfsetfillcolor{textcolor}%
\pgftext[x=0.228007in,y=1.537500in,,bottom,rotate=90.000000]{\color{textcolor}\sffamily\fontsize{8.800000}{10.560000}\selectfont LOO predictions}%
\end{pgfscope}%
\begin{pgfscope}%
\pgfpathrectangle{\pgfqpoint{0.625000in}{0.375000in}}{\pgfqpoint{3.875000in}{2.325000in}} %
\pgfusepath{clip}%
\pgfsetbuttcap%
\pgfsetroundjoin%
\definecolor{currentfill}{rgb}{0.000000,0.000000,0.000000}%
\pgfsetfillcolor{currentfill}%
\pgfsetlinewidth{2.007500pt}%
\definecolor{currentstroke}{rgb}{0.000000,0.000000,0.000000}%
\pgfsetstrokecolor{currentstroke}%
\pgfsetdash{}{0pt}%
\pgfsys@defobject{currentmarker}{\pgfqpoint{-0.038889in}{-0.038889in}}{\pgfqpoint{0.038889in}{0.038889in}}{%
\pgfpathmoveto{\pgfqpoint{-0.038889in}{-0.038889in}}%
\pgfpathlineto{\pgfqpoint{0.038889in}{0.038889in}}%
\pgfpathmoveto{\pgfqpoint{-0.038889in}{0.038889in}}%
\pgfpathlineto{\pgfqpoint{0.038889in}{-0.038889in}}%
\pgfusepath{stroke,fill}%
}%
\begin{pgfscope}%
\pgfsys@transformshift{3.069242in}{1.502400in}%
\pgfsys@useobject{currentmarker}{}%
\end{pgfscope}%
\begin{pgfscope}%
\pgfsys@transformshift{2.336981in}{1.585626in}%
\pgfsys@useobject{currentmarker}{}%
\end{pgfscope}%
\begin{pgfscope}%
\pgfsys@transformshift{2.260051in}{0.907780in}%
\pgfsys@useobject{currentmarker}{}%
\end{pgfscope}%
\begin{pgfscope}%
\pgfsys@transformshift{1.712992in}{1.259604in}%
\pgfsys@useobject{currentmarker}{}%
\end{pgfscope}%
\begin{pgfscope}%
\pgfsys@transformshift{1.262808in}{1.085049in}%
\pgfsys@useobject{currentmarker}{}%
\end{pgfscope}%
\begin{pgfscope}%
\pgfsys@transformshift{1.328341in}{1.114466in}%
\pgfsys@useobject{currentmarker}{}%
\end{pgfscope}%
\begin{pgfscope}%
\pgfsys@transformshift{2.077698in}{1.447027in}%
\pgfsys@useobject{currentmarker}{}%
\end{pgfscope}%
\begin{pgfscope}%
\pgfsys@transformshift{3.012256in}{1.724384in}%
\pgfsys@useobject{currentmarker}{}%
\end{pgfscope}%
\begin{pgfscope}%
\pgfsys@transformshift{3.357017in}{1.737488in}%
\pgfsys@useobject{currentmarker}{}%
\end{pgfscope}%
\begin{pgfscope}%
\pgfsys@transformshift{1.841209in}{1.618948in}%
\pgfsys@useobject{currentmarker}{}%
\end{pgfscope}%
\begin{pgfscope}%
\pgfsys@transformshift{2.234407in}{1.455929in}%
\pgfsys@useobject{currentmarker}{}%
\end{pgfscope}%
\begin{pgfscope}%
\pgfsys@transformshift{1.322642in}{1.004204in}%
\pgfsys@useobject{currentmarker}{}%
\end{pgfscope}%
\begin{pgfscope}%
\pgfsys@transformshift{3.334223in}{1.626397in}%
\pgfsys@useobject{currentmarker}{}%
\end{pgfscope}%
\begin{pgfscope}%
\pgfsys@transformshift{1.829812in}{1.296235in}%
\pgfsys@useobject{currentmarker}{}%
\end{pgfscope}%
\begin{pgfscope}%
\pgfsys@transformshift{1.553433in}{1.006016in}%
\pgfsys@useobject{currentmarker}{}%
\end{pgfscope}%
\begin{pgfscope}%
\pgfsys@transformshift{2.331282in}{1.312053in}%
\pgfsys@useobject{currentmarker}{}%
\end{pgfscope}%
\begin{pgfscope}%
\pgfsys@transformshift{3.639095in}{1.662430in}%
\pgfsys@useobject{currentmarker}{}%
\end{pgfscope}%
\begin{pgfscope}%
\pgfsys@transformshift{1.573378in}{1.225419in}%
\pgfsys@useobject{currentmarker}{}%
\end{pgfscope}%
\begin{pgfscope}%
\pgfsys@transformshift{1.504995in}{1.203369in}%
\pgfsys@useobject{currentmarker}{}%
\end{pgfscope}%
\begin{pgfscope}%
\pgfsys@transformshift{2.265749in}{1.060639in}%
\pgfsys@useobject{currentmarker}{}%
\end{pgfscope}%
\begin{pgfscope}%
\pgfsys@transformshift{1.807017in}{1.415103in}%
\pgfsys@useobject{currentmarker}{}%
\end{pgfscope}%
\begin{pgfscope}%
\pgfsys@transformshift{2.613359in}{1.830154in}%
\pgfsys@useobject{currentmarker}{}%
\end{pgfscope}%
\begin{pgfscope}%
\pgfsys@transformshift{3.126227in}{1.355796in}%
\pgfsys@useobject{currentmarker}{}%
\end{pgfscope}%
\begin{pgfscope}%
\pgfsys@transformshift{3.137624in}{2.075573in}%
\pgfsys@useobject{currentmarker}{}%
\end{pgfscope}%
\begin{pgfscope}%
\pgfsys@transformshift{2.365473in}{1.355059in}%
\pgfsys@useobject{currentmarker}{}%
\end{pgfscope}%
\begin{pgfscope}%
\pgfsys@transformshift{3.470988in}{1.785394in}%
\pgfsys@useobject{currentmarker}{}%
\end{pgfscope}%
\begin{pgfscope}%
\pgfsys@transformshift{2.596264in}{1.870701in}%
\pgfsys@useobject{currentmarker}{}%
\end{pgfscope}%
\begin{pgfscope}%
\pgfsys@transformshift{2.408212in}{1.260749in}%
\pgfsys@useobject{currentmarker}{}%
\end{pgfscope}%
\begin{pgfscope}%
\pgfsys@transformshift{2.097642in}{1.599172in}%
\pgfsys@useobject{currentmarker}{}%
\end{pgfscope}%
\begin{pgfscope}%
\pgfsys@transformshift{4.174756in}{1.779065in}%
\pgfsys@useobject{currentmarker}{}%
\end{pgfscope}%
\begin{pgfscope}%
\pgfsys@transformshift{1.878249in}{1.515681in}%
\pgfsys@useobject{currentmarker}{}%
\end{pgfscope}%
\begin{pgfscope}%
\pgfsys@transformshift{3.559315in}{1.964673in}%
\pgfsys@useobject{currentmarker}{}%
\end{pgfscope}%
\begin{pgfscope}%
\pgfsys@transformshift{2.849848in}{1.507238in}%
\pgfsys@useobject{currentmarker}{}%
\end{pgfscope}%
\begin{pgfscope}%
\pgfsys@transformshift{3.667587in}{2.181956in}%
\pgfsys@useobject{currentmarker}{}%
\end{pgfscope}%
\begin{pgfscope}%
\pgfsys@transformshift{3.616301in}{2.229460in}%
\pgfsys@useobject{currentmarker}{}%
\end{pgfscope}%
\begin{pgfscope}%
\pgfsys@transformshift{2.220161in}{1.818368in}%
\pgfsys@useobject{currentmarker}{}%
\end{pgfscope}%
\begin{pgfscope}%
\pgfsys@transformshift{3.539370in}{1.796039in}%
\pgfsys@useobject{currentmarker}{}%
\end{pgfscope}%
\begin{pgfscope}%
\pgfsys@transformshift{4.040841in}{2.073745in}%
\pgfsys@useobject{currentmarker}{}%
\end{pgfscope}%
\begin{pgfscope}%
\pgfsys@transformshift{2.311337in}{1.713222in}%
\pgfsys@useobject{currentmarker}{}%
\end{pgfscope}%
\begin{pgfscope}%
\pgfsys@transformshift{3.171815in}{1.827428in}%
\pgfsys@useobject{currentmarker}{}%
\end{pgfscope}%
\end{pgfscope}%
\begin{pgfscope}%
\pgfpathrectangle{\pgfqpoint{0.625000in}{0.375000in}}{\pgfqpoint{3.875000in}{2.325000in}} %
\pgfusepath{clip}%
\pgfsetbuttcap%
\pgfsetroundjoin%
\pgfsetlinewidth{0.752812pt}%
\definecolor{currentstroke}{rgb}{0.000000,0.000000,0.000000}%
\pgfsetstrokecolor{currentstroke}%
\pgfsetdash{{6.000000pt}{6.000000pt}}{0.000000pt}%
\pgfpathmoveto{\pgfqpoint{0.283088in}{0.169853in}}%
\pgfpathmoveto{\pgfqpoint{0.611111in}{0.366667in}}%
\pgfpathlineto{\pgfqpoint{4.513889in}{2.708333in}}%
\pgfusepath{stroke}%
\end{pgfscope}%
\begin{pgfscope}%
\pgfsetrectcap%
\pgfsetmiterjoin%
\pgfsetlinewidth{0.000000pt}%
\definecolor{currentstroke}{rgb}{1.000000,1.000000,1.000000}%
\pgfsetstrokecolor{currentstroke}%
\pgfsetdash{}{0pt}%
\pgfpathmoveto{\pgfqpoint{0.625000in}{2.700000in}}%
\pgfpathlineto{\pgfqpoint{4.500000in}{2.700000in}}%
\pgfusepath{}%
\end{pgfscope}%
\begin{pgfscope}%
\pgfsetrectcap%
\pgfsetmiterjoin%
\pgfsetlinewidth{0.000000pt}%
\definecolor{currentstroke}{rgb}{1.000000,1.000000,1.000000}%
\pgfsetstrokecolor{currentstroke}%
\pgfsetdash{}{0pt}%
\pgfpathmoveto{\pgfqpoint{4.500000in}{0.375000in}}%
\pgfpathlineto{\pgfqpoint{4.500000in}{2.700000in}}%
\pgfusepath{}%
\end{pgfscope}%
\begin{pgfscope}%
\pgfsetrectcap%
\pgfsetmiterjoin%
\pgfsetlinewidth{0.000000pt}%
\definecolor{currentstroke}{rgb}{1.000000,1.000000,1.000000}%
\pgfsetstrokecolor{currentstroke}%
\pgfsetdash{}{0pt}%
\pgfpathmoveto{\pgfqpoint{0.625000in}{0.375000in}}%
\pgfpathlineto{\pgfqpoint{4.500000in}{0.375000in}}%
\pgfusepath{}%
\end{pgfscope}%
\begin{pgfscope}%
\pgfsetrectcap%
\pgfsetmiterjoin%
\pgfsetlinewidth{0.000000pt}%
\definecolor{currentstroke}{rgb}{1.000000,1.000000,1.000000}%
\pgfsetstrokecolor{currentstroke}%
\pgfsetdash{}{0pt}%
\pgfpathmoveto{\pgfqpoint{0.625000in}{0.375000in}}%
\pgfpathlineto{\pgfqpoint{0.625000in}{2.700000in}}%
\pgfusepath{}%
\end{pgfscope}%
\end{pgfpicture}%
\makeatother%
\endgroup%

	\caption{Comparison between the two times registered for one throw.}
	\label{mloo_vs_real}
\end{figure}
