% -*- root: main.tex -*-
\section{Lab session 3}

In the previous section several designs of experiments (Doe) were implemented and analyzed in order to determine which was the best given some quality measures. Thus, for this case a latin hypercube design was used to generate 40 experiments which corresponds to one helicopter. Each helicopter was thrown 2 times from 5 to 6 meters height, 2 falling times for each throw were properly registered in seconds by different subjects.

\subsection{Analyzing Data}

The collected data was further analyzed in order to determine the observation noise, and other correlations of the variables used by the model. Inititally the 4 falling times were plotted, in fig \ref{fig_EX1_EX1} and \ref{fig_EX1_EX2} are illustrated two scatter plots wich represents the measurements from two observers for the same throw and from the same observer for different throws. Between different observers there is not a lot of noise as in the measurement for different throws. despite of that fact, in a first setting a Gaussian Process model was fitted using only two falling times, the average time of the first throw and the average time of the second throw.

\paragraph{}
In that setting the leave one out function had to be modified in order to retrieve the two observations for a same set of parameters or helicopter, then a $Q2$ quality measure was used to measure the prediction power of the proposed model, however the $Q2$ was very low, aproximately 0.1, the recommended value to state that the model was correct was at least 0.5, the model is better when the $Q2$ is close to one.

\paragraph{}
One of the assumptions for the $Q2$ to not be as high as expected was that the gathered data was very noisy and didn't expose the correlations between some of the variables and the falling time, according to some prior assumptions made, for example, that the wing angle an the falling time were linear increasingly correlated, the wing length to tail ratio was also expected to have a linear correlation with falling time. In fig ure
\ref{fig_wtr_vs_obs2} the linear correlation between the ratio and the falling time is not as clear as in the figure \ref{fig_wtr_vs_avg4}. While in the first figure only the measurment of one observer was ploted in the second figure the average of all four measurements was taken, exposing a better linear correlation, as expected.



\begin{figure}
	\begin{subfigure}[h]{.5\linewidth}
		%% Creator: Matplotlib, PGF backend
%%
%% To include the figure in your LaTeX document, write
%%   \input{<filename>.pgf}
%%
%% Make sure the required packages are loaded in your preamble
%%   \usepackage{pgf}
%%
%% Figures using additional raster images can only be included by \input if
%% they are in the same directory as the main LaTeX file. For loading figures
%% from other directories you can use the `import` package
%%   \usepackage{import}
%% and then include the figures with
%%   \import{<path to file>}{<filename>.pgf}
%%
%% Matplotlib used the following preamble
%%   \usepackage{fontspec}
%%   \setmainfont{Times New Roman}
%%   \setsansfont{Arial}
%%   \setmonofont{Andale Mono}
%%
\begingroup%
\makeatletter%
\begin{pgfpicture}%
\pgfpathrectangle{\pgfpointorigin}{\pgfqpoint{2.500000in}{2.500000in}}%
\pgfusepath{use as bounding box, clip}%
\begin{pgfscope}%
\pgfsetbuttcap%
\pgfsetmiterjoin%
\definecolor{currentfill}{rgb}{1.000000,1.000000,1.000000}%
\pgfsetfillcolor{currentfill}%
\pgfsetlinewidth{0.000000pt}%
\definecolor{currentstroke}{rgb}{1.000000,1.000000,1.000000}%
\pgfsetstrokecolor{currentstroke}%
\pgfsetdash{}{0pt}%
\pgfpathmoveto{\pgfqpoint{0.000000in}{0.000000in}}%
\pgfpathlineto{\pgfqpoint{2.500000in}{0.000000in}}%
\pgfpathlineto{\pgfqpoint{2.500000in}{2.500000in}}%
\pgfpathlineto{\pgfqpoint{0.000000in}{2.500000in}}%
\pgfpathclose%
\pgfusepath{fill}%
\end{pgfscope}%
\begin{pgfscope}%
\pgfsetbuttcap%
\pgfsetmiterjoin%
\definecolor{currentfill}{rgb}{0.917647,0.917647,0.949020}%
\pgfsetfillcolor{currentfill}%
\pgfsetlinewidth{0.000000pt}%
\definecolor{currentstroke}{rgb}{0.000000,0.000000,0.000000}%
\pgfsetstrokecolor{currentstroke}%
\pgfsetstrokeopacity{0.000000}%
\pgfsetdash{}{0pt}%
\pgfpathmoveto{\pgfqpoint{0.624479in}{0.566667in}}%
\pgfpathlineto{\pgfqpoint{2.262793in}{0.566667in}}%
\pgfpathlineto{\pgfqpoint{2.262793in}{2.287500in}}%
\pgfpathlineto{\pgfqpoint{0.624479in}{2.287500in}}%
\pgfpathclose%
\pgfusepath{fill}%
\end{pgfscope}%
\begin{pgfscope}%
\pgfpathrectangle{\pgfqpoint{0.624479in}{0.566667in}}{\pgfqpoint{1.638314in}{1.720833in}} %
\pgfusepath{clip}%
\pgfsetroundcap%
\pgfsetroundjoin%
\pgfsetlinewidth{1.003750pt}%
\definecolor{currentstroke}{rgb}{1.000000,1.000000,1.000000}%
\pgfsetstrokecolor{currentstroke}%
\pgfsetdash{}{0pt}%
\pgfpathmoveto{\pgfqpoint{0.624479in}{0.566667in}}%
\pgfpathlineto{\pgfqpoint{0.624479in}{2.287500in}}%
\pgfusepath{stroke}%
\end{pgfscope}%
\begin{pgfscope}%
\pgfsetbuttcap%
\pgfsetroundjoin%
\definecolor{currentfill}{rgb}{0.150000,0.150000,0.150000}%
\pgfsetfillcolor{currentfill}%
\pgfsetlinewidth{1.003750pt}%
\definecolor{currentstroke}{rgb}{0.150000,0.150000,0.150000}%
\pgfsetstrokecolor{currentstroke}%
\pgfsetdash{}{0pt}%
\pgfsys@defobject{currentmarker}{\pgfqpoint{0.000000in}{0.000000in}}{\pgfqpoint{0.000000in}{0.000000in}}{%
\pgfpathmoveto{\pgfqpoint{0.000000in}{0.000000in}}%
\pgfpathlineto{\pgfqpoint{0.000000in}{0.000000in}}%
\pgfusepath{stroke,fill}%
}%
\begin{pgfscope}%
\pgfsys@transformshift{0.624479in}{0.566667in}%
\pgfsys@useobject{currentmarker}{}%
\end{pgfscope}%
\end{pgfscope}%
\begin{pgfscope}%
\definecolor{textcolor}{rgb}{0.150000,0.150000,0.150000}%
\pgfsetstrokecolor{textcolor}%
\pgfsetfillcolor{textcolor}%
\pgftext[x=0.624479in,y=0.469444in,,top]{\color{textcolor}\sffamily\fontsize{10.000000}{12.000000}\selectfont 2.0}%
\end{pgfscope}%
\begin{pgfscope}%
\pgfpathrectangle{\pgfqpoint{0.624479in}{0.566667in}}{\pgfqpoint{1.638314in}{1.720833in}} %
\pgfusepath{clip}%
\pgfsetroundcap%
\pgfsetroundjoin%
\pgfsetlinewidth{1.003750pt}%
\definecolor{currentstroke}{rgb}{1.000000,1.000000,1.000000}%
\pgfsetstrokecolor{currentstroke}%
\pgfsetdash{}{0pt}%
\pgfpathmoveto{\pgfqpoint{0.806514in}{0.566667in}}%
\pgfpathlineto{\pgfqpoint{0.806514in}{2.287500in}}%
\pgfusepath{stroke}%
\end{pgfscope}%
\begin{pgfscope}%
\pgfsetbuttcap%
\pgfsetroundjoin%
\definecolor{currentfill}{rgb}{0.150000,0.150000,0.150000}%
\pgfsetfillcolor{currentfill}%
\pgfsetlinewidth{1.003750pt}%
\definecolor{currentstroke}{rgb}{0.150000,0.150000,0.150000}%
\pgfsetstrokecolor{currentstroke}%
\pgfsetdash{}{0pt}%
\pgfsys@defobject{currentmarker}{\pgfqpoint{0.000000in}{0.000000in}}{\pgfqpoint{0.000000in}{0.000000in}}{%
\pgfpathmoveto{\pgfqpoint{0.000000in}{0.000000in}}%
\pgfpathlineto{\pgfqpoint{0.000000in}{0.000000in}}%
\pgfusepath{stroke,fill}%
}%
\begin{pgfscope}%
\pgfsys@transformshift{0.806514in}{0.566667in}%
\pgfsys@useobject{currentmarker}{}%
\end{pgfscope}%
\end{pgfscope}%
\begin{pgfscope}%
\definecolor{textcolor}{rgb}{0.150000,0.150000,0.150000}%
\pgfsetstrokecolor{textcolor}%
\pgfsetfillcolor{textcolor}%
\pgftext[x=0.806514in,y=0.469444in,,top]{\color{textcolor}\sffamily\fontsize{10.000000}{12.000000}\selectfont 2.5}%
\end{pgfscope}%
\begin{pgfscope}%
\pgfpathrectangle{\pgfqpoint{0.624479in}{0.566667in}}{\pgfqpoint{1.638314in}{1.720833in}} %
\pgfusepath{clip}%
\pgfsetroundcap%
\pgfsetroundjoin%
\pgfsetlinewidth{1.003750pt}%
\definecolor{currentstroke}{rgb}{1.000000,1.000000,1.000000}%
\pgfsetstrokecolor{currentstroke}%
\pgfsetdash{}{0pt}%
\pgfpathmoveto{\pgfqpoint{0.988549in}{0.566667in}}%
\pgfpathlineto{\pgfqpoint{0.988549in}{2.287500in}}%
\pgfusepath{stroke}%
\end{pgfscope}%
\begin{pgfscope}%
\pgfsetbuttcap%
\pgfsetroundjoin%
\definecolor{currentfill}{rgb}{0.150000,0.150000,0.150000}%
\pgfsetfillcolor{currentfill}%
\pgfsetlinewidth{1.003750pt}%
\definecolor{currentstroke}{rgb}{0.150000,0.150000,0.150000}%
\pgfsetstrokecolor{currentstroke}%
\pgfsetdash{}{0pt}%
\pgfsys@defobject{currentmarker}{\pgfqpoint{0.000000in}{0.000000in}}{\pgfqpoint{0.000000in}{0.000000in}}{%
\pgfpathmoveto{\pgfqpoint{0.000000in}{0.000000in}}%
\pgfpathlineto{\pgfqpoint{0.000000in}{0.000000in}}%
\pgfusepath{stroke,fill}%
}%
\begin{pgfscope}%
\pgfsys@transformshift{0.988549in}{0.566667in}%
\pgfsys@useobject{currentmarker}{}%
\end{pgfscope}%
\end{pgfscope}%
\begin{pgfscope}%
\definecolor{textcolor}{rgb}{0.150000,0.150000,0.150000}%
\pgfsetstrokecolor{textcolor}%
\pgfsetfillcolor{textcolor}%
\pgftext[x=0.988549in,y=0.469444in,,top]{\color{textcolor}\sffamily\fontsize{10.000000}{12.000000}\selectfont 3.0}%
\end{pgfscope}%
\begin{pgfscope}%
\pgfpathrectangle{\pgfqpoint{0.624479in}{0.566667in}}{\pgfqpoint{1.638314in}{1.720833in}} %
\pgfusepath{clip}%
\pgfsetroundcap%
\pgfsetroundjoin%
\pgfsetlinewidth{1.003750pt}%
\definecolor{currentstroke}{rgb}{1.000000,1.000000,1.000000}%
\pgfsetstrokecolor{currentstroke}%
\pgfsetdash{}{0pt}%
\pgfpathmoveto{\pgfqpoint{1.170584in}{0.566667in}}%
\pgfpathlineto{\pgfqpoint{1.170584in}{2.287500in}}%
\pgfusepath{stroke}%
\end{pgfscope}%
\begin{pgfscope}%
\pgfsetbuttcap%
\pgfsetroundjoin%
\definecolor{currentfill}{rgb}{0.150000,0.150000,0.150000}%
\pgfsetfillcolor{currentfill}%
\pgfsetlinewidth{1.003750pt}%
\definecolor{currentstroke}{rgb}{0.150000,0.150000,0.150000}%
\pgfsetstrokecolor{currentstroke}%
\pgfsetdash{}{0pt}%
\pgfsys@defobject{currentmarker}{\pgfqpoint{0.000000in}{0.000000in}}{\pgfqpoint{0.000000in}{0.000000in}}{%
\pgfpathmoveto{\pgfqpoint{0.000000in}{0.000000in}}%
\pgfpathlineto{\pgfqpoint{0.000000in}{0.000000in}}%
\pgfusepath{stroke,fill}%
}%
\begin{pgfscope}%
\pgfsys@transformshift{1.170584in}{0.566667in}%
\pgfsys@useobject{currentmarker}{}%
\end{pgfscope}%
\end{pgfscope}%
\begin{pgfscope}%
\definecolor{textcolor}{rgb}{0.150000,0.150000,0.150000}%
\pgfsetstrokecolor{textcolor}%
\pgfsetfillcolor{textcolor}%
\pgftext[x=1.170584in,y=0.469444in,,top]{\color{textcolor}\sffamily\fontsize{10.000000}{12.000000}\selectfont 3.5}%
\end{pgfscope}%
\begin{pgfscope}%
\pgfpathrectangle{\pgfqpoint{0.624479in}{0.566667in}}{\pgfqpoint{1.638314in}{1.720833in}} %
\pgfusepath{clip}%
\pgfsetroundcap%
\pgfsetroundjoin%
\pgfsetlinewidth{1.003750pt}%
\definecolor{currentstroke}{rgb}{1.000000,1.000000,1.000000}%
\pgfsetstrokecolor{currentstroke}%
\pgfsetdash{}{0pt}%
\pgfpathmoveto{\pgfqpoint{1.352619in}{0.566667in}}%
\pgfpathlineto{\pgfqpoint{1.352619in}{2.287500in}}%
\pgfusepath{stroke}%
\end{pgfscope}%
\begin{pgfscope}%
\pgfsetbuttcap%
\pgfsetroundjoin%
\definecolor{currentfill}{rgb}{0.150000,0.150000,0.150000}%
\pgfsetfillcolor{currentfill}%
\pgfsetlinewidth{1.003750pt}%
\definecolor{currentstroke}{rgb}{0.150000,0.150000,0.150000}%
\pgfsetstrokecolor{currentstroke}%
\pgfsetdash{}{0pt}%
\pgfsys@defobject{currentmarker}{\pgfqpoint{0.000000in}{0.000000in}}{\pgfqpoint{0.000000in}{0.000000in}}{%
\pgfpathmoveto{\pgfqpoint{0.000000in}{0.000000in}}%
\pgfpathlineto{\pgfqpoint{0.000000in}{0.000000in}}%
\pgfusepath{stroke,fill}%
}%
\begin{pgfscope}%
\pgfsys@transformshift{1.352619in}{0.566667in}%
\pgfsys@useobject{currentmarker}{}%
\end{pgfscope}%
\end{pgfscope}%
\begin{pgfscope}%
\definecolor{textcolor}{rgb}{0.150000,0.150000,0.150000}%
\pgfsetstrokecolor{textcolor}%
\pgfsetfillcolor{textcolor}%
\pgftext[x=1.352619in,y=0.469444in,,top]{\color{textcolor}\sffamily\fontsize{10.000000}{12.000000}\selectfont 4.0}%
\end{pgfscope}%
\begin{pgfscope}%
\pgfpathrectangle{\pgfqpoint{0.624479in}{0.566667in}}{\pgfqpoint{1.638314in}{1.720833in}} %
\pgfusepath{clip}%
\pgfsetroundcap%
\pgfsetroundjoin%
\pgfsetlinewidth{1.003750pt}%
\definecolor{currentstroke}{rgb}{1.000000,1.000000,1.000000}%
\pgfsetstrokecolor{currentstroke}%
\pgfsetdash{}{0pt}%
\pgfpathmoveto{\pgfqpoint{1.534654in}{0.566667in}}%
\pgfpathlineto{\pgfqpoint{1.534654in}{2.287500in}}%
\pgfusepath{stroke}%
\end{pgfscope}%
\begin{pgfscope}%
\pgfsetbuttcap%
\pgfsetroundjoin%
\definecolor{currentfill}{rgb}{0.150000,0.150000,0.150000}%
\pgfsetfillcolor{currentfill}%
\pgfsetlinewidth{1.003750pt}%
\definecolor{currentstroke}{rgb}{0.150000,0.150000,0.150000}%
\pgfsetstrokecolor{currentstroke}%
\pgfsetdash{}{0pt}%
\pgfsys@defobject{currentmarker}{\pgfqpoint{0.000000in}{0.000000in}}{\pgfqpoint{0.000000in}{0.000000in}}{%
\pgfpathmoveto{\pgfqpoint{0.000000in}{0.000000in}}%
\pgfpathlineto{\pgfqpoint{0.000000in}{0.000000in}}%
\pgfusepath{stroke,fill}%
}%
\begin{pgfscope}%
\pgfsys@transformshift{1.534654in}{0.566667in}%
\pgfsys@useobject{currentmarker}{}%
\end{pgfscope}%
\end{pgfscope}%
\begin{pgfscope}%
\definecolor{textcolor}{rgb}{0.150000,0.150000,0.150000}%
\pgfsetstrokecolor{textcolor}%
\pgfsetfillcolor{textcolor}%
\pgftext[x=1.534654in,y=0.469444in,,top]{\color{textcolor}\sffamily\fontsize{10.000000}{12.000000}\selectfont 4.5}%
\end{pgfscope}%
\begin{pgfscope}%
\pgfpathrectangle{\pgfqpoint{0.624479in}{0.566667in}}{\pgfqpoint{1.638314in}{1.720833in}} %
\pgfusepath{clip}%
\pgfsetroundcap%
\pgfsetroundjoin%
\pgfsetlinewidth{1.003750pt}%
\definecolor{currentstroke}{rgb}{1.000000,1.000000,1.000000}%
\pgfsetstrokecolor{currentstroke}%
\pgfsetdash{}{0pt}%
\pgfpathmoveto{\pgfqpoint{1.716688in}{0.566667in}}%
\pgfpathlineto{\pgfqpoint{1.716688in}{2.287500in}}%
\pgfusepath{stroke}%
\end{pgfscope}%
\begin{pgfscope}%
\pgfsetbuttcap%
\pgfsetroundjoin%
\definecolor{currentfill}{rgb}{0.150000,0.150000,0.150000}%
\pgfsetfillcolor{currentfill}%
\pgfsetlinewidth{1.003750pt}%
\definecolor{currentstroke}{rgb}{0.150000,0.150000,0.150000}%
\pgfsetstrokecolor{currentstroke}%
\pgfsetdash{}{0pt}%
\pgfsys@defobject{currentmarker}{\pgfqpoint{0.000000in}{0.000000in}}{\pgfqpoint{0.000000in}{0.000000in}}{%
\pgfpathmoveto{\pgfqpoint{0.000000in}{0.000000in}}%
\pgfpathlineto{\pgfqpoint{0.000000in}{0.000000in}}%
\pgfusepath{stroke,fill}%
}%
\begin{pgfscope}%
\pgfsys@transformshift{1.716688in}{0.566667in}%
\pgfsys@useobject{currentmarker}{}%
\end{pgfscope}%
\end{pgfscope}%
\begin{pgfscope}%
\definecolor{textcolor}{rgb}{0.150000,0.150000,0.150000}%
\pgfsetstrokecolor{textcolor}%
\pgfsetfillcolor{textcolor}%
\pgftext[x=1.716688in,y=0.469444in,,top]{\color{textcolor}\sffamily\fontsize{10.000000}{12.000000}\selectfont 5.0}%
\end{pgfscope}%
\begin{pgfscope}%
\pgfpathrectangle{\pgfqpoint{0.624479in}{0.566667in}}{\pgfqpoint{1.638314in}{1.720833in}} %
\pgfusepath{clip}%
\pgfsetroundcap%
\pgfsetroundjoin%
\pgfsetlinewidth{1.003750pt}%
\definecolor{currentstroke}{rgb}{1.000000,1.000000,1.000000}%
\pgfsetstrokecolor{currentstroke}%
\pgfsetdash{}{0pt}%
\pgfpathmoveto{\pgfqpoint{1.898723in}{0.566667in}}%
\pgfpathlineto{\pgfqpoint{1.898723in}{2.287500in}}%
\pgfusepath{stroke}%
\end{pgfscope}%
\begin{pgfscope}%
\pgfsetbuttcap%
\pgfsetroundjoin%
\definecolor{currentfill}{rgb}{0.150000,0.150000,0.150000}%
\pgfsetfillcolor{currentfill}%
\pgfsetlinewidth{1.003750pt}%
\definecolor{currentstroke}{rgb}{0.150000,0.150000,0.150000}%
\pgfsetstrokecolor{currentstroke}%
\pgfsetdash{}{0pt}%
\pgfsys@defobject{currentmarker}{\pgfqpoint{0.000000in}{0.000000in}}{\pgfqpoint{0.000000in}{0.000000in}}{%
\pgfpathmoveto{\pgfqpoint{0.000000in}{0.000000in}}%
\pgfpathlineto{\pgfqpoint{0.000000in}{0.000000in}}%
\pgfusepath{stroke,fill}%
}%
\begin{pgfscope}%
\pgfsys@transformshift{1.898723in}{0.566667in}%
\pgfsys@useobject{currentmarker}{}%
\end{pgfscope}%
\end{pgfscope}%
\begin{pgfscope}%
\definecolor{textcolor}{rgb}{0.150000,0.150000,0.150000}%
\pgfsetstrokecolor{textcolor}%
\pgfsetfillcolor{textcolor}%
\pgftext[x=1.898723in,y=0.469444in,,top]{\color{textcolor}\sffamily\fontsize{10.000000}{12.000000}\selectfont 5.5}%
\end{pgfscope}%
\begin{pgfscope}%
\pgfpathrectangle{\pgfqpoint{0.624479in}{0.566667in}}{\pgfqpoint{1.638314in}{1.720833in}} %
\pgfusepath{clip}%
\pgfsetroundcap%
\pgfsetroundjoin%
\pgfsetlinewidth{1.003750pt}%
\definecolor{currentstroke}{rgb}{1.000000,1.000000,1.000000}%
\pgfsetstrokecolor{currentstroke}%
\pgfsetdash{}{0pt}%
\pgfpathmoveto{\pgfqpoint{2.080758in}{0.566667in}}%
\pgfpathlineto{\pgfqpoint{2.080758in}{2.287500in}}%
\pgfusepath{stroke}%
\end{pgfscope}%
\begin{pgfscope}%
\pgfsetbuttcap%
\pgfsetroundjoin%
\definecolor{currentfill}{rgb}{0.150000,0.150000,0.150000}%
\pgfsetfillcolor{currentfill}%
\pgfsetlinewidth{1.003750pt}%
\definecolor{currentstroke}{rgb}{0.150000,0.150000,0.150000}%
\pgfsetstrokecolor{currentstroke}%
\pgfsetdash{}{0pt}%
\pgfsys@defobject{currentmarker}{\pgfqpoint{0.000000in}{0.000000in}}{\pgfqpoint{0.000000in}{0.000000in}}{%
\pgfpathmoveto{\pgfqpoint{0.000000in}{0.000000in}}%
\pgfpathlineto{\pgfqpoint{0.000000in}{0.000000in}}%
\pgfusepath{stroke,fill}%
}%
\begin{pgfscope}%
\pgfsys@transformshift{2.080758in}{0.566667in}%
\pgfsys@useobject{currentmarker}{}%
\end{pgfscope}%
\end{pgfscope}%
\begin{pgfscope}%
\definecolor{textcolor}{rgb}{0.150000,0.150000,0.150000}%
\pgfsetstrokecolor{textcolor}%
\pgfsetfillcolor{textcolor}%
\pgftext[x=2.080758in,y=0.469444in,,top]{\color{textcolor}\sffamily\fontsize{10.000000}{12.000000}\selectfont 6.0}%
\end{pgfscope}%
\begin{pgfscope}%
\pgfpathrectangle{\pgfqpoint{0.624479in}{0.566667in}}{\pgfqpoint{1.638314in}{1.720833in}} %
\pgfusepath{clip}%
\pgfsetroundcap%
\pgfsetroundjoin%
\pgfsetlinewidth{1.003750pt}%
\definecolor{currentstroke}{rgb}{1.000000,1.000000,1.000000}%
\pgfsetstrokecolor{currentstroke}%
\pgfsetdash{}{0pt}%
\pgfpathmoveto{\pgfqpoint{2.262793in}{0.566667in}}%
\pgfpathlineto{\pgfqpoint{2.262793in}{2.287500in}}%
\pgfusepath{stroke}%
\end{pgfscope}%
\begin{pgfscope}%
\pgfsetbuttcap%
\pgfsetroundjoin%
\definecolor{currentfill}{rgb}{0.150000,0.150000,0.150000}%
\pgfsetfillcolor{currentfill}%
\pgfsetlinewidth{1.003750pt}%
\definecolor{currentstroke}{rgb}{0.150000,0.150000,0.150000}%
\pgfsetstrokecolor{currentstroke}%
\pgfsetdash{}{0pt}%
\pgfsys@defobject{currentmarker}{\pgfqpoint{0.000000in}{0.000000in}}{\pgfqpoint{0.000000in}{0.000000in}}{%
\pgfpathmoveto{\pgfqpoint{0.000000in}{0.000000in}}%
\pgfpathlineto{\pgfqpoint{0.000000in}{0.000000in}}%
\pgfusepath{stroke,fill}%
}%
\begin{pgfscope}%
\pgfsys@transformshift{2.262793in}{0.566667in}%
\pgfsys@useobject{currentmarker}{}%
\end{pgfscope}%
\end{pgfscope}%
\begin{pgfscope}%
\definecolor{textcolor}{rgb}{0.150000,0.150000,0.150000}%
\pgfsetstrokecolor{textcolor}%
\pgfsetfillcolor{textcolor}%
\pgftext[x=2.262793in,y=0.469444in,,top]{\color{textcolor}\sffamily\fontsize{10.000000}{12.000000}\selectfont 6.5}%
\end{pgfscope}%
\begin{pgfscope}%
\definecolor{textcolor}{rgb}{0.150000,0.150000,0.150000}%
\pgfsetstrokecolor{textcolor}%
\pgfsetfillcolor{textcolor}%
\pgftext[x=1.443636in,y=0.272979in,,top]{\color{textcolor}\sffamily\fontsize{11.000000}{13.200000}\selectfont Falling time realization 1 obs 1}%
\end{pgfscope}%
\begin{pgfscope}%
\pgfpathrectangle{\pgfqpoint{0.624479in}{0.566667in}}{\pgfqpoint{1.638314in}{1.720833in}} %
\pgfusepath{clip}%
\pgfsetroundcap%
\pgfsetroundjoin%
\pgfsetlinewidth{1.003750pt}%
\definecolor{currentstroke}{rgb}{1.000000,1.000000,1.000000}%
\pgfsetstrokecolor{currentstroke}%
\pgfsetdash{}{0pt}%
\pgfpathmoveto{\pgfqpoint{0.624479in}{0.566667in}}%
\pgfpathlineto{\pgfqpoint{2.262793in}{0.566667in}}%
\pgfusepath{stroke}%
\end{pgfscope}%
\begin{pgfscope}%
\pgfsetbuttcap%
\pgfsetroundjoin%
\definecolor{currentfill}{rgb}{0.150000,0.150000,0.150000}%
\pgfsetfillcolor{currentfill}%
\pgfsetlinewidth{1.003750pt}%
\definecolor{currentstroke}{rgb}{0.150000,0.150000,0.150000}%
\pgfsetstrokecolor{currentstroke}%
\pgfsetdash{}{0pt}%
\pgfsys@defobject{currentmarker}{\pgfqpoint{0.000000in}{0.000000in}}{\pgfqpoint{0.000000in}{0.000000in}}{%
\pgfpathmoveto{\pgfqpoint{0.000000in}{0.000000in}}%
\pgfpathlineto{\pgfqpoint{0.000000in}{0.000000in}}%
\pgfusepath{stroke,fill}%
}%
\begin{pgfscope}%
\pgfsys@transformshift{0.624479in}{0.566667in}%
\pgfsys@useobject{currentmarker}{}%
\end{pgfscope}%
\end{pgfscope}%
\begin{pgfscope}%
\definecolor{textcolor}{rgb}{0.150000,0.150000,0.150000}%
\pgfsetstrokecolor{textcolor}%
\pgfsetfillcolor{textcolor}%
\pgftext[x=0.527257in,y=0.566667in,right,]{\color{textcolor}\sffamily\fontsize{10.000000}{12.000000}\selectfont 2.5}%
\end{pgfscope}%
\begin{pgfscope}%
\pgfpathrectangle{\pgfqpoint{0.624479in}{0.566667in}}{\pgfqpoint{1.638314in}{1.720833in}} %
\pgfusepath{clip}%
\pgfsetroundcap%
\pgfsetroundjoin%
\pgfsetlinewidth{1.003750pt}%
\definecolor{currentstroke}{rgb}{1.000000,1.000000,1.000000}%
\pgfsetstrokecolor{currentstroke}%
\pgfsetdash{}{0pt}%
\pgfpathmoveto{\pgfqpoint{0.624479in}{0.781771in}}%
\pgfpathlineto{\pgfqpoint{2.262793in}{0.781771in}}%
\pgfusepath{stroke}%
\end{pgfscope}%
\begin{pgfscope}%
\pgfsetbuttcap%
\pgfsetroundjoin%
\definecolor{currentfill}{rgb}{0.150000,0.150000,0.150000}%
\pgfsetfillcolor{currentfill}%
\pgfsetlinewidth{1.003750pt}%
\definecolor{currentstroke}{rgb}{0.150000,0.150000,0.150000}%
\pgfsetstrokecolor{currentstroke}%
\pgfsetdash{}{0pt}%
\pgfsys@defobject{currentmarker}{\pgfqpoint{0.000000in}{0.000000in}}{\pgfqpoint{0.000000in}{0.000000in}}{%
\pgfpathmoveto{\pgfqpoint{0.000000in}{0.000000in}}%
\pgfpathlineto{\pgfqpoint{0.000000in}{0.000000in}}%
\pgfusepath{stroke,fill}%
}%
\begin{pgfscope}%
\pgfsys@transformshift{0.624479in}{0.781771in}%
\pgfsys@useobject{currentmarker}{}%
\end{pgfscope}%
\end{pgfscope}%
\begin{pgfscope}%
\definecolor{textcolor}{rgb}{0.150000,0.150000,0.150000}%
\pgfsetstrokecolor{textcolor}%
\pgfsetfillcolor{textcolor}%
\pgftext[x=0.527257in,y=0.781771in,right,]{\color{textcolor}\sffamily\fontsize{10.000000}{12.000000}\selectfont 3.0}%
\end{pgfscope}%
\begin{pgfscope}%
\pgfpathrectangle{\pgfqpoint{0.624479in}{0.566667in}}{\pgfqpoint{1.638314in}{1.720833in}} %
\pgfusepath{clip}%
\pgfsetroundcap%
\pgfsetroundjoin%
\pgfsetlinewidth{1.003750pt}%
\definecolor{currentstroke}{rgb}{1.000000,1.000000,1.000000}%
\pgfsetstrokecolor{currentstroke}%
\pgfsetdash{}{0pt}%
\pgfpathmoveto{\pgfqpoint{0.624479in}{0.996875in}}%
\pgfpathlineto{\pgfqpoint{2.262793in}{0.996875in}}%
\pgfusepath{stroke}%
\end{pgfscope}%
\begin{pgfscope}%
\pgfsetbuttcap%
\pgfsetroundjoin%
\definecolor{currentfill}{rgb}{0.150000,0.150000,0.150000}%
\pgfsetfillcolor{currentfill}%
\pgfsetlinewidth{1.003750pt}%
\definecolor{currentstroke}{rgb}{0.150000,0.150000,0.150000}%
\pgfsetstrokecolor{currentstroke}%
\pgfsetdash{}{0pt}%
\pgfsys@defobject{currentmarker}{\pgfqpoint{0.000000in}{0.000000in}}{\pgfqpoint{0.000000in}{0.000000in}}{%
\pgfpathmoveto{\pgfqpoint{0.000000in}{0.000000in}}%
\pgfpathlineto{\pgfqpoint{0.000000in}{0.000000in}}%
\pgfusepath{stroke,fill}%
}%
\begin{pgfscope}%
\pgfsys@transformshift{0.624479in}{0.996875in}%
\pgfsys@useobject{currentmarker}{}%
\end{pgfscope}%
\end{pgfscope}%
\begin{pgfscope}%
\definecolor{textcolor}{rgb}{0.150000,0.150000,0.150000}%
\pgfsetstrokecolor{textcolor}%
\pgfsetfillcolor{textcolor}%
\pgftext[x=0.527257in,y=0.996875in,right,]{\color{textcolor}\sffamily\fontsize{10.000000}{12.000000}\selectfont 3.5}%
\end{pgfscope}%
\begin{pgfscope}%
\pgfpathrectangle{\pgfqpoint{0.624479in}{0.566667in}}{\pgfqpoint{1.638314in}{1.720833in}} %
\pgfusepath{clip}%
\pgfsetroundcap%
\pgfsetroundjoin%
\pgfsetlinewidth{1.003750pt}%
\definecolor{currentstroke}{rgb}{1.000000,1.000000,1.000000}%
\pgfsetstrokecolor{currentstroke}%
\pgfsetdash{}{0pt}%
\pgfpathmoveto{\pgfqpoint{0.624479in}{1.211979in}}%
\pgfpathlineto{\pgfqpoint{2.262793in}{1.211979in}}%
\pgfusepath{stroke}%
\end{pgfscope}%
\begin{pgfscope}%
\pgfsetbuttcap%
\pgfsetroundjoin%
\definecolor{currentfill}{rgb}{0.150000,0.150000,0.150000}%
\pgfsetfillcolor{currentfill}%
\pgfsetlinewidth{1.003750pt}%
\definecolor{currentstroke}{rgb}{0.150000,0.150000,0.150000}%
\pgfsetstrokecolor{currentstroke}%
\pgfsetdash{}{0pt}%
\pgfsys@defobject{currentmarker}{\pgfqpoint{0.000000in}{0.000000in}}{\pgfqpoint{0.000000in}{0.000000in}}{%
\pgfpathmoveto{\pgfqpoint{0.000000in}{0.000000in}}%
\pgfpathlineto{\pgfqpoint{0.000000in}{0.000000in}}%
\pgfusepath{stroke,fill}%
}%
\begin{pgfscope}%
\pgfsys@transformshift{0.624479in}{1.211979in}%
\pgfsys@useobject{currentmarker}{}%
\end{pgfscope}%
\end{pgfscope}%
\begin{pgfscope}%
\definecolor{textcolor}{rgb}{0.150000,0.150000,0.150000}%
\pgfsetstrokecolor{textcolor}%
\pgfsetfillcolor{textcolor}%
\pgftext[x=0.527257in,y=1.211979in,right,]{\color{textcolor}\sffamily\fontsize{10.000000}{12.000000}\selectfont 4.0}%
\end{pgfscope}%
\begin{pgfscope}%
\pgfpathrectangle{\pgfqpoint{0.624479in}{0.566667in}}{\pgfqpoint{1.638314in}{1.720833in}} %
\pgfusepath{clip}%
\pgfsetroundcap%
\pgfsetroundjoin%
\pgfsetlinewidth{1.003750pt}%
\definecolor{currentstroke}{rgb}{1.000000,1.000000,1.000000}%
\pgfsetstrokecolor{currentstroke}%
\pgfsetdash{}{0pt}%
\pgfpathmoveto{\pgfqpoint{0.624479in}{1.427083in}}%
\pgfpathlineto{\pgfqpoint{2.262793in}{1.427083in}}%
\pgfusepath{stroke}%
\end{pgfscope}%
\begin{pgfscope}%
\pgfsetbuttcap%
\pgfsetroundjoin%
\definecolor{currentfill}{rgb}{0.150000,0.150000,0.150000}%
\pgfsetfillcolor{currentfill}%
\pgfsetlinewidth{1.003750pt}%
\definecolor{currentstroke}{rgb}{0.150000,0.150000,0.150000}%
\pgfsetstrokecolor{currentstroke}%
\pgfsetdash{}{0pt}%
\pgfsys@defobject{currentmarker}{\pgfqpoint{0.000000in}{0.000000in}}{\pgfqpoint{0.000000in}{0.000000in}}{%
\pgfpathmoveto{\pgfqpoint{0.000000in}{0.000000in}}%
\pgfpathlineto{\pgfqpoint{0.000000in}{0.000000in}}%
\pgfusepath{stroke,fill}%
}%
\begin{pgfscope}%
\pgfsys@transformshift{0.624479in}{1.427083in}%
\pgfsys@useobject{currentmarker}{}%
\end{pgfscope}%
\end{pgfscope}%
\begin{pgfscope}%
\definecolor{textcolor}{rgb}{0.150000,0.150000,0.150000}%
\pgfsetstrokecolor{textcolor}%
\pgfsetfillcolor{textcolor}%
\pgftext[x=0.527257in,y=1.427083in,right,]{\color{textcolor}\sffamily\fontsize{10.000000}{12.000000}\selectfont 4.5}%
\end{pgfscope}%
\begin{pgfscope}%
\pgfpathrectangle{\pgfqpoint{0.624479in}{0.566667in}}{\pgfqpoint{1.638314in}{1.720833in}} %
\pgfusepath{clip}%
\pgfsetroundcap%
\pgfsetroundjoin%
\pgfsetlinewidth{1.003750pt}%
\definecolor{currentstroke}{rgb}{1.000000,1.000000,1.000000}%
\pgfsetstrokecolor{currentstroke}%
\pgfsetdash{}{0pt}%
\pgfpathmoveto{\pgfqpoint{0.624479in}{1.642187in}}%
\pgfpathlineto{\pgfqpoint{2.262793in}{1.642187in}}%
\pgfusepath{stroke}%
\end{pgfscope}%
\begin{pgfscope}%
\pgfsetbuttcap%
\pgfsetroundjoin%
\definecolor{currentfill}{rgb}{0.150000,0.150000,0.150000}%
\pgfsetfillcolor{currentfill}%
\pgfsetlinewidth{1.003750pt}%
\definecolor{currentstroke}{rgb}{0.150000,0.150000,0.150000}%
\pgfsetstrokecolor{currentstroke}%
\pgfsetdash{}{0pt}%
\pgfsys@defobject{currentmarker}{\pgfqpoint{0.000000in}{0.000000in}}{\pgfqpoint{0.000000in}{0.000000in}}{%
\pgfpathmoveto{\pgfqpoint{0.000000in}{0.000000in}}%
\pgfpathlineto{\pgfqpoint{0.000000in}{0.000000in}}%
\pgfusepath{stroke,fill}%
}%
\begin{pgfscope}%
\pgfsys@transformshift{0.624479in}{1.642187in}%
\pgfsys@useobject{currentmarker}{}%
\end{pgfscope}%
\end{pgfscope}%
\begin{pgfscope}%
\definecolor{textcolor}{rgb}{0.150000,0.150000,0.150000}%
\pgfsetstrokecolor{textcolor}%
\pgfsetfillcolor{textcolor}%
\pgftext[x=0.527257in,y=1.642187in,right,]{\color{textcolor}\sffamily\fontsize{10.000000}{12.000000}\selectfont 5.0}%
\end{pgfscope}%
\begin{pgfscope}%
\pgfpathrectangle{\pgfqpoint{0.624479in}{0.566667in}}{\pgfqpoint{1.638314in}{1.720833in}} %
\pgfusepath{clip}%
\pgfsetroundcap%
\pgfsetroundjoin%
\pgfsetlinewidth{1.003750pt}%
\definecolor{currentstroke}{rgb}{1.000000,1.000000,1.000000}%
\pgfsetstrokecolor{currentstroke}%
\pgfsetdash{}{0pt}%
\pgfpathmoveto{\pgfqpoint{0.624479in}{1.857292in}}%
\pgfpathlineto{\pgfqpoint{2.262793in}{1.857292in}}%
\pgfusepath{stroke}%
\end{pgfscope}%
\begin{pgfscope}%
\pgfsetbuttcap%
\pgfsetroundjoin%
\definecolor{currentfill}{rgb}{0.150000,0.150000,0.150000}%
\pgfsetfillcolor{currentfill}%
\pgfsetlinewidth{1.003750pt}%
\definecolor{currentstroke}{rgb}{0.150000,0.150000,0.150000}%
\pgfsetstrokecolor{currentstroke}%
\pgfsetdash{}{0pt}%
\pgfsys@defobject{currentmarker}{\pgfqpoint{0.000000in}{0.000000in}}{\pgfqpoint{0.000000in}{0.000000in}}{%
\pgfpathmoveto{\pgfqpoint{0.000000in}{0.000000in}}%
\pgfpathlineto{\pgfqpoint{0.000000in}{0.000000in}}%
\pgfusepath{stroke,fill}%
}%
\begin{pgfscope}%
\pgfsys@transformshift{0.624479in}{1.857292in}%
\pgfsys@useobject{currentmarker}{}%
\end{pgfscope}%
\end{pgfscope}%
\begin{pgfscope}%
\definecolor{textcolor}{rgb}{0.150000,0.150000,0.150000}%
\pgfsetstrokecolor{textcolor}%
\pgfsetfillcolor{textcolor}%
\pgftext[x=0.527257in,y=1.857292in,right,]{\color{textcolor}\sffamily\fontsize{10.000000}{12.000000}\selectfont 5.5}%
\end{pgfscope}%
\begin{pgfscope}%
\pgfpathrectangle{\pgfqpoint{0.624479in}{0.566667in}}{\pgfqpoint{1.638314in}{1.720833in}} %
\pgfusepath{clip}%
\pgfsetroundcap%
\pgfsetroundjoin%
\pgfsetlinewidth{1.003750pt}%
\definecolor{currentstroke}{rgb}{1.000000,1.000000,1.000000}%
\pgfsetstrokecolor{currentstroke}%
\pgfsetdash{}{0pt}%
\pgfpathmoveto{\pgfqpoint{0.624479in}{2.072396in}}%
\pgfpathlineto{\pgfqpoint{2.262793in}{2.072396in}}%
\pgfusepath{stroke}%
\end{pgfscope}%
\begin{pgfscope}%
\pgfsetbuttcap%
\pgfsetroundjoin%
\definecolor{currentfill}{rgb}{0.150000,0.150000,0.150000}%
\pgfsetfillcolor{currentfill}%
\pgfsetlinewidth{1.003750pt}%
\definecolor{currentstroke}{rgb}{0.150000,0.150000,0.150000}%
\pgfsetstrokecolor{currentstroke}%
\pgfsetdash{}{0pt}%
\pgfsys@defobject{currentmarker}{\pgfqpoint{0.000000in}{0.000000in}}{\pgfqpoint{0.000000in}{0.000000in}}{%
\pgfpathmoveto{\pgfqpoint{0.000000in}{0.000000in}}%
\pgfpathlineto{\pgfqpoint{0.000000in}{0.000000in}}%
\pgfusepath{stroke,fill}%
}%
\begin{pgfscope}%
\pgfsys@transformshift{0.624479in}{2.072396in}%
\pgfsys@useobject{currentmarker}{}%
\end{pgfscope}%
\end{pgfscope}%
\begin{pgfscope}%
\definecolor{textcolor}{rgb}{0.150000,0.150000,0.150000}%
\pgfsetstrokecolor{textcolor}%
\pgfsetfillcolor{textcolor}%
\pgftext[x=0.527257in,y=2.072396in,right,]{\color{textcolor}\sffamily\fontsize{10.000000}{12.000000}\selectfont 6.0}%
\end{pgfscope}%
\begin{pgfscope}%
\pgfpathrectangle{\pgfqpoint{0.624479in}{0.566667in}}{\pgfqpoint{1.638314in}{1.720833in}} %
\pgfusepath{clip}%
\pgfsetroundcap%
\pgfsetroundjoin%
\pgfsetlinewidth{1.003750pt}%
\definecolor{currentstroke}{rgb}{1.000000,1.000000,1.000000}%
\pgfsetstrokecolor{currentstroke}%
\pgfsetdash{}{0pt}%
\pgfpathmoveto{\pgfqpoint{0.624479in}{2.287500in}}%
\pgfpathlineto{\pgfqpoint{2.262793in}{2.287500in}}%
\pgfusepath{stroke}%
\end{pgfscope}%
\begin{pgfscope}%
\pgfsetbuttcap%
\pgfsetroundjoin%
\definecolor{currentfill}{rgb}{0.150000,0.150000,0.150000}%
\pgfsetfillcolor{currentfill}%
\pgfsetlinewidth{1.003750pt}%
\definecolor{currentstroke}{rgb}{0.150000,0.150000,0.150000}%
\pgfsetstrokecolor{currentstroke}%
\pgfsetdash{}{0pt}%
\pgfsys@defobject{currentmarker}{\pgfqpoint{0.000000in}{0.000000in}}{\pgfqpoint{0.000000in}{0.000000in}}{%
\pgfpathmoveto{\pgfqpoint{0.000000in}{0.000000in}}%
\pgfpathlineto{\pgfqpoint{0.000000in}{0.000000in}}%
\pgfusepath{stroke,fill}%
}%
\begin{pgfscope}%
\pgfsys@transformshift{0.624479in}{2.287500in}%
\pgfsys@useobject{currentmarker}{}%
\end{pgfscope}%
\end{pgfscope}%
\begin{pgfscope}%
\definecolor{textcolor}{rgb}{0.150000,0.150000,0.150000}%
\pgfsetstrokecolor{textcolor}%
\pgfsetfillcolor{textcolor}%
\pgftext[x=0.527257in,y=2.287500in,right,]{\color{textcolor}\sffamily\fontsize{10.000000}{12.000000}\selectfont 6.5}%
\end{pgfscope}%
\begin{pgfscope}%
\definecolor{textcolor}{rgb}{0.150000,0.150000,0.150000}%
\pgfsetstrokecolor{textcolor}%
\pgfsetfillcolor{textcolor}%
\pgftext[x=0.264738in,y=1.427083in,,bottom,rotate=90.000000]{\color{textcolor}\sffamily\fontsize{11.000000}{13.200000}\selectfont Falling time realization 1 obs 2}%
\end{pgfscope}%
\begin{pgfscope}%
\pgfpathrectangle{\pgfqpoint{0.624479in}{0.566667in}}{\pgfqpoint{1.638314in}{1.720833in}} %
\pgfusepath{clip}%
\pgfsetbuttcap%
\pgfsetroundjoin%
\definecolor{currentfill}{rgb}{0.298039,0.447059,0.690196}%
\pgfsetfillcolor{currentfill}%
\pgfsetlinewidth{0.301125pt}%
\definecolor{currentstroke}{rgb}{1.000000,1.000000,1.000000}%
\pgfsetstrokecolor{currentstroke}%
\pgfsetdash{}{0pt}%
\pgfpathmoveto{\pgfqpoint{1.592905in}{1.611131in}}%
\pgfpathcurveto{\pgfqpoint{1.601141in}{1.611131in}}{\pgfqpoint{1.609041in}{1.614403in}}{\pgfqpoint{1.614865in}{1.620227in}}%
\pgfpathcurveto{\pgfqpoint{1.620689in}{1.626051in}}{\pgfqpoint{1.623961in}{1.633951in}}{\pgfqpoint{1.623961in}{1.642187in}}%
\pgfpathcurveto{\pgfqpoint{1.623961in}{1.650424in}}{\pgfqpoint{1.620689in}{1.658324in}}{\pgfqpoint{1.614865in}{1.664148in}}%
\pgfpathcurveto{\pgfqpoint{1.609041in}{1.669972in}}{\pgfqpoint{1.601141in}{1.673244in}}{\pgfqpoint{1.592905in}{1.673244in}}%
\pgfpathcurveto{\pgfqpoint{1.584668in}{1.673244in}}{\pgfqpoint{1.576768in}{1.669972in}}{\pgfqpoint{1.570944in}{1.664148in}}%
\pgfpathcurveto{\pgfqpoint{1.565120in}{1.658324in}}{\pgfqpoint{1.561848in}{1.650424in}}{\pgfqpoint{1.561848in}{1.642187in}}%
\pgfpathcurveto{\pgfqpoint{1.561848in}{1.633951in}}{\pgfqpoint{1.565120in}{1.626051in}}{\pgfqpoint{1.570944in}{1.620227in}}%
\pgfpathcurveto{\pgfqpoint{1.576768in}{1.614403in}}{\pgfqpoint{1.584668in}{1.611131in}}{\pgfqpoint{1.592905in}{1.611131in}}%
\pgfpathlineto{\pgfqpoint{1.592905in}{1.611131in}}%
\pgfusepath{stroke,fill}%
\end{pgfscope}%
\begin{pgfscope}%
\pgfpathrectangle{\pgfqpoint{0.624479in}{0.566667in}}{\pgfqpoint{1.638314in}{1.720833in}} %
\pgfusepath{clip}%
\pgfsetbuttcap%
\pgfsetroundjoin%
\definecolor{currentfill}{rgb}{0.298039,0.447059,0.690196}%
\pgfsetfillcolor{currentfill}%
\pgfsetlinewidth{0.301125pt}%
\definecolor{currentstroke}{rgb}{1.000000,1.000000,1.000000}%
\pgfsetstrokecolor{currentstroke}%
\pgfsetdash{}{0pt}%
\pgfpathmoveto{\pgfqpoint{1.319852in}{1.189527in}}%
\pgfpathcurveto{\pgfqpoint{1.328089in}{1.189527in}}{\pgfqpoint{1.335989in}{1.192799in}}{\pgfqpoint{1.341813in}{1.198623in}}%
\pgfpathcurveto{\pgfqpoint{1.347637in}{1.204447in}}{\pgfqpoint{1.350909in}{1.212347in}}{\pgfqpoint{1.350909in}{1.220583in}}%
\pgfpathcurveto{\pgfqpoint{1.350909in}{1.228820in}}{\pgfqpoint{1.347637in}{1.236720in}}{\pgfqpoint{1.341813in}{1.242544in}}%
\pgfpathcurveto{\pgfqpoint{1.335989in}{1.248368in}}{\pgfqpoint{1.328089in}{1.251640in}}{\pgfqpoint{1.319852in}{1.251640in}}%
\pgfpathcurveto{\pgfqpoint{1.311616in}{1.251640in}}{\pgfqpoint{1.303716in}{1.248368in}}{\pgfqpoint{1.297892in}{1.242544in}}%
\pgfpathcurveto{\pgfqpoint{1.292068in}{1.236720in}}{\pgfqpoint{1.288796in}{1.228820in}}{\pgfqpoint{1.288796in}{1.220583in}}%
\pgfpathcurveto{\pgfqpoint{1.288796in}{1.212347in}}{\pgfqpoint{1.292068in}{1.204447in}}{\pgfqpoint{1.297892in}{1.198623in}}%
\pgfpathcurveto{\pgfqpoint{1.303716in}{1.192799in}}{\pgfqpoint{1.311616in}{1.189527in}}{\pgfqpoint{1.319852in}{1.189527in}}%
\pgfpathlineto{\pgfqpoint{1.319852in}{1.189527in}}%
\pgfusepath{stroke,fill}%
\end{pgfscope}%
\begin{pgfscope}%
\pgfpathrectangle{\pgfqpoint{0.624479in}{0.566667in}}{\pgfqpoint{1.638314in}{1.720833in}} %
\pgfusepath{clip}%
\pgfsetbuttcap%
\pgfsetroundjoin%
\definecolor{currentfill}{rgb}{0.298039,0.447059,0.690196}%
\pgfsetfillcolor{currentfill}%
\pgfsetlinewidth{0.301125pt}%
\definecolor{currentstroke}{rgb}{1.000000,1.000000,1.000000}%
\pgfsetstrokecolor{currentstroke}%
\pgfsetdash{}{0pt}%
\pgfpathmoveto{\pgfqpoint{1.319852in}{1.159412in}}%
\pgfpathcurveto{\pgfqpoint{1.328089in}{1.159412in}}{\pgfqpoint{1.335989in}{1.162685in}}{\pgfqpoint{1.341813in}{1.168508in}}%
\pgfpathcurveto{\pgfqpoint{1.347637in}{1.174332in}}{\pgfqpoint{1.350909in}{1.182232in}}{\pgfqpoint{1.350909in}{1.190469in}}%
\pgfpathcurveto{\pgfqpoint{1.350909in}{1.198705in}}{\pgfqpoint{1.347637in}{1.206605in}}{\pgfqpoint{1.341813in}{1.212429in}}%
\pgfpathcurveto{\pgfqpoint{1.335989in}{1.218253in}}{\pgfqpoint{1.328089in}{1.221525in}}{\pgfqpoint{1.319852in}{1.221525in}}%
\pgfpathcurveto{\pgfqpoint{1.311616in}{1.221525in}}{\pgfqpoint{1.303716in}{1.218253in}}{\pgfqpoint{1.297892in}{1.212429in}}%
\pgfpathcurveto{\pgfqpoint{1.292068in}{1.206605in}}{\pgfqpoint{1.288796in}{1.198705in}}{\pgfqpoint{1.288796in}{1.190469in}}%
\pgfpathcurveto{\pgfqpoint{1.288796in}{1.182232in}}{\pgfqpoint{1.292068in}{1.174332in}}{\pgfqpoint{1.297892in}{1.168508in}}%
\pgfpathcurveto{\pgfqpoint{1.303716in}{1.162685in}}{\pgfqpoint{1.311616in}{1.159412in}}{\pgfqpoint{1.319852in}{1.159412in}}%
\pgfpathlineto{\pgfqpoint{1.319852in}{1.159412in}}%
\pgfusepath{stroke,fill}%
\end{pgfscope}%
\begin{pgfscope}%
\pgfpathrectangle{\pgfqpoint{0.624479in}{0.566667in}}{\pgfqpoint{1.638314in}{1.720833in}} %
\pgfusepath{clip}%
\pgfsetbuttcap%
\pgfsetroundjoin%
\definecolor{currentfill}{rgb}{0.298039,0.447059,0.690196}%
\pgfsetfillcolor{currentfill}%
\pgfsetlinewidth{0.301125pt}%
\definecolor{currentstroke}{rgb}{1.000000,1.000000,1.000000}%
\pgfsetstrokecolor{currentstroke}%
\pgfsetdash{}{0pt}%
\pgfpathmoveto{\pgfqpoint{0.952142in}{0.879777in}}%
\pgfpathcurveto{\pgfqpoint{0.960378in}{0.879777in}}{\pgfqpoint{0.968278in}{0.883049in}}{\pgfqpoint{0.974102in}{0.888873in}}%
\pgfpathcurveto{\pgfqpoint{0.979926in}{0.894697in}}{\pgfqpoint{0.983198in}{0.902597in}}{\pgfqpoint{0.983198in}{0.910833in}}%
\pgfpathcurveto{\pgfqpoint{0.983198in}{0.919070in}}{\pgfqpoint{0.979926in}{0.926970in}}{\pgfqpoint{0.974102in}{0.932794in}}%
\pgfpathcurveto{\pgfqpoint{0.968278in}{0.938618in}}{\pgfqpoint{0.960378in}{0.941890in}}{\pgfqpoint{0.952142in}{0.941890in}}%
\pgfpathcurveto{\pgfqpoint{0.943906in}{0.941890in}}{\pgfqpoint{0.936006in}{0.938618in}}{\pgfqpoint{0.930182in}{0.932794in}}%
\pgfpathcurveto{\pgfqpoint{0.924358in}{0.926970in}}{\pgfqpoint{0.921085in}{0.919070in}}{\pgfqpoint{0.921085in}{0.910833in}}%
\pgfpathcurveto{\pgfqpoint{0.921085in}{0.902597in}}{\pgfqpoint{0.924358in}{0.894697in}}{\pgfqpoint{0.930182in}{0.888873in}}%
\pgfpathcurveto{\pgfqpoint{0.936006in}{0.883049in}}{\pgfqpoint{0.943906in}{0.879777in}}{\pgfqpoint{0.952142in}{0.879777in}}%
\pgfpathlineto{\pgfqpoint{0.952142in}{0.879777in}}%
\pgfusepath{stroke,fill}%
\end{pgfscope}%
\begin{pgfscope}%
\pgfpathrectangle{\pgfqpoint{0.624479in}{0.566667in}}{\pgfqpoint{1.638314in}{1.720833in}} %
\pgfusepath{clip}%
\pgfsetbuttcap%
\pgfsetroundjoin%
\definecolor{currentfill}{rgb}{0.298039,0.447059,0.690196}%
\pgfsetfillcolor{currentfill}%
\pgfsetlinewidth{0.301125pt}%
\definecolor{currentstroke}{rgb}{1.000000,1.000000,1.000000}%
\pgfsetstrokecolor{currentstroke}%
\pgfsetdash{}{0pt}%
\pgfpathmoveto{\pgfqpoint{1.057722in}{0.879777in}}%
\pgfpathcurveto{\pgfqpoint{1.065958in}{0.879777in}}{\pgfqpoint{1.073858in}{0.883049in}}{\pgfqpoint{1.079682in}{0.888873in}}%
\pgfpathcurveto{\pgfqpoint{1.085506in}{0.894697in}}{\pgfqpoint{1.088779in}{0.902597in}}{\pgfqpoint{1.088779in}{0.910833in}}%
\pgfpathcurveto{\pgfqpoint{1.088779in}{0.919070in}}{\pgfqpoint{1.085506in}{0.926970in}}{\pgfqpoint{1.079682in}{0.932794in}}%
\pgfpathcurveto{\pgfqpoint{1.073858in}{0.938618in}}{\pgfqpoint{1.065958in}{0.941890in}}{\pgfqpoint{1.057722in}{0.941890in}}%
\pgfpathcurveto{\pgfqpoint{1.049486in}{0.941890in}}{\pgfqpoint{1.041586in}{0.938618in}}{\pgfqpoint{1.035762in}{0.932794in}}%
\pgfpathcurveto{\pgfqpoint{1.029938in}{0.926970in}}{\pgfqpoint{1.026666in}{0.919070in}}{\pgfqpoint{1.026666in}{0.910833in}}%
\pgfpathcurveto{\pgfqpoint{1.026666in}{0.902597in}}{\pgfqpoint{1.029938in}{0.894697in}}{\pgfqpoint{1.035762in}{0.888873in}}%
\pgfpathcurveto{\pgfqpoint{1.041586in}{0.883049in}}{\pgfqpoint{1.049486in}{0.879777in}}{\pgfqpoint{1.057722in}{0.879777in}}%
\pgfpathlineto{\pgfqpoint{1.057722in}{0.879777in}}%
\pgfusepath{stroke,fill}%
\end{pgfscope}%
\begin{pgfscope}%
\pgfpathrectangle{\pgfqpoint{0.624479in}{0.566667in}}{\pgfqpoint{1.638314in}{1.720833in}} %
\pgfusepath{clip}%
\pgfsetbuttcap%
\pgfsetroundjoin%
\definecolor{currentfill}{rgb}{0.298039,0.447059,0.690196}%
\pgfsetfillcolor{currentfill}%
\pgfsetlinewidth{0.301125pt}%
\definecolor{currentstroke}{rgb}{1.000000,1.000000,1.000000}%
\pgfsetstrokecolor{currentstroke}%
\pgfsetdash{}{0pt}%
\pgfpathmoveto{\pgfqpoint{0.813795in}{0.664673in}}%
\pgfpathcurveto{\pgfqpoint{0.822032in}{0.664673in}}{\pgfqpoint{0.829932in}{0.667945in}}{\pgfqpoint{0.835756in}{0.673769in}}%
\pgfpathcurveto{\pgfqpoint{0.841580in}{0.679593in}}{\pgfqpoint{0.844852in}{0.687493in}}{\pgfqpoint{0.844852in}{0.695729in}}%
\pgfpathcurveto{\pgfqpoint{0.844852in}{0.703965in}}{\pgfqpoint{0.841580in}{0.711865in}}{\pgfqpoint{0.835756in}{0.717689in}}%
\pgfpathcurveto{\pgfqpoint{0.829932in}{0.723513in}}{\pgfqpoint{0.822032in}{0.726786in}}{\pgfqpoint{0.813795in}{0.726786in}}%
\pgfpathcurveto{\pgfqpoint{0.805559in}{0.726786in}}{\pgfqpoint{0.797659in}{0.723513in}}{\pgfqpoint{0.791835in}{0.717689in}}%
\pgfpathcurveto{\pgfqpoint{0.786011in}{0.711865in}}{\pgfqpoint{0.782739in}{0.703965in}}{\pgfqpoint{0.782739in}{0.695729in}}%
\pgfpathcurveto{\pgfqpoint{0.782739in}{0.687493in}}{\pgfqpoint{0.786011in}{0.679593in}}{\pgfqpoint{0.791835in}{0.673769in}}%
\pgfpathcurveto{\pgfqpoint{0.797659in}{0.667945in}}{\pgfqpoint{0.805559in}{0.664673in}}{\pgfqpoint{0.813795in}{0.664673in}}%
\pgfpathlineto{\pgfqpoint{0.813795in}{0.664673in}}%
\pgfusepath{stroke,fill}%
\end{pgfscope}%
\begin{pgfscope}%
\pgfpathrectangle{\pgfqpoint{0.624479in}{0.566667in}}{\pgfqpoint{1.638314in}{1.720833in}} %
\pgfusepath{clip}%
\pgfsetbuttcap%
\pgfsetroundjoin%
\definecolor{currentfill}{rgb}{0.298039,0.447059,0.690196}%
\pgfsetfillcolor{currentfill}%
\pgfsetlinewidth{0.301125pt}%
\definecolor{currentstroke}{rgb}{1.000000,1.000000,1.000000}%
\pgfsetstrokecolor{currentstroke}%
\pgfsetdash{}{0pt}%
\pgfpathmoveto{\pgfqpoint{1.032237in}{1.086277in}}%
\pgfpathcurveto{\pgfqpoint{1.040474in}{1.086277in}}{\pgfqpoint{1.048374in}{1.089549in}}{\pgfqpoint{1.054198in}{1.095373in}}%
\pgfpathcurveto{\pgfqpoint{1.060021in}{1.101197in}}{\pgfqpoint{1.063294in}{1.109097in}}{\pgfqpoint{1.063294in}{1.117333in}}%
\pgfpathcurveto{\pgfqpoint{1.063294in}{1.125570in}}{\pgfqpoint{1.060021in}{1.133470in}}{\pgfqpoint{1.054198in}{1.139294in}}%
\pgfpathcurveto{\pgfqpoint{1.048374in}{1.145118in}}{\pgfqpoint{1.040474in}{1.148390in}}{\pgfqpoint{1.032237in}{1.148390in}}%
\pgfpathcurveto{\pgfqpoint{1.024001in}{1.148390in}}{\pgfqpoint{1.016101in}{1.145118in}}{\pgfqpoint{1.010277in}{1.139294in}}%
\pgfpathcurveto{\pgfqpoint{1.004453in}{1.133470in}}{\pgfqpoint{1.001181in}{1.125570in}}{\pgfqpoint{1.001181in}{1.117333in}}%
\pgfpathcurveto{\pgfqpoint{1.001181in}{1.109097in}}{\pgfqpoint{1.004453in}{1.101197in}}{\pgfqpoint{1.010277in}{1.095373in}}%
\pgfpathcurveto{\pgfqpoint{1.016101in}{1.089549in}}{\pgfqpoint{1.024001in}{1.086277in}}{\pgfqpoint{1.032237in}{1.086277in}}%
\pgfpathlineto{\pgfqpoint{1.032237in}{1.086277in}}%
\pgfusepath{stroke,fill}%
\end{pgfscope}%
\begin{pgfscope}%
\pgfpathrectangle{\pgfqpoint{0.624479in}{0.566667in}}{\pgfqpoint{1.638314in}{1.720833in}} %
\pgfusepath{clip}%
\pgfsetbuttcap%
\pgfsetroundjoin%
\definecolor{currentfill}{rgb}{0.298039,0.447059,0.690196}%
\pgfsetfillcolor{currentfill}%
\pgfsetlinewidth{0.301125pt}%
\definecolor{currentstroke}{rgb}{1.000000,1.000000,1.000000}%
\pgfsetstrokecolor{currentstroke}%
\pgfsetdash{}{0pt}%
\pgfpathmoveto{\pgfqpoint{1.483684in}{1.464860in}}%
\pgfpathcurveto{\pgfqpoint{1.491920in}{1.464860in}}{\pgfqpoint{1.499820in}{1.468132in}}{\pgfqpoint{1.505644in}{1.473956in}}%
\pgfpathcurveto{\pgfqpoint{1.511468in}{1.479780in}}{\pgfqpoint{1.514740in}{1.487680in}}{\pgfqpoint{1.514740in}{1.495917in}}%
\pgfpathcurveto{\pgfqpoint{1.514740in}{1.504153in}}{\pgfqpoint{1.511468in}{1.512053in}}{\pgfqpoint{1.505644in}{1.517877in}}%
\pgfpathcurveto{\pgfqpoint{1.499820in}{1.523701in}}{\pgfqpoint{1.491920in}{1.526973in}}{\pgfqpoint{1.483684in}{1.526973in}}%
\pgfpathcurveto{\pgfqpoint{1.475447in}{1.526973in}}{\pgfqpoint{1.467547in}{1.523701in}}{\pgfqpoint{1.461723in}{1.517877in}}%
\pgfpathcurveto{\pgfqpoint{1.455900in}{1.512053in}}{\pgfqpoint{1.452627in}{1.504153in}}{\pgfqpoint{1.452627in}{1.495917in}}%
\pgfpathcurveto{\pgfqpoint{1.452627in}{1.487680in}}{\pgfqpoint{1.455900in}{1.479780in}}{\pgfqpoint{1.461723in}{1.473956in}}%
\pgfpathcurveto{\pgfqpoint{1.467547in}{1.468132in}}{\pgfqpoint{1.475447in}{1.464860in}}{\pgfqpoint{1.483684in}{1.464860in}}%
\pgfpathlineto{\pgfqpoint{1.483684in}{1.464860in}}%
\pgfusepath{stroke,fill}%
\end{pgfscope}%
\begin{pgfscope}%
\pgfpathrectangle{\pgfqpoint{0.624479in}{0.566667in}}{\pgfqpoint{1.638314in}{1.720833in}} %
\pgfusepath{clip}%
\pgfsetbuttcap%
\pgfsetroundjoin%
\definecolor{currentfill}{rgb}{0.298039,0.447059,0.690196}%
\pgfsetfillcolor{currentfill}%
\pgfsetlinewidth{0.301125pt}%
\definecolor{currentstroke}{rgb}{1.000000,1.000000,1.000000}%
\pgfsetstrokecolor{currentstroke}%
\pgfsetdash{}{0pt}%
\pgfpathmoveto{\pgfqpoint{1.855035in}{1.602527in}}%
\pgfpathcurveto{\pgfqpoint{1.863271in}{1.602527in}}{\pgfqpoint{1.871171in}{1.605799in}}{\pgfqpoint{1.876995in}{1.611623in}}%
\pgfpathcurveto{\pgfqpoint{1.882819in}{1.617447in}}{\pgfqpoint{1.886091in}{1.625347in}}{\pgfqpoint{1.886091in}{1.633583in}}%
\pgfpathcurveto{\pgfqpoint{1.886091in}{1.641820in}}{\pgfqpoint{1.882819in}{1.649720in}}{\pgfqpoint{1.876995in}{1.655544in}}%
\pgfpathcurveto{\pgfqpoint{1.871171in}{1.661368in}}{\pgfqpoint{1.863271in}{1.664640in}}{\pgfqpoint{1.855035in}{1.664640in}}%
\pgfpathcurveto{\pgfqpoint{1.846799in}{1.664640in}}{\pgfqpoint{1.838899in}{1.661368in}}{\pgfqpoint{1.833075in}{1.655544in}}%
\pgfpathcurveto{\pgfqpoint{1.827251in}{1.649720in}}{\pgfqpoint{1.823978in}{1.641820in}}{\pgfqpoint{1.823978in}{1.633583in}}%
\pgfpathcurveto{\pgfqpoint{1.823978in}{1.625347in}}{\pgfqpoint{1.827251in}{1.617447in}}{\pgfqpoint{1.833075in}{1.611623in}}%
\pgfpathcurveto{\pgfqpoint{1.838899in}{1.605799in}}{\pgfqpoint{1.846799in}{1.602527in}}{\pgfqpoint{1.855035in}{1.602527in}}%
\pgfpathlineto{\pgfqpoint{1.855035in}{1.602527in}}%
\pgfusepath{stroke,fill}%
\end{pgfscope}%
\begin{pgfscope}%
\pgfpathrectangle{\pgfqpoint{0.624479in}{0.566667in}}{\pgfqpoint{1.638314in}{1.720833in}} %
\pgfusepath{clip}%
\pgfsetbuttcap%
\pgfsetroundjoin%
\definecolor{currentfill}{rgb}{0.298039,0.447059,0.690196}%
\pgfsetfillcolor{currentfill}%
\pgfsetlinewidth{0.301125pt}%
\definecolor{currentstroke}{rgb}{1.000000,1.000000,1.000000}%
\pgfsetstrokecolor{currentstroke}%
\pgfsetdash{}{0pt}%
\pgfpathmoveto{\pgfqpoint{1.188787in}{1.021746in}}%
\pgfpathcurveto{\pgfqpoint{1.197024in}{1.021746in}}{\pgfqpoint{1.204924in}{1.025018in}}{\pgfqpoint{1.210748in}{1.030842in}}%
\pgfpathcurveto{\pgfqpoint{1.216571in}{1.036666in}}{\pgfqpoint{1.219844in}{1.044566in}}{\pgfqpoint{1.219844in}{1.052802in}}%
\pgfpathcurveto{\pgfqpoint{1.219844in}{1.061038in}}{\pgfqpoint{1.216571in}{1.068938in}}{\pgfqpoint{1.210748in}{1.074762in}}%
\pgfpathcurveto{\pgfqpoint{1.204924in}{1.080586in}}{\pgfqpoint{1.197024in}{1.083859in}}{\pgfqpoint{1.188787in}{1.083859in}}%
\pgfpathcurveto{\pgfqpoint{1.180551in}{1.083859in}}{\pgfqpoint{1.172651in}{1.080586in}}{\pgfqpoint{1.166827in}{1.074762in}}%
\pgfpathcurveto{\pgfqpoint{1.161003in}{1.068938in}}{\pgfqpoint{1.157731in}{1.061038in}}{\pgfqpoint{1.157731in}{1.052802in}}%
\pgfpathcurveto{\pgfqpoint{1.157731in}{1.044566in}}{\pgfqpoint{1.161003in}{1.036666in}}{\pgfqpoint{1.166827in}{1.030842in}}%
\pgfpathcurveto{\pgfqpoint{1.172651in}{1.025018in}}{\pgfqpoint{1.180551in}{1.021746in}}{\pgfqpoint{1.188787in}{1.021746in}}%
\pgfpathlineto{\pgfqpoint{1.188787in}{1.021746in}}%
\pgfusepath{stroke,fill}%
\end{pgfscope}%
\begin{pgfscope}%
\pgfpathrectangle{\pgfqpoint{0.624479in}{0.566667in}}{\pgfqpoint{1.638314in}{1.720833in}} %
\pgfusepath{clip}%
\pgfsetbuttcap%
\pgfsetroundjoin%
\definecolor{currentfill}{rgb}{0.298039,0.447059,0.690196}%
\pgfsetfillcolor{currentfill}%
\pgfsetlinewidth{0.301125pt}%
\definecolor{currentstroke}{rgb}{1.000000,1.000000,1.000000}%
\pgfsetstrokecolor{currentstroke}%
\pgfsetdash{}{0pt}%
\pgfpathmoveto{\pgfqpoint{1.192428in}{1.034652in}}%
\pgfpathcurveto{\pgfqpoint{1.200664in}{1.034652in}}{\pgfqpoint{1.208564in}{1.037924in}}{\pgfqpoint{1.214388in}{1.043748in}}%
\pgfpathcurveto{\pgfqpoint{1.220212in}{1.049572in}}{\pgfqpoint{1.223484in}{1.057472in}}{\pgfqpoint{1.223484in}{1.065708in}}%
\pgfpathcurveto{\pgfqpoint{1.223484in}{1.073945in}}{\pgfqpoint{1.220212in}{1.081845in}}{\pgfqpoint{1.214388in}{1.087669in}}%
\pgfpathcurveto{\pgfqpoint{1.208564in}{1.093493in}}{\pgfqpoint{1.200664in}{1.096765in}}{\pgfqpoint{1.192428in}{1.096765in}}%
\pgfpathcurveto{\pgfqpoint{1.184192in}{1.096765in}}{\pgfqpoint{1.176292in}{1.093493in}}{\pgfqpoint{1.170468in}{1.087669in}}%
\pgfpathcurveto{\pgfqpoint{1.164644in}{1.081845in}}{\pgfqpoint{1.161371in}{1.073945in}}{\pgfqpoint{1.161371in}{1.065708in}}%
\pgfpathcurveto{\pgfqpoint{1.161371in}{1.057472in}}{\pgfqpoint{1.164644in}{1.049572in}}{\pgfqpoint{1.170468in}{1.043748in}}%
\pgfpathcurveto{\pgfqpoint{1.176292in}{1.037924in}}{\pgfqpoint{1.184192in}{1.034652in}}{\pgfqpoint{1.192428in}{1.034652in}}%
\pgfpathlineto{\pgfqpoint{1.192428in}{1.034652in}}%
\pgfusepath{stroke,fill}%
\end{pgfscope}%
\begin{pgfscope}%
\pgfpathrectangle{\pgfqpoint{0.624479in}{0.566667in}}{\pgfqpoint{1.638314in}{1.720833in}} %
\pgfusepath{clip}%
\pgfsetbuttcap%
\pgfsetroundjoin%
\definecolor{currentfill}{rgb}{0.298039,0.447059,0.690196}%
\pgfsetfillcolor{currentfill}%
\pgfsetlinewidth{0.301125pt}%
\definecolor{currentstroke}{rgb}{1.000000,1.000000,1.000000}%
\pgfsetstrokecolor{currentstroke}%
\pgfsetdash{}{0pt}%
\pgfpathmoveto{\pgfqpoint{0.846562in}{0.733506in}}%
\pgfpathcurveto{\pgfqpoint{0.854798in}{0.733506in}}{\pgfqpoint{0.862698in}{0.736778in}}{\pgfqpoint{0.868522in}{0.742602in}}%
\pgfpathcurveto{\pgfqpoint{0.874346in}{0.748426in}}{\pgfqpoint{0.877618in}{0.756326in}}{\pgfqpoint{0.877618in}{0.764562in}}%
\pgfpathcurveto{\pgfqpoint{0.877618in}{0.772799in}}{\pgfqpoint{0.874346in}{0.780699in}}{\pgfqpoint{0.868522in}{0.786523in}}%
\pgfpathcurveto{\pgfqpoint{0.862698in}{0.792347in}}{\pgfqpoint{0.854798in}{0.795619in}}{\pgfqpoint{0.846562in}{0.795619in}}%
\pgfpathcurveto{\pgfqpoint{0.838325in}{0.795619in}}{\pgfqpoint{0.830425in}{0.792347in}}{\pgfqpoint{0.824601in}{0.786523in}}%
\pgfpathcurveto{\pgfqpoint{0.818778in}{0.780699in}}{\pgfqpoint{0.815505in}{0.772799in}}{\pgfqpoint{0.815505in}{0.764562in}}%
\pgfpathcurveto{\pgfqpoint{0.815505in}{0.756326in}}{\pgfqpoint{0.818778in}{0.748426in}}{\pgfqpoint{0.824601in}{0.742602in}}%
\pgfpathcurveto{\pgfqpoint{0.830425in}{0.736778in}}{\pgfqpoint{0.838325in}{0.733506in}}{\pgfqpoint{0.846562in}{0.733506in}}%
\pgfpathlineto{\pgfqpoint{0.846562in}{0.733506in}}%
\pgfusepath{stroke,fill}%
\end{pgfscope}%
\begin{pgfscope}%
\pgfpathrectangle{\pgfqpoint{0.624479in}{0.566667in}}{\pgfqpoint{1.638314in}{1.720833in}} %
\pgfusepath{clip}%
\pgfsetbuttcap%
\pgfsetroundjoin%
\definecolor{currentfill}{rgb}{0.298039,0.447059,0.690196}%
\pgfsetfillcolor{currentfill}%
\pgfsetlinewidth{0.301125pt}%
\definecolor{currentstroke}{rgb}{1.000000,1.000000,1.000000}%
\pgfsetstrokecolor{currentstroke}%
\pgfsetdash{}{0pt}%
\pgfpathmoveto{\pgfqpoint{1.509169in}{1.761704in}}%
\pgfpathcurveto{\pgfqpoint{1.517405in}{1.761704in}}{\pgfqpoint{1.525305in}{1.764976in}}{\pgfqpoint{1.531129in}{1.770800in}}%
\pgfpathcurveto{\pgfqpoint{1.536953in}{1.776624in}}{\pgfqpoint{1.540225in}{1.784524in}}{\pgfqpoint{1.540225in}{1.792760in}}%
\pgfpathcurveto{\pgfqpoint{1.540225in}{1.800997in}}{\pgfqpoint{1.536953in}{1.808897in}}{\pgfqpoint{1.531129in}{1.814721in}}%
\pgfpathcurveto{\pgfqpoint{1.525305in}{1.820545in}}{\pgfqpoint{1.517405in}{1.823817in}}{\pgfqpoint{1.509169in}{1.823817in}}%
\pgfpathcurveto{\pgfqpoint{1.500932in}{1.823817in}}{\pgfqpoint{1.493032in}{1.820545in}}{\pgfqpoint{1.487208in}{1.814721in}}%
\pgfpathcurveto{\pgfqpoint{1.481384in}{1.808897in}}{\pgfqpoint{1.478112in}{1.800997in}}{\pgfqpoint{1.478112in}{1.792760in}}%
\pgfpathcurveto{\pgfqpoint{1.478112in}{1.784524in}}{\pgfqpoint{1.481384in}{1.776624in}}{\pgfqpoint{1.487208in}{1.770800in}}%
\pgfpathcurveto{\pgfqpoint{1.493032in}{1.764976in}}{\pgfqpoint{1.500932in}{1.761704in}}{\pgfqpoint{1.509169in}{1.761704in}}%
\pgfpathlineto{\pgfqpoint{1.509169in}{1.761704in}}%
\pgfusepath{stroke,fill}%
\end{pgfscope}%
\begin{pgfscope}%
\pgfpathrectangle{\pgfqpoint{0.624479in}{0.566667in}}{\pgfqpoint{1.638314in}{1.720833in}} %
\pgfusepath{clip}%
\pgfsetbuttcap%
\pgfsetroundjoin%
\definecolor{currentfill}{rgb}{0.298039,0.447059,0.690196}%
\pgfsetfillcolor{currentfill}%
\pgfsetlinewidth{0.301125pt}%
\definecolor{currentstroke}{rgb}{1.000000,1.000000,1.000000}%
\pgfsetstrokecolor{currentstroke}%
\pgfsetdash{}{0pt}%
\pgfpathmoveto{\pgfqpoint{1.199709in}{1.013141in}}%
\pgfpathcurveto{\pgfqpoint{1.207946in}{1.013141in}}{\pgfqpoint{1.215846in}{1.016414in}}{\pgfqpoint{1.221670in}{1.022238in}}%
\pgfpathcurveto{\pgfqpoint{1.227494in}{1.028062in}}{\pgfqpoint{1.230766in}{1.035962in}}{\pgfqpoint{1.230766in}{1.044198in}}%
\pgfpathcurveto{\pgfqpoint{1.230766in}{1.052434in}}{\pgfqpoint{1.227494in}{1.060334in}}{\pgfqpoint{1.221670in}{1.066158in}}%
\pgfpathcurveto{\pgfqpoint{1.215846in}{1.071982in}}{\pgfqpoint{1.207946in}{1.075254in}}{\pgfqpoint{1.199709in}{1.075254in}}%
\pgfpathcurveto{\pgfqpoint{1.191473in}{1.075254in}}{\pgfqpoint{1.183573in}{1.071982in}}{\pgfqpoint{1.177749in}{1.066158in}}%
\pgfpathcurveto{\pgfqpoint{1.171925in}{1.060334in}}{\pgfqpoint{1.168653in}{1.052434in}}{\pgfqpoint{1.168653in}{1.044198in}}%
\pgfpathcurveto{\pgfqpoint{1.168653in}{1.035962in}}{\pgfqpoint{1.171925in}{1.028062in}}{\pgfqpoint{1.177749in}{1.022238in}}%
\pgfpathcurveto{\pgfqpoint{1.183573in}{1.016414in}}{\pgfqpoint{1.191473in}{1.013141in}}{\pgfqpoint{1.199709in}{1.013141in}}%
\pgfpathlineto{\pgfqpoint{1.199709in}{1.013141in}}%
\pgfusepath{stroke,fill}%
\end{pgfscope}%
\begin{pgfscope}%
\pgfpathrectangle{\pgfqpoint{0.624479in}{0.566667in}}{\pgfqpoint{1.638314in}{1.720833in}} %
\pgfusepath{clip}%
\pgfsetbuttcap%
\pgfsetroundjoin%
\definecolor{currentfill}{rgb}{0.298039,0.447059,0.690196}%
\pgfsetfillcolor{currentfill}%
\pgfsetlinewidth{0.301125pt}%
\definecolor{currentstroke}{rgb}{1.000000,1.000000,1.000000}%
\pgfsetstrokecolor{currentstroke}%
\pgfsetdash{}{0pt}%
\pgfpathmoveto{\pgfqpoint{0.781029in}{0.819548in}}%
\pgfpathcurveto{\pgfqpoint{0.789265in}{0.819548in}}{\pgfqpoint{0.797165in}{0.822820in}}{\pgfqpoint{0.802989in}{0.828644in}}%
\pgfpathcurveto{\pgfqpoint{0.808813in}{0.834468in}}{\pgfqpoint{0.812086in}{0.842368in}}{\pgfqpoint{0.812086in}{0.850604in}}%
\pgfpathcurveto{\pgfqpoint{0.812086in}{0.858840in}}{\pgfqpoint{0.808813in}{0.866740in}}{\pgfqpoint{0.802989in}{0.872564in}}%
\pgfpathcurveto{\pgfqpoint{0.797165in}{0.878388in}}{\pgfqpoint{0.789265in}{0.881661in}}{\pgfqpoint{0.781029in}{0.881661in}}%
\pgfpathcurveto{\pgfqpoint{0.772793in}{0.881661in}}{\pgfqpoint{0.764893in}{0.878388in}}{\pgfqpoint{0.759069in}{0.872564in}}%
\pgfpathcurveto{\pgfqpoint{0.753245in}{0.866740in}}{\pgfqpoint{0.749973in}{0.858840in}}{\pgfqpoint{0.749973in}{0.850604in}}%
\pgfpathcurveto{\pgfqpoint{0.749973in}{0.842368in}}{\pgfqpoint{0.753245in}{0.834468in}}{\pgfqpoint{0.759069in}{0.828644in}}%
\pgfpathcurveto{\pgfqpoint{0.764893in}{0.822820in}}{\pgfqpoint{0.772793in}{0.819548in}}{\pgfqpoint{0.781029in}{0.819548in}}%
\pgfpathlineto{\pgfqpoint{0.781029in}{0.819548in}}%
\pgfusepath{stroke,fill}%
\end{pgfscope}%
\begin{pgfscope}%
\pgfpathrectangle{\pgfqpoint{0.624479in}{0.566667in}}{\pgfqpoint{1.638314in}{1.720833in}} %
\pgfusepath{clip}%
\pgfsetbuttcap%
\pgfsetroundjoin%
\definecolor{currentfill}{rgb}{0.298039,0.447059,0.690196}%
\pgfsetfillcolor{currentfill}%
\pgfsetlinewidth{0.301125pt}%
\definecolor{currentstroke}{rgb}{1.000000,1.000000,1.000000}%
\pgfsetstrokecolor{currentstroke}%
\pgfsetdash{}{0pt}%
\pgfpathmoveto{\pgfqpoint{1.054081in}{0.832454in}}%
\pgfpathcurveto{\pgfqpoint{1.062318in}{0.832454in}}{\pgfqpoint{1.070218in}{0.835726in}}{\pgfqpoint{1.076042in}{0.841550in}}%
\pgfpathcurveto{\pgfqpoint{1.081866in}{0.847374in}}{\pgfqpoint{1.085138in}{0.855274in}}{\pgfqpoint{1.085138in}{0.863510in}}%
\pgfpathcurveto{\pgfqpoint{1.085138in}{0.871747in}}{\pgfqpoint{1.081866in}{0.879647in}}{\pgfqpoint{1.076042in}{0.885471in}}%
\pgfpathcurveto{\pgfqpoint{1.070218in}{0.891295in}}{\pgfqpoint{1.062318in}{0.894567in}}{\pgfqpoint{1.054081in}{0.894567in}}%
\pgfpathcurveto{\pgfqpoint{1.045845in}{0.894567in}}{\pgfqpoint{1.037945in}{0.891295in}}{\pgfqpoint{1.032121in}{0.885471in}}%
\pgfpathcurveto{\pgfqpoint{1.026297in}{0.879647in}}{\pgfqpoint{1.023025in}{0.871747in}}{\pgfqpoint{1.023025in}{0.863510in}}%
\pgfpathcurveto{\pgfqpoint{1.023025in}{0.855274in}}{\pgfqpoint{1.026297in}{0.847374in}}{\pgfqpoint{1.032121in}{0.841550in}}%
\pgfpathcurveto{\pgfqpoint{1.037945in}{0.835726in}}{\pgfqpoint{1.045845in}{0.832454in}}{\pgfqpoint{1.054081in}{0.832454in}}%
\pgfpathlineto{\pgfqpoint{1.054081in}{0.832454in}}%
\pgfusepath{stroke,fill}%
\end{pgfscope}%
\begin{pgfscope}%
\pgfpathrectangle{\pgfqpoint{0.624479in}{0.566667in}}{\pgfqpoint{1.638314in}{1.720833in}} %
\pgfusepath{clip}%
\pgfsetbuttcap%
\pgfsetroundjoin%
\definecolor{currentfill}{rgb}{0.298039,0.447059,0.690196}%
\pgfsetfillcolor{currentfill}%
\pgfsetlinewidth{0.301125pt}%
\definecolor{currentstroke}{rgb}{1.000000,1.000000,1.000000}%
\pgfsetstrokecolor{currentstroke}%
\pgfsetdash{}{0pt}%
\pgfpathmoveto{\pgfqpoint{1.676641in}{1.679964in}}%
\pgfpathcurveto{\pgfqpoint{1.684877in}{1.679964in}}{\pgfqpoint{1.692777in}{1.683237in}}{\pgfqpoint{1.698601in}{1.689061in}}%
\pgfpathcurveto{\pgfqpoint{1.704425in}{1.694885in}}{\pgfqpoint{1.707697in}{1.702785in}}{\pgfqpoint{1.707697in}{1.711021in}}%
\pgfpathcurveto{\pgfqpoint{1.707697in}{1.719257in}}{\pgfqpoint{1.704425in}{1.727157in}}{\pgfqpoint{1.698601in}{1.732981in}}%
\pgfpathcurveto{\pgfqpoint{1.692777in}{1.738805in}}{\pgfqpoint{1.684877in}{1.742077in}}{\pgfqpoint{1.676641in}{1.742077in}}%
\pgfpathcurveto{\pgfqpoint{1.668404in}{1.742077in}}{\pgfqpoint{1.660504in}{1.738805in}}{\pgfqpoint{1.654680in}{1.732981in}}%
\pgfpathcurveto{\pgfqpoint{1.648857in}{1.727157in}}{\pgfqpoint{1.645584in}{1.719257in}}{\pgfqpoint{1.645584in}{1.711021in}}%
\pgfpathcurveto{\pgfqpoint{1.645584in}{1.702785in}}{\pgfqpoint{1.648857in}{1.694885in}}{\pgfqpoint{1.654680in}{1.689061in}}%
\pgfpathcurveto{\pgfqpoint{1.660504in}{1.683237in}}{\pgfqpoint{1.668404in}{1.679964in}}{\pgfqpoint{1.676641in}{1.679964in}}%
\pgfpathlineto{\pgfqpoint{1.676641in}{1.679964in}}%
\pgfusepath{stroke,fill}%
\end{pgfscope}%
\begin{pgfscope}%
\pgfpathrectangle{\pgfqpoint{0.624479in}{0.566667in}}{\pgfqpoint{1.638314in}{1.720833in}} %
\pgfusepath{clip}%
\pgfsetbuttcap%
\pgfsetroundjoin%
\definecolor{currentfill}{rgb}{0.298039,0.447059,0.690196}%
\pgfsetfillcolor{currentfill}%
\pgfsetlinewidth{0.301125pt}%
\definecolor{currentstroke}{rgb}{1.000000,1.000000,1.000000}%
\pgfsetstrokecolor{currentstroke}%
\pgfsetdash{}{0pt}%
\pgfpathmoveto{\pgfqpoint{0.923016in}{0.703391in}}%
\pgfpathcurveto{\pgfqpoint{0.931253in}{0.703391in}}{\pgfqpoint{0.939153in}{0.706664in}}{\pgfqpoint{0.944977in}{0.712488in}}%
\pgfpathcurveto{\pgfqpoint{0.950801in}{0.718312in}}{\pgfqpoint{0.954073in}{0.726212in}}{\pgfqpoint{0.954073in}{0.734448in}}%
\pgfpathcurveto{\pgfqpoint{0.954073in}{0.742684in}}{\pgfqpoint{0.950801in}{0.750584in}}{\pgfqpoint{0.944977in}{0.756408in}}%
\pgfpathcurveto{\pgfqpoint{0.939153in}{0.762232in}}{\pgfqpoint{0.931253in}{0.765504in}}{\pgfqpoint{0.923016in}{0.765504in}}%
\pgfpathcurveto{\pgfqpoint{0.914780in}{0.765504in}}{\pgfqpoint{0.906880in}{0.762232in}}{\pgfqpoint{0.901056in}{0.756408in}}%
\pgfpathcurveto{\pgfqpoint{0.895232in}{0.750584in}}{\pgfqpoint{0.891960in}{0.742684in}}{\pgfqpoint{0.891960in}{0.734448in}}%
\pgfpathcurveto{\pgfqpoint{0.891960in}{0.726212in}}{\pgfqpoint{0.895232in}{0.718312in}}{\pgfqpoint{0.901056in}{0.712488in}}%
\pgfpathcurveto{\pgfqpoint{0.906880in}{0.706664in}}{\pgfqpoint{0.914780in}{0.703391in}}{\pgfqpoint{0.923016in}{0.703391in}}%
\pgfpathlineto{\pgfqpoint{0.923016in}{0.703391in}}%
\pgfusepath{stroke,fill}%
\end{pgfscope}%
\begin{pgfscope}%
\pgfpathrectangle{\pgfqpoint{0.624479in}{0.566667in}}{\pgfqpoint{1.638314in}{1.720833in}} %
\pgfusepath{clip}%
\pgfsetbuttcap%
\pgfsetroundjoin%
\definecolor{currentfill}{rgb}{0.298039,0.447059,0.690196}%
\pgfsetfillcolor{currentfill}%
\pgfsetlinewidth{0.301125pt}%
\definecolor{currentstroke}{rgb}{1.000000,1.000000,1.000000}%
\pgfsetstrokecolor{currentstroke}%
\pgfsetdash{}{0pt}%
\pgfpathmoveto{\pgfqpoint{0.897531in}{0.806641in}}%
\pgfpathcurveto{\pgfqpoint{0.905768in}{0.806641in}}{\pgfqpoint{0.913668in}{0.809914in}}{\pgfqpoint{0.919492in}{0.815738in}}%
\pgfpathcurveto{\pgfqpoint{0.925316in}{0.821562in}}{\pgfqpoint{0.928588in}{0.829462in}}{\pgfqpoint{0.928588in}{0.837698in}}%
\pgfpathcurveto{\pgfqpoint{0.928588in}{0.845934in}}{\pgfqpoint{0.925316in}{0.853834in}}{\pgfqpoint{0.919492in}{0.859658in}}%
\pgfpathcurveto{\pgfqpoint{0.913668in}{0.865482in}}{\pgfqpoint{0.905768in}{0.868754in}}{\pgfqpoint{0.897531in}{0.868754in}}%
\pgfpathcurveto{\pgfqpoint{0.889295in}{0.868754in}}{\pgfqpoint{0.881395in}{0.865482in}}{\pgfqpoint{0.875571in}{0.859658in}}%
\pgfpathcurveto{\pgfqpoint{0.869747in}{0.853834in}}{\pgfqpoint{0.866475in}{0.845934in}}{\pgfqpoint{0.866475in}{0.837698in}}%
\pgfpathcurveto{\pgfqpoint{0.866475in}{0.829462in}}{\pgfqpoint{0.869747in}{0.821562in}}{\pgfqpoint{0.875571in}{0.815738in}}%
\pgfpathcurveto{\pgfqpoint{0.881395in}{0.809914in}}{\pgfqpoint{0.889295in}{0.806641in}}{\pgfqpoint{0.897531in}{0.806641in}}%
\pgfpathlineto{\pgfqpoint{0.897531in}{0.806641in}}%
\pgfusepath{stroke,fill}%
\end{pgfscope}%
\begin{pgfscope}%
\pgfpathrectangle{\pgfqpoint{0.624479in}{0.566667in}}{\pgfqpoint{1.638314in}{1.720833in}} %
\pgfusepath{clip}%
\pgfsetbuttcap%
\pgfsetroundjoin%
\definecolor{currentfill}{rgb}{0.298039,0.447059,0.690196}%
\pgfsetfillcolor{currentfill}%
\pgfsetlinewidth{0.301125pt}%
\definecolor{currentstroke}{rgb}{1.000000,1.000000,1.000000}%
\pgfsetstrokecolor{currentstroke}%
\pgfsetdash{}{0pt}%
\pgfpathmoveto{\pgfqpoint{1.294367in}{1.129298in}}%
\pgfpathcurveto{\pgfqpoint{1.302604in}{1.129298in}}{\pgfqpoint{1.310504in}{1.132570in}}{\pgfqpoint{1.316328in}{1.138394in}}%
\pgfpathcurveto{\pgfqpoint{1.322152in}{1.144218in}}{\pgfqpoint{1.325424in}{1.152118in}}{\pgfqpoint{1.325424in}{1.160354in}}%
\pgfpathcurveto{\pgfqpoint{1.325424in}{1.168590in}}{\pgfqpoint{1.322152in}{1.176490in}}{\pgfqpoint{1.316328in}{1.182314in}}%
\pgfpathcurveto{\pgfqpoint{1.310504in}{1.188138in}}{\pgfqpoint{1.302604in}{1.191411in}}{\pgfqpoint{1.294367in}{1.191411in}}%
\pgfpathcurveto{\pgfqpoint{1.286131in}{1.191411in}}{\pgfqpoint{1.278231in}{1.188138in}}{\pgfqpoint{1.272407in}{1.182314in}}%
\pgfpathcurveto{\pgfqpoint{1.266583in}{1.176490in}}{\pgfqpoint{1.263311in}{1.168590in}}{\pgfqpoint{1.263311in}{1.160354in}}%
\pgfpathcurveto{\pgfqpoint{1.263311in}{1.152118in}}{\pgfqpoint{1.266583in}{1.144218in}}{\pgfqpoint{1.272407in}{1.138394in}}%
\pgfpathcurveto{\pgfqpoint{1.278231in}{1.132570in}}{\pgfqpoint{1.286131in}{1.129298in}}{\pgfqpoint{1.294367in}{1.129298in}}%
\pgfpathlineto{\pgfqpoint{1.294367in}{1.129298in}}%
\pgfusepath{stroke,fill}%
\end{pgfscope}%
\begin{pgfscope}%
\pgfpathrectangle{\pgfqpoint{0.624479in}{0.566667in}}{\pgfqpoint{1.638314in}{1.720833in}} %
\pgfusepath{clip}%
\pgfsetbuttcap%
\pgfsetroundjoin%
\definecolor{currentfill}{rgb}{0.298039,0.447059,0.690196}%
\pgfsetfillcolor{currentfill}%
\pgfsetlinewidth{0.301125pt}%
\definecolor{currentstroke}{rgb}{1.000000,1.000000,1.000000}%
\pgfsetstrokecolor{currentstroke}%
\pgfsetdash{}{0pt}%
\pgfpathmoveto{\pgfqpoint{1.156021in}{0.892683in}}%
\pgfpathcurveto{\pgfqpoint{1.164257in}{0.892683in}}{\pgfqpoint{1.172157in}{0.895955in}}{\pgfqpoint{1.177981in}{0.901779in}}%
\pgfpathcurveto{\pgfqpoint{1.183805in}{0.907603in}}{\pgfqpoint{1.187077in}{0.915503in}}{\pgfqpoint{1.187077in}{0.923740in}}%
\pgfpathcurveto{\pgfqpoint{1.187077in}{0.931976in}}{\pgfqpoint{1.183805in}{0.939876in}}{\pgfqpoint{1.177981in}{0.945700in}}%
\pgfpathcurveto{\pgfqpoint{1.172157in}{0.951524in}}{\pgfqpoint{1.164257in}{0.954796in}}{\pgfqpoint{1.156021in}{0.954796in}}%
\pgfpathcurveto{\pgfqpoint{1.147785in}{0.954796in}}{\pgfqpoint{1.139885in}{0.951524in}}{\pgfqpoint{1.134061in}{0.945700in}}%
\pgfpathcurveto{\pgfqpoint{1.128237in}{0.939876in}}{\pgfqpoint{1.124964in}{0.931976in}}{\pgfqpoint{1.124964in}{0.923740in}}%
\pgfpathcurveto{\pgfqpoint{1.124964in}{0.915503in}}{\pgfqpoint{1.128237in}{0.907603in}}{\pgfqpoint{1.134061in}{0.901779in}}%
\pgfpathcurveto{\pgfqpoint{1.139885in}{0.895955in}}{\pgfqpoint{1.147785in}{0.892683in}}{\pgfqpoint{1.156021in}{0.892683in}}%
\pgfpathlineto{\pgfqpoint{1.156021in}{0.892683in}}%
\pgfusepath{stroke,fill}%
\end{pgfscope}%
\begin{pgfscope}%
\pgfpathrectangle{\pgfqpoint{0.624479in}{0.566667in}}{\pgfqpoint{1.638314in}{1.720833in}} %
\pgfusepath{clip}%
\pgfsetbuttcap%
\pgfsetroundjoin%
\definecolor{currentfill}{rgb}{0.298039,0.447059,0.690196}%
\pgfsetfillcolor{currentfill}%
\pgfsetlinewidth{0.301125pt}%
\definecolor{currentstroke}{rgb}{1.000000,1.000000,1.000000}%
\pgfsetstrokecolor{currentstroke}%
\pgfsetdash{}{0pt}%
\pgfpathmoveto{\pgfqpoint{1.166943in}{1.064766in}}%
\pgfpathcurveto{\pgfqpoint{1.175179in}{1.064766in}}{\pgfqpoint{1.183079in}{1.068039in}}{\pgfqpoint{1.188903in}{1.073863in}}%
\pgfpathcurveto{\pgfqpoint{1.194727in}{1.079687in}}{\pgfqpoint{1.198000in}{1.087587in}}{\pgfqpoint{1.198000in}{1.095823in}}%
\pgfpathcurveto{\pgfqpoint{1.198000in}{1.104059in}}{\pgfqpoint{1.194727in}{1.111959in}}{\pgfqpoint{1.188903in}{1.117783in}}%
\pgfpathcurveto{\pgfqpoint{1.183079in}{1.123607in}}{\pgfqpoint{1.175179in}{1.126879in}}{\pgfqpoint{1.166943in}{1.126879in}}%
\pgfpathcurveto{\pgfqpoint{1.158707in}{1.126879in}}{\pgfqpoint{1.150807in}{1.123607in}}{\pgfqpoint{1.144983in}{1.117783in}}%
\pgfpathcurveto{\pgfqpoint{1.139159in}{1.111959in}}{\pgfqpoint{1.135887in}{1.104059in}}{\pgfqpoint{1.135887in}{1.095823in}}%
\pgfpathcurveto{\pgfqpoint{1.135887in}{1.087587in}}{\pgfqpoint{1.139159in}{1.079687in}}{\pgfqpoint{1.144983in}{1.073863in}}%
\pgfpathcurveto{\pgfqpoint{1.150807in}{1.068039in}}{\pgfqpoint{1.158707in}{1.064766in}}{\pgfqpoint{1.166943in}{1.064766in}}%
\pgfpathlineto{\pgfqpoint{1.166943in}{1.064766in}}%
\pgfusepath{stroke,fill}%
\end{pgfscope}%
\begin{pgfscope}%
\pgfpathrectangle{\pgfqpoint{0.624479in}{0.566667in}}{\pgfqpoint{1.638314in}{1.720833in}} %
\pgfusepath{clip}%
\pgfsetbuttcap%
\pgfsetroundjoin%
\definecolor{currentfill}{rgb}{0.298039,0.447059,0.690196}%
\pgfsetfillcolor{currentfill}%
\pgfsetlinewidth{0.301125pt}%
\definecolor{currentstroke}{rgb}{1.000000,1.000000,1.000000}%
\pgfsetstrokecolor{currentstroke}%
\pgfsetdash{}{0pt}%
\pgfpathmoveto{\pgfqpoint{1.348978in}{1.383121in}}%
\pgfpathcurveto{\pgfqpoint{1.357214in}{1.383121in}}{\pgfqpoint{1.365114in}{1.386393in}}{\pgfqpoint{1.370938in}{1.392217in}}%
\pgfpathcurveto{\pgfqpoint{1.376762in}{1.398041in}}{\pgfqpoint{1.380034in}{1.405941in}}{\pgfqpoint{1.380034in}{1.414177in}}%
\pgfpathcurveto{\pgfqpoint{1.380034in}{1.422413in}}{\pgfqpoint{1.376762in}{1.430313in}}{\pgfqpoint{1.370938in}{1.436137in}}%
\pgfpathcurveto{\pgfqpoint{1.365114in}{1.441961in}}{\pgfqpoint{1.357214in}{1.445234in}}{\pgfqpoint{1.348978in}{1.445234in}}%
\pgfpathcurveto{\pgfqpoint{1.340742in}{1.445234in}}{\pgfqpoint{1.332842in}{1.441961in}}{\pgfqpoint{1.327018in}{1.436137in}}%
\pgfpathcurveto{\pgfqpoint{1.321194in}{1.430313in}}{\pgfqpoint{1.317921in}{1.422413in}}{\pgfqpoint{1.317921in}{1.414177in}}%
\pgfpathcurveto{\pgfqpoint{1.317921in}{1.405941in}}{\pgfqpoint{1.321194in}{1.398041in}}{\pgfqpoint{1.327018in}{1.392217in}}%
\pgfpathcurveto{\pgfqpoint{1.332842in}{1.386393in}}{\pgfqpoint{1.340742in}{1.383121in}}{\pgfqpoint{1.348978in}{1.383121in}}%
\pgfpathlineto{\pgfqpoint{1.348978in}{1.383121in}}%
\pgfusepath{stroke,fill}%
\end{pgfscope}%
\begin{pgfscope}%
\pgfpathrectangle{\pgfqpoint{0.624479in}{0.566667in}}{\pgfqpoint{1.638314in}{1.720833in}} %
\pgfusepath{clip}%
\pgfsetbuttcap%
\pgfsetroundjoin%
\definecolor{currentfill}{rgb}{0.298039,0.447059,0.690196}%
\pgfsetfillcolor{currentfill}%
\pgfsetlinewidth{0.301125pt}%
\definecolor{currentstroke}{rgb}{1.000000,1.000000,1.000000}%
\pgfsetstrokecolor{currentstroke}%
\pgfsetdash{}{0pt}%
\pgfpathmoveto{\pgfqpoint{1.469121in}{1.417537in}}%
\pgfpathcurveto{\pgfqpoint{1.477357in}{1.417537in}}{\pgfqpoint{1.485257in}{1.420810in}}{\pgfqpoint{1.491081in}{1.426633in}}%
\pgfpathcurveto{\pgfqpoint{1.496905in}{1.432457in}}{\pgfqpoint{1.500177in}{1.440357in}}{\pgfqpoint{1.500177in}{1.448594in}}%
\pgfpathcurveto{\pgfqpoint{1.500177in}{1.456830in}}{\pgfqpoint{1.496905in}{1.464730in}}{\pgfqpoint{1.491081in}{1.470554in}}%
\pgfpathcurveto{\pgfqpoint{1.485257in}{1.476378in}}{\pgfqpoint{1.477357in}{1.479650in}}{\pgfqpoint{1.469121in}{1.479650in}}%
\pgfpathcurveto{\pgfqpoint{1.460885in}{1.479650in}}{\pgfqpoint{1.452985in}{1.476378in}}{\pgfqpoint{1.447161in}{1.470554in}}%
\pgfpathcurveto{\pgfqpoint{1.441337in}{1.464730in}}{\pgfqpoint{1.438064in}{1.456830in}}{\pgfqpoint{1.438064in}{1.448594in}}%
\pgfpathcurveto{\pgfqpoint{1.438064in}{1.440357in}}{\pgfqpoint{1.441337in}{1.432457in}}{\pgfqpoint{1.447161in}{1.426633in}}%
\pgfpathcurveto{\pgfqpoint{1.452985in}{1.420810in}}{\pgfqpoint{1.460885in}{1.417537in}}{\pgfqpoint{1.469121in}{1.417537in}}%
\pgfpathlineto{\pgfqpoint{1.469121in}{1.417537in}}%
\pgfusepath{stroke,fill}%
\end{pgfscope}%
\begin{pgfscope}%
\pgfpathrectangle{\pgfqpoint{0.624479in}{0.566667in}}{\pgfqpoint{1.638314in}{1.720833in}} %
\pgfusepath{clip}%
\pgfsetbuttcap%
\pgfsetroundjoin%
\definecolor{currentfill}{rgb}{0.298039,0.447059,0.690196}%
\pgfsetfillcolor{currentfill}%
\pgfsetlinewidth{0.301125pt}%
\definecolor{currentstroke}{rgb}{1.000000,1.000000,1.000000}%
\pgfsetstrokecolor{currentstroke}%
\pgfsetdash{}{0pt}%
\pgfpathmoveto{\pgfqpoint{1.225194in}{1.155110in}}%
\pgfpathcurveto{\pgfqpoint{1.233431in}{1.155110in}}{\pgfqpoint{1.241331in}{1.158382in}}{\pgfqpoint{1.247154in}{1.164206in}}%
\pgfpathcurveto{\pgfqpoint{1.252978in}{1.170030in}}{\pgfqpoint{1.256251in}{1.177930in}}{\pgfqpoint{1.256251in}{1.186167in}}%
\pgfpathcurveto{\pgfqpoint{1.256251in}{1.194403in}}{\pgfqpoint{1.252978in}{1.202303in}}{\pgfqpoint{1.247154in}{1.208127in}}%
\pgfpathcurveto{\pgfqpoint{1.241331in}{1.213951in}}{\pgfqpoint{1.233431in}{1.217223in}}{\pgfqpoint{1.225194in}{1.217223in}}%
\pgfpathcurveto{\pgfqpoint{1.216958in}{1.217223in}}{\pgfqpoint{1.209058in}{1.213951in}}{\pgfqpoint{1.203234in}{1.208127in}}%
\pgfpathcurveto{\pgfqpoint{1.197410in}{1.202303in}}{\pgfqpoint{1.194138in}{1.194403in}}{\pgfqpoint{1.194138in}{1.186167in}}%
\pgfpathcurveto{\pgfqpoint{1.194138in}{1.177930in}}{\pgfqpoint{1.197410in}{1.170030in}}{\pgfqpoint{1.203234in}{1.164206in}}%
\pgfpathcurveto{\pgfqpoint{1.209058in}{1.158382in}}{\pgfqpoint{1.216958in}{1.155110in}}{\pgfqpoint{1.225194in}{1.155110in}}%
\pgfpathlineto{\pgfqpoint{1.225194in}{1.155110in}}%
\pgfusepath{stroke,fill}%
\end{pgfscope}%
\begin{pgfscope}%
\pgfpathrectangle{\pgfqpoint{0.624479in}{0.566667in}}{\pgfqpoint{1.638314in}{1.720833in}} %
\pgfusepath{clip}%
\pgfsetbuttcap%
\pgfsetroundjoin%
\definecolor{currentfill}{rgb}{0.298039,0.447059,0.690196}%
\pgfsetfillcolor{currentfill}%
\pgfsetlinewidth{0.301125pt}%
\definecolor{currentstroke}{rgb}{1.000000,1.000000,1.000000}%
\pgfsetstrokecolor{currentstroke}%
\pgfsetdash{}{0pt}%
\pgfpathmoveto{\pgfqpoint{1.807706in}{1.701475in}}%
\pgfpathcurveto{\pgfqpoint{1.815942in}{1.701475in}}{\pgfqpoint{1.823842in}{1.704747in}}{\pgfqpoint{1.829666in}{1.710571in}}%
\pgfpathcurveto{\pgfqpoint{1.835490in}{1.716395in}}{\pgfqpoint{1.838762in}{1.724295in}}{\pgfqpoint{1.838762in}{1.732531in}}%
\pgfpathcurveto{\pgfqpoint{1.838762in}{1.740768in}}{\pgfqpoint{1.835490in}{1.748668in}}{\pgfqpoint{1.829666in}{1.754492in}}%
\pgfpathcurveto{\pgfqpoint{1.823842in}{1.760315in}}{\pgfqpoint{1.815942in}{1.763588in}}{\pgfqpoint{1.807706in}{1.763588in}}%
\pgfpathcurveto{\pgfqpoint{1.799470in}{1.763588in}}{\pgfqpoint{1.791569in}{1.760315in}}{\pgfqpoint{1.785746in}{1.754492in}}%
\pgfpathcurveto{\pgfqpoint{1.779922in}{1.748668in}}{\pgfqpoint{1.776649in}{1.740768in}}{\pgfqpoint{1.776649in}{1.732531in}}%
\pgfpathcurveto{\pgfqpoint{1.776649in}{1.724295in}}{\pgfqpoint{1.779922in}{1.716395in}}{\pgfqpoint{1.785746in}{1.710571in}}%
\pgfpathcurveto{\pgfqpoint{1.791569in}{1.704747in}}{\pgfqpoint{1.799470in}{1.701475in}}{\pgfqpoint{1.807706in}{1.701475in}}%
\pgfpathlineto{\pgfqpoint{1.807706in}{1.701475in}}%
\pgfusepath{stroke,fill}%
\end{pgfscope}%
\begin{pgfscope}%
\pgfpathrectangle{\pgfqpoint{0.624479in}{0.566667in}}{\pgfqpoint{1.638314in}{1.720833in}} %
\pgfusepath{clip}%
\pgfsetbuttcap%
\pgfsetroundjoin%
\definecolor{currentfill}{rgb}{0.298039,0.447059,0.690196}%
\pgfsetfillcolor{currentfill}%
\pgfsetlinewidth{0.301125pt}%
\definecolor{currentstroke}{rgb}{1.000000,1.000000,1.000000}%
\pgfsetstrokecolor{currentstroke}%
\pgfsetdash{}{0pt}%
\pgfpathmoveto{\pgfqpoint{1.225194in}{1.211037in}}%
\pgfpathcurveto{\pgfqpoint{1.233431in}{1.211037in}}{\pgfqpoint{1.241331in}{1.214310in}}{\pgfqpoint{1.247154in}{1.220133in}}%
\pgfpathcurveto{\pgfqpoint{1.252978in}{1.225957in}}{\pgfqpoint{1.256251in}{1.233857in}}{\pgfqpoint{1.256251in}{1.242094in}}%
\pgfpathcurveto{\pgfqpoint{1.256251in}{1.250330in}}{\pgfqpoint{1.252978in}{1.258230in}}{\pgfqpoint{1.247154in}{1.264054in}}%
\pgfpathcurveto{\pgfqpoint{1.241331in}{1.269878in}}{\pgfqpoint{1.233431in}{1.273150in}}{\pgfqpoint{1.225194in}{1.273150in}}%
\pgfpathcurveto{\pgfqpoint{1.216958in}{1.273150in}}{\pgfqpoint{1.209058in}{1.269878in}}{\pgfqpoint{1.203234in}{1.264054in}}%
\pgfpathcurveto{\pgfqpoint{1.197410in}{1.258230in}}{\pgfqpoint{1.194138in}{1.250330in}}{\pgfqpoint{1.194138in}{1.242094in}}%
\pgfpathcurveto{\pgfqpoint{1.194138in}{1.233857in}}{\pgfqpoint{1.197410in}{1.225957in}}{\pgfqpoint{1.203234in}{1.220133in}}%
\pgfpathcurveto{\pgfqpoint{1.209058in}{1.214310in}}{\pgfqpoint{1.216958in}{1.211037in}}{\pgfqpoint{1.225194in}{1.211037in}}%
\pgfpathlineto{\pgfqpoint{1.225194in}{1.211037in}}%
\pgfusepath{stroke,fill}%
\end{pgfscope}%
\begin{pgfscope}%
\pgfpathrectangle{\pgfqpoint{0.624479in}{0.566667in}}{\pgfqpoint{1.638314in}{1.720833in}} %
\pgfusepath{clip}%
\pgfsetbuttcap%
\pgfsetroundjoin%
\definecolor{currentfill}{rgb}{0.298039,0.447059,0.690196}%
\pgfsetfillcolor{currentfill}%
\pgfsetlinewidth{0.301125pt}%
\definecolor{currentstroke}{rgb}{1.000000,1.000000,1.000000}%
\pgfsetstrokecolor{currentstroke}%
\pgfsetdash{}{0pt}%
\pgfpathmoveto{\pgfqpoint{1.148740in}{1.090579in}}%
\pgfpathcurveto{\pgfqpoint{1.156976in}{1.090579in}}{\pgfqpoint{1.164876in}{1.093851in}}{\pgfqpoint{1.170700in}{1.099675in}}%
\pgfpathcurveto{\pgfqpoint{1.176524in}{1.105499in}}{\pgfqpoint{1.179796in}{1.113399in}}{\pgfqpoint{1.179796in}{1.121635in}}%
\pgfpathcurveto{\pgfqpoint{1.179796in}{1.129872in}}{\pgfqpoint{1.176524in}{1.137772in}}{\pgfqpoint{1.170700in}{1.143596in}}%
\pgfpathcurveto{\pgfqpoint{1.164876in}{1.149420in}}{\pgfqpoint{1.156976in}{1.152692in}}{\pgfqpoint{1.148740in}{1.152692in}}%
\pgfpathcurveto{\pgfqpoint{1.140503in}{1.152692in}}{\pgfqpoint{1.132603in}{1.149420in}}{\pgfqpoint{1.126779in}{1.143596in}}%
\pgfpathcurveto{\pgfqpoint{1.120955in}{1.137772in}}{\pgfqpoint{1.117683in}{1.129872in}}{\pgfqpoint{1.117683in}{1.121635in}}%
\pgfpathcurveto{\pgfqpoint{1.117683in}{1.113399in}}{\pgfqpoint{1.120955in}{1.105499in}}{\pgfqpoint{1.126779in}{1.099675in}}%
\pgfpathcurveto{\pgfqpoint{1.132603in}{1.093851in}}{\pgfqpoint{1.140503in}{1.090579in}}{\pgfqpoint{1.148740in}{1.090579in}}%
\pgfpathlineto{\pgfqpoint{1.148740in}{1.090579in}}%
\pgfusepath{stroke,fill}%
\end{pgfscope}%
\begin{pgfscope}%
\pgfpathrectangle{\pgfqpoint{0.624479in}{0.566667in}}{\pgfqpoint{1.638314in}{1.720833in}} %
\pgfusepath{clip}%
\pgfsetbuttcap%
\pgfsetroundjoin%
\definecolor{currentfill}{rgb}{0.298039,0.447059,0.690196}%
\pgfsetfillcolor{currentfill}%
\pgfsetlinewidth{0.301125pt}%
\definecolor{currentstroke}{rgb}{1.000000,1.000000,1.000000}%
\pgfsetstrokecolor{currentstroke}%
\pgfsetdash{}{0pt}%
\pgfpathmoveto{\pgfqpoint{1.308930in}{1.112089in}}%
\pgfpathcurveto{\pgfqpoint{1.317167in}{1.112089in}}{\pgfqpoint{1.325067in}{1.115362in}}{\pgfqpoint{1.330891in}{1.121186in}}%
\pgfpathcurveto{\pgfqpoint{1.336714in}{1.127010in}}{\pgfqpoint{1.339987in}{1.134910in}}{\pgfqpoint{1.339987in}{1.143146in}}%
\pgfpathcurveto{\pgfqpoint{1.339987in}{1.151382in}}{\pgfqpoint{1.336714in}{1.159282in}}{\pgfqpoint{1.330891in}{1.165106in}}%
\pgfpathcurveto{\pgfqpoint{1.325067in}{1.170930in}}{\pgfqpoint{1.317167in}{1.174202in}}{\pgfqpoint{1.308930in}{1.174202in}}%
\pgfpathcurveto{\pgfqpoint{1.300694in}{1.174202in}}{\pgfqpoint{1.292794in}{1.170930in}}{\pgfqpoint{1.286970in}{1.165106in}}%
\pgfpathcurveto{\pgfqpoint{1.281146in}{1.159282in}}{\pgfqpoint{1.277874in}{1.151382in}}{\pgfqpoint{1.277874in}{1.143146in}}%
\pgfpathcurveto{\pgfqpoint{1.277874in}{1.134910in}}{\pgfqpoint{1.281146in}{1.127010in}}{\pgfqpoint{1.286970in}{1.121186in}}%
\pgfpathcurveto{\pgfqpoint{1.292794in}{1.115362in}}{\pgfqpoint{1.300694in}{1.112089in}}{\pgfqpoint{1.308930in}{1.112089in}}%
\pgfpathlineto{\pgfqpoint{1.308930in}{1.112089in}}%
\pgfusepath{stroke,fill}%
\end{pgfscope}%
\begin{pgfscope}%
\pgfpathrectangle{\pgfqpoint{0.624479in}{0.566667in}}{\pgfqpoint{1.638314in}{1.720833in}} %
\pgfusepath{clip}%
\pgfsetbuttcap%
\pgfsetroundjoin%
\definecolor{currentfill}{rgb}{0.298039,0.447059,0.690196}%
\pgfsetfillcolor{currentfill}%
\pgfsetlinewidth{0.301125pt}%
\definecolor{currentstroke}{rgb}{1.000000,1.000000,1.000000}%
\pgfsetstrokecolor{currentstroke}%
\pgfsetdash{}{0pt}%
\pgfpathmoveto{\pgfqpoint{1.534654in}{1.602527in}}%
\pgfpathcurveto{\pgfqpoint{1.542890in}{1.602527in}}{\pgfqpoint{1.550790in}{1.605799in}}{\pgfqpoint{1.556614in}{1.611623in}}%
\pgfpathcurveto{\pgfqpoint{1.562438in}{1.617447in}}{\pgfqpoint{1.565710in}{1.625347in}}{\pgfqpoint{1.565710in}{1.633583in}}%
\pgfpathcurveto{\pgfqpoint{1.565710in}{1.641820in}}{\pgfqpoint{1.562438in}{1.649720in}}{\pgfqpoint{1.556614in}{1.655544in}}%
\pgfpathcurveto{\pgfqpoint{1.550790in}{1.661368in}}{\pgfqpoint{1.542890in}{1.664640in}}{\pgfqpoint{1.534654in}{1.664640in}}%
\pgfpathcurveto{\pgfqpoint{1.526417in}{1.664640in}}{\pgfqpoint{1.518517in}{1.661368in}}{\pgfqpoint{1.512693in}{1.655544in}}%
\pgfpathcurveto{\pgfqpoint{1.506869in}{1.649720in}}{\pgfqpoint{1.503597in}{1.641820in}}{\pgfqpoint{1.503597in}{1.633583in}}%
\pgfpathcurveto{\pgfqpoint{1.503597in}{1.625347in}}{\pgfqpoint{1.506869in}{1.617447in}}{\pgfqpoint{1.512693in}{1.611623in}}%
\pgfpathcurveto{\pgfqpoint{1.518517in}{1.605799in}}{\pgfqpoint{1.526417in}{1.602527in}}{\pgfqpoint{1.534654in}{1.602527in}}%
\pgfpathlineto{\pgfqpoint{1.534654in}{1.602527in}}%
\pgfusepath{stroke,fill}%
\end{pgfscope}%
\begin{pgfscope}%
\pgfpathrectangle{\pgfqpoint{0.624479in}{0.566667in}}{\pgfqpoint{1.638314in}{1.720833in}} %
\pgfusepath{clip}%
\pgfsetbuttcap%
\pgfsetroundjoin%
\definecolor{currentfill}{rgb}{0.298039,0.447059,0.690196}%
\pgfsetfillcolor{currentfill}%
\pgfsetlinewidth{0.301125pt}%
\definecolor{currentstroke}{rgb}{1.000000,1.000000,1.000000}%
\pgfsetstrokecolor{currentstroke}%
\pgfsetdash{}{0pt}%
\pgfpathmoveto{\pgfqpoint{1.181506in}{1.021746in}}%
\pgfpathcurveto{\pgfqpoint{1.189742in}{1.021746in}}{\pgfqpoint{1.197642in}{1.025018in}}{\pgfqpoint{1.203466in}{1.030842in}}%
\pgfpathcurveto{\pgfqpoint{1.209290in}{1.036666in}}{\pgfqpoint{1.212562in}{1.044566in}}{\pgfqpoint{1.212562in}{1.052802in}}%
\pgfpathcurveto{\pgfqpoint{1.212562in}{1.061038in}}{\pgfqpoint{1.209290in}{1.068938in}}{\pgfqpoint{1.203466in}{1.074762in}}%
\pgfpathcurveto{\pgfqpoint{1.197642in}{1.080586in}}{\pgfqpoint{1.189742in}{1.083859in}}{\pgfqpoint{1.181506in}{1.083859in}}%
\pgfpathcurveto{\pgfqpoint{1.173270in}{1.083859in}}{\pgfqpoint{1.165370in}{1.080586in}}{\pgfqpoint{1.159546in}{1.074762in}}%
\pgfpathcurveto{\pgfqpoint{1.153722in}{1.068938in}}{\pgfqpoint{1.150449in}{1.061038in}}{\pgfqpoint{1.150449in}{1.052802in}}%
\pgfpathcurveto{\pgfqpoint{1.150449in}{1.044566in}}{\pgfqpoint{1.153722in}{1.036666in}}{\pgfqpoint{1.159546in}{1.030842in}}%
\pgfpathcurveto{\pgfqpoint{1.165370in}{1.025018in}}{\pgfqpoint{1.173270in}{1.021746in}}{\pgfqpoint{1.181506in}{1.021746in}}%
\pgfpathlineto{\pgfqpoint{1.181506in}{1.021746in}}%
\pgfusepath{stroke,fill}%
\end{pgfscope}%
\begin{pgfscope}%
\pgfpathrectangle{\pgfqpoint{0.624479in}{0.566667in}}{\pgfqpoint{1.638314in}{1.720833in}} %
\pgfusepath{clip}%
\pgfsetbuttcap%
\pgfsetroundjoin%
\definecolor{currentfill}{rgb}{0.298039,0.447059,0.690196}%
\pgfsetfillcolor{currentfill}%
\pgfsetlinewidth{0.301125pt}%
\definecolor{currentstroke}{rgb}{1.000000,1.000000,1.000000}%
\pgfsetstrokecolor{currentstroke}%
\pgfsetdash{}{0pt}%
\pgfpathmoveto{\pgfqpoint{1.902364in}{1.834839in}}%
\pgfpathcurveto{\pgfqpoint{1.910600in}{1.834839in}}{\pgfqpoint{1.918500in}{1.838112in}}{\pgfqpoint{1.924324in}{1.843936in}}%
\pgfpathcurveto{\pgfqpoint{1.930148in}{1.849760in}}{\pgfqpoint{1.933420in}{1.857660in}}{\pgfqpoint{1.933420in}{1.865896in}}%
\pgfpathcurveto{\pgfqpoint{1.933420in}{1.874132in}}{\pgfqpoint{1.930148in}{1.882032in}}{\pgfqpoint{1.924324in}{1.887856in}}%
\pgfpathcurveto{\pgfqpoint{1.918500in}{1.893680in}}{\pgfqpoint{1.910600in}{1.896952in}}{\pgfqpoint{1.902364in}{1.896952in}}%
\pgfpathcurveto{\pgfqpoint{1.894128in}{1.896952in}}{\pgfqpoint{1.886228in}{1.893680in}}{\pgfqpoint{1.880404in}{1.887856in}}%
\pgfpathcurveto{\pgfqpoint{1.874580in}{1.882032in}}{\pgfqpoint{1.871307in}{1.874132in}}{\pgfqpoint{1.871307in}{1.865896in}}%
\pgfpathcurveto{\pgfqpoint{1.871307in}{1.857660in}}{\pgfqpoint{1.874580in}{1.849760in}}{\pgfqpoint{1.880404in}{1.843936in}}%
\pgfpathcurveto{\pgfqpoint{1.886228in}{1.838112in}}{\pgfqpoint{1.894128in}{1.834839in}}{\pgfqpoint{1.902364in}{1.834839in}}%
\pgfpathlineto{\pgfqpoint{1.902364in}{1.834839in}}%
\pgfusepath{stroke,fill}%
\end{pgfscope}%
\begin{pgfscope}%
\pgfpathrectangle{\pgfqpoint{0.624479in}{0.566667in}}{\pgfqpoint{1.638314in}{1.720833in}} %
\pgfusepath{clip}%
\pgfsetbuttcap%
\pgfsetroundjoin%
\definecolor{currentfill}{rgb}{0.298039,0.447059,0.690196}%
\pgfsetfillcolor{currentfill}%
\pgfsetlinewidth{0.301125pt}%
\definecolor{currentstroke}{rgb}{1.000000,1.000000,1.000000}%
\pgfsetstrokecolor{currentstroke}%
\pgfsetdash{}{0pt}%
\pgfpathmoveto{\pgfqpoint{1.545576in}{1.516485in}}%
\pgfpathcurveto{\pgfqpoint{1.553812in}{1.516485in}}{\pgfqpoint{1.561712in}{1.519757in}}{\pgfqpoint{1.567536in}{1.525581in}}%
\pgfpathcurveto{\pgfqpoint{1.573360in}{1.531405in}}{\pgfqpoint{1.576632in}{1.539305in}}{\pgfqpoint{1.576632in}{1.547542in}}%
\pgfpathcurveto{\pgfqpoint{1.576632in}{1.555778in}}{\pgfqpoint{1.573360in}{1.563678in}}{\pgfqpoint{1.567536in}{1.569502in}}%
\pgfpathcurveto{\pgfqpoint{1.561712in}{1.575326in}}{\pgfqpoint{1.553812in}{1.578598in}}{\pgfqpoint{1.545576in}{1.578598in}}%
\pgfpathcurveto{\pgfqpoint{1.537339in}{1.578598in}}{\pgfqpoint{1.529439in}{1.575326in}}{\pgfqpoint{1.523615in}{1.569502in}}%
\pgfpathcurveto{\pgfqpoint{1.517791in}{1.563678in}}{\pgfqpoint{1.514519in}{1.555778in}}{\pgfqpoint{1.514519in}{1.547542in}}%
\pgfpathcurveto{\pgfqpoint{1.514519in}{1.539305in}}{\pgfqpoint{1.517791in}{1.531405in}}{\pgfqpoint{1.523615in}{1.525581in}}%
\pgfpathcurveto{\pgfqpoint{1.529439in}{1.519757in}}{\pgfqpoint{1.537339in}{1.516485in}}{\pgfqpoint{1.545576in}{1.516485in}}%
\pgfpathlineto{\pgfqpoint{1.545576in}{1.516485in}}%
\pgfusepath{stroke,fill}%
\end{pgfscope}%
\begin{pgfscope}%
\pgfpathrectangle{\pgfqpoint{0.624479in}{0.566667in}}{\pgfqpoint{1.638314in}{1.720833in}} %
\pgfusepath{clip}%
\pgfsetbuttcap%
\pgfsetroundjoin%
\definecolor{currentfill}{rgb}{0.298039,0.447059,0.690196}%
\pgfsetfillcolor{currentfill}%
\pgfsetlinewidth{0.301125pt}%
\definecolor{currentstroke}{rgb}{1.000000,1.000000,1.000000}%
\pgfsetstrokecolor{currentstroke}%
\pgfsetdash{}{0pt}%
\pgfpathmoveto{\pgfqpoint{1.793143in}{1.714381in}}%
\pgfpathcurveto{\pgfqpoint{1.801379in}{1.714381in}}{\pgfqpoint{1.809279in}{1.717653in}}{\pgfqpoint{1.815103in}{1.723477in}}%
\pgfpathcurveto{\pgfqpoint{1.820927in}{1.729301in}}{\pgfqpoint{1.824200in}{1.737201in}}{\pgfqpoint{1.824200in}{1.745437in}}%
\pgfpathcurveto{\pgfqpoint{1.824200in}{1.753674in}}{\pgfqpoint{1.820927in}{1.761574in}}{\pgfqpoint{1.815103in}{1.767398in}}%
\pgfpathcurveto{\pgfqpoint{1.809279in}{1.773222in}}{\pgfqpoint{1.801379in}{1.776494in}}{\pgfqpoint{1.793143in}{1.776494in}}%
\pgfpathcurveto{\pgfqpoint{1.784907in}{1.776494in}}{\pgfqpoint{1.777007in}{1.773222in}}{\pgfqpoint{1.771183in}{1.767398in}}%
\pgfpathcurveto{\pgfqpoint{1.765359in}{1.761574in}}{\pgfqpoint{1.762087in}{1.753674in}}{\pgfqpoint{1.762087in}{1.745437in}}%
\pgfpathcurveto{\pgfqpoint{1.762087in}{1.737201in}}{\pgfqpoint{1.765359in}{1.729301in}}{\pgfqpoint{1.771183in}{1.723477in}}%
\pgfpathcurveto{\pgfqpoint{1.777007in}{1.717653in}}{\pgfqpoint{1.784907in}{1.714381in}}{\pgfqpoint{1.793143in}{1.714381in}}%
\pgfpathlineto{\pgfqpoint{1.793143in}{1.714381in}}%
\pgfusepath{stroke,fill}%
\end{pgfscope}%
\begin{pgfscope}%
\pgfpathrectangle{\pgfqpoint{0.624479in}{0.566667in}}{\pgfqpoint{1.638314in}{1.720833in}} %
\pgfusepath{clip}%
\pgfsetbuttcap%
\pgfsetroundjoin%
\definecolor{currentfill}{rgb}{0.298039,0.447059,0.690196}%
\pgfsetfillcolor{currentfill}%
\pgfsetlinewidth{0.301125pt}%
\definecolor{currentstroke}{rgb}{1.000000,1.000000,1.000000}%
\pgfsetstrokecolor{currentstroke}%
\pgfsetdash{}{0pt}%
\pgfpathmoveto{\pgfqpoint{2.084399in}{1.981110in}}%
\pgfpathcurveto{\pgfqpoint{2.092635in}{1.981110in}}{\pgfqpoint{2.100535in}{1.984382in}}{\pgfqpoint{2.106359in}{1.990206in}}%
\pgfpathcurveto{\pgfqpoint{2.112183in}{1.996030in}}{\pgfqpoint{2.115455in}{2.003930in}}{\pgfqpoint{2.115455in}{2.012167in}}%
\pgfpathcurveto{\pgfqpoint{2.115455in}{2.020403in}}{\pgfqpoint{2.112183in}{2.028303in}}{\pgfqpoint{2.106359in}{2.034127in}}%
\pgfpathcurveto{\pgfqpoint{2.100535in}{2.039951in}}{\pgfqpoint{2.092635in}{2.043223in}}{\pgfqpoint{2.084399in}{2.043223in}}%
\pgfpathcurveto{\pgfqpoint{2.076163in}{2.043223in}}{\pgfqpoint{2.068262in}{2.039951in}}{\pgfqpoint{2.062439in}{2.034127in}}%
\pgfpathcurveto{\pgfqpoint{2.056615in}{2.028303in}}{\pgfqpoint{2.053342in}{2.020403in}}{\pgfqpoint{2.053342in}{2.012167in}}%
\pgfpathcurveto{\pgfqpoint{2.053342in}{2.003930in}}{\pgfqpoint{2.056615in}{1.996030in}}{\pgfqpoint{2.062439in}{1.990206in}}%
\pgfpathcurveto{\pgfqpoint{2.068262in}{1.984382in}}{\pgfqpoint{2.076163in}{1.981110in}}{\pgfqpoint{2.084399in}{1.981110in}}%
\pgfpathlineto{\pgfqpoint{2.084399in}{1.981110in}}%
\pgfusepath{stroke,fill}%
\end{pgfscope}%
\begin{pgfscope}%
\pgfpathrectangle{\pgfqpoint{0.624479in}{0.566667in}}{\pgfqpoint{1.638314in}{1.720833in}} %
\pgfusepath{clip}%
\pgfsetbuttcap%
\pgfsetroundjoin%
\definecolor{currentfill}{rgb}{0.298039,0.447059,0.690196}%
\pgfsetfillcolor{currentfill}%
\pgfsetlinewidth{0.301125pt}%
\definecolor{currentstroke}{rgb}{1.000000,1.000000,1.000000}%
\pgfsetstrokecolor{currentstroke}%
\pgfsetdash{}{0pt}%
\pgfpathmoveto{\pgfqpoint{1.170584in}{1.043256in}}%
\pgfpathcurveto{\pgfqpoint{1.178820in}{1.043256in}}{\pgfqpoint{1.186720in}{1.046528in}}{\pgfqpoint{1.192544in}{1.052352in}}%
\pgfpathcurveto{\pgfqpoint{1.198368in}{1.058176in}}{\pgfqpoint{1.201640in}{1.066076in}}{\pgfqpoint{1.201640in}{1.074313in}}%
\pgfpathcurveto{\pgfqpoint{1.201640in}{1.082549in}}{\pgfqpoint{1.198368in}{1.090449in}}{\pgfqpoint{1.192544in}{1.096273in}}%
\pgfpathcurveto{\pgfqpoint{1.186720in}{1.102097in}}{\pgfqpoint{1.178820in}{1.105369in}}{\pgfqpoint{1.170584in}{1.105369in}}%
\pgfpathcurveto{\pgfqpoint{1.162347in}{1.105369in}}{\pgfqpoint{1.154447in}{1.102097in}}{\pgfqpoint{1.148624in}{1.096273in}}%
\pgfpathcurveto{\pgfqpoint{1.142800in}{1.090449in}}{\pgfqpoint{1.139527in}{1.082549in}}{\pgfqpoint{1.139527in}{1.074313in}}%
\pgfpathcurveto{\pgfqpoint{1.139527in}{1.066076in}}{\pgfqpoint{1.142800in}{1.058176in}}{\pgfqpoint{1.148624in}{1.052352in}}%
\pgfpathcurveto{\pgfqpoint{1.154447in}{1.046528in}}{\pgfqpoint{1.162347in}{1.043256in}}{\pgfqpoint{1.170584in}{1.043256in}}%
\pgfpathlineto{\pgfqpoint{1.170584in}{1.043256in}}%
\pgfusepath{stroke,fill}%
\end{pgfscope}%
\begin{pgfscope}%
\pgfpathrectangle{\pgfqpoint{0.624479in}{0.566667in}}{\pgfqpoint{1.638314in}{1.720833in}} %
\pgfusepath{clip}%
\pgfsetbuttcap%
\pgfsetroundjoin%
\definecolor{currentfill}{rgb}{0.298039,0.447059,0.690196}%
\pgfsetfillcolor{currentfill}%
\pgfsetlinewidth{0.301125pt}%
\definecolor{currentstroke}{rgb}{1.000000,1.000000,1.000000}%
\pgfsetstrokecolor{currentstroke}%
\pgfsetdash{}{0pt}%
\pgfpathmoveto{\pgfqpoint{1.887801in}{1.830537in}}%
\pgfpathcurveto{\pgfqpoint{1.896037in}{1.830537in}}{\pgfqpoint{1.903937in}{1.833810in}}{\pgfqpoint{1.909761in}{1.839633in}}%
\pgfpathcurveto{\pgfqpoint{1.915585in}{1.845457in}}{\pgfqpoint{1.918858in}{1.853357in}}{\pgfqpoint{1.918858in}{1.861594in}}%
\pgfpathcurveto{\pgfqpoint{1.918858in}{1.869830in}}{\pgfqpoint{1.915585in}{1.877730in}}{\pgfqpoint{1.909761in}{1.883554in}}%
\pgfpathcurveto{\pgfqpoint{1.903937in}{1.889378in}}{\pgfqpoint{1.896037in}{1.892650in}}{\pgfqpoint{1.887801in}{1.892650in}}%
\pgfpathcurveto{\pgfqpoint{1.879565in}{1.892650in}}{\pgfqpoint{1.871665in}{1.889378in}}{\pgfqpoint{1.865841in}{1.883554in}}%
\pgfpathcurveto{\pgfqpoint{1.860017in}{1.877730in}}{\pgfqpoint{1.856745in}{1.869830in}}{\pgfqpoint{1.856745in}{1.861594in}}%
\pgfpathcurveto{\pgfqpoint{1.856745in}{1.853357in}}{\pgfqpoint{1.860017in}{1.845457in}}{\pgfqpoint{1.865841in}{1.839633in}}%
\pgfpathcurveto{\pgfqpoint{1.871665in}{1.833810in}}{\pgfqpoint{1.879565in}{1.830537in}}{\pgfqpoint{1.887801in}{1.830537in}}%
\pgfpathlineto{\pgfqpoint{1.887801in}{1.830537in}}%
\pgfusepath{stroke,fill}%
\end{pgfscope}%
\begin{pgfscope}%
\pgfpathrectangle{\pgfqpoint{0.624479in}{0.566667in}}{\pgfqpoint{1.638314in}{1.720833in}} %
\pgfusepath{clip}%
\pgfsetbuttcap%
\pgfsetroundjoin%
\definecolor{currentfill}{rgb}{0.298039,0.447059,0.690196}%
\pgfsetfillcolor{currentfill}%
\pgfsetlinewidth{0.301125pt}%
\definecolor{currentstroke}{rgb}{1.000000,1.000000,1.000000}%
\pgfsetstrokecolor{currentstroke}%
\pgfsetdash{}{0pt}%
\pgfpathmoveto{\pgfqpoint{1.683922in}{1.679964in}}%
\pgfpathcurveto{\pgfqpoint{1.692158in}{1.679964in}}{\pgfqpoint{1.700058in}{1.683237in}}{\pgfqpoint{1.705882in}{1.689061in}}%
\pgfpathcurveto{\pgfqpoint{1.711706in}{1.694885in}}{\pgfqpoint{1.714979in}{1.702785in}}{\pgfqpoint{1.714979in}{1.711021in}}%
\pgfpathcurveto{\pgfqpoint{1.714979in}{1.719257in}}{\pgfqpoint{1.711706in}{1.727157in}}{\pgfqpoint{1.705882in}{1.732981in}}%
\pgfpathcurveto{\pgfqpoint{1.700058in}{1.738805in}}{\pgfqpoint{1.692158in}{1.742077in}}{\pgfqpoint{1.683922in}{1.742077in}}%
\pgfpathcurveto{\pgfqpoint{1.675686in}{1.742077in}}{\pgfqpoint{1.667786in}{1.738805in}}{\pgfqpoint{1.661962in}{1.732981in}}%
\pgfpathcurveto{\pgfqpoint{1.656138in}{1.727157in}}{\pgfqpoint{1.652866in}{1.719257in}}{\pgfqpoint{1.652866in}{1.711021in}}%
\pgfpathcurveto{\pgfqpoint{1.652866in}{1.702785in}}{\pgfqpoint{1.656138in}{1.694885in}}{\pgfqpoint{1.661962in}{1.689061in}}%
\pgfpathcurveto{\pgfqpoint{1.667786in}{1.683237in}}{\pgfqpoint{1.675686in}{1.679964in}}{\pgfqpoint{1.683922in}{1.679964in}}%
\pgfpathlineto{\pgfqpoint{1.683922in}{1.679964in}}%
\pgfusepath{stroke,fill}%
\end{pgfscope}%
\begin{pgfscope}%
\pgfpathrectangle{\pgfqpoint{0.624479in}{0.566667in}}{\pgfqpoint{1.638314in}{1.720833in}} %
\pgfusepath{clip}%
\pgfsetbuttcap%
\pgfsetroundjoin%
\definecolor{currentfill}{rgb}{0.298039,0.447059,0.690196}%
\pgfsetfillcolor{currentfill}%
\pgfsetlinewidth{0.301125pt}%
\definecolor{currentstroke}{rgb}{1.000000,1.000000,1.000000}%
\pgfsetstrokecolor{currentstroke}%
\pgfsetdash{}{0pt}%
\pgfpathmoveto{\pgfqpoint{1.334415in}{1.249756in}}%
\pgfpathcurveto{\pgfqpoint{1.342651in}{1.249756in}}{\pgfqpoint{1.350551in}{1.253028in}}{\pgfqpoint{1.356375in}{1.258852in}}%
\pgfpathcurveto{\pgfqpoint{1.362199in}{1.264676in}}{\pgfqpoint{1.365472in}{1.272576in}}{\pgfqpoint{1.365472in}{1.280812in}}%
\pgfpathcurveto{\pgfqpoint{1.365472in}{1.289049in}}{\pgfqpoint{1.362199in}{1.296949in}}{\pgfqpoint{1.356375in}{1.302773in}}%
\pgfpathcurveto{\pgfqpoint{1.350551in}{1.308597in}}{\pgfqpoint{1.342651in}{1.311869in}}{\pgfqpoint{1.334415in}{1.311869in}}%
\pgfpathcurveto{\pgfqpoint{1.326179in}{1.311869in}}{\pgfqpoint{1.318279in}{1.308597in}}{\pgfqpoint{1.312455in}{1.302773in}}%
\pgfpathcurveto{\pgfqpoint{1.306631in}{1.296949in}}{\pgfqpoint{1.303359in}{1.289049in}}{\pgfqpoint{1.303359in}{1.280812in}}%
\pgfpathcurveto{\pgfqpoint{1.303359in}{1.272576in}}{\pgfqpoint{1.306631in}{1.264676in}}{\pgfqpoint{1.312455in}{1.258852in}}%
\pgfpathcurveto{\pgfqpoint{1.318279in}{1.253028in}}{\pgfqpoint{1.326179in}{1.249756in}}{\pgfqpoint{1.334415in}{1.249756in}}%
\pgfpathlineto{\pgfqpoint{1.334415in}{1.249756in}}%
\pgfusepath{stroke,fill}%
\end{pgfscope}%
\begin{pgfscope}%
\pgfpathrectangle{\pgfqpoint{0.624479in}{0.566667in}}{\pgfqpoint{1.638314in}{1.720833in}} %
\pgfusepath{clip}%
\pgfsetbuttcap%
\pgfsetroundjoin%
\definecolor{currentfill}{rgb}{0.298039,0.447059,0.690196}%
\pgfsetfillcolor{currentfill}%
\pgfsetlinewidth{0.301125pt}%
\definecolor{currentstroke}{rgb}{1.000000,1.000000,1.000000}%
\pgfsetstrokecolor{currentstroke}%
\pgfsetdash{}{0pt}%
\pgfpathmoveto{\pgfqpoint{1.571060in}{1.525089in}}%
\pgfpathcurveto{\pgfqpoint{1.579297in}{1.525089in}}{\pgfqpoint{1.587197in}{1.528362in}}{\pgfqpoint{1.593021in}{1.534186in}}%
\pgfpathcurveto{\pgfqpoint{1.598845in}{1.540010in}}{\pgfqpoint{1.602117in}{1.547910in}}{\pgfqpoint{1.602117in}{1.556146in}}%
\pgfpathcurveto{\pgfqpoint{1.602117in}{1.564382in}}{\pgfqpoint{1.598845in}{1.572282in}}{\pgfqpoint{1.593021in}{1.578106in}}%
\pgfpathcurveto{\pgfqpoint{1.587197in}{1.583930in}}{\pgfqpoint{1.579297in}{1.587202in}}{\pgfqpoint{1.571060in}{1.587202in}}%
\pgfpathcurveto{\pgfqpoint{1.562824in}{1.587202in}}{\pgfqpoint{1.554924in}{1.583930in}}{\pgfqpoint{1.549100in}{1.578106in}}%
\pgfpathcurveto{\pgfqpoint{1.543276in}{1.572282in}}{\pgfqpoint{1.540004in}{1.564382in}}{\pgfqpoint{1.540004in}{1.556146in}}%
\pgfpathcurveto{\pgfqpoint{1.540004in}{1.547910in}}{\pgfqpoint{1.543276in}{1.540010in}}{\pgfqpoint{1.549100in}{1.534186in}}%
\pgfpathcurveto{\pgfqpoint{1.554924in}{1.528362in}}{\pgfqpoint{1.562824in}{1.525089in}}{\pgfqpoint{1.571060in}{1.525089in}}%
\pgfpathlineto{\pgfqpoint{1.571060in}{1.525089in}}%
\pgfusepath{stroke,fill}%
\end{pgfscope}%
\begin{pgfscope}%
\pgfsetrectcap%
\pgfsetmiterjoin%
\pgfsetlinewidth{0.000000pt}%
\definecolor{currentstroke}{rgb}{1.000000,1.000000,1.000000}%
\pgfsetstrokecolor{currentstroke}%
\pgfsetdash{}{0pt}%
\pgfpathmoveto{\pgfqpoint{0.624479in}{0.566667in}}%
\pgfpathlineto{\pgfqpoint{2.262793in}{0.566667in}}%
\pgfusepath{}%
\end{pgfscope}%
\begin{pgfscope}%
\pgfsetrectcap%
\pgfsetmiterjoin%
\pgfsetlinewidth{0.000000pt}%
\definecolor{currentstroke}{rgb}{1.000000,1.000000,1.000000}%
\pgfsetstrokecolor{currentstroke}%
\pgfsetdash{}{0pt}%
\pgfpathmoveto{\pgfqpoint{0.624479in}{0.566667in}}%
\pgfpathlineto{\pgfqpoint{0.624479in}{2.287500in}}%
\pgfusepath{}%
\end{pgfscope}%
\end{pgfpicture}%
\makeatother%
\endgroup%

		\caption{Comparison between the two times registered for one throw.}
		\label{fig_EX1_EX1}
	\end{subfigure}
	\begin{subfigure}[h]{.5\linewidth}
		%% Creator: Matplotlib, PGF backend
%%
%% To include the figure in your LaTeX document, write
%%   \input{<filename>.pgf}
%%
%% Make sure the required packages are loaded in your preamble
%%   \usepackage{pgf}
%%
%% Figures using additional raster images can only be included by \input if
%% they are in the same directory as the main LaTeX file. For loading figures
%% from other directories you can use the `import` package
%%   \usepackage{import}
%% and then include the figures with
%%   \import{<path to file>}{<filename>.pgf}
%%
%% Matplotlib used the following preamble
%%   \usepackage[utf8x]{inputenc}
%%   \usepackage[T1]{fontenc}
%%   \usepackage{cmbright}
%%
\begingroup%
\makeatletter%
\begin{pgfpicture}%
\pgfpathrectangle{\pgfpointorigin}{\pgfqpoint{2.500000in}{2.500000in}}%
\pgfusepath{use as bounding box, clip}%
\begin{pgfscope}%
\pgfsetbuttcap%
\pgfsetmiterjoin%
\definecolor{currentfill}{rgb}{1.000000,1.000000,1.000000}%
\pgfsetfillcolor{currentfill}%
\pgfsetlinewidth{0.000000pt}%
\definecolor{currentstroke}{rgb}{1.000000,1.000000,1.000000}%
\pgfsetstrokecolor{currentstroke}%
\pgfsetdash{}{0pt}%
\pgfpathmoveto{\pgfqpoint{0.000000in}{0.000000in}}%
\pgfpathlineto{\pgfqpoint{2.500000in}{0.000000in}}%
\pgfpathlineto{\pgfqpoint{2.500000in}{2.500000in}}%
\pgfpathlineto{\pgfqpoint{0.000000in}{2.500000in}}%
\pgfpathclose%
\pgfusepath{fill}%
\end{pgfscope}%
\begin{pgfscope}%
\pgfsetbuttcap%
\pgfsetmiterjoin%
\definecolor{currentfill}{rgb}{0.917647,0.917647,0.949020}%
\pgfsetfillcolor{currentfill}%
\pgfsetlinewidth{0.000000pt}%
\definecolor{currentstroke}{rgb}{0.000000,0.000000,0.000000}%
\pgfsetstrokecolor{currentstroke}%
\pgfsetstrokeopacity{0.000000}%
\pgfsetdash{}{0pt}%
\pgfpathmoveto{\pgfqpoint{0.556847in}{0.516222in}}%
\pgfpathlineto{\pgfqpoint{2.279437in}{0.516222in}}%
\pgfpathlineto{\pgfqpoint{2.279437in}{2.299750in}}%
\pgfpathlineto{\pgfqpoint{0.556847in}{2.299750in}}%
\pgfpathclose%
\pgfusepath{fill}%
\end{pgfscope}%
\begin{pgfscope}%
\pgfpathrectangle{\pgfqpoint{0.556847in}{0.516222in}}{\pgfqpoint{1.722590in}{1.783528in}} %
\pgfusepath{clip}%
\pgfsetroundcap%
\pgfsetroundjoin%
\pgfsetlinewidth{0.803000pt}%
\definecolor{currentstroke}{rgb}{1.000000,1.000000,1.000000}%
\pgfsetstrokecolor{currentstroke}%
\pgfsetdash{}{0pt}%
\pgfpathmoveto{\pgfqpoint{0.556847in}{0.516222in}}%
\pgfpathlineto{\pgfqpoint{0.556847in}{2.299750in}}%
\pgfusepath{stroke}%
\end{pgfscope}%
\begin{pgfscope}%
\pgfsetbuttcap%
\pgfsetroundjoin%
\definecolor{currentfill}{rgb}{0.150000,0.150000,0.150000}%
\pgfsetfillcolor{currentfill}%
\pgfsetlinewidth{0.803000pt}%
\definecolor{currentstroke}{rgb}{0.150000,0.150000,0.150000}%
\pgfsetstrokecolor{currentstroke}%
\pgfsetdash{}{0pt}%
\pgfsys@defobject{currentmarker}{\pgfqpoint{0.000000in}{0.000000in}}{\pgfqpoint{0.000000in}{0.000000in}}{%
\pgfpathmoveto{\pgfqpoint{0.000000in}{0.000000in}}%
\pgfpathlineto{\pgfqpoint{0.000000in}{0.000000in}}%
\pgfusepath{stroke,fill}%
}%
\begin{pgfscope}%
\pgfsys@transformshift{0.556847in}{0.516222in}%
\pgfsys@useobject{currentmarker}{}%
\end{pgfscope}%
\end{pgfscope}%
\begin{pgfscope}%
\definecolor{textcolor}{rgb}{0.150000,0.150000,0.150000}%
\pgfsetstrokecolor{textcolor}%
\pgfsetfillcolor{textcolor}%
\pgftext[x=0.556847in,y=0.438444in,,top]{\color{textcolor}\sffamily\fontsize{8.000000}{9.600000}\selectfont 2.0}%
\end{pgfscope}%
\begin{pgfscope}%
\pgfpathrectangle{\pgfqpoint{0.556847in}{0.516222in}}{\pgfqpoint{1.722590in}{1.783528in}} %
\pgfusepath{clip}%
\pgfsetroundcap%
\pgfsetroundjoin%
\pgfsetlinewidth{0.803000pt}%
\definecolor{currentstroke}{rgb}{1.000000,1.000000,1.000000}%
\pgfsetstrokecolor{currentstroke}%
\pgfsetdash{}{0pt}%
\pgfpathmoveto{\pgfqpoint{0.748246in}{0.516222in}}%
\pgfpathlineto{\pgfqpoint{0.748246in}{2.299750in}}%
\pgfusepath{stroke}%
\end{pgfscope}%
\begin{pgfscope}%
\pgfsetbuttcap%
\pgfsetroundjoin%
\definecolor{currentfill}{rgb}{0.150000,0.150000,0.150000}%
\pgfsetfillcolor{currentfill}%
\pgfsetlinewidth{0.803000pt}%
\definecolor{currentstroke}{rgb}{0.150000,0.150000,0.150000}%
\pgfsetstrokecolor{currentstroke}%
\pgfsetdash{}{0pt}%
\pgfsys@defobject{currentmarker}{\pgfqpoint{0.000000in}{0.000000in}}{\pgfqpoint{0.000000in}{0.000000in}}{%
\pgfpathmoveto{\pgfqpoint{0.000000in}{0.000000in}}%
\pgfpathlineto{\pgfqpoint{0.000000in}{0.000000in}}%
\pgfusepath{stroke,fill}%
}%
\begin{pgfscope}%
\pgfsys@transformshift{0.748246in}{0.516222in}%
\pgfsys@useobject{currentmarker}{}%
\end{pgfscope}%
\end{pgfscope}%
\begin{pgfscope}%
\definecolor{textcolor}{rgb}{0.150000,0.150000,0.150000}%
\pgfsetstrokecolor{textcolor}%
\pgfsetfillcolor{textcolor}%
\pgftext[x=0.748246in,y=0.438444in,,top]{\color{textcolor}\sffamily\fontsize{8.000000}{9.600000}\selectfont 2.5}%
\end{pgfscope}%
\begin{pgfscope}%
\pgfpathrectangle{\pgfqpoint{0.556847in}{0.516222in}}{\pgfqpoint{1.722590in}{1.783528in}} %
\pgfusepath{clip}%
\pgfsetroundcap%
\pgfsetroundjoin%
\pgfsetlinewidth{0.803000pt}%
\definecolor{currentstroke}{rgb}{1.000000,1.000000,1.000000}%
\pgfsetstrokecolor{currentstroke}%
\pgfsetdash{}{0pt}%
\pgfpathmoveto{\pgfqpoint{0.939645in}{0.516222in}}%
\pgfpathlineto{\pgfqpoint{0.939645in}{2.299750in}}%
\pgfusepath{stroke}%
\end{pgfscope}%
\begin{pgfscope}%
\pgfsetbuttcap%
\pgfsetroundjoin%
\definecolor{currentfill}{rgb}{0.150000,0.150000,0.150000}%
\pgfsetfillcolor{currentfill}%
\pgfsetlinewidth{0.803000pt}%
\definecolor{currentstroke}{rgb}{0.150000,0.150000,0.150000}%
\pgfsetstrokecolor{currentstroke}%
\pgfsetdash{}{0pt}%
\pgfsys@defobject{currentmarker}{\pgfqpoint{0.000000in}{0.000000in}}{\pgfqpoint{0.000000in}{0.000000in}}{%
\pgfpathmoveto{\pgfqpoint{0.000000in}{0.000000in}}%
\pgfpathlineto{\pgfqpoint{0.000000in}{0.000000in}}%
\pgfusepath{stroke,fill}%
}%
\begin{pgfscope}%
\pgfsys@transformshift{0.939645in}{0.516222in}%
\pgfsys@useobject{currentmarker}{}%
\end{pgfscope}%
\end{pgfscope}%
\begin{pgfscope}%
\definecolor{textcolor}{rgb}{0.150000,0.150000,0.150000}%
\pgfsetstrokecolor{textcolor}%
\pgfsetfillcolor{textcolor}%
\pgftext[x=0.939645in,y=0.438444in,,top]{\color{textcolor}\sffamily\fontsize{8.000000}{9.600000}\selectfont 3.0}%
\end{pgfscope}%
\begin{pgfscope}%
\pgfpathrectangle{\pgfqpoint{0.556847in}{0.516222in}}{\pgfqpoint{1.722590in}{1.783528in}} %
\pgfusepath{clip}%
\pgfsetroundcap%
\pgfsetroundjoin%
\pgfsetlinewidth{0.803000pt}%
\definecolor{currentstroke}{rgb}{1.000000,1.000000,1.000000}%
\pgfsetstrokecolor{currentstroke}%
\pgfsetdash{}{0pt}%
\pgfpathmoveto{\pgfqpoint{1.131044in}{0.516222in}}%
\pgfpathlineto{\pgfqpoint{1.131044in}{2.299750in}}%
\pgfusepath{stroke}%
\end{pgfscope}%
\begin{pgfscope}%
\pgfsetbuttcap%
\pgfsetroundjoin%
\definecolor{currentfill}{rgb}{0.150000,0.150000,0.150000}%
\pgfsetfillcolor{currentfill}%
\pgfsetlinewidth{0.803000pt}%
\definecolor{currentstroke}{rgb}{0.150000,0.150000,0.150000}%
\pgfsetstrokecolor{currentstroke}%
\pgfsetdash{}{0pt}%
\pgfsys@defobject{currentmarker}{\pgfqpoint{0.000000in}{0.000000in}}{\pgfqpoint{0.000000in}{0.000000in}}{%
\pgfpathmoveto{\pgfqpoint{0.000000in}{0.000000in}}%
\pgfpathlineto{\pgfqpoint{0.000000in}{0.000000in}}%
\pgfusepath{stroke,fill}%
}%
\begin{pgfscope}%
\pgfsys@transformshift{1.131044in}{0.516222in}%
\pgfsys@useobject{currentmarker}{}%
\end{pgfscope}%
\end{pgfscope}%
\begin{pgfscope}%
\definecolor{textcolor}{rgb}{0.150000,0.150000,0.150000}%
\pgfsetstrokecolor{textcolor}%
\pgfsetfillcolor{textcolor}%
\pgftext[x=1.131044in,y=0.438444in,,top]{\color{textcolor}\sffamily\fontsize{8.000000}{9.600000}\selectfont 3.5}%
\end{pgfscope}%
\begin{pgfscope}%
\pgfpathrectangle{\pgfqpoint{0.556847in}{0.516222in}}{\pgfqpoint{1.722590in}{1.783528in}} %
\pgfusepath{clip}%
\pgfsetroundcap%
\pgfsetroundjoin%
\pgfsetlinewidth{0.803000pt}%
\definecolor{currentstroke}{rgb}{1.000000,1.000000,1.000000}%
\pgfsetstrokecolor{currentstroke}%
\pgfsetdash{}{0pt}%
\pgfpathmoveto{\pgfqpoint{1.322443in}{0.516222in}}%
\pgfpathlineto{\pgfqpoint{1.322443in}{2.299750in}}%
\pgfusepath{stroke}%
\end{pgfscope}%
\begin{pgfscope}%
\pgfsetbuttcap%
\pgfsetroundjoin%
\definecolor{currentfill}{rgb}{0.150000,0.150000,0.150000}%
\pgfsetfillcolor{currentfill}%
\pgfsetlinewidth{0.803000pt}%
\definecolor{currentstroke}{rgb}{0.150000,0.150000,0.150000}%
\pgfsetstrokecolor{currentstroke}%
\pgfsetdash{}{0pt}%
\pgfsys@defobject{currentmarker}{\pgfqpoint{0.000000in}{0.000000in}}{\pgfqpoint{0.000000in}{0.000000in}}{%
\pgfpathmoveto{\pgfqpoint{0.000000in}{0.000000in}}%
\pgfpathlineto{\pgfqpoint{0.000000in}{0.000000in}}%
\pgfusepath{stroke,fill}%
}%
\begin{pgfscope}%
\pgfsys@transformshift{1.322443in}{0.516222in}%
\pgfsys@useobject{currentmarker}{}%
\end{pgfscope}%
\end{pgfscope}%
\begin{pgfscope}%
\definecolor{textcolor}{rgb}{0.150000,0.150000,0.150000}%
\pgfsetstrokecolor{textcolor}%
\pgfsetfillcolor{textcolor}%
\pgftext[x=1.322443in,y=0.438444in,,top]{\color{textcolor}\sffamily\fontsize{8.000000}{9.600000}\selectfont 4.0}%
\end{pgfscope}%
\begin{pgfscope}%
\pgfpathrectangle{\pgfqpoint{0.556847in}{0.516222in}}{\pgfqpoint{1.722590in}{1.783528in}} %
\pgfusepath{clip}%
\pgfsetroundcap%
\pgfsetroundjoin%
\pgfsetlinewidth{0.803000pt}%
\definecolor{currentstroke}{rgb}{1.000000,1.000000,1.000000}%
\pgfsetstrokecolor{currentstroke}%
\pgfsetdash{}{0pt}%
\pgfpathmoveto{\pgfqpoint{1.513842in}{0.516222in}}%
\pgfpathlineto{\pgfqpoint{1.513842in}{2.299750in}}%
\pgfusepath{stroke}%
\end{pgfscope}%
\begin{pgfscope}%
\pgfsetbuttcap%
\pgfsetroundjoin%
\definecolor{currentfill}{rgb}{0.150000,0.150000,0.150000}%
\pgfsetfillcolor{currentfill}%
\pgfsetlinewidth{0.803000pt}%
\definecolor{currentstroke}{rgb}{0.150000,0.150000,0.150000}%
\pgfsetstrokecolor{currentstroke}%
\pgfsetdash{}{0pt}%
\pgfsys@defobject{currentmarker}{\pgfqpoint{0.000000in}{0.000000in}}{\pgfqpoint{0.000000in}{0.000000in}}{%
\pgfpathmoveto{\pgfqpoint{0.000000in}{0.000000in}}%
\pgfpathlineto{\pgfqpoint{0.000000in}{0.000000in}}%
\pgfusepath{stroke,fill}%
}%
\begin{pgfscope}%
\pgfsys@transformshift{1.513842in}{0.516222in}%
\pgfsys@useobject{currentmarker}{}%
\end{pgfscope}%
\end{pgfscope}%
\begin{pgfscope}%
\definecolor{textcolor}{rgb}{0.150000,0.150000,0.150000}%
\pgfsetstrokecolor{textcolor}%
\pgfsetfillcolor{textcolor}%
\pgftext[x=1.513842in,y=0.438444in,,top]{\color{textcolor}\sffamily\fontsize{8.000000}{9.600000}\selectfont 4.5}%
\end{pgfscope}%
\begin{pgfscope}%
\pgfpathrectangle{\pgfqpoint{0.556847in}{0.516222in}}{\pgfqpoint{1.722590in}{1.783528in}} %
\pgfusepath{clip}%
\pgfsetroundcap%
\pgfsetroundjoin%
\pgfsetlinewidth{0.803000pt}%
\definecolor{currentstroke}{rgb}{1.000000,1.000000,1.000000}%
\pgfsetstrokecolor{currentstroke}%
\pgfsetdash{}{0pt}%
\pgfpathmoveto{\pgfqpoint{1.705241in}{0.516222in}}%
\pgfpathlineto{\pgfqpoint{1.705241in}{2.299750in}}%
\pgfusepath{stroke}%
\end{pgfscope}%
\begin{pgfscope}%
\pgfsetbuttcap%
\pgfsetroundjoin%
\definecolor{currentfill}{rgb}{0.150000,0.150000,0.150000}%
\pgfsetfillcolor{currentfill}%
\pgfsetlinewidth{0.803000pt}%
\definecolor{currentstroke}{rgb}{0.150000,0.150000,0.150000}%
\pgfsetstrokecolor{currentstroke}%
\pgfsetdash{}{0pt}%
\pgfsys@defobject{currentmarker}{\pgfqpoint{0.000000in}{0.000000in}}{\pgfqpoint{0.000000in}{0.000000in}}{%
\pgfpathmoveto{\pgfqpoint{0.000000in}{0.000000in}}%
\pgfpathlineto{\pgfqpoint{0.000000in}{0.000000in}}%
\pgfusepath{stroke,fill}%
}%
\begin{pgfscope}%
\pgfsys@transformshift{1.705241in}{0.516222in}%
\pgfsys@useobject{currentmarker}{}%
\end{pgfscope}%
\end{pgfscope}%
\begin{pgfscope}%
\definecolor{textcolor}{rgb}{0.150000,0.150000,0.150000}%
\pgfsetstrokecolor{textcolor}%
\pgfsetfillcolor{textcolor}%
\pgftext[x=1.705241in,y=0.438444in,,top]{\color{textcolor}\sffamily\fontsize{8.000000}{9.600000}\selectfont 5.0}%
\end{pgfscope}%
\begin{pgfscope}%
\pgfpathrectangle{\pgfqpoint{0.556847in}{0.516222in}}{\pgfqpoint{1.722590in}{1.783528in}} %
\pgfusepath{clip}%
\pgfsetroundcap%
\pgfsetroundjoin%
\pgfsetlinewidth{0.803000pt}%
\definecolor{currentstroke}{rgb}{1.000000,1.000000,1.000000}%
\pgfsetstrokecolor{currentstroke}%
\pgfsetdash{}{0pt}%
\pgfpathmoveto{\pgfqpoint{1.896640in}{0.516222in}}%
\pgfpathlineto{\pgfqpoint{1.896640in}{2.299750in}}%
\pgfusepath{stroke}%
\end{pgfscope}%
\begin{pgfscope}%
\pgfsetbuttcap%
\pgfsetroundjoin%
\definecolor{currentfill}{rgb}{0.150000,0.150000,0.150000}%
\pgfsetfillcolor{currentfill}%
\pgfsetlinewidth{0.803000pt}%
\definecolor{currentstroke}{rgb}{0.150000,0.150000,0.150000}%
\pgfsetstrokecolor{currentstroke}%
\pgfsetdash{}{0pt}%
\pgfsys@defobject{currentmarker}{\pgfqpoint{0.000000in}{0.000000in}}{\pgfqpoint{0.000000in}{0.000000in}}{%
\pgfpathmoveto{\pgfqpoint{0.000000in}{0.000000in}}%
\pgfpathlineto{\pgfqpoint{0.000000in}{0.000000in}}%
\pgfusepath{stroke,fill}%
}%
\begin{pgfscope}%
\pgfsys@transformshift{1.896640in}{0.516222in}%
\pgfsys@useobject{currentmarker}{}%
\end{pgfscope}%
\end{pgfscope}%
\begin{pgfscope}%
\definecolor{textcolor}{rgb}{0.150000,0.150000,0.150000}%
\pgfsetstrokecolor{textcolor}%
\pgfsetfillcolor{textcolor}%
\pgftext[x=1.896640in,y=0.438444in,,top]{\color{textcolor}\sffamily\fontsize{8.000000}{9.600000}\selectfont 5.5}%
\end{pgfscope}%
\begin{pgfscope}%
\pgfpathrectangle{\pgfqpoint{0.556847in}{0.516222in}}{\pgfqpoint{1.722590in}{1.783528in}} %
\pgfusepath{clip}%
\pgfsetroundcap%
\pgfsetroundjoin%
\pgfsetlinewidth{0.803000pt}%
\definecolor{currentstroke}{rgb}{1.000000,1.000000,1.000000}%
\pgfsetstrokecolor{currentstroke}%
\pgfsetdash{}{0pt}%
\pgfpathmoveto{\pgfqpoint{2.088039in}{0.516222in}}%
\pgfpathlineto{\pgfqpoint{2.088039in}{2.299750in}}%
\pgfusepath{stroke}%
\end{pgfscope}%
\begin{pgfscope}%
\pgfsetbuttcap%
\pgfsetroundjoin%
\definecolor{currentfill}{rgb}{0.150000,0.150000,0.150000}%
\pgfsetfillcolor{currentfill}%
\pgfsetlinewidth{0.803000pt}%
\definecolor{currentstroke}{rgb}{0.150000,0.150000,0.150000}%
\pgfsetstrokecolor{currentstroke}%
\pgfsetdash{}{0pt}%
\pgfsys@defobject{currentmarker}{\pgfqpoint{0.000000in}{0.000000in}}{\pgfqpoint{0.000000in}{0.000000in}}{%
\pgfpathmoveto{\pgfqpoint{0.000000in}{0.000000in}}%
\pgfpathlineto{\pgfqpoint{0.000000in}{0.000000in}}%
\pgfusepath{stroke,fill}%
}%
\begin{pgfscope}%
\pgfsys@transformshift{2.088039in}{0.516222in}%
\pgfsys@useobject{currentmarker}{}%
\end{pgfscope}%
\end{pgfscope}%
\begin{pgfscope}%
\definecolor{textcolor}{rgb}{0.150000,0.150000,0.150000}%
\pgfsetstrokecolor{textcolor}%
\pgfsetfillcolor{textcolor}%
\pgftext[x=2.088039in,y=0.438444in,,top]{\color{textcolor}\sffamily\fontsize{8.000000}{9.600000}\selectfont 6.0}%
\end{pgfscope}%
\begin{pgfscope}%
\pgfpathrectangle{\pgfqpoint{0.556847in}{0.516222in}}{\pgfqpoint{1.722590in}{1.783528in}} %
\pgfusepath{clip}%
\pgfsetroundcap%
\pgfsetroundjoin%
\pgfsetlinewidth{0.803000pt}%
\definecolor{currentstroke}{rgb}{1.000000,1.000000,1.000000}%
\pgfsetstrokecolor{currentstroke}%
\pgfsetdash{}{0pt}%
\pgfpathmoveto{\pgfqpoint{2.279437in}{0.516222in}}%
\pgfpathlineto{\pgfqpoint{2.279437in}{2.299750in}}%
\pgfusepath{stroke}%
\end{pgfscope}%
\begin{pgfscope}%
\pgfsetbuttcap%
\pgfsetroundjoin%
\definecolor{currentfill}{rgb}{0.150000,0.150000,0.150000}%
\pgfsetfillcolor{currentfill}%
\pgfsetlinewidth{0.803000pt}%
\definecolor{currentstroke}{rgb}{0.150000,0.150000,0.150000}%
\pgfsetstrokecolor{currentstroke}%
\pgfsetdash{}{0pt}%
\pgfsys@defobject{currentmarker}{\pgfqpoint{0.000000in}{0.000000in}}{\pgfqpoint{0.000000in}{0.000000in}}{%
\pgfpathmoveto{\pgfqpoint{0.000000in}{0.000000in}}%
\pgfpathlineto{\pgfqpoint{0.000000in}{0.000000in}}%
\pgfusepath{stroke,fill}%
}%
\begin{pgfscope}%
\pgfsys@transformshift{2.279437in}{0.516222in}%
\pgfsys@useobject{currentmarker}{}%
\end{pgfscope}%
\end{pgfscope}%
\begin{pgfscope}%
\definecolor{textcolor}{rgb}{0.150000,0.150000,0.150000}%
\pgfsetstrokecolor{textcolor}%
\pgfsetfillcolor{textcolor}%
\pgftext[x=2.279437in,y=0.438444in,,top]{\color{textcolor}\sffamily\fontsize{8.000000}{9.600000}\selectfont 6.5}%
\end{pgfscope}%
\begin{pgfscope}%
\definecolor{textcolor}{rgb}{0.150000,0.150000,0.150000}%
\pgfsetstrokecolor{textcolor}%
\pgfsetfillcolor{textcolor}%
\pgftext[x=1.418142in,y=0.273321in,,top]{\color{textcolor}\sffamily\fontsize{8.800000}{10.560000}\selectfont Falling time realization 1 obs 1}%
\end{pgfscope}%
\begin{pgfscope}%
\pgfpathrectangle{\pgfqpoint{0.556847in}{0.516222in}}{\pgfqpoint{1.722590in}{1.783528in}} %
\pgfusepath{clip}%
\pgfsetroundcap%
\pgfsetroundjoin%
\pgfsetlinewidth{0.803000pt}%
\definecolor{currentstroke}{rgb}{1.000000,1.000000,1.000000}%
\pgfsetstrokecolor{currentstroke}%
\pgfsetdash{}{0pt}%
\pgfpathmoveto{\pgfqpoint{0.556847in}{0.516222in}}%
\pgfpathlineto{\pgfqpoint{2.279437in}{0.516222in}}%
\pgfusepath{stroke}%
\end{pgfscope}%
\begin{pgfscope}%
\pgfsetbuttcap%
\pgfsetroundjoin%
\definecolor{currentfill}{rgb}{0.150000,0.150000,0.150000}%
\pgfsetfillcolor{currentfill}%
\pgfsetlinewidth{0.803000pt}%
\definecolor{currentstroke}{rgb}{0.150000,0.150000,0.150000}%
\pgfsetstrokecolor{currentstroke}%
\pgfsetdash{}{0pt}%
\pgfsys@defobject{currentmarker}{\pgfqpoint{0.000000in}{0.000000in}}{\pgfqpoint{0.000000in}{0.000000in}}{%
\pgfpathmoveto{\pgfqpoint{0.000000in}{0.000000in}}%
\pgfpathlineto{\pgfqpoint{0.000000in}{0.000000in}}%
\pgfusepath{stroke,fill}%
}%
\begin{pgfscope}%
\pgfsys@transformshift{0.556847in}{0.516222in}%
\pgfsys@useobject{currentmarker}{}%
\end{pgfscope}%
\end{pgfscope}%
\begin{pgfscope}%
\definecolor{textcolor}{rgb}{0.150000,0.150000,0.150000}%
\pgfsetstrokecolor{textcolor}%
\pgfsetfillcolor{textcolor}%
\pgftext[x=0.479069in,y=0.516222in,right,]{\color{textcolor}\sffamily\fontsize{8.000000}{9.600000}\selectfont 2.5}%
\end{pgfscope}%
\begin{pgfscope}%
\pgfpathrectangle{\pgfqpoint{0.556847in}{0.516222in}}{\pgfqpoint{1.722590in}{1.783528in}} %
\pgfusepath{clip}%
\pgfsetroundcap%
\pgfsetroundjoin%
\pgfsetlinewidth{0.803000pt}%
\definecolor{currentstroke}{rgb}{1.000000,1.000000,1.000000}%
\pgfsetstrokecolor{currentstroke}%
\pgfsetdash{}{0pt}%
\pgfpathmoveto{\pgfqpoint{0.556847in}{0.739163in}}%
\pgfpathlineto{\pgfqpoint{2.279437in}{0.739163in}}%
\pgfusepath{stroke}%
\end{pgfscope}%
\begin{pgfscope}%
\pgfsetbuttcap%
\pgfsetroundjoin%
\definecolor{currentfill}{rgb}{0.150000,0.150000,0.150000}%
\pgfsetfillcolor{currentfill}%
\pgfsetlinewidth{0.803000pt}%
\definecolor{currentstroke}{rgb}{0.150000,0.150000,0.150000}%
\pgfsetstrokecolor{currentstroke}%
\pgfsetdash{}{0pt}%
\pgfsys@defobject{currentmarker}{\pgfqpoint{0.000000in}{0.000000in}}{\pgfqpoint{0.000000in}{0.000000in}}{%
\pgfpathmoveto{\pgfqpoint{0.000000in}{0.000000in}}%
\pgfpathlineto{\pgfqpoint{0.000000in}{0.000000in}}%
\pgfusepath{stroke,fill}%
}%
\begin{pgfscope}%
\pgfsys@transformshift{0.556847in}{0.739163in}%
\pgfsys@useobject{currentmarker}{}%
\end{pgfscope}%
\end{pgfscope}%
\begin{pgfscope}%
\definecolor{textcolor}{rgb}{0.150000,0.150000,0.150000}%
\pgfsetstrokecolor{textcolor}%
\pgfsetfillcolor{textcolor}%
\pgftext[x=0.479069in,y=0.739163in,right,]{\color{textcolor}\sffamily\fontsize{8.000000}{9.600000}\selectfont 3.0}%
\end{pgfscope}%
\begin{pgfscope}%
\pgfpathrectangle{\pgfqpoint{0.556847in}{0.516222in}}{\pgfqpoint{1.722590in}{1.783528in}} %
\pgfusepath{clip}%
\pgfsetroundcap%
\pgfsetroundjoin%
\pgfsetlinewidth{0.803000pt}%
\definecolor{currentstroke}{rgb}{1.000000,1.000000,1.000000}%
\pgfsetstrokecolor{currentstroke}%
\pgfsetdash{}{0pt}%
\pgfpathmoveto{\pgfqpoint{0.556847in}{0.962104in}}%
\pgfpathlineto{\pgfqpoint{2.279437in}{0.962104in}}%
\pgfusepath{stroke}%
\end{pgfscope}%
\begin{pgfscope}%
\pgfsetbuttcap%
\pgfsetroundjoin%
\definecolor{currentfill}{rgb}{0.150000,0.150000,0.150000}%
\pgfsetfillcolor{currentfill}%
\pgfsetlinewidth{0.803000pt}%
\definecolor{currentstroke}{rgb}{0.150000,0.150000,0.150000}%
\pgfsetstrokecolor{currentstroke}%
\pgfsetdash{}{0pt}%
\pgfsys@defobject{currentmarker}{\pgfqpoint{0.000000in}{0.000000in}}{\pgfqpoint{0.000000in}{0.000000in}}{%
\pgfpathmoveto{\pgfqpoint{0.000000in}{0.000000in}}%
\pgfpathlineto{\pgfqpoint{0.000000in}{0.000000in}}%
\pgfusepath{stroke,fill}%
}%
\begin{pgfscope}%
\pgfsys@transformshift{0.556847in}{0.962104in}%
\pgfsys@useobject{currentmarker}{}%
\end{pgfscope}%
\end{pgfscope}%
\begin{pgfscope}%
\definecolor{textcolor}{rgb}{0.150000,0.150000,0.150000}%
\pgfsetstrokecolor{textcolor}%
\pgfsetfillcolor{textcolor}%
\pgftext[x=0.479069in,y=0.962104in,right,]{\color{textcolor}\sffamily\fontsize{8.000000}{9.600000}\selectfont 3.5}%
\end{pgfscope}%
\begin{pgfscope}%
\pgfpathrectangle{\pgfqpoint{0.556847in}{0.516222in}}{\pgfqpoint{1.722590in}{1.783528in}} %
\pgfusepath{clip}%
\pgfsetroundcap%
\pgfsetroundjoin%
\pgfsetlinewidth{0.803000pt}%
\definecolor{currentstroke}{rgb}{1.000000,1.000000,1.000000}%
\pgfsetstrokecolor{currentstroke}%
\pgfsetdash{}{0pt}%
\pgfpathmoveto{\pgfqpoint{0.556847in}{1.185045in}}%
\pgfpathlineto{\pgfqpoint{2.279437in}{1.185045in}}%
\pgfusepath{stroke}%
\end{pgfscope}%
\begin{pgfscope}%
\pgfsetbuttcap%
\pgfsetroundjoin%
\definecolor{currentfill}{rgb}{0.150000,0.150000,0.150000}%
\pgfsetfillcolor{currentfill}%
\pgfsetlinewidth{0.803000pt}%
\definecolor{currentstroke}{rgb}{0.150000,0.150000,0.150000}%
\pgfsetstrokecolor{currentstroke}%
\pgfsetdash{}{0pt}%
\pgfsys@defobject{currentmarker}{\pgfqpoint{0.000000in}{0.000000in}}{\pgfqpoint{0.000000in}{0.000000in}}{%
\pgfpathmoveto{\pgfqpoint{0.000000in}{0.000000in}}%
\pgfpathlineto{\pgfqpoint{0.000000in}{0.000000in}}%
\pgfusepath{stroke,fill}%
}%
\begin{pgfscope}%
\pgfsys@transformshift{0.556847in}{1.185045in}%
\pgfsys@useobject{currentmarker}{}%
\end{pgfscope}%
\end{pgfscope}%
\begin{pgfscope}%
\definecolor{textcolor}{rgb}{0.150000,0.150000,0.150000}%
\pgfsetstrokecolor{textcolor}%
\pgfsetfillcolor{textcolor}%
\pgftext[x=0.479069in,y=1.185045in,right,]{\color{textcolor}\sffamily\fontsize{8.000000}{9.600000}\selectfont 4.0}%
\end{pgfscope}%
\begin{pgfscope}%
\pgfpathrectangle{\pgfqpoint{0.556847in}{0.516222in}}{\pgfqpoint{1.722590in}{1.783528in}} %
\pgfusepath{clip}%
\pgfsetroundcap%
\pgfsetroundjoin%
\pgfsetlinewidth{0.803000pt}%
\definecolor{currentstroke}{rgb}{1.000000,1.000000,1.000000}%
\pgfsetstrokecolor{currentstroke}%
\pgfsetdash{}{0pt}%
\pgfpathmoveto{\pgfqpoint{0.556847in}{1.407986in}}%
\pgfpathlineto{\pgfqpoint{2.279437in}{1.407986in}}%
\pgfusepath{stroke}%
\end{pgfscope}%
\begin{pgfscope}%
\pgfsetbuttcap%
\pgfsetroundjoin%
\definecolor{currentfill}{rgb}{0.150000,0.150000,0.150000}%
\pgfsetfillcolor{currentfill}%
\pgfsetlinewidth{0.803000pt}%
\definecolor{currentstroke}{rgb}{0.150000,0.150000,0.150000}%
\pgfsetstrokecolor{currentstroke}%
\pgfsetdash{}{0pt}%
\pgfsys@defobject{currentmarker}{\pgfqpoint{0.000000in}{0.000000in}}{\pgfqpoint{0.000000in}{0.000000in}}{%
\pgfpathmoveto{\pgfqpoint{0.000000in}{0.000000in}}%
\pgfpathlineto{\pgfqpoint{0.000000in}{0.000000in}}%
\pgfusepath{stroke,fill}%
}%
\begin{pgfscope}%
\pgfsys@transformshift{0.556847in}{1.407986in}%
\pgfsys@useobject{currentmarker}{}%
\end{pgfscope}%
\end{pgfscope}%
\begin{pgfscope}%
\definecolor{textcolor}{rgb}{0.150000,0.150000,0.150000}%
\pgfsetstrokecolor{textcolor}%
\pgfsetfillcolor{textcolor}%
\pgftext[x=0.479069in,y=1.407986in,right,]{\color{textcolor}\sffamily\fontsize{8.000000}{9.600000}\selectfont 4.5}%
\end{pgfscope}%
\begin{pgfscope}%
\pgfpathrectangle{\pgfqpoint{0.556847in}{0.516222in}}{\pgfqpoint{1.722590in}{1.783528in}} %
\pgfusepath{clip}%
\pgfsetroundcap%
\pgfsetroundjoin%
\pgfsetlinewidth{0.803000pt}%
\definecolor{currentstroke}{rgb}{1.000000,1.000000,1.000000}%
\pgfsetstrokecolor{currentstroke}%
\pgfsetdash{}{0pt}%
\pgfpathmoveto{\pgfqpoint{0.556847in}{1.630927in}}%
\pgfpathlineto{\pgfqpoint{2.279437in}{1.630927in}}%
\pgfusepath{stroke}%
\end{pgfscope}%
\begin{pgfscope}%
\pgfsetbuttcap%
\pgfsetroundjoin%
\definecolor{currentfill}{rgb}{0.150000,0.150000,0.150000}%
\pgfsetfillcolor{currentfill}%
\pgfsetlinewidth{0.803000pt}%
\definecolor{currentstroke}{rgb}{0.150000,0.150000,0.150000}%
\pgfsetstrokecolor{currentstroke}%
\pgfsetdash{}{0pt}%
\pgfsys@defobject{currentmarker}{\pgfqpoint{0.000000in}{0.000000in}}{\pgfqpoint{0.000000in}{0.000000in}}{%
\pgfpathmoveto{\pgfqpoint{0.000000in}{0.000000in}}%
\pgfpathlineto{\pgfqpoint{0.000000in}{0.000000in}}%
\pgfusepath{stroke,fill}%
}%
\begin{pgfscope}%
\pgfsys@transformshift{0.556847in}{1.630927in}%
\pgfsys@useobject{currentmarker}{}%
\end{pgfscope}%
\end{pgfscope}%
\begin{pgfscope}%
\definecolor{textcolor}{rgb}{0.150000,0.150000,0.150000}%
\pgfsetstrokecolor{textcolor}%
\pgfsetfillcolor{textcolor}%
\pgftext[x=0.479069in,y=1.630927in,right,]{\color{textcolor}\sffamily\fontsize{8.000000}{9.600000}\selectfont 5.0}%
\end{pgfscope}%
\begin{pgfscope}%
\pgfpathrectangle{\pgfqpoint{0.556847in}{0.516222in}}{\pgfqpoint{1.722590in}{1.783528in}} %
\pgfusepath{clip}%
\pgfsetroundcap%
\pgfsetroundjoin%
\pgfsetlinewidth{0.803000pt}%
\definecolor{currentstroke}{rgb}{1.000000,1.000000,1.000000}%
\pgfsetstrokecolor{currentstroke}%
\pgfsetdash{}{0pt}%
\pgfpathmoveto{\pgfqpoint{0.556847in}{1.853868in}}%
\pgfpathlineto{\pgfqpoint{2.279437in}{1.853868in}}%
\pgfusepath{stroke}%
\end{pgfscope}%
\begin{pgfscope}%
\pgfsetbuttcap%
\pgfsetroundjoin%
\definecolor{currentfill}{rgb}{0.150000,0.150000,0.150000}%
\pgfsetfillcolor{currentfill}%
\pgfsetlinewidth{0.803000pt}%
\definecolor{currentstroke}{rgb}{0.150000,0.150000,0.150000}%
\pgfsetstrokecolor{currentstroke}%
\pgfsetdash{}{0pt}%
\pgfsys@defobject{currentmarker}{\pgfqpoint{0.000000in}{0.000000in}}{\pgfqpoint{0.000000in}{0.000000in}}{%
\pgfpathmoveto{\pgfqpoint{0.000000in}{0.000000in}}%
\pgfpathlineto{\pgfqpoint{0.000000in}{0.000000in}}%
\pgfusepath{stroke,fill}%
}%
\begin{pgfscope}%
\pgfsys@transformshift{0.556847in}{1.853868in}%
\pgfsys@useobject{currentmarker}{}%
\end{pgfscope}%
\end{pgfscope}%
\begin{pgfscope}%
\definecolor{textcolor}{rgb}{0.150000,0.150000,0.150000}%
\pgfsetstrokecolor{textcolor}%
\pgfsetfillcolor{textcolor}%
\pgftext[x=0.479069in,y=1.853868in,right,]{\color{textcolor}\sffamily\fontsize{8.000000}{9.600000}\selectfont 5.5}%
\end{pgfscope}%
\begin{pgfscope}%
\pgfpathrectangle{\pgfqpoint{0.556847in}{0.516222in}}{\pgfqpoint{1.722590in}{1.783528in}} %
\pgfusepath{clip}%
\pgfsetroundcap%
\pgfsetroundjoin%
\pgfsetlinewidth{0.803000pt}%
\definecolor{currentstroke}{rgb}{1.000000,1.000000,1.000000}%
\pgfsetstrokecolor{currentstroke}%
\pgfsetdash{}{0pt}%
\pgfpathmoveto{\pgfqpoint{0.556847in}{2.076809in}}%
\pgfpathlineto{\pgfqpoint{2.279437in}{2.076809in}}%
\pgfusepath{stroke}%
\end{pgfscope}%
\begin{pgfscope}%
\pgfsetbuttcap%
\pgfsetroundjoin%
\definecolor{currentfill}{rgb}{0.150000,0.150000,0.150000}%
\pgfsetfillcolor{currentfill}%
\pgfsetlinewidth{0.803000pt}%
\definecolor{currentstroke}{rgb}{0.150000,0.150000,0.150000}%
\pgfsetstrokecolor{currentstroke}%
\pgfsetdash{}{0pt}%
\pgfsys@defobject{currentmarker}{\pgfqpoint{0.000000in}{0.000000in}}{\pgfqpoint{0.000000in}{0.000000in}}{%
\pgfpathmoveto{\pgfqpoint{0.000000in}{0.000000in}}%
\pgfpathlineto{\pgfqpoint{0.000000in}{0.000000in}}%
\pgfusepath{stroke,fill}%
}%
\begin{pgfscope}%
\pgfsys@transformshift{0.556847in}{2.076809in}%
\pgfsys@useobject{currentmarker}{}%
\end{pgfscope}%
\end{pgfscope}%
\begin{pgfscope}%
\definecolor{textcolor}{rgb}{0.150000,0.150000,0.150000}%
\pgfsetstrokecolor{textcolor}%
\pgfsetfillcolor{textcolor}%
\pgftext[x=0.479069in,y=2.076809in,right,]{\color{textcolor}\sffamily\fontsize{8.000000}{9.600000}\selectfont 6.0}%
\end{pgfscope}%
\begin{pgfscope}%
\pgfpathrectangle{\pgfqpoint{0.556847in}{0.516222in}}{\pgfqpoint{1.722590in}{1.783528in}} %
\pgfusepath{clip}%
\pgfsetroundcap%
\pgfsetroundjoin%
\pgfsetlinewidth{0.803000pt}%
\definecolor{currentstroke}{rgb}{1.000000,1.000000,1.000000}%
\pgfsetstrokecolor{currentstroke}%
\pgfsetdash{}{0pt}%
\pgfpathmoveto{\pgfqpoint{0.556847in}{2.299750in}}%
\pgfpathlineto{\pgfqpoint{2.279437in}{2.299750in}}%
\pgfusepath{stroke}%
\end{pgfscope}%
\begin{pgfscope}%
\pgfsetbuttcap%
\pgfsetroundjoin%
\definecolor{currentfill}{rgb}{0.150000,0.150000,0.150000}%
\pgfsetfillcolor{currentfill}%
\pgfsetlinewidth{0.803000pt}%
\definecolor{currentstroke}{rgb}{0.150000,0.150000,0.150000}%
\pgfsetstrokecolor{currentstroke}%
\pgfsetdash{}{0pt}%
\pgfsys@defobject{currentmarker}{\pgfqpoint{0.000000in}{0.000000in}}{\pgfqpoint{0.000000in}{0.000000in}}{%
\pgfpathmoveto{\pgfqpoint{0.000000in}{0.000000in}}%
\pgfpathlineto{\pgfqpoint{0.000000in}{0.000000in}}%
\pgfusepath{stroke,fill}%
}%
\begin{pgfscope}%
\pgfsys@transformshift{0.556847in}{2.299750in}%
\pgfsys@useobject{currentmarker}{}%
\end{pgfscope}%
\end{pgfscope}%
\begin{pgfscope}%
\definecolor{textcolor}{rgb}{0.150000,0.150000,0.150000}%
\pgfsetstrokecolor{textcolor}%
\pgfsetfillcolor{textcolor}%
\pgftext[x=0.479069in,y=2.299750in,right,]{\color{textcolor}\sffamily\fontsize{8.000000}{9.600000}\selectfont 6.5}%
\end{pgfscope}%
\begin{pgfscope}%
\definecolor{textcolor}{rgb}{0.150000,0.150000,0.150000}%
\pgfsetstrokecolor{textcolor}%
\pgfsetfillcolor{textcolor}%
\pgftext[x=0.251677in,y=1.407986in,,bottom,rotate=90.000000]{\color{textcolor}\sffamily\fontsize{8.800000}{10.560000}\selectfont Falling time realization 2 obs 1}%
\end{pgfscope}%
\begin{pgfscope}%
\pgfpathrectangle{\pgfqpoint{0.556847in}{0.516222in}}{\pgfqpoint{1.722590in}{1.783528in}} %
\pgfusepath{clip}%
\pgfsetbuttcap%
\pgfsetroundjoin%
\definecolor{currentfill}{rgb}{0.298039,0.447059,0.690196}%
\pgfsetfillcolor{currentfill}%
\pgfsetlinewidth{0.240900pt}%
\definecolor{currentstroke}{rgb}{1.000000,1.000000,1.000000}%
\pgfsetstrokecolor{currentstroke}%
\pgfsetdash{}{0pt}%
\pgfpathmoveto{\pgfqpoint{1.575089in}{1.180742in}}%
\pgfpathcurveto{\pgfqpoint{1.583326in}{1.180742in}}{\pgfqpoint{1.591226in}{1.184014in}}{\pgfqpoint{1.597050in}{1.189838in}}%
\pgfpathcurveto{\pgfqpoint{1.602874in}{1.195662in}}{\pgfqpoint{1.606146in}{1.203562in}}{\pgfqpoint{1.606146in}{1.211798in}}%
\pgfpathcurveto{\pgfqpoint{1.606146in}{1.220034in}}{\pgfqpoint{1.602874in}{1.227934in}}{\pgfqpoint{1.597050in}{1.233758in}}%
\pgfpathcurveto{\pgfqpoint{1.591226in}{1.239582in}}{\pgfqpoint{1.583326in}{1.242855in}}{\pgfqpoint{1.575089in}{1.242855in}}%
\pgfpathcurveto{\pgfqpoint{1.566853in}{1.242855in}}{\pgfqpoint{1.558953in}{1.239582in}}{\pgfqpoint{1.553129in}{1.233758in}}%
\pgfpathcurveto{\pgfqpoint{1.547305in}{1.227934in}}{\pgfqpoint{1.544033in}{1.220034in}}{\pgfqpoint{1.544033in}{1.211798in}}%
\pgfpathcurveto{\pgfqpoint{1.544033in}{1.203562in}}{\pgfqpoint{1.547305in}{1.195662in}}{\pgfqpoint{1.553129in}{1.189838in}}%
\pgfpathcurveto{\pgfqpoint{1.558953in}{1.184014in}}{\pgfqpoint{1.566853in}{1.180742in}}{\pgfqpoint{1.575089in}{1.180742in}}%
\pgfpathclose%
\pgfusepath{stroke,fill}%
\end{pgfscope}%
\begin{pgfscope}%
\pgfpathrectangle{\pgfqpoint{0.556847in}{0.516222in}}{\pgfqpoint{1.722590in}{1.783528in}} %
\pgfusepath{clip}%
\pgfsetbuttcap%
\pgfsetroundjoin%
\definecolor{currentfill}{rgb}{0.298039,0.447059,0.690196}%
\pgfsetfillcolor{currentfill}%
\pgfsetlinewidth{0.240900pt}%
\definecolor{currentstroke}{rgb}{1.000000,1.000000,1.000000}%
\pgfsetstrokecolor{currentstroke}%
\pgfsetdash{}{0pt}%
\pgfpathmoveto{\pgfqpoint{1.287991in}{1.064812in}}%
\pgfpathcurveto{\pgfqpoint{1.296227in}{1.064812in}}{\pgfqpoint{1.304127in}{1.068085in}}{\pgfqpoint{1.309951in}{1.073908in}}%
\pgfpathcurveto{\pgfqpoint{1.315775in}{1.079732in}}{\pgfqpoint{1.319048in}{1.087632in}}{\pgfqpoint{1.319048in}{1.095869in}}%
\pgfpathcurveto{\pgfqpoint{1.319048in}{1.104105in}}{\pgfqpoint{1.315775in}{1.112005in}}{\pgfqpoint{1.309951in}{1.117829in}}%
\pgfpathcurveto{\pgfqpoint{1.304127in}{1.123653in}}{\pgfqpoint{1.296227in}{1.126925in}}{\pgfqpoint{1.287991in}{1.126925in}}%
\pgfpathcurveto{\pgfqpoint{1.279755in}{1.126925in}}{\pgfqpoint{1.271855in}{1.123653in}}{\pgfqpoint{1.266031in}{1.117829in}}%
\pgfpathcurveto{\pgfqpoint{1.260207in}{1.112005in}}{\pgfqpoint{1.256935in}{1.104105in}}{\pgfqpoint{1.256935in}{1.095869in}}%
\pgfpathcurveto{\pgfqpoint{1.256935in}{1.087632in}}{\pgfqpoint{1.260207in}{1.079732in}}{\pgfqpoint{1.266031in}{1.073908in}}%
\pgfpathcurveto{\pgfqpoint{1.271855in}{1.068085in}}{\pgfqpoint{1.279755in}{1.064812in}}{\pgfqpoint{1.287991in}{1.064812in}}%
\pgfpathclose%
\pgfusepath{stroke,fill}%
\end{pgfscope}%
\begin{pgfscope}%
\pgfpathrectangle{\pgfqpoint{0.556847in}{0.516222in}}{\pgfqpoint{1.722590in}{1.783528in}} %
\pgfusepath{clip}%
\pgfsetbuttcap%
\pgfsetroundjoin%
\definecolor{currentfill}{rgb}{0.298039,0.447059,0.690196}%
\pgfsetfillcolor{currentfill}%
\pgfsetlinewidth{0.240900pt}%
\definecolor{currentstroke}{rgb}{1.000000,1.000000,1.000000}%
\pgfsetstrokecolor{currentstroke}%
\pgfsetdash{}{0pt}%
\pgfpathmoveto{\pgfqpoint{1.287991in}{1.015765in}}%
\pgfpathcurveto{\pgfqpoint{1.296227in}{1.015765in}}{\pgfqpoint{1.304127in}{1.019038in}}{\pgfqpoint{1.309951in}{1.024861in}}%
\pgfpathcurveto{\pgfqpoint{1.315775in}{1.030685in}}{\pgfqpoint{1.319048in}{1.038585in}}{\pgfqpoint{1.319048in}{1.046822in}}%
\pgfpathcurveto{\pgfqpoint{1.319048in}{1.055058in}}{\pgfqpoint{1.315775in}{1.062958in}}{\pgfqpoint{1.309951in}{1.068782in}}%
\pgfpathcurveto{\pgfqpoint{1.304127in}{1.074606in}}{\pgfqpoint{1.296227in}{1.077878in}}{\pgfqpoint{1.287991in}{1.077878in}}%
\pgfpathcurveto{\pgfqpoint{1.279755in}{1.077878in}}{\pgfqpoint{1.271855in}{1.074606in}}{\pgfqpoint{1.266031in}{1.068782in}}%
\pgfpathcurveto{\pgfqpoint{1.260207in}{1.062958in}}{\pgfqpoint{1.256935in}{1.055058in}}{\pgfqpoint{1.256935in}{1.046822in}}%
\pgfpathcurveto{\pgfqpoint{1.256935in}{1.038585in}}{\pgfqpoint{1.260207in}{1.030685in}}{\pgfqpoint{1.266031in}{1.024861in}}%
\pgfpathcurveto{\pgfqpoint{1.271855in}{1.019038in}}{\pgfqpoint{1.279755in}{1.015765in}}{\pgfqpoint{1.287991in}{1.015765in}}%
\pgfpathclose%
\pgfusepath{stroke,fill}%
\end{pgfscope}%
\begin{pgfscope}%
\pgfpathrectangle{\pgfqpoint{0.556847in}{0.516222in}}{\pgfqpoint{1.722590in}{1.783528in}} %
\pgfusepath{clip}%
\pgfsetbuttcap%
\pgfsetroundjoin%
\definecolor{currentfill}{rgb}{0.298039,0.447059,0.690196}%
\pgfsetfillcolor{currentfill}%
\pgfsetlinewidth{0.240900pt}%
\definecolor{currentstroke}{rgb}{1.000000,1.000000,1.000000}%
\pgfsetstrokecolor{currentstroke}%
\pgfsetdash{}{0pt}%
\pgfpathmoveto{\pgfqpoint{0.901365in}{0.953342in}}%
\pgfpathcurveto{\pgfqpoint{0.909602in}{0.953342in}}{\pgfqpoint{0.917502in}{0.956614in}}{\pgfqpoint{0.923326in}{0.962438in}}%
\pgfpathcurveto{\pgfqpoint{0.929149in}{0.968262in}}{\pgfqpoint{0.932422in}{0.976162in}}{\pgfqpoint{0.932422in}{0.984398in}}%
\pgfpathcurveto{\pgfqpoint{0.932422in}{0.992635in}}{\pgfqpoint{0.929149in}{1.000535in}}{\pgfqpoint{0.923326in}{1.006359in}}%
\pgfpathcurveto{\pgfqpoint{0.917502in}{1.012182in}}{\pgfqpoint{0.909602in}{1.015455in}}{\pgfqpoint{0.901365in}{1.015455in}}%
\pgfpathcurveto{\pgfqpoint{0.893129in}{1.015455in}}{\pgfqpoint{0.885229in}{1.012182in}}{\pgfqpoint{0.879405in}{1.006359in}}%
\pgfpathcurveto{\pgfqpoint{0.873581in}{1.000535in}}{\pgfqpoint{0.870309in}{0.992635in}}{\pgfqpoint{0.870309in}{0.984398in}}%
\pgfpathcurveto{\pgfqpoint{0.870309in}{0.976162in}}{\pgfqpoint{0.873581in}{0.968262in}}{\pgfqpoint{0.879405in}{0.962438in}}%
\pgfpathcurveto{\pgfqpoint{0.885229in}{0.956614in}}{\pgfqpoint{0.893129in}{0.953342in}}{\pgfqpoint{0.901365in}{0.953342in}}%
\pgfpathclose%
\pgfusepath{stroke,fill}%
\end{pgfscope}%
\begin{pgfscope}%
\pgfpathrectangle{\pgfqpoint{0.556847in}{0.516222in}}{\pgfqpoint{1.722590in}{1.783528in}} %
\pgfusepath{clip}%
\pgfsetbuttcap%
\pgfsetroundjoin%
\definecolor{currentfill}{rgb}{0.298039,0.447059,0.690196}%
\pgfsetfillcolor{currentfill}%
\pgfsetlinewidth{0.240900pt}%
\definecolor{currentstroke}{rgb}{1.000000,1.000000,1.000000}%
\pgfsetstrokecolor{currentstroke}%
\pgfsetdash{}{0pt}%
\pgfpathmoveto{\pgfqpoint{1.012377in}{0.574342in}}%
\pgfpathcurveto{\pgfqpoint{1.020613in}{0.574342in}}{\pgfqpoint{1.028513in}{0.577614in}}{\pgfqpoint{1.034337in}{0.583438in}}%
\pgfpathcurveto{\pgfqpoint{1.040161in}{0.589262in}}{\pgfqpoint{1.043433in}{0.597162in}}{\pgfqpoint{1.043433in}{0.605399in}}%
\pgfpathcurveto{\pgfqpoint{1.043433in}{0.613635in}}{\pgfqpoint{1.040161in}{0.621535in}}{\pgfqpoint{1.034337in}{0.627359in}}%
\pgfpathcurveto{\pgfqpoint{1.028513in}{0.633183in}}{\pgfqpoint{1.020613in}{0.636455in}}{\pgfqpoint{1.012377in}{0.636455in}}%
\pgfpathcurveto{\pgfqpoint{1.004140in}{0.636455in}}{\pgfqpoint{0.996240in}{0.633183in}}{\pgfqpoint{0.990416in}{0.627359in}}%
\pgfpathcurveto{\pgfqpoint{0.984592in}{0.621535in}}{\pgfqpoint{0.981320in}{0.613635in}}{\pgfqpoint{0.981320in}{0.605399in}}%
\pgfpathcurveto{\pgfqpoint{0.981320in}{0.597162in}}{\pgfqpoint{0.984592in}{0.589262in}}{\pgfqpoint{0.990416in}{0.583438in}}%
\pgfpathcurveto{\pgfqpoint{0.996240in}{0.577614in}}{\pgfqpoint{1.004140in}{0.574342in}}{\pgfqpoint{1.012377in}{0.574342in}}%
\pgfpathclose%
\pgfusepath{stroke,fill}%
\end{pgfscope}%
\begin{pgfscope}%
\pgfpathrectangle{\pgfqpoint{0.556847in}{0.516222in}}{\pgfqpoint{1.722590in}{1.783528in}} %
\pgfusepath{clip}%
\pgfsetbuttcap%
\pgfsetroundjoin%
\definecolor{currentfill}{rgb}{0.298039,0.447059,0.690196}%
\pgfsetfillcolor{currentfill}%
\pgfsetlinewidth{0.240900pt}%
\definecolor{currentstroke}{rgb}{1.000000,1.000000,1.000000}%
\pgfsetstrokecolor{currentstroke}%
\pgfsetdash{}{0pt}%
\pgfpathmoveto{\pgfqpoint{0.755902in}{0.873083in}}%
\pgfpathcurveto{\pgfqpoint{0.764138in}{0.873083in}}{\pgfqpoint{0.772038in}{0.876355in}}{\pgfqpoint{0.777862in}{0.882179in}}%
\pgfpathcurveto{\pgfqpoint{0.783686in}{0.888003in}}{\pgfqpoint{0.786959in}{0.895903in}}{\pgfqpoint{0.786959in}{0.904140in}}%
\pgfpathcurveto{\pgfqpoint{0.786959in}{0.912376in}}{\pgfqpoint{0.783686in}{0.920276in}}{\pgfqpoint{0.777862in}{0.926100in}}%
\pgfpathcurveto{\pgfqpoint{0.772038in}{0.931924in}}{\pgfqpoint{0.764138in}{0.935196in}}{\pgfqpoint{0.755902in}{0.935196in}}%
\pgfpathcurveto{\pgfqpoint{0.747666in}{0.935196in}}{\pgfqpoint{0.739766in}{0.931924in}}{\pgfqpoint{0.733942in}{0.926100in}}%
\pgfpathcurveto{\pgfqpoint{0.728118in}{0.920276in}}{\pgfqpoint{0.724846in}{0.912376in}}{\pgfqpoint{0.724846in}{0.904140in}}%
\pgfpathcurveto{\pgfqpoint{0.724846in}{0.895903in}}{\pgfqpoint{0.728118in}{0.888003in}}{\pgfqpoint{0.733942in}{0.882179in}}%
\pgfpathcurveto{\pgfqpoint{0.739766in}{0.876355in}}{\pgfqpoint{0.747666in}{0.873083in}}{\pgfqpoint{0.755902in}{0.873083in}}%
\pgfpathclose%
\pgfusepath{stroke,fill}%
\end{pgfscope}%
\begin{pgfscope}%
\pgfpathrectangle{\pgfqpoint{0.556847in}{0.516222in}}{\pgfqpoint{1.722590in}{1.783528in}} %
\pgfusepath{clip}%
\pgfsetbuttcap%
\pgfsetroundjoin%
\definecolor{currentfill}{rgb}{0.298039,0.447059,0.690196}%
\pgfsetfillcolor{currentfill}%
\pgfsetlinewidth{0.240900pt}%
\definecolor{currentstroke}{rgb}{1.000000,1.000000,1.000000}%
\pgfsetstrokecolor{currentstroke}%
\pgfsetdash{}{0pt}%
\pgfpathmoveto{\pgfqpoint{0.985581in}{1.087106in}}%
\pgfpathcurveto{\pgfqpoint{0.993817in}{1.087106in}}{\pgfqpoint{1.001717in}{1.090379in}}{\pgfqpoint{1.007541in}{1.096203in}}%
\pgfpathcurveto{\pgfqpoint{1.013365in}{1.102027in}}{\pgfqpoint{1.016637in}{1.109927in}}{\pgfqpoint{1.016637in}{1.118163in}}%
\pgfpathcurveto{\pgfqpoint{1.016637in}{1.126399in}}{\pgfqpoint{1.013365in}{1.134299in}}{\pgfqpoint{1.007541in}{1.140123in}}%
\pgfpathcurveto{\pgfqpoint{1.001717in}{1.145947in}}{\pgfqpoint{0.993817in}{1.149219in}}{\pgfqpoint{0.985581in}{1.149219in}}%
\pgfpathcurveto{\pgfqpoint{0.977345in}{1.149219in}}{\pgfqpoint{0.969444in}{1.145947in}}{\pgfqpoint{0.963621in}{1.140123in}}%
\pgfpathcurveto{\pgfqpoint{0.957797in}{1.134299in}}{\pgfqpoint{0.954524in}{1.126399in}}{\pgfqpoint{0.954524in}{1.118163in}}%
\pgfpathcurveto{\pgfqpoint{0.954524in}{1.109927in}}{\pgfqpoint{0.957797in}{1.102027in}}{\pgfqpoint{0.963621in}{1.096203in}}%
\pgfpathcurveto{\pgfqpoint{0.969444in}{1.090379in}}{\pgfqpoint{0.977345in}{1.087106in}}{\pgfqpoint{0.985581in}{1.087106in}}%
\pgfpathclose%
\pgfusepath{stroke,fill}%
\end{pgfscope}%
\begin{pgfscope}%
\pgfpathrectangle{\pgfqpoint{0.556847in}{0.516222in}}{\pgfqpoint{1.722590in}{1.783528in}} %
\pgfusepath{clip}%
\pgfsetbuttcap%
\pgfsetroundjoin%
\definecolor{currentfill}{rgb}{0.298039,0.447059,0.690196}%
\pgfsetfillcolor{currentfill}%
\pgfsetlinewidth{0.240900pt}%
\definecolor{currentstroke}{rgb}{1.000000,1.000000,1.000000}%
\pgfsetstrokecolor{currentstroke}%
\pgfsetdash{}{0pt}%
\pgfpathmoveto{\pgfqpoint{1.460250in}{1.376930in}}%
\pgfpathcurveto{\pgfqpoint{1.468486in}{1.376930in}}{\pgfqpoint{1.476386in}{1.380202in}}{\pgfqpoint{1.482210in}{1.386026in}}%
\pgfpathcurveto{\pgfqpoint{1.488034in}{1.391850in}}{\pgfqpoint{1.491307in}{1.399750in}}{\pgfqpoint{1.491307in}{1.407986in}}%
\pgfpathcurveto{\pgfqpoint{1.491307in}{1.416222in}}{\pgfqpoint{1.488034in}{1.424122in}}{\pgfqpoint{1.482210in}{1.429946in}}%
\pgfpathcurveto{\pgfqpoint{1.476386in}{1.435770in}}{\pgfqpoint{1.468486in}{1.439043in}}{\pgfqpoint{1.460250in}{1.439043in}}%
\pgfpathcurveto{\pgfqpoint{1.452014in}{1.439043in}}{\pgfqpoint{1.444114in}{1.435770in}}{\pgfqpoint{1.438290in}{1.429946in}}%
\pgfpathcurveto{\pgfqpoint{1.432466in}{1.424122in}}{\pgfqpoint{1.429194in}{1.416222in}}{\pgfqpoint{1.429194in}{1.407986in}}%
\pgfpathcurveto{\pgfqpoint{1.429194in}{1.399750in}}{\pgfqpoint{1.432466in}{1.391850in}}{\pgfqpoint{1.438290in}{1.386026in}}%
\pgfpathcurveto{\pgfqpoint{1.444114in}{1.380202in}}{\pgfqpoint{1.452014in}{1.376930in}}{\pgfqpoint{1.460250in}{1.376930in}}%
\pgfpathclose%
\pgfusepath{stroke,fill}%
\end{pgfscope}%
\begin{pgfscope}%
\pgfpathrectangle{\pgfqpoint{0.556847in}{0.516222in}}{\pgfqpoint{1.722590in}{1.783528in}} %
\pgfusepath{clip}%
\pgfsetbuttcap%
\pgfsetroundjoin%
\definecolor{currentfill}{rgb}{0.298039,0.447059,0.690196}%
\pgfsetfillcolor{currentfill}%
\pgfsetlinewidth{0.240900pt}%
\definecolor{currentstroke}{rgb}{1.000000,1.000000,1.000000}%
\pgfsetstrokecolor{currentstroke}%
\pgfsetdash{}{0pt}%
\pgfpathmoveto{\pgfqpoint{1.850704in}{1.372471in}}%
\pgfpathcurveto{\pgfqpoint{1.858940in}{1.372471in}}{\pgfqpoint{1.866840in}{1.375743in}}{\pgfqpoint{1.872664in}{1.381567in}}%
\pgfpathcurveto{\pgfqpoint{1.878488in}{1.387391in}}{\pgfqpoint{1.881760in}{1.395291in}}{\pgfqpoint{1.881760in}{1.403527in}}%
\pgfpathcurveto{\pgfqpoint{1.881760in}{1.411764in}}{\pgfqpoint{1.878488in}{1.419664in}}{\pgfqpoint{1.872664in}{1.425488in}}%
\pgfpathcurveto{\pgfqpoint{1.866840in}{1.431311in}}{\pgfqpoint{1.858940in}{1.434584in}}{\pgfqpoint{1.850704in}{1.434584in}}%
\pgfpathcurveto{\pgfqpoint{1.842468in}{1.434584in}}{\pgfqpoint{1.834568in}{1.431311in}}{\pgfqpoint{1.828744in}{1.425488in}}%
\pgfpathcurveto{\pgfqpoint{1.822920in}{1.419664in}}{\pgfqpoint{1.819647in}{1.411764in}}{\pgfqpoint{1.819647in}{1.403527in}}%
\pgfpathcurveto{\pgfqpoint{1.819647in}{1.395291in}}{\pgfqpoint{1.822920in}{1.387391in}}{\pgfqpoint{1.828744in}{1.381567in}}%
\pgfpathcurveto{\pgfqpoint{1.834568in}{1.375743in}}{\pgfqpoint{1.842468in}{1.372471in}}{\pgfqpoint{1.850704in}{1.372471in}}%
\pgfpathclose%
\pgfusepath{stroke,fill}%
\end{pgfscope}%
\begin{pgfscope}%
\pgfpathrectangle{\pgfqpoint{0.556847in}{0.516222in}}{\pgfqpoint{1.722590in}{1.783528in}} %
\pgfusepath{clip}%
\pgfsetbuttcap%
\pgfsetroundjoin%
\definecolor{currentfill}{rgb}{0.298039,0.447059,0.690196}%
\pgfsetfillcolor{currentfill}%
\pgfsetlinewidth{0.240900pt}%
\definecolor{currentstroke}{rgb}{1.000000,1.000000,1.000000}%
\pgfsetstrokecolor{currentstroke}%
\pgfsetdash{}{0pt}%
\pgfpathmoveto{\pgfqpoint{1.150184in}{0.788365in}}%
\pgfpathcurveto{\pgfqpoint{1.158420in}{0.788365in}}{\pgfqpoint{1.166320in}{0.791638in}}{\pgfqpoint{1.172144in}{0.797462in}}%
\pgfpathcurveto{\pgfqpoint{1.177968in}{0.803286in}}{\pgfqpoint{1.181240in}{0.811186in}}{\pgfqpoint{1.181240in}{0.819422in}}%
\pgfpathcurveto{\pgfqpoint{1.181240in}{0.827658in}}{\pgfqpoint{1.177968in}{0.835558in}}{\pgfqpoint{1.172144in}{0.841382in}}%
\pgfpathcurveto{\pgfqpoint{1.166320in}{0.847206in}}{\pgfqpoint{1.158420in}{0.850478in}}{\pgfqpoint{1.150184in}{0.850478in}}%
\pgfpathcurveto{\pgfqpoint{1.141948in}{0.850478in}}{\pgfqpoint{1.134048in}{0.847206in}}{\pgfqpoint{1.128224in}{0.841382in}}%
\pgfpathcurveto{\pgfqpoint{1.122400in}{0.835558in}}{\pgfqpoint{1.119127in}{0.827658in}}{\pgfqpoint{1.119127in}{0.819422in}}%
\pgfpathcurveto{\pgfqpoint{1.119127in}{0.811186in}}{\pgfqpoint{1.122400in}{0.803286in}}{\pgfqpoint{1.128224in}{0.797462in}}%
\pgfpathcurveto{\pgfqpoint{1.134048in}{0.791638in}}{\pgfqpoint{1.141948in}{0.788365in}}{\pgfqpoint{1.150184in}{0.788365in}}%
\pgfpathclose%
\pgfusepath{stroke,fill}%
\end{pgfscope}%
\begin{pgfscope}%
\pgfpathrectangle{\pgfqpoint{0.556847in}{0.516222in}}{\pgfqpoint{1.722590in}{1.783528in}} %
\pgfusepath{clip}%
\pgfsetbuttcap%
\pgfsetroundjoin%
\definecolor{currentfill}{rgb}{0.298039,0.447059,0.690196}%
\pgfsetfillcolor{currentfill}%
\pgfsetlinewidth{0.240900pt}%
\definecolor{currentstroke}{rgb}{1.000000,1.000000,1.000000}%
\pgfsetstrokecolor{currentstroke}%
\pgfsetdash{}{0pt}%
\pgfpathmoveto{\pgfqpoint{1.154012in}{1.176283in}}%
\pgfpathcurveto{\pgfqpoint{1.162248in}{1.176283in}}{\pgfqpoint{1.170148in}{1.179555in}}{\pgfqpoint{1.175972in}{1.185379in}}%
\pgfpathcurveto{\pgfqpoint{1.181796in}{1.191203in}}{\pgfqpoint{1.185068in}{1.199103in}}{\pgfqpoint{1.185068in}{1.207339in}}%
\pgfpathcurveto{\pgfqpoint{1.185068in}{1.215576in}}{\pgfqpoint{1.181796in}{1.223476in}}{\pgfqpoint{1.175972in}{1.229299in}}%
\pgfpathcurveto{\pgfqpoint{1.170148in}{1.235123in}}{\pgfqpoint{1.162248in}{1.238396in}}{\pgfqpoint{1.154012in}{1.238396in}}%
\pgfpathcurveto{\pgfqpoint{1.145776in}{1.238396in}}{\pgfqpoint{1.137876in}{1.235123in}}{\pgfqpoint{1.132052in}{1.229299in}}%
\pgfpathcurveto{\pgfqpoint{1.126228in}{1.223476in}}{\pgfqpoint{1.122955in}{1.215576in}}{\pgfqpoint{1.122955in}{1.207339in}}%
\pgfpathcurveto{\pgfqpoint{1.122955in}{1.199103in}}{\pgfqpoint{1.126228in}{1.191203in}}{\pgfqpoint{1.132052in}{1.185379in}}%
\pgfpathcurveto{\pgfqpoint{1.137876in}{1.179555in}}{\pgfqpoint{1.145776in}{1.176283in}}{\pgfqpoint{1.154012in}{1.176283in}}%
\pgfpathclose%
\pgfusepath{stroke,fill}%
\end{pgfscope}%
\begin{pgfscope}%
\pgfpathrectangle{\pgfqpoint{0.556847in}{0.516222in}}{\pgfqpoint{1.722590in}{1.783528in}} %
\pgfusepath{clip}%
\pgfsetbuttcap%
\pgfsetroundjoin%
\definecolor{currentfill}{rgb}{0.298039,0.447059,0.690196}%
\pgfsetfillcolor{currentfill}%
\pgfsetlinewidth{0.240900pt}%
\definecolor{currentstroke}{rgb}{1.000000,1.000000,1.000000}%
\pgfsetstrokecolor{currentstroke}%
\pgfsetdash{}{0pt}%
\pgfpathmoveto{\pgfqpoint{0.790354in}{0.717024in}}%
\pgfpathcurveto{\pgfqpoint{0.798590in}{0.717024in}}{\pgfqpoint{0.806490in}{0.720297in}}{\pgfqpoint{0.812314in}{0.726121in}}%
\pgfpathcurveto{\pgfqpoint{0.818138in}{0.731945in}}{\pgfqpoint{0.821410in}{0.739845in}}{\pgfqpoint{0.821410in}{0.748081in}}%
\pgfpathcurveto{\pgfqpoint{0.821410in}{0.756317in}}{\pgfqpoint{0.818138in}{0.764217in}}{\pgfqpoint{0.812314in}{0.770041in}}%
\pgfpathcurveto{\pgfqpoint{0.806490in}{0.775865in}}{\pgfqpoint{0.798590in}{0.779137in}}{\pgfqpoint{0.790354in}{0.779137in}}%
\pgfpathcurveto{\pgfqpoint{0.782118in}{0.779137in}}{\pgfqpoint{0.774218in}{0.775865in}}{\pgfqpoint{0.768394in}{0.770041in}}%
\pgfpathcurveto{\pgfqpoint{0.762570in}{0.764217in}}{\pgfqpoint{0.759297in}{0.756317in}}{\pgfqpoint{0.759297in}{0.748081in}}%
\pgfpathcurveto{\pgfqpoint{0.759297in}{0.739845in}}{\pgfqpoint{0.762570in}{0.731945in}}{\pgfqpoint{0.768394in}{0.726121in}}%
\pgfpathcurveto{\pgfqpoint{0.774218in}{0.720297in}}{\pgfqpoint{0.782118in}{0.717024in}}{\pgfqpoint{0.790354in}{0.717024in}}%
\pgfpathclose%
\pgfusepath{stroke,fill}%
\end{pgfscope}%
\begin{pgfscope}%
\pgfpathrectangle{\pgfqpoint{0.556847in}{0.516222in}}{\pgfqpoint{1.722590in}{1.783528in}} %
\pgfusepath{clip}%
\pgfsetbuttcap%
\pgfsetroundjoin%
\definecolor{currentfill}{rgb}{0.298039,0.447059,0.690196}%
\pgfsetfillcolor{currentfill}%
\pgfsetlinewidth{0.240900pt}%
\definecolor{currentstroke}{rgb}{1.000000,1.000000,1.000000}%
\pgfsetstrokecolor{currentstroke}%
\pgfsetdash{}{0pt}%
\pgfpathmoveto{\pgfqpoint{1.487046in}{1.403683in}}%
\pgfpathcurveto{\pgfqpoint{1.495282in}{1.403683in}}{\pgfqpoint{1.503182in}{1.406955in}}{\pgfqpoint{1.509006in}{1.412779in}}%
\pgfpathcurveto{\pgfqpoint{1.514830in}{1.418603in}}{\pgfqpoint{1.518102in}{1.426503in}}{\pgfqpoint{1.518102in}{1.434739in}}%
\pgfpathcurveto{\pgfqpoint{1.518102in}{1.442975in}}{\pgfqpoint{1.514830in}{1.450875in}}{\pgfqpoint{1.509006in}{1.456699in}}%
\pgfpathcurveto{\pgfqpoint{1.503182in}{1.462523in}}{\pgfqpoint{1.495282in}{1.465796in}}{\pgfqpoint{1.487046in}{1.465796in}}%
\pgfpathcurveto{\pgfqpoint{1.478810in}{1.465796in}}{\pgfqpoint{1.470910in}{1.462523in}}{\pgfqpoint{1.465086in}{1.456699in}}%
\pgfpathcurveto{\pgfqpoint{1.459262in}{1.450875in}}{\pgfqpoint{1.455989in}{1.442975in}}{\pgfqpoint{1.455989in}{1.434739in}}%
\pgfpathcurveto{\pgfqpoint{1.455989in}{1.426503in}}{\pgfqpoint{1.459262in}{1.418603in}}{\pgfqpoint{1.465086in}{1.412779in}}%
\pgfpathcurveto{\pgfqpoint{1.470910in}{1.406955in}}{\pgfqpoint{1.478810in}{1.403683in}}{\pgfqpoint{1.487046in}{1.403683in}}%
\pgfpathclose%
\pgfusepath{stroke,fill}%
\end{pgfscope}%
\begin{pgfscope}%
\pgfpathrectangle{\pgfqpoint{0.556847in}{0.516222in}}{\pgfqpoint{1.722590in}{1.783528in}} %
\pgfusepath{clip}%
\pgfsetbuttcap%
\pgfsetroundjoin%
\definecolor{currentfill}{rgb}{0.298039,0.447059,0.690196}%
\pgfsetfillcolor{currentfill}%
\pgfsetlinewidth{0.240900pt}%
\definecolor{currentstroke}{rgb}{1.000000,1.000000,1.000000}%
\pgfsetstrokecolor{currentstroke}%
\pgfsetdash{}{0pt}%
\pgfpathmoveto{\pgfqpoint{1.161668in}{0.873083in}}%
\pgfpathcurveto{\pgfqpoint{1.169904in}{0.873083in}}{\pgfqpoint{1.177804in}{0.876355in}}{\pgfqpoint{1.183628in}{0.882179in}}%
\pgfpathcurveto{\pgfqpoint{1.189452in}{0.888003in}}{\pgfqpoint{1.192724in}{0.895903in}}{\pgfqpoint{1.192724in}{0.904140in}}%
\pgfpathcurveto{\pgfqpoint{1.192724in}{0.912376in}}{\pgfqpoint{1.189452in}{0.920276in}}{\pgfqpoint{1.183628in}{0.926100in}}%
\pgfpathcurveto{\pgfqpoint{1.177804in}{0.931924in}}{\pgfqpoint{1.169904in}{0.935196in}}{\pgfqpoint{1.161668in}{0.935196in}}%
\pgfpathcurveto{\pgfqpoint{1.153432in}{0.935196in}}{\pgfqpoint{1.145531in}{0.931924in}}{\pgfqpoint{1.139708in}{0.926100in}}%
\pgfpathcurveto{\pgfqpoint{1.133884in}{0.920276in}}{\pgfqpoint{1.130611in}{0.912376in}}{\pgfqpoint{1.130611in}{0.904140in}}%
\pgfpathcurveto{\pgfqpoint{1.130611in}{0.895903in}}{\pgfqpoint{1.133884in}{0.888003in}}{\pgfqpoint{1.139708in}{0.882179in}}%
\pgfpathcurveto{\pgfqpoint{1.145531in}{0.876355in}}{\pgfqpoint{1.153432in}{0.873083in}}{\pgfqpoint{1.161668in}{0.873083in}}%
\pgfpathclose%
\pgfusepath{stroke,fill}%
\end{pgfscope}%
\begin{pgfscope}%
\pgfpathrectangle{\pgfqpoint{0.556847in}{0.516222in}}{\pgfqpoint{1.722590in}{1.783528in}} %
\pgfusepath{clip}%
\pgfsetbuttcap%
\pgfsetroundjoin%
\definecolor{currentfill}{rgb}{0.298039,0.447059,0.690196}%
\pgfsetfillcolor{currentfill}%
\pgfsetlinewidth{0.240900pt}%
\definecolor{currentstroke}{rgb}{1.000000,1.000000,1.000000}%
\pgfsetstrokecolor{currentstroke}%
\pgfsetdash{}{0pt}%
\pgfpathmoveto{\pgfqpoint{0.721450in}{0.971177in}}%
\pgfpathcurveto{\pgfqpoint{0.729687in}{0.971177in}}{\pgfqpoint{0.737587in}{0.974449in}}{\pgfqpoint{0.743411in}{0.980273in}}%
\pgfpathcurveto{\pgfqpoint{0.749234in}{0.986097in}}{\pgfqpoint{0.752507in}{0.993997in}}{\pgfqpoint{0.752507in}{1.002234in}}%
\pgfpathcurveto{\pgfqpoint{0.752507in}{1.010470in}}{\pgfqpoint{0.749234in}{1.018370in}}{\pgfqpoint{0.743411in}{1.024194in}}%
\pgfpathcurveto{\pgfqpoint{0.737587in}{1.030018in}}{\pgfqpoint{0.729687in}{1.033290in}}{\pgfqpoint{0.721450in}{1.033290in}}%
\pgfpathcurveto{\pgfqpoint{0.713214in}{1.033290in}}{\pgfqpoint{0.705314in}{1.030018in}}{\pgfqpoint{0.699490in}{1.024194in}}%
\pgfpathcurveto{\pgfqpoint{0.693666in}{1.018370in}}{\pgfqpoint{0.690394in}{1.010470in}}{\pgfqpoint{0.690394in}{1.002234in}}%
\pgfpathcurveto{\pgfqpoint{0.690394in}{0.993997in}}{\pgfqpoint{0.693666in}{0.986097in}}{\pgfqpoint{0.699490in}{0.980273in}}%
\pgfpathcurveto{\pgfqpoint{0.705314in}{0.974449in}}{\pgfqpoint{0.713214in}{0.971177in}}{\pgfqpoint{0.721450in}{0.971177in}}%
\pgfpathclose%
\pgfusepath{stroke,fill}%
\end{pgfscope}%
\begin{pgfscope}%
\pgfpathrectangle{\pgfqpoint{0.556847in}{0.516222in}}{\pgfqpoint{1.722590in}{1.783528in}} %
\pgfusepath{clip}%
\pgfsetbuttcap%
\pgfsetroundjoin%
\definecolor{currentfill}{rgb}{0.298039,0.447059,0.690196}%
\pgfsetfillcolor{currentfill}%
\pgfsetlinewidth{0.240900pt}%
\definecolor{currentstroke}{rgb}{1.000000,1.000000,1.000000}%
\pgfsetstrokecolor{currentstroke}%
\pgfsetdash{}{0pt}%
\pgfpathmoveto{\pgfqpoint{1.008549in}{1.403683in}}%
\pgfpathcurveto{\pgfqpoint{1.016785in}{1.403683in}}{\pgfqpoint{1.024685in}{1.406955in}}{\pgfqpoint{1.030509in}{1.412779in}}%
\pgfpathcurveto{\pgfqpoint{1.036333in}{1.418603in}}{\pgfqpoint{1.039605in}{1.426503in}}{\pgfqpoint{1.039605in}{1.434739in}}%
\pgfpathcurveto{\pgfqpoint{1.039605in}{1.442975in}}{\pgfqpoint{1.036333in}{1.450875in}}{\pgfqpoint{1.030509in}{1.456699in}}%
\pgfpathcurveto{\pgfqpoint{1.024685in}{1.462523in}}{\pgfqpoint{1.016785in}{1.465796in}}{\pgfqpoint{1.008549in}{1.465796in}}%
\pgfpathcurveto{\pgfqpoint{1.000312in}{1.465796in}}{\pgfqpoint{0.992412in}{1.462523in}}{\pgfqpoint{0.986588in}{1.456699in}}%
\pgfpathcurveto{\pgfqpoint{0.980764in}{1.450875in}}{\pgfqpoint{0.977492in}{1.442975in}}{\pgfqpoint{0.977492in}{1.434739in}}%
\pgfpathcurveto{\pgfqpoint{0.977492in}{1.426503in}}{\pgfqpoint{0.980764in}{1.418603in}}{\pgfqpoint{0.986588in}{1.412779in}}%
\pgfpathcurveto{\pgfqpoint{0.992412in}{1.406955in}}{\pgfqpoint{1.000312in}{1.403683in}}{\pgfqpoint{1.008549in}{1.403683in}}%
\pgfpathclose%
\pgfusepath{stroke,fill}%
\end{pgfscope}%
\begin{pgfscope}%
\pgfpathrectangle{\pgfqpoint{0.556847in}{0.516222in}}{\pgfqpoint{1.722590in}{1.783528in}} %
\pgfusepath{clip}%
\pgfsetbuttcap%
\pgfsetroundjoin%
\definecolor{currentfill}{rgb}{0.298039,0.447059,0.690196}%
\pgfsetfillcolor{currentfill}%
\pgfsetlinewidth{0.240900pt}%
\definecolor{currentstroke}{rgb}{1.000000,1.000000,1.000000}%
\pgfsetstrokecolor{currentstroke}%
\pgfsetdash{}{0pt}%
\pgfpathmoveto{\pgfqpoint{1.663133in}{1.537447in}}%
\pgfpathcurveto{\pgfqpoint{1.671369in}{1.537447in}}{\pgfqpoint{1.679269in}{1.540719in}}{\pgfqpoint{1.685093in}{1.546543in}}%
\pgfpathcurveto{\pgfqpoint{1.690917in}{1.552367in}}{\pgfqpoint{1.694189in}{1.560267in}}{\pgfqpoint{1.694189in}{1.568504in}}%
\pgfpathcurveto{\pgfqpoint{1.694189in}{1.576740in}}{\pgfqpoint{1.690917in}{1.584640in}}{\pgfqpoint{1.685093in}{1.590464in}}%
\pgfpathcurveto{\pgfqpoint{1.679269in}{1.596288in}}{\pgfqpoint{1.671369in}{1.599560in}}{\pgfqpoint{1.663133in}{1.599560in}}%
\pgfpathcurveto{\pgfqpoint{1.654897in}{1.599560in}}{\pgfqpoint{1.646997in}{1.596288in}}{\pgfqpoint{1.641173in}{1.590464in}}%
\pgfpathcurveto{\pgfqpoint{1.635349in}{1.584640in}}{\pgfqpoint{1.632076in}{1.576740in}}{\pgfqpoint{1.632076in}{1.568504in}}%
\pgfpathcurveto{\pgfqpoint{1.632076in}{1.560267in}}{\pgfqpoint{1.635349in}{1.552367in}}{\pgfqpoint{1.641173in}{1.546543in}}%
\pgfpathcurveto{\pgfqpoint{1.646997in}{1.540719in}}{\pgfqpoint{1.654897in}{1.537447in}}{\pgfqpoint{1.663133in}{1.537447in}}%
\pgfpathclose%
\pgfusepath{stroke,fill}%
\end{pgfscope}%
\begin{pgfscope}%
\pgfpathrectangle{\pgfqpoint{0.556847in}{0.516222in}}{\pgfqpoint{1.722590in}{1.783528in}} %
\pgfusepath{clip}%
\pgfsetbuttcap%
\pgfsetroundjoin%
\definecolor{currentfill}{rgb}{0.298039,0.447059,0.690196}%
\pgfsetfillcolor{currentfill}%
\pgfsetlinewidth{0.240900pt}%
\definecolor{currentstroke}{rgb}{1.000000,1.000000,1.000000}%
\pgfsetstrokecolor{currentstroke}%
\pgfsetdash{}{0pt}%
\pgfpathmoveto{\pgfqpoint{0.870741in}{0.877542in}}%
\pgfpathcurveto{\pgfqpoint{0.878978in}{0.877542in}}{\pgfqpoint{0.886878in}{0.880814in}}{\pgfqpoint{0.892702in}{0.886638in}}%
\pgfpathcurveto{\pgfqpoint{0.898526in}{0.892462in}}{\pgfqpoint{0.901798in}{0.900362in}}{\pgfqpoint{0.901798in}{0.908598in}}%
\pgfpathcurveto{\pgfqpoint{0.901798in}{0.916835in}}{\pgfqpoint{0.898526in}{0.924735in}}{\pgfqpoint{0.892702in}{0.930559in}}%
\pgfpathcurveto{\pgfqpoint{0.886878in}{0.936383in}}{\pgfqpoint{0.878978in}{0.939655in}}{\pgfqpoint{0.870741in}{0.939655in}}%
\pgfpathcurveto{\pgfqpoint{0.862505in}{0.939655in}}{\pgfqpoint{0.854605in}{0.936383in}}{\pgfqpoint{0.848781in}{0.930559in}}%
\pgfpathcurveto{\pgfqpoint{0.842957in}{0.924735in}}{\pgfqpoint{0.839685in}{0.916835in}}{\pgfqpoint{0.839685in}{0.908598in}}%
\pgfpathcurveto{\pgfqpoint{0.839685in}{0.900362in}}{\pgfqpoint{0.842957in}{0.892462in}}{\pgfqpoint{0.848781in}{0.886638in}}%
\pgfpathcurveto{\pgfqpoint{0.854605in}{0.880814in}}{\pgfqpoint{0.862505in}{0.877542in}}{\pgfqpoint{0.870741in}{0.877542in}}%
\pgfpathclose%
\pgfusepath{stroke,fill}%
\end{pgfscope}%
\begin{pgfscope}%
\pgfpathrectangle{\pgfqpoint{0.556847in}{0.516222in}}{\pgfqpoint{1.722590in}{1.783528in}} %
\pgfusepath{clip}%
\pgfsetbuttcap%
\pgfsetroundjoin%
\definecolor{currentfill}{rgb}{0.298039,0.447059,0.690196}%
\pgfsetfillcolor{currentfill}%
\pgfsetlinewidth{0.240900pt}%
\definecolor{currentstroke}{rgb}{1.000000,1.000000,1.000000}%
\pgfsetstrokecolor{currentstroke}%
\pgfsetdash{}{0pt}%
\pgfpathmoveto{\pgfqpoint{0.843946in}{0.815118in}}%
\pgfpathcurveto{\pgfqpoint{0.852182in}{0.815118in}}{\pgfqpoint{0.860082in}{0.818391in}}{\pgfqpoint{0.865906in}{0.824215in}}%
\pgfpathcurveto{\pgfqpoint{0.871730in}{0.830039in}}{\pgfqpoint{0.875002in}{0.837939in}}{\pgfqpoint{0.875002in}{0.846175in}}%
\pgfpathcurveto{\pgfqpoint{0.875002in}{0.854411in}}{\pgfqpoint{0.871730in}{0.862311in}}{\pgfqpoint{0.865906in}{0.868135in}}%
\pgfpathcurveto{\pgfqpoint{0.860082in}{0.873959in}}{\pgfqpoint{0.852182in}{0.877231in}}{\pgfqpoint{0.843946in}{0.877231in}}%
\pgfpathcurveto{\pgfqpoint{0.835709in}{0.877231in}}{\pgfqpoint{0.827809in}{0.873959in}}{\pgfqpoint{0.821985in}{0.868135in}}%
\pgfpathcurveto{\pgfqpoint{0.816161in}{0.862311in}}{\pgfqpoint{0.812889in}{0.854411in}}{\pgfqpoint{0.812889in}{0.846175in}}%
\pgfpathcurveto{\pgfqpoint{0.812889in}{0.837939in}}{\pgfqpoint{0.816161in}{0.830039in}}{\pgfqpoint{0.821985in}{0.824215in}}%
\pgfpathcurveto{\pgfqpoint{0.827809in}{0.818391in}}{\pgfqpoint{0.835709in}{0.815118in}}{\pgfqpoint{0.843946in}{0.815118in}}%
\pgfpathclose%
\pgfusepath{stroke,fill}%
\end{pgfscope}%
\begin{pgfscope}%
\pgfpathrectangle{\pgfqpoint{0.556847in}{0.516222in}}{\pgfqpoint{1.722590in}{1.783528in}} %
\pgfusepath{clip}%
\pgfsetbuttcap%
\pgfsetroundjoin%
\definecolor{currentfill}{rgb}{0.298039,0.447059,0.690196}%
\pgfsetfillcolor{currentfill}%
\pgfsetlinewidth{0.240900pt}%
\definecolor{currentstroke}{rgb}{1.000000,1.000000,1.000000}%
\pgfsetstrokecolor{currentstroke}%
\pgfsetdash{}{0pt}%
\pgfpathmoveto{\pgfqpoint{1.261195in}{0.993471in}}%
\pgfpathcurveto{\pgfqpoint{1.269432in}{0.993471in}}{\pgfqpoint{1.277332in}{0.996743in}}{\pgfqpoint{1.283156in}{1.002567in}}%
\pgfpathcurveto{\pgfqpoint{1.288979in}{1.008391in}}{\pgfqpoint{1.292252in}{1.016291in}}{\pgfqpoint{1.292252in}{1.024528in}}%
\pgfpathcurveto{\pgfqpoint{1.292252in}{1.032764in}}{\pgfqpoint{1.288979in}{1.040664in}}{\pgfqpoint{1.283156in}{1.046488in}}%
\pgfpathcurveto{\pgfqpoint{1.277332in}{1.052312in}}{\pgfqpoint{1.269432in}{1.055584in}}{\pgfqpoint{1.261195in}{1.055584in}}%
\pgfpathcurveto{\pgfqpoint{1.252959in}{1.055584in}}{\pgfqpoint{1.245059in}{1.052312in}}{\pgfqpoint{1.239235in}{1.046488in}}%
\pgfpathcurveto{\pgfqpoint{1.233411in}{1.040664in}}{\pgfqpoint{1.230139in}{1.032764in}}{\pgfqpoint{1.230139in}{1.024528in}}%
\pgfpathcurveto{\pgfqpoint{1.230139in}{1.016291in}}{\pgfqpoint{1.233411in}{1.008391in}}{\pgfqpoint{1.239235in}{1.002567in}}%
\pgfpathcurveto{\pgfqpoint{1.245059in}{0.996743in}}{\pgfqpoint{1.252959in}{0.993471in}}{\pgfqpoint{1.261195in}{0.993471in}}%
\pgfpathclose%
\pgfusepath{stroke,fill}%
\end{pgfscope}%
\begin{pgfscope}%
\pgfpathrectangle{\pgfqpoint{0.556847in}{0.516222in}}{\pgfqpoint{1.722590in}{1.783528in}} %
\pgfusepath{clip}%
\pgfsetbuttcap%
\pgfsetroundjoin%
\definecolor{currentfill}{rgb}{0.298039,0.447059,0.690196}%
\pgfsetfillcolor{currentfill}%
\pgfsetlinewidth{0.240900pt}%
\definecolor{currentstroke}{rgb}{1.000000,1.000000,1.000000}%
\pgfsetstrokecolor{currentstroke}%
\pgfsetdash{}{0pt}%
\pgfpathmoveto{\pgfqpoint{1.115732in}{0.819577in}}%
\pgfpathcurveto{\pgfqpoint{1.123968in}{0.819577in}}{\pgfqpoint{1.131868in}{0.822849in}}{\pgfqpoint{1.137692in}{0.828673in}}%
\pgfpathcurveto{\pgfqpoint{1.143516in}{0.834497in}}{\pgfqpoint{1.146789in}{0.842397in}}{\pgfqpoint{1.146789in}{0.850634in}}%
\pgfpathcurveto{\pgfqpoint{1.146789in}{0.858870in}}{\pgfqpoint{1.143516in}{0.866770in}}{\pgfqpoint{1.137692in}{0.872594in}}%
\pgfpathcurveto{\pgfqpoint{1.131868in}{0.878418in}}{\pgfqpoint{1.123968in}{0.881690in}}{\pgfqpoint{1.115732in}{0.881690in}}%
\pgfpathcurveto{\pgfqpoint{1.107496in}{0.881690in}}{\pgfqpoint{1.099596in}{0.878418in}}{\pgfqpoint{1.093772in}{0.872594in}}%
\pgfpathcurveto{\pgfqpoint{1.087948in}{0.866770in}}{\pgfqpoint{1.084676in}{0.858870in}}{\pgfqpoint{1.084676in}{0.850634in}}%
\pgfpathcurveto{\pgfqpoint{1.084676in}{0.842397in}}{\pgfqpoint{1.087948in}{0.834497in}}{\pgfqpoint{1.093772in}{0.828673in}}%
\pgfpathcurveto{\pgfqpoint{1.099596in}{0.822849in}}{\pgfqpoint{1.107496in}{0.819577in}}{\pgfqpoint{1.115732in}{0.819577in}}%
\pgfpathclose%
\pgfusepath{stroke,fill}%
\end{pgfscope}%
\begin{pgfscope}%
\pgfpathrectangle{\pgfqpoint{0.556847in}{0.516222in}}{\pgfqpoint{1.722590in}{1.783528in}} %
\pgfusepath{clip}%
\pgfsetbuttcap%
\pgfsetroundjoin%
\definecolor{currentfill}{rgb}{0.298039,0.447059,0.690196}%
\pgfsetfillcolor{currentfill}%
\pgfsetlinewidth{0.240900pt}%
\definecolor{currentstroke}{rgb}{1.000000,1.000000,1.000000}%
\pgfsetstrokecolor{currentstroke}%
\pgfsetdash{}{0pt}%
\pgfpathmoveto{\pgfqpoint{1.127216in}{1.390306in}}%
\pgfpathcurveto{\pgfqpoint{1.135452in}{1.390306in}}{\pgfqpoint{1.143352in}{1.393578in}}{\pgfqpoint{1.149176in}{1.399402in}}%
\pgfpathcurveto{\pgfqpoint{1.155000in}{1.405226in}}{\pgfqpoint{1.158273in}{1.413126in}}{\pgfqpoint{1.158273in}{1.421363in}}%
\pgfpathcurveto{\pgfqpoint{1.158273in}{1.429599in}}{\pgfqpoint{1.155000in}{1.437499in}}{\pgfqpoint{1.149176in}{1.443323in}}%
\pgfpathcurveto{\pgfqpoint{1.143352in}{1.449147in}}{\pgfqpoint{1.135452in}{1.452419in}}{\pgfqpoint{1.127216in}{1.452419in}}%
\pgfpathcurveto{\pgfqpoint{1.118980in}{1.452419in}}{\pgfqpoint{1.111080in}{1.449147in}}{\pgfqpoint{1.105256in}{1.443323in}}%
\pgfpathcurveto{\pgfqpoint{1.099432in}{1.437499in}}{\pgfqpoint{1.096160in}{1.429599in}}{\pgfqpoint{1.096160in}{1.421363in}}%
\pgfpathcurveto{\pgfqpoint{1.096160in}{1.413126in}}{\pgfqpoint{1.099432in}{1.405226in}}{\pgfqpoint{1.105256in}{1.399402in}}%
\pgfpathcurveto{\pgfqpoint{1.111080in}{1.393578in}}{\pgfqpoint{1.118980in}{1.390306in}}{\pgfqpoint{1.127216in}{1.390306in}}%
\pgfpathclose%
\pgfusepath{stroke,fill}%
\end{pgfscope}%
\begin{pgfscope}%
\pgfpathrectangle{\pgfqpoint{0.556847in}{0.516222in}}{\pgfqpoint{1.722590in}{1.783528in}} %
\pgfusepath{clip}%
\pgfsetbuttcap%
\pgfsetroundjoin%
\definecolor{currentfill}{rgb}{0.298039,0.447059,0.690196}%
\pgfsetfillcolor{currentfill}%
\pgfsetlinewidth{0.240900pt}%
\definecolor{currentstroke}{rgb}{1.000000,1.000000,1.000000}%
\pgfsetstrokecolor{currentstroke}%
\pgfsetdash{}{0pt}%
\pgfpathmoveto{\pgfqpoint{1.318615in}{1.488400in}}%
\pgfpathcurveto{\pgfqpoint{1.326851in}{1.488400in}}{\pgfqpoint{1.334751in}{1.491672in}}{\pgfqpoint{1.340575in}{1.497496in}}%
\pgfpathcurveto{\pgfqpoint{1.346399in}{1.503320in}}{\pgfqpoint{1.349671in}{1.511220in}}{\pgfqpoint{1.349671in}{1.519457in}}%
\pgfpathcurveto{\pgfqpoint{1.349671in}{1.527693in}}{\pgfqpoint{1.346399in}{1.535593in}}{\pgfqpoint{1.340575in}{1.541417in}}%
\pgfpathcurveto{\pgfqpoint{1.334751in}{1.547241in}}{\pgfqpoint{1.326851in}{1.550513in}}{\pgfqpoint{1.318615in}{1.550513in}}%
\pgfpathcurveto{\pgfqpoint{1.310379in}{1.550513in}}{\pgfqpoint{1.302479in}{1.547241in}}{\pgfqpoint{1.296655in}{1.541417in}}%
\pgfpathcurveto{\pgfqpoint{1.290831in}{1.535593in}}{\pgfqpoint{1.287558in}{1.527693in}}{\pgfqpoint{1.287558in}{1.519457in}}%
\pgfpathcurveto{\pgfqpoint{1.287558in}{1.511220in}}{\pgfqpoint{1.290831in}{1.503320in}}{\pgfqpoint{1.296655in}{1.497496in}}%
\pgfpathcurveto{\pgfqpoint{1.302479in}{1.491672in}}{\pgfqpoint{1.310379in}{1.488400in}}{\pgfqpoint{1.318615in}{1.488400in}}%
\pgfpathclose%
\pgfusepath{stroke,fill}%
\end{pgfscope}%
\begin{pgfscope}%
\pgfpathrectangle{\pgfqpoint{0.556847in}{0.516222in}}{\pgfqpoint{1.722590in}{1.783528in}} %
\pgfusepath{clip}%
\pgfsetbuttcap%
\pgfsetroundjoin%
\definecolor{currentfill}{rgb}{0.298039,0.447059,0.690196}%
\pgfsetfillcolor{currentfill}%
\pgfsetlinewidth{0.240900pt}%
\definecolor{currentstroke}{rgb}{1.000000,1.000000,1.000000}%
\pgfsetstrokecolor{currentstroke}%
\pgfsetdash{}{0pt}%
\pgfpathmoveto{\pgfqpoint{1.444938in}{1.457188in}}%
\pgfpathcurveto{\pgfqpoint{1.453174in}{1.457188in}}{\pgfqpoint{1.461075in}{1.460461in}}{\pgfqpoint{1.466898in}{1.466285in}}%
\pgfpathcurveto{\pgfqpoint{1.472722in}{1.472109in}}{\pgfqpoint{1.475995in}{1.480009in}}{\pgfqpoint{1.475995in}{1.488245in}}%
\pgfpathcurveto{\pgfqpoint{1.475995in}{1.496481in}}{\pgfqpoint{1.472722in}{1.504381in}}{\pgfqpoint{1.466898in}{1.510205in}}%
\pgfpathcurveto{\pgfqpoint{1.461075in}{1.516029in}}{\pgfqpoint{1.453174in}{1.519301in}}{\pgfqpoint{1.444938in}{1.519301in}}%
\pgfpathcurveto{\pgfqpoint{1.436702in}{1.519301in}}{\pgfqpoint{1.428802in}{1.516029in}}{\pgfqpoint{1.422978in}{1.510205in}}%
\pgfpathcurveto{\pgfqpoint{1.417154in}{1.504381in}}{\pgfqpoint{1.413882in}{1.496481in}}{\pgfqpoint{1.413882in}{1.488245in}}%
\pgfpathcurveto{\pgfqpoint{1.413882in}{1.480009in}}{\pgfqpoint{1.417154in}{1.472109in}}{\pgfqpoint{1.422978in}{1.466285in}}%
\pgfpathcurveto{\pgfqpoint{1.428802in}{1.460461in}}{\pgfqpoint{1.436702in}{1.457188in}}{\pgfqpoint{1.444938in}{1.457188in}}%
\pgfpathclose%
\pgfusepath{stroke,fill}%
\end{pgfscope}%
\begin{pgfscope}%
\pgfpathrectangle{\pgfqpoint{0.556847in}{0.516222in}}{\pgfqpoint{1.722590in}{1.783528in}} %
\pgfusepath{clip}%
\pgfsetbuttcap%
\pgfsetroundjoin%
\definecolor{currentfill}{rgb}{0.298039,0.447059,0.690196}%
\pgfsetfillcolor{currentfill}%
\pgfsetlinewidth{0.240900pt}%
\definecolor{currentstroke}{rgb}{1.000000,1.000000,1.000000}%
\pgfsetstrokecolor{currentstroke}%
\pgfsetdash{}{0pt}%
\pgfpathmoveto{\pgfqpoint{1.188464in}{1.153989in}}%
\pgfpathcurveto{\pgfqpoint{1.196700in}{1.153989in}}{\pgfqpoint{1.204600in}{1.157261in}}{\pgfqpoint{1.210424in}{1.163085in}}%
\pgfpathcurveto{\pgfqpoint{1.216248in}{1.168909in}}{\pgfqpoint{1.219520in}{1.176809in}}{\pgfqpoint{1.219520in}{1.185045in}}%
\pgfpathcurveto{\pgfqpoint{1.219520in}{1.193281in}}{\pgfqpoint{1.216248in}{1.201181in}}{\pgfqpoint{1.210424in}{1.207005in}}%
\pgfpathcurveto{\pgfqpoint{1.204600in}{1.212829in}}{\pgfqpoint{1.196700in}{1.216102in}}{\pgfqpoint{1.188464in}{1.216102in}}%
\pgfpathcurveto{\pgfqpoint{1.180227in}{1.216102in}}{\pgfqpoint{1.172327in}{1.212829in}}{\pgfqpoint{1.166503in}{1.207005in}}%
\pgfpathcurveto{\pgfqpoint{1.160679in}{1.201181in}}{\pgfqpoint{1.157407in}{1.193281in}}{\pgfqpoint{1.157407in}{1.185045in}}%
\pgfpathcurveto{\pgfqpoint{1.157407in}{1.176809in}}{\pgfqpoint{1.160679in}{1.168909in}}{\pgfqpoint{1.166503in}{1.163085in}}%
\pgfpathcurveto{\pgfqpoint{1.172327in}{1.157261in}}{\pgfqpoint{1.180227in}{1.153989in}}{\pgfqpoint{1.188464in}{1.153989in}}%
\pgfpathclose%
\pgfusepath{stroke,fill}%
\end{pgfscope}%
\begin{pgfscope}%
\pgfpathrectangle{\pgfqpoint{0.556847in}{0.516222in}}{\pgfqpoint{1.722590in}{1.783528in}} %
\pgfusepath{clip}%
\pgfsetbuttcap%
\pgfsetroundjoin%
\definecolor{currentfill}{rgb}{0.298039,0.447059,0.690196}%
\pgfsetfillcolor{currentfill}%
\pgfsetlinewidth{0.240900pt}%
\definecolor{currentstroke}{rgb}{1.000000,1.000000,1.000000}%
\pgfsetstrokecolor{currentstroke}%
\pgfsetdash{}{0pt}%
\pgfpathmoveto{\pgfqpoint{1.800940in}{1.368012in}}%
\pgfpathcurveto{\pgfqpoint{1.809176in}{1.368012in}}{\pgfqpoint{1.817077in}{1.371284in}}{\pgfqpoint{1.822900in}{1.377108in}}%
\pgfpathcurveto{\pgfqpoint{1.828724in}{1.382932in}}{\pgfqpoint{1.831997in}{1.390832in}}{\pgfqpoint{1.831997in}{1.399068in}}%
\pgfpathcurveto{\pgfqpoint{1.831997in}{1.407305in}}{\pgfqpoint{1.828724in}{1.415205in}}{\pgfqpoint{1.822900in}{1.421029in}}%
\pgfpathcurveto{\pgfqpoint{1.817077in}{1.426853in}}{\pgfqpoint{1.809176in}{1.430125in}}{\pgfqpoint{1.800940in}{1.430125in}}%
\pgfpathcurveto{\pgfqpoint{1.792704in}{1.430125in}}{\pgfqpoint{1.784804in}{1.426853in}}{\pgfqpoint{1.778980in}{1.421029in}}%
\pgfpathcurveto{\pgfqpoint{1.773156in}{1.415205in}}{\pgfqpoint{1.769884in}{1.407305in}}{\pgfqpoint{1.769884in}{1.399068in}}%
\pgfpathcurveto{\pgfqpoint{1.769884in}{1.390832in}}{\pgfqpoint{1.773156in}{1.382932in}}{\pgfqpoint{1.778980in}{1.377108in}}%
\pgfpathcurveto{\pgfqpoint{1.784804in}{1.371284in}}{\pgfqpoint{1.792704in}{1.368012in}}{\pgfqpoint{1.800940in}{1.368012in}}%
\pgfpathclose%
\pgfusepath{stroke,fill}%
\end{pgfscope}%
\begin{pgfscope}%
\pgfpathrectangle{\pgfqpoint{0.556847in}{0.516222in}}{\pgfqpoint{1.722590in}{1.783528in}} %
\pgfusepath{clip}%
\pgfsetbuttcap%
\pgfsetroundjoin%
\definecolor{currentfill}{rgb}{0.298039,0.447059,0.690196}%
\pgfsetfillcolor{currentfill}%
\pgfsetlinewidth{0.240900pt}%
\definecolor{currentstroke}{rgb}{1.000000,1.000000,1.000000}%
\pgfsetstrokecolor{currentstroke}%
\pgfsetdash{}{0pt}%
\pgfpathmoveto{\pgfqpoint{1.188464in}{1.359094in}}%
\pgfpathcurveto{\pgfqpoint{1.196700in}{1.359094in}}{\pgfqpoint{1.204600in}{1.362367in}}{\pgfqpoint{1.210424in}{1.368191in}}%
\pgfpathcurveto{\pgfqpoint{1.216248in}{1.374015in}}{\pgfqpoint{1.219520in}{1.381915in}}{\pgfqpoint{1.219520in}{1.390151in}}%
\pgfpathcurveto{\pgfqpoint{1.219520in}{1.398387in}}{\pgfqpoint{1.216248in}{1.406287in}}{\pgfqpoint{1.210424in}{1.412111in}}%
\pgfpathcurveto{\pgfqpoint{1.204600in}{1.417935in}}{\pgfqpoint{1.196700in}{1.421207in}}{\pgfqpoint{1.188464in}{1.421207in}}%
\pgfpathcurveto{\pgfqpoint{1.180227in}{1.421207in}}{\pgfqpoint{1.172327in}{1.417935in}}{\pgfqpoint{1.166503in}{1.412111in}}%
\pgfpathcurveto{\pgfqpoint{1.160679in}{1.406287in}}{\pgfqpoint{1.157407in}{1.398387in}}{\pgfqpoint{1.157407in}{1.390151in}}%
\pgfpathcurveto{\pgfqpoint{1.157407in}{1.381915in}}{\pgfqpoint{1.160679in}{1.374015in}}{\pgfqpoint{1.166503in}{1.368191in}}%
\pgfpathcurveto{\pgfqpoint{1.172327in}{1.362367in}}{\pgfqpoint{1.180227in}{1.359094in}}{\pgfqpoint{1.188464in}{1.359094in}}%
\pgfpathclose%
\pgfusepath{stroke,fill}%
\end{pgfscope}%
\begin{pgfscope}%
\pgfpathrectangle{\pgfqpoint{0.556847in}{0.516222in}}{\pgfqpoint{1.722590in}{1.783528in}} %
\pgfusepath{clip}%
\pgfsetbuttcap%
\pgfsetroundjoin%
\definecolor{currentfill}{rgb}{0.298039,0.447059,0.690196}%
\pgfsetfillcolor{currentfill}%
\pgfsetlinewidth{0.240900pt}%
\definecolor{currentstroke}{rgb}{1.000000,1.000000,1.000000}%
\pgfsetstrokecolor{currentstroke}%
\pgfsetdash{}{0pt}%
\pgfpathmoveto{\pgfqpoint{1.108076in}{1.278836in}}%
\pgfpathcurveto{\pgfqpoint{1.116312in}{1.278836in}}{\pgfqpoint{1.124212in}{1.282108in}}{\pgfqpoint{1.130036in}{1.287932in}}%
\pgfpathcurveto{\pgfqpoint{1.135860in}{1.293756in}}{\pgfqpoint{1.139133in}{1.301656in}}{\pgfqpoint{1.139133in}{1.309892in}}%
\pgfpathcurveto{\pgfqpoint{1.139133in}{1.318128in}}{\pgfqpoint{1.135860in}{1.326028in}}{\pgfqpoint{1.130036in}{1.331852in}}%
\pgfpathcurveto{\pgfqpoint{1.124212in}{1.337676in}}{\pgfqpoint{1.116312in}{1.340949in}}{\pgfqpoint{1.108076in}{1.340949in}}%
\pgfpathcurveto{\pgfqpoint{1.099840in}{1.340949in}}{\pgfqpoint{1.091940in}{1.337676in}}{\pgfqpoint{1.086116in}{1.331852in}}%
\pgfpathcurveto{\pgfqpoint{1.080292in}{1.326028in}}{\pgfqpoint{1.077020in}{1.318128in}}{\pgfqpoint{1.077020in}{1.309892in}}%
\pgfpathcurveto{\pgfqpoint{1.077020in}{1.301656in}}{\pgfqpoint{1.080292in}{1.293756in}}{\pgfqpoint{1.086116in}{1.287932in}}%
\pgfpathcurveto{\pgfqpoint{1.091940in}{1.282108in}}{\pgfqpoint{1.099840in}{1.278836in}}{\pgfqpoint{1.108076in}{1.278836in}}%
\pgfpathclose%
\pgfusepath{stroke,fill}%
\end{pgfscope}%
\begin{pgfscope}%
\pgfpathrectangle{\pgfqpoint{0.556847in}{0.516222in}}{\pgfqpoint{1.722590in}{1.783528in}} %
\pgfusepath{clip}%
\pgfsetbuttcap%
\pgfsetroundjoin%
\definecolor{currentfill}{rgb}{0.298039,0.447059,0.690196}%
\pgfsetfillcolor{currentfill}%
\pgfsetlinewidth{0.240900pt}%
\definecolor{currentstroke}{rgb}{1.000000,1.000000,1.000000}%
\pgfsetstrokecolor{currentstroke}%
\pgfsetdash{}{0pt}%
\pgfpathmoveto{\pgfqpoint{1.276507in}{0.931048in}}%
\pgfpathcurveto{\pgfqpoint{1.284743in}{0.931048in}}{\pgfqpoint{1.292643in}{0.934320in}}{\pgfqpoint{1.298467in}{0.940144in}}%
\pgfpathcurveto{\pgfqpoint{1.304291in}{0.945968in}}{\pgfqpoint{1.307564in}{0.953868in}}{\pgfqpoint{1.307564in}{0.962104in}}%
\pgfpathcurveto{\pgfqpoint{1.307564in}{0.970340in}}{\pgfqpoint{1.304291in}{0.978240in}}{\pgfqpoint{1.298467in}{0.984064in}}%
\pgfpathcurveto{\pgfqpoint{1.292643in}{0.989888in}}{\pgfqpoint{1.284743in}{0.993161in}}{\pgfqpoint{1.276507in}{0.993161in}}%
\pgfpathcurveto{\pgfqpoint{1.268271in}{0.993161in}}{\pgfqpoint{1.260371in}{0.989888in}}{\pgfqpoint{1.254547in}{0.984064in}}%
\pgfpathcurveto{\pgfqpoint{1.248723in}{0.978240in}}{\pgfqpoint{1.245451in}{0.970340in}}{\pgfqpoint{1.245451in}{0.962104in}}%
\pgfpathcurveto{\pgfqpoint{1.245451in}{0.953868in}}{\pgfqpoint{1.248723in}{0.945968in}}{\pgfqpoint{1.254547in}{0.940144in}}%
\pgfpathcurveto{\pgfqpoint{1.260371in}{0.934320in}}{\pgfqpoint{1.268271in}{0.931048in}}{\pgfqpoint{1.276507in}{0.931048in}}%
\pgfpathclose%
\pgfusepath{stroke,fill}%
\end{pgfscope}%
\begin{pgfscope}%
\pgfpathrectangle{\pgfqpoint{0.556847in}{0.516222in}}{\pgfqpoint{1.722590in}{1.783528in}} %
\pgfusepath{clip}%
\pgfsetbuttcap%
\pgfsetroundjoin%
\definecolor{currentfill}{rgb}{0.298039,0.447059,0.690196}%
\pgfsetfillcolor{currentfill}%
\pgfsetlinewidth{0.240900pt}%
\definecolor{currentstroke}{rgb}{1.000000,1.000000,1.000000}%
\pgfsetstrokecolor{currentstroke}%
\pgfsetdash{}{0pt}%
\pgfpathmoveto{\pgfqpoint{1.513842in}{2.090341in}}%
\pgfpathcurveto{\pgfqpoint{1.522078in}{2.090341in}}{\pgfqpoint{1.529978in}{2.093613in}}{\pgfqpoint{1.535802in}{2.099437in}}%
\pgfpathcurveto{\pgfqpoint{1.541626in}{2.105261in}}{\pgfqpoint{1.544898in}{2.113161in}}{\pgfqpoint{1.544898in}{2.121397in}}%
\pgfpathcurveto{\pgfqpoint{1.544898in}{2.129634in}}{\pgfqpoint{1.541626in}{2.137534in}}{\pgfqpoint{1.535802in}{2.143357in}}%
\pgfpathcurveto{\pgfqpoint{1.529978in}{2.149181in}}{\pgfqpoint{1.522078in}{2.152454in}}{\pgfqpoint{1.513842in}{2.152454in}}%
\pgfpathcurveto{\pgfqpoint{1.505606in}{2.152454in}}{\pgfqpoint{1.497705in}{2.149181in}}{\pgfqpoint{1.491882in}{2.143357in}}%
\pgfpathcurveto{\pgfqpoint{1.486058in}{2.137534in}}{\pgfqpoint{1.482785in}{2.129634in}}{\pgfqpoint{1.482785in}{2.121397in}}%
\pgfpathcurveto{\pgfqpoint{1.482785in}{2.113161in}}{\pgfqpoint{1.486058in}{2.105261in}}{\pgfqpoint{1.491882in}{2.099437in}}%
\pgfpathcurveto{\pgfqpoint{1.497705in}{2.093613in}}{\pgfqpoint{1.505606in}{2.090341in}}{\pgfqpoint{1.513842in}{2.090341in}}%
\pgfpathclose%
\pgfusepath{stroke,fill}%
\end{pgfscope}%
\begin{pgfscope}%
\pgfpathrectangle{\pgfqpoint{0.556847in}{0.516222in}}{\pgfqpoint{1.722590in}{1.783528in}} %
\pgfusepath{clip}%
\pgfsetbuttcap%
\pgfsetroundjoin%
\definecolor{currentfill}{rgb}{0.298039,0.447059,0.690196}%
\pgfsetfillcolor{currentfill}%
\pgfsetlinewidth{0.240900pt}%
\definecolor{currentstroke}{rgb}{1.000000,1.000000,1.000000}%
\pgfsetstrokecolor{currentstroke}%
\pgfsetdash{}{0pt}%
\pgfpathmoveto{\pgfqpoint{1.142528in}{0.810660in}}%
\pgfpathcurveto{\pgfqpoint{1.150764in}{0.810660in}}{\pgfqpoint{1.158664in}{0.813932in}}{\pgfqpoint{1.164488in}{0.819756in}}%
\pgfpathcurveto{\pgfqpoint{1.170312in}{0.825580in}}{\pgfqpoint{1.173584in}{0.833480in}}{\pgfqpoint{1.173584in}{0.841716in}}%
\pgfpathcurveto{\pgfqpoint{1.173584in}{0.849952in}}{\pgfqpoint{1.170312in}{0.857852in}}{\pgfqpoint{1.164488in}{0.863676in}}%
\pgfpathcurveto{\pgfqpoint{1.158664in}{0.869500in}}{\pgfqpoint{1.150764in}{0.872773in}}{\pgfqpoint{1.142528in}{0.872773in}}%
\pgfpathcurveto{\pgfqpoint{1.134292in}{0.872773in}}{\pgfqpoint{1.126392in}{0.869500in}}{\pgfqpoint{1.120568in}{0.863676in}}%
\pgfpathcurveto{\pgfqpoint{1.114744in}{0.857852in}}{\pgfqpoint{1.111471in}{0.849952in}}{\pgfqpoint{1.111471in}{0.841716in}}%
\pgfpathcurveto{\pgfqpoint{1.111471in}{0.833480in}}{\pgfqpoint{1.114744in}{0.825580in}}{\pgfqpoint{1.120568in}{0.819756in}}%
\pgfpathcurveto{\pgfqpoint{1.126392in}{0.813932in}}{\pgfqpoint{1.134292in}{0.810660in}}{\pgfqpoint{1.142528in}{0.810660in}}%
\pgfpathclose%
\pgfusepath{stroke,fill}%
\end{pgfscope}%
\begin{pgfscope}%
\pgfpathrectangle{\pgfqpoint{0.556847in}{0.516222in}}{\pgfqpoint{1.722590in}{1.783528in}} %
\pgfusepath{clip}%
\pgfsetbuttcap%
\pgfsetroundjoin%
\definecolor{currentfill}{rgb}{0.298039,0.447059,0.690196}%
\pgfsetfillcolor{currentfill}%
\pgfsetlinewidth{0.240900pt}%
\definecolor{currentstroke}{rgb}{1.000000,1.000000,1.000000}%
\pgfsetstrokecolor{currentstroke}%
\pgfsetdash{}{0pt}%
\pgfpathmoveto{\pgfqpoint{1.900468in}{1.265459in}}%
\pgfpathcurveto{\pgfqpoint{1.908704in}{1.265459in}}{\pgfqpoint{1.916604in}{1.268731in}}{\pgfqpoint{1.922428in}{1.274555in}}%
\pgfpathcurveto{\pgfqpoint{1.928252in}{1.280379in}}{\pgfqpoint{1.931524in}{1.288279in}}{\pgfqpoint{1.931524in}{1.296516in}}%
\pgfpathcurveto{\pgfqpoint{1.931524in}{1.304752in}}{\pgfqpoint{1.928252in}{1.312652in}}{\pgfqpoint{1.922428in}{1.318476in}}%
\pgfpathcurveto{\pgfqpoint{1.916604in}{1.324300in}}{\pgfqpoint{1.908704in}{1.327572in}}{\pgfqpoint{1.900468in}{1.327572in}}%
\pgfpathcurveto{\pgfqpoint{1.892231in}{1.327572in}}{\pgfqpoint{1.884331in}{1.324300in}}{\pgfqpoint{1.878507in}{1.318476in}}%
\pgfpathcurveto{\pgfqpoint{1.872683in}{1.312652in}}{\pgfqpoint{1.869411in}{1.304752in}}{\pgfqpoint{1.869411in}{1.296516in}}%
\pgfpathcurveto{\pgfqpoint{1.869411in}{1.288279in}}{\pgfqpoint{1.872683in}{1.280379in}}{\pgfqpoint{1.878507in}{1.274555in}}%
\pgfpathcurveto{\pgfqpoint{1.884331in}{1.268731in}}{\pgfqpoint{1.892231in}{1.265459in}}{\pgfqpoint{1.900468in}{1.265459in}}%
\pgfpathclose%
\pgfusepath{stroke,fill}%
\end{pgfscope}%
\begin{pgfscope}%
\pgfpathrectangle{\pgfqpoint{0.556847in}{0.516222in}}{\pgfqpoint{1.722590in}{1.783528in}} %
\pgfusepath{clip}%
\pgfsetbuttcap%
\pgfsetroundjoin%
\definecolor{currentfill}{rgb}{0.298039,0.447059,0.690196}%
\pgfsetfillcolor{currentfill}%
\pgfsetlinewidth{0.240900pt}%
\definecolor{currentstroke}{rgb}{1.000000,1.000000,1.000000}%
\pgfsetstrokecolor{currentstroke}%
\pgfsetdash{}{0pt}%
\pgfpathmoveto{\pgfqpoint{1.525326in}{1.211953in}}%
\pgfpathcurveto{\pgfqpoint{1.533562in}{1.211953in}}{\pgfqpoint{1.541462in}{1.215226in}}{\pgfqpoint{1.547286in}{1.221050in}}%
\pgfpathcurveto{\pgfqpoint{1.553110in}{1.226873in}}{\pgfqpoint{1.556382in}{1.234774in}}{\pgfqpoint{1.556382in}{1.243010in}}%
\pgfpathcurveto{\pgfqpoint{1.556382in}{1.251246in}}{\pgfqpoint{1.553110in}{1.259146in}}{\pgfqpoint{1.547286in}{1.264970in}}%
\pgfpathcurveto{\pgfqpoint{1.541462in}{1.270794in}}{\pgfqpoint{1.533562in}{1.274066in}}{\pgfqpoint{1.525326in}{1.274066in}}%
\pgfpathcurveto{\pgfqpoint{1.517089in}{1.274066in}}{\pgfqpoint{1.509189in}{1.270794in}}{\pgfqpoint{1.503365in}{1.264970in}}%
\pgfpathcurveto{\pgfqpoint{1.497542in}{1.259146in}}{\pgfqpoint{1.494269in}{1.251246in}}{\pgfqpoint{1.494269in}{1.243010in}}%
\pgfpathcurveto{\pgfqpoint{1.494269in}{1.234774in}}{\pgfqpoint{1.497542in}{1.226873in}}{\pgfqpoint{1.503365in}{1.221050in}}%
\pgfpathcurveto{\pgfqpoint{1.509189in}{1.215226in}}{\pgfqpoint{1.517089in}{1.211953in}}{\pgfqpoint{1.525326in}{1.211953in}}%
\pgfpathclose%
\pgfusepath{stroke,fill}%
\end{pgfscope}%
\begin{pgfscope}%
\pgfpathrectangle{\pgfqpoint{0.556847in}{0.516222in}}{\pgfqpoint{1.722590in}{1.783528in}} %
\pgfusepath{clip}%
\pgfsetbuttcap%
\pgfsetroundjoin%
\definecolor{currentfill}{rgb}{0.298039,0.447059,0.690196}%
\pgfsetfillcolor{currentfill}%
\pgfsetlinewidth{0.240900pt}%
\definecolor{currentstroke}{rgb}{1.000000,1.000000,1.000000}%
\pgfsetstrokecolor{currentstroke}%
\pgfsetdash{}{0pt}%
\pgfpathmoveto{\pgfqpoint{1.785628in}{1.559741in}}%
\pgfpathcurveto{\pgfqpoint{1.793865in}{1.559741in}}{\pgfqpoint{1.801765in}{1.563014in}}{\pgfqpoint{1.807589in}{1.568837in}}%
\pgfpathcurveto{\pgfqpoint{1.813412in}{1.574661in}}{\pgfqpoint{1.816685in}{1.582561in}}{\pgfqpoint{1.816685in}{1.590798in}}%
\pgfpathcurveto{\pgfqpoint{1.816685in}{1.599034in}}{\pgfqpoint{1.813412in}{1.606934in}}{\pgfqpoint{1.807589in}{1.612758in}}%
\pgfpathcurveto{\pgfqpoint{1.801765in}{1.618582in}}{\pgfqpoint{1.793865in}{1.621854in}}{\pgfqpoint{1.785628in}{1.621854in}}%
\pgfpathcurveto{\pgfqpoint{1.777392in}{1.621854in}}{\pgfqpoint{1.769492in}{1.618582in}}{\pgfqpoint{1.763668in}{1.612758in}}%
\pgfpathcurveto{\pgfqpoint{1.757844in}{1.606934in}}{\pgfqpoint{1.754572in}{1.599034in}}{\pgfqpoint{1.754572in}{1.590798in}}%
\pgfpathcurveto{\pgfqpoint{1.754572in}{1.582561in}}{\pgfqpoint{1.757844in}{1.574661in}}{\pgfqpoint{1.763668in}{1.568837in}}%
\pgfpathcurveto{\pgfqpoint{1.769492in}{1.563014in}}{\pgfqpoint{1.777392in}{1.559741in}}{\pgfqpoint{1.785628in}{1.559741in}}%
\pgfpathclose%
\pgfusepath{stroke,fill}%
\end{pgfscope}%
\begin{pgfscope}%
\pgfpathrectangle{\pgfqpoint{0.556847in}{0.516222in}}{\pgfqpoint{1.722590in}{1.783528in}} %
\pgfusepath{clip}%
\pgfsetbuttcap%
\pgfsetroundjoin%
\definecolor{currentfill}{rgb}{0.298039,0.447059,0.690196}%
\pgfsetfillcolor{currentfill}%
\pgfsetlinewidth{0.240900pt}%
\definecolor{currentstroke}{rgb}{1.000000,1.000000,1.000000}%
\pgfsetstrokecolor{currentstroke}%
\pgfsetdash{}{0pt}%
\pgfpathmoveto{\pgfqpoint{2.091867in}{0.939965in}}%
\pgfpathcurveto{\pgfqpoint{2.100103in}{0.939965in}}{\pgfqpoint{2.108003in}{0.943238in}}{\pgfqpoint{2.113827in}{0.949062in}}%
\pgfpathcurveto{\pgfqpoint{2.119651in}{0.954885in}}{\pgfqpoint{2.122923in}{0.962786in}}{\pgfqpoint{2.122923in}{0.971022in}}%
\pgfpathcurveto{\pgfqpoint{2.122923in}{0.979258in}}{\pgfqpoint{2.119651in}{0.987158in}}{\pgfqpoint{2.113827in}{0.992982in}}%
\pgfpathcurveto{\pgfqpoint{2.108003in}{0.998806in}}{\pgfqpoint{2.100103in}{1.002078in}}{\pgfqpoint{2.091867in}{1.002078in}}%
\pgfpathcurveto{\pgfqpoint{2.083630in}{1.002078in}}{\pgfqpoint{2.075730in}{0.998806in}}{\pgfqpoint{2.069906in}{0.992982in}}%
\pgfpathcurveto{\pgfqpoint{2.064082in}{0.987158in}}{\pgfqpoint{2.060810in}{0.979258in}}{\pgfqpoint{2.060810in}{0.971022in}}%
\pgfpathcurveto{\pgfqpoint{2.060810in}{0.962786in}}{\pgfqpoint{2.064082in}{0.954885in}}{\pgfqpoint{2.069906in}{0.949062in}}%
\pgfpathcurveto{\pgfqpoint{2.075730in}{0.943238in}}{\pgfqpoint{2.083630in}{0.939965in}}{\pgfqpoint{2.091867in}{0.939965in}}%
\pgfpathclose%
\pgfusepath{stroke,fill}%
\end{pgfscope}%
\begin{pgfscope}%
\pgfpathrectangle{\pgfqpoint{0.556847in}{0.516222in}}{\pgfqpoint{1.722590in}{1.783528in}} %
\pgfusepath{clip}%
\pgfsetbuttcap%
\pgfsetroundjoin%
\definecolor{currentfill}{rgb}{0.298039,0.447059,0.690196}%
\pgfsetfillcolor{currentfill}%
\pgfsetlinewidth{0.240900pt}%
\definecolor{currentstroke}{rgb}{1.000000,1.000000,1.000000}%
\pgfsetstrokecolor{currentstroke}%
\pgfsetdash{}{0pt}%
\pgfpathmoveto{\pgfqpoint{1.131044in}{1.158447in}}%
\pgfpathcurveto{\pgfqpoint{1.139280in}{1.158447in}}{\pgfqpoint{1.147180in}{1.161720in}}{\pgfqpoint{1.153004in}{1.167544in}}%
\pgfpathcurveto{\pgfqpoint{1.158828in}{1.173368in}}{\pgfqpoint{1.162100in}{1.181268in}}{\pgfqpoint{1.162100in}{1.189504in}}%
\pgfpathcurveto{\pgfqpoint{1.162100in}{1.197740in}}{\pgfqpoint{1.158828in}{1.205640in}}{\pgfqpoint{1.153004in}{1.211464in}}%
\pgfpathcurveto{\pgfqpoint{1.147180in}{1.217288in}}{\pgfqpoint{1.139280in}{1.220560in}}{\pgfqpoint{1.131044in}{1.220560in}}%
\pgfpathcurveto{\pgfqpoint{1.122808in}{1.220560in}}{\pgfqpoint{1.114908in}{1.217288in}}{\pgfqpoint{1.109084in}{1.211464in}}%
\pgfpathcurveto{\pgfqpoint{1.103260in}{1.205640in}}{\pgfqpoint{1.099987in}{1.197740in}}{\pgfqpoint{1.099987in}{1.189504in}}%
\pgfpathcurveto{\pgfqpoint{1.099987in}{1.181268in}}{\pgfqpoint{1.103260in}{1.173368in}}{\pgfqpoint{1.109084in}{1.167544in}}%
\pgfpathcurveto{\pgfqpoint{1.114908in}{1.161720in}}{\pgfqpoint{1.122808in}{1.158447in}}{\pgfqpoint{1.131044in}{1.158447in}}%
\pgfpathclose%
\pgfusepath{stroke,fill}%
\end{pgfscope}%
\begin{pgfscope}%
\pgfpathrectangle{\pgfqpoint{0.556847in}{0.516222in}}{\pgfqpoint{1.722590in}{1.783528in}} %
\pgfusepath{clip}%
\pgfsetbuttcap%
\pgfsetroundjoin%
\definecolor{currentfill}{rgb}{0.298039,0.447059,0.690196}%
\pgfsetfillcolor{currentfill}%
\pgfsetlinewidth{0.240900pt}%
\definecolor{currentstroke}{rgb}{1.000000,1.000000,1.000000}%
\pgfsetstrokecolor{currentstroke}%
\pgfsetdash{}{0pt}%
\pgfpathmoveto{\pgfqpoint{1.885156in}{1.265459in}}%
\pgfpathcurveto{\pgfqpoint{1.893392in}{1.265459in}}{\pgfqpoint{1.901292in}{1.268731in}}{\pgfqpoint{1.907116in}{1.274555in}}%
\pgfpathcurveto{\pgfqpoint{1.912940in}{1.280379in}}{\pgfqpoint{1.916212in}{1.288279in}}{\pgfqpoint{1.916212in}{1.296516in}}%
\pgfpathcurveto{\pgfqpoint{1.916212in}{1.304752in}}{\pgfqpoint{1.912940in}{1.312652in}}{\pgfqpoint{1.907116in}{1.318476in}}%
\pgfpathcurveto{\pgfqpoint{1.901292in}{1.324300in}}{\pgfqpoint{1.893392in}{1.327572in}}{\pgfqpoint{1.885156in}{1.327572in}}%
\pgfpathcurveto{\pgfqpoint{1.876919in}{1.327572in}}{\pgfqpoint{1.869019in}{1.324300in}}{\pgfqpoint{1.863195in}{1.318476in}}%
\pgfpathcurveto{\pgfqpoint{1.857372in}{1.312652in}}{\pgfqpoint{1.854099in}{1.304752in}}{\pgfqpoint{1.854099in}{1.296516in}}%
\pgfpathcurveto{\pgfqpoint{1.854099in}{1.288279in}}{\pgfqpoint{1.857372in}{1.280379in}}{\pgfqpoint{1.863195in}{1.274555in}}%
\pgfpathcurveto{\pgfqpoint{1.869019in}{1.268731in}}{\pgfqpoint{1.876919in}{1.265459in}}{\pgfqpoint{1.885156in}{1.265459in}}%
\pgfpathclose%
\pgfusepath{stroke,fill}%
\end{pgfscope}%
\begin{pgfscope}%
\pgfpathrectangle{\pgfqpoint{0.556847in}{0.516222in}}{\pgfqpoint{1.722590in}{1.783528in}} %
\pgfusepath{clip}%
\pgfsetbuttcap%
\pgfsetroundjoin%
\definecolor{currentfill}{rgb}{0.298039,0.447059,0.690196}%
\pgfsetfillcolor{currentfill}%
\pgfsetlinewidth{0.240900pt}%
\definecolor{currentstroke}{rgb}{1.000000,1.000000,1.000000}%
\pgfsetstrokecolor{currentstroke}%
\pgfsetdash{}{0pt}%
\pgfpathmoveto{\pgfqpoint{1.670789in}{1.920906in}}%
\pgfpathcurveto{\pgfqpoint{1.679025in}{1.920906in}}{\pgfqpoint{1.686925in}{1.924178in}}{\pgfqpoint{1.692749in}{1.930002in}}%
\pgfpathcurveto{\pgfqpoint{1.698573in}{1.935826in}}{\pgfqpoint{1.701845in}{1.943726in}}{\pgfqpoint{1.701845in}{1.951962in}}%
\pgfpathcurveto{\pgfqpoint{1.701845in}{1.960198in}}{\pgfqpoint{1.698573in}{1.968098in}}{\pgfqpoint{1.692749in}{1.973922in}}%
\pgfpathcurveto{\pgfqpoint{1.686925in}{1.979746in}}{\pgfqpoint{1.679025in}{1.983019in}}{\pgfqpoint{1.670789in}{1.983019in}}%
\pgfpathcurveto{\pgfqpoint{1.662553in}{1.983019in}}{\pgfqpoint{1.654653in}{1.979746in}}{\pgfqpoint{1.648829in}{1.973922in}}%
\pgfpathcurveto{\pgfqpoint{1.643005in}{1.968098in}}{\pgfqpoint{1.639732in}{1.960198in}}{\pgfqpoint{1.639732in}{1.951962in}}%
\pgfpathcurveto{\pgfqpoint{1.639732in}{1.943726in}}{\pgfqpoint{1.643005in}{1.935826in}}{\pgfqpoint{1.648829in}{1.930002in}}%
\pgfpathcurveto{\pgfqpoint{1.654653in}{1.924178in}}{\pgfqpoint{1.662553in}{1.920906in}}{\pgfqpoint{1.670789in}{1.920906in}}%
\pgfpathclose%
\pgfusepath{stroke,fill}%
\end{pgfscope}%
\begin{pgfscope}%
\pgfpathrectangle{\pgfqpoint{0.556847in}{0.516222in}}{\pgfqpoint{1.722590in}{1.783528in}} %
\pgfusepath{clip}%
\pgfsetbuttcap%
\pgfsetroundjoin%
\definecolor{currentfill}{rgb}{0.298039,0.447059,0.690196}%
\pgfsetfillcolor{currentfill}%
\pgfsetlinewidth{0.240900pt}%
\definecolor{currentstroke}{rgb}{1.000000,1.000000,1.000000}%
\pgfsetstrokecolor{currentstroke}%
\pgfsetdash{}{0pt}%
\pgfpathmoveto{\pgfqpoint{1.303303in}{1.051436in}}%
\pgfpathcurveto{\pgfqpoint{1.311539in}{1.051436in}}{\pgfqpoint{1.319439in}{1.054708in}}{\pgfqpoint{1.325263in}{1.060532in}}%
\pgfpathcurveto{\pgfqpoint{1.331087in}{1.066356in}}{\pgfqpoint{1.334360in}{1.074256in}}{\pgfqpoint{1.334360in}{1.082492in}}%
\pgfpathcurveto{\pgfqpoint{1.334360in}{1.090729in}}{\pgfqpoint{1.331087in}{1.098629in}}{\pgfqpoint{1.325263in}{1.104453in}}%
\pgfpathcurveto{\pgfqpoint{1.319439in}{1.110276in}}{\pgfqpoint{1.311539in}{1.113549in}}{\pgfqpoint{1.303303in}{1.113549in}}%
\pgfpathcurveto{\pgfqpoint{1.295067in}{1.113549in}}{\pgfqpoint{1.287167in}{1.110276in}}{\pgfqpoint{1.281343in}{1.104453in}}%
\pgfpathcurveto{\pgfqpoint{1.275519in}{1.098629in}}{\pgfqpoint{1.272247in}{1.090729in}}{\pgfqpoint{1.272247in}{1.082492in}}%
\pgfpathcurveto{\pgfqpoint{1.272247in}{1.074256in}}{\pgfqpoint{1.275519in}{1.066356in}}{\pgfqpoint{1.281343in}{1.060532in}}%
\pgfpathcurveto{\pgfqpoint{1.287167in}{1.054708in}}{\pgfqpoint{1.295067in}{1.051436in}}{\pgfqpoint{1.303303in}{1.051436in}}%
\pgfpathclose%
\pgfusepath{stroke,fill}%
\end{pgfscope}%
\begin{pgfscope}%
\pgfpathrectangle{\pgfqpoint{0.556847in}{0.516222in}}{\pgfqpoint{1.722590in}{1.783528in}} %
\pgfusepath{clip}%
\pgfsetbuttcap%
\pgfsetroundjoin%
\definecolor{currentfill}{rgb}{0.298039,0.447059,0.690196}%
\pgfsetfillcolor{currentfill}%
\pgfsetlinewidth{0.240900pt}%
\definecolor{currentstroke}{rgb}{1.000000,1.000000,1.000000}%
\pgfsetstrokecolor{currentstroke}%
\pgfsetdash{}{0pt}%
\pgfpathmoveto{\pgfqpoint{1.552122in}{1.345718in}}%
\pgfpathcurveto{\pgfqpoint{1.560358in}{1.345718in}}{\pgfqpoint{1.568258in}{1.348990in}}{\pgfqpoint{1.574082in}{1.354814in}}%
\pgfpathcurveto{\pgfqpoint{1.579906in}{1.360638in}}{\pgfqpoint{1.583178in}{1.368538in}}{\pgfqpoint{1.583178in}{1.376774in}}%
\pgfpathcurveto{\pgfqpoint{1.583178in}{1.385011in}}{\pgfqpoint{1.579906in}{1.392911in}}{\pgfqpoint{1.574082in}{1.398735in}}%
\pgfpathcurveto{\pgfqpoint{1.568258in}{1.404559in}}{\pgfqpoint{1.560358in}{1.407831in}}{\pgfqpoint{1.552122in}{1.407831in}}%
\pgfpathcurveto{\pgfqpoint{1.543885in}{1.407831in}}{\pgfqpoint{1.535985in}{1.404559in}}{\pgfqpoint{1.530161in}{1.398735in}}%
\pgfpathcurveto{\pgfqpoint{1.524337in}{1.392911in}}{\pgfqpoint{1.521065in}{1.385011in}}{\pgfqpoint{1.521065in}{1.376774in}}%
\pgfpathcurveto{\pgfqpoint{1.521065in}{1.368538in}}{\pgfqpoint{1.524337in}{1.360638in}}{\pgfqpoint{1.530161in}{1.354814in}}%
\pgfpathcurveto{\pgfqpoint{1.535985in}{1.348990in}}{\pgfqpoint{1.543885in}{1.345718in}}{\pgfqpoint{1.552122in}{1.345718in}}%
\pgfpathclose%
\pgfusepath{stroke,fill}%
\end{pgfscope}%
\begin{pgfscope}%
\pgfsetrectcap%
\pgfsetmiterjoin%
\pgfsetlinewidth{0.000000pt}%
\definecolor{currentstroke}{rgb}{1.000000,1.000000,1.000000}%
\pgfsetstrokecolor{currentstroke}%
\pgfsetdash{}{0pt}%
\pgfpathmoveto{\pgfqpoint{0.556847in}{0.516222in}}%
\pgfpathlineto{\pgfqpoint{0.556847in}{2.299750in}}%
\pgfusepath{}%
\end{pgfscope}%
\begin{pgfscope}%
\pgfsetrectcap%
\pgfsetmiterjoin%
\pgfsetlinewidth{0.000000pt}%
\definecolor{currentstroke}{rgb}{1.000000,1.000000,1.000000}%
\pgfsetstrokecolor{currentstroke}%
\pgfsetdash{}{0pt}%
\pgfpathmoveto{\pgfqpoint{0.556847in}{0.516222in}}%
\pgfpathlineto{\pgfqpoint{2.279437in}{0.516222in}}%
\pgfusepath{}%
\end{pgfscope}%
\end{pgfpicture}%
\makeatother%
\endgroup%

		\caption{Comparison between the two times from different throws.}
		\label{fig_EX1_EX2}
	\end{subfigure}
	\caption{Plots of time measurments made by two observers(obs)}
\end{figure}

\subsection{Fitting a GP model with average data}

Another Assumption made to explain the $Q2$ value was the simplicity of the model, initially a GP with a Matern 32 kernel was used. Thus, after modifying the time variable to be the average of the four measurements and adding some other 2 RBF kernels the $Q2$ measure increased in 0.4 giving exactly a value of $0.58$ aproximately. The $Q2$ measure was enough for us to follow with the optimization stage. With the new data was not neccesary to modify the leave one out function to delete the repeated point with the other time observation.

\begin{figure}
	\begin{subfigure}[h]{.5\linewidth}
		%% Creator: Matplotlib, PGF backend
%%
%% To include the figure in your LaTeX document, write
%%   \input{<filename>.pgf}
%%
%% Make sure the required packages are loaded in your preamble
%%   \usepackage{pgf}
%%
%% Figures using additional raster images can only be included by \input if
%% they are in the same directory as the main LaTeX file. For loading figures
%% from other directories you can use the `import` package
%%   \usepackage{import}
%% and then include the figures with
%%   \import{<path to file>}{<filename>.pgf}
%%
%% Matplotlib used the following preamble
%%   \usepackage[utf8x]{inputenc}
%%   \usepackage[T1]{fontenc}
%%   \usepackage{cmbright}
%%
\begingroup%
\makeatletter%
\begin{pgfpicture}%
\pgfpathrectangle{\pgfpointorigin}{\pgfqpoint{2.500000in}{2.500000in}}%
\pgfusepath{use as bounding box, clip}%
\begin{pgfscope}%
\pgfsetbuttcap%
\pgfsetmiterjoin%
\definecolor{currentfill}{rgb}{1.000000,1.000000,1.000000}%
\pgfsetfillcolor{currentfill}%
\pgfsetlinewidth{0.000000pt}%
\definecolor{currentstroke}{rgb}{1.000000,1.000000,1.000000}%
\pgfsetstrokecolor{currentstroke}%
\pgfsetdash{}{0pt}%
\pgfpathmoveto{\pgfqpoint{0.000000in}{0.000000in}}%
\pgfpathlineto{\pgfqpoint{2.500000in}{0.000000in}}%
\pgfpathlineto{\pgfqpoint{2.500000in}{2.500000in}}%
\pgfpathlineto{\pgfqpoint{0.000000in}{2.500000in}}%
\pgfpathclose%
\pgfusepath{fill}%
\end{pgfscope}%
\begin{pgfscope}%
\pgfsetbuttcap%
\pgfsetmiterjoin%
\definecolor{currentfill}{rgb}{0.917647,0.917647,0.949020}%
\pgfsetfillcolor{currentfill}%
\pgfsetlinewidth{0.000000pt}%
\definecolor{currentstroke}{rgb}{0.000000,0.000000,0.000000}%
\pgfsetstrokecolor{currentstroke}%
\pgfsetstrokeopacity{0.000000}%
\pgfsetdash{}{0pt}%
\pgfpathmoveto{\pgfqpoint{0.548058in}{0.516222in}}%
\pgfpathlineto{\pgfqpoint{2.287641in}{0.516222in}}%
\pgfpathlineto{\pgfqpoint{2.287641in}{2.299750in}}%
\pgfpathlineto{\pgfqpoint{0.548058in}{2.299750in}}%
\pgfpathclose%
\pgfusepath{fill}%
\end{pgfscope}%
\begin{pgfscope}%
\pgfpathrectangle{\pgfqpoint{0.548058in}{0.516222in}}{\pgfqpoint{1.739582in}{1.783528in}} %
\pgfusepath{clip}%
\pgfsetroundcap%
\pgfsetroundjoin%
\pgfsetlinewidth{0.803000pt}%
\definecolor{currentstroke}{rgb}{1.000000,1.000000,1.000000}%
\pgfsetstrokecolor{currentstroke}%
\pgfsetdash{}{0pt}%
\pgfpathmoveto{\pgfqpoint{0.548058in}{0.516222in}}%
\pgfpathlineto{\pgfqpoint{0.548058in}{2.299750in}}%
\pgfusepath{stroke}%
\end{pgfscope}%
\begin{pgfscope}%
\pgfsetbuttcap%
\pgfsetroundjoin%
\definecolor{currentfill}{rgb}{0.150000,0.150000,0.150000}%
\pgfsetfillcolor{currentfill}%
\pgfsetlinewidth{0.803000pt}%
\definecolor{currentstroke}{rgb}{0.150000,0.150000,0.150000}%
\pgfsetstrokecolor{currentstroke}%
\pgfsetdash{}{0pt}%
\pgfsys@defobject{currentmarker}{\pgfqpoint{0.000000in}{0.000000in}}{\pgfqpoint{0.000000in}{0.000000in}}{%
\pgfpathmoveto{\pgfqpoint{0.000000in}{0.000000in}}%
\pgfpathlineto{\pgfqpoint{0.000000in}{0.000000in}}%
\pgfusepath{stroke,fill}%
}%
\begin{pgfscope}%
\pgfsys@transformshift{0.548058in}{0.516222in}%
\pgfsys@useobject{currentmarker}{}%
\end{pgfscope}%
\end{pgfscope}%
\begin{pgfscope}%
\definecolor{textcolor}{rgb}{0.150000,0.150000,0.150000}%
\pgfsetstrokecolor{textcolor}%
\pgfsetfillcolor{textcolor}%
\pgftext[x=0.548058in,y=0.438444in,,top]{\color{textcolor}\sffamily\fontsize{8.000000}{9.600000}\selectfont −0.2}%
\end{pgfscope}%
\begin{pgfscope}%
\pgfpathrectangle{\pgfqpoint{0.548058in}{0.516222in}}{\pgfqpoint{1.739582in}{1.783528in}} %
\pgfusepath{clip}%
\pgfsetroundcap%
\pgfsetroundjoin%
\pgfsetlinewidth{0.803000pt}%
\definecolor{currentstroke}{rgb}{1.000000,1.000000,1.000000}%
\pgfsetstrokecolor{currentstroke}%
\pgfsetdash{}{0pt}%
\pgfpathmoveto{\pgfqpoint{0.796570in}{0.516222in}}%
\pgfpathlineto{\pgfqpoint{0.796570in}{2.299750in}}%
\pgfusepath{stroke}%
\end{pgfscope}%
\begin{pgfscope}%
\pgfsetbuttcap%
\pgfsetroundjoin%
\definecolor{currentfill}{rgb}{0.150000,0.150000,0.150000}%
\pgfsetfillcolor{currentfill}%
\pgfsetlinewidth{0.803000pt}%
\definecolor{currentstroke}{rgb}{0.150000,0.150000,0.150000}%
\pgfsetstrokecolor{currentstroke}%
\pgfsetdash{}{0pt}%
\pgfsys@defobject{currentmarker}{\pgfqpoint{0.000000in}{0.000000in}}{\pgfqpoint{0.000000in}{0.000000in}}{%
\pgfpathmoveto{\pgfqpoint{0.000000in}{0.000000in}}%
\pgfpathlineto{\pgfqpoint{0.000000in}{0.000000in}}%
\pgfusepath{stroke,fill}%
}%
\begin{pgfscope}%
\pgfsys@transformshift{0.796570in}{0.516222in}%
\pgfsys@useobject{currentmarker}{}%
\end{pgfscope}%
\end{pgfscope}%
\begin{pgfscope}%
\definecolor{textcolor}{rgb}{0.150000,0.150000,0.150000}%
\pgfsetstrokecolor{textcolor}%
\pgfsetfillcolor{textcolor}%
\pgftext[x=0.796570in,y=0.438444in,,top]{\color{textcolor}\sffamily\fontsize{8.000000}{9.600000}\selectfont 0.0}%
\end{pgfscope}%
\begin{pgfscope}%
\pgfpathrectangle{\pgfqpoint{0.548058in}{0.516222in}}{\pgfqpoint{1.739582in}{1.783528in}} %
\pgfusepath{clip}%
\pgfsetroundcap%
\pgfsetroundjoin%
\pgfsetlinewidth{0.803000pt}%
\definecolor{currentstroke}{rgb}{1.000000,1.000000,1.000000}%
\pgfsetstrokecolor{currentstroke}%
\pgfsetdash{}{0pt}%
\pgfpathmoveto{\pgfqpoint{1.045082in}{0.516222in}}%
\pgfpathlineto{\pgfqpoint{1.045082in}{2.299750in}}%
\pgfusepath{stroke}%
\end{pgfscope}%
\begin{pgfscope}%
\pgfsetbuttcap%
\pgfsetroundjoin%
\definecolor{currentfill}{rgb}{0.150000,0.150000,0.150000}%
\pgfsetfillcolor{currentfill}%
\pgfsetlinewidth{0.803000pt}%
\definecolor{currentstroke}{rgb}{0.150000,0.150000,0.150000}%
\pgfsetstrokecolor{currentstroke}%
\pgfsetdash{}{0pt}%
\pgfsys@defobject{currentmarker}{\pgfqpoint{0.000000in}{0.000000in}}{\pgfqpoint{0.000000in}{0.000000in}}{%
\pgfpathmoveto{\pgfqpoint{0.000000in}{0.000000in}}%
\pgfpathlineto{\pgfqpoint{0.000000in}{0.000000in}}%
\pgfusepath{stroke,fill}%
}%
\begin{pgfscope}%
\pgfsys@transformshift{1.045082in}{0.516222in}%
\pgfsys@useobject{currentmarker}{}%
\end{pgfscope}%
\end{pgfscope}%
\begin{pgfscope}%
\definecolor{textcolor}{rgb}{0.150000,0.150000,0.150000}%
\pgfsetstrokecolor{textcolor}%
\pgfsetfillcolor{textcolor}%
\pgftext[x=1.045082in,y=0.438444in,,top]{\color{textcolor}\sffamily\fontsize{8.000000}{9.600000}\selectfont 0.2}%
\end{pgfscope}%
\begin{pgfscope}%
\pgfpathrectangle{\pgfqpoint{0.548058in}{0.516222in}}{\pgfqpoint{1.739582in}{1.783528in}} %
\pgfusepath{clip}%
\pgfsetroundcap%
\pgfsetroundjoin%
\pgfsetlinewidth{0.803000pt}%
\definecolor{currentstroke}{rgb}{1.000000,1.000000,1.000000}%
\pgfsetstrokecolor{currentstroke}%
\pgfsetdash{}{0pt}%
\pgfpathmoveto{\pgfqpoint{1.293594in}{0.516222in}}%
\pgfpathlineto{\pgfqpoint{1.293594in}{2.299750in}}%
\pgfusepath{stroke}%
\end{pgfscope}%
\begin{pgfscope}%
\pgfsetbuttcap%
\pgfsetroundjoin%
\definecolor{currentfill}{rgb}{0.150000,0.150000,0.150000}%
\pgfsetfillcolor{currentfill}%
\pgfsetlinewidth{0.803000pt}%
\definecolor{currentstroke}{rgb}{0.150000,0.150000,0.150000}%
\pgfsetstrokecolor{currentstroke}%
\pgfsetdash{}{0pt}%
\pgfsys@defobject{currentmarker}{\pgfqpoint{0.000000in}{0.000000in}}{\pgfqpoint{0.000000in}{0.000000in}}{%
\pgfpathmoveto{\pgfqpoint{0.000000in}{0.000000in}}%
\pgfpathlineto{\pgfqpoint{0.000000in}{0.000000in}}%
\pgfusepath{stroke,fill}%
}%
\begin{pgfscope}%
\pgfsys@transformshift{1.293594in}{0.516222in}%
\pgfsys@useobject{currentmarker}{}%
\end{pgfscope}%
\end{pgfscope}%
\begin{pgfscope}%
\definecolor{textcolor}{rgb}{0.150000,0.150000,0.150000}%
\pgfsetstrokecolor{textcolor}%
\pgfsetfillcolor{textcolor}%
\pgftext[x=1.293594in,y=0.438444in,,top]{\color{textcolor}\sffamily\fontsize{8.000000}{9.600000}\selectfont 0.4}%
\end{pgfscope}%
\begin{pgfscope}%
\pgfpathrectangle{\pgfqpoint{0.548058in}{0.516222in}}{\pgfqpoint{1.739582in}{1.783528in}} %
\pgfusepath{clip}%
\pgfsetroundcap%
\pgfsetroundjoin%
\pgfsetlinewidth{0.803000pt}%
\definecolor{currentstroke}{rgb}{1.000000,1.000000,1.000000}%
\pgfsetstrokecolor{currentstroke}%
\pgfsetdash{}{0pt}%
\pgfpathmoveto{\pgfqpoint{1.542105in}{0.516222in}}%
\pgfpathlineto{\pgfqpoint{1.542105in}{2.299750in}}%
\pgfusepath{stroke}%
\end{pgfscope}%
\begin{pgfscope}%
\pgfsetbuttcap%
\pgfsetroundjoin%
\definecolor{currentfill}{rgb}{0.150000,0.150000,0.150000}%
\pgfsetfillcolor{currentfill}%
\pgfsetlinewidth{0.803000pt}%
\definecolor{currentstroke}{rgb}{0.150000,0.150000,0.150000}%
\pgfsetstrokecolor{currentstroke}%
\pgfsetdash{}{0pt}%
\pgfsys@defobject{currentmarker}{\pgfqpoint{0.000000in}{0.000000in}}{\pgfqpoint{0.000000in}{0.000000in}}{%
\pgfpathmoveto{\pgfqpoint{0.000000in}{0.000000in}}%
\pgfpathlineto{\pgfqpoint{0.000000in}{0.000000in}}%
\pgfusepath{stroke,fill}%
}%
\begin{pgfscope}%
\pgfsys@transformshift{1.542105in}{0.516222in}%
\pgfsys@useobject{currentmarker}{}%
\end{pgfscope}%
\end{pgfscope}%
\begin{pgfscope}%
\definecolor{textcolor}{rgb}{0.150000,0.150000,0.150000}%
\pgfsetstrokecolor{textcolor}%
\pgfsetfillcolor{textcolor}%
\pgftext[x=1.542105in,y=0.438444in,,top]{\color{textcolor}\sffamily\fontsize{8.000000}{9.600000}\selectfont 0.6}%
\end{pgfscope}%
\begin{pgfscope}%
\pgfpathrectangle{\pgfqpoint{0.548058in}{0.516222in}}{\pgfqpoint{1.739582in}{1.783528in}} %
\pgfusepath{clip}%
\pgfsetroundcap%
\pgfsetroundjoin%
\pgfsetlinewidth{0.803000pt}%
\definecolor{currentstroke}{rgb}{1.000000,1.000000,1.000000}%
\pgfsetstrokecolor{currentstroke}%
\pgfsetdash{}{0pt}%
\pgfpathmoveto{\pgfqpoint{1.790617in}{0.516222in}}%
\pgfpathlineto{\pgfqpoint{1.790617in}{2.299750in}}%
\pgfusepath{stroke}%
\end{pgfscope}%
\begin{pgfscope}%
\pgfsetbuttcap%
\pgfsetroundjoin%
\definecolor{currentfill}{rgb}{0.150000,0.150000,0.150000}%
\pgfsetfillcolor{currentfill}%
\pgfsetlinewidth{0.803000pt}%
\definecolor{currentstroke}{rgb}{0.150000,0.150000,0.150000}%
\pgfsetstrokecolor{currentstroke}%
\pgfsetdash{}{0pt}%
\pgfsys@defobject{currentmarker}{\pgfqpoint{0.000000in}{0.000000in}}{\pgfqpoint{0.000000in}{0.000000in}}{%
\pgfpathmoveto{\pgfqpoint{0.000000in}{0.000000in}}%
\pgfpathlineto{\pgfqpoint{0.000000in}{0.000000in}}%
\pgfusepath{stroke,fill}%
}%
\begin{pgfscope}%
\pgfsys@transformshift{1.790617in}{0.516222in}%
\pgfsys@useobject{currentmarker}{}%
\end{pgfscope}%
\end{pgfscope}%
\begin{pgfscope}%
\definecolor{textcolor}{rgb}{0.150000,0.150000,0.150000}%
\pgfsetstrokecolor{textcolor}%
\pgfsetfillcolor{textcolor}%
\pgftext[x=1.790617in,y=0.438444in,,top]{\color{textcolor}\sffamily\fontsize{8.000000}{9.600000}\selectfont 0.8}%
\end{pgfscope}%
\begin{pgfscope}%
\pgfpathrectangle{\pgfqpoint{0.548058in}{0.516222in}}{\pgfqpoint{1.739582in}{1.783528in}} %
\pgfusepath{clip}%
\pgfsetroundcap%
\pgfsetroundjoin%
\pgfsetlinewidth{0.803000pt}%
\definecolor{currentstroke}{rgb}{1.000000,1.000000,1.000000}%
\pgfsetstrokecolor{currentstroke}%
\pgfsetdash{}{0pt}%
\pgfpathmoveto{\pgfqpoint{2.039129in}{0.516222in}}%
\pgfpathlineto{\pgfqpoint{2.039129in}{2.299750in}}%
\pgfusepath{stroke}%
\end{pgfscope}%
\begin{pgfscope}%
\pgfsetbuttcap%
\pgfsetroundjoin%
\definecolor{currentfill}{rgb}{0.150000,0.150000,0.150000}%
\pgfsetfillcolor{currentfill}%
\pgfsetlinewidth{0.803000pt}%
\definecolor{currentstroke}{rgb}{0.150000,0.150000,0.150000}%
\pgfsetstrokecolor{currentstroke}%
\pgfsetdash{}{0pt}%
\pgfsys@defobject{currentmarker}{\pgfqpoint{0.000000in}{0.000000in}}{\pgfqpoint{0.000000in}{0.000000in}}{%
\pgfpathmoveto{\pgfqpoint{0.000000in}{0.000000in}}%
\pgfpathlineto{\pgfqpoint{0.000000in}{0.000000in}}%
\pgfusepath{stroke,fill}%
}%
\begin{pgfscope}%
\pgfsys@transformshift{2.039129in}{0.516222in}%
\pgfsys@useobject{currentmarker}{}%
\end{pgfscope}%
\end{pgfscope}%
\begin{pgfscope}%
\definecolor{textcolor}{rgb}{0.150000,0.150000,0.150000}%
\pgfsetstrokecolor{textcolor}%
\pgfsetfillcolor{textcolor}%
\pgftext[x=2.039129in,y=0.438444in,,top]{\color{textcolor}\sffamily\fontsize{8.000000}{9.600000}\selectfont 1.0}%
\end{pgfscope}%
\begin{pgfscope}%
\pgfpathrectangle{\pgfqpoint{0.548058in}{0.516222in}}{\pgfqpoint{1.739582in}{1.783528in}} %
\pgfusepath{clip}%
\pgfsetroundcap%
\pgfsetroundjoin%
\pgfsetlinewidth{0.803000pt}%
\definecolor{currentstroke}{rgb}{1.000000,1.000000,1.000000}%
\pgfsetstrokecolor{currentstroke}%
\pgfsetdash{}{0pt}%
\pgfpathmoveto{\pgfqpoint{2.287641in}{0.516222in}}%
\pgfpathlineto{\pgfqpoint{2.287641in}{2.299750in}}%
\pgfusepath{stroke}%
\end{pgfscope}%
\begin{pgfscope}%
\pgfsetbuttcap%
\pgfsetroundjoin%
\definecolor{currentfill}{rgb}{0.150000,0.150000,0.150000}%
\pgfsetfillcolor{currentfill}%
\pgfsetlinewidth{0.803000pt}%
\definecolor{currentstroke}{rgb}{0.150000,0.150000,0.150000}%
\pgfsetstrokecolor{currentstroke}%
\pgfsetdash{}{0pt}%
\pgfsys@defobject{currentmarker}{\pgfqpoint{0.000000in}{0.000000in}}{\pgfqpoint{0.000000in}{0.000000in}}{%
\pgfpathmoveto{\pgfqpoint{0.000000in}{0.000000in}}%
\pgfpathlineto{\pgfqpoint{0.000000in}{0.000000in}}%
\pgfusepath{stroke,fill}%
}%
\begin{pgfscope}%
\pgfsys@transformshift{2.287641in}{0.516222in}%
\pgfsys@useobject{currentmarker}{}%
\end{pgfscope}%
\end{pgfscope}%
\begin{pgfscope}%
\definecolor{textcolor}{rgb}{0.150000,0.150000,0.150000}%
\pgfsetstrokecolor{textcolor}%
\pgfsetfillcolor{textcolor}%
\pgftext[x=2.287641in,y=0.438444in,,top]{\color{textcolor}\sffamily\fontsize{8.000000}{9.600000}\selectfont 1.2}%
\end{pgfscope}%
\begin{pgfscope}%
\definecolor{textcolor}{rgb}{0.150000,0.150000,0.150000}%
\pgfsetstrokecolor{textcolor}%
\pgfsetfillcolor{textcolor}%
\pgftext[x=1.417849in,y=0.273321in,,top]{\color{textcolor}\sffamily\fontsize{8.800000}{10.560000}\selectfont wing tail ratio}%
\end{pgfscope}%
\begin{pgfscope}%
\pgfpathrectangle{\pgfqpoint{0.548058in}{0.516222in}}{\pgfqpoint{1.739582in}{1.783528in}} %
\pgfusepath{clip}%
\pgfsetroundcap%
\pgfsetroundjoin%
\pgfsetlinewidth{0.803000pt}%
\definecolor{currentstroke}{rgb}{1.000000,1.000000,1.000000}%
\pgfsetstrokecolor{currentstroke}%
\pgfsetdash{}{0pt}%
\pgfpathmoveto{\pgfqpoint{0.548058in}{0.516222in}}%
\pgfpathlineto{\pgfqpoint{2.287641in}{0.516222in}}%
\pgfusepath{stroke}%
\end{pgfscope}%
\begin{pgfscope}%
\pgfsetbuttcap%
\pgfsetroundjoin%
\definecolor{currentfill}{rgb}{0.150000,0.150000,0.150000}%
\pgfsetfillcolor{currentfill}%
\pgfsetlinewidth{0.803000pt}%
\definecolor{currentstroke}{rgb}{0.150000,0.150000,0.150000}%
\pgfsetstrokecolor{currentstroke}%
\pgfsetdash{}{0pt}%
\pgfsys@defobject{currentmarker}{\pgfqpoint{0.000000in}{0.000000in}}{\pgfqpoint{0.000000in}{0.000000in}}{%
\pgfpathmoveto{\pgfqpoint{0.000000in}{0.000000in}}%
\pgfpathlineto{\pgfqpoint{0.000000in}{0.000000in}}%
\pgfusepath{stroke,fill}%
}%
\begin{pgfscope}%
\pgfsys@transformshift{0.548058in}{0.516222in}%
\pgfsys@useobject{currentmarker}{}%
\end{pgfscope}%
\end{pgfscope}%
\begin{pgfscope}%
\definecolor{textcolor}{rgb}{0.150000,0.150000,0.150000}%
\pgfsetstrokecolor{textcolor}%
\pgfsetfillcolor{textcolor}%
\pgftext[x=0.470280in,y=0.516222in,right,]{\color{textcolor}\sffamily\fontsize{8.000000}{9.600000}\selectfont 2.5}%
\end{pgfscope}%
\begin{pgfscope}%
\pgfpathrectangle{\pgfqpoint{0.548058in}{0.516222in}}{\pgfqpoint{1.739582in}{1.783528in}} %
\pgfusepath{clip}%
\pgfsetroundcap%
\pgfsetroundjoin%
\pgfsetlinewidth{0.803000pt}%
\definecolor{currentstroke}{rgb}{1.000000,1.000000,1.000000}%
\pgfsetstrokecolor{currentstroke}%
\pgfsetdash{}{0pt}%
\pgfpathmoveto{\pgfqpoint{0.548058in}{0.739163in}}%
\pgfpathlineto{\pgfqpoint{2.287641in}{0.739163in}}%
\pgfusepath{stroke}%
\end{pgfscope}%
\begin{pgfscope}%
\pgfsetbuttcap%
\pgfsetroundjoin%
\definecolor{currentfill}{rgb}{0.150000,0.150000,0.150000}%
\pgfsetfillcolor{currentfill}%
\pgfsetlinewidth{0.803000pt}%
\definecolor{currentstroke}{rgb}{0.150000,0.150000,0.150000}%
\pgfsetstrokecolor{currentstroke}%
\pgfsetdash{}{0pt}%
\pgfsys@defobject{currentmarker}{\pgfqpoint{0.000000in}{0.000000in}}{\pgfqpoint{0.000000in}{0.000000in}}{%
\pgfpathmoveto{\pgfqpoint{0.000000in}{0.000000in}}%
\pgfpathlineto{\pgfqpoint{0.000000in}{0.000000in}}%
\pgfusepath{stroke,fill}%
}%
\begin{pgfscope}%
\pgfsys@transformshift{0.548058in}{0.739163in}%
\pgfsys@useobject{currentmarker}{}%
\end{pgfscope}%
\end{pgfscope}%
\begin{pgfscope}%
\definecolor{textcolor}{rgb}{0.150000,0.150000,0.150000}%
\pgfsetstrokecolor{textcolor}%
\pgfsetfillcolor{textcolor}%
\pgftext[x=0.470280in,y=0.739163in,right,]{\color{textcolor}\sffamily\fontsize{8.000000}{9.600000}\selectfont 3.0}%
\end{pgfscope}%
\begin{pgfscope}%
\pgfpathrectangle{\pgfqpoint{0.548058in}{0.516222in}}{\pgfqpoint{1.739582in}{1.783528in}} %
\pgfusepath{clip}%
\pgfsetroundcap%
\pgfsetroundjoin%
\pgfsetlinewidth{0.803000pt}%
\definecolor{currentstroke}{rgb}{1.000000,1.000000,1.000000}%
\pgfsetstrokecolor{currentstroke}%
\pgfsetdash{}{0pt}%
\pgfpathmoveto{\pgfqpoint{0.548058in}{0.962104in}}%
\pgfpathlineto{\pgfqpoint{2.287641in}{0.962104in}}%
\pgfusepath{stroke}%
\end{pgfscope}%
\begin{pgfscope}%
\pgfsetbuttcap%
\pgfsetroundjoin%
\definecolor{currentfill}{rgb}{0.150000,0.150000,0.150000}%
\pgfsetfillcolor{currentfill}%
\pgfsetlinewidth{0.803000pt}%
\definecolor{currentstroke}{rgb}{0.150000,0.150000,0.150000}%
\pgfsetstrokecolor{currentstroke}%
\pgfsetdash{}{0pt}%
\pgfsys@defobject{currentmarker}{\pgfqpoint{0.000000in}{0.000000in}}{\pgfqpoint{0.000000in}{0.000000in}}{%
\pgfpathmoveto{\pgfqpoint{0.000000in}{0.000000in}}%
\pgfpathlineto{\pgfqpoint{0.000000in}{0.000000in}}%
\pgfusepath{stroke,fill}%
}%
\begin{pgfscope}%
\pgfsys@transformshift{0.548058in}{0.962104in}%
\pgfsys@useobject{currentmarker}{}%
\end{pgfscope}%
\end{pgfscope}%
\begin{pgfscope}%
\definecolor{textcolor}{rgb}{0.150000,0.150000,0.150000}%
\pgfsetstrokecolor{textcolor}%
\pgfsetfillcolor{textcolor}%
\pgftext[x=0.470280in,y=0.962104in,right,]{\color{textcolor}\sffamily\fontsize{8.000000}{9.600000}\selectfont 3.5}%
\end{pgfscope}%
\begin{pgfscope}%
\pgfpathrectangle{\pgfqpoint{0.548058in}{0.516222in}}{\pgfqpoint{1.739582in}{1.783528in}} %
\pgfusepath{clip}%
\pgfsetroundcap%
\pgfsetroundjoin%
\pgfsetlinewidth{0.803000pt}%
\definecolor{currentstroke}{rgb}{1.000000,1.000000,1.000000}%
\pgfsetstrokecolor{currentstroke}%
\pgfsetdash{}{0pt}%
\pgfpathmoveto{\pgfqpoint{0.548058in}{1.185045in}}%
\pgfpathlineto{\pgfqpoint{2.287641in}{1.185045in}}%
\pgfusepath{stroke}%
\end{pgfscope}%
\begin{pgfscope}%
\pgfsetbuttcap%
\pgfsetroundjoin%
\definecolor{currentfill}{rgb}{0.150000,0.150000,0.150000}%
\pgfsetfillcolor{currentfill}%
\pgfsetlinewidth{0.803000pt}%
\definecolor{currentstroke}{rgb}{0.150000,0.150000,0.150000}%
\pgfsetstrokecolor{currentstroke}%
\pgfsetdash{}{0pt}%
\pgfsys@defobject{currentmarker}{\pgfqpoint{0.000000in}{0.000000in}}{\pgfqpoint{0.000000in}{0.000000in}}{%
\pgfpathmoveto{\pgfqpoint{0.000000in}{0.000000in}}%
\pgfpathlineto{\pgfqpoint{0.000000in}{0.000000in}}%
\pgfusepath{stroke,fill}%
}%
\begin{pgfscope}%
\pgfsys@transformshift{0.548058in}{1.185045in}%
\pgfsys@useobject{currentmarker}{}%
\end{pgfscope}%
\end{pgfscope}%
\begin{pgfscope}%
\definecolor{textcolor}{rgb}{0.150000,0.150000,0.150000}%
\pgfsetstrokecolor{textcolor}%
\pgfsetfillcolor{textcolor}%
\pgftext[x=0.470280in,y=1.185045in,right,]{\color{textcolor}\sffamily\fontsize{8.000000}{9.600000}\selectfont 4.0}%
\end{pgfscope}%
\begin{pgfscope}%
\pgfpathrectangle{\pgfqpoint{0.548058in}{0.516222in}}{\pgfqpoint{1.739582in}{1.783528in}} %
\pgfusepath{clip}%
\pgfsetroundcap%
\pgfsetroundjoin%
\pgfsetlinewidth{0.803000pt}%
\definecolor{currentstroke}{rgb}{1.000000,1.000000,1.000000}%
\pgfsetstrokecolor{currentstroke}%
\pgfsetdash{}{0pt}%
\pgfpathmoveto{\pgfqpoint{0.548058in}{1.407986in}}%
\pgfpathlineto{\pgfqpoint{2.287641in}{1.407986in}}%
\pgfusepath{stroke}%
\end{pgfscope}%
\begin{pgfscope}%
\pgfsetbuttcap%
\pgfsetroundjoin%
\definecolor{currentfill}{rgb}{0.150000,0.150000,0.150000}%
\pgfsetfillcolor{currentfill}%
\pgfsetlinewidth{0.803000pt}%
\definecolor{currentstroke}{rgb}{0.150000,0.150000,0.150000}%
\pgfsetstrokecolor{currentstroke}%
\pgfsetdash{}{0pt}%
\pgfsys@defobject{currentmarker}{\pgfqpoint{0.000000in}{0.000000in}}{\pgfqpoint{0.000000in}{0.000000in}}{%
\pgfpathmoveto{\pgfqpoint{0.000000in}{0.000000in}}%
\pgfpathlineto{\pgfqpoint{0.000000in}{0.000000in}}%
\pgfusepath{stroke,fill}%
}%
\begin{pgfscope}%
\pgfsys@transformshift{0.548058in}{1.407986in}%
\pgfsys@useobject{currentmarker}{}%
\end{pgfscope}%
\end{pgfscope}%
\begin{pgfscope}%
\definecolor{textcolor}{rgb}{0.150000,0.150000,0.150000}%
\pgfsetstrokecolor{textcolor}%
\pgfsetfillcolor{textcolor}%
\pgftext[x=0.470280in,y=1.407986in,right,]{\color{textcolor}\sffamily\fontsize{8.000000}{9.600000}\selectfont 4.5}%
\end{pgfscope}%
\begin{pgfscope}%
\pgfpathrectangle{\pgfqpoint{0.548058in}{0.516222in}}{\pgfqpoint{1.739582in}{1.783528in}} %
\pgfusepath{clip}%
\pgfsetroundcap%
\pgfsetroundjoin%
\pgfsetlinewidth{0.803000pt}%
\definecolor{currentstroke}{rgb}{1.000000,1.000000,1.000000}%
\pgfsetstrokecolor{currentstroke}%
\pgfsetdash{}{0pt}%
\pgfpathmoveto{\pgfqpoint{0.548058in}{1.630927in}}%
\pgfpathlineto{\pgfqpoint{2.287641in}{1.630927in}}%
\pgfusepath{stroke}%
\end{pgfscope}%
\begin{pgfscope}%
\pgfsetbuttcap%
\pgfsetroundjoin%
\definecolor{currentfill}{rgb}{0.150000,0.150000,0.150000}%
\pgfsetfillcolor{currentfill}%
\pgfsetlinewidth{0.803000pt}%
\definecolor{currentstroke}{rgb}{0.150000,0.150000,0.150000}%
\pgfsetstrokecolor{currentstroke}%
\pgfsetdash{}{0pt}%
\pgfsys@defobject{currentmarker}{\pgfqpoint{0.000000in}{0.000000in}}{\pgfqpoint{0.000000in}{0.000000in}}{%
\pgfpathmoveto{\pgfqpoint{0.000000in}{0.000000in}}%
\pgfpathlineto{\pgfqpoint{0.000000in}{0.000000in}}%
\pgfusepath{stroke,fill}%
}%
\begin{pgfscope}%
\pgfsys@transformshift{0.548058in}{1.630927in}%
\pgfsys@useobject{currentmarker}{}%
\end{pgfscope}%
\end{pgfscope}%
\begin{pgfscope}%
\definecolor{textcolor}{rgb}{0.150000,0.150000,0.150000}%
\pgfsetstrokecolor{textcolor}%
\pgfsetfillcolor{textcolor}%
\pgftext[x=0.470280in,y=1.630927in,right,]{\color{textcolor}\sffamily\fontsize{8.000000}{9.600000}\selectfont 5.0}%
\end{pgfscope}%
\begin{pgfscope}%
\pgfpathrectangle{\pgfqpoint{0.548058in}{0.516222in}}{\pgfqpoint{1.739582in}{1.783528in}} %
\pgfusepath{clip}%
\pgfsetroundcap%
\pgfsetroundjoin%
\pgfsetlinewidth{0.803000pt}%
\definecolor{currentstroke}{rgb}{1.000000,1.000000,1.000000}%
\pgfsetstrokecolor{currentstroke}%
\pgfsetdash{}{0pt}%
\pgfpathmoveto{\pgfqpoint{0.548058in}{1.853868in}}%
\pgfpathlineto{\pgfqpoint{2.287641in}{1.853868in}}%
\pgfusepath{stroke}%
\end{pgfscope}%
\begin{pgfscope}%
\pgfsetbuttcap%
\pgfsetroundjoin%
\definecolor{currentfill}{rgb}{0.150000,0.150000,0.150000}%
\pgfsetfillcolor{currentfill}%
\pgfsetlinewidth{0.803000pt}%
\definecolor{currentstroke}{rgb}{0.150000,0.150000,0.150000}%
\pgfsetstrokecolor{currentstroke}%
\pgfsetdash{}{0pt}%
\pgfsys@defobject{currentmarker}{\pgfqpoint{0.000000in}{0.000000in}}{\pgfqpoint{0.000000in}{0.000000in}}{%
\pgfpathmoveto{\pgfqpoint{0.000000in}{0.000000in}}%
\pgfpathlineto{\pgfqpoint{0.000000in}{0.000000in}}%
\pgfusepath{stroke,fill}%
}%
\begin{pgfscope}%
\pgfsys@transformshift{0.548058in}{1.853868in}%
\pgfsys@useobject{currentmarker}{}%
\end{pgfscope}%
\end{pgfscope}%
\begin{pgfscope}%
\definecolor{textcolor}{rgb}{0.150000,0.150000,0.150000}%
\pgfsetstrokecolor{textcolor}%
\pgfsetfillcolor{textcolor}%
\pgftext[x=0.470280in,y=1.853868in,right,]{\color{textcolor}\sffamily\fontsize{8.000000}{9.600000}\selectfont 5.5}%
\end{pgfscope}%
\begin{pgfscope}%
\pgfpathrectangle{\pgfqpoint{0.548058in}{0.516222in}}{\pgfqpoint{1.739582in}{1.783528in}} %
\pgfusepath{clip}%
\pgfsetroundcap%
\pgfsetroundjoin%
\pgfsetlinewidth{0.803000pt}%
\definecolor{currentstroke}{rgb}{1.000000,1.000000,1.000000}%
\pgfsetstrokecolor{currentstroke}%
\pgfsetdash{}{0pt}%
\pgfpathmoveto{\pgfqpoint{0.548058in}{2.076809in}}%
\pgfpathlineto{\pgfqpoint{2.287641in}{2.076809in}}%
\pgfusepath{stroke}%
\end{pgfscope}%
\begin{pgfscope}%
\pgfsetbuttcap%
\pgfsetroundjoin%
\definecolor{currentfill}{rgb}{0.150000,0.150000,0.150000}%
\pgfsetfillcolor{currentfill}%
\pgfsetlinewidth{0.803000pt}%
\definecolor{currentstroke}{rgb}{0.150000,0.150000,0.150000}%
\pgfsetstrokecolor{currentstroke}%
\pgfsetdash{}{0pt}%
\pgfsys@defobject{currentmarker}{\pgfqpoint{0.000000in}{0.000000in}}{\pgfqpoint{0.000000in}{0.000000in}}{%
\pgfpathmoveto{\pgfqpoint{0.000000in}{0.000000in}}%
\pgfpathlineto{\pgfqpoint{0.000000in}{0.000000in}}%
\pgfusepath{stroke,fill}%
}%
\begin{pgfscope}%
\pgfsys@transformshift{0.548058in}{2.076809in}%
\pgfsys@useobject{currentmarker}{}%
\end{pgfscope}%
\end{pgfscope}%
\begin{pgfscope}%
\definecolor{textcolor}{rgb}{0.150000,0.150000,0.150000}%
\pgfsetstrokecolor{textcolor}%
\pgfsetfillcolor{textcolor}%
\pgftext[x=0.470280in,y=2.076809in,right,]{\color{textcolor}\sffamily\fontsize{8.000000}{9.600000}\selectfont 6.0}%
\end{pgfscope}%
\begin{pgfscope}%
\pgfpathrectangle{\pgfqpoint{0.548058in}{0.516222in}}{\pgfqpoint{1.739582in}{1.783528in}} %
\pgfusepath{clip}%
\pgfsetroundcap%
\pgfsetroundjoin%
\pgfsetlinewidth{0.803000pt}%
\definecolor{currentstroke}{rgb}{1.000000,1.000000,1.000000}%
\pgfsetstrokecolor{currentstroke}%
\pgfsetdash{}{0pt}%
\pgfpathmoveto{\pgfqpoint{0.548058in}{2.299750in}}%
\pgfpathlineto{\pgfqpoint{2.287641in}{2.299750in}}%
\pgfusepath{stroke}%
\end{pgfscope}%
\begin{pgfscope}%
\pgfsetbuttcap%
\pgfsetroundjoin%
\definecolor{currentfill}{rgb}{0.150000,0.150000,0.150000}%
\pgfsetfillcolor{currentfill}%
\pgfsetlinewidth{0.803000pt}%
\definecolor{currentstroke}{rgb}{0.150000,0.150000,0.150000}%
\pgfsetstrokecolor{currentstroke}%
\pgfsetdash{}{0pt}%
\pgfsys@defobject{currentmarker}{\pgfqpoint{0.000000in}{0.000000in}}{\pgfqpoint{0.000000in}{0.000000in}}{%
\pgfpathmoveto{\pgfqpoint{0.000000in}{0.000000in}}%
\pgfpathlineto{\pgfqpoint{0.000000in}{0.000000in}}%
\pgfusepath{stroke,fill}%
}%
\begin{pgfscope}%
\pgfsys@transformshift{0.548058in}{2.299750in}%
\pgfsys@useobject{currentmarker}{}%
\end{pgfscope}%
\end{pgfscope}%
\begin{pgfscope}%
\definecolor{textcolor}{rgb}{0.150000,0.150000,0.150000}%
\pgfsetstrokecolor{textcolor}%
\pgfsetfillcolor{textcolor}%
\pgftext[x=0.470280in,y=2.299750in,right,]{\color{textcolor}\sffamily\fontsize{8.000000}{9.600000}\selectfont 6.5}%
\end{pgfscope}%
\begin{pgfscope}%
\definecolor{textcolor}{rgb}{0.150000,0.150000,0.150000}%
\pgfsetstrokecolor{textcolor}%
\pgfsetfillcolor{textcolor}%
\pgftext[x=0.242888in,y=1.407986in,,bottom,rotate=90.000000]{\color{textcolor}\sffamily\fontsize{8.800000}{10.560000}\selectfont fall time exp 2 obs 1}%
\end{pgfscope}%
\begin{pgfscope}%
\pgfpathrectangle{\pgfqpoint{0.548058in}{0.516222in}}{\pgfqpoint{1.739582in}{1.783528in}} %
\pgfusepath{clip}%
\pgfsetbuttcap%
\pgfsetroundjoin%
\definecolor{currentfill}{rgb}{0.298039,0.447059,0.690196}%
\pgfsetfillcolor{currentfill}%
\pgfsetlinewidth{0.240900pt}%
\definecolor{currentstroke}{rgb}{1.000000,1.000000,1.000000}%
\pgfsetstrokecolor{currentstroke}%
\pgfsetdash{}{0pt}%
\pgfpathmoveto{\pgfqpoint{0.796570in}{1.180742in}}%
\pgfpathcurveto{\pgfqpoint{0.804806in}{1.180742in}}{\pgfqpoint{0.812706in}{1.184014in}}{\pgfqpoint{0.818530in}{1.189838in}}%
\pgfpathcurveto{\pgfqpoint{0.824354in}{1.195662in}}{\pgfqpoint{0.827626in}{1.203562in}}{\pgfqpoint{0.827626in}{1.211798in}}%
\pgfpathcurveto{\pgfqpoint{0.827626in}{1.220034in}}{\pgfqpoint{0.824354in}{1.227934in}}{\pgfqpoint{0.818530in}{1.233758in}}%
\pgfpathcurveto{\pgfqpoint{0.812706in}{1.239582in}}{\pgfqpoint{0.804806in}{1.242855in}}{\pgfqpoint{0.796570in}{1.242855in}}%
\pgfpathcurveto{\pgfqpoint{0.788334in}{1.242855in}}{\pgfqpoint{0.780434in}{1.239582in}}{\pgfqpoint{0.774610in}{1.233758in}}%
\pgfpathcurveto{\pgfqpoint{0.768786in}{1.227934in}}{\pgfqpoint{0.765513in}{1.220034in}}{\pgfqpoint{0.765513in}{1.211798in}}%
\pgfpathcurveto{\pgfqpoint{0.765513in}{1.203562in}}{\pgfqpoint{0.768786in}{1.195662in}}{\pgfqpoint{0.774610in}{1.189838in}}%
\pgfpathcurveto{\pgfqpoint{0.780434in}{1.184014in}}{\pgfqpoint{0.788334in}{1.180742in}}{\pgfqpoint{0.796570in}{1.180742in}}%
\pgfpathlineto{\pgfqpoint{0.796570in}{1.180742in}}%
\pgfusepath{stroke,fill}%
\end{pgfscope}%
\begin{pgfscope}%
\pgfpathrectangle{\pgfqpoint{0.548058in}{0.516222in}}{\pgfqpoint{1.739582in}{1.783528in}} %
\pgfusepath{clip}%
\pgfsetbuttcap%
\pgfsetroundjoin%
\definecolor{currentfill}{rgb}{0.298039,0.447059,0.690196}%
\pgfsetfillcolor{currentfill}%
\pgfsetlinewidth{0.240900pt}%
\definecolor{currentstroke}{rgb}{1.000000,1.000000,1.000000}%
\pgfsetstrokecolor{currentstroke}%
\pgfsetdash{}{0pt}%
\pgfpathmoveto{\pgfqpoint{1.019593in}{1.064812in}}%
\pgfpathcurveto{\pgfqpoint{1.027830in}{1.064812in}}{\pgfqpoint{1.035730in}{1.068085in}}{\pgfqpoint{1.041554in}{1.073908in}}%
\pgfpathcurveto{\pgfqpoint{1.047378in}{1.079732in}}{\pgfqpoint{1.050650in}{1.087632in}}{\pgfqpoint{1.050650in}{1.095869in}}%
\pgfpathcurveto{\pgfqpoint{1.050650in}{1.104105in}}{\pgfqpoint{1.047378in}{1.112005in}}{\pgfqpoint{1.041554in}{1.117829in}}%
\pgfpathcurveto{\pgfqpoint{1.035730in}{1.123653in}}{\pgfqpoint{1.027830in}{1.126925in}}{\pgfqpoint{1.019593in}{1.126925in}}%
\pgfpathcurveto{\pgfqpoint{1.011357in}{1.126925in}}{\pgfqpoint{1.003457in}{1.123653in}}{\pgfqpoint{0.997633in}{1.117829in}}%
\pgfpathcurveto{\pgfqpoint{0.991809in}{1.112005in}}{\pgfqpoint{0.988537in}{1.104105in}}{\pgfqpoint{0.988537in}{1.095869in}}%
\pgfpathcurveto{\pgfqpoint{0.988537in}{1.087632in}}{\pgfqpoint{0.991809in}{1.079732in}}{\pgfqpoint{0.997633in}{1.073908in}}%
\pgfpathcurveto{\pgfqpoint{1.003457in}{1.068085in}}{\pgfqpoint{1.011357in}{1.064812in}}{\pgfqpoint{1.019593in}{1.064812in}}%
\pgfpathlineto{\pgfqpoint{1.019593in}{1.064812in}}%
\pgfusepath{stroke,fill}%
\end{pgfscope}%
\begin{pgfscope}%
\pgfpathrectangle{\pgfqpoint{0.548058in}{0.516222in}}{\pgfqpoint{1.739582in}{1.783528in}} %
\pgfusepath{clip}%
\pgfsetbuttcap%
\pgfsetroundjoin%
\definecolor{currentfill}{rgb}{0.298039,0.447059,0.690196}%
\pgfsetfillcolor{currentfill}%
\pgfsetlinewidth{0.240900pt}%
\definecolor{currentstroke}{rgb}{1.000000,1.000000,1.000000}%
\pgfsetstrokecolor{currentstroke}%
\pgfsetdash{}{0pt}%
\pgfpathmoveto{\pgfqpoint{1.847966in}{1.015765in}}%
\pgfpathcurveto{\pgfqpoint{1.856202in}{1.015765in}}{\pgfqpoint{1.864102in}{1.019038in}}{\pgfqpoint{1.869926in}{1.024861in}}%
\pgfpathcurveto{\pgfqpoint{1.875750in}{1.030685in}}{\pgfqpoint{1.879022in}{1.038585in}}{\pgfqpoint{1.879022in}{1.046822in}}%
\pgfpathcurveto{\pgfqpoint{1.879022in}{1.055058in}}{\pgfqpoint{1.875750in}{1.062958in}}{\pgfqpoint{1.869926in}{1.068782in}}%
\pgfpathcurveto{\pgfqpoint{1.864102in}{1.074606in}}{\pgfqpoint{1.856202in}{1.077878in}}{\pgfqpoint{1.847966in}{1.077878in}}%
\pgfpathcurveto{\pgfqpoint{1.839730in}{1.077878in}}{\pgfqpoint{1.831830in}{1.074606in}}{\pgfqpoint{1.826006in}{1.068782in}}%
\pgfpathcurveto{\pgfqpoint{1.820182in}{1.062958in}}{\pgfqpoint{1.816909in}{1.055058in}}{\pgfqpoint{1.816909in}{1.046822in}}%
\pgfpathcurveto{\pgfqpoint{1.816909in}{1.038585in}}{\pgfqpoint{1.820182in}{1.030685in}}{\pgfqpoint{1.826006in}{1.024861in}}%
\pgfpathcurveto{\pgfqpoint{1.831830in}{1.019038in}}{\pgfqpoint{1.839730in}{1.015765in}}{\pgfqpoint{1.847966in}{1.015765in}}%
\pgfpathlineto{\pgfqpoint{1.847966in}{1.015765in}}%
\pgfusepath{stroke,fill}%
\end{pgfscope}%
\begin{pgfscope}%
\pgfpathrectangle{\pgfqpoint{0.548058in}{0.516222in}}{\pgfqpoint{1.739582in}{1.783528in}} %
\pgfusepath{clip}%
\pgfsetbuttcap%
\pgfsetroundjoin%
\definecolor{currentfill}{rgb}{0.298039,0.447059,0.690196}%
\pgfsetfillcolor{currentfill}%
\pgfsetlinewidth{0.240900pt}%
\definecolor{currentstroke}{rgb}{1.000000,1.000000,1.000000}%
\pgfsetstrokecolor{currentstroke}%
\pgfsetdash{}{0pt}%
\pgfpathmoveto{\pgfqpoint{2.039129in}{0.953342in}}%
\pgfpathcurveto{\pgfqpoint{2.047365in}{0.953342in}}{\pgfqpoint{2.055265in}{0.956614in}}{\pgfqpoint{2.061089in}{0.962438in}}%
\pgfpathcurveto{\pgfqpoint{2.066913in}{0.968262in}}{\pgfqpoint{2.070185in}{0.976162in}}{\pgfqpoint{2.070185in}{0.984398in}}%
\pgfpathcurveto{\pgfqpoint{2.070185in}{0.992635in}}{\pgfqpoint{2.066913in}{1.000535in}}{\pgfqpoint{2.061089in}{1.006359in}}%
\pgfpathcurveto{\pgfqpoint{2.055265in}{1.012182in}}{\pgfqpoint{2.047365in}{1.015455in}}{\pgfqpoint{2.039129in}{1.015455in}}%
\pgfpathcurveto{\pgfqpoint{2.030893in}{1.015455in}}{\pgfqpoint{2.022993in}{1.012182in}}{\pgfqpoint{2.017169in}{1.006359in}}%
\pgfpathcurveto{\pgfqpoint{2.011345in}{1.000535in}}{\pgfqpoint{2.008072in}{0.992635in}}{\pgfqpoint{2.008072in}{0.984398in}}%
\pgfpathcurveto{\pgfqpoint{2.008072in}{0.976162in}}{\pgfqpoint{2.011345in}{0.968262in}}{\pgfqpoint{2.017169in}{0.962438in}}%
\pgfpathcurveto{\pgfqpoint{2.022993in}{0.956614in}}{\pgfqpoint{2.030893in}{0.953342in}}{\pgfqpoint{2.039129in}{0.953342in}}%
\pgfpathlineto{\pgfqpoint{2.039129in}{0.953342in}}%
\pgfusepath{stroke,fill}%
\end{pgfscope}%
\begin{pgfscope}%
\pgfpathrectangle{\pgfqpoint{0.548058in}{0.516222in}}{\pgfqpoint{1.739582in}{1.783528in}} %
\pgfusepath{clip}%
\pgfsetbuttcap%
\pgfsetroundjoin%
\definecolor{currentfill}{rgb}{0.298039,0.447059,0.690196}%
\pgfsetfillcolor{currentfill}%
\pgfsetlinewidth{0.240900pt}%
\definecolor{currentstroke}{rgb}{1.000000,1.000000,1.000000}%
\pgfsetstrokecolor{currentstroke}%
\pgfsetdash{}{0pt}%
\pgfpathmoveto{\pgfqpoint{1.656803in}{0.574342in}}%
\pgfpathcurveto{\pgfqpoint{1.665039in}{0.574342in}}{\pgfqpoint{1.672939in}{0.577614in}}{\pgfqpoint{1.678763in}{0.583438in}}%
\pgfpathcurveto{\pgfqpoint{1.684587in}{0.589262in}}{\pgfqpoint{1.687860in}{0.597162in}}{\pgfqpoint{1.687860in}{0.605399in}}%
\pgfpathcurveto{\pgfqpoint{1.687860in}{0.613635in}}{\pgfqpoint{1.684587in}{0.621535in}}{\pgfqpoint{1.678763in}{0.627359in}}%
\pgfpathcurveto{\pgfqpoint{1.672939in}{0.633183in}}{\pgfqpoint{1.665039in}{0.636455in}}{\pgfqpoint{1.656803in}{0.636455in}}%
\pgfpathcurveto{\pgfqpoint{1.648567in}{0.636455in}}{\pgfqpoint{1.640667in}{0.633183in}}{\pgfqpoint{1.634843in}{0.627359in}}%
\pgfpathcurveto{\pgfqpoint{1.629019in}{0.621535in}}{\pgfqpoint{1.625747in}{0.613635in}}{\pgfqpoint{1.625747in}{0.605399in}}%
\pgfpathcurveto{\pgfqpoint{1.625747in}{0.597162in}}{\pgfqpoint{1.629019in}{0.589262in}}{\pgfqpoint{1.634843in}{0.583438in}}%
\pgfpathcurveto{\pgfqpoint{1.640667in}{0.577614in}}{\pgfqpoint{1.648567in}{0.574342in}}{\pgfqpoint{1.656803in}{0.574342in}}%
\pgfpathlineto{\pgfqpoint{1.656803in}{0.574342in}}%
\pgfusepath{stroke,fill}%
\end{pgfscope}%
\begin{pgfscope}%
\pgfpathrectangle{\pgfqpoint{0.548058in}{0.516222in}}{\pgfqpoint{1.739582in}{1.783528in}} %
\pgfusepath{clip}%
\pgfsetbuttcap%
\pgfsetroundjoin%
\definecolor{currentfill}{rgb}{0.298039,0.447059,0.690196}%
\pgfsetfillcolor{currentfill}%
\pgfsetlinewidth{0.240900pt}%
\definecolor{currentstroke}{rgb}{1.000000,1.000000,1.000000}%
\pgfsetstrokecolor{currentstroke}%
\pgfsetdash{}{0pt}%
\pgfpathmoveto{\pgfqpoint{1.816105in}{0.873083in}}%
\pgfpathcurveto{\pgfqpoint{1.824342in}{0.873083in}}{\pgfqpoint{1.832242in}{0.876355in}}{\pgfqpoint{1.838066in}{0.882179in}}%
\pgfpathcurveto{\pgfqpoint{1.843890in}{0.888003in}}{\pgfqpoint{1.847162in}{0.895903in}}{\pgfqpoint{1.847162in}{0.904140in}}%
\pgfpathcurveto{\pgfqpoint{1.847162in}{0.912376in}}{\pgfqpoint{1.843890in}{0.920276in}}{\pgfqpoint{1.838066in}{0.926100in}}%
\pgfpathcurveto{\pgfqpoint{1.832242in}{0.931924in}}{\pgfqpoint{1.824342in}{0.935196in}}{\pgfqpoint{1.816105in}{0.935196in}}%
\pgfpathcurveto{\pgfqpoint{1.807869in}{0.935196in}}{\pgfqpoint{1.799969in}{0.931924in}}{\pgfqpoint{1.794145in}{0.926100in}}%
\pgfpathcurveto{\pgfqpoint{1.788321in}{0.920276in}}{\pgfqpoint{1.785049in}{0.912376in}}{\pgfqpoint{1.785049in}{0.904140in}}%
\pgfpathcurveto{\pgfqpoint{1.785049in}{0.895903in}}{\pgfqpoint{1.788321in}{0.888003in}}{\pgfqpoint{1.794145in}{0.882179in}}%
\pgfpathcurveto{\pgfqpoint{1.799969in}{0.876355in}}{\pgfqpoint{1.807869in}{0.873083in}}{\pgfqpoint{1.816105in}{0.873083in}}%
\pgfpathlineto{\pgfqpoint{1.816105in}{0.873083in}}%
\pgfusepath{stroke,fill}%
\end{pgfscope}%
\begin{pgfscope}%
\pgfpathrectangle{\pgfqpoint{0.548058in}{0.516222in}}{\pgfqpoint{1.739582in}{1.783528in}} %
\pgfusepath{clip}%
\pgfsetbuttcap%
\pgfsetroundjoin%
\definecolor{currentfill}{rgb}{0.298039,0.447059,0.690196}%
\pgfsetfillcolor{currentfill}%
\pgfsetlinewidth{0.240900pt}%
\definecolor{currentstroke}{rgb}{1.000000,1.000000,1.000000}%
\pgfsetstrokecolor{currentstroke}%
\pgfsetdash{}{0pt}%
\pgfpathmoveto{\pgfqpoint{1.051454in}{1.087106in}}%
\pgfpathcurveto{\pgfqpoint{1.059690in}{1.087106in}}{\pgfqpoint{1.067590in}{1.090379in}}{\pgfqpoint{1.073414in}{1.096203in}}%
\pgfpathcurveto{\pgfqpoint{1.079238in}{1.102027in}}{\pgfqpoint{1.082510in}{1.109927in}}{\pgfqpoint{1.082510in}{1.118163in}}%
\pgfpathcurveto{\pgfqpoint{1.082510in}{1.126399in}}{\pgfqpoint{1.079238in}{1.134299in}}{\pgfqpoint{1.073414in}{1.140123in}}%
\pgfpathcurveto{\pgfqpoint{1.067590in}{1.145947in}}{\pgfqpoint{1.059690in}{1.149219in}}{\pgfqpoint{1.051454in}{1.149219in}}%
\pgfpathcurveto{\pgfqpoint{1.043218in}{1.149219in}}{\pgfqpoint{1.035317in}{1.145947in}}{\pgfqpoint{1.029494in}{1.140123in}}%
\pgfpathcurveto{\pgfqpoint{1.023670in}{1.134299in}}{\pgfqpoint{1.020397in}{1.126399in}}{\pgfqpoint{1.020397in}{1.118163in}}%
\pgfpathcurveto{\pgfqpoint{1.020397in}{1.109927in}}{\pgfqpoint{1.023670in}{1.102027in}}{\pgfqpoint{1.029494in}{1.096203in}}%
\pgfpathcurveto{\pgfqpoint{1.035317in}{1.090379in}}{\pgfqpoint{1.043218in}{1.087106in}}{\pgfqpoint{1.051454in}{1.087106in}}%
\pgfpathlineto{\pgfqpoint{1.051454in}{1.087106in}}%
\pgfusepath{stroke,fill}%
\end{pgfscope}%
\begin{pgfscope}%
\pgfpathrectangle{\pgfqpoint{0.548058in}{0.516222in}}{\pgfqpoint{1.739582in}{1.783528in}} %
\pgfusepath{clip}%
\pgfsetbuttcap%
\pgfsetroundjoin%
\definecolor{currentfill}{rgb}{0.298039,0.447059,0.690196}%
\pgfsetfillcolor{currentfill}%
\pgfsetlinewidth{0.240900pt}%
\definecolor{currentstroke}{rgb}{1.000000,1.000000,1.000000}%
\pgfsetstrokecolor{currentstroke}%
\pgfsetdash{}{0pt}%
\pgfpathmoveto{\pgfqpoint{1.433780in}{1.376930in}}%
\pgfpathcurveto{\pgfqpoint{1.442016in}{1.376930in}}{\pgfqpoint{1.449916in}{1.380202in}}{\pgfqpoint{1.455740in}{1.386026in}}%
\pgfpathcurveto{\pgfqpoint{1.461564in}{1.391850in}}{\pgfqpoint{1.464836in}{1.399750in}}{\pgfqpoint{1.464836in}{1.407986in}}%
\pgfpathcurveto{\pgfqpoint{1.464836in}{1.416222in}}{\pgfqpoint{1.461564in}{1.424122in}}{\pgfqpoint{1.455740in}{1.429946in}}%
\pgfpathcurveto{\pgfqpoint{1.449916in}{1.435770in}}{\pgfqpoint{1.442016in}{1.439043in}}{\pgfqpoint{1.433780in}{1.439043in}}%
\pgfpathcurveto{\pgfqpoint{1.425543in}{1.439043in}}{\pgfqpoint{1.417643in}{1.435770in}}{\pgfqpoint{1.411819in}{1.429946in}}%
\pgfpathcurveto{\pgfqpoint{1.405995in}{1.424122in}}{\pgfqpoint{1.402723in}{1.416222in}}{\pgfqpoint{1.402723in}{1.407986in}}%
\pgfpathcurveto{\pgfqpoint{1.402723in}{1.399750in}}{\pgfqpoint{1.405995in}{1.391850in}}{\pgfqpoint{1.411819in}{1.386026in}}%
\pgfpathcurveto{\pgfqpoint{1.417643in}{1.380202in}}{\pgfqpoint{1.425543in}{1.376930in}}{\pgfqpoint{1.433780in}{1.376930in}}%
\pgfpathlineto{\pgfqpoint{1.433780in}{1.376930in}}%
\pgfusepath{stroke,fill}%
\end{pgfscope}%
\begin{pgfscope}%
\pgfpathrectangle{\pgfqpoint{0.548058in}{0.516222in}}{\pgfqpoint{1.739582in}{1.783528in}} %
\pgfusepath{clip}%
\pgfsetbuttcap%
\pgfsetroundjoin%
\definecolor{currentfill}{rgb}{0.298039,0.447059,0.690196}%
\pgfsetfillcolor{currentfill}%
\pgfsetlinewidth{0.240900pt}%
\definecolor{currentstroke}{rgb}{1.000000,1.000000,1.000000}%
\pgfsetstrokecolor{currentstroke}%
\pgfsetdash{}{0pt}%
\pgfpathmoveto{\pgfqpoint{1.370059in}{1.372471in}}%
\pgfpathcurveto{\pgfqpoint{1.378295in}{1.372471in}}{\pgfqpoint{1.386195in}{1.375743in}}{\pgfqpoint{1.392019in}{1.381567in}}%
\pgfpathcurveto{\pgfqpoint{1.397843in}{1.387391in}}{\pgfqpoint{1.401115in}{1.395291in}}{\pgfqpoint{1.401115in}{1.403527in}}%
\pgfpathcurveto{\pgfqpoint{1.401115in}{1.411764in}}{\pgfqpoint{1.397843in}{1.419664in}}{\pgfqpoint{1.392019in}{1.425488in}}%
\pgfpathcurveto{\pgfqpoint{1.386195in}{1.431311in}}{\pgfqpoint{1.378295in}{1.434584in}}{\pgfqpoint{1.370059in}{1.434584in}}%
\pgfpathcurveto{\pgfqpoint{1.361822in}{1.434584in}}{\pgfqpoint{1.353922in}{1.431311in}}{\pgfqpoint{1.348098in}{1.425488in}}%
\pgfpathcurveto{\pgfqpoint{1.342274in}{1.419664in}}{\pgfqpoint{1.339002in}{1.411764in}}{\pgfqpoint{1.339002in}{1.403527in}}%
\pgfpathcurveto{\pgfqpoint{1.339002in}{1.395291in}}{\pgfqpoint{1.342274in}{1.387391in}}{\pgfqpoint{1.348098in}{1.381567in}}%
\pgfpathcurveto{\pgfqpoint{1.353922in}{1.375743in}}{\pgfqpoint{1.361822in}{1.372471in}}{\pgfqpoint{1.370059in}{1.372471in}}%
\pgfpathlineto{\pgfqpoint{1.370059in}{1.372471in}}%
\pgfusepath{stroke,fill}%
\end{pgfscope}%
\begin{pgfscope}%
\pgfpathrectangle{\pgfqpoint{0.548058in}{0.516222in}}{\pgfqpoint{1.739582in}{1.783528in}} %
\pgfusepath{clip}%
\pgfsetbuttcap%
\pgfsetroundjoin%
\definecolor{currentfill}{rgb}{0.298039,0.447059,0.690196}%
\pgfsetfillcolor{currentfill}%
\pgfsetlinewidth{0.240900pt}%
\definecolor{currentstroke}{rgb}{1.000000,1.000000,1.000000}%
\pgfsetstrokecolor{currentstroke}%
\pgfsetdash{}{0pt}%
\pgfpathmoveto{\pgfqpoint{1.083314in}{0.788365in}}%
\pgfpathcurveto{\pgfqpoint{1.091551in}{0.788365in}}{\pgfqpoint{1.099451in}{0.791638in}}{\pgfqpoint{1.105275in}{0.797462in}}%
\pgfpathcurveto{\pgfqpoint{1.111098in}{0.803286in}}{\pgfqpoint{1.114371in}{0.811186in}}{\pgfqpoint{1.114371in}{0.819422in}}%
\pgfpathcurveto{\pgfqpoint{1.114371in}{0.827658in}}{\pgfqpoint{1.111098in}{0.835558in}}{\pgfqpoint{1.105275in}{0.841382in}}%
\pgfpathcurveto{\pgfqpoint{1.099451in}{0.847206in}}{\pgfqpoint{1.091551in}{0.850478in}}{\pgfqpoint{1.083314in}{0.850478in}}%
\pgfpathcurveto{\pgfqpoint{1.075078in}{0.850478in}}{\pgfqpoint{1.067178in}{0.847206in}}{\pgfqpoint{1.061354in}{0.841382in}}%
\pgfpathcurveto{\pgfqpoint{1.055530in}{0.835558in}}{\pgfqpoint{1.052258in}{0.827658in}}{\pgfqpoint{1.052258in}{0.819422in}}%
\pgfpathcurveto{\pgfqpoint{1.052258in}{0.811186in}}{\pgfqpoint{1.055530in}{0.803286in}}{\pgfqpoint{1.061354in}{0.797462in}}%
\pgfpathcurveto{\pgfqpoint{1.067178in}{0.791638in}}{\pgfqpoint{1.075078in}{0.788365in}}{\pgfqpoint{1.083314in}{0.788365in}}%
\pgfpathlineto{\pgfqpoint{1.083314in}{0.788365in}}%
\pgfusepath{stroke,fill}%
\end{pgfscope}%
\begin{pgfscope}%
\pgfpathrectangle{\pgfqpoint{0.548058in}{0.516222in}}{\pgfqpoint{1.739582in}{1.783528in}} %
\pgfusepath{clip}%
\pgfsetbuttcap%
\pgfsetroundjoin%
\definecolor{currentfill}{rgb}{0.298039,0.447059,0.690196}%
\pgfsetfillcolor{currentfill}%
\pgfsetlinewidth{0.240900pt}%
\definecolor{currentstroke}{rgb}{1.000000,1.000000,1.000000}%
\pgfsetstrokecolor{currentstroke}%
\pgfsetdash{}{0pt}%
\pgfpathmoveto{\pgfqpoint{1.497501in}{1.176283in}}%
\pgfpathcurveto{\pgfqpoint{1.505737in}{1.176283in}}{\pgfqpoint{1.513637in}{1.179555in}}{\pgfqpoint{1.519461in}{1.185379in}}%
\pgfpathcurveto{\pgfqpoint{1.525285in}{1.191203in}}{\pgfqpoint{1.528557in}{1.199103in}}{\pgfqpoint{1.528557in}{1.207339in}}%
\pgfpathcurveto{\pgfqpoint{1.528557in}{1.215576in}}{\pgfqpoint{1.525285in}{1.223476in}}{\pgfqpoint{1.519461in}{1.229299in}}%
\pgfpathcurveto{\pgfqpoint{1.513637in}{1.235123in}}{\pgfqpoint{1.505737in}{1.238396in}}{\pgfqpoint{1.497501in}{1.238396in}}%
\pgfpathcurveto{\pgfqpoint{1.489264in}{1.238396in}}{\pgfqpoint{1.481364in}{1.235123in}}{\pgfqpoint{1.475540in}{1.229299in}}%
\pgfpathcurveto{\pgfqpoint{1.469716in}{1.223476in}}{\pgfqpoint{1.466444in}{1.215576in}}{\pgfqpoint{1.466444in}{1.207339in}}%
\pgfpathcurveto{\pgfqpoint{1.466444in}{1.199103in}}{\pgfqpoint{1.469716in}{1.191203in}}{\pgfqpoint{1.475540in}{1.185379in}}%
\pgfpathcurveto{\pgfqpoint{1.481364in}{1.179555in}}{\pgfqpoint{1.489264in}{1.176283in}}{\pgfqpoint{1.497501in}{1.176283in}}%
\pgfpathlineto{\pgfqpoint{1.497501in}{1.176283in}}%
\pgfusepath{stroke,fill}%
\end{pgfscope}%
\begin{pgfscope}%
\pgfpathrectangle{\pgfqpoint{0.548058in}{0.516222in}}{\pgfqpoint{1.739582in}{1.783528in}} %
\pgfusepath{clip}%
\pgfsetbuttcap%
\pgfsetroundjoin%
\definecolor{currentfill}{rgb}{0.298039,0.447059,0.690196}%
\pgfsetfillcolor{currentfill}%
\pgfsetlinewidth{0.240900pt}%
\definecolor{currentstroke}{rgb}{1.000000,1.000000,1.000000}%
\pgfsetstrokecolor{currentstroke}%
\pgfsetdash{}{0pt}%
\pgfpathmoveto{\pgfqpoint{1.688664in}{0.717024in}}%
\pgfpathcurveto{\pgfqpoint{1.696900in}{0.717024in}}{\pgfqpoint{1.704800in}{0.720297in}}{\pgfqpoint{1.710624in}{0.726121in}}%
\pgfpathcurveto{\pgfqpoint{1.716448in}{0.731945in}}{\pgfqpoint{1.719720in}{0.739845in}}{\pgfqpoint{1.719720in}{0.748081in}}%
\pgfpathcurveto{\pgfqpoint{1.719720in}{0.756317in}}{\pgfqpoint{1.716448in}{0.764217in}}{\pgfqpoint{1.710624in}{0.770041in}}%
\pgfpathcurveto{\pgfqpoint{1.704800in}{0.775865in}}{\pgfqpoint{1.696900in}{0.779137in}}{\pgfqpoint{1.688664in}{0.779137in}}%
\pgfpathcurveto{\pgfqpoint{1.680427in}{0.779137in}}{\pgfqpoint{1.672527in}{0.775865in}}{\pgfqpoint{1.666703in}{0.770041in}}%
\pgfpathcurveto{\pgfqpoint{1.660879in}{0.764217in}}{\pgfqpoint{1.657607in}{0.756317in}}{\pgfqpoint{1.657607in}{0.748081in}}%
\pgfpathcurveto{\pgfqpoint{1.657607in}{0.739845in}}{\pgfqpoint{1.660879in}{0.731945in}}{\pgfqpoint{1.666703in}{0.726121in}}%
\pgfpathcurveto{\pgfqpoint{1.672527in}{0.720297in}}{\pgfqpoint{1.680427in}{0.717024in}}{\pgfqpoint{1.688664in}{0.717024in}}%
\pgfpathlineto{\pgfqpoint{1.688664in}{0.717024in}}%
\pgfusepath{stroke,fill}%
\end{pgfscope}%
\begin{pgfscope}%
\pgfpathrectangle{\pgfqpoint{0.548058in}{0.516222in}}{\pgfqpoint{1.739582in}{1.783528in}} %
\pgfusepath{clip}%
\pgfsetbuttcap%
\pgfsetroundjoin%
\definecolor{currentfill}{rgb}{0.298039,0.447059,0.690196}%
\pgfsetfillcolor{currentfill}%
\pgfsetlinewidth{0.240900pt}%
\definecolor{currentstroke}{rgb}{1.000000,1.000000,1.000000}%
\pgfsetstrokecolor{currentstroke}%
\pgfsetdash{}{0pt}%
\pgfpathmoveto{\pgfqpoint{0.987733in}{1.403683in}}%
\pgfpathcurveto{\pgfqpoint{0.995969in}{1.403683in}}{\pgfqpoint{1.003869in}{1.406955in}}{\pgfqpoint{1.009693in}{1.412779in}}%
\pgfpathcurveto{\pgfqpoint{1.015517in}{1.418603in}}{\pgfqpoint{1.018789in}{1.426503in}}{\pgfqpoint{1.018789in}{1.434739in}}%
\pgfpathcurveto{\pgfqpoint{1.018789in}{1.442975in}}{\pgfqpoint{1.015517in}{1.450875in}}{\pgfqpoint{1.009693in}{1.456699in}}%
\pgfpathcurveto{\pgfqpoint{1.003869in}{1.462523in}}{\pgfqpoint{0.995969in}{1.465796in}}{\pgfqpoint{0.987733in}{1.465796in}}%
\pgfpathcurveto{\pgfqpoint{0.979497in}{1.465796in}}{\pgfqpoint{0.971597in}{1.462523in}}{\pgfqpoint{0.965773in}{1.456699in}}%
\pgfpathcurveto{\pgfqpoint{0.959949in}{1.450875in}}{\pgfqpoint{0.956676in}{1.442975in}}{\pgfqpoint{0.956676in}{1.434739in}}%
\pgfpathcurveto{\pgfqpoint{0.956676in}{1.426503in}}{\pgfqpoint{0.959949in}{1.418603in}}{\pgfqpoint{0.965773in}{1.412779in}}%
\pgfpathcurveto{\pgfqpoint{0.971597in}{1.406955in}}{\pgfqpoint{0.979497in}{1.403683in}}{\pgfqpoint{0.987733in}{1.403683in}}%
\pgfpathlineto{\pgfqpoint{0.987733in}{1.403683in}}%
\pgfusepath{stroke,fill}%
\end{pgfscope}%
\begin{pgfscope}%
\pgfpathrectangle{\pgfqpoint{0.548058in}{0.516222in}}{\pgfqpoint{1.739582in}{1.783528in}} %
\pgfusepath{clip}%
\pgfsetbuttcap%
\pgfsetroundjoin%
\definecolor{currentfill}{rgb}{0.298039,0.447059,0.690196}%
\pgfsetfillcolor{currentfill}%
\pgfsetlinewidth{0.240900pt}%
\definecolor{currentstroke}{rgb}{1.000000,1.000000,1.000000}%
\pgfsetstrokecolor{currentstroke}%
\pgfsetdash{}{0pt}%
\pgfpathmoveto{\pgfqpoint{1.401919in}{0.873083in}}%
\pgfpathcurveto{\pgfqpoint{1.410155in}{0.873083in}}{\pgfqpoint{1.418055in}{0.876355in}}{\pgfqpoint{1.423879in}{0.882179in}}%
\pgfpathcurveto{\pgfqpoint{1.429703in}{0.888003in}}{\pgfqpoint{1.432976in}{0.895903in}}{\pgfqpoint{1.432976in}{0.904140in}}%
\pgfpathcurveto{\pgfqpoint{1.432976in}{0.912376in}}{\pgfqpoint{1.429703in}{0.920276in}}{\pgfqpoint{1.423879in}{0.926100in}}%
\pgfpathcurveto{\pgfqpoint{1.418055in}{0.931924in}}{\pgfqpoint{1.410155in}{0.935196in}}{\pgfqpoint{1.401919in}{0.935196in}}%
\pgfpathcurveto{\pgfqpoint{1.393683in}{0.935196in}}{\pgfqpoint{1.385783in}{0.931924in}}{\pgfqpoint{1.379959in}{0.926100in}}%
\pgfpathcurveto{\pgfqpoint{1.374135in}{0.920276in}}{\pgfqpoint{1.370863in}{0.912376in}}{\pgfqpoint{1.370863in}{0.904140in}}%
\pgfpathcurveto{\pgfqpoint{1.370863in}{0.895903in}}{\pgfqpoint{1.374135in}{0.888003in}}{\pgfqpoint{1.379959in}{0.882179in}}%
\pgfpathcurveto{\pgfqpoint{1.385783in}{0.876355in}}{\pgfqpoint{1.393683in}{0.873083in}}{\pgfqpoint{1.401919in}{0.873083in}}%
\pgfpathlineto{\pgfqpoint{1.401919in}{0.873083in}}%
\pgfusepath{stroke,fill}%
\end{pgfscope}%
\begin{pgfscope}%
\pgfpathrectangle{\pgfqpoint{0.548058in}{0.516222in}}{\pgfqpoint{1.739582in}{1.783528in}} %
\pgfusepath{clip}%
\pgfsetbuttcap%
\pgfsetroundjoin%
\definecolor{currentfill}{rgb}{0.298039,0.447059,0.690196}%
\pgfsetfillcolor{currentfill}%
\pgfsetlinewidth{0.240900pt}%
\definecolor{currentstroke}{rgb}{1.000000,1.000000,1.000000}%
\pgfsetstrokecolor{currentstroke}%
\pgfsetdash{}{0pt}%
\pgfpathmoveto{\pgfqpoint{1.943547in}{0.971177in}}%
\pgfpathcurveto{\pgfqpoint{1.951784in}{0.971177in}}{\pgfqpoint{1.959684in}{0.974449in}}{\pgfqpoint{1.965508in}{0.980273in}}%
\pgfpathcurveto{\pgfqpoint{1.971332in}{0.986097in}}{\pgfqpoint{1.974604in}{0.993997in}}{\pgfqpoint{1.974604in}{1.002234in}}%
\pgfpathcurveto{\pgfqpoint{1.974604in}{1.010470in}}{\pgfqpoint{1.971332in}{1.018370in}}{\pgfqpoint{1.965508in}{1.024194in}}%
\pgfpathcurveto{\pgfqpoint{1.959684in}{1.030018in}}{\pgfqpoint{1.951784in}{1.033290in}}{\pgfqpoint{1.943547in}{1.033290in}}%
\pgfpathcurveto{\pgfqpoint{1.935311in}{1.033290in}}{\pgfqpoint{1.927411in}{1.030018in}}{\pgfqpoint{1.921587in}{1.024194in}}%
\pgfpathcurveto{\pgfqpoint{1.915763in}{1.018370in}}{\pgfqpoint{1.912491in}{1.010470in}}{\pgfqpoint{1.912491in}{1.002234in}}%
\pgfpathcurveto{\pgfqpoint{1.912491in}{0.993997in}}{\pgfqpoint{1.915763in}{0.986097in}}{\pgfqpoint{1.921587in}{0.980273in}}%
\pgfpathcurveto{\pgfqpoint{1.927411in}{0.974449in}}{\pgfqpoint{1.935311in}{0.971177in}}{\pgfqpoint{1.943547in}{0.971177in}}%
\pgfpathlineto{\pgfqpoint{1.943547in}{0.971177in}}%
\pgfusepath{stroke,fill}%
\end{pgfscope}%
\begin{pgfscope}%
\pgfpathrectangle{\pgfqpoint{0.548058in}{0.516222in}}{\pgfqpoint{1.739582in}{1.783528in}} %
\pgfusepath{clip}%
\pgfsetbuttcap%
\pgfsetroundjoin%
\definecolor{currentfill}{rgb}{0.298039,0.447059,0.690196}%
\pgfsetfillcolor{currentfill}%
\pgfsetlinewidth{0.240900pt}%
\definecolor{currentstroke}{rgb}{1.000000,1.000000,1.000000}%
\pgfsetstrokecolor{currentstroke}%
\pgfsetdash{}{0pt}%
\pgfpathmoveto{\pgfqpoint{1.242617in}{1.403683in}}%
\pgfpathcurveto{\pgfqpoint{1.250853in}{1.403683in}}{\pgfqpoint{1.258753in}{1.406955in}}{\pgfqpoint{1.264577in}{1.412779in}}%
\pgfpathcurveto{\pgfqpoint{1.270401in}{1.418603in}}{\pgfqpoint{1.273673in}{1.426503in}}{\pgfqpoint{1.273673in}{1.434739in}}%
\pgfpathcurveto{\pgfqpoint{1.273673in}{1.442975in}}{\pgfqpoint{1.270401in}{1.450875in}}{\pgfqpoint{1.264577in}{1.456699in}}%
\pgfpathcurveto{\pgfqpoint{1.258753in}{1.462523in}}{\pgfqpoint{1.250853in}{1.465796in}}{\pgfqpoint{1.242617in}{1.465796in}}%
\pgfpathcurveto{\pgfqpoint{1.234380in}{1.465796in}}{\pgfqpoint{1.226480in}{1.462523in}}{\pgfqpoint{1.220656in}{1.456699in}}%
\pgfpathcurveto{\pgfqpoint{1.214833in}{1.450875in}}{\pgfqpoint{1.211560in}{1.442975in}}{\pgfqpoint{1.211560in}{1.434739in}}%
\pgfpathcurveto{\pgfqpoint{1.211560in}{1.426503in}}{\pgfqpoint{1.214833in}{1.418603in}}{\pgfqpoint{1.220656in}{1.412779in}}%
\pgfpathcurveto{\pgfqpoint{1.226480in}{1.406955in}}{\pgfqpoint{1.234380in}{1.403683in}}{\pgfqpoint{1.242617in}{1.403683in}}%
\pgfpathlineto{\pgfqpoint{1.242617in}{1.403683in}}%
\pgfusepath{stroke,fill}%
\end{pgfscope}%
\begin{pgfscope}%
\pgfpathrectangle{\pgfqpoint{0.548058in}{0.516222in}}{\pgfqpoint{1.739582in}{1.783528in}} %
\pgfusepath{clip}%
\pgfsetbuttcap%
\pgfsetroundjoin%
\definecolor{currentfill}{rgb}{0.298039,0.447059,0.690196}%
\pgfsetfillcolor{currentfill}%
\pgfsetlinewidth{0.240900pt}%
\definecolor{currentstroke}{rgb}{1.000000,1.000000,1.000000}%
\pgfsetstrokecolor{currentstroke}%
\pgfsetdash{}{0pt}%
\pgfpathmoveto{\pgfqpoint{0.828430in}{1.537447in}}%
\pgfpathcurveto{\pgfqpoint{0.836667in}{1.537447in}}{\pgfqpoint{0.844567in}{1.540719in}}{\pgfqpoint{0.850391in}{1.546543in}}%
\pgfpathcurveto{\pgfqpoint{0.856215in}{1.552367in}}{\pgfqpoint{0.859487in}{1.560267in}}{\pgfqpoint{0.859487in}{1.568504in}}%
\pgfpathcurveto{\pgfqpoint{0.859487in}{1.576740in}}{\pgfqpoint{0.856215in}{1.584640in}}{\pgfqpoint{0.850391in}{1.590464in}}%
\pgfpathcurveto{\pgfqpoint{0.844567in}{1.596288in}}{\pgfqpoint{0.836667in}{1.599560in}}{\pgfqpoint{0.828430in}{1.599560in}}%
\pgfpathcurveto{\pgfqpoint{0.820194in}{1.599560in}}{\pgfqpoint{0.812294in}{1.596288in}}{\pgfqpoint{0.806470in}{1.590464in}}%
\pgfpathcurveto{\pgfqpoint{0.800646in}{1.584640in}}{\pgfqpoint{0.797374in}{1.576740in}}{\pgfqpoint{0.797374in}{1.568504in}}%
\pgfpathcurveto{\pgfqpoint{0.797374in}{1.560267in}}{\pgfqpoint{0.800646in}{1.552367in}}{\pgfqpoint{0.806470in}{1.546543in}}%
\pgfpathcurveto{\pgfqpoint{0.812294in}{1.540719in}}{\pgfqpoint{0.820194in}{1.537447in}}{\pgfqpoint{0.828430in}{1.537447in}}%
\pgfpathlineto{\pgfqpoint{0.828430in}{1.537447in}}%
\pgfusepath{stroke,fill}%
\end{pgfscope}%
\begin{pgfscope}%
\pgfpathrectangle{\pgfqpoint{0.548058in}{0.516222in}}{\pgfqpoint{1.739582in}{1.783528in}} %
\pgfusepath{clip}%
\pgfsetbuttcap%
\pgfsetroundjoin%
\definecolor{currentfill}{rgb}{0.298039,0.447059,0.690196}%
\pgfsetfillcolor{currentfill}%
\pgfsetlinewidth{0.240900pt}%
\definecolor{currentstroke}{rgb}{1.000000,1.000000,1.000000}%
\pgfsetstrokecolor{currentstroke}%
\pgfsetdash{}{0pt}%
\pgfpathmoveto{\pgfqpoint{1.338198in}{0.877542in}}%
\pgfpathcurveto{\pgfqpoint{1.346434in}{0.877542in}}{\pgfqpoint{1.354335in}{0.880814in}}{\pgfqpoint{1.360158in}{0.886638in}}%
\pgfpathcurveto{\pgfqpoint{1.365982in}{0.892462in}}{\pgfqpoint{1.369255in}{0.900362in}}{\pgfqpoint{1.369255in}{0.908598in}}%
\pgfpathcurveto{\pgfqpoint{1.369255in}{0.916835in}}{\pgfqpoint{1.365982in}{0.924735in}}{\pgfqpoint{1.360158in}{0.930559in}}%
\pgfpathcurveto{\pgfqpoint{1.354335in}{0.936383in}}{\pgfqpoint{1.346434in}{0.939655in}}{\pgfqpoint{1.338198in}{0.939655in}}%
\pgfpathcurveto{\pgfqpoint{1.329962in}{0.939655in}}{\pgfqpoint{1.322062in}{0.936383in}}{\pgfqpoint{1.316238in}{0.930559in}}%
\pgfpathcurveto{\pgfqpoint{1.310414in}{0.924735in}}{\pgfqpoint{1.307142in}{0.916835in}}{\pgfqpoint{1.307142in}{0.908598in}}%
\pgfpathcurveto{\pgfqpoint{1.307142in}{0.900362in}}{\pgfqpoint{1.310414in}{0.892462in}}{\pgfqpoint{1.316238in}{0.886638in}}%
\pgfpathcurveto{\pgfqpoint{1.322062in}{0.880814in}}{\pgfqpoint{1.329962in}{0.877542in}}{\pgfqpoint{1.338198in}{0.877542in}}%
\pgfpathlineto{\pgfqpoint{1.338198in}{0.877542in}}%
\pgfusepath{stroke,fill}%
\end{pgfscope}%
\begin{pgfscope}%
\pgfpathrectangle{\pgfqpoint{0.548058in}{0.516222in}}{\pgfqpoint{1.739582in}{1.783528in}} %
\pgfusepath{clip}%
\pgfsetbuttcap%
\pgfsetroundjoin%
\definecolor{currentfill}{rgb}{0.298039,0.447059,0.690196}%
\pgfsetfillcolor{currentfill}%
\pgfsetlinewidth{0.240900pt}%
\definecolor{currentstroke}{rgb}{1.000000,1.000000,1.000000}%
\pgfsetstrokecolor{currentstroke}%
\pgfsetdash{}{0pt}%
\pgfpathmoveto{\pgfqpoint{1.911687in}{0.815118in}}%
\pgfpathcurveto{\pgfqpoint{1.919923in}{0.815118in}}{\pgfqpoint{1.927823in}{0.818391in}}{\pgfqpoint{1.933647in}{0.824215in}}%
\pgfpathcurveto{\pgfqpoint{1.939471in}{0.830039in}}{\pgfqpoint{1.942743in}{0.837939in}}{\pgfqpoint{1.942743in}{0.846175in}}%
\pgfpathcurveto{\pgfqpoint{1.942743in}{0.854411in}}{\pgfqpoint{1.939471in}{0.862311in}}{\pgfqpoint{1.933647in}{0.868135in}}%
\pgfpathcurveto{\pgfqpoint{1.927823in}{0.873959in}}{\pgfqpoint{1.919923in}{0.877231in}}{\pgfqpoint{1.911687in}{0.877231in}}%
\pgfpathcurveto{\pgfqpoint{1.903451in}{0.877231in}}{\pgfqpoint{1.895551in}{0.873959in}}{\pgfqpoint{1.889727in}{0.868135in}}%
\pgfpathcurveto{\pgfqpoint{1.883903in}{0.862311in}}{\pgfqpoint{1.880630in}{0.854411in}}{\pgfqpoint{1.880630in}{0.846175in}}%
\pgfpathcurveto{\pgfqpoint{1.880630in}{0.837939in}}{\pgfqpoint{1.883903in}{0.830039in}}{\pgfqpoint{1.889727in}{0.824215in}}%
\pgfpathcurveto{\pgfqpoint{1.895551in}{0.818391in}}{\pgfqpoint{1.903451in}{0.815118in}}{\pgfqpoint{1.911687in}{0.815118in}}%
\pgfpathlineto{\pgfqpoint{1.911687in}{0.815118in}}%
\pgfusepath{stroke,fill}%
\end{pgfscope}%
\begin{pgfscope}%
\pgfpathrectangle{\pgfqpoint{0.548058in}{0.516222in}}{\pgfqpoint{1.739582in}{1.783528in}} %
\pgfusepath{clip}%
\pgfsetbuttcap%
\pgfsetroundjoin%
\definecolor{currentfill}{rgb}{0.298039,0.447059,0.690196}%
\pgfsetfillcolor{currentfill}%
\pgfsetlinewidth{0.240900pt}%
\definecolor{currentstroke}{rgb}{1.000000,1.000000,1.000000}%
\pgfsetstrokecolor{currentstroke}%
\pgfsetdash{}{0pt}%
\pgfpathmoveto{\pgfqpoint{1.720524in}{0.993471in}}%
\pgfpathcurveto{\pgfqpoint{1.728760in}{0.993471in}}{\pgfqpoint{1.736660in}{0.996743in}}{\pgfqpoint{1.742484in}{1.002567in}}%
\pgfpathcurveto{\pgfqpoint{1.748308in}{1.008391in}}{\pgfqpoint{1.751580in}{1.016291in}}{\pgfqpoint{1.751580in}{1.024528in}}%
\pgfpathcurveto{\pgfqpoint{1.751580in}{1.032764in}}{\pgfqpoint{1.748308in}{1.040664in}}{\pgfqpoint{1.742484in}{1.046488in}}%
\pgfpathcurveto{\pgfqpoint{1.736660in}{1.052312in}}{\pgfqpoint{1.728760in}{1.055584in}}{\pgfqpoint{1.720524in}{1.055584in}}%
\pgfpathcurveto{\pgfqpoint{1.712288in}{1.055584in}}{\pgfqpoint{1.704388in}{1.052312in}}{\pgfqpoint{1.698564in}{1.046488in}}%
\pgfpathcurveto{\pgfqpoint{1.692740in}{1.040664in}}{\pgfqpoint{1.689467in}{1.032764in}}{\pgfqpoint{1.689467in}{1.024528in}}%
\pgfpathcurveto{\pgfqpoint{1.689467in}{1.016291in}}{\pgfqpoint{1.692740in}{1.008391in}}{\pgfqpoint{1.698564in}{1.002567in}}%
\pgfpathcurveto{\pgfqpoint{1.704388in}{0.996743in}}{\pgfqpoint{1.712288in}{0.993471in}}{\pgfqpoint{1.720524in}{0.993471in}}%
\pgfpathlineto{\pgfqpoint{1.720524in}{0.993471in}}%
\pgfusepath{stroke,fill}%
\end{pgfscope}%
\begin{pgfscope}%
\pgfpathrectangle{\pgfqpoint{0.548058in}{0.516222in}}{\pgfqpoint{1.739582in}{1.783528in}} %
\pgfusepath{clip}%
\pgfsetbuttcap%
\pgfsetroundjoin%
\definecolor{currentfill}{rgb}{0.298039,0.447059,0.690196}%
\pgfsetfillcolor{currentfill}%
\pgfsetlinewidth{0.240900pt}%
\definecolor{currentstroke}{rgb}{1.000000,1.000000,1.000000}%
\pgfsetstrokecolor{currentstroke}%
\pgfsetdash{}{0pt}%
\pgfpathmoveto{\pgfqpoint{1.465640in}{0.819577in}}%
\pgfpathcurveto{\pgfqpoint{1.473876in}{0.819577in}}{\pgfqpoint{1.481776in}{0.822849in}}{\pgfqpoint{1.487600in}{0.828673in}}%
\pgfpathcurveto{\pgfqpoint{1.493424in}{0.834497in}}{\pgfqpoint{1.496697in}{0.842397in}}{\pgfqpoint{1.496697in}{0.850634in}}%
\pgfpathcurveto{\pgfqpoint{1.496697in}{0.858870in}}{\pgfqpoint{1.493424in}{0.866770in}}{\pgfqpoint{1.487600in}{0.872594in}}%
\pgfpathcurveto{\pgfqpoint{1.481776in}{0.878418in}}{\pgfqpoint{1.473876in}{0.881690in}}{\pgfqpoint{1.465640in}{0.881690in}}%
\pgfpathcurveto{\pgfqpoint{1.457404in}{0.881690in}}{\pgfqpoint{1.449504in}{0.878418in}}{\pgfqpoint{1.443680in}{0.872594in}}%
\pgfpathcurveto{\pgfqpoint{1.437856in}{0.866770in}}{\pgfqpoint{1.434584in}{0.858870in}}{\pgfqpoint{1.434584in}{0.850634in}}%
\pgfpathcurveto{\pgfqpoint{1.434584in}{0.842397in}}{\pgfqpoint{1.437856in}{0.834497in}}{\pgfqpoint{1.443680in}{0.828673in}}%
\pgfpathcurveto{\pgfqpoint{1.449504in}{0.822849in}}{\pgfqpoint{1.457404in}{0.819577in}}{\pgfqpoint{1.465640in}{0.819577in}}%
\pgfpathlineto{\pgfqpoint{1.465640in}{0.819577in}}%
\pgfusepath{stroke,fill}%
\end{pgfscope}%
\begin{pgfscope}%
\pgfpathrectangle{\pgfqpoint{0.548058in}{0.516222in}}{\pgfqpoint{1.739582in}{1.783528in}} %
\pgfusepath{clip}%
\pgfsetbuttcap%
\pgfsetroundjoin%
\definecolor{currentfill}{rgb}{0.298039,0.447059,0.690196}%
\pgfsetfillcolor{currentfill}%
\pgfsetlinewidth{0.240900pt}%
\definecolor{currentstroke}{rgb}{1.000000,1.000000,1.000000}%
\pgfsetstrokecolor{currentstroke}%
\pgfsetdash{}{0pt}%
\pgfpathmoveto{\pgfqpoint{0.955872in}{1.390306in}}%
\pgfpathcurveto{\pgfqpoint{0.964109in}{1.390306in}}{\pgfqpoint{0.972009in}{1.393578in}}{\pgfqpoint{0.977833in}{1.399402in}}%
\pgfpathcurveto{\pgfqpoint{0.983657in}{1.405226in}}{\pgfqpoint{0.986929in}{1.413126in}}{\pgfqpoint{0.986929in}{1.421363in}}%
\pgfpathcurveto{\pgfqpoint{0.986929in}{1.429599in}}{\pgfqpoint{0.983657in}{1.437499in}}{\pgfqpoint{0.977833in}{1.443323in}}%
\pgfpathcurveto{\pgfqpoint{0.972009in}{1.449147in}}{\pgfqpoint{0.964109in}{1.452419in}}{\pgfqpoint{0.955872in}{1.452419in}}%
\pgfpathcurveto{\pgfqpoint{0.947636in}{1.452419in}}{\pgfqpoint{0.939736in}{1.449147in}}{\pgfqpoint{0.933912in}{1.443323in}}%
\pgfpathcurveto{\pgfqpoint{0.928088in}{1.437499in}}{\pgfqpoint{0.924816in}{1.429599in}}{\pgfqpoint{0.924816in}{1.421363in}}%
\pgfpathcurveto{\pgfqpoint{0.924816in}{1.413126in}}{\pgfqpoint{0.928088in}{1.405226in}}{\pgfqpoint{0.933912in}{1.399402in}}%
\pgfpathcurveto{\pgfqpoint{0.939736in}{1.393578in}}{\pgfqpoint{0.947636in}{1.390306in}}{\pgfqpoint{0.955872in}{1.390306in}}%
\pgfpathlineto{\pgfqpoint{0.955872in}{1.390306in}}%
\pgfusepath{stroke,fill}%
\end{pgfscope}%
\begin{pgfscope}%
\pgfpathrectangle{\pgfqpoint{0.548058in}{0.516222in}}{\pgfqpoint{1.739582in}{1.783528in}} %
\pgfusepath{clip}%
\pgfsetbuttcap%
\pgfsetroundjoin%
\definecolor{currentfill}{rgb}{0.298039,0.447059,0.690196}%
\pgfsetfillcolor{currentfill}%
\pgfsetlinewidth{0.240900pt}%
\definecolor{currentstroke}{rgb}{1.000000,1.000000,1.000000}%
\pgfsetstrokecolor{currentstroke}%
\pgfsetdash{}{0pt}%
\pgfpathmoveto{\pgfqpoint{1.306338in}{1.488400in}}%
\pgfpathcurveto{\pgfqpoint{1.314574in}{1.488400in}}{\pgfqpoint{1.322474in}{1.491672in}}{\pgfqpoint{1.328298in}{1.497496in}}%
\pgfpathcurveto{\pgfqpoint{1.334122in}{1.503320in}}{\pgfqpoint{1.337394in}{1.511220in}}{\pgfqpoint{1.337394in}{1.519457in}}%
\pgfpathcurveto{\pgfqpoint{1.337394in}{1.527693in}}{\pgfqpoint{1.334122in}{1.535593in}}{\pgfqpoint{1.328298in}{1.541417in}}%
\pgfpathcurveto{\pgfqpoint{1.322474in}{1.547241in}}{\pgfqpoint{1.314574in}{1.550513in}}{\pgfqpoint{1.306338in}{1.550513in}}%
\pgfpathcurveto{\pgfqpoint{1.298101in}{1.550513in}}{\pgfqpoint{1.290201in}{1.547241in}}{\pgfqpoint{1.284377in}{1.541417in}}%
\pgfpathcurveto{\pgfqpoint{1.278554in}{1.535593in}}{\pgfqpoint{1.275281in}{1.527693in}}{\pgfqpoint{1.275281in}{1.519457in}}%
\pgfpathcurveto{\pgfqpoint{1.275281in}{1.511220in}}{\pgfqpoint{1.278554in}{1.503320in}}{\pgfqpoint{1.284377in}{1.497496in}}%
\pgfpathcurveto{\pgfqpoint{1.290201in}{1.491672in}}{\pgfqpoint{1.298101in}{1.488400in}}{\pgfqpoint{1.306338in}{1.488400in}}%
\pgfpathlineto{\pgfqpoint{1.306338in}{1.488400in}}%
\pgfusepath{stroke,fill}%
\end{pgfscope}%
\begin{pgfscope}%
\pgfpathrectangle{\pgfqpoint{0.548058in}{0.516222in}}{\pgfqpoint{1.739582in}{1.783528in}} %
\pgfusepath{clip}%
\pgfsetbuttcap%
\pgfsetroundjoin%
\definecolor{currentfill}{rgb}{0.298039,0.447059,0.690196}%
\pgfsetfillcolor{currentfill}%
\pgfsetlinewidth{0.240900pt}%
\definecolor{currentstroke}{rgb}{1.000000,1.000000,1.000000}%
\pgfsetstrokecolor{currentstroke}%
\pgfsetdash{}{0pt}%
\pgfpathmoveto{\pgfqpoint{0.860291in}{1.457188in}}%
\pgfpathcurveto{\pgfqpoint{0.868527in}{1.457188in}}{\pgfqpoint{0.876427in}{1.460461in}}{\pgfqpoint{0.882251in}{1.466285in}}%
\pgfpathcurveto{\pgfqpoint{0.888075in}{1.472109in}}{\pgfqpoint{0.891347in}{1.480009in}}{\pgfqpoint{0.891347in}{1.488245in}}%
\pgfpathcurveto{\pgfqpoint{0.891347in}{1.496481in}}{\pgfqpoint{0.888075in}{1.504381in}}{\pgfqpoint{0.882251in}{1.510205in}}%
\pgfpathcurveto{\pgfqpoint{0.876427in}{1.516029in}}{\pgfqpoint{0.868527in}{1.519301in}}{\pgfqpoint{0.860291in}{1.519301in}}%
\pgfpathcurveto{\pgfqpoint{0.852055in}{1.519301in}}{\pgfqpoint{0.844155in}{1.516029in}}{\pgfqpoint{0.838331in}{1.510205in}}%
\pgfpathcurveto{\pgfqpoint{0.832507in}{1.504381in}}{\pgfqpoint{0.829234in}{1.496481in}}{\pgfqpoint{0.829234in}{1.488245in}}%
\pgfpathcurveto{\pgfqpoint{0.829234in}{1.480009in}}{\pgfqpoint{0.832507in}{1.472109in}}{\pgfqpoint{0.838331in}{1.466285in}}%
\pgfpathcurveto{\pgfqpoint{0.844155in}{1.460461in}}{\pgfqpoint{0.852055in}{1.457188in}}{\pgfqpoint{0.860291in}{1.457188in}}%
\pgfpathlineto{\pgfqpoint{0.860291in}{1.457188in}}%
\pgfusepath{stroke,fill}%
\end{pgfscope}%
\begin{pgfscope}%
\pgfpathrectangle{\pgfqpoint{0.548058in}{0.516222in}}{\pgfqpoint{1.739582in}{1.783528in}} %
\pgfusepath{clip}%
\pgfsetbuttcap%
\pgfsetroundjoin%
\definecolor{currentfill}{rgb}{0.298039,0.447059,0.690196}%
\pgfsetfillcolor{currentfill}%
\pgfsetlinewidth{0.240900pt}%
\definecolor{currentstroke}{rgb}{1.000000,1.000000,1.000000}%
\pgfsetstrokecolor{currentstroke}%
\pgfsetdash{}{0pt}%
\pgfpathmoveto{\pgfqpoint{1.752384in}{1.153989in}}%
\pgfpathcurveto{\pgfqpoint{1.760621in}{1.153989in}}{\pgfqpoint{1.768521in}{1.157261in}}{\pgfqpoint{1.774345in}{1.163085in}}%
\pgfpathcurveto{\pgfqpoint{1.780169in}{1.168909in}}{\pgfqpoint{1.783441in}{1.176809in}}{\pgfqpoint{1.783441in}{1.185045in}}%
\pgfpathcurveto{\pgfqpoint{1.783441in}{1.193281in}}{\pgfqpoint{1.780169in}{1.201181in}}{\pgfqpoint{1.774345in}{1.207005in}}%
\pgfpathcurveto{\pgfqpoint{1.768521in}{1.212829in}}{\pgfqpoint{1.760621in}{1.216102in}}{\pgfqpoint{1.752384in}{1.216102in}}%
\pgfpathcurveto{\pgfqpoint{1.744148in}{1.216102in}}{\pgfqpoint{1.736248in}{1.212829in}}{\pgfqpoint{1.730424in}{1.207005in}}%
\pgfpathcurveto{\pgfqpoint{1.724600in}{1.201181in}}{\pgfqpoint{1.721328in}{1.193281in}}{\pgfqpoint{1.721328in}{1.185045in}}%
\pgfpathcurveto{\pgfqpoint{1.721328in}{1.176809in}}{\pgfqpoint{1.724600in}{1.168909in}}{\pgfqpoint{1.730424in}{1.163085in}}%
\pgfpathcurveto{\pgfqpoint{1.736248in}{1.157261in}}{\pgfqpoint{1.744148in}{1.153989in}}{\pgfqpoint{1.752384in}{1.153989in}}%
\pgfpathlineto{\pgfqpoint{1.752384in}{1.153989in}}%
\pgfusepath{stroke,fill}%
\end{pgfscope}%
\begin{pgfscope}%
\pgfpathrectangle{\pgfqpoint{0.548058in}{0.516222in}}{\pgfqpoint{1.739582in}{1.783528in}} %
\pgfusepath{clip}%
\pgfsetbuttcap%
\pgfsetroundjoin%
\definecolor{currentfill}{rgb}{0.298039,0.447059,0.690196}%
\pgfsetfillcolor{currentfill}%
\pgfsetlinewidth{0.240900pt}%
\definecolor{currentstroke}{rgb}{1.000000,1.000000,1.000000}%
\pgfsetstrokecolor{currentstroke}%
\pgfsetdash{}{0pt}%
\pgfpathmoveto{\pgfqpoint{1.147035in}{1.368012in}}%
\pgfpathcurveto{\pgfqpoint{1.155272in}{1.368012in}}{\pgfqpoint{1.163172in}{1.371284in}}{\pgfqpoint{1.168996in}{1.377108in}}%
\pgfpathcurveto{\pgfqpoint{1.174819in}{1.382932in}}{\pgfqpoint{1.178092in}{1.390832in}}{\pgfqpoint{1.178092in}{1.399068in}}%
\pgfpathcurveto{\pgfqpoint{1.178092in}{1.407305in}}{\pgfqpoint{1.174819in}{1.415205in}}{\pgfqpoint{1.168996in}{1.421029in}}%
\pgfpathcurveto{\pgfqpoint{1.163172in}{1.426853in}}{\pgfqpoint{1.155272in}{1.430125in}}{\pgfqpoint{1.147035in}{1.430125in}}%
\pgfpathcurveto{\pgfqpoint{1.138799in}{1.430125in}}{\pgfqpoint{1.130899in}{1.426853in}}{\pgfqpoint{1.125075in}{1.421029in}}%
\pgfpathcurveto{\pgfqpoint{1.119251in}{1.415205in}}{\pgfqpoint{1.115979in}{1.407305in}}{\pgfqpoint{1.115979in}{1.399068in}}%
\pgfpathcurveto{\pgfqpoint{1.115979in}{1.390832in}}{\pgfqpoint{1.119251in}{1.382932in}}{\pgfqpoint{1.125075in}{1.377108in}}%
\pgfpathcurveto{\pgfqpoint{1.130899in}{1.371284in}}{\pgfqpoint{1.138799in}{1.368012in}}{\pgfqpoint{1.147035in}{1.368012in}}%
\pgfpathlineto{\pgfqpoint{1.147035in}{1.368012in}}%
\pgfusepath{stroke,fill}%
\end{pgfscope}%
\begin{pgfscope}%
\pgfpathrectangle{\pgfqpoint{0.548058in}{0.516222in}}{\pgfqpoint{1.739582in}{1.783528in}} %
\pgfusepath{clip}%
\pgfsetbuttcap%
\pgfsetroundjoin%
\definecolor{currentfill}{rgb}{0.298039,0.447059,0.690196}%
\pgfsetfillcolor{currentfill}%
\pgfsetlinewidth{0.240900pt}%
\definecolor{currentstroke}{rgb}{1.000000,1.000000,1.000000}%
\pgfsetstrokecolor{currentstroke}%
\pgfsetdash{}{0pt}%
\pgfpathmoveto{\pgfqpoint{1.178896in}{1.359094in}}%
\pgfpathcurveto{\pgfqpoint{1.187132in}{1.359094in}}{\pgfqpoint{1.195032in}{1.362367in}}{\pgfqpoint{1.200856in}{1.368191in}}%
\pgfpathcurveto{\pgfqpoint{1.206680in}{1.374015in}}{\pgfqpoint{1.209952in}{1.381915in}}{\pgfqpoint{1.209952in}{1.390151in}}%
\pgfpathcurveto{\pgfqpoint{1.209952in}{1.398387in}}{\pgfqpoint{1.206680in}{1.406287in}}{\pgfqpoint{1.200856in}{1.412111in}}%
\pgfpathcurveto{\pgfqpoint{1.195032in}{1.417935in}}{\pgfqpoint{1.187132in}{1.421207in}}{\pgfqpoint{1.178896in}{1.421207in}}%
\pgfpathcurveto{\pgfqpoint{1.170659in}{1.421207in}}{\pgfqpoint{1.162759in}{1.417935in}}{\pgfqpoint{1.156935in}{1.412111in}}%
\pgfpathcurveto{\pgfqpoint{1.151112in}{1.406287in}}{\pgfqpoint{1.147839in}{1.398387in}}{\pgfqpoint{1.147839in}{1.390151in}}%
\pgfpathcurveto{\pgfqpoint{1.147839in}{1.381915in}}{\pgfqpoint{1.151112in}{1.374015in}}{\pgfqpoint{1.156935in}{1.368191in}}%
\pgfpathcurveto{\pgfqpoint{1.162759in}{1.362367in}}{\pgfqpoint{1.170659in}{1.359094in}}{\pgfqpoint{1.178896in}{1.359094in}}%
\pgfpathlineto{\pgfqpoint{1.178896in}{1.359094in}}%
\pgfusepath{stroke,fill}%
\end{pgfscope}%
\begin{pgfscope}%
\pgfpathrectangle{\pgfqpoint{0.548058in}{0.516222in}}{\pgfqpoint{1.739582in}{1.783528in}} %
\pgfusepath{clip}%
\pgfsetbuttcap%
\pgfsetroundjoin%
\definecolor{currentfill}{rgb}{0.298039,0.447059,0.690196}%
\pgfsetfillcolor{currentfill}%
\pgfsetlinewidth{0.240900pt}%
\definecolor{currentstroke}{rgb}{1.000000,1.000000,1.000000}%
\pgfsetstrokecolor{currentstroke}%
\pgfsetdash{}{0pt}%
\pgfpathmoveto{\pgfqpoint{2.007268in}{1.278836in}}%
\pgfpathcurveto{\pgfqpoint{2.015505in}{1.278836in}}{\pgfqpoint{2.023405in}{1.282108in}}{\pgfqpoint{2.029229in}{1.287932in}}%
\pgfpathcurveto{\pgfqpoint{2.035053in}{1.293756in}}{\pgfqpoint{2.038325in}{1.301656in}}{\pgfqpoint{2.038325in}{1.309892in}}%
\pgfpathcurveto{\pgfqpoint{2.038325in}{1.318128in}}{\pgfqpoint{2.035053in}{1.326028in}}{\pgfqpoint{2.029229in}{1.331852in}}%
\pgfpathcurveto{\pgfqpoint{2.023405in}{1.337676in}}{\pgfqpoint{2.015505in}{1.340949in}}{\pgfqpoint{2.007268in}{1.340949in}}%
\pgfpathcurveto{\pgfqpoint{1.999032in}{1.340949in}}{\pgfqpoint{1.991132in}{1.337676in}}{\pgfqpoint{1.985308in}{1.331852in}}%
\pgfpathcurveto{\pgfqpoint{1.979484in}{1.326028in}}{\pgfqpoint{1.976212in}{1.318128in}}{\pgfqpoint{1.976212in}{1.309892in}}%
\pgfpathcurveto{\pgfqpoint{1.976212in}{1.301656in}}{\pgfqpoint{1.979484in}{1.293756in}}{\pgfqpoint{1.985308in}{1.287932in}}%
\pgfpathcurveto{\pgfqpoint{1.991132in}{1.282108in}}{\pgfqpoint{1.999032in}{1.278836in}}{\pgfqpoint{2.007268in}{1.278836in}}%
\pgfpathlineto{\pgfqpoint{2.007268in}{1.278836in}}%
\pgfusepath{stroke,fill}%
\end{pgfscope}%
\begin{pgfscope}%
\pgfpathrectangle{\pgfqpoint{0.548058in}{0.516222in}}{\pgfqpoint{1.739582in}{1.783528in}} %
\pgfusepath{clip}%
\pgfsetbuttcap%
\pgfsetroundjoin%
\definecolor{currentfill}{rgb}{0.298039,0.447059,0.690196}%
\pgfsetfillcolor{currentfill}%
\pgfsetlinewidth{0.240900pt}%
\definecolor{currentstroke}{rgb}{1.000000,1.000000,1.000000}%
\pgfsetstrokecolor{currentstroke}%
\pgfsetdash{}{0pt}%
\pgfpathmoveto{\pgfqpoint{1.784245in}{0.931048in}}%
\pgfpathcurveto{\pgfqpoint{1.792481in}{0.931048in}}{\pgfqpoint{1.800381in}{0.934320in}}{\pgfqpoint{1.806205in}{0.940144in}}%
\pgfpathcurveto{\pgfqpoint{1.812029in}{0.945968in}}{\pgfqpoint{1.815301in}{0.953868in}}{\pgfqpoint{1.815301in}{0.962104in}}%
\pgfpathcurveto{\pgfqpoint{1.815301in}{0.970340in}}{\pgfqpoint{1.812029in}{0.978240in}}{\pgfqpoint{1.806205in}{0.984064in}}%
\pgfpathcurveto{\pgfqpoint{1.800381in}{0.989888in}}{\pgfqpoint{1.792481in}{0.993161in}}{\pgfqpoint{1.784245in}{0.993161in}}%
\pgfpathcurveto{\pgfqpoint{1.776009in}{0.993161in}}{\pgfqpoint{1.768109in}{0.989888in}}{\pgfqpoint{1.762285in}{0.984064in}}%
\pgfpathcurveto{\pgfqpoint{1.756461in}{0.978240in}}{\pgfqpoint{1.753188in}{0.970340in}}{\pgfqpoint{1.753188in}{0.962104in}}%
\pgfpathcurveto{\pgfqpoint{1.753188in}{0.953868in}}{\pgfqpoint{1.756461in}{0.945968in}}{\pgfqpoint{1.762285in}{0.940144in}}%
\pgfpathcurveto{\pgfqpoint{1.768109in}{0.934320in}}{\pgfqpoint{1.776009in}{0.931048in}}{\pgfqpoint{1.784245in}{0.931048in}}%
\pgfpathlineto{\pgfqpoint{1.784245in}{0.931048in}}%
\pgfusepath{stroke,fill}%
\end{pgfscope}%
\begin{pgfscope}%
\pgfpathrectangle{\pgfqpoint{0.548058in}{0.516222in}}{\pgfqpoint{1.739582in}{1.783528in}} %
\pgfusepath{clip}%
\pgfsetbuttcap%
\pgfsetroundjoin%
\definecolor{currentfill}{rgb}{0.298039,0.447059,0.690196}%
\pgfsetfillcolor{currentfill}%
\pgfsetlinewidth{0.240900pt}%
\definecolor{currentstroke}{rgb}{1.000000,1.000000,1.000000}%
\pgfsetstrokecolor{currentstroke}%
\pgfsetdash{}{0pt}%
\pgfpathmoveto{\pgfqpoint{0.924012in}{2.090341in}}%
\pgfpathcurveto{\pgfqpoint{0.932248in}{2.090341in}}{\pgfqpoint{0.940148in}{2.093613in}}{\pgfqpoint{0.945972in}{2.099437in}}%
\pgfpathcurveto{\pgfqpoint{0.951796in}{2.105261in}}{\pgfqpoint{0.955068in}{2.113161in}}{\pgfqpoint{0.955068in}{2.121397in}}%
\pgfpathcurveto{\pgfqpoint{0.955068in}{2.129634in}}{\pgfqpoint{0.951796in}{2.137534in}}{\pgfqpoint{0.945972in}{2.143357in}}%
\pgfpathcurveto{\pgfqpoint{0.940148in}{2.149181in}}{\pgfqpoint{0.932248in}{2.152454in}}{\pgfqpoint{0.924012in}{2.152454in}}%
\pgfpathcurveto{\pgfqpoint{0.915776in}{2.152454in}}{\pgfqpoint{0.907876in}{2.149181in}}{\pgfqpoint{0.902052in}{2.143357in}}%
\pgfpathcurveto{\pgfqpoint{0.896228in}{2.137534in}}{\pgfqpoint{0.892955in}{2.129634in}}{\pgfqpoint{0.892955in}{2.121397in}}%
\pgfpathcurveto{\pgfqpoint{0.892955in}{2.113161in}}{\pgfqpoint{0.896228in}{2.105261in}}{\pgfqpoint{0.902052in}{2.099437in}}%
\pgfpathcurveto{\pgfqpoint{0.907876in}{2.093613in}}{\pgfqpoint{0.915776in}{2.090341in}}{\pgfqpoint{0.924012in}{2.090341in}}%
\pgfpathlineto{\pgfqpoint{0.924012in}{2.090341in}}%
\pgfusepath{stroke,fill}%
\end{pgfscope}%
\begin{pgfscope}%
\pgfpathrectangle{\pgfqpoint{0.548058in}{0.516222in}}{\pgfqpoint{1.739582in}{1.783528in}} %
\pgfusepath{clip}%
\pgfsetbuttcap%
\pgfsetroundjoin%
\definecolor{currentfill}{rgb}{0.298039,0.447059,0.690196}%
\pgfsetfillcolor{currentfill}%
\pgfsetlinewidth{0.240900pt}%
\definecolor{currentstroke}{rgb}{1.000000,1.000000,1.000000}%
\pgfsetstrokecolor{currentstroke}%
\pgfsetdash{}{0pt}%
\pgfpathmoveto{\pgfqpoint{1.975408in}{0.810660in}}%
\pgfpathcurveto{\pgfqpoint{1.983644in}{0.810660in}}{\pgfqpoint{1.991544in}{0.813932in}}{\pgfqpoint{1.997368in}{0.819756in}}%
\pgfpathcurveto{\pgfqpoint{2.003192in}{0.825580in}}{\pgfqpoint{2.006464in}{0.833480in}}{\pgfqpoint{2.006464in}{0.841716in}}%
\pgfpathcurveto{\pgfqpoint{2.006464in}{0.849952in}}{\pgfqpoint{2.003192in}{0.857852in}}{\pgfqpoint{1.997368in}{0.863676in}}%
\pgfpathcurveto{\pgfqpoint{1.991544in}{0.869500in}}{\pgfqpoint{1.983644in}{0.872773in}}{\pgfqpoint{1.975408in}{0.872773in}}%
\pgfpathcurveto{\pgfqpoint{1.967172in}{0.872773in}}{\pgfqpoint{1.959272in}{0.869500in}}{\pgfqpoint{1.953448in}{0.863676in}}%
\pgfpathcurveto{\pgfqpoint{1.947624in}{0.857852in}}{\pgfqpoint{1.944351in}{0.849952in}}{\pgfqpoint{1.944351in}{0.841716in}}%
\pgfpathcurveto{\pgfqpoint{1.944351in}{0.833480in}}{\pgfqpoint{1.947624in}{0.825580in}}{\pgfqpoint{1.953448in}{0.819756in}}%
\pgfpathcurveto{\pgfqpoint{1.959272in}{0.813932in}}{\pgfqpoint{1.967172in}{0.810660in}}{\pgfqpoint{1.975408in}{0.810660in}}%
\pgfpathlineto{\pgfqpoint{1.975408in}{0.810660in}}%
\pgfusepath{stroke,fill}%
\end{pgfscope}%
\begin{pgfscope}%
\pgfpathrectangle{\pgfqpoint{0.548058in}{0.516222in}}{\pgfqpoint{1.739582in}{1.783528in}} %
\pgfusepath{clip}%
\pgfsetbuttcap%
\pgfsetroundjoin%
\definecolor{currentfill}{rgb}{0.298039,0.447059,0.690196}%
\pgfsetfillcolor{currentfill}%
\pgfsetlinewidth{0.240900pt}%
\definecolor{currentstroke}{rgb}{1.000000,1.000000,1.000000}%
\pgfsetstrokecolor{currentstroke}%
\pgfsetdash{}{0pt}%
\pgfpathmoveto{\pgfqpoint{1.624943in}{1.265459in}}%
\pgfpathcurveto{\pgfqpoint{1.633179in}{1.265459in}}{\pgfqpoint{1.641079in}{1.268731in}}{\pgfqpoint{1.646903in}{1.274555in}}%
\pgfpathcurveto{\pgfqpoint{1.652727in}{1.280379in}}{\pgfqpoint{1.655999in}{1.288279in}}{\pgfqpoint{1.655999in}{1.296516in}}%
\pgfpathcurveto{\pgfqpoint{1.655999in}{1.304752in}}{\pgfqpoint{1.652727in}{1.312652in}}{\pgfqpoint{1.646903in}{1.318476in}}%
\pgfpathcurveto{\pgfqpoint{1.641079in}{1.324300in}}{\pgfqpoint{1.633179in}{1.327572in}}{\pgfqpoint{1.624943in}{1.327572in}}%
\pgfpathcurveto{\pgfqpoint{1.616706in}{1.327572in}}{\pgfqpoint{1.608806in}{1.324300in}}{\pgfqpoint{1.602982in}{1.318476in}}%
\pgfpathcurveto{\pgfqpoint{1.597158in}{1.312652in}}{\pgfqpoint{1.593886in}{1.304752in}}{\pgfqpoint{1.593886in}{1.296516in}}%
\pgfpathcurveto{\pgfqpoint{1.593886in}{1.288279in}}{\pgfqpoint{1.597158in}{1.280379in}}{\pgfqpoint{1.602982in}{1.274555in}}%
\pgfpathcurveto{\pgfqpoint{1.608806in}{1.268731in}}{\pgfqpoint{1.616706in}{1.265459in}}{\pgfqpoint{1.624943in}{1.265459in}}%
\pgfpathlineto{\pgfqpoint{1.624943in}{1.265459in}}%
\pgfusepath{stroke,fill}%
\end{pgfscope}%
\begin{pgfscope}%
\pgfpathrectangle{\pgfqpoint{0.548058in}{0.516222in}}{\pgfqpoint{1.739582in}{1.783528in}} %
\pgfusepath{clip}%
\pgfsetbuttcap%
\pgfsetroundjoin%
\definecolor{currentfill}{rgb}{0.298039,0.447059,0.690196}%
\pgfsetfillcolor{currentfill}%
\pgfsetlinewidth{0.240900pt}%
\definecolor{currentstroke}{rgb}{1.000000,1.000000,1.000000}%
\pgfsetstrokecolor{currentstroke}%
\pgfsetdash{}{0pt}%
\pgfpathmoveto{\pgfqpoint{1.879826in}{1.211953in}}%
\pgfpathcurveto{\pgfqpoint{1.888063in}{1.211953in}}{\pgfqpoint{1.895963in}{1.215226in}}{\pgfqpoint{1.901787in}{1.221050in}}%
\pgfpathcurveto{\pgfqpoint{1.907611in}{1.226873in}}{\pgfqpoint{1.910883in}{1.234774in}}{\pgfqpoint{1.910883in}{1.243010in}}%
\pgfpathcurveto{\pgfqpoint{1.910883in}{1.251246in}}{\pgfqpoint{1.907611in}{1.259146in}}{\pgfqpoint{1.901787in}{1.264970in}}%
\pgfpathcurveto{\pgfqpoint{1.895963in}{1.270794in}}{\pgfqpoint{1.888063in}{1.274066in}}{\pgfqpoint{1.879826in}{1.274066in}}%
\pgfpathcurveto{\pgfqpoint{1.871590in}{1.274066in}}{\pgfqpoint{1.863690in}{1.270794in}}{\pgfqpoint{1.857866in}{1.264970in}}%
\pgfpathcurveto{\pgfqpoint{1.852042in}{1.259146in}}{\pgfqpoint{1.848770in}{1.251246in}}{\pgfqpoint{1.848770in}{1.243010in}}%
\pgfpathcurveto{\pgfqpoint{1.848770in}{1.234774in}}{\pgfqpoint{1.852042in}{1.226873in}}{\pgfqpoint{1.857866in}{1.221050in}}%
\pgfpathcurveto{\pgfqpoint{1.863690in}{1.215226in}}{\pgfqpoint{1.871590in}{1.211953in}}{\pgfqpoint{1.879826in}{1.211953in}}%
\pgfpathlineto{\pgfqpoint{1.879826in}{1.211953in}}%
\pgfusepath{stroke,fill}%
\end{pgfscope}%
\begin{pgfscope}%
\pgfpathrectangle{\pgfqpoint{0.548058in}{0.516222in}}{\pgfqpoint{1.739582in}{1.783528in}} %
\pgfusepath{clip}%
\pgfsetbuttcap%
\pgfsetroundjoin%
\definecolor{currentfill}{rgb}{0.298039,0.447059,0.690196}%
\pgfsetfillcolor{currentfill}%
\pgfsetlinewidth{0.240900pt}%
\definecolor{currentstroke}{rgb}{1.000000,1.000000,1.000000}%
\pgfsetstrokecolor{currentstroke}%
\pgfsetdash{}{0pt}%
\pgfpathmoveto{\pgfqpoint{1.593082in}{1.559741in}}%
\pgfpathcurveto{\pgfqpoint{1.601318in}{1.559741in}}{\pgfqpoint{1.609218in}{1.563014in}}{\pgfqpoint{1.615042in}{1.568837in}}%
\pgfpathcurveto{\pgfqpoint{1.620866in}{1.574661in}}{\pgfqpoint{1.624139in}{1.582561in}}{\pgfqpoint{1.624139in}{1.590798in}}%
\pgfpathcurveto{\pgfqpoint{1.624139in}{1.599034in}}{\pgfqpoint{1.620866in}{1.606934in}}{\pgfqpoint{1.615042in}{1.612758in}}%
\pgfpathcurveto{\pgfqpoint{1.609218in}{1.618582in}}{\pgfqpoint{1.601318in}{1.621854in}}{\pgfqpoint{1.593082in}{1.621854in}}%
\pgfpathcurveto{\pgfqpoint{1.584846in}{1.621854in}}{\pgfqpoint{1.576946in}{1.618582in}}{\pgfqpoint{1.571122in}{1.612758in}}%
\pgfpathcurveto{\pgfqpoint{1.565298in}{1.606934in}}{\pgfqpoint{1.562026in}{1.599034in}}{\pgfqpoint{1.562026in}{1.590798in}}%
\pgfpathcurveto{\pgfqpoint{1.562026in}{1.582561in}}{\pgfqpoint{1.565298in}{1.574661in}}{\pgfqpoint{1.571122in}{1.568837in}}%
\pgfpathcurveto{\pgfqpoint{1.576946in}{1.563014in}}{\pgfqpoint{1.584846in}{1.559741in}}{\pgfqpoint{1.593082in}{1.559741in}}%
\pgfpathlineto{\pgfqpoint{1.593082in}{1.559741in}}%
\pgfusepath{stroke,fill}%
\end{pgfscope}%
\begin{pgfscope}%
\pgfpathrectangle{\pgfqpoint{0.548058in}{0.516222in}}{\pgfqpoint{1.739582in}{1.783528in}} %
\pgfusepath{clip}%
\pgfsetbuttcap%
\pgfsetroundjoin%
\definecolor{currentfill}{rgb}{0.298039,0.447059,0.690196}%
\pgfsetfillcolor{currentfill}%
\pgfsetlinewidth{0.240900pt}%
\definecolor{currentstroke}{rgb}{1.000000,1.000000,1.000000}%
\pgfsetstrokecolor{currentstroke}%
\pgfsetdash{}{0pt}%
\pgfpathmoveto{\pgfqpoint{1.529361in}{0.939965in}}%
\pgfpathcurveto{\pgfqpoint{1.537597in}{0.939965in}}{\pgfqpoint{1.545497in}{0.943238in}}{\pgfqpoint{1.551321in}{0.949062in}}%
\pgfpathcurveto{\pgfqpoint{1.557145in}{0.954885in}}{\pgfqpoint{1.560418in}{0.962786in}}{\pgfqpoint{1.560418in}{0.971022in}}%
\pgfpathcurveto{\pgfqpoint{1.560418in}{0.979258in}}{\pgfqpoint{1.557145in}{0.987158in}}{\pgfqpoint{1.551321in}{0.992982in}}%
\pgfpathcurveto{\pgfqpoint{1.545497in}{0.998806in}}{\pgfqpoint{1.537597in}{1.002078in}}{\pgfqpoint{1.529361in}{1.002078in}}%
\pgfpathcurveto{\pgfqpoint{1.521125in}{1.002078in}}{\pgfqpoint{1.513225in}{0.998806in}}{\pgfqpoint{1.507401in}{0.992982in}}%
\pgfpathcurveto{\pgfqpoint{1.501577in}{0.987158in}}{\pgfqpoint{1.498305in}{0.979258in}}{\pgfqpoint{1.498305in}{0.971022in}}%
\pgfpathcurveto{\pgfqpoint{1.498305in}{0.962786in}}{\pgfqpoint{1.501577in}{0.954885in}}{\pgfqpoint{1.507401in}{0.949062in}}%
\pgfpathcurveto{\pgfqpoint{1.513225in}{0.943238in}}{\pgfqpoint{1.521125in}{0.939965in}}{\pgfqpoint{1.529361in}{0.939965in}}%
\pgfpathlineto{\pgfqpoint{1.529361in}{0.939965in}}%
\pgfusepath{stroke,fill}%
\end{pgfscope}%
\begin{pgfscope}%
\pgfpathrectangle{\pgfqpoint{0.548058in}{0.516222in}}{\pgfqpoint{1.739582in}{1.783528in}} %
\pgfusepath{clip}%
\pgfsetbuttcap%
\pgfsetroundjoin%
\definecolor{currentfill}{rgb}{0.298039,0.447059,0.690196}%
\pgfsetfillcolor{currentfill}%
\pgfsetlinewidth{0.240900pt}%
\definecolor{currentstroke}{rgb}{1.000000,1.000000,1.000000}%
\pgfsetstrokecolor{currentstroke}%
\pgfsetdash{}{0pt}%
\pgfpathmoveto{\pgfqpoint{1.115175in}{1.158447in}}%
\pgfpathcurveto{\pgfqpoint{1.123411in}{1.158447in}}{\pgfqpoint{1.131311in}{1.161720in}}{\pgfqpoint{1.137135in}{1.167544in}}%
\pgfpathcurveto{\pgfqpoint{1.142959in}{1.173368in}}{\pgfqpoint{1.146231in}{1.181268in}}{\pgfqpoint{1.146231in}{1.189504in}}%
\pgfpathcurveto{\pgfqpoint{1.146231in}{1.197740in}}{\pgfqpoint{1.142959in}{1.205640in}}{\pgfqpoint{1.137135in}{1.211464in}}%
\pgfpathcurveto{\pgfqpoint{1.131311in}{1.217288in}}{\pgfqpoint{1.123411in}{1.220560in}}{\pgfqpoint{1.115175in}{1.220560in}}%
\pgfpathcurveto{\pgfqpoint{1.106939in}{1.220560in}}{\pgfqpoint{1.099038in}{1.217288in}}{\pgfqpoint{1.093215in}{1.211464in}}%
\pgfpathcurveto{\pgfqpoint{1.087391in}{1.205640in}}{\pgfqpoint{1.084118in}{1.197740in}}{\pgfqpoint{1.084118in}{1.189504in}}%
\pgfpathcurveto{\pgfqpoint{1.084118in}{1.181268in}}{\pgfqpoint{1.087391in}{1.173368in}}{\pgfqpoint{1.093215in}{1.167544in}}%
\pgfpathcurveto{\pgfqpoint{1.099038in}{1.161720in}}{\pgfqpoint{1.106939in}{1.158447in}}{\pgfqpoint{1.115175in}{1.158447in}}%
\pgfpathlineto{\pgfqpoint{1.115175in}{1.158447in}}%
\pgfusepath{stroke,fill}%
\end{pgfscope}%
\begin{pgfscope}%
\pgfpathrectangle{\pgfqpoint{0.548058in}{0.516222in}}{\pgfqpoint{1.739582in}{1.783528in}} %
\pgfusepath{clip}%
\pgfsetbuttcap%
\pgfsetroundjoin%
\definecolor{currentfill}{rgb}{0.298039,0.447059,0.690196}%
\pgfsetfillcolor{currentfill}%
\pgfsetlinewidth{0.240900pt}%
\definecolor{currentstroke}{rgb}{1.000000,1.000000,1.000000}%
\pgfsetstrokecolor{currentstroke}%
\pgfsetdash{}{0pt}%
\pgfpathmoveto{\pgfqpoint{1.274477in}{1.265459in}}%
\pgfpathcurveto{\pgfqpoint{1.282713in}{1.265459in}}{\pgfqpoint{1.290614in}{1.268731in}}{\pgfqpoint{1.296437in}{1.274555in}}%
\pgfpathcurveto{\pgfqpoint{1.302261in}{1.280379in}}{\pgfqpoint{1.305534in}{1.288279in}}{\pgfqpoint{1.305534in}{1.296516in}}%
\pgfpathcurveto{\pgfqpoint{1.305534in}{1.304752in}}{\pgfqpoint{1.302261in}{1.312652in}}{\pgfqpoint{1.296437in}{1.318476in}}%
\pgfpathcurveto{\pgfqpoint{1.290614in}{1.324300in}}{\pgfqpoint{1.282713in}{1.327572in}}{\pgfqpoint{1.274477in}{1.327572in}}%
\pgfpathcurveto{\pgfqpoint{1.266241in}{1.327572in}}{\pgfqpoint{1.258341in}{1.324300in}}{\pgfqpoint{1.252517in}{1.318476in}}%
\pgfpathcurveto{\pgfqpoint{1.246693in}{1.312652in}}{\pgfqpoint{1.243421in}{1.304752in}}{\pgfqpoint{1.243421in}{1.296516in}}%
\pgfpathcurveto{\pgfqpoint{1.243421in}{1.288279in}}{\pgfqpoint{1.246693in}{1.280379in}}{\pgfqpoint{1.252517in}{1.274555in}}%
\pgfpathcurveto{\pgfqpoint{1.258341in}{1.268731in}}{\pgfqpoint{1.266241in}{1.265459in}}{\pgfqpoint{1.274477in}{1.265459in}}%
\pgfpathlineto{\pgfqpoint{1.274477in}{1.265459in}}%
\pgfusepath{stroke,fill}%
\end{pgfscope}%
\begin{pgfscope}%
\pgfpathrectangle{\pgfqpoint{0.548058in}{0.516222in}}{\pgfqpoint{1.739582in}{1.783528in}} %
\pgfusepath{clip}%
\pgfsetbuttcap%
\pgfsetroundjoin%
\definecolor{currentfill}{rgb}{0.298039,0.447059,0.690196}%
\pgfsetfillcolor{currentfill}%
\pgfsetlinewidth{0.240900pt}%
\definecolor{currentstroke}{rgb}{1.000000,1.000000,1.000000}%
\pgfsetstrokecolor{currentstroke}%
\pgfsetdash{}{0pt}%
\pgfpathmoveto{\pgfqpoint{1.561222in}{1.920906in}}%
\pgfpathcurveto{\pgfqpoint{1.569458in}{1.920906in}}{\pgfqpoint{1.577358in}{1.924178in}}{\pgfqpoint{1.583182in}{1.930002in}}%
\pgfpathcurveto{\pgfqpoint{1.589006in}{1.935826in}}{\pgfqpoint{1.592278in}{1.943726in}}{\pgfqpoint{1.592278in}{1.951962in}}%
\pgfpathcurveto{\pgfqpoint{1.592278in}{1.960198in}}{\pgfqpoint{1.589006in}{1.968098in}}{\pgfqpoint{1.583182in}{1.973922in}}%
\pgfpathcurveto{\pgfqpoint{1.577358in}{1.979746in}}{\pgfqpoint{1.569458in}{1.983019in}}{\pgfqpoint{1.561222in}{1.983019in}}%
\pgfpathcurveto{\pgfqpoint{1.552985in}{1.983019in}}{\pgfqpoint{1.545085in}{1.979746in}}{\pgfqpoint{1.539261in}{1.973922in}}%
\pgfpathcurveto{\pgfqpoint{1.533437in}{1.968098in}}{\pgfqpoint{1.530165in}{1.960198in}}{\pgfqpoint{1.530165in}{1.951962in}}%
\pgfpathcurveto{\pgfqpoint{1.530165in}{1.943726in}}{\pgfqpoint{1.533437in}{1.935826in}}{\pgfqpoint{1.539261in}{1.930002in}}%
\pgfpathcurveto{\pgfqpoint{1.545085in}{1.924178in}}{\pgfqpoint{1.552985in}{1.920906in}}{\pgfqpoint{1.561222in}{1.920906in}}%
\pgfpathlineto{\pgfqpoint{1.561222in}{1.920906in}}%
\pgfusepath{stroke,fill}%
\end{pgfscope}%
\begin{pgfscope}%
\pgfpathrectangle{\pgfqpoint{0.548058in}{0.516222in}}{\pgfqpoint{1.739582in}{1.783528in}} %
\pgfusepath{clip}%
\pgfsetbuttcap%
\pgfsetroundjoin%
\definecolor{currentfill}{rgb}{0.298039,0.447059,0.690196}%
\pgfsetfillcolor{currentfill}%
\pgfsetlinewidth{0.240900pt}%
\definecolor{currentstroke}{rgb}{1.000000,1.000000,1.000000}%
\pgfsetstrokecolor{currentstroke}%
\pgfsetdash{}{0pt}%
\pgfpathmoveto{\pgfqpoint{0.892151in}{1.051436in}}%
\pgfpathcurveto{\pgfqpoint{0.900388in}{1.051436in}}{\pgfqpoint{0.908288in}{1.054708in}}{\pgfqpoint{0.914112in}{1.060532in}}%
\pgfpathcurveto{\pgfqpoint{0.919936in}{1.066356in}}{\pgfqpoint{0.923208in}{1.074256in}}{\pgfqpoint{0.923208in}{1.082492in}}%
\pgfpathcurveto{\pgfqpoint{0.923208in}{1.090729in}}{\pgfqpoint{0.919936in}{1.098629in}}{\pgfqpoint{0.914112in}{1.104453in}}%
\pgfpathcurveto{\pgfqpoint{0.908288in}{1.110276in}}{\pgfqpoint{0.900388in}{1.113549in}}{\pgfqpoint{0.892151in}{1.113549in}}%
\pgfpathcurveto{\pgfqpoint{0.883915in}{1.113549in}}{\pgfqpoint{0.876015in}{1.110276in}}{\pgfqpoint{0.870191in}{1.104453in}}%
\pgfpathcurveto{\pgfqpoint{0.864367in}{1.098629in}}{\pgfqpoint{0.861095in}{1.090729in}}{\pgfqpoint{0.861095in}{1.082492in}}%
\pgfpathcurveto{\pgfqpoint{0.861095in}{1.074256in}}{\pgfqpoint{0.864367in}{1.066356in}}{\pgfqpoint{0.870191in}{1.060532in}}%
\pgfpathcurveto{\pgfqpoint{0.876015in}{1.054708in}}{\pgfqpoint{0.883915in}{1.051436in}}{\pgfqpoint{0.892151in}{1.051436in}}%
\pgfpathlineto{\pgfqpoint{0.892151in}{1.051436in}}%
\pgfusepath{stroke,fill}%
\end{pgfscope}%
\begin{pgfscope}%
\pgfpathrectangle{\pgfqpoint{0.548058in}{0.516222in}}{\pgfqpoint{1.739582in}{1.783528in}} %
\pgfusepath{clip}%
\pgfsetbuttcap%
\pgfsetroundjoin%
\definecolor{currentfill}{rgb}{0.298039,0.447059,0.690196}%
\pgfsetfillcolor{currentfill}%
\pgfsetlinewidth{0.240900pt}%
\definecolor{currentstroke}{rgb}{1.000000,1.000000,1.000000}%
\pgfsetstrokecolor{currentstroke}%
\pgfsetdash{}{0pt}%
\pgfpathmoveto{\pgfqpoint{1.210756in}{1.345718in}}%
\pgfpathcurveto{\pgfqpoint{1.218993in}{1.345718in}}{\pgfqpoint{1.226893in}{1.348990in}}{\pgfqpoint{1.232717in}{1.354814in}}%
\pgfpathcurveto{\pgfqpoint{1.238540in}{1.360638in}}{\pgfqpoint{1.241813in}{1.368538in}}{\pgfqpoint{1.241813in}{1.376774in}}%
\pgfpathcurveto{\pgfqpoint{1.241813in}{1.385011in}}{\pgfqpoint{1.238540in}{1.392911in}}{\pgfqpoint{1.232717in}{1.398735in}}%
\pgfpathcurveto{\pgfqpoint{1.226893in}{1.404559in}}{\pgfqpoint{1.218993in}{1.407831in}}{\pgfqpoint{1.210756in}{1.407831in}}%
\pgfpathcurveto{\pgfqpoint{1.202520in}{1.407831in}}{\pgfqpoint{1.194620in}{1.404559in}}{\pgfqpoint{1.188796in}{1.398735in}}%
\pgfpathcurveto{\pgfqpoint{1.182972in}{1.392911in}}{\pgfqpoint{1.179700in}{1.385011in}}{\pgfqpoint{1.179700in}{1.376774in}}%
\pgfpathcurveto{\pgfqpoint{1.179700in}{1.368538in}}{\pgfqpoint{1.182972in}{1.360638in}}{\pgfqpoint{1.188796in}{1.354814in}}%
\pgfpathcurveto{\pgfqpoint{1.194620in}{1.348990in}}{\pgfqpoint{1.202520in}{1.345718in}}{\pgfqpoint{1.210756in}{1.345718in}}%
\pgfpathlineto{\pgfqpoint{1.210756in}{1.345718in}}%
\pgfusepath{stroke,fill}%
\end{pgfscope}%
\begin{pgfscope}%
\pgfsetrectcap%
\pgfsetmiterjoin%
\pgfsetlinewidth{0.000000pt}%
\definecolor{currentstroke}{rgb}{1.000000,1.000000,1.000000}%
\pgfsetstrokecolor{currentstroke}%
\pgfsetdash{}{0pt}%
\pgfpathmoveto{\pgfqpoint{0.548058in}{0.516222in}}%
\pgfpathlineto{\pgfqpoint{2.287641in}{0.516222in}}%
\pgfusepath{}%
\end{pgfscope}%
\begin{pgfscope}%
\pgfsetrectcap%
\pgfsetmiterjoin%
\pgfsetlinewidth{0.000000pt}%
\definecolor{currentstroke}{rgb}{1.000000,1.000000,1.000000}%
\pgfsetstrokecolor{currentstroke}%
\pgfsetdash{}{0pt}%
\pgfpathmoveto{\pgfqpoint{0.548058in}{0.516222in}}%
\pgfpathlineto{\pgfqpoint{0.548058in}{2.299750in}}%
\pgfusepath{}%
\end{pgfscope}%
\end{pgfpicture}%
\makeatother%
\endgroup%

		\caption{Comparison between the two times registered for one throw.}
		\label{fig_wtr_vs_obs2}
	\end{subfigure}
	\begin{subfigure}[h]{.5\linewidth}
		%% Creator: Matplotlib, PGF backend
%%
%% To include the figure in your LaTeX document, write
%%   \input{<filename>.pgf}
%%
%% Make sure the required packages are loaded in your preamble
%%   \usepackage{pgf}
%%
%% Figures using additional raster images can only be included by \input if
%% they are in the same directory as the main LaTeX file. For loading figures
%% from other directories you can use the `import` package
%%   \usepackage{import}
%% and then include the figures with
%%   \import{<path to file>}{<filename>.pgf}
%%
%% Matplotlib used the following preamble
%%   \usepackage{fontspec}
%%   \setmainfont{Times New Roman}
%%   \setsansfont{Arial}
%%   \setmonofont{Andale Mono}
%%
\begingroup%
\makeatletter%
\begin{pgfpicture}%
\pgfpathrectangle{\pgfpointorigin}{\pgfqpoint{2.500000in}{2.500000in}}%
\pgfusepath{use as bounding box, clip}%
\begin{pgfscope}%
\pgfsetbuttcap%
\pgfsetmiterjoin%
\definecolor{currentfill}{rgb}{1.000000,1.000000,1.000000}%
\pgfsetfillcolor{currentfill}%
\pgfsetlinewidth{0.000000pt}%
\definecolor{currentstroke}{rgb}{1.000000,1.000000,1.000000}%
\pgfsetstrokecolor{currentstroke}%
\pgfsetdash{}{0pt}%
\pgfpathmoveto{\pgfqpoint{0.000000in}{0.000000in}}%
\pgfpathlineto{\pgfqpoint{2.500000in}{0.000000in}}%
\pgfpathlineto{\pgfqpoint{2.500000in}{2.500000in}}%
\pgfpathlineto{\pgfqpoint{0.000000in}{2.500000in}}%
\pgfpathclose%
\pgfusepath{fill}%
\end{pgfscope}%
\begin{pgfscope}%
\pgfsetbuttcap%
\pgfsetmiterjoin%
\definecolor{currentfill}{rgb}{0.917647,0.917647,0.949020}%
\pgfsetfillcolor{currentfill}%
\pgfsetlinewidth{0.000000pt}%
\definecolor{currentstroke}{rgb}{0.000000,0.000000,0.000000}%
\pgfsetstrokecolor{currentstroke}%
\pgfsetstrokeopacity{0.000000}%
\pgfsetdash{}{0pt}%
\pgfpathmoveto{\pgfqpoint{0.624479in}{0.567253in}}%
\pgfpathlineto{\pgfqpoint{2.262793in}{0.567253in}}%
\pgfpathlineto{\pgfqpoint{2.262793in}{2.287500in}}%
\pgfpathlineto{\pgfqpoint{0.624479in}{2.287500in}}%
\pgfpathclose%
\pgfusepath{fill}%
\end{pgfscope}%
\begin{pgfscope}%
\pgfpathrectangle{\pgfqpoint{0.624479in}{0.567253in}}{\pgfqpoint{1.638314in}{1.720247in}} %
\pgfusepath{clip}%
\pgfsetroundcap%
\pgfsetroundjoin%
\pgfsetlinewidth{1.003750pt}%
\definecolor{currentstroke}{rgb}{1.000000,1.000000,1.000000}%
\pgfsetstrokecolor{currentstroke}%
\pgfsetdash{}{0pt}%
\pgfpathmoveto{\pgfqpoint{0.624479in}{0.567253in}}%
\pgfpathlineto{\pgfqpoint{0.624479in}{2.287500in}}%
\pgfusepath{stroke}%
\end{pgfscope}%
\begin{pgfscope}%
\pgfsetbuttcap%
\pgfsetroundjoin%
\definecolor{currentfill}{rgb}{0.150000,0.150000,0.150000}%
\pgfsetfillcolor{currentfill}%
\pgfsetlinewidth{1.003750pt}%
\definecolor{currentstroke}{rgb}{0.150000,0.150000,0.150000}%
\pgfsetstrokecolor{currentstroke}%
\pgfsetdash{}{0pt}%
\pgfsys@defobject{currentmarker}{\pgfqpoint{0.000000in}{0.000000in}}{\pgfqpoint{0.000000in}{0.000000in}}{%
\pgfpathmoveto{\pgfqpoint{0.000000in}{0.000000in}}%
\pgfpathlineto{\pgfqpoint{0.000000in}{0.000000in}}%
\pgfusepath{stroke,fill}%
}%
\begin{pgfscope}%
\pgfsys@transformshift{0.624479in}{0.567253in}%
\pgfsys@useobject{currentmarker}{}%
\end{pgfscope}%
\end{pgfscope}%
\begin{pgfscope}%
\definecolor{textcolor}{rgb}{0.150000,0.150000,0.150000}%
\pgfsetstrokecolor{textcolor}%
\pgfsetfillcolor{textcolor}%
\pgftext[x=0.624479in,y=0.470030in,,top]{\color{textcolor}\sffamily\fontsize{10.000000}{12.000000}\selectfont 0.8}%
\end{pgfscope}%
\begin{pgfscope}%
\pgfpathrectangle{\pgfqpoint{0.624479in}{0.567253in}}{\pgfqpoint{1.638314in}{1.720247in}} %
\pgfusepath{clip}%
\pgfsetroundcap%
\pgfsetroundjoin%
\pgfsetlinewidth{1.003750pt}%
\definecolor{currentstroke}{rgb}{1.000000,1.000000,1.000000}%
\pgfsetstrokecolor{currentstroke}%
\pgfsetdash{}{0pt}%
\pgfpathmoveto{\pgfqpoint{0.858524in}{0.567253in}}%
\pgfpathlineto{\pgfqpoint{0.858524in}{2.287500in}}%
\pgfusepath{stroke}%
\end{pgfscope}%
\begin{pgfscope}%
\pgfsetbuttcap%
\pgfsetroundjoin%
\definecolor{currentfill}{rgb}{0.150000,0.150000,0.150000}%
\pgfsetfillcolor{currentfill}%
\pgfsetlinewidth{1.003750pt}%
\definecolor{currentstroke}{rgb}{0.150000,0.150000,0.150000}%
\pgfsetstrokecolor{currentstroke}%
\pgfsetdash{}{0pt}%
\pgfsys@defobject{currentmarker}{\pgfqpoint{0.000000in}{0.000000in}}{\pgfqpoint{0.000000in}{0.000000in}}{%
\pgfpathmoveto{\pgfqpoint{0.000000in}{0.000000in}}%
\pgfpathlineto{\pgfqpoint{0.000000in}{0.000000in}}%
\pgfusepath{stroke,fill}%
}%
\begin{pgfscope}%
\pgfsys@transformshift{0.858524in}{0.567253in}%
\pgfsys@useobject{currentmarker}{}%
\end{pgfscope}%
\end{pgfscope}%
\begin{pgfscope}%
\definecolor{textcolor}{rgb}{0.150000,0.150000,0.150000}%
\pgfsetstrokecolor{textcolor}%
\pgfsetfillcolor{textcolor}%
\pgftext[x=0.858524in,y=0.470030in,,top]{\color{textcolor}\sffamily\fontsize{10.000000}{12.000000}\selectfont 1.0}%
\end{pgfscope}%
\begin{pgfscope}%
\pgfpathrectangle{\pgfqpoint{0.624479in}{0.567253in}}{\pgfqpoint{1.638314in}{1.720247in}} %
\pgfusepath{clip}%
\pgfsetroundcap%
\pgfsetroundjoin%
\pgfsetlinewidth{1.003750pt}%
\definecolor{currentstroke}{rgb}{1.000000,1.000000,1.000000}%
\pgfsetstrokecolor{currentstroke}%
\pgfsetdash{}{0pt}%
\pgfpathmoveto{\pgfqpoint{1.092569in}{0.567253in}}%
\pgfpathlineto{\pgfqpoint{1.092569in}{2.287500in}}%
\pgfusepath{stroke}%
\end{pgfscope}%
\begin{pgfscope}%
\pgfsetbuttcap%
\pgfsetroundjoin%
\definecolor{currentfill}{rgb}{0.150000,0.150000,0.150000}%
\pgfsetfillcolor{currentfill}%
\pgfsetlinewidth{1.003750pt}%
\definecolor{currentstroke}{rgb}{0.150000,0.150000,0.150000}%
\pgfsetstrokecolor{currentstroke}%
\pgfsetdash{}{0pt}%
\pgfsys@defobject{currentmarker}{\pgfqpoint{0.000000in}{0.000000in}}{\pgfqpoint{0.000000in}{0.000000in}}{%
\pgfpathmoveto{\pgfqpoint{0.000000in}{0.000000in}}%
\pgfpathlineto{\pgfqpoint{0.000000in}{0.000000in}}%
\pgfusepath{stroke,fill}%
}%
\begin{pgfscope}%
\pgfsys@transformshift{1.092569in}{0.567253in}%
\pgfsys@useobject{currentmarker}{}%
\end{pgfscope}%
\end{pgfscope}%
\begin{pgfscope}%
\definecolor{textcolor}{rgb}{0.150000,0.150000,0.150000}%
\pgfsetstrokecolor{textcolor}%
\pgfsetfillcolor{textcolor}%
\pgftext[x=1.092569in,y=0.470030in,,top]{\color{textcolor}\sffamily\fontsize{10.000000}{12.000000}\selectfont 1.2}%
\end{pgfscope}%
\begin{pgfscope}%
\pgfpathrectangle{\pgfqpoint{0.624479in}{0.567253in}}{\pgfqpoint{1.638314in}{1.720247in}} %
\pgfusepath{clip}%
\pgfsetroundcap%
\pgfsetroundjoin%
\pgfsetlinewidth{1.003750pt}%
\definecolor{currentstroke}{rgb}{1.000000,1.000000,1.000000}%
\pgfsetstrokecolor{currentstroke}%
\pgfsetdash{}{0pt}%
\pgfpathmoveto{\pgfqpoint{1.326614in}{0.567253in}}%
\pgfpathlineto{\pgfqpoint{1.326614in}{2.287500in}}%
\pgfusepath{stroke}%
\end{pgfscope}%
\begin{pgfscope}%
\pgfsetbuttcap%
\pgfsetroundjoin%
\definecolor{currentfill}{rgb}{0.150000,0.150000,0.150000}%
\pgfsetfillcolor{currentfill}%
\pgfsetlinewidth{1.003750pt}%
\definecolor{currentstroke}{rgb}{0.150000,0.150000,0.150000}%
\pgfsetstrokecolor{currentstroke}%
\pgfsetdash{}{0pt}%
\pgfsys@defobject{currentmarker}{\pgfqpoint{0.000000in}{0.000000in}}{\pgfqpoint{0.000000in}{0.000000in}}{%
\pgfpathmoveto{\pgfqpoint{0.000000in}{0.000000in}}%
\pgfpathlineto{\pgfqpoint{0.000000in}{0.000000in}}%
\pgfusepath{stroke,fill}%
}%
\begin{pgfscope}%
\pgfsys@transformshift{1.326614in}{0.567253in}%
\pgfsys@useobject{currentmarker}{}%
\end{pgfscope}%
\end{pgfscope}%
\begin{pgfscope}%
\definecolor{textcolor}{rgb}{0.150000,0.150000,0.150000}%
\pgfsetstrokecolor{textcolor}%
\pgfsetfillcolor{textcolor}%
\pgftext[x=1.326614in,y=0.470030in,,top]{\color{textcolor}\sffamily\fontsize{10.000000}{12.000000}\selectfont 1.4}%
\end{pgfscope}%
\begin{pgfscope}%
\pgfpathrectangle{\pgfqpoint{0.624479in}{0.567253in}}{\pgfqpoint{1.638314in}{1.720247in}} %
\pgfusepath{clip}%
\pgfsetroundcap%
\pgfsetroundjoin%
\pgfsetlinewidth{1.003750pt}%
\definecolor{currentstroke}{rgb}{1.000000,1.000000,1.000000}%
\pgfsetstrokecolor{currentstroke}%
\pgfsetdash{}{0pt}%
\pgfpathmoveto{\pgfqpoint{1.560658in}{0.567253in}}%
\pgfpathlineto{\pgfqpoint{1.560658in}{2.287500in}}%
\pgfusepath{stroke}%
\end{pgfscope}%
\begin{pgfscope}%
\pgfsetbuttcap%
\pgfsetroundjoin%
\definecolor{currentfill}{rgb}{0.150000,0.150000,0.150000}%
\pgfsetfillcolor{currentfill}%
\pgfsetlinewidth{1.003750pt}%
\definecolor{currentstroke}{rgb}{0.150000,0.150000,0.150000}%
\pgfsetstrokecolor{currentstroke}%
\pgfsetdash{}{0pt}%
\pgfsys@defobject{currentmarker}{\pgfqpoint{0.000000in}{0.000000in}}{\pgfqpoint{0.000000in}{0.000000in}}{%
\pgfpathmoveto{\pgfqpoint{0.000000in}{0.000000in}}%
\pgfpathlineto{\pgfqpoint{0.000000in}{0.000000in}}%
\pgfusepath{stroke,fill}%
}%
\begin{pgfscope}%
\pgfsys@transformshift{1.560658in}{0.567253in}%
\pgfsys@useobject{currentmarker}{}%
\end{pgfscope}%
\end{pgfscope}%
\begin{pgfscope}%
\definecolor{textcolor}{rgb}{0.150000,0.150000,0.150000}%
\pgfsetstrokecolor{textcolor}%
\pgfsetfillcolor{textcolor}%
\pgftext[x=1.560658in,y=0.470030in,,top]{\color{textcolor}\sffamily\fontsize{10.000000}{12.000000}\selectfont 1.6}%
\end{pgfscope}%
\begin{pgfscope}%
\pgfpathrectangle{\pgfqpoint{0.624479in}{0.567253in}}{\pgfqpoint{1.638314in}{1.720247in}} %
\pgfusepath{clip}%
\pgfsetroundcap%
\pgfsetroundjoin%
\pgfsetlinewidth{1.003750pt}%
\definecolor{currentstroke}{rgb}{1.000000,1.000000,1.000000}%
\pgfsetstrokecolor{currentstroke}%
\pgfsetdash{}{0pt}%
\pgfpathmoveto{\pgfqpoint{1.794703in}{0.567253in}}%
\pgfpathlineto{\pgfqpoint{1.794703in}{2.287500in}}%
\pgfusepath{stroke}%
\end{pgfscope}%
\begin{pgfscope}%
\pgfsetbuttcap%
\pgfsetroundjoin%
\definecolor{currentfill}{rgb}{0.150000,0.150000,0.150000}%
\pgfsetfillcolor{currentfill}%
\pgfsetlinewidth{1.003750pt}%
\definecolor{currentstroke}{rgb}{0.150000,0.150000,0.150000}%
\pgfsetstrokecolor{currentstroke}%
\pgfsetdash{}{0pt}%
\pgfsys@defobject{currentmarker}{\pgfqpoint{0.000000in}{0.000000in}}{\pgfqpoint{0.000000in}{0.000000in}}{%
\pgfpathmoveto{\pgfqpoint{0.000000in}{0.000000in}}%
\pgfpathlineto{\pgfqpoint{0.000000in}{0.000000in}}%
\pgfusepath{stroke,fill}%
}%
\begin{pgfscope}%
\pgfsys@transformshift{1.794703in}{0.567253in}%
\pgfsys@useobject{currentmarker}{}%
\end{pgfscope}%
\end{pgfscope}%
\begin{pgfscope}%
\definecolor{textcolor}{rgb}{0.150000,0.150000,0.150000}%
\pgfsetstrokecolor{textcolor}%
\pgfsetfillcolor{textcolor}%
\pgftext[x=1.794703in,y=0.470030in,,top]{\color{textcolor}\sffamily\fontsize{10.000000}{12.000000}\selectfont 1.8}%
\end{pgfscope}%
\begin{pgfscope}%
\pgfpathrectangle{\pgfqpoint{0.624479in}{0.567253in}}{\pgfqpoint{1.638314in}{1.720247in}} %
\pgfusepath{clip}%
\pgfsetroundcap%
\pgfsetroundjoin%
\pgfsetlinewidth{1.003750pt}%
\definecolor{currentstroke}{rgb}{1.000000,1.000000,1.000000}%
\pgfsetstrokecolor{currentstroke}%
\pgfsetdash{}{0pt}%
\pgfpathmoveto{\pgfqpoint{2.028748in}{0.567253in}}%
\pgfpathlineto{\pgfqpoint{2.028748in}{2.287500in}}%
\pgfusepath{stroke}%
\end{pgfscope}%
\begin{pgfscope}%
\pgfsetbuttcap%
\pgfsetroundjoin%
\definecolor{currentfill}{rgb}{0.150000,0.150000,0.150000}%
\pgfsetfillcolor{currentfill}%
\pgfsetlinewidth{1.003750pt}%
\definecolor{currentstroke}{rgb}{0.150000,0.150000,0.150000}%
\pgfsetstrokecolor{currentstroke}%
\pgfsetdash{}{0pt}%
\pgfsys@defobject{currentmarker}{\pgfqpoint{0.000000in}{0.000000in}}{\pgfqpoint{0.000000in}{0.000000in}}{%
\pgfpathmoveto{\pgfqpoint{0.000000in}{0.000000in}}%
\pgfpathlineto{\pgfqpoint{0.000000in}{0.000000in}}%
\pgfusepath{stroke,fill}%
}%
\begin{pgfscope}%
\pgfsys@transformshift{2.028748in}{0.567253in}%
\pgfsys@useobject{currentmarker}{}%
\end{pgfscope}%
\end{pgfscope}%
\begin{pgfscope}%
\definecolor{textcolor}{rgb}{0.150000,0.150000,0.150000}%
\pgfsetstrokecolor{textcolor}%
\pgfsetfillcolor{textcolor}%
\pgftext[x=2.028748in,y=0.470030in,,top]{\color{textcolor}\sffamily\fontsize{10.000000}{12.000000}\selectfont 2.0}%
\end{pgfscope}%
\begin{pgfscope}%
\pgfpathrectangle{\pgfqpoint{0.624479in}{0.567253in}}{\pgfqpoint{1.638314in}{1.720247in}} %
\pgfusepath{clip}%
\pgfsetroundcap%
\pgfsetroundjoin%
\pgfsetlinewidth{1.003750pt}%
\definecolor{currentstroke}{rgb}{1.000000,1.000000,1.000000}%
\pgfsetstrokecolor{currentstroke}%
\pgfsetdash{}{0pt}%
\pgfpathmoveto{\pgfqpoint{2.262793in}{0.567253in}}%
\pgfpathlineto{\pgfqpoint{2.262793in}{2.287500in}}%
\pgfusepath{stroke}%
\end{pgfscope}%
\begin{pgfscope}%
\pgfsetbuttcap%
\pgfsetroundjoin%
\definecolor{currentfill}{rgb}{0.150000,0.150000,0.150000}%
\pgfsetfillcolor{currentfill}%
\pgfsetlinewidth{1.003750pt}%
\definecolor{currentstroke}{rgb}{0.150000,0.150000,0.150000}%
\pgfsetstrokecolor{currentstroke}%
\pgfsetdash{}{0pt}%
\pgfsys@defobject{currentmarker}{\pgfqpoint{0.000000in}{0.000000in}}{\pgfqpoint{0.000000in}{0.000000in}}{%
\pgfpathmoveto{\pgfqpoint{0.000000in}{0.000000in}}%
\pgfpathlineto{\pgfqpoint{0.000000in}{0.000000in}}%
\pgfusepath{stroke,fill}%
}%
\begin{pgfscope}%
\pgfsys@transformshift{2.262793in}{0.567253in}%
\pgfsys@useobject{currentmarker}{}%
\end{pgfscope}%
\end{pgfscope}%
\begin{pgfscope}%
\definecolor{textcolor}{rgb}{0.150000,0.150000,0.150000}%
\pgfsetstrokecolor{textcolor}%
\pgfsetfillcolor{textcolor}%
\pgftext[x=2.262793in,y=0.470030in,,top]{\color{textcolor}\sffamily\fontsize{10.000000}{12.000000}\selectfont 2.2}%
\end{pgfscope}%
\begin{pgfscope}%
\definecolor{textcolor}{rgb}{0.150000,0.150000,0.150000}%
\pgfsetstrokecolor{textcolor}%
\pgfsetfillcolor{textcolor}%
\pgftext[x=1.443636in,y=0.273565in,,top]{\color{textcolor}\sffamily\fontsize{11.000000}{13.200000}\selectfont wing tail ratio}%
\end{pgfscope}%
\begin{pgfscope}%
\pgfpathrectangle{\pgfqpoint{0.624479in}{0.567253in}}{\pgfqpoint{1.638314in}{1.720247in}} %
\pgfusepath{clip}%
\pgfsetroundcap%
\pgfsetroundjoin%
\pgfsetlinewidth{1.003750pt}%
\definecolor{currentstroke}{rgb}{1.000000,1.000000,1.000000}%
\pgfsetstrokecolor{currentstroke}%
\pgfsetdash{}{0pt}%
\pgfpathmoveto{\pgfqpoint{0.624479in}{0.567253in}}%
\pgfpathlineto{\pgfqpoint{2.262793in}{0.567253in}}%
\pgfusepath{stroke}%
\end{pgfscope}%
\begin{pgfscope}%
\pgfsetbuttcap%
\pgfsetroundjoin%
\definecolor{currentfill}{rgb}{0.150000,0.150000,0.150000}%
\pgfsetfillcolor{currentfill}%
\pgfsetlinewidth{1.003750pt}%
\definecolor{currentstroke}{rgb}{0.150000,0.150000,0.150000}%
\pgfsetstrokecolor{currentstroke}%
\pgfsetdash{}{0pt}%
\pgfsys@defobject{currentmarker}{\pgfqpoint{0.000000in}{0.000000in}}{\pgfqpoint{0.000000in}{0.000000in}}{%
\pgfpathmoveto{\pgfqpoint{0.000000in}{0.000000in}}%
\pgfpathlineto{\pgfqpoint{0.000000in}{0.000000in}}%
\pgfusepath{stroke,fill}%
}%
\begin{pgfscope}%
\pgfsys@transformshift{0.624479in}{0.567253in}%
\pgfsys@useobject{currentmarker}{}%
\end{pgfscope}%
\end{pgfscope}%
\begin{pgfscope}%
\definecolor{textcolor}{rgb}{0.150000,0.150000,0.150000}%
\pgfsetstrokecolor{textcolor}%
\pgfsetfillcolor{textcolor}%
\pgftext[x=0.527257in,y=0.567253in,right,]{\color{textcolor}\sffamily\fontsize{10.000000}{12.000000}\selectfont 2.5}%
\end{pgfscope}%
\begin{pgfscope}%
\pgfpathrectangle{\pgfqpoint{0.624479in}{0.567253in}}{\pgfqpoint{1.638314in}{1.720247in}} %
\pgfusepath{clip}%
\pgfsetroundcap%
\pgfsetroundjoin%
\pgfsetlinewidth{1.003750pt}%
\definecolor{currentstroke}{rgb}{1.000000,1.000000,1.000000}%
\pgfsetstrokecolor{currentstroke}%
\pgfsetdash{}{0pt}%
\pgfpathmoveto{\pgfqpoint{0.624479in}{0.813002in}}%
\pgfpathlineto{\pgfqpoint{2.262793in}{0.813002in}}%
\pgfusepath{stroke}%
\end{pgfscope}%
\begin{pgfscope}%
\pgfsetbuttcap%
\pgfsetroundjoin%
\definecolor{currentfill}{rgb}{0.150000,0.150000,0.150000}%
\pgfsetfillcolor{currentfill}%
\pgfsetlinewidth{1.003750pt}%
\definecolor{currentstroke}{rgb}{0.150000,0.150000,0.150000}%
\pgfsetstrokecolor{currentstroke}%
\pgfsetdash{}{0pt}%
\pgfsys@defobject{currentmarker}{\pgfqpoint{0.000000in}{0.000000in}}{\pgfqpoint{0.000000in}{0.000000in}}{%
\pgfpathmoveto{\pgfqpoint{0.000000in}{0.000000in}}%
\pgfpathlineto{\pgfqpoint{0.000000in}{0.000000in}}%
\pgfusepath{stroke,fill}%
}%
\begin{pgfscope}%
\pgfsys@transformshift{0.624479in}{0.813002in}%
\pgfsys@useobject{currentmarker}{}%
\end{pgfscope}%
\end{pgfscope}%
\begin{pgfscope}%
\definecolor{textcolor}{rgb}{0.150000,0.150000,0.150000}%
\pgfsetstrokecolor{textcolor}%
\pgfsetfillcolor{textcolor}%
\pgftext[x=0.527257in,y=0.813002in,right,]{\color{textcolor}\sffamily\fontsize{10.000000}{12.000000}\selectfont 3.0}%
\end{pgfscope}%
\begin{pgfscope}%
\pgfpathrectangle{\pgfqpoint{0.624479in}{0.567253in}}{\pgfqpoint{1.638314in}{1.720247in}} %
\pgfusepath{clip}%
\pgfsetroundcap%
\pgfsetroundjoin%
\pgfsetlinewidth{1.003750pt}%
\definecolor{currentstroke}{rgb}{1.000000,1.000000,1.000000}%
\pgfsetstrokecolor{currentstroke}%
\pgfsetdash{}{0pt}%
\pgfpathmoveto{\pgfqpoint{0.624479in}{1.058752in}}%
\pgfpathlineto{\pgfqpoint{2.262793in}{1.058752in}}%
\pgfusepath{stroke}%
\end{pgfscope}%
\begin{pgfscope}%
\pgfsetbuttcap%
\pgfsetroundjoin%
\definecolor{currentfill}{rgb}{0.150000,0.150000,0.150000}%
\pgfsetfillcolor{currentfill}%
\pgfsetlinewidth{1.003750pt}%
\definecolor{currentstroke}{rgb}{0.150000,0.150000,0.150000}%
\pgfsetstrokecolor{currentstroke}%
\pgfsetdash{}{0pt}%
\pgfsys@defobject{currentmarker}{\pgfqpoint{0.000000in}{0.000000in}}{\pgfqpoint{0.000000in}{0.000000in}}{%
\pgfpathmoveto{\pgfqpoint{0.000000in}{0.000000in}}%
\pgfpathlineto{\pgfqpoint{0.000000in}{0.000000in}}%
\pgfusepath{stroke,fill}%
}%
\begin{pgfscope}%
\pgfsys@transformshift{0.624479in}{1.058752in}%
\pgfsys@useobject{currentmarker}{}%
\end{pgfscope}%
\end{pgfscope}%
\begin{pgfscope}%
\definecolor{textcolor}{rgb}{0.150000,0.150000,0.150000}%
\pgfsetstrokecolor{textcolor}%
\pgfsetfillcolor{textcolor}%
\pgftext[x=0.527257in,y=1.058752in,right,]{\color{textcolor}\sffamily\fontsize{10.000000}{12.000000}\selectfont 3.5}%
\end{pgfscope}%
\begin{pgfscope}%
\pgfpathrectangle{\pgfqpoint{0.624479in}{0.567253in}}{\pgfqpoint{1.638314in}{1.720247in}} %
\pgfusepath{clip}%
\pgfsetroundcap%
\pgfsetroundjoin%
\pgfsetlinewidth{1.003750pt}%
\definecolor{currentstroke}{rgb}{1.000000,1.000000,1.000000}%
\pgfsetstrokecolor{currentstroke}%
\pgfsetdash{}{0pt}%
\pgfpathmoveto{\pgfqpoint{0.624479in}{1.304501in}}%
\pgfpathlineto{\pgfqpoint{2.262793in}{1.304501in}}%
\pgfusepath{stroke}%
\end{pgfscope}%
\begin{pgfscope}%
\pgfsetbuttcap%
\pgfsetroundjoin%
\definecolor{currentfill}{rgb}{0.150000,0.150000,0.150000}%
\pgfsetfillcolor{currentfill}%
\pgfsetlinewidth{1.003750pt}%
\definecolor{currentstroke}{rgb}{0.150000,0.150000,0.150000}%
\pgfsetstrokecolor{currentstroke}%
\pgfsetdash{}{0pt}%
\pgfsys@defobject{currentmarker}{\pgfqpoint{0.000000in}{0.000000in}}{\pgfqpoint{0.000000in}{0.000000in}}{%
\pgfpathmoveto{\pgfqpoint{0.000000in}{0.000000in}}%
\pgfpathlineto{\pgfqpoint{0.000000in}{0.000000in}}%
\pgfusepath{stroke,fill}%
}%
\begin{pgfscope}%
\pgfsys@transformshift{0.624479in}{1.304501in}%
\pgfsys@useobject{currentmarker}{}%
\end{pgfscope}%
\end{pgfscope}%
\begin{pgfscope}%
\definecolor{textcolor}{rgb}{0.150000,0.150000,0.150000}%
\pgfsetstrokecolor{textcolor}%
\pgfsetfillcolor{textcolor}%
\pgftext[x=0.527257in,y=1.304501in,right,]{\color{textcolor}\sffamily\fontsize{10.000000}{12.000000}\selectfont 4.0}%
\end{pgfscope}%
\begin{pgfscope}%
\pgfpathrectangle{\pgfqpoint{0.624479in}{0.567253in}}{\pgfqpoint{1.638314in}{1.720247in}} %
\pgfusepath{clip}%
\pgfsetroundcap%
\pgfsetroundjoin%
\pgfsetlinewidth{1.003750pt}%
\definecolor{currentstroke}{rgb}{1.000000,1.000000,1.000000}%
\pgfsetstrokecolor{currentstroke}%
\pgfsetdash{}{0pt}%
\pgfpathmoveto{\pgfqpoint{0.624479in}{1.550251in}}%
\pgfpathlineto{\pgfqpoint{2.262793in}{1.550251in}}%
\pgfusepath{stroke}%
\end{pgfscope}%
\begin{pgfscope}%
\pgfsetbuttcap%
\pgfsetroundjoin%
\definecolor{currentfill}{rgb}{0.150000,0.150000,0.150000}%
\pgfsetfillcolor{currentfill}%
\pgfsetlinewidth{1.003750pt}%
\definecolor{currentstroke}{rgb}{0.150000,0.150000,0.150000}%
\pgfsetstrokecolor{currentstroke}%
\pgfsetdash{}{0pt}%
\pgfsys@defobject{currentmarker}{\pgfqpoint{0.000000in}{0.000000in}}{\pgfqpoint{0.000000in}{0.000000in}}{%
\pgfpathmoveto{\pgfqpoint{0.000000in}{0.000000in}}%
\pgfpathlineto{\pgfqpoint{0.000000in}{0.000000in}}%
\pgfusepath{stroke,fill}%
}%
\begin{pgfscope}%
\pgfsys@transformshift{0.624479in}{1.550251in}%
\pgfsys@useobject{currentmarker}{}%
\end{pgfscope}%
\end{pgfscope}%
\begin{pgfscope}%
\definecolor{textcolor}{rgb}{0.150000,0.150000,0.150000}%
\pgfsetstrokecolor{textcolor}%
\pgfsetfillcolor{textcolor}%
\pgftext[x=0.527257in,y=1.550251in,right,]{\color{textcolor}\sffamily\fontsize{10.000000}{12.000000}\selectfont 4.5}%
\end{pgfscope}%
\begin{pgfscope}%
\pgfpathrectangle{\pgfqpoint{0.624479in}{0.567253in}}{\pgfqpoint{1.638314in}{1.720247in}} %
\pgfusepath{clip}%
\pgfsetroundcap%
\pgfsetroundjoin%
\pgfsetlinewidth{1.003750pt}%
\definecolor{currentstroke}{rgb}{1.000000,1.000000,1.000000}%
\pgfsetstrokecolor{currentstroke}%
\pgfsetdash{}{0pt}%
\pgfpathmoveto{\pgfqpoint{0.624479in}{1.796001in}}%
\pgfpathlineto{\pgfqpoint{2.262793in}{1.796001in}}%
\pgfusepath{stroke}%
\end{pgfscope}%
\begin{pgfscope}%
\pgfsetbuttcap%
\pgfsetroundjoin%
\definecolor{currentfill}{rgb}{0.150000,0.150000,0.150000}%
\pgfsetfillcolor{currentfill}%
\pgfsetlinewidth{1.003750pt}%
\definecolor{currentstroke}{rgb}{0.150000,0.150000,0.150000}%
\pgfsetstrokecolor{currentstroke}%
\pgfsetdash{}{0pt}%
\pgfsys@defobject{currentmarker}{\pgfqpoint{0.000000in}{0.000000in}}{\pgfqpoint{0.000000in}{0.000000in}}{%
\pgfpathmoveto{\pgfqpoint{0.000000in}{0.000000in}}%
\pgfpathlineto{\pgfqpoint{0.000000in}{0.000000in}}%
\pgfusepath{stroke,fill}%
}%
\begin{pgfscope}%
\pgfsys@transformshift{0.624479in}{1.796001in}%
\pgfsys@useobject{currentmarker}{}%
\end{pgfscope}%
\end{pgfscope}%
\begin{pgfscope}%
\definecolor{textcolor}{rgb}{0.150000,0.150000,0.150000}%
\pgfsetstrokecolor{textcolor}%
\pgfsetfillcolor{textcolor}%
\pgftext[x=0.527257in,y=1.796001in,right,]{\color{textcolor}\sffamily\fontsize{10.000000}{12.000000}\selectfont 5.0}%
\end{pgfscope}%
\begin{pgfscope}%
\pgfpathrectangle{\pgfqpoint{0.624479in}{0.567253in}}{\pgfqpoint{1.638314in}{1.720247in}} %
\pgfusepath{clip}%
\pgfsetroundcap%
\pgfsetroundjoin%
\pgfsetlinewidth{1.003750pt}%
\definecolor{currentstroke}{rgb}{1.000000,1.000000,1.000000}%
\pgfsetstrokecolor{currentstroke}%
\pgfsetdash{}{0pt}%
\pgfpathmoveto{\pgfqpoint{0.624479in}{2.041750in}}%
\pgfpathlineto{\pgfqpoint{2.262793in}{2.041750in}}%
\pgfusepath{stroke}%
\end{pgfscope}%
\begin{pgfscope}%
\pgfsetbuttcap%
\pgfsetroundjoin%
\definecolor{currentfill}{rgb}{0.150000,0.150000,0.150000}%
\pgfsetfillcolor{currentfill}%
\pgfsetlinewidth{1.003750pt}%
\definecolor{currentstroke}{rgb}{0.150000,0.150000,0.150000}%
\pgfsetstrokecolor{currentstroke}%
\pgfsetdash{}{0pt}%
\pgfsys@defobject{currentmarker}{\pgfqpoint{0.000000in}{0.000000in}}{\pgfqpoint{0.000000in}{0.000000in}}{%
\pgfpathmoveto{\pgfqpoint{0.000000in}{0.000000in}}%
\pgfpathlineto{\pgfqpoint{0.000000in}{0.000000in}}%
\pgfusepath{stroke,fill}%
}%
\begin{pgfscope}%
\pgfsys@transformshift{0.624479in}{2.041750in}%
\pgfsys@useobject{currentmarker}{}%
\end{pgfscope}%
\end{pgfscope}%
\begin{pgfscope}%
\definecolor{textcolor}{rgb}{0.150000,0.150000,0.150000}%
\pgfsetstrokecolor{textcolor}%
\pgfsetfillcolor{textcolor}%
\pgftext[x=0.527257in,y=2.041750in,right,]{\color{textcolor}\sffamily\fontsize{10.000000}{12.000000}\selectfont 5.5}%
\end{pgfscope}%
\begin{pgfscope}%
\pgfpathrectangle{\pgfqpoint{0.624479in}{0.567253in}}{\pgfqpoint{1.638314in}{1.720247in}} %
\pgfusepath{clip}%
\pgfsetroundcap%
\pgfsetroundjoin%
\pgfsetlinewidth{1.003750pt}%
\definecolor{currentstroke}{rgb}{1.000000,1.000000,1.000000}%
\pgfsetstrokecolor{currentstroke}%
\pgfsetdash{}{0pt}%
\pgfpathmoveto{\pgfqpoint{0.624479in}{2.287500in}}%
\pgfpathlineto{\pgfqpoint{2.262793in}{2.287500in}}%
\pgfusepath{stroke}%
\end{pgfscope}%
\begin{pgfscope}%
\pgfsetbuttcap%
\pgfsetroundjoin%
\definecolor{currentfill}{rgb}{0.150000,0.150000,0.150000}%
\pgfsetfillcolor{currentfill}%
\pgfsetlinewidth{1.003750pt}%
\definecolor{currentstroke}{rgb}{0.150000,0.150000,0.150000}%
\pgfsetstrokecolor{currentstroke}%
\pgfsetdash{}{0pt}%
\pgfsys@defobject{currentmarker}{\pgfqpoint{0.000000in}{0.000000in}}{\pgfqpoint{0.000000in}{0.000000in}}{%
\pgfpathmoveto{\pgfqpoint{0.000000in}{0.000000in}}%
\pgfpathlineto{\pgfqpoint{0.000000in}{0.000000in}}%
\pgfusepath{stroke,fill}%
}%
\begin{pgfscope}%
\pgfsys@transformshift{0.624479in}{2.287500in}%
\pgfsys@useobject{currentmarker}{}%
\end{pgfscope}%
\end{pgfscope}%
\begin{pgfscope}%
\definecolor{textcolor}{rgb}{0.150000,0.150000,0.150000}%
\pgfsetstrokecolor{textcolor}%
\pgfsetfillcolor{textcolor}%
\pgftext[x=0.527257in,y=2.287500in,right,]{\color{textcolor}\sffamily\fontsize{10.000000}{12.000000}\selectfont 6.0}%
\end{pgfscope}%
\begin{pgfscope}%
\definecolor{textcolor}{rgb}{0.150000,0.150000,0.150000}%
\pgfsetstrokecolor{textcolor}%
\pgfsetfillcolor{textcolor}%
\pgftext[x=0.264738in,y=1.427376in,,bottom,rotate=90.000000]{\color{textcolor}\sffamily\fontsize{11.000000}{13.200000}\selectfont Average of four falling times}%
\end{pgfscope}%
\begin{pgfscope}%
\pgfpathrectangle{\pgfqpoint{0.624479in}{0.567253in}}{\pgfqpoint{1.638314in}{1.720247in}} %
\pgfusepath{clip}%
\pgfsetbuttcap%
\pgfsetroundjoin%
\definecolor{currentfill}{rgb}{0.298039,0.447059,0.690196}%
\pgfsetfillcolor{currentfill}%
\pgfsetlinewidth{0.301125pt}%
\definecolor{currentstroke}{rgb}{1.000000,1.000000,1.000000}%
\pgfsetstrokecolor{currentstroke}%
\pgfsetdash{}{0pt}%
\pgfpathmoveto{\pgfqpoint{0.858524in}{1.515508in}}%
\pgfpathcurveto{\pgfqpoint{0.866760in}{1.515508in}}{\pgfqpoint{0.874660in}{1.518781in}}{\pgfqpoint{0.880484in}{1.524605in}}%
\pgfpathcurveto{\pgfqpoint{0.886308in}{1.530429in}}{\pgfqpoint{0.889580in}{1.538329in}}{\pgfqpoint{0.889580in}{1.546565in}}%
\pgfpathcurveto{\pgfqpoint{0.889580in}{1.554801in}}{\pgfqpoint{0.886308in}{1.562701in}}{\pgfqpoint{0.880484in}{1.568525in}}%
\pgfpathcurveto{\pgfqpoint{0.874660in}{1.574349in}}{\pgfqpoint{0.866760in}{1.577621in}}{\pgfqpoint{0.858524in}{1.577621in}}%
\pgfpathcurveto{\pgfqpoint{0.850288in}{1.577621in}}{\pgfqpoint{0.842388in}{1.574349in}}{\pgfqpoint{0.836564in}{1.568525in}}%
\pgfpathcurveto{\pgfqpoint{0.830740in}{1.562701in}}{\pgfqpoint{0.827467in}{1.554801in}}{\pgfqpoint{0.827467in}{1.546565in}}%
\pgfpathcurveto{\pgfqpoint{0.827467in}{1.538329in}}{\pgfqpoint{0.830740in}{1.530429in}}{\pgfqpoint{0.836564in}{1.524605in}}%
\pgfpathcurveto{\pgfqpoint{0.842388in}{1.518781in}}{\pgfqpoint{0.850288in}{1.515508in}}{\pgfqpoint{0.858524in}{1.515508in}}%
\pgfpathlineto{\pgfqpoint{0.858524in}{1.515508in}}%
\pgfusepath{stroke,fill}%
\end{pgfscope}%
\begin{pgfscope}%
\pgfpathrectangle{\pgfqpoint{0.624479in}{0.567253in}}{\pgfqpoint{1.638314in}{1.720247in}} %
\pgfusepath{clip}%
\pgfsetbuttcap%
\pgfsetroundjoin%
\definecolor{currentfill}{rgb}{0.298039,0.447059,0.690196}%
\pgfsetfillcolor{currentfill}%
\pgfsetlinewidth{0.301125pt}%
\definecolor{currentstroke}{rgb}{1.000000,1.000000,1.000000}%
\pgfsetstrokecolor{currentstroke}%
\pgfsetdash{}{0pt}%
\pgfpathmoveto{\pgfqpoint{1.068564in}{1.199720in}}%
\pgfpathcurveto{\pgfqpoint{1.076801in}{1.199720in}}{\pgfqpoint{1.084701in}{1.202992in}}{\pgfqpoint{1.090524in}{1.208816in}}%
\pgfpathcurveto{\pgfqpoint{1.096348in}{1.214640in}}{\pgfqpoint{1.099621in}{1.222540in}}{\pgfqpoint{1.099621in}{1.230777in}}%
\pgfpathcurveto{\pgfqpoint{1.099621in}{1.239013in}}{\pgfqpoint{1.096348in}{1.246913in}}{\pgfqpoint{1.090524in}{1.252737in}}%
\pgfpathcurveto{\pgfqpoint{1.084701in}{1.258561in}}{\pgfqpoint{1.076801in}{1.261833in}}{\pgfqpoint{1.068564in}{1.261833in}}%
\pgfpathcurveto{\pgfqpoint{1.060328in}{1.261833in}}{\pgfqpoint{1.052428in}{1.258561in}}{\pgfqpoint{1.046604in}{1.252737in}}%
\pgfpathcurveto{\pgfqpoint{1.040780in}{1.246913in}}{\pgfqpoint{1.037508in}{1.239013in}}{\pgfqpoint{1.037508in}{1.230777in}}%
\pgfpathcurveto{\pgfqpoint{1.037508in}{1.222540in}}{\pgfqpoint{1.040780in}{1.214640in}}{\pgfqpoint{1.046604in}{1.208816in}}%
\pgfpathcurveto{\pgfqpoint{1.052428in}{1.202992in}}{\pgfqpoint{1.060328in}{1.199720in}}{\pgfqpoint{1.068564in}{1.199720in}}%
\pgfpathlineto{\pgfqpoint{1.068564in}{1.199720in}}%
\pgfusepath{stroke,fill}%
\end{pgfscope}%
\begin{pgfscope}%
\pgfpathrectangle{\pgfqpoint{0.624479in}{0.567253in}}{\pgfqpoint{1.638314in}{1.720247in}} %
\pgfusepath{clip}%
\pgfsetbuttcap%
\pgfsetroundjoin%
\definecolor{currentfill}{rgb}{0.298039,0.447059,0.690196}%
\pgfsetfillcolor{currentfill}%
\pgfsetlinewidth{0.301125pt}%
\definecolor{currentstroke}{rgb}{1.000000,1.000000,1.000000}%
\pgfsetstrokecolor{currentstroke}%
\pgfsetdash{}{0pt}%
\pgfpathmoveto{\pgfqpoint{1.848714in}{1.166544in}}%
\pgfpathcurveto{\pgfqpoint{1.856950in}{1.166544in}}{\pgfqpoint{1.864850in}{1.169816in}}{\pgfqpoint{1.870674in}{1.175640in}}%
\pgfpathcurveto{\pgfqpoint{1.876498in}{1.181464in}}{\pgfqpoint{1.879770in}{1.189364in}}{\pgfqpoint{1.879770in}{1.197600in}}%
\pgfpathcurveto{\pgfqpoint{1.879770in}{1.205837in}}{\pgfqpoint{1.876498in}{1.213737in}}{\pgfqpoint{1.870674in}{1.219561in}}%
\pgfpathcurveto{\pgfqpoint{1.864850in}{1.225385in}}{\pgfqpoint{1.856950in}{1.228657in}}{\pgfqpoint{1.848714in}{1.228657in}}%
\pgfpathcurveto{\pgfqpoint{1.840477in}{1.228657in}}{\pgfqpoint{1.832577in}{1.225385in}}{\pgfqpoint{1.826753in}{1.219561in}}%
\pgfpathcurveto{\pgfqpoint{1.820929in}{1.213737in}}{\pgfqpoint{1.817657in}{1.205837in}}{\pgfqpoint{1.817657in}{1.197600in}}%
\pgfpathcurveto{\pgfqpoint{1.817657in}{1.189364in}}{\pgfqpoint{1.820929in}{1.181464in}}{\pgfqpoint{1.826753in}{1.175640in}}%
\pgfpathcurveto{\pgfqpoint{1.832577in}{1.169816in}}{\pgfqpoint{1.840477in}{1.166544in}}{\pgfqpoint{1.848714in}{1.166544in}}%
\pgfpathlineto{\pgfqpoint{1.848714in}{1.166544in}}%
\pgfusepath{stroke,fill}%
\end{pgfscope}%
\begin{pgfscope}%
\pgfpathrectangle{\pgfqpoint{0.624479in}{0.567253in}}{\pgfqpoint{1.638314in}{1.720247in}} %
\pgfusepath{clip}%
\pgfsetbuttcap%
\pgfsetroundjoin%
\definecolor{currentfill}{rgb}{0.298039,0.447059,0.690196}%
\pgfsetfillcolor{currentfill}%
\pgfsetlinewidth{0.301125pt}%
\definecolor{currentstroke}{rgb}{1.000000,1.000000,1.000000}%
\pgfsetstrokecolor{currentstroke}%
\pgfsetdash{}{0pt}%
\pgfpathmoveto{\pgfqpoint{2.028748in}{0.930624in}}%
\pgfpathcurveto{\pgfqpoint{2.036984in}{0.930624in}}{\pgfqpoint{2.044884in}{0.933897in}}{\pgfqpoint{2.050708in}{0.939720in}}%
\pgfpathcurveto{\pgfqpoint{2.056532in}{0.945544in}}{\pgfqpoint{2.059805in}{0.953444in}}{\pgfqpoint{2.059805in}{0.961681in}}%
\pgfpathcurveto{\pgfqpoint{2.059805in}{0.969917in}}{\pgfqpoint{2.056532in}{0.977817in}}{\pgfqpoint{2.050708in}{0.983641in}}%
\pgfpathcurveto{\pgfqpoint{2.044884in}{0.989465in}}{\pgfqpoint{2.036984in}{0.992737in}}{\pgfqpoint{2.028748in}{0.992737in}}%
\pgfpathcurveto{\pgfqpoint{2.020512in}{0.992737in}}{\pgfqpoint{2.012612in}{0.989465in}}{\pgfqpoint{2.006788in}{0.983641in}}%
\pgfpathcurveto{\pgfqpoint{2.000964in}{0.977817in}}{\pgfqpoint{1.997692in}{0.969917in}}{\pgfqpoint{1.997692in}{0.961681in}}%
\pgfpathcurveto{\pgfqpoint{1.997692in}{0.953444in}}{\pgfqpoint{2.000964in}{0.945544in}}{\pgfqpoint{2.006788in}{0.939720in}}%
\pgfpathcurveto{\pgfqpoint{2.012612in}{0.933897in}}{\pgfqpoint{2.020512in}{0.930624in}}{\pgfqpoint{2.028748in}{0.930624in}}%
\pgfpathlineto{\pgfqpoint{2.028748in}{0.930624in}}%
\pgfusepath{stroke,fill}%
\end{pgfscope}%
\begin{pgfscope}%
\pgfpathrectangle{\pgfqpoint{0.624479in}{0.567253in}}{\pgfqpoint{1.638314in}{1.720247in}} %
\pgfusepath{clip}%
\pgfsetbuttcap%
\pgfsetroundjoin%
\definecolor{currentfill}{rgb}{0.298039,0.447059,0.690196}%
\pgfsetfillcolor{currentfill}%
\pgfsetlinewidth{0.301125pt}%
\definecolor{currentstroke}{rgb}{1.000000,1.000000,1.000000}%
\pgfsetstrokecolor{currentstroke}%
\pgfsetdash{}{0pt}%
\pgfpathmoveto{\pgfqpoint{1.668679in}{0.736482in}}%
\pgfpathcurveto{\pgfqpoint{1.676915in}{0.736482in}}{\pgfqpoint{1.684816in}{0.739754in}}{\pgfqpoint{1.690639in}{0.745578in}}%
\pgfpathcurveto{\pgfqpoint{1.696463in}{0.751402in}}{\pgfqpoint{1.699736in}{0.759302in}}{\pgfqpoint{1.699736in}{0.767539in}}%
\pgfpathcurveto{\pgfqpoint{1.699736in}{0.775775in}}{\pgfqpoint{1.696463in}{0.783675in}}{\pgfqpoint{1.690639in}{0.789499in}}%
\pgfpathcurveto{\pgfqpoint{1.684816in}{0.795323in}}{\pgfqpoint{1.676915in}{0.798595in}}{\pgfqpoint{1.668679in}{0.798595in}}%
\pgfpathcurveto{\pgfqpoint{1.660443in}{0.798595in}}{\pgfqpoint{1.652543in}{0.795323in}}{\pgfqpoint{1.646719in}{0.789499in}}%
\pgfpathcurveto{\pgfqpoint{1.640895in}{0.783675in}}{\pgfqpoint{1.637623in}{0.775775in}}{\pgfqpoint{1.637623in}{0.767539in}}%
\pgfpathcurveto{\pgfqpoint{1.637623in}{0.759302in}}{\pgfqpoint{1.640895in}{0.751402in}}{\pgfqpoint{1.646719in}{0.745578in}}%
\pgfpathcurveto{\pgfqpoint{1.652543in}{0.739754in}}{\pgfqpoint{1.660443in}{0.736482in}}{\pgfqpoint{1.668679in}{0.736482in}}%
\pgfpathlineto{\pgfqpoint{1.668679in}{0.736482in}}%
\pgfusepath{stroke,fill}%
\end{pgfscope}%
\begin{pgfscope}%
\pgfpathrectangle{\pgfqpoint{0.624479in}{0.567253in}}{\pgfqpoint{1.638314in}{1.720247in}} %
\pgfusepath{clip}%
\pgfsetbuttcap%
\pgfsetroundjoin%
\definecolor{currentfill}{rgb}{0.298039,0.447059,0.690196}%
\pgfsetfillcolor{currentfill}%
\pgfsetlinewidth{0.301125pt}%
\definecolor{currentstroke}{rgb}{1.000000,1.000000,1.000000}%
\pgfsetstrokecolor{currentstroke}%
\pgfsetdash{}{0pt}%
\pgfpathmoveto{\pgfqpoint{1.818708in}{0.764743in}}%
\pgfpathcurveto{\pgfqpoint{1.826944in}{0.764743in}}{\pgfqpoint{1.834844in}{0.768016in}}{\pgfqpoint{1.840668in}{0.773839in}}%
\pgfpathcurveto{\pgfqpoint{1.846492in}{0.779663in}}{\pgfqpoint{1.849764in}{0.787563in}}{\pgfqpoint{1.849764in}{0.795800in}}%
\pgfpathcurveto{\pgfqpoint{1.849764in}{0.804036in}}{\pgfqpoint{1.846492in}{0.811936in}}{\pgfqpoint{1.840668in}{0.817760in}}%
\pgfpathcurveto{\pgfqpoint{1.834844in}{0.823584in}}{\pgfqpoint{1.826944in}{0.826856in}}{\pgfqpoint{1.818708in}{0.826856in}}%
\pgfpathcurveto{\pgfqpoint{1.810472in}{0.826856in}}{\pgfqpoint{1.802572in}{0.823584in}}{\pgfqpoint{1.796748in}{0.817760in}}%
\pgfpathcurveto{\pgfqpoint{1.790924in}{0.811936in}}{\pgfqpoint{1.787651in}{0.804036in}}{\pgfqpoint{1.787651in}{0.795800in}}%
\pgfpathcurveto{\pgfqpoint{1.787651in}{0.787563in}}{\pgfqpoint{1.790924in}{0.779663in}}{\pgfqpoint{1.796748in}{0.773839in}}%
\pgfpathcurveto{\pgfqpoint{1.802572in}{0.768016in}}{\pgfqpoint{1.810472in}{0.764743in}}{\pgfqpoint{1.818708in}{0.764743in}}%
\pgfpathlineto{\pgfqpoint{1.818708in}{0.764743in}}%
\pgfusepath{stroke,fill}%
\end{pgfscope}%
\begin{pgfscope}%
\pgfpathrectangle{\pgfqpoint{0.624479in}{0.567253in}}{\pgfqpoint{1.638314in}{1.720247in}} %
\pgfusepath{clip}%
\pgfsetbuttcap%
\pgfsetroundjoin%
\definecolor{currentfill}{rgb}{0.298039,0.447059,0.690196}%
\pgfsetfillcolor{currentfill}%
\pgfsetlinewidth{0.301125pt}%
\definecolor{currentstroke}{rgb}{1.000000,1.000000,1.000000}%
\pgfsetstrokecolor{currentstroke}%
\pgfsetdash{}{0pt}%
\pgfpathmoveto{\pgfqpoint{1.098570in}{1.087904in}}%
\pgfpathcurveto{\pgfqpoint{1.106806in}{1.087904in}}{\pgfqpoint{1.114706in}{1.091176in}}{\pgfqpoint{1.120530in}{1.097000in}}%
\pgfpathcurveto{\pgfqpoint{1.126354in}{1.102824in}}{\pgfqpoint{1.129626in}{1.110724in}}{\pgfqpoint{1.129626in}{1.118961in}}%
\pgfpathcurveto{\pgfqpoint{1.129626in}{1.127197in}}{\pgfqpoint{1.126354in}{1.135097in}}{\pgfqpoint{1.120530in}{1.140921in}}%
\pgfpathcurveto{\pgfqpoint{1.114706in}{1.146745in}}{\pgfqpoint{1.106806in}{1.150017in}}{\pgfqpoint{1.098570in}{1.150017in}}%
\pgfpathcurveto{\pgfqpoint{1.090334in}{1.150017in}}{\pgfqpoint{1.082434in}{1.146745in}}{\pgfqpoint{1.076610in}{1.140921in}}%
\pgfpathcurveto{\pgfqpoint{1.070786in}{1.135097in}}{\pgfqpoint{1.067513in}{1.127197in}}{\pgfqpoint{1.067513in}{1.118961in}}%
\pgfpathcurveto{\pgfqpoint{1.067513in}{1.110724in}}{\pgfqpoint{1.070786in}{1.102824in}}{\pgfqpoint{1.076610in}{1.097000in}}%
\pgfpathcurveto{\pgfqpoint{1.082434in}{1.091176in}}{\pgfqpoint{1.090334in}{1.087904in}}{\pgfqpoint{1.098570in}{1.087904in}}%
\pgfpathlineto{\pgfqpoint{1.098570in}{1.087904in}}%
\pgfusepath{stroke,fill}%
\end{pgfscope}%
\begin{pgfscope}%
\pgfpathrectangle{\pgfqpoint{0.624479in}{0.567253in}}{\pgfqpoint{1.638314in}{1.720247in}} %
\pgfusepath{clip}%
\pgfsetbuttcap%
\pgfsetroundjoin%
\definecolor{currentfill}{rgb}{0.298039,0.447059,0.690196}%
\pgfsetfillcolor{currentfill}%
\pgfsetlinewidth{0.301125pt}%
\definecolor{currentstroke}{rgb}{1.000000,1.000000,1.000000}%
\pgfsetstrokecolor{currentstroke}%
\pgfsetdash{}{0pt}%
\pgfpathmoveto{\pgfqpoint{1.458639in}{1.490933in}}%
\pgfpathcurveto{\pgfqpoint{1.466875in}{1.490933in}}{\pgfqpoint{1.474775in}{1.494206in}}{\pgfqpoint{1.480599in}{1.500030in}}%
\pgfpathcurveto{\pgfqpoint{1.486423in}{1.505854in}}{\pgfqpoint{1.489695in}{1.513754in}}{\pgfqpoint{1.489695in}{1.521990in}}%
\pgfpathcurveto{\pgfqpoint{1.489695in}{1.530226in}}{\pgfqpoint{1.486423in}{1.538126in}}{\pgfqpoint{1.480599in}{1.543950in}}%
\pgfpathcurveto{\pgfqpoint{1.474775in}{1.549774in}}{\pgfqpoint{1.466875in}{1.553046in}}{\pgfqpoint{1.458639in}{1.553046in}}%
\pgfpathcurveto{\pgfqpoint{1.450403in}{1.553046in}}{\pgfqpoint{1.442503in}{1.549774in}}{\pgfqpoint{1.436679in}{1.543950in}}%
\pgfpathcurveto{\pgfqpoint{1.430855in}{1.538126in}}{\pgfqpoint{1.427582in}{1.530226in}}{\pgfqpoint{1.427582in}{1.521990in}}%
\pgfpathcurveto{\pgfqpoint{1.427582in}{1.513754in}}{\pgfqpoint{1.430855in}{1.505854in}}{\pgfqpoint{1.436679in}{1.500030in}}%
\pgfpathcurveto{\pgfqpoint{1.442503in}{1.494206in}}{\pgfqpoint{1.450403in}{1.490933in}}{\pgfqpoint{1.458639in}{1.490933in}}%
\pgfpathlineto{\pgfqpoint{1.458639in}{1.490933in}}%
\pgfusepath{stroke,fill}%
\end{pgfscope}%
\begin{pgfscope}%
\pgfpathrectangle{\pgfqpoint{0.624479in}{0.567253in}}{\pgfqpoint{1.638314in}{1.720247in}} %
\pgfusepath{clip}%
\pgfsetbuttcap%
\pgfsetroundjoin%
\definecolor{currentfill}{rgb}{0.298039,0.447059,0.690196}%
\pgfsetfillcolor{currentfill}%
\pgfsetlinewidth{0.301125pt}%
\definecolor{currentstroke}{rgb}{1.000000,1.000000,1.000000}%
\pgfsetstrokecolor{currentstroke}%
\pgfsetdash{}{0pt}%
\pgfpathmoveto{\pgfqpoint{1.398627in}{1.639612in}}%
\pgfpathcurveto{\pgfqpoint{1.406864in}{1.639612in}}{\pgfqpoint{1.414764in}{1.642884in}}{\pgfqpoint{1.420588in}{1.648708in}}%
\pgfpathcurveto{\pgfqpoint{1.426412in}{1.654532in}}{\pgfqpoint{1.429684in}{1.662432in}}{\pgfqpoint{1.429684in}{1.670668in}}%
\pgfpathcurveto{\pgfqpoint{1.429684in}{1.678905in}}{\pgfqpoint{1.426412in}{1.686805in}}{\pgfqpoint{1.420588in}{1.692629in}}%
\pgfpathcurveto{\pgfqpoint{1.414764in}{1.698453in}}{\pgfqpoint{1.406864in}{1.701725in}}{\pgfqpoint{1.398627in}{1.701725in}}%
\pgfpathcurveto{\pgfqpoint{1.390391in}{1.701725in}}{\pgfqpoint{1.382491in}{1.698453in}}{\pgfqpoint{1.376667in}{1.692629in}}%
\pgfpathcurveto{\pgfqpoint{1.370843in}{1.686805in}}{\pgfqpoint{1.367571in}{1.678905in}}{\pgfqpoint{1.367571in}{1.670668in}}%
\pgfpathcurveto{\pgfqpoint{1.367571in}{1.662432in}}{\pgfqpoint{1.370843in}{1.654532in}}{\pgfqpoint{1.376667in}{1.648708in}}%
\pgfpathcurveto{\pgfqpoint{1.382491in}{1.642884in}}{\pgfqpoint{1.390391in}{1.639612in}}{\pgfqpoint{1.398627in}{1.639612in}}%
\pgfpathlineto{\pgfqpoint{1.398627in}{1.639612in}}%
\pgfusepath{stroke,fill}%
\end{pgfscope}%
\begin{pgfscope}%
\pgfpathrectangle{\pgfqpoint{0.624479in}{0.567253in}}{\pgfqpoint{1.638314in}{1.720247in}} %
\pgfusepath{clip}%
\pgfsetbuttcap%
\pgfsetroundjoin%
\definecolor{currentfill}{rgb}{0.298039,0.447059,0.690196}%
\pgfsetfillcolor{currentfill}%
\pgfsetlinewidth{0.301125pt}%
\definecolor{currentstroke}{rgb}{1.000000,1.000000,1.000000}%
\pgfsetstrokecolor{currentstroke}%
\pgfsetdash{}{0pt}%
\pgfpathmoveto{\pgfqpoint{1.128576in}{0.985918in}}%
\pgfpathcurveto{\pgfqpoint{1.136812in}{0.985918in}}{\pgfqpoint{1.144712in}{0.989190in}}{\pgfqpoint{1.150536in}{0.995014in}}%
\pgfpathcurveto{\pgfqpoint{1.156360in}{1.000838in}}{\pgfqpoint{1.159632in}{1.008738in}}{\pgfqpoint{1.159632in}{1.016974in}}%
\pgfpathcurveto{\pgfqpoint{1.159632in}{1.025211in}}{\pgfqpoint{1.156360in}{1.033111in}}{\pgfqpoint{1.150536in}{1.038935in}}%
\pgfpathcurveto{\pgfqpoint{1.144712in}{1.044759in}}{\pgfqpoint{1.136812in}{1.048031in}}{\pgfqpoint{1.128576in}{1.048031in}}%
\pgfpathcurveto{\pgfqpoint{1.120339in}{1.048031in}}{\pgfqpoint{1.112439in}{1.044759in}}{\pgfqpoint{1.106615in}{1.038935in}}%
\pgfpathcurveto{\pgfqpoint{1.100792in}{1.033111in}}{\pgfqpoint{1.097519in}{1.025211in}}{\pgfqpoint{1.097519in}{1.016974in}}%
\pgfpathcurveto{\pgfqpoint{1.097519in}{1.008738in}}{\pgfqpoint{1.100792in}{1.000838in}}{\pgfqpoint{1.106615in}{0.995014in}}%
\pgfpathcurveto{\pgfqpoint{1.112439in}{0.989190in}}{\pgfqpoint{1.120339in}{0.985918in}}{\pgfqpoint{1.128576in}{0.985918in}}%
\pgfpathlineto{\pgfqpoint{1.128576in}{0.985918in}}%
\pgfusepath{stroke,fill}%
\end{pgfscope}%
\begin{pgfscope}%
\pgfpathrectangle{\pgfqpoint{0.624479in}{0.567253in}}{\pgfqpoint{1.638314in}{1.720247in}} %
\pgfusepath{clip}%
\pgfsetbuttcap%
\pgfsetroundjoin%
\definecolor{currentfill}{rgb}{0.298039,0.447059,0.690196}%
\pgfsetfillcolor{currentfill}%
\pgfsetlinewidth{0.301125pt}%
\definecolor{currentstroke}{rgb}{1.000000,1.000000,1.000000}%
\pgfsetstrokecolor{currentstroke}%
\pgfsetdash{}{0pt}%
\pgfpathmoveto{\pgfqpoint{1.518650in}{1.155485in}}%
\pgfpathcurveto{\pgfqpoint{1.526887in}{1.155485in}}{\pgfqpoint{1.534787in}{1.158757in}}{\pgfqpoint{1.540611in}{1.164581in}}%
\pgfpathcurveto{\pgfqpoint{1.546435in}{1.170405in}}{\pgfqpoint{1.549707in}{1.178305in}}{\pgfqpoint{1.549707in}{1.186542in}}%
\pgfpathcurveto{\pgfqpoint{1.549707in}{1.194778in}}{\pgfqpoint{1.546435in}{1.202678in}}{\pgfqpoint{1.540611in}{1.208502in}}%
\pgfpathcurveto{\pgfqpoint{1.534787in}{1.214326in}}{\pgfqpoint{1.526887in}{1.217598in}}{\pgfqpoint{1.518650in}{1.217598in}}%
\pgfpathcurveto{\pgfqpoint{1.510414in}{1.217598in}}{\pgfqpoint{1.502514in}{1.214326in}}{\pgfqpoint{1.496690in}{1.208502in}}%
\pgfpathcurveto{\pgfqpoint{1.490866in}{1.202678in}}{\pgfqpoint{1.487594in}{1.194778in}}{\pgfqpoint{1.487594in}{1.186542in}}%
\pgfpathcurveto{\pgfqpoint{1.487594in}{1.178305in}}{\pgfqpoint{1.490866in}{1.170405in}}{\pgfqpoint{1.496690in}{1.164581in}}%
\pgfpathcurveto{\pgfqpoint{1.502514in}{1.158757in}}{\pgfqpoint{1.510414in}{1.155485in}}{\pgfqpoint{1.518650in}{1.155485in}}%
\pgfpathlineto{\pgfqpoint{1.518650in}{1.155485in}}%
\pgfusepath{stroke,fill}%
\end{pgfscope}%
\begin{pgfscope}%
\pgfpathrectangle{\pgfqpoint{0.624479in}{0.567253in}}{\pgfqpoint{1.638314in}{1.720247in}} %
\pgfusepath{clip}%
\pgfsetbuttcap%
\pgfsetroundjoin%
\definecolor{currentfill}{rgb}{0.298039,0.447059,0.690196}%
\pgfsetfillcolor{currentfill}%
\pgfsetlinewidth{0.301125pt}%
\definecolor{currentstroke}{rgb}{1.000000,1.000000,1.000000}%
\pgfsetstrokecolor{currentstroke}%
\pgfsetdash{}{0pt}%
\pgfpathmoveto{\pgfqpoint{1.698685in}{0.762286in}}%
\pgfpathcurveto{\pgfqpoint{1.706921in}{0.762286in}}{\pgfqpoint{1.714821in}{0.765558in}}{\pgfqpoint{1.720645in}{0.771382in}}%
\pgfpathcurveto{\pgfqpoint{1.726469in}{0.777206in}}{\pgfqpoint{1.729741in}{0.785106in}}{\pgfqpoint{1.729741in}{0.793342in}}%
\pgfpathcurveto{\pgfqpoint{1.729741in}{0.801579in}}{\pgfqpoint{1.726469in}{0.809479in}}{\pgfqpoint{1.720645in}{0.815303in}}%
\pgfpathcurveto{\pgfqpoint{1.714821in}{0.821126in}}{\pgfqpoint{1.706921in}{0.824399in}}{\pgfqpoint{1.698685in}{0.824399in}}%
\pgfpathcurveto{\pgfqpoint{1.690449in}{0.824399in}}{\pgfqpoint{1.682549in}{0.821126in}}{\pgfqpoint{1.676725in}{0.815303in}}%
\pgfpathcurveto{\pgfqpoint{1.670901in}{0.809479in}}{\pgfqpoint{1.667628in}{0.801579in}}{\pgfqpoint{1.667628in}{0.793342in}}%
\pgfpathcurveto{\pgfqpoint{1.667628in}{0.785106in}}{\pgfqpoint{1.670901in}{0.777206in}}{\pgfqpoint{1.676725in}{0.771382in}}%
\pgfpathcurveto{\pgfqpoint{1.682549in}{0.765558in}}{\pgfqpoint{1.690449in}{0.762286in}}{\pgfqpoint{1.698685in}{0.762286in}}%
\pgfpathlineto{\pgfqpoint{1.698685in}{0.762286in}}%
\pgfusepath{stroke,fill}%
\end{pgfscope}%
\begin{pgfscope}%
\pgfpathrectangle{\pgfqpoint{0.624479in}{0.567253in}}{\pgfqpoint{1.638314in}{1.720247in}} %
\pgfusepath{clip}%
\pgfsetbuttcap%
\pgfsetroundjoin%
\definecolor{currentfill}{rgb}{0.298039,0.447059,0.690196}%
\pgfsetfillcolor{currentfill}%
\pgfsetlinewidth{0.301125pt}%
\definecolor{currentstroke}{rgb}{1.000000,1.000000,1.000000}%
\pgfsetstrokecolor{currentstroke}%
\pgfsetdash{}{0pt}%
\pgfpathmoveto{\pgfqpoint{1.038558in}{1.629782in}}%
\pgfpathcurveto{\pgfqpoint{1.046795in}{1.629782in}}{\pgfqpoint{1.054695in}{1.633054in}}{\pgfqpoint{1.060519in}{1.638878in}}%
\pgfpathcurveto{\pgfqpoint{1.066343in}{1.644702in}}{\pgfqpoint{1.069615in}{1.652602in}}{\pgfqpoint{1.069615in}{1.660838in}}%
\pgfpathcurveto{\pgfqpoint{1.069615in}{1.669075in}}{\pgfqpoint{1.066343in}{1.676975in}}{\pgfqpoint{1.060519in}{1.682799in}}%
\pgfpathcurveto{\pgfqpoint{1.054695in}{1.688623in}}{\pgfqpoint{1.046795in}{1.691895in}}{\pgfqpoint{1.038558in}{1.691895in}}%
\pgfpathcurveto{\pgfqpoint{1.030322in}{1.691895in}}{\pgfqpoint{1.022422in}{1.688623in}}{\pgfqpoint{1.016598in}{1.682799in}}%
\pgfpathcurveto{\pgfqpoint{1.010774in}{1.676975in}}{\pgfqpoint{1.007502in}{1.669075in}}{\pgfqpoint{1.007502in}{1.660838in}}%
\pgfpathcurveto{\pgfqpoint{1.007502in}{1.652602in}}{\pgfqpoint{1.010774in}{1.644702in}}{\pgfqpoint{1.016598in}{1.638878in}}%
\pgfpathcurveto{\pgfqpoint{1.022422in}{1.633054in}}{\pgfqpoint{1.030322in}{1.629782in}}{\pgfqpoint{1.038558in}{1.629782in}}%
\pgfpathlineto{\pgfqpoint{1.038558in}{1.629782in}}%
\pgfusepath{stroke,fill}%
\end{pgfscope}%
\begin{pgfscope}%
\pgfpathrectangle{\pgfqpoint{0.624479in}{0.567253in}}{\pgfqpoint{1.638314in}{1.720247in}} %
\pgfusepath{clip}%
\pgfsetbuttcap%
\pgfsetroundjoin%
\definecolor{currentfill}{rgb}{0.298039,0.447059,0.690196}%
\pgfsetfillcolor{currentfill}%
\pgfsetlinewidth{0.301125pt}%
\definecolor{currentstroke}{rgb}{1.000000,1.000000,1.000000}%
\pgfsetstrokecolor{currentstroke}%
\pgfsetdash{}{0pt}%
\pgfpathmoveto{\pgfqpoint{1.428633in}{0.981003in}}%
\pgfpathcurveto{\pgfqpoint{1.436869in}{0.981003in}}{\pgfqpoint{1.444770in}{0.984275in}}{\pgfqpoint{1.450593in}{0.990099in}}%
\pgfpathcurveto{\pgfqpoint{1.456417in}{0.995923in}}{\pgfqpoint{1.459690in}{1.003823in}}{\pgfqpoint{1.459690in}{1.012059in}}%
\pgfpathcurveto{\pgfqpoint{1.459690in}{1.020296in}}{\pgfqpoint{1.456417in}{1.028196in}}{\pgfqpoint{1.450593in}{1.034020in}}%
\pgfpathcurveto{\pgfqpoint{1.444770in}{1.039844in}}{\pgfqpoint{1.436869in}{1.043116in}}{\pgfqpoint{1.428633in}{1.043116in}}%
\pgfpathcurveto{\pgfqpoint{1.420397in}{1.043116in}}{\pgfqpoint{1.412497in}{1.039844in}}{\pgfqpoint{1.406673in}{1.034020in}}%
\pgfpathcurveto{\pgfqpoint{1.400849in}{1.028196in}}{\pgfqpoint{1.397577in}{1.020296in}}{\pgfqpoint{1.397577in}{1.012059in}}%
\pgfpathcurveto{\pgfqpoint{1.397577in}{1.003823in}}{\pgfqpoint{1.400849in}{0.995923in}}{\pgfqpoint{1.406673in}{0.990099in}}%
\pgfpathcurveto{\pgfqpoint{1.412497in}{0.984275in}}{\pgfqpoint{1.420397in}{0.981003in}}{\pgfqpoint{1.428633in}{0.981003in}}%
\pgfpathlineto{\pgfqpoint{1.428633in}{0.981003in}}%
\pgfusepath{stroke,fill}%
\end{pgfscope}%
\begin{pgfscope}%
\pgfpathrectangle{\pgfqpoint{0.624479in}{0.567253in}}{\pgfqpoint{1.638314in}{1.720247in}} %
\pgfusepath{clip}%
\pgfsetbuttcap%
\pgfsetroundjoin%
\definecolor{currentfill}{rgb}{0.298039,0.447059,0.690196}%
\pgfsetfillcolor{currentfill}%
\pgfsetlinewidth{0.301125pt}%
\definecolor{currentstroke}{rgb}{1.000000,1.000000,1.000000}%
\pgfsetstrokecolor{currentstroke}%
\pgfsetdash{}{0pt}%
\pgfpathmoveto{\pgfqpoint{1.938731in}{0.861814in}}%
\pgfpathcurveto{\pgfqpoint{1.946967in}{0.861814in}}{\pgfqpoint{1.954867in}{0.865087in}}{\pgfqpoint{1.960691in}{0.870911in}}%
\pgfpathcurveto{\pgfqpoint{1.966515in}{0.876735in}}{\pgfqpoint{1.969787in}{0.884635in}}{\pgfqpoint{1.969787in}{0.892871in}}%
\pgfpathcurveto{\pgfqpoint{1.969787in}{0.901107in}}{\pgfqpoint{1.966515in}{0.909007in}}{\pgfqpoint{1.960691in}{0.914831in}}%
\pgfpathcurveto{\pgfqpoint{1.954867in}{0.920655in}}{\pgfqpoint{1.946967in}{0.923927in}}{\pgfqpoint{1.938731in}{0.923927in}}%
\pgfpathcurveto{\pgfqpoint{1.930495in}{0.923927in}}{\pgfqpoint{1.922595in}{0.920655in}}{\pgfqpoint{1.916771in}{0.914831in}}%
\pgfpathcurveto{\pgfqpoint{1.910947in}{0.909007in}}{\pgfqpoint{1.907674in}{0.901107in}}{\pgfqpoint{1.907674in}{0.892871in}}%
\pgfpathcurveto{\pgfqpoint{1.907674in}{0.884635in}}{\pgfqpoint{1.910947in}{0.876735in}}{\pgfqpoint{1.916771in}{0.870911in}}%
\pgfpathcurveto{\pgfqpoint{1.922595in}{0.865087in}}{\pgfqpoint{1.930495in}{0.861814in}}{\pgfqpoint{1.938731in}{0.861814in}}%
\pgfpathlineto{\pgfqpoint{1.938731in}{0.861814in}}%
\pgfusepath{stroke,fill}%
\end{pgfscope}%
\begin{pgfscope}%
\pgfpathrectangle{\pgfqpoint{0.624479in}{0.567253in}}{\pgfqpoint{1.638314in}{1.720247in}} %
\pgfusepath{clip}%
\pgfsetbuttcap%
\pgfsetroundjoin%
\definecolor{currentfill}{rgb}{0.298039,0.447059,0.690196}%
\pgfsetfillcolor{currentfill}%
\pgfsetlinewidth{0.301125pt}%
\definecolor{currentstroke}{rgb}{1.000000,1.000000,1.000000}%
\pgfsetstrokecolor{currentstroke}%
\pgfsetdash{}{0pt}%
\pgfpathmoveto{\pgfqpoint{1.278604in}{1.197263in}}%
\pgfpathcurveto{\pgfqpoint{1.286841in}{1.197263in}}{\pgfqpoint{1.294741in}{1.200535in}}{\pgfqpoint{1.300565in}{1.206359in}}%
\pgfpathcurveto{\pgfqpoint{1.306389in}{1.212183in}}{\pgfqpoint{1.309661in}{1.220083in}}{\pgfqpoint{1.309661in}{1.228319in}}%
\pgfpathcurveto{\pgfqpoint{1.309661in}{1.236555in}}{\pgfqpoint{1.306389in}{1.244455in}}{\pgfqpoint{1.300565in}{1.250279in}}%
\pgfpathcurveto{\pgfqpoint{1.294741in}{1.256103in}}{\pgfqpoint{1.286841in}{1.259376in}}{\pgfqpoint{1.278604in}{1.259376in}}%
\pgfpathcurveto{\pgfqpoint{1.270368in}{1.259376in}}{\pgfqpoint{1.262468in}{1.256103in}}{\pgfqpoint{1.256644in}{1.250279in}}%
\pgfpathcurveto{\pgfqpoint{1.250820in}{1.244455in}}{\pgfqpoint{1.247548in}{1.236555in}}{\pgfqpoint{1.247548in}{1.228319in}}%
\pgfpathcurveto{\pgfqpoint{1.247548in}{1.220083in}}{\pgfqpoint{1.250820in}{1.212183in}}{\pgfqpoint{1.256644in}{1.206359in}}%
\pgfpathcurveto{\pgfqpoint{1.262468in}{1.200535in}}{\pgfqpoint{1.270368in}{1.197263in}}{\pgfqpoint{1.278604in}{1.197263in}}%
\pgfpathlineto{\pgfqpoint{1.278604in}{1.197263in}}%
\pgfusepath{stroke,fill}%
\end{pgfscope}%
\begin{pgfscope}%
\pgfpathrectangle{\pgfqpoint{0.624479in}{0.567253in}}{\pgfqpoint{1.638314in}{1.720247in}} %
\pgfusepath{clip}%
\pgfsetbuttcap%
\pgfsetroundjoin%
\definecolor{currentfill}{rgb}{0.298039,0.447059,0.690196}%
\pgfsetfillcolor{currentfill}%
\pgfsetlinewidth{0.301125pt}%
\definecolor{currentstroke}{rgb}{1.000000,1.000000,1.000000}%
\pgfsetstrokecolor{currentstroke}%
\pgfsetdash{}{0pt}%
\pgfpathmoveto{\pgfqpoint{0.888530in}{1.761258in}}%
\pgfpathcurveto{\pgfqpoint{0.896766in}{1.761258in}}{\pgfqpoint{0.904666in}{1.764530in}}{\pgfqpoint{0.910490in}{1.770354in}}%
\pgfpathcurveto{\pgfqpoint{0.916314in}{1.776178in}}{\pgfqpoint{0.919586in}{1.784078in}}{\pgfqpoint{0.919586in}{1.792314in}}%
\pgfpathcurveto{\pgfqpoint{0.919586in}{1.800551in}}{\pgfqpoint{0.916314in}{1.808451in}}{\pgfqpoint{0.910490in}{1.814275in}}%
\pgfpathcurveto{\pgfqpoint{0.904666in}{1.820099in}}{\pgfqpoint{0.896766in}{1.823371in}}{\pgfqpoint{0.888530in}{1.823371in}}%
\pgfpathcurveto{\pgfqpoint{0.880293in}{1.823371in}}{\pgfqpoint{0.872393in}{1.820099in}}{\pgfqpoint{0.866569in}{1.814275in}}%
\pgfpathcurveto{\pgfqpoint{0.860746in}{1.808451in}}{\pgfqpoint{0.857473in}{1.800551in}}{\pgfqpoint{0.857473in}{1.792314in}}%
\pgfpathcurveto{\pgfqpoint{0.857473in}{1.784078in}}{\pgfqpoint{0.860746in}{1.776178in}}{\pgfqpoint{0.866569in}{1.770354in}}%
\pgfpathcurveto{\pgfqpoint{0.872393in}{1.764530in}}{\pgfqpoint{0.880293in}{1.761258in}}{\pgfqpoint{0.888530in}{1.761258in}}%
\pgfpathlineto{\pgfqpoint{0.888530in}{1.761258in}}%
\pgfusepath{stroke,fill}%
\end{pgfscope}%
\begin{pgfscope}%
\pgfpathrectangle{\pgfqpoint{0.624479in}{0.567253in}}{\pgfqpoint{1.638314in}{1.720247in}} %
\pgfusepath{clip}%
\pgfsetbuttcap%
\pgfsetroundjoin%
\definecolor{currentfill}{rgb}{0.298039,0.447059,0.690196}%
\pgfsetfillcolor{currentfill}%
\pgfsetlinewidth{0.301125pt}%
\definecolor{currentstroke}{rgb}{1.000000,1.000000,1.000000}%
\pgfsetstrokecolor{currentstroke}%
\pgfsetdash{}{0pt}%
\pgfpathmoveto{\pgfqpoint{1.368622in}{0.870416in}}%
\pgfpathcurveto{\pgfqpoint{1.376858in}{0.870416in}}{\pgfqpoint{1.384758in}{0.873688in}}{\pgfqpoint{1.390582in}{0.879512in}}%
\pgfpathcurveto{\pgfqpoint{1.396406in}{0.885336in}}{\pgfqpoint{1.399678in}{0.893236in}}{\pgfqpoint{1.399678in}{0.901472in}}%
\pgfpathcurveto{\pgfqpoint{1.399678in}{0.909708in}}{\pgfqpoint{1.396406in}{0.917608in}}{\pgfqpoint{1.390582in}{0.923432in}}%
\pgfpathcurveto{\pgfqpoint{1.384758in}{0.929256in}}{\pgfqpoint{1.376858in}{0.932529in}}{\pgfqpoint{1.368622in}{0.932529in}}%
\pgfpathcurveto{\pgfqpoint{1.360385in}{0.932529in}}{\pgfqpoint{1.352485in}{0.929256in}}{\pgfqpoint{1.346661in}{0.923432in}}%
\pgfpathcurveto{\pgfqpoint{1.340838in}{0.917608in}}{\pgfqpoint{1.337565in}{0.909708in}}{\pgfqpoint{1.337565in}{0.901472in}}%
\pgfpathcurveto{\pgfqpoint{1.337565in}{0.893236in}}{\pgfqpoint{1.340838in}{0.885336in}}{\pgfqpoint{1.346661in}{0.879512in}}%
\pgfpathcurveto{\pgfqpoint{1.352485in}{0.873688in}}{\pgfqpoint{1.360385in}{0.870416in}}{\pgfqpoint{1.368622in}{0.870416in}}%
\pgfpathlineto{\pgfqpoint{1.368622in}{0.870416in}}%
\pgfusepath{stroke,fill}%
\end{pgfscope}%
\begin{pgfscope}%
\pgfpathrectangle{\pgfqpoint{0.624479in}{0.567253in}}{\pgfqpoint{1.638314in}{1.720247in}} %
\pgfusepath{clip}%
\pgfsetbuttcap%
\pgfsetroundjoin%
\definecolor{currentfill}{rgb}{0.298039,0.447059,0.690196}%
\pgfsetfillcolor{currentfill}%
\pgfsetlinewidth{0.301125pt}%
\definecolor{currentstroke}{rgb}{1.000000,1.000000,1.000000}%
\pgfsetstrokecolor{currentstroke}%
\pgfsetdash{}{0pt}%
\pgfpathmoveto{\pgfqpoint{1.908725in}{0.840926in}}%
\pgfpathcurveto{\pgfqpoint{1.916961in}{0.840926in}}{\pgfqpoint{1.924861in}{0.844198in}}{\pgfqpoint{1.930685in}{0.850022in}}%
\pgfpathcurveto{\pgfqpoint{1.936509in}{0.855846in}}{\pgfqpoint{1.939782in}{0.863746in}}{\pgfqpoint{1.939782in}{0.871982in}}%
\pgfpathcurveto{\pgfqpoint{1.939782in}{0.880218in}}{\pgfqpoint{1.936509in}{0.888118in}}{\pgfqpoint{1.930685in}{0.893942in}}%
\pgfpathcurveto{\pgfqpoint{1.924861in}{0.899766in}}{\pgfqpoint{1.916961in}{0.903039in}}{\pgfqpoint{1.908725in}{0.903039in}}%
\pgfpathcurveto{\pgfqpoint{1.900489in}{0.903039in}}{\pgfqpoint{1.892589in}{0.899766in}}{\pgfqpoint{1.886765in}{0.893942in}}%
\pgfpathcurveto{\pgfqpoint{1.880941in}{0.888118in}}{\pgfqpoint{1.877669in}{0.880218in}}{\pgfqpoint{1.877669in}{0.871982in}}%
\pgfpathcurveto{\pgfqpoint{1.877669in}{0.863746in}}{\pgfqpoint{1.880941in}{0.855846in}}{\pgfqpoint{1.886765in}{0.850022in}}%
\pgfpathcurveto{\pgfqpoint{1.892589in}{0.844198in}}{\pgfqpoint{1.900489in}{0.840926in}}{\pgfqpoint{1.908725in}{0.840926in}}%
\pgfpathlineto{\pgfqpoint{1.908725in}{0.840926in}}%
\pgfusepath{stroke,fill}%
\end{pgfscope}%
\begin{pgfscope}%
\pgfpathrectangle{\pgfqpoint{0.624479in}{0.567253in}}{\pgfqpoint{1.638314in}{1.720247in}} %
\pgfusepath{clip}%
\pgfsetbuttcap%
\pgfsetroundjoin%
\definecolor{currentfill}{rgb}{0.298039,0.447059,0.690196}%
\pgfsetfillcolor{currentfill}%
\pgfsetlinewidth{0.301125pt}%
\definecolor{currentstroke}{rgb}{1.000000,1.000000,1.000000}%
\pgfsetstrokecolor{currentstroke}%
\pgfsetdash{}{0pt}%
\pgfpathmoveto{\pgfqpoint{1.728691in}{1.169001in}}%
\pgfpathcurveto{\pgfqpoint{1.736927in}{1.169001in}}{\pgfqpoint{1.744827in}{1.172274in}}{\pgfqpoint{1.750651in}{1.178098in}}%
\pgfpathcurveto{\pgfqpoint{1.756475in}{1.183922in}}{\pgfqpoint{1.759747in}{1.191822in}}{\pgfqpoint{1.759747in}{1.200058in}}%
\pgfpathcurveto{\pgfqpoint{1.759747in}{1.208294in}}{\pgfqpoint{1.756475in}{1.216194in}}{\pgfqpoint{1.750651in}{1.222018in}}%
\pgfpathcurveto{\pgfqpoint{1.744827in}{1.227842in}}{\pgfqpoint{1.736927in}{1.231114in}}{\pgfqpoint{1.728691in}{1.231114in}}%
\pgfpathcurveto{\pgfqpoint{1.720454in}{1.231114in}}{\pgfqpoint{1.712554in}{1.227842in}}{\pgfqpoint{1.706730in}{1.222018in}}%
\pgfpathcurveto{\pgfqpoint{1.700906in}{1.216194in}}{\pgfqpoint{1.697634in}{1.208294in}}{\pgfqpoint{1.697634in}{1.200058in}}%
\pgfpathcurveto{\pgfqpoint{1.697634in}{1.191822in}}{\pgfqpoint{1.700906in}{1.183922in}}{\pgfqpoint{1.706730in}{1.178098in}}%
\pgfpathcurveto{\pgfqpoint{1.712554in}{1.172274in}}{\pgfqpoint{1.720454in}{1.169001in}}{\pgfqpoint{1.728691in}{1.169001in}}%
\pgfpathlineto{\pgfqpoint{1.728691in}{1.169001in}}%
\pgfusepath{stroke,fill}%
\end{pgfscope}%
\begin{pgfscope}%
\pgfpathrectangle{\pgfqpoint{0.624479in}{0.567253in}}{\pgfqpoint{1.638314in}{1.720247in}} %
\pgfusepath{clip}%
\pgfsetbuttcap%
\pgfsetroundjoin%
\definecolor{currentfill}{rgb}{0.298039,0.447059,0.690196}%
\pgfsetfillcolor{currentfill}%
\pgfsetlinewidth{0.301125pt}%
\definecolor{currentstroke}{rgb}{1.000000,1.000000,1.000000}%
\pgfsetstrokecolor{currentstroke}%
\pgfsetdash{}{0pt}%
\pgfpathmoveto{\pgfqpoint{1.488645in}{0.971173in}}%
\pgfpathcurveto{\pgfqpoint{1.496881in}{0.971173in}}{\pgfqpoint{1.504781in}{0.974445in}}{\pgfqpoint{1.510605in}{0.980269in}}%
\pgfpathcurveto{\pgfqpoint{1.516429in}{0.986093in}}{\pgfqpoint{1.519701in}{0.993993in}}{\pgfqpoint{1.519701in}{1.002229in}}%
\pgfpathcurveto{\pgfqpoint{1.519701in}{1.010466in}}{\pgfqpoint{1.516429in}{1.018366in}}{\pgfqpoint{1.510605in}{1.024190in}}%
\pgfpathcurveto{\pgfqpoint{1.504781in}{1.030014in}}{\pgfqpoint{1.496881in}{1.033286in}}{\pgfqpoint{1.488645in}{1.033286in}}%
\pgfpathcurveto{\pgfqpoint{1.480408in}{1.033286in}}{\pgfqpoint{1.472508in}{1.030014in}}{\pgfqpoint{1.466684in}{1.024190in}}%
\pgfpathcurveto{\pgfqpoint{1.460860in}{1.018366in}}{\pgfqpoint{1.457588in}{1.010466in}}{\pgfqpoint{1.457588in}{1.002229in}}%
\pgfpathcurveto{\pgfqpoint{1.457588in}{0.993993in}}{\pgfqpoint{1.460860in}{0.986093in}}{\pgfqpoint{1.466684in}{0.980269in}}%
\pgfpathcurveto{\pgfqpoint{1.472508in}{0.974445in}}{\pgfqpoint{1.480408in}{0.971173in}}{\pgfqpoint{1.488645in}{0.971173in}}%
\pgfpathlineto{\pgfqpoint{1.488645in}{0.971173in}}%
\pgfusepath{stroke,fill}%
\end{pgfscope}%
\begin{pgfscope}%
\pgfpathrectangle{\pgfqpoint{0.624479in}{0.567253in}}{\pgfqpoint{1.638314in}{1.720247in}} %
\pgfusepath{clip}%
\pgfsetbuttcap%
\pgfsetroundjoin%
\definecolor{currentfill}{rgb}{0.298039,0.447059,0.690196}%
\pgfsetfillcolor{currentfill}%
\pgfsetlinewidth{0.301125pt}%
\definecolor{currentstroke}{rgb}{1.000000,1.000000,1.000000}%
\pgfsetstrokecolor{currentstroke}%
\pgfsetdash{}{0pt}%
\pgfpathmoveto{\pgfqpoint{1.008553in}{1.318909in}}%
\pgfpathcurveto{\pgfqpoint{1.016789in}{1.318909in}}{\pgfqpoint{1.024689in}{1.322181in}}{\pgfqpoint{1.030513in}{1.328005in}}%
\pgfpathcurveto{\pgfqpoint{1.036337in}{1.333829in}}{\pgfqpoint{1.039609in}{1.341729in}}{\pgfqpoint{1.039609in}{1.349965in}}%
\pgfpathcurveto{\pgfqpoint{1.039609in}{1.358201in}}{\pgfqpoint{1.036337in}{1.366102in}}{\pgfqpoint{1.030513in}{1.371925in}}%
\pgfpathcurveto{\pgfqpoint{1.024689in}{1.377749in}}{\pgfqpoint{1.016789in}{1.381022in}}{\pgfqpoint{1.008553in}{1.381022in}}%
\pgfpathcurveto{\pgfqpoint{1.000316in}{1.381022in}}{\pgfqpoint{0.992416in}{1.377749in}}{\pgfqpoint{0.986592in}{1.371925in}}%
\pgfpathcurveto{\pgfqpoint{0.980769in}{1.366102in}}{\pgfqpoint{0.977496in}{1.358201in}}{\pgfqpoint{0.977496in}{1.349965in}}%
\pgfpathcurveto{\pgfqpoint{0.977496in}{1.341729in}}{\pgfqpoint{0.980769in}{1.333829in}}{\pgfqpoint{0.986592in}{1.328005in}}%
\pgfpathcurveto{\pgfqpoint{0.992416in}{1.322181in}}{\pgfqpoint{1.000316in}{1.318909in}}{\pgfqpoint{1.008553in}{1.318909in}}%
\pgfpathlineto{\pgfqpoint{1.008553in}{1.318909in}}%
\pgfusepath{stroke,fill}%
\end{pgfscope}%
\begin{pgfscope}%
\pgfpathrectangle{\pgfqpoint{0.624479in}{0.567253in}}{\pgfqpoint{1.638314in}{1.720247in}} %
\pgfusepath{clip}%
\pgfsetbuttcap%
\pgfsetroundjoin%
\definecolor{currentfill}{rgb}{0.298039,0.447059,0.690196}%
\pgfsetfillcolor{currentfill}%
\pgfsetlinewidth{0.301125pt}%
\definecolor{currentstroke}{rgb}{1.000000,1.000000,1.000000}%
\pgfsetstrokecolor{currentstroke}%
\pgfsetdash{}{0pt}%
\pgfpathmoveto{\pgfqpoint{1.338616in}{1.540083in}}%
\pgfpathcurveto{\pgfqpoint{1.346852in}{1.540083in}}{\pgfqpoint{1.354752in}{1.543356in}}{\pgfqpoint{1.360576in}{1.549180in}}%
\pgfpathcurveto{\pgfqpoint{1.366400in}{1.555004in}}{\pgfqpoint{1.369672in}{1.562904in}}{\pgfqpoint{1.369672in}{1.571140in}}%
\pgfpathcurveto{\pgfqpoint{1.369672in}{1.579376in}}{\pgfqpoint{1.366400in}{1.587276in}}{\pgfqpoint{1.360576in}{1.593100in}}%
\pgfpathcurveto{\pgfqpoint{1.354752in}{1.598924in}}{\pgfqpoint{1.346852in}{1.602196in}}{\pgfqpoint{1.338616in}{1.602196in}}%
\pgfpathcurveto{\pgfqpoint{1.330380in}{1.602196in}}{\pgfqpoint{1.322480in}{1.598924in}}{\pgfqpoint{1.316656in}{1.593100in}}%
\pgfpathcurveto{\pgfqpoint{1.310832in}{1.587276in}}{\pgfqpoint{1.307559in}{1.579376in}}{\pgfqpoint{1.307559in}{1.571140in}}%
\pgfpathcurveto{\pgfqpoint{1.307559in}{1.562904in}}{\pgfqpoint{1.310832in}{1.555004in}}{\pgfqpoint{1.316656in}{1.549180in}}%
\pgfpathcurveto{\pgfqpoint{1.322480in}{1.543356in}}{\pgfqpoint{1.330380in}{1.540083in}}{\pgfqpoint{1.338616in}{1.540083in}}%
\pgfpathlineto{\pgfqpoint{1.338616in}{1.540083in}}%
\pgfusepath{stroke,fill}%
\end{pgfscope}%
\begin{pgfscope}%
\pgfpathrectangle{\pgfqpoint{0.624479in}{0.567253in}}{\pgfqpoint{1.638314in}{1.720247in}} %
\pgfusepath{clip}%
\pgfsetbuttcap%
\pgfsetroundjoin%
\definecolor{currentfill}{rgb}{0.298039,0.447059,0.690196}%
\pgfsetfillcolor{currentfill}%
\pgfsetlinewidth{0.301125pt}%
\definecolor{currentstroke}{rgb}{1.000000,1.000000,1.000000}%
\pgfsetstrokecolor{currentstroke}%
\pgfsetdash{}{0pt}%
\pgfpathmoveto{\pgfqpoint{0.918535in}{1.544998in}}%
\pgfpathcurveto{\pgfqpoint{0.926772in}{1.544998in}}{\pgfqpoint{0.934672in}{1.548271in}}{\pgfqpoint{0.940496in}{1.554095in}}%
\pgfpathcurveto{\pgfqpoint{0.946320in}{1.559918in}}{\pgfqpoint{0.949592in}{1.567819in}}{\pgfqpoint{0.949592in}{1.576055in}}%
\pgfpathcurveto{\pgfqpoint{0.949592in}{1.584291in}}{\pgfqpoint{0.946320in}{1.592191in}}{\pgfqpoint{0.940496in}{1.598015in}}%
\pgfpathcurveto{\pgfqpoint{0.934672in}{1.603839in}}{\pgfqpoint{0.926772in}{1.607111in}}{\pgfqpoint{0.918535in}{1.607111in}}%
\pgfpathcurveto{\pgfqpoint{0.910299in}{1.607111in}}{\pgfqpoint{0.902399in}{1.603839in}}{\pgfqpoint{0.896575in}{1.598015in}}%
\pgfpathcurveto{\pgfqpoint{0.890751in}{1.592191in}}{\pgfqpoint{0.887479in}{1.584291in}}{\pgfqpoint{0.887479in}{1.576055in}}%
\pgfpathcurveto{\pgfqpoint{0.887479in}{1.567819in}}{\pgfqpoint{0.890751in}{1.559918in}}{\pgfqpoint{0.896575in}{1.554095in}}%
\pgfpathcurveto{\pgfqpoint{0.902399in}{1.548271in}}{\pgfqpoint{0.910299in}{1.544998in}}{\pgfqpoint{0.918535in}{1.544998in}}%
\pgfpathlineto{\pgfqpoint{0.918535in}{1.544998in}}%
\pgfusepath{stroke,fill}%
\end{pgfscope}%
\begin{pgfscope}%
\pgfpathrectangle{\pgfqpoint{0.624479in}{0.567253in}}{\pgfqpoint{1.638314in}{1.720247in}} %
\pgfusepath{clip}%
\pgfsetbuttcap%
\pgfsetroundjoin%
\definecolor{currentfill}{rgb}{0.298039,0.447059,0.690196}%
\pgfsetfillcolor{currentfill}%
\pgfsetlinewidth{0.301125pt}%
\definecolor{currentstroke}{rgb}{1.000000,1.000000,1.000000}%
\pgfsetstrokecolor{currentstroke}%
\pgfsetdash{}{0pt}%
\pgfpathmoveto{\pgfqpoint{1.758696in}{1.212008in}}%
\pgfpathcurveto{\pgfqpoint{1.766933in}{1.212008in}}{\pgfqpoint{1.774833in}{1.215280in}}{\pgfqpoint{1.780657in}{1.221104in}}%
\pgfpathcurveto{\pgfqpoint{1.786481in}{1.226928in}}{\pgfqpoint{1.789753in}{1.234828in}}{\pgfqpoint{1.789753in}{1.243064in}}%
\pgfpathcurveto{\pgfqpoint{1.789753in}{1.251300in}}{\pgfqpoint{1.786481in}{1.259200in}}{\pgfqpoint{1.780657in}{1.265024in}}%
\pgfpathcurveto{\pgfqpoint{1.774833in}{1.270848in}}{\pgfqpoint{1.766933in}{1.274121in}}{\pgfqpoint{1.758696in}{1.274121in}}%
\pgfpathcurveto{\pgfqpoint{1.750460in}{1.274121in}}{\pgfqpoint{1.742560in}{1.270848in}}{\pgfqpoint{1.736736in}{1.265024in}}%
\pgfpathcurveto{\pgfqpoint{1.730912in}{1.259200in}}{\pgfqpoint{1.727640in}{1.251300in}}{\pgfqpoint{1.727640in}{1.243064in}}%
\pgfpathcurveto{\pgfqpoint{1.727640in}{1.234828in}}{\pgfqpoint{1.730912in}{1.226928in}}{\pgfqpoint{1.736736in}{1.221104in}}%
\pgfpathcurveto{\pgfqpoint{1.742560in}{1.215280in}}{\pgfqpoint{1.750460in}{1.212008in}}{\pgfqpoint{1.758696in}{1.212008in}}%
\pgfpathlineto{\pgfqpoint{1.758696in}{1.212008in}}%
\pgfusepath{stroke,fill}%
\end{pgfscope}%
\begin{pgfscope}%
\pgfpathrectangle{\pgfqpoint{0.624479in}{0.567253in}}{\pgfqpoint{1.638314in}{1.720247in}} %
\pgfusepath{clip}%
\pgfsetbuttcap%
\pgfsetroundjoin%
\definecolor{currentfill}{rgb}{0.298039,0.447059,0.690196}%
\pgfsetfillcolor{currentfill}%
\pgfsetlinewidth{0.301125pt}%
\definecolor{currentstroke}{rgb}{1.000000,1.000000,1.000000}%
\pgfsetstrokecolor{currentstroke}%
\pgfsetdash{}{0pt}%
\pgfpathmoveto{\pgfqpoint{1.188587in}{1.688762in}}%
\pgfpathcurveto{\pgfqpoint{1.196823in}{1.688762in}}{\pgfqpoint{1.204724in}{1.692034in}}{\pgfqpoint{1.210547in}{1.697858in}}%
\pgfpathcurveto{\pgfqpoint{1.216371in}{1.703682in}}{\pgfqpoint{1.219644in}{1.711582in}}{\pgfqpoint{1.219644in}{1.719818in}}%
\pgfpathcurveto{\pgfqpoint{1.219644in}{1.728055in}}{\pgfqpoint{1.216371in}{1.735955in}}{\pgfqpoint{1.210547in}{1.741779in}}%
\pgfpathcurveto{\pgfqpoint{1.204724in}{1.747603in}}{\pgfqpoint{1.196823in}{1.750875in}}{\pgfqpoint{1.188587in}{1.750875in}}%
\pgfpathcurveto{\pgfqpoint{1.180351in}{1.750875in}}{\pgfqpoint{1.172451in}{1.747603in}}{\pgfqpoint{1.166627in}{1.741779in}}%
\pgfpathcurveto{\pgfqpoint{1.160803in}{1.735955in}}{\pgfqpoint{1.157531in}{1.728055in}}{\pgfqpoint{1.157531in}{1.719818in}}%
\pgfpathcurveto{\pgfqpoint{1.157531in}{1.711582in}}{\pgfqpoint{1.160803in}{1.703682in}}{\pgfqpoint{1.166627in}{1.697858in}}%
\pgfpathcurveto{\pgfqpoint{1.172451in}{1.692034in}}{\pgfqpoint{1.180351in}{1.688762in}}{\pgfqpoint{1.188587in}{1.688762in}}%
\pgfpathlineto{\pgfqpoint{1.188587in}{1.688762in}}%
\pgfusepath{stroke,fill}%
\end{pgfscope}%
\begin{pgfscope}%
\pgfpathrectangle{\pgfqpoint{0.624479in}{0.567253in}}{\pgfqpoint{1.638314in}{1.720247in}} %
\pgfusepath{clip}%
\pgfsetbuttcap%
\pgfsetroundjoin%
\definecolor{currentfill}{rgb}{0.298039,0.447059,0.690196}%
\pgfsetfillcolor{currentfill}%
\pgfsetlinewidth{0.301125pt}%
\definecolor{currentstroke}{rgb}{1.000000,1.000000,1.000000}%
\pgfsetstrokecolor{currentstroke}%
\pgfsetdash{}{0pt}%
\pgfpathmoveto{\pgfqpoint{1.218593in}{1.311536in}}%
\pgfpathcurveto{\pgfqpoint{1.226829in}{1.311536in}}{\pgfqpoint{1.234729in}{1.314808in}}{\pgfqpoint{1.240553in}{1.320632in}}%
\pgfpathcurveto{\pgfqpoint{1.246377in}{1.326456in}}{\pgfqpoint{1.249649in}{1.334356in}}{\pgfqpoint{1.249649in}{1.342593in}}%
\pgfpathcurveto{\pgfqpoint{1.249649in}{1.350829in}}{\pgfqpoint{1.246377in}{1.358729in}}{\pgfqpoint{1.240553in}{1.364553in}}%
\pgfpathcurveto{\pgfqpoint{1.234729in}{1.370377in}}{\pgfqpoint{1.226829in}{1.373649in}}{\pgfqpoint{1.218593in}{1.373649in}}%
\pgfpathcurveto{\pgfqpoint{1.210357in}{1.373649in}}{\pgfqpoint{1.202457in}{1.370377in}}{\pgfqpoint{1.196633in}{1.364553in}}%
\pgfpathcurveto{\pgfqpoint{1.190809in}{1.358729in}}{\pgfqpoint{1.187536in}{1.350829in}}{\pgfqpoint{1.187536in}{1.342593in}}%
\pgfpathcurveto{\pgfqpoint{1.187536in}{1.334356in}}{\pgfqpoint{1.190809in}{1.326456in}}{\pgfqpoint{1.196633in}{1.320632in}}%
\pgfpathcurveto{\pgfqpoint{1.202457in}{1.314808in}}{\pgfqpoint{1.210357in}{1.311536in}}{\pgfqpoint{1.218593in}{1.311536in}}%
\pgfpathlineto{\pgfqpoint{1.218593in}{1.311536in}}%
\pgfusepath{stroke,fill}%
\end{pgfscope}%
\begin{pgfscope}%
\pgfpathrectangle{\pgfqpoint{0.624479in}{0.567253in}}{\pgfqpoint{1.638314in}{1.720247in}} %
\pgfusepath{clip}%
\pgfsetbuttcap%
\pgfsetroundjoin%
\definecolor{currentfill}{rgb}{0.298039,0.447059,0.690196}%
\pgfsetfillcolor{currentfill}%
\pgfsetlinewidth{0.301125pt}%
\definecolor{currentstroke}{rgb}{1.000000,1.000000,1.000000}%
\pgfsetstrokecolor{currentstroke}%
\pgfsetdash{}{0pt}%
\pgfpathmoveto{\pgfqpoint{1.998742in}{1.230439in}}%
\pgfpathcurveto{\pgfqpoint{2.006979in}{1.230439in}}{\pgfqpoint{2.014879in}{1.233711in}}{\pgfqpoint{2.020703in}{1.239535in}}%
\pgfpathcurveto{\pgfqpoint{2.026527in}{1.245359in}}{\pgfqpoint{2.029799in}{1.253259in}}{\pgfqpoint{2.029799in}{1.261495in}}%
\pgfpathcurveto{\pgfqpoint{2.029799in}{1.269732in}}{\pgfqpoint{2.026527in}{1.277632in}}{\pgfqpoint{2.020703in}{1.283456in}}%
\pgfpathcurveto{\pgfqpoint{2.014879in}{1.289279in}}{\pgfqpoint{2.006979in}{1.292552in}}{\pgfqpoint{1.998742in}{1.292552in}}%
\pgfpathcurveto{\pgfqpoint{1.990506in}{1.292552in}}{\pgfqpoint{1.982606in}{1.289279in}}{\pgfqpoint{1.976782in}{1.283456in}}%
\pgfpathcurveto{\pgfqpoint{1.970958in}{1.277632in}}{\pgfqpoint{1.967686in}{1.269732in}}{\pgfqpoint{1.967686in}{1.261495in}}%
\pgfpathcurveto{\pgfqpoint{1.967686in}{1.253259in}}{\pgfqpoint{1.970958in}{1.245359in}}{\pgfqpoint{1.976782in}{1.239535in}}%
\pgfpathcurveto{\pgfqpoint{1.982606in}{1.233711in}}{\pgfqpoint{1.990506in}{1.230439in}}{\pgfqpoint{1.998742in}{1.230439in}}%
\pgfpathlineto{\pgfqpoint{1.998742in}{1.230439in}}%
\pgfusepath{stroke,fill}%
\end{pgfscope}%
\begin{pgfscope}%
\pgfpathrectangle{\pgfqpoint{0.624479in}{0.567253in}}{\pgfqpoint{1.638314in}{1.720247in}} %
\pgfusepath{clip}%
\pgfsetbuttcap%
\pgfsetroundjoin%
\definecolor{currentfill}{rgb}{0.298039,0.447059,0.690196}%
\pgfsetfillcolor{currentfill}%
\pgfsetlinewidth{0.301125pt}%
\definecolor{currentstroke}{rgb}{1.000000,1.000000,1.000000}%
\pgfsetstrokecolor{currentstroke}%
\pgfsetdash{}{0pt}%
\pgfpathmoveto{\pgfqpoint{1.788702in}{1.096505in}}%
\pgfpathcurveto{\pgfqpoint{1.796938in}{1.096505in}}{\pgfqpoint{1.804838in}{1.099778in}}{\pgfqpoint{1.810662in}{1.105601in}}%
\pgfpathcurveto{\pgfqpoint{1.816486in}{1.111425in}}{\pgfqpoint{1.819759in}{1.119325in}}{\pgfqpoint{1.819759in}{1.127562in}}%
\pgfpathcurveto{\pgfqpoint{1.819759in}{1.135798in}}{\pgfqpoint{1.816486in}{1.143698in}}{\pgfqpoint{1.810662in}{1.149522in}}%
\pgfpathcurveto{\pgfqpoint{1.804838in}{1.155346in}}{\pgfqpoint{1.796938in}{1.158618in}}{\pgfqpoint{1.788702in}{1.158618in}}%
\pgfpathcurveto{\pgfqpoint{1.780466in}{1.158618in}}{\pgfqpoint{1.772566in}{1.155346in}}{\pgfqpoint{1.766742in}{1.149522in}}%
\pgfpathcurveto{\pgfqpoint{1.760918in}{1.143698in}}{\pgfqpoint{1.757646in}{1.135798in}}{\pgfqpoint{1.757646in}{1.127562in}}%
\pgfpathcurveto{\pgfqpoint{1.757646in}{1.119325in}}{\pgfqpoint{1.760918in}{1.111425in}}{\pgfqpoint{1.766742in}{1.105601in}}%
\pgfpathcurveto{\pgfqpoint{1.772566in}{1.099778in}}{\pgfqpoint{1.780466in}{1.096505in}}{\pgfqpoint{1.788702in}{1.096505in}}%
\pgfpathlineto{\pgfqpoint{1.788702in}{1.096505in}}%
\pgfusepath{stroke,fill}%
\end{pgfscope}%
\begin{pgfscope}%
\pgfpathrectangle{\pgfqpoint{0.624479in}{0.567253in}}{\pgfqpoint{1.638314in}{1.720247in}} %
\pgfusepath{clip}%
\pgfsetbuttcap%
\pgfsetroundjoin%
\definecolor{currentfill}{rgb}{0.298039,0.447059,0.690196}%
\pgfsetfillcolor{currentfill}%
\pgfsetlinewidth{0.301125pt}%
\definecolor{currentstroke}{rgb}{1.000000,1.000000,1.000000}%
\pgfsetstrokecolor{currentstroke}%
\pgfsetdash{}{0pt}%
\pgfpathmoveto{\pgfqpoint{0.978547in}{1.992263in}}%
\pgfpathcurveto{\pgfqpoint{0.986783in}{1.992263in}}{\pgfqpoint{0.994683in}{1.995535in}}{\pgfqpoint{1.000507in}{2.001359in}}%
\pgfpathcurveto{\pgfqpoint{1.006331in}{2.007183in}}{\pgfqpoint{1.009603in}{2.015083in}}{\pgfqpoint{1.009603in}{2.023319in}}%
\pgfpathcurveto{\pgfqpoint{1.009603in}{2.031555in}}{\pgfqpoint{1.006331in}{2.039455in}}{\pgfqpoint{1.000507in}{2.045279in}}%
\pgfpathcurveto{\pgfqpoint{0.994683in}{2.051103in}}{\pgfqpoint{0.986783in}{2.054376in}}{\pgfqpoint{0.978547in}{2.054376in}}%
\pgfpathcurveto{\pgfqpoint{0.970311in}{2.054376in}}{\pgfqpoint{0.962411in}{2.051103in}}{\pgfqpoint{0.956587in}{2.045279in}}%
\pgfpathcurveto{\pgfqpoint{0.950763in}{2.039455in}}{\pgfqpoint{0.947490in}{2.031555in}}{\pgfqpoint{0.947490in}{2.023319in}}%
\pgfpathcurveto{\pgfqpoint{0.947490in}{2.015083in}}{\pgfqpoint{0.950763in}{2.007183in}}{\pgfqpoint{0.956587in}{2.001359in}}%
\pgfpathcurveto{\pgfqpoint{0.962411in}{1.995535in}}{\pgfqpoint{0.970311in}{1.992263in}}{\pgfqpoint{0.978547in}{1.992263in}}%
\pgfpathlineto{\pgfqpoint{0.978547in}{1.992263in}}%
\pgfusepath{stroke,fill}%
\end{pgfscope}%
\begin{pgfscope}%
\pgfpathrectangle{\pgfqpoint{0.624479in}{0.567253in}}{\pgfqpoint{1.638314in}{1.720247in}} %
\pgfusepath{clip}%
\pgfsetbuttcap%
\pgfsetroundjoin%
\definecolor{currentfill}{rgb}{0.298039,0.447059,0.690196}%
\pgfsetfillcolor{currentfill}%
\pgfsetlinewidth{0.301125pt}%
\definecolor{currentstroke}{rgb}{1.000000,1.000000,1.000000}%
\pgfsetstrokecolor{currentstroke}%
\pgfsetdash{}{0pt}%
\pgfpathmoveto{\pgfqpoint{1.968737in}{1.001892in}}%
\pgfpathcurveto{\pgfqpoint{1.976973in}{1.001892in}}{\pgfqpoint{1.984873in}{1.005164in}}{\pgfqpoint{1.990697in}{1.010988in}}%
\pgfpathcurveto{\pgfqpoint{1.996521in}{1.016812in}}{\pgfqpoint{1.999793in}{1.024712in}}{\pgfqpoint{1.999793in}{1.032948in}}%
\pgfpathcurveto{\pgfqpoint{1.999793in}{1.041184in}}{\pgfqpoint{1.996521in}{1.049084in}}{\pgfqpoint{1.990697in}{1.054908in}}%
\pgfpathcurveto{\pgfqpoint{1.984873in}{1.060732in}}{\pgfqpoint{1.976973in}{1.064005in}}{\pgfqpoint{1.968737in}{1.064005in}}%
\pgfpathcurveto{\pgfqpoint{1.960500in}{1.064005in}}{\pgfqpoint{1.952600in}{1.060732in}}{\pgfqpoint{1.946776in}{1.054908in}}%
\pgfpathcurveto{\pgfqpoint{1.940952in}{1.049084in}}{\pgfqpoint{1.937680in}{1.041184in}}{\pgfqpoint{1.937680in}{1.032948in}}%
\pgfpathcurveto{\pgfqpoint{1.937680in}{1.024712in}}{\pgfqpoint{1.940952in}{1.016812in}}{\pgfqpoint{1.946776in}{1.010988in}}%
\pgfpathcurveto{\pgfqpoint{1.952600in}{1.005164in}}{\pgfqpoint{1.960500in}{1.001892in}}{\pgfqpoint{1.968737in}{1.001892in}}%
\pgfpathlineto{\pgfqpoint{1.968737in}{1.001892in}}%
\pgfusepath{stroke,fill}%
\end{pgfscope}%
\begin{pgfscope}%
\pgfpathrectangle{\pgfqpoint{0.624479in}{0.567253in}}{\pgfqpoint{1.638314in}{1.720247in}} %
\pgfusepath{clip}%
\pgfsetbuttcap%
\pgfsetroundjoin%
\definecolor{currentfill}{rgb}{0.298039,0.447059,0.690196}%
\pgfsetfillcolor{currentfill}%
\pgfsetlinewidth{0.301125pt}%
\definecolor{currentstroke}{rgb}{1.000000,1.000000,1.000000}%
\pgfsetstrokecolor{currentstroke}%
\pgfsetdash{}{0pt}%
\pgfpathmoveto{\pgfqpoint{1.638673in}{1.726853in}}%
\pgfpathcurveto{\pgfqpoint{1.646910in}{1.726853in}}{\pgfqpoint{1.654810in}{1.730125in}}{\pgfqpoint{1.660634in}{1.735949in}}%
\pgfpathcurveto{\pgfqpoint{1.666458in}{1.741773in}}{\pgfqpoint{1.669730in}{1.749673in}}{\pgfqpoint{1.669730in}{1.757910in}}%
\pgfpathcurveto{\pgfqpoint{1.669730in}{1.766146in}}{\pgfqpoint{1.666458in}{1.774046in}}{\pgfqpoint{1.660634in}{1.779870in}}%
\pgfpathcurveto{\pgfqpoint{1.654810in}{1.785694in}}{\pgfqpoint{1.646910in}{1.788966in}}{\pgfqpoint{1.638673in}{1.788966in}}%
\pgfpathcurveto{\pgfqpoint{1.630437in}{1.788966in}}{\pgfqpoint{1.622537in}{1.785694in}}{\pgfqpoint{1.616713in}{1.779870in}}%
\pgfpathcurveto{\pgfqpoint{1.610889in}{1.774046in}}{\pgfqpoint{1.607617in}{1.766146in}}{\pgfqpoint{1.607617in}{1.757910in}}%
\pgfpathcurveto{\pgfqpoint{1.607617in}{1.749673in}}{\pgfqpoint{1.610889in}{1.741773in}}{\pgfqpoint{1.616713in}{1.735949in}}%
\pgfpathcurveto{\pgfqpoint{1.622537in}{1.730125in}}{\pgfqpoint{1.630437in}{1.726853in}}{\pgfqpoint{1.638673in}{1.726853in}}%
\pgfpathlineto{\pgfqpoint{1.638673in}{1.726853in}}%
\pgfusepath{stroke,fill}%
\end{pgfscope}%
\begin{pgfscope}%
\pgfpathrectangle{\pgfqpoint{0.624479in}{0.567253in}}{\pgfqpoint{1.638314in}{1.720247in}} %
\pgfusepath{clip}%
\pgfsetbuttcap%
\pgfsetroundjoin%
\definecolor{currentfill}{rgb}{0.298039,0.447059,0.690196}%
\pgfsetfillcolor{currentfill}%
\pgfsetlinewidth{0.301125pt}%
\definecolor{currentstroke}{rgb}{1.000000,1.000000,1.000000}%
\pgfsetstrokecolor{currentstroke}%
\pgfsetdash{}{0pt}%
\pgfpathmoveto{\pgfqpoint{1.878719in}{1.420895in}}%
\pgfpathcurveto{\pgfqpoint{1.886956in}{1.420895in}}{\pgfqpoint{1.894856in}{1.424167in}}{\pgfqpoint{1.900680in}{1.429991in}}%
\pgfpathcurveto{\pgfqpoint{1.906504in}{1.435815in}}{\pgfqpoint{1.909776in}{1.443715in}}{\pgfqpoint{1.909776in}{1.451951in}}%
\pgfpathcurveto{\pgfqpoint{1.909776in}{1.460188in}}{\pgfqpoint{1.906504in}{1.468088in}}{\pgfqpoint{1.900680in}{1.473912in}}%
\pgfpathcurveto{\pgfqpoint{1.894856in}{1.479735in}}{\pgfqpoint{1.886956in}{1.483008in}}{\pgfqpoint{1.878719in}{1.483008in}}%
\pgfpathcurveto{\pgfqpoint{1.870483in}{1.483008in}}{\pgfqpoint{1.862583in}{1.479735in}}{\pgfqpoint{1.856759in}{1.473912in}}%
\pgfpathcurveto{\pgfqpoint{1.850935in}{1.468088in}}{\pgfqpoint{1.847663in}{1.460188in}}{\pgfqpoint{1.847663in}{1.451951in}}%
\pgfpathcurveto{\pgfqpoint{1.847663in}{1.443715in}}{\pgfqpoint{1.850935in}{1.435815in}}{\pgfqpoint{1.856759in}{1.429991in}}%
\pgfpathcurveto{\pgfqpoint{1.862583in}{1.424167in}}{\pgfqpoint{1.870483in}{1.420895in}}{\pgfqpoint{1.878719in}{1.420895in}}%
\pgfpathlineto{\pgfqpoint{1.878719in}{1.420895in}}%
\pgfusepath{stroke,fill}%
\end{pgfscope}%
\begin{pgfscope}%
\pgfpathrectangle{\pgfqpoint{0.624479in}{0.567253in}}{\pgfqpoint{1.638314in}{1.720247in}} %
\pgfusepath{clip}%
\pgfsetbuttcap%
\pgfsetroundjoin%
\definecolor{currentfill}{rgb}{0.298039,0.447059,0.690196}%
\pgfsetfillcolor{currentfill}%
\pgfsetlinewidth{0.301125pt}%
\definecolor{currentstroke}{rgb}{1.000000,1.000000,1.000000}%
\pgfsetstrokecolor{currentstroke}%
\pgfsetdash{}{0pt}%
\pgfpathmoveto{\pgfqpoint{1.608668in}{1.773545in}}%
\pgfpathcurveto{\pgfqpoint{1.616904in}{1.773545in}}{\pgfqpoint{1.624804in}{1.776818in}}{\pgfqpoint{1.630628in}{1.782642in}}%
\pgfpathcurveto{\pgfqpoint{1.636452in}{1.788466in}}{\pgfqpoint{1.639724in}{1.796366in}}{\pgfqpoint{1.639724in}{1.804602in}}%
\pgfpathcurveto{\pgfqpoint{1.639724in}{1.812838in}}{\pgfqpoint{1.636452in}{1.820738in}}{\pgfqpoint{1.630628in}{1.826562in}}%
\pgfpathcurveto{\pgfqpoint{1.624804in}{1.832386in}}{\pgfqpoint{1.616904in}{1.835658in}}{\pgfqpoint{1.608668in}{1.835658in}}%
\pgfpathcurveto{\pgfqpoint{1.600431in}{1.835658in}}{\pgfqpoint{1.592531in}{1.832386in}}{\pgfqpoint{1.586707in}{1.826562in}}%
\pgfpathcurveto{\pgfqpoint{1.580883in}{1.820738in}}{\pgfqpoint{1.577611in}{1.812838in}}{\pgfqpoint{1.577611in}{1.804602in}}%
\pgfpathcurveto{\pgfqpoint{1.577611in}{1.796366in}}{\pgfqpoint{1.580883in}{1.788466in}}{\pgfqpoint{1.586707in}{1.782642in}}%
\pgfpathcurveto{\pgfqpoint{1.592531in}{1.776818in}}{\pgfqpoint{1.600431in}{1.773545in}}{\pgfqpoint{1.608668in}{1.773545in}}%
\pgfpathlineto{\pgfqpoint{1.608668in}{1.773545in}}%
\pgfusepath{stroke,fill}%
\end{pgfscope}%
\begin{pgfscope}%
\pgfpathrectangle{\pgfqpoint{0.624479in}{0.567253in}}{\pgfqpoint{1.638314in}{1.720247in}} %
\pgfusepath{clip}%
\pgfsetbuttcap%
\pgfsetroundjoin%
\definecolor{currentfill}{rgb}{0.298039,0.447059,0.690196}%
\pgfsetfillcolor{currentfill}%
\pgfsetlinewidth{0.301125pt}%
\definecolor{currentstroke}{rgb}{1.000000,1.000000,1.000000}%
\pgfsetstrokecolor{currentstroke}%
\pgfsetdash{}{0pt}%
\pgfpathmoveto{\pgfqpoint{1.548656in}{1.751428in}}%
\pgfpathcurveto{\pgfqpoint{1.556892in}{1.751428in}}{\pgfqpoint{1.564793in}{1.754700in}}{\pgfqpoint{1.570616in}{1.760524in}}%
\pgfpathcurveto{\pgfqpoint{1.576440in}{1.766348in}}{\pgfqpoint{1.579713in}{1.774248in}}{\pgfqpoint{1.579713in}{1.782485in}}%
\pgfpathcurveto{\pgfqpoint{1.579713in}{1.790721in}}{\pgfqpoint{1.576440in}{1.798621in}}{\pgfqpoint{1.570616in}{1.804445in}}%
\pgfpathcurveto{\pgfqpoint{1.564793in}{1.810269in}}{\pgfqpoint{1.556892in}{1.813541in}}{\pgfqpoint{1.548656in}{1.813541in}}%
\pgfpathcurveto{\pgfqpoint{1.540420in}{1.813541in}}{\pgfqpoint{1.532520in}{1.810269in}}{\pgfqpoint{1.526696in}{1.804445in}}%
\pgfpathcurveto{\pgfqpoint{1.520872in}{1.798621in}}{\pgfqpoint{1.517600in}{1.790721in}}{\pgfqpoint{1.517600in}{1.782485in}}%
\pgfpathcurveto{\pgfqpoint{1.517600in}{1.774248in}}{\pgfqpoint{1.520872in}{1.766348in}}{\pgfqpoint{1.526696in}{1.760524in}}%
\pgfpathcurveto{\pgfqpoint{1.532520in}{1.754700in}}{\pgfqpoint{1.540420in}{1.751428in}}{\pgfqpoint{1.548656in}{1.751428in}}%
\pgfpathlineto{\pgfqpoint{1.548656in}{1.751428in}}%
\pgfusepath{stroke,fill}%
\end{pgfscope}%
\begin{pgfscope}%
\pgfpathrectangle{\pgfqpoint{0.624479in}{0.567253in}}{\pgfqpoint{1.638314in}{1.720247in}} %
\pgfusepath{clip}%
\pgfsetbuttcap%
\pgfsetroundjoin%
\definecolor{currentfill}{rgb}{0.298039,0.447059,0.690196}%
\pgfsetfillcolor{currentfill}%
\pgfsetlinewidth{0.301125pt}%
\definecolor{currentstroke}{rgb}{1.000000,1.000000,1.000000}%
\pgfsetstrokecolor{currentstroke}%
\pgfsetdash{}{0pt}%
\pgfpathmoveto{\pgfqpoint{1.158581in}{1.149341in}}%
\pgfpathcurveto{\pgfqpoint{1.166818in}{1.149341in}}{\pgfqpoint{1.174718in}{1.152614in}}{\pgfqpoint{1.180542in}{1.158438in}}%
\pgfpathcurveto{\pgfqpoint{1.186366in}{1.164262in}}{\pgfqpoint{1.189638in}{1.172162in}}{\pgfqpoint{1.189638in}{1.180398in}}%
\pgfpathcurveto{\pgfqpoint{1.189638in}{1.188634in}}{\pgfqpoint{1.186366in}{1.196534in}}{\pgfqpoint{1.180542in}{1.202358in}}%
\pgfpathcurveto{\pgfqpoint{1.174718in}{1.208182in}}{\pgfqpoint{1.166818in}{1.211454in}}{\pgfqpoint{1.158581in}{1.211454in}}%
\pgfpathcurveto{\pgfqpoint{1.150345in}{1.211454in}}{\pgfqpoint{1.142445in}{1.208182in}}{\pgfqpoint{1.136621in}{1.202358in}}%
\pgfpathcurveto{\pgfqpoint{1.130797in}{1.196534in}}{\pgfqpoint{1.127525in}{1.188634in}}{\pgfqpoint{1.127525in}{1.180398in}}%
\pgfpathcurveto{\pgfqpoint{1.127525in}{1.172162in}}{\pgfqpoint{1.130797in}{1.164262in}}{\pgfqpoint{1.136621in}{1.158438in}}%
\pgfpathcurveto{\pgfqpoint{1.142445in}{1.152614in}}{\pgfqpoint{1.150345in}{1.149341in}}{\pgfqpoint{1.158581in}{1.149341in}}%
\pgfpathlineto{\pgfqpoint{1.158581in}{1.149341in}}%
\pgfusepath{stroke,fill}%
\end{pgfscope}%
\begin{pgfscope}%
\pgfpathrectangle{\pgfqpoint{0.624479in}{0.567253in}}{\pgfqpoint{1.638314in}{1.720247in}} %
\pgfusepath{clip}%
\pgfsetbuttcap%
\pgfsetroundjoin%
\definecolor{currentfill}{rgb}{0.298039,0.447059,0.690196}%
\pgfsetfillcolor{currentfill}%
\pgfsetlinewidth{0.301125pt}%
\definecolor{currentstroke}{rgb}{1.000000,1.000000,1.000000}%
\pgfsetstrokecolor{currentstroke}%
\pgfsetdash{}{0pt}%
\pgfpathmoveto{\pgfqpoint{1.308610in}{1.718252in}}%
\pgfpathcurveto{\pgfqpoint{1.316846in}{1.718252in}}{\pgfqpoint{1.324747in}{1.721524in}}{\pgfqpoint{1.330570in}{1.727348in}}%
\pgfpathcurveto{\pgfqpoint{1.336394in}{1.733172in}}{\pgfqpoint{1.339667in}{1.741072in}}{\pgfqpoint{1.339667in}{1.749308in}}%
\pgfpathcurveto{\pgfqpoint{1.339667in}{1.757545in}}{\pgfqpoint{1.336394in}{1.765445in}}{\pgfqpoint{1.330570in}{1.771269in}}%
\pgfpathcurveto{\pgfqpoint{1.324747in}{1.777093in}}{\pgfqpoint{1.316846in}{1.780365in}}{\pgfqpoint{1.308610in}{1.780365in}}%
\pgfpathcurveto{\pgfqpoint{1.300374in}{1.780365in}}{\pgfqpoint{1.292474in}{1.777093in}}{\pgfqpoint{1.286650in}{1.771269in}}%
\pgfpathcurveto{\pgfqpoint{1.280826in}{1.765445in}}{\pgfqpoint{1.277554in}{1.757545in}}{\pgfqpoint{1.277554in}{1.749308in}}%
\pgfpathcurveto{\pgfqpoint{1.277554in}{1.741072in}}{\pgfqpoint{1.280826in}{1.733172in}}{\pgfqpoint{1.286650in}{1.727348in}}%
\pgfpathcurveto{\pgfqpoint{1.292474in}{1.721524in}}{\pgfqpoint{1.300374in}{1.718252in}}{\pgfqpoint{1.308610in}{1.718252in}}%
\pgfpathlineto{\pgfqpoint{1.308610in}{1.718252in}}%
\pgfusepath{stroke,fill}%
\end{pgfscope}%
\begin{pgfscope}%
\pgfpathrectangle{\pgfqpoint{0.624479in}{0.567253in}}{\pgfqpoint{1.638314in}{1.720247in}} %
\pgfusepath{clip}%
\pgfsetbuttcap%
\pgfsetroundjoin%
\definecolor{currentfill}{rgb}{0.298039,0.447059,0.690196}%
\pgfsetfillcolor{currentfill}%
\pgfsetlinewidth{0.301125pt}%
\definecolor{currentstroke}{rgb}{1.000000,1.000000,1.000000}%
\pgfsetstrokecolor{currentstroke}%
\pgfsetdash{}{0pt}%
\pgfpathmoveto{\pgfqpoint{1.578662in}{1.934511in}}%
\pgfpathcurveto{\pgfqpoint{1.586898in}{1.934511in}}{\pgfqpoint{1.594798in}{1.937784in}}{\pgfqpoint{1.600622in}{1.943608in}}%
\pgfpathcurveto{\pgfqpoint{1.606446in}{1.949432in}}{\pgfqpoint{1.609718in}{1.957332in}}{\pgfqpoint{1.609718in}{1.965568in}}%
\pgfpathcurveto{\pgfqpoint{1.609718in}{1.973804in}}{\pgfqpoint{1.606446in}{1.981704in}}{\pgfqpoint{1.600622in}{1.987528in}}%
\pgfpathcurveto{\pgfqpoint{1.594798in}{1.993352in}}{\pgfqpoint{1.586898in}{1.996624in}}{\pgfqpoint{1.578662in}{1.996624in}}%
\pgfpathcurveto{\pgfqpoint{1.570426in}{1.996624in}}{\pgfqpoint{1.562526in}{1.993352in}}{\pgfqpoint{1.556702in}{1.987528in}}%
\pgfpathcurveto{\pgfqpoint{1.550878in}{1.981704in}}{\pgfqpoint{1.547605in}{1.973804in}}{\pgfqpoint{1.547605in}{1.965568in}}%
\pgfpathcurveto{\pgfqpoint{1.547605in}{1.957332in}}{\pgfqpoint{1.550878in}{1.949432in}}{\pgfqpoint{1.556702in}{1.943608in}}%
\pgfpathcurveto{\pgfqpoint{1.562526in}{1.937784in}}{\pgfqpoint{1.570426in}{1.934511in}}{\pgfqpoint{1.578662in}{1.934511in}}%
\pgfpathlineto{\pgfqpoint{1.578662in}{1.934511in}}%
\pgfusepath{stroke,fill}%
\end{pgfscope}%
\begin{pgfscope}%
\pgfpathrectangle{\pgfqpoint{0.624479in}{0.567253in}}{\pgfqpoint{1.638314in}{1.720247in}} %
\pgfusepath{clip}%
\pgfsetbuttcap%
\pgfsetroundjoin%
\definecolor{currentfill}{rgb}{0.298039,0.447059,0.690196}%
\pgfsetfillcolor{currentfill}%
\pgfsetlinewidth{0.301125pt}%
\definecolor{currentstroke}{rgb}{1.000000,1.000000,1.000000}%
\pgfsetstrokecolor{currentstroke}%
\pgfsetdash{}{0pt}%
\pgfpathmoveto{\pgfqpoint{0.948541in}{1.188661in}}%
\pgfpathcurveto{\pgfqpoint{0.956778in}{1.188661in}}{\pgfqpoint{0.964678in}{1.191934in}}{\pgfqpoint{0.970501in}{1.197758in}}%
\pgfpathcurveto{\pgfqpoint{0.976325in}{1.203582in}}{\pgfqpoint{0.979598in}{1.211482in}}{\pgfqpoint{0.979598in}{1.219718in}}%
\pgfpathcurveto{\pgfqpoint{0.979598in}{1.227954in}}{\pgfqpoint{0.976325in}{1.235854in}}{\pgfqpoint{0.970501in}{1.241678in}}%
\pgfpathcurveto{\pgfqpoint{0.964678in}{1.247502in}}{\pgfqpoint{0.956778in}{1.250774in}}{\pgfqpoint{0.948541in}{1.250774in}}%
\pgfpathcurveto{\pgfqpoint{0.940305in}{1.250774in}}{\pgfqpoint{0.932405in}{1.247502in}}{\pgfqpoint{0.926581in}{1.241678in}}%
\pgfpathcurveto{\pgfqpoint{0.920757in}{1.235854in}}{\pgfqpoint{0.917485in}{1.227954in}}{\pgfqpoint{0.917485in}{1.219718in}}%
\pgfpathcurveto{\pgfqpoint{0.917485in}{1.211482in}}{\pgfqpoint{0.920757in}{1.203582in}}{\pgfqpoint{0.926581in}{1.197758in}}%
\pgfpathcurveto{\pgfqpoint{0.932405in}{1.191934in}}{\pgfqpoint{0.940305in}{1.188661in}}{\pgfqpoint{0.948541in}{1.188661in}}%
\pgfpathlineto{\pgfqpoint{0.948541in}{1.188661in}}%
\pgfusepath{stroke,fill}%
\end{pgfscope}%
\begin{pgfscope}%
\pgfpathrectangle{\pgfqpoint{0.624479in}{0.567253in}}{\pgfqpoint{1.638314in}{1.720247in}} %
\pgfusepath{clip}%
\pgfsetbuttcap%
\pgfsetroundjoin%
\definecolor{currentfill}{rgb}{0.298039,0.447059,0.690196}%
\pgfsetfillcolor{currentfill}%
\pgfsetlinewidth{0.301125pt}%
\definecolor{currentstroke}{rgb}{1.000000,1.000000,1.000000}%
\pgfsetstrokecolor{currentstroke}%
\pgfsetdash{}{0pt}%
\pgfpathmoveto{\pgfqpoint{1.248599in}{1.559743in}}%
\pgfpathcurveto{\pgfqpoint{1.256835in}{1.559743in}}{\pgfqpoint{1.264735in}{1.563016in}}{\pgfqpoint{1.270559in}{1.568840in}}%
\pgfpathcurveto{\pgfqpoint{1.276383in}{1.574663in}}{\pgfqpoint{1.279655in}{1.582564in}}{\pgfqpoint{1.279655in}{1.590800in}}%
\pgfpathcurveto{\pgfqpoint{1.279655in}{1.599036in}}{\pgfqpoint{1.276383in}{1.606936in}}{\pgfqpoint{1.270559in}{1.612760in}}%
\pgfpathcurveto{\pgfqpoint{1.264735in}{1.618584in}}{\pgfqpoint{1.256835in}{1.621856in}}{\pgfqpoint{1.248599in}{1.621856in}}%
\pgfpathcurveto{\pgfqpoint{1.240362in}{1.621856in}}{\pgfqpoint{1.232462in}{1.618584in}}{\pgfqpoint{1.226638in}{1.612760in}}%
\pgfpathcurveto{\pgfqpoint{1.220815in}{1.606936in}}{\pgfqpoint{1.217542in}{1.599036in}}{\pgfqpoint{1.217542in}{1.590800in}}%
\pgfpathcurveto{\pgfqpoint{1.217542in}{1.582564in}}{\pgfqpoint{1.220815in}{1.574663in}}{\pgfqpoint{1.226638in}{1.568840in}}%
\pgfpathcurveto{\pgfqpoint{1.232462in}{1.563016in}}{\pgfqpoint{1.240362in}{1.559743in}}{\pgfqpoint{1.248599in}{1.559743in}}%
\pgfpathlineto{\pgfqpoint{1.248599in}{1.559743in}}%
\pgfusepath{stroke,fill}%
\end{pgfscope}%
\begin{pgfscope}%
\pgfsetrectcap%
\pgfsetmiterjoin%
\pgfsetlinewidth{0.000000pt}%
\definecolor{currentstroke}{rgb}{1.000000,1.000000,1.000000}%
\pgfsetstrokecolor{currentstroke}%
\pgfsetdash{}{0pt}%
\pgfpathmoveto{\pgfqpoint{0.624479in}{0.567253in}}%
\pgfpathlineto{\pgfqpoint{2.262793in}{0.567253in}}%
\pgfusepath{}%
\end{pgfscope}%
\begin{pgfscope}%
\pgfsetrectcap%
\pgfsetmiterjoin%
\pgfsetlinewidth{0.000000pt}%
\definecolor{currentstroke}{rgb}{1.000000,1.000000,1.000000}%
\pgfsetstrokecolor{currentstroke}%
\pgfsetdash{}{0pt}%
\pgfpathmoveto{\pgfqpoint{0.624479in}{0.567253in}}%
\pgfpathlineto{\pgfqpoint{0.624479in}{2.287500in}}%
\pgfusepath{}%
\end{pgfscope}%
\end{pgfpicture}%
\makeatother%
\endgroup%

		\caption{Comparison between the two times from different throws.}
		\label{fig_wtr_vs_avg4}
	\end{subfigure}
	\caption{Plots of time measurments made by two observers(obs)}
\end{figure}


\begin{figure}[h]{.5\linewidth}
	%% Creator: Matplotlib, PGF backend
%%
%% To include the figure in your LaTeX document, write
%%   \input{<filename>.pgf}
%%
%% Make sure the required packages are loaded in your preamble
%%   \usepackage{pgf}
%%
%% Figures using additional raster images can only be included by \input if
%% they are in the same directory as the main LaTeX file. For loading figures
%% from other directories you can use the `import` package
%%   \usepackage{import}
%% and then include the figures with
%%   \import{<path to file>}{<filename>.pgf}
%%
%% Matplotlib used the following preamble
%%   \usepackage[utf8x]{inputenc}
%%   \usepackage[T1]{fontenc}
%%   \usepackage{cmbright}
%%
\begingroup%
\makeatletter%
\begin{pgfpicture}%
\pgfpathrectangle{\pgfpointorigin}{\pgfqpoint{5.000000in}{3.000000in}}%
\pgfusepath{use as bounding box, clip}%
\begin{pgfscope}%
\pgfsetbuttcap%
\pgfsetmiterjoin%
\definecolor{currentfill}{rgb}{1.000000,1.000000,1.000000}%
\pgfsetfillcolor{currentfill}%
\pgfsetlinewidth{0.000000pt}%
\definecolor{currentstroke}{rgb}{1.000000,1.000000,1.000000}%
\pgfsetstrokecolor{currentstroke}%
\pgfsetdash{}{0pt}%
\pgfpathmoveto{\pgfqpoint{0.000000in}{0.000000in}}%
\pgfpathlineto{\pgfqpoint{5.000000in}{0.000000in}}%
\pgfpathlineto{\pgfqpoint{5.000000in}{3.000000in}}%
\pgfpathlineto{\pgfqpoint{0.000000in}{3.000000in}}%
\pgfpathclose%
\pgfusepath{fill}%
\end{pgfscope}%
\begin{pgfscope}%
\pgfsetbuttcap%
\pgfsetmiterjoin%
\definecolor{currentfill}{rgb}{0.917647,0.917647,0.949020}%
\pgfsetfillcolor{currentfill}%
\pgfsetlinewidth{0.000000pt}%
\definecolor{currentstroke}{rgb}{0.000000,0.000000,0.000000}%
\pgfsetstrokecolor{currentstroke}%
\pgfsetstrokeopacity{0.000000}%
\pgfsetdash{}{0pt}%
\pgfpathmoveto{\pgfqpoint{0.625000in}{0.375000in}}%
\pgfpathlineto{\pgfqpoint{4.500000in}{0.375000in}}%
\pgfpathlineto{\pgfqpoint{4.500000in}{2.700000in}}%
\pgfpathlineto{\pgfqpoint{0.625000in}{2.700000in}}%
\pgfpathclose%
\pgfusepath{fill}%
\end{pgfscope}%
\begin{pgfscope}%
\pgfpathrectangle{\pgfqpoint{0.625000in}{0.375000in}}{\pgfqpoint{3.875000in}{2.325000in}} %
\pgfusepath{clip}%
\pgfsetroundcap%
\pgfsetroundjoin%
\pgfsetlinewidth{0.803000pt}%
\definecolor{currentstroke}{rgb}{1.000000,1.000000,1.000000}%
\pgfsetstrokecolor{currentstroke}%
\pgfsetdash{}{0pt}%
\pgfpathmoveto{\pgfqpoint{0.852941in}{0.375000in}}%
\pgfpathlineto{\pgfqpoint{0.852941in}{2.700000in}}%
\pgfusepath{stroke}%
\end{pgfscope}%
\begin{pgfscope}%
\pgfsetbuttcap%
\pgfsetroundjoin%
\definecolor{currentfill}{rgb}{0.150000,0.150000,0.150000}%
\pgfsetfillcolor{currentfill}%
\pgfsetlinewidth{0.803000pt}%
\definecolor{currentstroke}{rgb}{0.150000,0.150000,0.150000}%
\pgfsetstrokecolor{currentstroke}%
\pgfsetdash{}{0pt}%
\pgfsys@defobject{currentmarker}{\pgfqpoint{0.000000in}{0.000000in}}{\pgfqpoint{0.000000in}{0.000000in}}{%
\pgfpathmoveto{\pgfqpoint{0.000000in}{0.000000in}}%
\pgfpathlineto{\pgfqpoint{0.000000in}{0.000000in}}%
\pgfusepath{stroke,fill}%
}%
\begin{pgfscope}%
\pgfsys@transformshift{0.852941in}{0.375000in}%
\pgfsys@useobject{currentmarker}{}%
\end{pgfscope}%
\end{pgfscope}%
\begin{pgfscope}%
\pgfsetbuttcap%
\pgfsetroundjoin%
\definecolor{currentfill}{rgb}{0.150000,0.150000,0.150000}%
\pgfsetfillcolor{currentfill}%
\pgfsetlinewidth{0.803000pt}%
\definecolor{currentstroke}{rgb}{0.150000,0.150000,0.150000}%
\pgfsetstrokecolor{currentstroke}%
\pgfsetdash{}{0pt}%
\pgfsys@defobject{currentmarker}{\pgfqpoint{0.000000in}{0.000000in}}{\pgfqpoint{0.000000in}{0.000000in}}{%
\pgfpathmoveto{\pgfqpoint{0.000000in}{0.000000in}}%
\pgfpathlineto{\pgfqpoint{0.000000in}{0.000000in}}%
\pgfusepath{stroke,fill}%
}%
\begin{pgfscope}%
\pgfsys@transformshift{0.852941in}{2.700000in}%
\pgfsys@useobject{currentmarker}{}%
\end{pgfscope}%
\end{pgfscope}%
\begin{pgfscope}%
\definecolor{textcolor}{rgb}{0.150000,0.150000,0.150000}%
\pgfsetstrokecolor{textcolor}%
\pgfsetfillcolor{textcolor}%
\pgftext[x=0.852941in,y=0.297222in,,top]{\color{textcolor}\sffamily\fontsize{8.000000}{9.600000}\selectfont −1.5}%
\end{pgfscope}%
\begin{pgfscope}%
\pgfpathrectangle{\pgfqpoint{0.625000in}{0.375000in}}{\pgfqpoint{3.875000in}{2.325000in}} %
\pgfusepath{clip}%
\pgfsetroundcap%
\pgfsetroundjoin%
\pgfsetlinewidth{0.803000pt}%
\definecolor{currentstroke}{rgb}{1.000000,1.000000,1.000000}%
\pgfsetstrokecolor{currentstroke}%
\pgfsetdash{}{0pt}%
\pgfpathmoveto{\pgfqpoint{1.422794in}{0.375000in}}%
\pgfpathlineto{\pgfqpoint{1.422794in}{2.700000in}}%
\pgfusepath{stroke}%
\end{pgfscope}%
\begin{pgfscope}%
\pgfsetbuttcap%
\pgfsetroundjoin%
\definecolor{currentfill}{rgb}{0.150000,0.150000,0.150000}%
\pgfsetfillcolor{currentfill}%
\pgfsetlinewidth{0.803000pt}%
\definecolor{currentstroke}{rgb}{0.150000,0.150000,0.150000}%
\pgfsetstrokecolor{currentstroke}%
\pgfsetdash{}{0pt}%
\pgfsys@defobject{currentmarker}{\pgfqpoint{0.000000in}{0.000000in}}{\pgfqpoint{0.000000in}{0.000000in}}{%
\pgfpathmoveto{\pgfqpoint{0.000000in}{0.000000in}}%
\pgfpathlineto{\pgfqpoint{0.000000in}{0.000000in}}%
\pgfusepath{stroke,fill}%
}%
\begin{pgfscope}%
\pgfsys@transformshift{1.422794in}{0.375000in}%
\pgfsys@useobject{currentmarker}{}%
\end{pgfscope}%
\end{pgfscope}%
\begin{pgfscope}%
\pgfsetbuttcap%
\pgfsetroundjoin%
\definecolor{currentfill}{rgb}{0.150000,0.150000,0.150000}%
\pgfsetfillcolor{currentfill}%
\pgfsetlinewidth{0.803000pt}%
\definecolor{currentstroke}{rgb}{0.150000,0.150000,0.150000}%
\pgfsetstrokecolor{currentstroke}%
\pgfsetdash{}{0pt}%
\pgfsys@defobject{currentmarker}{\pgfqpoint{0.000000in}{0.000000in}}{\pgfqpoint{0.000000in}{0.000000in}}{%
\pgfpathmoveto{\pgfqpoint{0.000000in}{0.000000in}}%
\pgfpathlineto{\pgfqpoint{0.000000in}{0.000000in}}%
\pgfusepath{stroke,fill}%
}%
\begin{pgfscope}%
\pgfsys@transformshift{1.422794in}{2.700000in}%
\pgfsys@useobject{currentmarker}{}%
\end{pgfscope}%
\end{pgfscope}%
\begin{pgfscope}%
\definecolor{textcolor}{rgb}{0.150000,0.150000,0.150000}%
\pgfsetstrokecolor{textcolor}%
\pgfsetfillcolor{textcolor}%
\pgftext[x=1.422794in,y=0.297222in,,top]{\color{textcolor}\sffamily\fontsize{8.000000}{9.600000}\selectfont −1.0}%
\end{pgfscope}%
\begin{pgfscope}%
\pgfpathrectangle{\pgfqpoint{0.625000in}{0.375000in}}{\pgfqpoint{3.875000in}{2.325000in}} %
\pgfusepath{clip}%
\pgfsetroundcap%
\pgfsetroundjoin%
\pgfsetlinewidth{0.803000pt}%
\definecolor{currentstroke}{rgb}{1.000000,1.000000,1.000000}%
\pgfsetstrokecolor{currentstroke}%
\pgfsetdash{}{0pt}%
\pgfpathmoveto{\pgfqpoint{1.992647in}{0.375000in}}%
\pgfpathlineto{\pgfqpoint{1.992647in}{2.700000in}}%
\pgfusepath{stroke}%
\end{pgfscope}%
\begin{pgfscope}%
\pgfsetbuttcap%
\pgfsetroundjoin%
\definecolor{currentfill}{rgb}{0.150000,0.150000,0.150000}%
\pgfsetfillcolor{currentfill}%
\pgfsetlinewidth{0.803000pt}%
\definecolor{currentstroke}{rgb}{0.150000,0.150000,0.150000}%
\pgfsetstrokecolor{currentstroke}%
\pgfsetdash{}{0pt}%
\pgfsys@defobject{currentmarker}{\pgfqpoint{0.000000in}{0.000000in}}{\pgfqpoint{0.000000in}{0.000000in}}{%
\pgfpathmoveto{\pgfqpoint{0.000000in}{0.000000in}}%
\pgfpathlineto{\pgfqpoint{0.000000in}{0.000000in}}%
\pgfusepath{stroke,fill}%
}%
\begin{pgfscope}%
\pgfsys@transformshift{1.992647in}{0.375000in}%
\pgfsys@useobject{currentmarker}{}%
\end{pgfscope}%
\end{pgfscope}%
\begin{pgfscope}%
\pgfsetbuttcap%
\pgfsetroundjoin%
\definecolor{currentfill}{rgb}{0.150000,0.150000,0.150000}%
\pgfsetfillcolor{currentfill}%
\pgfsetlinewidth{0.803000pt}%
\definecolor{currentstroke}{rgb}{0.150000,0.150000,0.150000}%
\pgfsetstrokecolor{currentstroke}%
\pgfsetdash{}{0pt}%
\pgfsys@defobject{currentmarker}{\pgfqpoint{0.000000in}{0.000000in}}{\pgfqpoint{0.000000in}{0.000000in}}{%
\pgfpathmoveto{\pgfqpoint{0.000000in}{0.000000in}}%
\pgfpathlineto{\pgfqpoint{0.000000in}{0.000000in}}%
\pgfusepath{stroke,fill}%
}%
\begin{pgfscope}%
\pgfsys@transformshift{1.992647in}{2.700000in}%
\pgfsys@useobject{currentmarker}{}%
\end{pgfscope}%
\end{pgfscope}%
\begin{pgfscope}%
\definecolor{textcolor}{rgb}{0.150000,0.150000,0.150000}%
\pgfsetstrokecolor{textcolor}%
\pgfsetfillcolor{textcolor}%
\pgftext[x=1.992647in,y=0.297222in,,top]{\color{textcolor}\sffamily\fontsize{8.000000}{9.600000}\selectfont −0.5}%
\end{pgfscope}%
\begin{pgfscope}%
\pgfpathrectangle{\pgfqpoint{0.625000in}{0.375000in}}{\pgfqpoint{3.875000in}{2.325000in}} %
\pgfusepath{clip}%
\pgfsetroundcap%
\pgfsetroundjoin%
\pgfsetlinewidth{0.803000pt}%
\definecolor{currentstroke}{rgb}{1.000000,1.000000,1.000000}%
\pgfsetstrokecolor{currentstroke}%
\pgfsetdash{}{0pt}%
\pgfpathmoveto{\pgfqpoint{2.562500in}{0.375000in}}%
\pgfpathlineto{\pgfqpoint{2.562500in}{2.700000in}}%
\pgfusepath{stroke}%
\end{pgfscope}%
\begin{pgfscope}%
\pgfsetbuttcap%
\pgfsetroundjoin%
\definecolor{currentfill}{rgb}{0.150000,0.150000,0.150000}%
\pgfsetfillcolor{currentfill}%
\pgfsetlinewidth{0.803000pt}%
\definecolor{currentstroke}{rgb}{0.150000,0.150000,0.150000}%
\pgfsetstrokecolor{currentstroke}%
\pgfsetdash{}{0pt}%
\pgfsys@defobject{currentmarker}{\pgfqpoint{0.000000in}{0.000000in}}{\pgfqpoint{0.000000in}{0.000000in}}{%
\pgfpathmoveto{\pgfqpoint{0.000000in}{0.000000in}}%
\pgfpathlineto{\pgfqpoint{0.000000in}{0.000000in}}%
\pgfusepath{stroke,fill}%
}%
\begin{pgfscope}%
\pgfsys@transformshift{2.562500in}{0.375000in}%
\pgfsys@useobject{currentmarker}{}%
\end{pgfscope}%
\end{pgfscope}%
\begin{pgfscope}%
\pgfsetbuttcap%
\pgfsetroundjoin%
\definecolor{currentfill}{rgb}{0.150000,0.150000,0.150000}%
\pgfsetfillcolor{currentfill}%
\pgfsetlinewidth{0.803000pt}%
\definecolor{currentstroke}{rgb}{0.150000,0.150000,0.150000}%
\pgfsetstrokecolor{currentstroke}%
\pgfsetdash{}{0pt}%
\pgfsys@defobject{currentmarker}{\pgfqpoint{0.000000in}{0.000000in}}{\pgfqpoint{0.000000in}{0.000000in}}{%
\pgfpathmoveto{\pgfqpoint{0.000000in}{0.000000in}}%
\pgfpathlineto{\pgfqpoint{0.000000in}{0.000000in}}%
\pgfusepath{stroke,fill}%
}%
\begin{pgfscope}%
\pgfsys@transformshift{2.562500in}{2.700000in}%
\pgfsys@useobject{currentmarker}{}%
\end{pgfscope}%
\end{pgfscope}%
\begin{pgfscope}%
\definecolor{textcolor}{rgb}{0.150000,0.150000,0.150000}%
\pgfsetstrokecolor{textcolor}%
\pgfsetfillcolor{textcolor}%
\pgftext[x=2.562500in,y=0.297222in,,top]{\color{textcolor}\sffamily\fontsize{8.000000}{9.600000}\selectfont 0.0}%
\end{pgfscope}%
\begin{pgfscope}%
\pgfpathrectangle{\pgfqpoint{0.625000in}{0.375000in}}{\pgfqpoint{3.875000in}{2.325000in}} %
\pgfusepath{clip}%
\pgfsetroundcap%
\pgfsetroundjoin%
\pgfsetlinewidth{0.803000pt}%
\definecolor{currentstroke}{rgb}{1.000000,1.000000,1.000000}%
\pgfsetstrokecolor{currentstroke}%
\pgfsetdash{}{0pt}%
\pgfpathmoveto{\pgfqpoint{3.132353in}{0.375000in}}%
\pgfpathlineto{\pgfqpoint{3.132353in}{2.700000in}}%
\pgfusepath{stroke}%
\end{pgfscope}%
\begin{pgfscope}%
\pgfsetbuttcap%
\pgfsetroundjoin%
\definecolor{currentfill}{rgb}{0.150000,0.150000,0.150000}%
\pgfsetfillcolor{currentfill}%
\pgfsetlinewidth{0.803000pt}%
\definecolor{currentstroke}{rgb}{0.150000,0.150000,0.150000}%
\pgfsetstrokecolor{currentstroke}%
\pgfsetdash{}{0pt}%
\pgfsys@defobject{currentmarker}{\pgfqpoint{0.000000in}{0.000000in}}{\pgfqpoint{0.000000in}{0.000000in}}{%
\pgfpathmoveto{\pgfqpoint{0.000000in}{0.000000in}}%
\pgfpathlineto{\pgfqpoint{0.000000in}{0.000000in}}%
\pgfusepath{stroke,fill}%
}%
\begin{pgfscope}%
\pgfsys@transformshift{3.132353in}{0.375000in}%
\pgfsys@useobject{currentmarker}{}%
\end{pgfscope}%
\end{pgfscope}%
\begin{pgfscope}%
\pgfsetbuttcap%
\pgfsetroundjoin%
\definecolor{currentfill}{rgb}{0.150000,0.150000,0.150000}%
\pgfsetfillcolor{currentfill}%
\pgfsetlinewidth{0.803000pt}%
\definecolor{currentstroke}{rgb}{0.150000,0.150000,0.150000}%
\pgfsetstrokecolor{currentstroke}%
\pgfsetdash{}{0pt}%
\pgfsys@defobject{currentmarker}{\pgfqpoint{0.000000in}{0.000000in}}{\pgfqpoint{0.000000in}{0.000000in}}{%
\pgfpathmoveto{\pgfqpoint{0.000000in}{0.000000in}}%
\pgfpathlineto{\pgfqpoint{0.000000in}{0.000000in}}%
\pgfusepath{stroke,fill}%
}%
\begin{pgfscope}%
\pgfsys@transformshift{3.132353in}{2.700000in}%
\pgfsys@useobject{currentmarker}{}%
\end{pgfscope}%
\end{pgfscope}%
\begin{pgfscope}%
\definecolor{textcolor}{rgb}{0.150000,0.150000,0.150000}%
\pgfsetstrokecolor{textcolor}%
\pgfsetfillcolor{textcolor}%
\pgftext[x=3.132353in,y=0.297222in,,top]{\color{textcolor}\sffamily\fontsize{8.000000}{9.600000}\selectfont 0.5}%
\end{pgfscope}%
\begin{pgfscope}%
\pgfpathrectangle{\pgfqpoint{0.625000in}{0.375000in}}{\pgfqpoint{3.875000in}{2.325000in}} %
\pgfusepath{clip}%
\pgfsetroundcap%
\pgfsetroundjoin%
\pgfsetlinewidth{0.803000pt}%
\definecolor{currentstroke}{rgb}{1.000000,1.000000,1.000000}%
\pgfsetstrokecolor{currentstroke}%
\pgfsetdash{}{0pt}%
\pgfpathmoveto{\pgfqpoint{3.702206in}{0.375000in}}%
\pgfpathlineto{\pgfqpoint{3.702206in}{2.700000in}}%
\pgfusepath{stroke}%
\end{pgfscope}%
\begin{pgfscope}%
\pgfsetbuttcap%
\pgfsetroundjoin%
\definecolor{currentfill}{rgb}{0.150000,0.150000,0.150000}%
\pgfsetfillcolor{currentfill}%
\pgfsetlinewidth{0.803000pt}%
\definecolor{currentstroke}{rgb}{0.150000,0.150000,0.150000}%
\pgfsetstrokecolor{currentstroke}%
\pgfsetdash{}{0pt}%
\pgfsys@defobject{currentmarker}{\pgfqpoint{0.000000in}{0.000000in}}{\pgfqpoint{0.000000in}{0.000000in}}{%
\pgfpathmoveto{\pgfqpoint{0.000000in}{0.000000in}}%
\pgfpathlineto{\pgfqpoint{0.000000in}{0.000000in}}%
\pgfusepath{stroke,fill}%
}%
\begin{pgfscope}%
\pgfsys@transformshift{3.702206in}{0.375000in}%
\pgfsys@useobject{currentmarker}{}%
\end{pgfscope}%
\end{pgfscope}%
\begin{pgfscope}%
\pgfsetbuttcap%
\pgfsetroundjoin%
\definecolor{currentfill}{rgb}{0.150000,0.150000,0.150000}%
\pgfsetfillcolor{currentfill}%
\pgfsetlinewidth{0.803000pt}%
\definecolor{currentstroke}{rgb}{0.150000,0.150000,0.150000}%
\pgfsetstrokecolor{currentstroke}%
\pgfsetdash{}{0pt}%
\pgfsys@defobject{currentmarker}{\pgfqpoint{0.000000in}{0.000000in}}{\pgfqpoint{0.000000in}{0.000000in}}{%
\pgfpathmoveto{\pgfqpoint{0.000000in}{0.000000in}}%
\pgfpathlineto{\pgfqpoint{0.000000in}{0.000000in}}%
\pgfusepath{stroke,fill}%
}%
\begin{pgfscope}%
\pgfsys@transformshift{3.702206in}{2.700000in}%
\pgfsys@useobject{currentmarker}{}%
\end{pgfscope}%
\end{pgfscope}%
\begin{pgfscope}%
\definecolor{textcolor}{rgb}{0.150000,0.150000,0.150000}%
\pgfsetstrokecolor{textcolor}%
\pgfsetfillcolor{textcolor}%
\pgftext[x=3.702206in,y=0.297222in,,top]{\color{textcolor}\sffamily\fontsize{8.000000}{9.600000}\selectfont 1.0}%
\end{pgfscope}%
\begin{pgfscope}%
\pgfpathrectangle{\pgfqpoint{0.625000in}{0.375000in}}{\pgfqpoint{3.875000in}{2.325000in}} %
\pgfusepath{clip}%
\pgfsetroundcap%
\pgfsetroundjoin%
\pgfsetlinewidth{0.803000pt}%
\definecolor{currentstroke}{rgb}{1.000000,1.000000,1.000000}%
\pgfsetstrokecolor{currentstroke}%
\pgfsetdash{}{0pt}%
\pgfpathmoveto{\pgfqpoint{4.272059in}{0.375000in}}%
\pgfpathlineto{\pgfqpoint{4.272059in}{2.700000in}}%
\pgfusepath{stroke}%
\end{pgfscope}%
\begin{pgfscope}%
\pgfsetbuttcap%
\pgfsetroundjoin%
\definecolor{currentfill}{rgb}{0.150000,0.150000,0.150000}%
\pgfsetfillcolor{currentfill}%
\pgfsetlinewidth{0.803000pt}%
\definecolor{currentstroke}{rgb}{0.150000,0.150000,0.150000}%
\pgfsetstrokecolor{currentstroke}%
\pgfsetdash{}{0pt}%
\pgfsys@defobject{currentmarker}{\pgfqpoint{0.000000in}{0.000000in}}{\pgfqpoint{0.000000in}{0.000000in}}{%
\pgfpathmoveto{\pgfqpoint{0.000000in}{0.000000in}}%
\pgfpathlineto{\pgfqpoint{0.000000in}{0.000000in}}%
\pgfusepath{stroke,fill}%
}%
\begin{pgfscope}%
\pgfsys@transformshift{4.272059in}{0.375000in}%
\pgfsys@useobject{currentmarker}{}%
\end{pgfscope}%
\end{pgfscope}%
\begin{pgfscope}%
\pgfsetbuttcap%
\pgfsetroundjoin%
\definecolor{currentfill}{rgb}{0.150000,0.150000,0.150000}%
\pgfsetfillcolor{currentfill}%
\pgfsetlinewidth{0.803000pt}%
\definecolor{currentstroke}{rgb}{0.150000,0.150000,0.150000}%
\pgfsetstrokecolor{currentstroke}%
\pgfsetdash{}{0pt}%
\pgfsys@defobject{currentmarker}{\pgfqpoint{0.000000in}{0.000000in}}{\pgfqpoint{0.000000in}{0.000000in}}{%
\pgfpathmoveto{\pgfqpoint{0.000000in}{0.000000in}}%
\pgfpathlineto{\pgfqpoint{0.000000in}{0.000000in}}%
\pgfusepath{stroke,fill}%
}%
\begin{pgfscope}%
\pgfsys@transformshift{4.272059in}{2.700000in}%
\pgfsys@useobject{currentmarker}{}%
\end{pgfscope}%
\end{pgfscope}%
\begin{pgfscope}%
\definecolor{textcolor}{rgb}{0.150000,0.150000,0.150000}%
\pgfsetstrokecolor{textcolor}%
\pgfsetfillcolor{textcolor}%
\pgftext[x=4.272059in,y=0.297222in,,top]{\color{textcolor}\sffamily\fontsize{8.000000}{9.600000}\selectfont 1.5}%
\end{pgfscope}%
\begin{pgfscope}%
\definecolor{textcolor}{rgb}{0.150000,0.150000,0.150000}%
\pgfsetstrokecolor{textcolor}%
\pgfsetfillcolor{textcolor}%
\pgftext[x=2.562500in,y=0.132099in,,top]{\color{textcolor}\sffamily\fontsize{8.800000}{10.560000}\selectfont real values}%
\end{pgfscope}%
\begin{pgfscope}%
\pgfpathrectangle{\pgfqpoint{0.625000in}{0.375000in}}{\pgfqpoint{3.875000in}{2.325000in}} %
\pgfusepath{clip}%
\pgfsetroundcap%
\pgfsetroundjoin%
\pgfsetlinewidth{0.803000pt}%
\definecolor{currentstroke}{rgb}{1.000000,1.000000,1.000000}%
\pgfsetstrokecolor{currentstroke}%
\pgfsetdash{}{0pt}%
\pgfpathmoveto{\pgfqpoint{0.625000in}{0.511765in}}%
\pgfpathlineto{\pgfqpoint{4.500000in}{0.511765in}}%
\pgfusepath{stroke}%
\end{pgfscope}%
\begin{pgfscope}%
\pgfsetbuttcap%
\pgfsetroundjoin%
\definecolor{currentfill}{rgb}{0.150000,0.150000,0.150000}%
\pgfsetfillcolor{currentfill}%
\pgfsetlinewidth{0.803000pt}%
\definecolor{currentstroke}{rgb}{0.150000,0.150000,0.150000}%
\pgfsetstrokecolor{currentstroke}%
\pgfsetdash{}{0pt}%
\pgfsys@defobject{currentmarker}{\pgfqpoint{0.000000in}{0.000000in}}{\pgfqpoint{0.000000in}{0.000000in}}{%
\pgfpathmoveto{\pgfqpoint{0.000000in}{0.000000in}}%
\pgfpathlineto{\pgfqpoint{0.000000in}{0.000000in}}%
\pgfusepath{stroke,fill}%
}%
\begin{pgfscope}%
\pgfsys@transformshift{0.625000in}{0.511765in}%
\pgfsys@useobject{currentmarker}{}%
\end{pgfscope}%
\end{pgfscope}%
\begin{pgfscope}%
\pgfsetbuttcap%
\pgfsetroundjoin%
\definecolor{currentfill}{rgb}{0.150000,0.150000,0.150000}%
\pgfsetfillcolor{currentfill}%
\pgfsetlinewidth{0.803000pt}%
\definecolor{currentstroke}{rgb}{0.150000,0.150000,0.150000}%
\pgfsetstrokecolor{currentstroke}%
\pgfsetdash{}{0pt}%
\pgfsys@defobject{currentmarker}{\pgfqpoint{0.000000in}{0.000000in}}{\pgfqpoint{0.000000in}{0.000000in}}{%
\pgfpathmoveto{\pgfqpoint{0.000000in}{0.000000in}}%
\pgfpathlineto{\pgfqpoint{0.000000in}{0.000000in}}%
\pgfusepath{stroke,fill}%
}%
\begin{pgfscope}%
\pgfsys@transformshift{4.500000in}{0.511765in}%
\pgfsys@useobject{currentmarker}{}%
\end{pgfscope}%
\end{pgfscope}%
\begin{pgfscope}%
\definecolor{textcolor}{rgb}{0.150000,0.150000,0.150000}%
\pgfsetstrokecolor{textcolor}%
\pgfsetfillcolor{textcolor}%
\pgftext[x=0.547222in,y=0.511765in,right,]{\color{textcolor}\sffamily\fontsize{8.000000}{9.600000}\selectfont −1.5}%
\end{pgfscope}%
\begin{pgfscope}%
\pgfpathrectangle{\pgfqpoint{0.625000in}{0.375000in}}{\pgfqpoint{3.875000in}{2.325000in}} %
\pgfusepath{clip}%
\pgfsetroundcap%
\pgfsetroundjoin%
\pgfsetlinewidth{0.803000pt}%
\definecolor{currentstroke}{rgb}{1.000000,1.000000,1.000000}%
\pgfsetstrokecolor{currentstroke}%
\pgfsetdash{}{0pt}%
\pgfpathmoveto{\pgfqpoint{0.625000in}{0.853676in}}%
\pgfpathlineto{\pgfqpoint{4.500000in}{0.853676in}}%
\pgfusepath{stroke}%
\end{pgfscope}%
\begin{pgfscope}%
\pgfsetbuttcap%
\pgfsetroundjoin%
\definecolor{currentfill}{rgb}{0.150000,0.150000,0.150000}%
\pgfsetfillcolor{currentfill}%
\pgfsetlinewidth{0.803000pt}%
\definecolor{currentstroke}{rgb}{0.150000,0.150000,0.150000}%
\pgfsetstrokecolor{currentstroke}%
\pgfsetdash{}{0pt}%
\pgfsys@defobject{currentmarker}{\pgfqpoint{0.000000in}{0.000000in}}{\pgfqpoint{0.000000in}{0.000000in}}{%
\pgfpathmoveto{\pgfqpoint{0.000000in}{0.000000in}}%
\pgfpathlineto{\pgfqpoint{0.000000in}{0.000000in}}%
\pgfusepath{stroke,fill}%
}%
\begin{pgfscope}%
\pgfsys@transformshift{0.625000in}{0.853676in}%
\pgfsys@useobject{currentmarker}{}%
\end{pgfscope}%
\end{pgfscope}%
\begin{pgfscope}%
\pgfsetbuttcap%
\pgfsetroundjoin%
\definecolor{currentfill}{rgb}{0.150000,0.150000,0.150000}%
\pgfsetfillcolor{currentfill}%
\pgfsetlinewidth{0.803000pt}%
\definecolor{currentstroke}{rgb}{0.150000,0.150000,0.150000}%
\pgfsetstrokecolor{currentstroke}%
\pgfsetdash{}{0pt}%
\pgfsys@defobject{currentmarker}{\pgfqpoint{0.000000in}{0.000000in}}{\pgfqpoint{0.000000in}{0.000000in}}{%
\pgfpathmoveto{\pgfqpoint{0.000000in}{0.000000in}}%
\pgfpathlineto{\pgfqpoint{0.000000in}{0.000000in}}%
\pgfusepath{stroke,fill}%
}%
\begin{pgfscope}%
\pgfsys@transformshift{4.500000in}{0.853676in}%
\pgfsys@useobject{currentmarker}{}%
\end{pgfscope}%
\end{pgfscope}%
\begin{pgfscope}%
\definecolor{textcolor}{rgb}{0.150000,0.150000,0.150000}%
\pgfsetstrokecolor{textcolor}%
\pgfsetfillcolor{textcolor}%
\pgftext[x=0.547222in,y=0.853676in,right,]{\color{textcolor}\sffamily\fontsize{8.000000}{9.600000}\selectfont −1.0}%
\end{pgfscope}%
\begin{pgfscope}%
\pgfpathrectangle{\pgfqpoint{0.625000in}{0.375000in}}{\pgfqpoint{3.875000in}{2.325000in}} %
\pgfusepath{clip}%
\pgfsetroundcap%
\pgfsetroundjoin%
\pgfsetlinewidth{0.803000pt}%
\definecolor{currentstroke}{rgb}{1.000000,1.000000,1.000000}%
\pgfsetstrokecolor{currentstroke}%
\pgfsetdash{}{0pt}%
\pgfpathmoveto{\pgfqpoint{0.625000in}{1.195588in}}%
\pgfpathlineto{\pgfqpoint{4.500000in}{1.195588in}}%
\pgfusepath{stroke}%
\end{pgfscope}%
\begin{pgfscope}%
\pgfsetbuttcap%
\pgfsetroundjoin%
\definecolor{currentfill}{rgb}{0.150000,0.150000,0.150000}%
\pgfsetfillcolor{currentfill}%
\pgfsetlinewidth{0.803000pt}%
\definecolor{currentstroke}{rgb}{0.150000,0.150000,0.150000}%
\pgfsetstrokecolor{currentstroke}%
\pgfsetdash{}{0pt}%
\pgfsys@defobject{currentmarker}{\pgfqpoint{0.000000in}{0.000000in}}{\pgfqpoint{0.000000in}{0.000000in}}{%
\pgfpathmoveto{\pgfqpoint{0.000000in}{0.000000in}}%
\pgfpathlineto{\pgfqpoint{0.000000in}{0.000000in}}%
\pgfusepath{stroke,fill}%
}%
\begin{pgfscope}%
\pgfsys@transformshift{0.625000in}{1.195588in}%
\pgfsys@useobject{currentmarker}{}%
\end{pgfscope}%
\end{pgfscope}%
\begin{pgfscope}%
\pgfsetbuttcap%
\pgfsetroundjoin%
\definecolor{currentfill}{rgb}{0.150000,0.150000,0.150000}%
\pgfsetfillcolor{currentfill}%
\pgfsetlinewidth{0.803000pt}%
\definecolor{currentstroke}{rgb}{0.150000,0.150000,0.150000}%
\pgfsetstrokecolor{currentstroke}%
\pgfsetdash{}{0pt}%
\pgfsys@defobject{currentmarker}{\pgfqpoint{0.000000in}{0.000000in}}{\pgfqpoint{0.000000in}{0.000000in}}{%
\pgfpathmoveto{\pgfqpoint{0.000000in}{0.000000in}}%
\pgfpathlineto{\pgfqpoint{0.000000in}{0.000000in}}%
\pgfusepath{stroke,fill}%
}%
\begin{pgfscope}%
\pgfsys@transformshift{4.500000in}{1.195588in}%
\pgfsys@useobject{currentmarker}{}%
\end{pgfscope}%
\end{pgfscope}%
\begin{pgfscope}%
\definecolor{textcolor}{rgb}{0.150000,0.150000,0.150000}%
\pgfsetstrokecolor{textcolor}%
\pgfsetfillcolor{textcolor}%
\pgftext[x=0.547222in,y=1.195588in,right,]{\color{textcolor}\sffamily\fontsize{8.000000}{9.600000}\selectfont −0.5}%
\end{pgfscope}%
\begin{pgfscope}%
\pgfpathrectangle{\pgfqpoint{0.625000in}{0.375000in}}{\pgfqpoint{3.875000in}{2.325000in}} %
\pgfusepath{clip}%
\pgfsetroundcap%
\pgfsetroundjoin%
\pgfsetlinewidth{0.803000pt}%
\definecolor{currentstroke}{rgb}{1.000000,1.000000,1.000000}%
\pgfsetstrokecolor{currentstroke}%
\pgfsetdash{}{0pt}%
\pgfpathmoveto{\pgfqpoint{0.625000in}{1.537500in}}%
\pgfpathlineto{\pgfqpoint{4.500000in}{1.537500in}}%
\pgfusepath{stroke}%
\end{pgfscope}%
\begin{pgfscope}%
\pgfsetbuttcap%
\pgfsetroundjoin%
\definecolor{currentfill}{rgb}{0.150000,0.150000,0.150000}%
\pgfsetfillcolor{currentfill}%
\pgfsetlinewidth{0.803000pt}%
\definecolor{currentstroke}{rgb}{0.150000,0.150000,0.150000}%
\pgfsetstrokecolor{currentstroke}%
\pgfsetdash{}{0pt}%
\pgfsys@defobject{currentmarker}{\pgfqpoint{0.000000in}{0.000000in}}{\pgfqpoint{0.000000in}{0.000000in}}{%
\pgfpathmoveto{\pgfqpoint{0.000000in}{0.000000in}}%
\pgfpathlineto{\pgfqpoint{0.000000in}{0.000000in}}%
\pgfusepath{stroke,fill}%
}%
\begin{pgfscope}%
\pgfsys@transformshift{0.625000in}{1.537500in}%
\pgfsys@useobject{currentmarker}{}%
\end{pgfscope}%
\end{pgfscope}%
\begin{pgfscope}%
\pgfsetbuttcap%
\pgfsetroundjoin%
\definecolor{currentfill}{rgb}{0.150000,0.150000,0.150000}%
\pgfsetfillcolor{currentfill}%
\pgfsetlinewidth{0.803000pt}%
\definecolor{currentstroke}{rgb}{0.150000,0.150000,0.150000}%
\pgfsetstrokecolor{currentstroke}%
\pgfsetdash{}{0pt}%
\pgfsys@defobject{currentmarker}{\pgfqpoint{0.000000in}{0.000000in}}{\pgfqpoint{0.000000in}{0.000000in}}{%
\pgfpathmoveto{\pgfqpoint{0.000000in}{0.000000in}}%
\pgfpathlineto{\pgfqpoint{0.000000in}{0.000000in}}%
\pgfusepath{stroke,fill}%
}%
\begin{pgfscope}%
\pgfsys@transformshift{4.500000in}{1.537500in}%
\pgfsys@useobject{currentmarker}{}%
\end{pgfscope}%
\end{pgfscope}%
\begin{pgfscope}%
\definecolor{textcolor}{rgb}{0.150000,0.150000,0.150000}%
\pgfsetstrokecolor{textcolor}%
\pgfsetfillcolor{textcolor}%
\pgftext[x=0.547222in,y=1.537500in,right,]{\color{textcolor}\sffamily\fontsize{8.000000}{9.600000}\selectfont 0.0}%
\end{pgfscope}%
\begin{pgfscope}%
\pgfpathrectangle{\pgfqpoint{0.625000in}{0.375000in}}{\pgfqpoint{3.875000in}{2.325000in}} %
\pgfusepath{clip}%
\pgfsetroundcap%
\pgfsetroundjoin%
\pgfsetlinewidth{0.803000pt}%
\definecolor{currentstroke}{rgb}{1.000000,1.000000,1.000000}%
\pgfsetstrokecolor{currentstroke}%
\pgfsetdash{}{0pt}%
\pgfpathmoveto{\pgfqpoint{0.625000in}{1.879412in}}%
\pgfpathlineto{\pgfqpoint{4.500000in}{1.879412in}}%
\pgfusepath{stroke}%
\end{pgfscope}%
\begin{pgfscope}%
\pgfsetbuttcap%
\pgfsetroundjoin%
\definecolor{currentfill}{rgb}{0.150000,0.150000,0.150000}%
\pgfsetfillcolor{currentfill}%
\pgfsetlinewidth{0.803000pt}%
\definecolor{currentstroke}{rgb}{0.150000,0.150000,0.150000}%
\pgfsetstrokecolor{currentstroke}%
\pgfsetdash{}{0pt}%
\pgfsys@defobject{currentmarker}{\pgfqpoint{0.000000in}{0.000000in}}{\pgfqpoint{0.000000in}{0.000000in}}{%
\pgfpathmoveto{\pgfqpoint{0.000000in}{0.000000in}}%
\pgfpathlineto{\pgfqpoint{0.000000in}{0.000000in}}%
\pgfusepath{stroke,fill}%
}%
\begin{pgfscope}%
\pgfsys@transformshift{0.625000in}{1.879412in}%
\pgfsys@useobject{currentmarker}{}%
\end{pgfscope}%
\end{pgfscope}%
\begin{pgfscope}%
\pgfsetbuttcap%
\pgfsetroundjoin%
\definecolor{currentfill}{rgb}{0.150000,0.150000,0.150000}%
\pgfsetfillcolor{currentfill}%
\pgfsetlinewidth{0.803000pt}%
\definecolor{currentstroke}{rgb}{0.150000,0.150000,0.150000}%
\pgfsetstrokecolor{currentstroke}%
\pgfsetdash{}{0pt}%
\pgfsys@defobject{currentmarker}{\pgfqpoint{0.000000in}{0.000000in}}{\pgfqpoint{0.000000in}{0.000000in}}{%
\pgfpathmoveto{\pgfqpoint{0.000000in}{0.000000in}}%
\pgfpathlineto{\pgfqpoint{0.000000in}{0.000000in}}%
\pgfusepath{stroke,fill}%
}%
\begin{pgfscope}%
\pgfsys@transformshift{4.500000in}{1.879412in}%
\pgfsys@useobject{currentmarker}{}%
\end{pgfscope}%
\end{pgfscope}%
\begin{pgfscope}%
\definecolor{textcolor}{rgb}{0.150000,0.150000,0.150000}%
\pgfsetstrokecolor{textcolor}%
\pgfsetfillcolor{textcolor}%
\pgftext[x=0.547222in,y=1.879412in,right,]{\color{textcolor}\sffamily\fontsize{8.000000}{9.600000}\selectfont 0.5}%
\end{pgfscope}%
\begin{pgfscope}%
\pgfpathrectangle{\pgfqpoint{0.625000in}{0.375000in}}{\pgfqpoint{3.875000in}{2.325000in}} %
\pgfusepath{clip}%
\pgfsetroundcap%
\pgfsetroundjoin%
\pgfsetlinewidth{0.803000pt}%
\definecolor{currentstroke}{rgb}{1.000000,1.000000,1.000000}%
\pgfsetstrokecolor{currentstroke}%
\pgfsetdash{}{0pt}%
\pgfpathmoveto{\pgfqpoint{0.625000in}{2.221324in}}%
\pgfpathlineto{\pgfqpoint{4.500000in}{2.221324in}}%
\pgfusepath{stroke}%
\end{pgfscope}%
\begin{pgfscope}%
\pgfsetbuttcap%
\pgfsetroundjoin%
\definecolor{currentfill}{rgb}{0.150000,0.150000,0.150000}%
\pgfsetfillcolor{currentfill}%
\pgfsetlinewidth{0.803000pt}%
\definecolor{currentstroke}{rgb}{0.150000,0.150000,0.150000}%
\pgfsetstrokecolor{currentstroke}%
\pgfsetdash{}{0pt}%
\pgfsys@defobject{currentmarker}{\pgfqpoint{0.000000in}{0.000000in}}{\pgfqpoint{0.000000in}{0.000000in}}{%
\pgfpathmoveto{\pgfqpoint{0.000000in}{0.000000in}}%
\pgfpathlineto{\pgfqpoint{0.000000in}{0.000000in}}%
\pgfusepath{stroke,fill}%
}%
\begin{pgfscope}%
\pgfsys@transformshift{0.625000in}{2.221324in}%
\pgfsys@useobject{currentmarker}{}%
\end{pgfscope}%
\end{pgfscope}%
\begin{pgfscope}%
\pgfsetbuttcap%
\pgfsetroundjoin%
\definecolor{currentfill}{rgb}{0.150000,0.150000,0.150000}%
\pgfsetfillcolor{currentfill}%
\pgfsetlinewidth{0.803000pt}%
\definecolor{currentstroke}{rgb}{0.150000,0.150000,0.150000}%
\pgfsetstrokecolor{currentstroke}%
\pgfsetdash{}{0pt}%
\pgfsys@defobject{currentmarker}{\pgfqpoint{0.000000in}{0.000000in}}{\pgfqpoint{0.000000in}{0.000000in}}{%
\pgfpathmoveto{\pgfqpoint{0.000000in}{0.000000in}}%
\pgfpathlineto{\pgfqpoint{0.000000in}{0.000000in}}%
\pgfusepath{stroke,fill}%
}%
\begin{pgfscope}%
\pgfsys@transformshift{4.500000in}{2.221324in}%
\pgfsys@useobject{currentmarker}{}%
\end{pgfscope}%
\end{pgfscope}%
\begin{pgfscope}%
\definecolor{textcolor}{rgb}{0.150000,0.150000,0.150000}%
\pgfsetstrokecolor{textcolor}%
\pgfsetfillcolor{textcolor}%
\pgftext[x=0.547222in,y=2.221324in,right,]{\color{textcolor}\sffamily\fontsize{8.000000}{9.600000}\selectfont 1.0}%
\end{pgfscope}%
\begin{pgfscope}%
\pgfpathrectangle{\pgfqpoint{0.625000in}{0.375000in}}{\pgfqpoint{3.875000in}{2.325000in}} %
\pgfusepath{clip}%
\pgfsetroundcap%
\pgfsetroundjoin%
\pgfsetlinewidth{0.803000pt}%
\definecolor{currentstroke}{rgb}{1.000000,1.000000,1.000000}%
\pgfsetstrokecolor{currentstroke}%
\pgfsetdash{}{0pt}%
\pgfpathmoveto{\pgfqpoint{0.625000in}{2.563235in}}%
\pgfpathlineto{\pgfqpoint{4.500000in}{2.563235in}}%
\pgfusepath{stroke}%
\end{pgfscope}%
\begin{pgfscope}%
\pgfsetbuttcap%
\pgfsetroundjoin%
\definecolor{currentfill}{rgb}{0.150000,0.150000,0.150000}%
\pgfsetfillcolor{currentfill}%
\pgfsetlinewidth{0.803000pt}%
\definecolor{currentstroke}{rgb}{0.150000,0.150000,0.150000}%
\pgfsetstrokecolor{currentstroke}%
\pgfsetdash{}{0pt}%
\pgfsys@defobject{currentmarker}{\pgfqpoint{0.000000in}{0.000000in}}{\pgfqpoint{0.000000in}{0.000000in}}{%
\pgfpathmoveto{\pgfqpoint{0.000000in}{0.000000in}}%
\pgfpathlineto{\pgfqpoint{0.000000in}{0.000000in}}%
\pgfusepath{stroke,fill}%
}%
\begin{pgfscope}%
\pgfsys@transformshift{0.625000in}{2.563235in}%
\pgfsys@useobject{currentmarker}{}%
\end{pgfscope}%
\end{pgfscope}%
\begin{pgfscope}%
\pgfsetbuttcap%
\pgfsetroundjoin%
\definecolor{currentfill}{rgb}{0.150000,0.150000,0.150000}%
\pgfsetfillcolor{currentfill}%
\pgfsetlinewidth{0.803000pt}%
\definecolor{currentstroke}{rgb}{0.150000,0.150000,0.150000}%
\pgfsetstrokecolor{currentstroke}%
\pgfsetdash{}{0pt}%
\pgfsys@defobject{currentmarker}{\pgfqpoint{0.000000in}{0.000000in}}{\pgfqpoint{0.000000in}{0.000000in}}{%
\pgfpathmoveto{\pgfqpoint{0.000000in}{0.000000in}}%
\pgfpathlineto{\pgfqpoint{0.000000in}{0.000000in}}%
\pgfusepath{stroke,fill}%
}%
\begin{pgfscope}%
\pgfsys@transformshift{4.500000in}{2.563235in}%
\pgfsys@useobject{currentmarker}{}%
\end{pgfscope}%
\end{pgfscope}%
\begin{pgfscope}%
\definecolor{textcolor}{rgb}{0.150000,0.150000,0.150000}%
\pgfsetstrokecolor{textcolor}%
\pgfsetfillcolor{textcolor}%
\pgftext[x=0.547222in,y=2.563235in,right,]{\color{textcolor}\sffamily\fontsize{8.000000}{9.600000}\selectfont 1.5}%
\end{pgfscope}%
\begin{pgfscope}%
\definecolor{textcolor}{rgb}{0.150000,0.150000,0.150000}%
\pgfsetstrokecolor{textcolor}%
\pgfsetfillcolor{textcolor}%
\pgftext[x=0.228007in,y=1.537500in,,bottom,rotate=90.000000]{\color{textcolor}\sffamily\fontsize{8.800000}{10.560000}\selectfont LOO predictions}%
\end{pgfscope}%
\begin{pgfscope}%
\pgfpathrectangle{\pgfqpoint{0.625000in}{0.375000in}}{\pgfqpoint{3.875000in}{2.325000in}} %
\pgfusepath{clip}%
\pgfsetbuttcap%
\pgfsetroundjoin%
\definecolor{currentfill}{rgb}{0.000000,0.000000,0.000000}%
\pgfsetfillcolor{currentfill}%
\pgfsetlinewidth{2.007500pt}%
\definecolor{currentstroke}{rgb}{0.000000,0.000000,0.000000}%
\pgfsetstrokecolor{currentstroke}%
\pgfsetdash{}{0pt}%
\pgfsys@defobject{currentmarker}{\pgfqpoint{-0.038889in}{-0.038889in}}{\pgfqpoint{0.038889in}{0.038889in}}{%
\pgfpathmoveto{\pgfqpoint{-0.038889in}{-0.038889in}}%
\pgfpathlineto{\pgfqpoint{0.038889in}{0.038889in}}%
\pgfpathmoveto{\pgfqpoint{-0.038889in}{0.038889in}}%
\pgfpathlineto{\pgfqpoint{0.038889in}{-0.038889in}}%
\pgfusepath{stroke,fill}%
}%
\begin{pgfscope}%
\pgfsys@transformshift{3.069242in}{1.502400in}%
\pgfsys@useobject{currentmarker}{}%
\end{pgfscope}%
\begin{pgfscope}%
\pgfsys@transformshift{2.336981in}{1.585626in}%
\pgfsys@useobject{currentmarker}{}%
\end{pgfscope}%
\begin{pgfscope}%
\pgfsys@transformshift{2.260051in}{0.907780in}%
\pgfsys@useobject{currentmarker}{}%
\end{pgfscope}%
\begin{pgfscope}%
\pgfsys@transformshift{1.712992in}{1.259604in}%
\pgfsys@useobject{currentmarker}{}%
\end{pgfscope}%
\begin{pgfscope}%
\pgfsys@transformshift{1.262808in}{1.085049in}%
\pgfsys@useobject{currentmarker}{}%
\end{pgfscope}%
\begin{pgfscope}%
\pgfsys@transformshift{1.328341in}{1.114466in}%
\pgfsys@useobject{currentmarker}{}%
\end{pgfscope}%
\begin{pgfscope}%
\pgfsys@transformshift{2.077698in}{1.447027in}%
\pgfsys@useobject{currentmarker}{}%
\end{pgfscope}%
\begin{pgfscope}%
\pgfsys@transformshift{3.012256in}{1.724384in}%
\pgfsys@useobject{currentmarker}{}%
\end{pgfscope}%
\begin{pgfscope}%
\pgfsys@transformshift{3.357017in}{1.737488in}%
\pgfsys@useobject{currentmarker}{}%
\end{pgfscope}%
\begin{pgfscope}%
\pgfsys@transformshift{1.841209in}{1.618948in}%
\pgfsys@useobject{currentmarker}{}%
\end{pgfscope}%
\begin{pgfscope}%
\pgfsys@transformshift{2.234407in}{1.455929in}%
\pgfsys@useobject{currentmarker}{}%
\end{pgfscope}%
\begin{pgfscope}%
\pgfsys@transformshift{1.322642in}{1.004204in}%
\pgfsys@useobject{currentmarker}{}%
\end{pgfscope}%
\begin{pgfscope}%
\pgfsys@transformshift{3.334223in}{1.626397in}%
\pgfsys@useobject{currentmarker}{}%
\end{pgfscope}%
\begin{pgfscope}%
\pgfsys@transformshift{1.829812in}{1.296235in}%
\pgfsys@useobject{currentmarker}{}%
\end{pgfscope}%
\begin{pgfscope}%
\pgfsys@transformshift{1.553433in}{1.006016in}%
\pgfsys@useobject{currentmarker}{}%
\end{pgfscope}%
\begin{pgfscope}%
\pgfsys@transformshift{2.331282in}{1.312053in}%
\pgfsys@useobject{currentmarker}{}%
\end{pgfscope}%
\begin{pgfscope}%
\pgfsys@transformshift{3.639095in}{1.662430in}%
\pgfsys@useobject{currentmarker}{}%
\end{pgfscope}%
\begin{pgfscope}%
\pgfsys@transformshift{1.573378in}{1.225419in}%
\pgfsys@useobject{currentmarker}{}%
\end{pgfscope}%
\begin{pgfscope}%
\pgfsys@transformshift{1.504995in}{1.203369in}%
\pgfsys@useobject{currentmarker}{}%
\end{pgfscope}%
\begin{pgfscope}%
\pgfsys@transformshift{2.265749in}{1.060639in}%
\pgfsys@useobject{currentmarker}{}%
\end{pgfscope}%
\begin{pgfscope}%
\pgfsys@transformshift{1.807017in}{1.415103in}%
\pgfsys@useobject{currentmarker}{}%
\end{pgfscope}%
\begin{pgfscope}%
\pgfsys@transformshift{2.613359in}{1.830154in}%
\pgfsys@useobject{currentmarker}{}%
\end{pgfscope}%
\begin{pgfscope}%
\pgfsys@transformshift{3.126227in}{1.355796in}%
\pgfsys@useobject{currentmarker}{}%
\end{pgfscope}%
\begin{pgfscope}%
\pgfsys@transformshift{3.137624in}{2.075573in}%
\pgfsys@useobject{currentmarker}{}%
\end{pgfscope}%
\begin{pgfscope}%
\pgfsys@transformshift{2.365473in}{1.355059in}%
\pgfsys@useobject{currentmarker}{}%
\end{pgfscope}%
\begin{pgfscope}%
\pgfsys@transformshift{3.470988in}{1.785394in}%
\pgfsys@useobject{currentmarker}{}%
\end{pgfscope}%
\begin{pgfscope}%
\pgfsys@transformshift{2.596264in}{1.870701in}%
\pgfsys@useobject{currentmarker}{}%
\end{pgfscope}%
\begin{pgfscope}%
\pgfsys@transformshift{2.408212in}{1.260749in}%
\pgfsys@useobject{currentmarker}{}%
\end{pgfscope}%
\begin{pgfscope}%
\pgfsys@transformshift{2.097642in}{1.599172in}%
\pgfsys@useobject{currentmarker}{}%
\end{pgfscope}%
\begin{pgfscope}%
\pgfsys@transformshift{4.174756in}{1.779065in}%
\pgfsys@useobject{currentmarker}{}%
\end{pgfscope}%
\begin{pgfscope}%
\pgfsys@transformshift{1.878249in}{1.515681in}%
\pgfsys@useobject{currentmarker}{}%
\end{pgfscope}%
\begin{pgfscope}%
\pgfsys@transformshift{3.559315in}{1.964673in}%
\pgfsys@useobject{currentmarker}{}%
\end{pgfscope}%
\begin{pgfscope}%
\pgfsys@transformshift{2.849848in}{1.507238in}%
\pgfsys@useobject{currentmarker}{}%
\end{pgfscope}%
\begin{pgfscope}%
\pgfsys@transformshift{3.667587in}{2.181956in}%
\pgfsys@useobject{currentmarker}{}%
\end{pgfscope}%
\begin{pgfscope}%
\pgfsys@transformshift{3.616301in}{2.229460in}%
\pgfsys@useobject{currentmarker}{}%
\end{pgfscope}%
\begin{pgfscope}%
\pgfsys@transformshift{2.220161in}{1.818368in}%
\pgfsys@useobject{currentmarker}{}%
\end{pgfscope}%
\begin{pgfscope}%
\pgfsys@transformshift{3.539370in}{1.796039in}%
\pgfsys@useobject{currentmarker}{}%
\end{pgfscope}%
\begin{pgfscope}%
\pgfsys@transformshift{4.040841in}{2.073745in}%
\pgfsys@useobject{currentmarker}{}%
\end{pgfscope}%
\begin{pgfscope}%
\pgfsys@transformshift{2.311337in}{1.713222in}%
\pgfsys@useobject{currentmarker}{}%
\end{pgfscope}%
\begin{pgfscope}%
\pgfsys@transformshift{3.171815in}{1.827428in}%
\pgfsys@useobject{currentmarker}{}%
\end{pgfscope}%
\end{pgfscope}%
\begin{pgfscope}%
\pgfpathrectangle{\pgfqpoint{0.625000in}{0.375000in}}{\pgfqpoint{3.875000in}{2.325000in}} %
\pgfusepath{clip}%
\pgfsetbuttcap%
\pgfsetroundjoin%
\pgfsetlinewidth{0.752812pt}%
\definecolor{currentstroke}{rgb}{0.000000,0.000000,0.000000}%
\pgfsetstrokecolor{currentstroke}%
\pgfsetdash{{6.000000pt}{6.000000pt}}{0.000000pt}%
\pgfpathmoveto{\pgfqpoint{0.283088in}{0.169853in}}%
\pgfpathmoveto{\pgfqpoint{0.611111in}{0.366667in}}%
\pgfpathlineto{\pgfqpoint{4.513889in}{2.708333in}}%
\pgfusepath{stroke}%
\end{pgfscope}%
\begin{pgfscope}%
\pgfsetrectcap%
\pgfsetmiterjoin%
\pgfsetlinewidth{0.000000pt}%
\definecolor{currentstroke}{rgb}{1.000000,1.000000,1.000000}%
\pgfsetstrokecolor{currentstroke}%
\pgfsetdash{}{0pt}%
\pgfpathmoveto{\pgfqpoint{0.625000in}{2.700000in}}%
\pgfpathlineto{\pgfqpoint{4.500000in}{2.700000in}}%
\pgfusepath{}%
\end{pgfscope}%
\begin{pgfscope}%
\pgfsetrectcap%
\pgfsetmiterjoin%
\pgfsetlinewidth{0.000000pt}%
\definecolor{currentstroke}{rgb}{1.000000,1.000000,1.000000}%
\pgfsetstrokecolor{currentstroke}%
\pgfsetdash{}{0pt}%
\pgfpathmoveto{\pgfqpoint{4.500000in}{0.375000in}}%
\pgfpathlineto{\pgfqpoint{4.500000in}{2.700000in}}%
\pgfusepath{}%
\end{pgfscope}%
\begin{pgfscope}%
\pgfsetrectcap%
\pgfsetmiterjoin%
\pgfsetlinewidth{0.000000pt}%
\definecolor{currentstroke}{rgb}{1.000000,1.000000,1.000000}%
\pgfsetstrokecolor{currentstroke}%
\pgfsetdash{}{0pt}%
\pgfpathmoveto{\pgfqpoint{0.625000in}{0.375000in}}%
\pgfpathlineto{\pgfqpoint{4.500000in}{0.375000in}}%
\pgfusepath{}%
\end{pgfscope}%
\begin{pgfscope}%
\pgfsetrectcap%
\pgfsetmiterjoin%
\pgfsetlinewidth{0.000000pt}%
\definecolor{currentstroke}{rgb}{1.000000,1.000000,1.000000}%
\pgfsetstrokecolor{currentstroke}%
\pgfsetdash{}{0pt}%
\pgfpathmoveto{\pgfqpoint{0.625000in}{0.375000in}}%
\pgfpathlineto{\pgfqpoint{0.625000in}{2.700000in}}%
\pgfusepath{}%
\end{pgfscope}%
\end{pgfpicture}%
\makeatother%
\endgroup%

	\caption{Comparison between the two times registered for one throw.}
	\label{mloo_vs_real}
\end{figure}
